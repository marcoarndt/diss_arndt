% !TeX document-id = {2e27ddd9-8725-4684-b0b8-a4328d10ae60}
% !TeX program = txs:///lualatex
% !TEX spellcheck = de-DE
% !BIB program = txs:///biber
%% magic comments ab automatically switch to correct compiler in TexStudio


\documentclass[
%accepted, % uncomment this after your oral exam for the final print (changes title page)
% print,% use this flag for printing (turns links black)
english, ngerman,% Language: last one is the main language used in the document
smallfont, % if you wish a smaller font. This is 12pt for the manuscript, 
twoside, % one of {oneside, twoside} -> onesided or twosided layout. for twosided print choose twoside
toc=bib,
BCOR=6mm, % binding correction, 6 mm for 200 pages, 80 g/m² Paper (half of thickness of book content)
]{isw_smb_diss} % Relative path to class. If class is in subfolder (e.g. as git submodule) then \documentclass[...]{isw_smb_diss/isw_smb_diss}

% Rules for the design of your dissertation: http://dx.doi.org/10.18419/opus-10327 (S. 25)

% Suppress single lines on previous/next page:
\clubpenalty=10000
\widowpenalty=10000
\displaywidowpenalty=10000

%% ======= Recommended packages (not necessarily needed) ======= 

\usepackage{shellesc} % shell escape alternative: more flexible than --shell-escape flag
					  % for glossaries-extra, tikz external package

% use this to debug where borders are hurt:
%\usepackage{showframe}

\usepackage{csquotes} % for bibliography and glossaries
\usepackage{xpatch}

% TikZ - Packages to include graphics
  \usepackage{pgfplots}
  \pgfplotsset{compat=newest}
  %% the following commands are needed for some matlab2tikz features
  \usetikzlibrary{plotmarks}
  \usetikzlibrary{arrows.meta}
  \usepgfplotslibrary{patchplots}
  \usepackage{grffile}
  \usepackage{amsmath}

% Citation style. If your institute has specific requirements, configure here:
\usepackage[style=numeric, % see https://www.overleaf.com/learn/latex/Biblatex_citation_styles for a list of predefined styles
sorting=none,
maxcitenames=2,
maxbibnames=99,
giveninits=true,
uniquelist=false,
backend=biber,
natbib=true,
bibwarn=true,
sortcites=true, % sort citations when multiple are cited, e.g. \cite{A,B,C} -> [1,2,3] instead of [3,1,2]
isbn=true]{biblatex}

% Hack: print url only if no doi is present, credits to samcarter_is_at_topanswers.xyz on https://tex.stackexchange.com/a/424775
\renewbibmacro*{doi+eprint+url}{%
	\printfield{doi}%
	\newunit\newblock%
	\iftoggle{bbx:eprint}{%
		\usebibmacro{eprint}%
	}{}%
	\newunit\newblock%
	\iffieldundef{doi}{%
		\usebibmacro{url+urldate}}%
	{}%
}

\DefineBibliographyStrings{ngerman}{
   andothers = {et\addabbrvspace al\adddot}
}

\usepackage{amsmath, amssymb, bm} % math commands
\numberwithin{equation}{chapter} %ensure that equations enumeration starts at 1 for every chapter

\usepackage{breakurl} %break links in bibliography

\RequirePackage{graphicx} % images
\usepackage{xcolor} % Markierung im Text

% Nice looking tables (https://nhigham.com/2019/11/19/better-latex-tables-with-booktabs/latex subfig)
\usepackage{booktabs}

% si-units: auto-rounding numbers, set same format for numbers globally, add and correclty typeset units for numbers
\usepackage{siunitx}
\sisetup{%round-mode = places, 
	%round-precision = 3, % 
	range-phrase=~--~, % --, % \qtyrange{from}{to}{unit}
	table-number-alignment = center, 
	table-omit-exponent=false, % turn of exp representation in tables
	exponent-product=\cdot,
	output-decimal-marker={,}, % "," instead of "." as decimal separator (for German theses)
	%exponent-mode = fixed, fixed-exponent = 0, % turn of exponent mode with Latex3e
}
% The following line should help, if you experience problems with missing hyphenation of composed words with dash 
% (LaTeX turns off hyphenation when a dash is present in a word)
% Source: https://tex.stackexchange.com/questions/63232/why-can-words-with-hyphen-char-not-be-hyphenated/63234#63234
% Alternatively, explicitely hyphenate with "= (https://de.wikibooks.org/wiki/LaTeX-W%C3%B6rterbuch:_Silbentrennung)
\defaulthyphenchar=127

% Just for demonstration use
\usepackage{blindtext}

% harvey balls: e.g \harveyBallFull
% \usepackage{harveyballs}

%% mark changes after your Oral presentation for review (http://mirrors.ibiblio.org/CTAN/macros/latex/contrib/changes/changes.english.pdf):
%% You can even list all changes so they can easily be checked with "\listofchanges"
%\usepackage[%final % keyword "final" is to hide changes highlighting
%				]{changes}
%\definechangesauthor[name=Prof. Haupt Berichter, color=Plum]{MyProf}
%\definechangesauthor[name=Prof. Mit berichter, color=OliveGreen]{OtherProf}

% Other settings you may find useful
\input{settings/definitions} % commands
\input{settings/tikz} % TikZ settings
\input{settings/tikz_blocks} % block diagrams using tikz 

%% ======= End of recommended packages ======= 

% TODO: add your own bibliography here:
\addbibresource{bibliography.bib}

% search for images in folder img/, i.e. instead of  \includegraphics{img/...}, just use \includegraphics{...}
\graphicspath{{img/}}


\title{Entwicklung Effizienter Multivariater Lebensdauertests}

% English title: set proper title case: https://titlecaseconverter.com/
\subtitle{Efficient Multivariate Lifetime Testing}

\author{Marco Arndt,~M.Sc.}
\birthplace{Ravensburg}

\examiner{PD Dr.-Ing. habil. Martin Dazer}
\coexaminer{Univ.-Prof.~Dr.-Ing.~Mit~Berichter}
\faculty{Konstruktions-, Produktions- und Fahrzeugtechnik} % Fak. 7
\institute{Institut für Maschinenelemente}
\dateoforalexamination{1. Januar 2026}
\yearofpublication{2026}

% Manual hyphenation. 
% More info for German hyphenation in LaTeX: https://de.wikibooks.org/wiki/LaTeX-W%C3%B6rterbuch:_Silbentrennung
% \hyphenation{Dis-am-bi-gu-ie-rung, Sil-ben-trenn-ung}

%%% Acronyms and glossary:
% Glossary (Used symbols, acronyms, ...)
\input{settings/glossaries}

% Singular: \ac{cdf} Plural: \acp{cdf} 
%%%
\newacronym{ALT}{ALT}{Accelerated Lifetime Testing}
\newacronym[plural=BEVs, longplural=Battery Electric Vehicles]{BEV}{BEV}{Battery Electric Vehicle}
\newacronym[plural=cdfs, longplural=Cumulative Distribution Functions]{cdf}{\textit{cdf}}{Cumulative Distribution Function}
\newacronym[plural=CIs, longplural=Confidence Intervals]{CI}{CI}{Confidence Interval}
\newacronym{DoE}{DoE}{Design of Experiments}
\newacronym{L-DoE}{L-DoE}{Lifetime Design of Experiments}
\newacronym[plural=ECUs, longplural=Electronic Control Units]{ECU}{ECU}{Electronic Control Unit}
\newacronym{EoL}{EoL}{End-of-Life}
\newacronym[plural=MLEs, longplural=Maximum-Likelihood-Estimations]{MLE}{MLE}{Maximum-Likelihood-Estimation}
\newacronym{MMR}{MMR}{Median-Rank-Regression-Methode}
\newacronym{OLS}{OLS}{Ordinary Least Squares}
\newacronym[plural=pdfs, longplural=Probability Density Functions]{pdf}{\textit{pdf}}{Probability Density Function}
\newacronym{IMA}{IMA}{Institut für Maschinenelemente}
\newacronym{ANOVA}{ANOVA}{Varianzanalyse, engl. Analysis of Variance}
\newacronym{OFAT}{OFAT}{One Factor At Time}

% Indices - use in text: \sym{i}, also needs \input{settings/glossaries}
\newglossaryentry{i}{type=indices, name={\ensuremath{i}}, description={Laufvariable}}
\newglossaryentry{0}{type=indices, name={\ensuremath{0}}, description={zum Zeitpunkt $\sym{t}=0$}}
\newglossaryentry{q}{type=indices, name={\ensuremath{q}}, description={Quantilwert}}
\newglossaryentry{j}{type=indices, name={\ensuremath{j}}, description={Laufvariable}}
\newglossaryentry{l}{type=indices, name={\ensuremath{l}}, description={Laufvariable}}
\newglossaryentry{O}{type=indices, name={\ensuremath{O}}, description={Beobachtet (Observed)}}
\newglossaryentry{u}{type=indices, name={\ensuremath{u}}, description={Untere Grenze des Vertrauensbereichs}}
\newglossaryentry{o}{type=indices, name={\ensuremath{o}}, description={Obere Grenze des Vertrauensbereichs}}


%%%
% Glossary - use in text: \sym{R}, also needs \input{settings/glossaries}
\glsxtrnewsymbol[type=formulas, description={Weibull-Formparameter (Weibull-Modul)}, text={\ensuremath{b}}]{b}{\ensuremath{b}}
\glsxtrnewsymbol[type=formulas, description={Statusindikator (1=Ausfall, 0=Zensiert)}, text={\ensuremath{\delta}}]{delta}{\ensuremath{\delta}}
\glsxtrnewsymbol[type=formulas, description={Erwartungswert}, text={\ensuremath{E}}]{E}{\ensuremath{E(\cdot)}}
\glsxtrnewsymbol[type=formulas, description={Dichtefunktion}, text={\ensuremath{f}}]{f}{\ensuremath{f(\cdot)}}
\glsxtrnewsymbol[type=formulas, description={Ausfallwahrscheinlichkeit, Verteilungsfunktion}, text={\ensuremath{F}}]{F}{\ensuremath{F(\cdot)}}
\glsxtrnewsymbol[type=formulas, description={Gamma-Funktion}, text={\ensuremath{\Gamma}}]{Gamma}{\ensuremath{\Gamma(\cdot)}}
\glsxtrnewsymbol[type=formulas, description={Anzahl der Modellparameter}, text={\ensuremath{k}}]{k}{\ensuremath{k}}
\glsxtrnewsymbol[type=formulas, description={Ausfallrate}, text={\ensuremath{\lambda}}]{lambda}{\ensuremath{\lambda(\cdot)}}
\glsxtrnewsymbol[type=formulas, description={Log-Likelihood-Funktion}, text={\ensuremath{\Lambda}}]{Lambda}{\ensuremath{\Lambda(\cdot)}}
\glsxtrnewsymbol[type=formulas, description={Likelihood-Funktion}, text={\ensuremath{\mathcal{L}}}]{L_like}{\ensuremath{\mathcal{L}(\cdot)}}
\glsxtrnewsymbol[type=formulas, description={Erwartungswert der Lebensdauer (theoretisch)}, text={\ensuremath{\mu}}]{mu}{\ensuremath{\mu}}
\glsxtrnewsymbol[type=formulas, description={Stichprobenumfang}, text={\ensuremath{n}}]{n}{\ensuremath{n}}
\glsxtrnewsymbol[type=formulas, description={Wahrscheinlichkeit}, text={\ensuremath{\Pr}}]{Pr}{\ensuremath{\Pr}}
\glsxtrnewsymbol[type=formulas, description={Zuverlässigkeit}]{R}{\ensuremath{R}}
\glsxtrnewsymbol[type=formulas, description={Menge der reellen Zahlen}]{RR}{\ensuremath{\mathbb{R}}}
\glsxtrnewsymbol[type=formulas, description={Empirische Standardabweichung (von x)}, text={\ensuremath{s}}]{s}{\ensuremath{s}}
\glsxtrnewsymbol[type=formulas, description={Empirische Varianz (von x)}, text={\ensuremath{s^2}}]{s_sq}{\ensuremath{s^2}}
\glsxtrnewsymbol[type=formulas, description={Varianz der Lebensdauer (theoretisch)}, text={\ensuremath{\sigma^2}}]{sigma_sq}{\ensuremath{\sigma^2}}
\glsxtrnewsymbol[type=formulas, description={Zeit, Lebensdauermerimal}]{t}{\ensuremath{t}}
\glsxtrnewsymbol[type=formulas, description={Vektor der Ausfallzeiten}, text={\ensuremath{\mathbf{t}}}]{t_vec}{\ensuremath{\mathbf{t}}}
\glsxtrnewsymbol[type=formulas, description={Charakteristische Lebensdauer (Skalenparameter)}, text={\ensuremath{T}}]{T}{\ensuremath{T}}
\glsxtrnewsymbol[type=formulas, description={kontinuierliche Zufallsvariable}]{tau}{\ensuremath{\tau}}
\glsxtrnewsymbol[type=formulas, description={Vektor der Modellparameter}, text={\ensuremath{\bm{\theta}}}]{theta}{\ensuremath{\bm{\theta}}}
\glsxtrnewsymbol[type=formulas, description={Varianz-Operator}, text={\ensuremath{\mathrm{Var}}}]{Var}{\ensuremath{\mathrm{Var}[\cdot]}}
\glsxtrnewsymbol[type=formulas, description={Covarianz-Operator}, text={\ensuremath{\mathrm{Cov}}}]{Cov}{\ensuremath{\mathrm{Cov}[\cdot]}}
\glsxtrnewsymbol[type=formulas, description={Weibull-Verteilung (Notation)}, text={\ensuremath{\mathcal{W}}}]{W}{\ensuremath{\mathcal{W}(\cdot, \cdot)}}
\glsxtrnewsymbol[type=formulas, description={Messwert, Variable}, text={\ensuremath{x}}]{x}{\ensuremath{x}}
\glsxtrnewsymbol[type=formulas, description={Arithmetischer Mittelwert (empirisch)}, text={\ensuremath{\bar{x}}}]{x_bar}{\ensuremath{\bar{x}}}
\glsxtrnewsymbol[type=formulas, description={Integrationsvariable (Gamma-Funktion)}, text={\ensuremath{\upsilon}}]{ups}{\ensuremath{\upsilon}}
\glsxtrnewsymbol[type=formulas, description={Fisher-Informationsmatrix}, text={\ensuremath{\mathbf{F}}}]{FIM}{\ensuremath{\mathbf{F}}}
\glsxtrnewsymbol[type=formulas, description={Hessian-Matrix (der Log-Likelihood-Funktion)}, text={\ensuremath{\mathbf{H}}}]{H}{\ensuremath{\mathbf{H}}}
\glsxtrnewsymbol[type=formulas, description={Varianz-Kovarianz-Matrix}, text={\ensuremath{\mathbf{V}}}]{V}{\ensuremath{\mathbf{V}}}
\glsxtrnewsymbol[type=formulas, description={Standardfehler (Standard Error)}, text={\ensuremath{\mathit{se}}}]{se}{\ensuremath{\mathit{se}}}
\glsxtrnewsymbol[type=formulas, description={Signifikanzniveau}, text={\ensuremath{\alpha}}]{alpha}{\ensuremath{\alpha}}
\glsxtrnewsymbol[type=formulas, description={Quantil der Standardnormalverteilung}, text={\ensuremath{q}}]{q_norm}{\ensuremath{q}}
\glsxtrnewsymbol[type=formulas, description={Quantil der Standardnormalverteilung}, text={\ensuremath{z}}]{z}{\ensuremath{z}}
\glsxtrnewsymbol[type=formulas, description={Abgeleitete Funktion (für Delta-Methode)}, text={\ensuremath{g}}]{g_func}{\ensuremath{g(\cdot)}}
\glsxtrnewsymbol[type=formulas, description={Gradientenvektor der Funktion g}, text={\ensuremath{\mathbf{g}'}}]{g_grad}{\ensuremath{\mathbf{g}'}}
\glsxtrnewsymbol[type=formulas, description={Systemantwort, -variable}, text={\ensuremath{y}}]{y}{\ensuremath{y}}
\glsxtrnewsymbol[type=formulas, description={Nullhypothese}, text={\ensuremath{H_0}}]{H0}{\ensuremath{H_0}}
\glsxtrnewsymbol[type=formulas, description={Alternativhypothese}, text={\ensuremath{H_1}}]{H1}{\ensuremath{H_1}}


%  Show page borders (useful for final checks)
%\usepackage{showframe}

\begin{document}

\frontmatter
\maketitle
\onehalfspacing % line spacing: 1,5

\addchap{Vorwort}
% Formel-Beispiel
Thanks for your service.

% Beispiel fürs Referenzieren eines Akronyms:
% \ac{ISW}.
% Automatische Bindestriche bei zusammengesetzten Akronymen: \ac{FE}[-Modelle]

\cleardoublepage
\addchap{Kurzfassung/Abstract}
\colorbox{yellow}{--TODO--TODO--TODO--TODO--TODO--TODO--TODO--TODO--TODO--TODO--TODO--}
\todo{
    \begin{itemize}
        \item
        \item Abstract / Kurzfassung / Danksagung
        \item Bilder prüfen
        \item "mark changes after your Oral presentation for review"
        \item "textcolor blue" in definitions ändern
        \item "Print" Funktion aktivieren
        \item "nocite" Funktion deaktivieren in main Z.269
    \end{itemize}
}\
\colorbox{yellow}{--TODO--TODO--TODO--TODO--TODO--TODO--TODO--TODO--TODO--TODO--TODO--}

\cleardoublepage
\ifpdf
    \phantomsection
    \pdfbookmark[0]{Inhaltsverzeichnis}{toc}
\fi
\tableofcontents

% % List of acronyms/symbols
%%
\cleardoublepage
\phantomsection
\addchap{Nomenklatur}
\label{chap:nomenklatur}

% --- Abkürzungen ---
\section*{Abkürzungen}
\addcontentsline{toc}{section}{Abkürzungen}
\glssetwidest{Electronic Control Unit~}
\begingroup
\renewcommand*{\glossarysection}[2][]{}%
\printglossary[type=\acronymtype,style=alttree,title={},nonumberlist,nogroupskip]
\endgroup

% --- Indizes ---
\section*{Indizes}
\addcontentsline{toc}{section}{Indizes}
\begingroup
\renewcommand*{\glossarysection}[2][]{}%
\printglossary[type=indices,style=linelesssymbunitlong,title={},nonumberlist,nogroupskip]
\endgroup

% --- Formelverzeichnis ---
\section*{Formelzeichen}
\addcontentsline{toc}{section}{Formelzeichen}
\begingroup
\renewcommand*{\glossarysection}[2][]{}%
\printglossary[type=formulas,style=linelesssymbunitlong,title={},nonumberlist,nogroupskip]
\endgroup




%%

%\printglossary[type=symbols,style={symbunitlong}, title={Symbolverzeichnis}]


\cleardoublepage
\listoffigures

\cleardoublepage
\listoftables

\raggedbottom % prefer having whitespace on bottom of pages (instead of stretching content to have common bottom line)
% This is required by the publisher

\mainmatter % Main contents. Tips:
% - Have separate .tex files per chapter, include them using \input{...}
% - When using versioning with git, try to start each sentence in a new line (this makes diffs cleaner)

%%%%%%%%%%%%%%%%%%%%%%%%%%%%%%%%%%%%%%%%%%%%%%%%%%%%%%%%%%%%%%%%%%%%%%%%%%%%%%%%%%%%%%%%%%%%%%%%%%%%%%%%%%%%%%%%%
%% Kapitel 1 - Einleitung %%%%%%%%%%%%%%%%%%%%%%%%%%%%%%%%%%%%%%%%%%%%%%%%%%%%%%%%%%%%%%%%%%%%%%%%%%%%%%%%%%%%%%%
%%%%%%%%%%%%%%%%%%%%%%%%%%%%%%%%%%%%%%%%%%%%%%%%%%%%%%%%%%%%%%%%%%%%%%%%%%%%%%%%%%%%%%%%%%%%%%%%%%%%%%%%%%%%%%%%%

\chapter{Einleitung}
Die Fähigkeit, die Zuverlässigkeit technischer Systeme und ingenieurwissenschaftlicher Erzeugnisse präzise vorherzusagen, ist vermehrt von zentraler, unternehmerischer Bedeutung.
Prognosen über die Lebensdauer und Ausfallwahrscheinlichkeit rücken damit in den Fokus strategischer Entscheidungen – von der Auslegung sicherheitskritischer Komponenten bis hin zur wirtschaftlichen Optimierung von Wartungsintervallen. Dabei ist die Modellbildung umso komplexer, je mehr Einflussgrößen simultan zu berücksichtigen sind. Sie übernimmt sogar eine zunehmend dominierende Rolle: Elektrifizierung, Gewichts- und Performance-Optimierung durch neuartige Material- und Strukturkonglomerate sowie Digitalisierung lassen vielfältige Expositionen gegenüber Beanspruchungen und Ausfallmechanismen realisieren, deren strukturelles Verhalten nicht ohne weiteres modellierbar ist. Genau hier setzt die multivariate Zuverlässigkeitsmodellierung an: sie erlaubt nicht nur eine realitätsnahe Abbildung des Systemverhaltens, sondern schafft die Voraussetzung für belastbare Aussagen über die Wirkung kombinierter Belastungsfaktoren gemäß klassischer Zuverlässigkeitsmethodik. Besonders in frühen Entwicklungsphasen, in denen physikalische Ausfallmechanismen (Physics of Failure, PoF) noch nicht vollständig verstanden sind, wird ein systematischer experimenteller Ansatz unverzichtbar.  Um dennoch belastbare Modelle zu erzeugen oder Nachweise zu erbringen, kommen beschleunigte Lebensdauertests (Accelerated Life Testing, ALT) zum Einsatz. Dabei werden gezielt erhöhte Belastungsniveaus appliziert, um Versagensprozesse zu initiieren, ohne den eigentlichen Ausfallmechanismus zu verfälschen. Die resultierenden Daten dienen der Generierung von Lebensdauermodellen, häufig mittels verallgemeinerter log-linearer Regressionsmodelle (GLMs)..

Da in der Regel mehrere kontinuierliche Einflussgrößen gleichzeitig auf das Versagensverhalten wirken, ist die Entwicklung eines statistisch fundierten Versuchsplans essenziell. Die Response Surface Methodology (RSM) bietet hierfür ein leistungsfähiges Instrumentarium, insbesondere durch die Anwendung zentral zusammengesetzter Versuchspläne wie dem Central Composite Design (CCD). Diese Designs ermöglichen durch ihre orthogonale und rotierbare Struktur eine unabhängige Schätzung linearer, quadratischer und interaktiver Effekte – ein entscheidender Vorteil für die Modellierung komplexer Systeme, deren Parameterraum zunächst nur unzureichend bekannt ist.

CCD zeichnet sich insbesondere durch seine Fähigkeit aus, gleichmäßige Vorhersagegenauigkeiten radial zum Zentrum des Versuchsraumes zu gewährleisten. Dies prädestiniert den Ansatz für explorative Teststrategien in Fällen mit hohem Unsicherheitsgrad, etwa bei neuartigen Technologien oder unvollständig erforschten Werkstoffverhalten. Im Vergleich zu anderen optimalen Designs, die spezifisches Vorwissen über relevante Einflussgrößen erfordern, liefert CCD eine robuste Ausgangsbasis für die strukturierte Erfassung von Wechselwirkungen und nichtlinearen Zusammenhängen. Dies ist von unschätzbarem Wert für Produkte, deren physikalische Ausfallmodelle noch nicht vollständig formuliert sind.

Die konsequente Anwendung dieser Methodik erfordert jedoch auch eine Erweiterung klassischer Modellierungsansätze. Insbesondere bei klinischen oder biomedizinischen Anwendungen – zunehmend aber auch bei technischen Systemen mit diskreten oder korrelierten Antwortgrößen – stoßen klassische lineare Modelle an ihre Grenzen. Hier bietet die Klasse der verallgemeinerten linearen Modelle (GLMs) ein erweitertes Rahmenwerk, das sowohl binäre als auch Poisson-verteilte oder andere nicht-normalverteilte Zielgrößen adäquat abbilden kann. Allerdings stellt die Abhängigkeit optimaler Versuchspläne von a priori unbekannten Modellparametern eine Herausforderung dar, die bislang in der Literatur nur begrenzt adressiert wurde.

Diese Arbeit entwickelt vor diesem Hintergrund ein umfassendes Modellkonzept, das die Prinzipien der Versuchsplanung mit der Analyse ökonomischer und qualitativer Zielgrößen in der Modellbildung verknüpft. Ziel ist es, eine strukturierte und effiziente Methodik zur Durchführung multivariater Lebensdauertests zu etablieren – als Grundlage für moderne, belastbare und praxisgerechte Zuverlässigkeitsprognosen.



Die Untersuchung von Zuverlässigkeit in der Ingenieurwissenschaft und der Produktentwicklung stellt eine essenzielle Herausforderung dar, insbesondere da zuverlässige technische Systeme und Produkte unter normalen Einsatzbedingungen Lebensdauern aufweisen können, die für Testzwecke unpraktisch sind. Um die Grundlage für Lebensdauermodelle zu schaffen oder die Zuverlässigkeit nachzuweisen, werden häufig beschleunigte Lebensdauertests (Accelerated Life Testing, ALT) durchgeführt.
Dabei werden Belastungsniveaus erhöht, um Versagensmechanismen in angemessener Zeit auszulösen, ohne dabei den Versagensmodus am Ende der Lebensdauer (End-of-Life, EoL) zu verändern. Anhand der Testergebnisse wird ein Regressionsmodell, häufig unter Verwendung verallgemeinerter log-linearer Modelle (GLM) und der Maximum-Likelihood-Methode (MLE), entwickelt, um ein Lebensdauer-Belastungs-Modell auf Basis empirischer Daten abzuleiten.
Zahlreiche Studien zeigen, dass die Modellierung der Lebensdauer häufig von mehreren kontinuierlichen Einflussfaktoren abhängt, was ein statistisches Versuchsdesign unerlässlich macht.
Die Response Surface Methodology (RSM) bietet dabei entscheidende Vorteile für experimentelle Designs im Zuverlässigkeitsengineering. Insbesondere factorial designs wie das Central Composite Design (CCD) sind aufgrund ihrer orthogonalen und rotierbaren Struktur besonders geeignet.
CCD ermöglicht die unabhängige Schätzung von Effekten und Wechselwirkungen zwischen mehreren Faktoren, was eine zentrale Voraussetzung für die Testung industrieller Anwendungen darstellt.
Die resultierenden Antwortflächen liefern zudem konstante Vorhersagegüten oder Vorhersagevarianzen, die radial vom Zentrum des experimentellen Designs ausgehen.
Dies ist ideal für die Entwicklung eines funktionsfähigen Zuverlässigkeitsmodells, insbesondere wenn der ideale Parameterraum für die Testung einer Technologie zu Beginn nicht bekannt ist – eine häufige Situation in der Praxis. CCD ordnet die Testpunkte auf effiziente Weise an und ermöglicht die Erfassung von Wechselwirkungen und quadratischen Effekten.
Dies ist besonders vorteilhaft bei technischen Produkten, bei denen die zugrunde liegenden physikalischen Versagensmechanismen (Physics of Failure, PoF) noch nicht vollständig verstanden sind.
Während alternative optimale Designs auf spezifische Leistungsanforderungen zugeschnitten sind, bietet CCD eine robuste Grundlage für Szenarien ohne Vorwissen.
Dies macht CCD zu einem wesentlichen Werkzeug für die Untersuchung von technischen Produkten, deren optimale Testbedingungen oder Zuverlässigkeitsmodelle erst entwickelt werden müssen.
Abschließend wird in dieser Arbeit ein umfassendes Modell vorgestellt, das die Prinzipien der Versuchsplanung mit der Analyse von Kosten und Qualität in der Modellbildung verbindet. Ziel ist es, eine methodische Grundlage für Zuverlässigkeitstests zu schaffen, die sowohl technische als auch ökonomische Anforderungen erfüllt.

\section{Forschungperspektive und Problembeschreibung}
\begin{itemize}
    \item {Ausgangssituation und Problemstellung}
\end{itemize}

\section{Beitrag dieser Arbeit}
\begin{itemize}
    \item {Ziele}
\end{itemize}

\section{Aufbau der Arbeit}
\begin{itemize}
    \item {Kapitelzusammensetzung und Struktur etc}
\end{itemize}
%%%%%%%%%%%%%%%%%%%%%%%%%%%%%%%%%%%%%%%%%%%%%%%%%%%%%%%%%%%%%%%%%%%%%%%%%%%%%%%%%%%%%%%%%%%%%%%%%%%%%%%%%%%%%%%%%
%% Kapitel 2 - Stand der Forschung und Technik %%%%%%%%%%%%%%%%%%%%%%%%%%%%%%%%%%%%%%%%%%%%%%%%%%%%%%%%%%%%%%%%%%
%%%%%%%%%%%%%%%%%%%%%%%%%%%%%%%%%%%%%%%%%%%%%%%%%%%%%%%%%%%%%%%%%%%%%%%%%%%%%%%%%%%%%%%%%%%%%%%%%%%%%%%%%%%%%%%%%
\chapter{Stand der Forschung und Technik} \label{chap:stand}

Dieses Kapitel stellt die für diese Arbeit erforderlichen technischen und methodischen Grundlagen bereit. Zunächst werde in Abschnitt~\ref{sec:zuv} zentrale Begriffe und Konzepte der Zuverlässigkeitstechnik sowie das grundlegende statistische Verfahren zur Lebensdauer-Datenanalyse in Kombination mit Versuchsplänen erläutert.
Darauf aufbauend folgen in Abschnitt~\ref{sec:doe} die Einführung und die Einordnung von \acs{DoE} sowie der multivariaten Lebensdauermodellierung aus dem Stand der Technik und der Wissenschaft, die beide für die Entwicklung effizienter Lebensdauerversuchspläne maßgeblich sind.
Im Kontext der Lebensdauererprobung umfasst dies insbesondere typische, statistische Versuchspläne sowie Metriken und Indikatoren zur allgemeinen Bewertung der Versuchspläne.

\section{Zuverlässigkeitstechnik und Wahrscheinlichkeitstheorie} \label{sec:zuv}
Die Zuverlässigkeitstechnik befasst sich mit der probabilistischen Beschreibung der Lebensdauer technischer Produkte und Systeme.
Ziel ist die statistische Modellierung des Ausfallverhaltens unter Berücksichtigung der Funktionalität des Produkts unter relevanten Randbedingungen.
Eine zentrale Aufgabe besteht somit in der statistischen Charakterisierung des Ausfallbegriffs mithilfe deskriptiver Statistik sowie in der Parametrisierung geeigneter Verteilungen zur Abbildung des Lebensdauerverhaltens.
Die Modellierung kann - abhängig von den Randbedingungen - auf Basis \textit{einer einzelnen} Belastungsgröße oder \textit{mehrerer} Beanspruchungsparameter erfolgen, die gemeinsam den Produktausfall determinieren.
Ein grundlegendes Verständnis des Umgangs mit zufallsverteilten Lebensdauerereignissen ist daher eine elementare Voraussetzung für die statistische Versuchsplanung im Rahmen der Zuverlässigkeitstechnik.
Weiterführende Konzepte und vertiefte methodische Ansätze zur Zuverlässigkeitstechnik sowie zur statistischen Testplanung sind allen voran in der Standardliteratur von Bertsche und Dazer \cite{Bertsche.2022} dargelegt, an deren Vorgehensweise sich die nachfolgenden Ausführungen orientieren.

\subsection{Begriffe und Definitionen} \label{subsec:begriffezuv}
Der \textbf{Ausfall} eines technischen Produkts bezeichnet den Zeitpunkt innerhalb seiner Lebensdauer, zu dem die geforderte Funktionalität unter definierten Umgebungs- und Randbedingungen nicht mehr erfüllt ist.
Die \textbf{Ausfallzeit}, welche diese Zustandsänderung zeitlich definiert, wird im Allgemeinen als kontinuierliche Zufallsvariable $\sym{tau}>0$ aufgefasst.
So ergibt sich die Wahrscheinlichkeit, dass ein Produkt im Zeitraum bis $\sym{t}$ einen Funktionsverlust erleidet, zu
\begin{equation} \label{eq:probdef}
    \sym{F}(\sym{t})=\sym{Pr}(\sym{tau}\leq \sym{t}).
\end{equation}
Diese Funktion beschreibt die \textbf{Ausfallwahrscheinlichkeit} $\sym{F}(\sym{t})$, während die \textbf{Zuverlässigkeit}
\begin{equation} \label{eq:reldef}
    \sym{R}(\sym{t})=\sym{Pr}(\sym{tau} > \sym{t}) = 1 - \sym{F}(\sym{t}) = \int_{-\infty}^{\sym{t}} \sym{f}(\sym{t}) \,d\sym{t} , \quad \sym{t} \geq 0.
\end{equation}
komplementär diejenige Wahrscheinlichkeit $\sym{R}(\sym{t}): \sym{RR}_{\geq 0} \rightarrow [0,1] \subset \sym{RR}$ quantifiziert, zu der das nicht reparierbare Produkt die realisierte Zeit $\sym{t}$ überlebt: also frei von Funktionsverlust bleibt und funktionsfähig ist \cite{Bertsche.2022,Birolini.2017,Meeker.2022,Yang.2007}.
Damit ist die Zuverlässigkeit mathematisch als reellwertige, monoton fallende und stetige Funktion definiert.
Die \textbf{Wahrscheinlichkeitsdichtefunktion} $\sym{f}(\sym{t})$ der Ausfallzeit beschreibt, wie sich die Wahrscheinlichkeiten der Ausfälle über der Zeit verteilen.
Sie folgt der Ableitung der Verteilungsfunktion $F(\sym{t})$ und kann als Maß für die Ausfallintensität pro Zeiteinheit interpretiert werden:
\begin{equation} \label{eq:pdfdef}
    \sym{f}(\sym{t}) = \frac{d}{d\sym{t}}\sym{F}(\sym{t}) = \frac{d}{d\sym{t}}\sym{Pr}(\sym{tau} \leq \sym{t}), \quad \sym{t} \geq 0.
\end{equation}
Damit beschreibt $\sym{f}(\sym{t})$ die lokale Änderungsrate der Zuverlässigkeit - also, wie schnell die Wahrscheinlichkeit der Funktionserfüllung respektive der Lebensdauer ab- bzw. zunimmt.
Das \textbf{Quantil} $\symsub{t}{q}$ einer Lebensdauer definiert den Zeitpunkt, zu dem die kumulative Ausfallwahrscheinlichkeit $\sym{F}(\sym{t})$ mindestens den Anteil $\sym{q}$ (bzw. das \textbf{Perzentil} in Prozentpunkten) erreicht:
\begin{equation} \label{eq:quantildef}
    \sym{F}(\symsub{t}{q}) = \sym{q}, \quad \sym{q} \in [0,1].
\end{equation}
Damit gibt das $\sym{q}$~-~Quantil denjenigen Lebensdauerwert an, unterhalb dessen der Anteil $\sym{q}$ aller betrachteten Produkte ausgefallen ist.

Ein spezieller und häufig verwendeter Fall ist der \textbf{Median}, also das Quantil zur Lebensdauer $\sym{t}_{0.5}$, bei dem die Eintrittswahrscheinlichkeit zu einem Ereignis bzw. die Ausfallwahrscheinlichkeit 50\,\% beträgt:
\begin{equation} \label{eq:mediandef}
    \sym{F}(\sym{t}_{0.5}) = 0.5.
\end{equation}
Der Median beschreibt somit den Zeitpunkt, zu dem die Hälfte aller Produkte ausgefallen ist.
Er stellt eine robuste, lagebezogene Kenngröße der zentralen Tendenz dar, da er - im Gegensatz zum \textbf{arithmetischen Mittelwert} - gegenüber Ausreißern und schiefen Verteilungen unempfindlich ist und auf der Verteilungsfunktion und nicht auf der Dichte basiert.
Während der Mittelwert hingegen sensitiv gegenüber Ausreißern ist, erfassen \textbf{Varianz} und \textbf{Standardabweichung} lediglich die Streuung der Zufallsvariablen um den Erwartungswert.





\subsection{Deskriptive Statistik für Lebensdauerdaten} \label{subsec:stat}

\subsection{Parameterschätzverfahren} \label{subsec:schätzer}

\section{Statistische Versuchsplanung und Modellbildung} \label{sec:doe}

\subsection{Grundbegriffe der statistischen Versuchsplanung} \label{subsec:begriffedoe}

\subsection{Statistische Lebensdauer-Versuchspläne} \label{subsec:pläne}

\subsection{Statistische Modellbildung} \label{subsec:model}

%%%%%%%%%%%%%%%%%%%%%%%%%%%%%%%%%%%%%%%%%%%%%%%%%%%%%%%%%%%%%%%%%%%%%%%%%%%%%%%%%%%%%%%%%%%%%%%%%%%%%%%%%%%%%%%%%
%% Kapitel 3 - Ansatz %%%%%%%%%%%%%%%%%%%%%%%%%%%%%%%%%%%%%%%%%%%%%%%%%%%%%%%%%%%%%%%%%%%%%%%%%%%%%%%%%%%%%%%%%%%
%%%%%%%%%%%%%%%%%%%%%%%%%%%%%%%%%%%%%%%%%%%%%%%%%%%%%%%%%%%%%%%%%%%%%%%%%%%%%%%%%%%%%%%%%%%%%%%%%%%%%%%%%%%%%%%%%

\chapter{Ansätze zur Effizienzsteigerung in der Planung von ausfallbasierten Lebensdauertests mit mehreren Faktoren} \label{chap:ansatz}

\section{Bewertung des Standes der Forschung und Technik}

\begin{itemize}
    \item \cite[Kap. 9.1.4]{Myers.2016}
    \item \cite[Kap. 9.2.1]{Myers.2016}
\end{itemize}

\subsubsection{Besonderheiten der Effizienz in der Lebensdauerprüfung}
Der Begriff der Effizienz erfährt im Kontext der Planung multivariater Lebensdauertests (\ac{ALT} oder Reliability Demonstration Tests) eine signifikante Erweiterung gegenüber der klassischen linearen Versuchsplanung.
Während Standard-Designs primär die Varianz der Parameterschätzer minimieren, unterliegen Lebensdauertests der zusätzlichen Restriktion, dass die Information (der Ausfall eines Bauteils) stochastisch über die Zeit generiert wird und oft durch \textbf{Zensierung} limitiert ist.

Eine zentrale Herausforderung besteht in der \textbf{Modellabhängigkeit} (engl. Model Dependence) der Informationsmatrix.
Bei nicht-linearen Modellen, wie der in der Zuverlässigkeitstechnik omnipräsenten Weibull- oder Lognormal-Regression, ist die Fisher-Informationsmatrix $\sym{FIM}$ nicht mehr allein von der Versuchsplanmatrix $\sym{X}$ abhängig, sondern auch von den wahren, aber unbekannten Verteilungsparametern $\sym{theta}$ (z.\,B. Formparameter $\sym{b}$):
\begin{equation}
    \sym{M}(\sym{X}, \sym{theta}) = \sym{E} \left[ - \frac{\partial^2 \sym{L_like}(\sym{theta})}{\partial \sym{theta}^2} \right].
    \label{eq:fisher_info_depend}
\end{equation}
Daraus resultiert das Paradoxon, dass zur Konstruktion eines optimalen Plans bereits Kenntnisse über die zu ermittelnden Parameter vorliegen müssen.
Klassische Optimalitätskriterien (D-, A-Optimalität) wandeln sich daher zu \textbf{lokalen Optimalitäten}, die nur für einen spezifischen Parametervektor $\sym{theta}_0$ („Best Guess“) gültig sind.
Um Robustheit gegenüber Fehlannahmen dieser Startwerte zu gewährleisten, werden in der Entwicklung effizienter Lebensdauertests häufig \textbf{Bayes-Optimale Versuchspläne} eingesetzt, welche die Effizienz über eine A-Priori-Verteilung der Parameter maximieren \cite{Meeker.2022, Goos.2011}.

Zudem muss die \textbf{zeitliche Effizienz} berücksichtigt werden.
Ein Versuchsplan gilt im Kontext der Lebensdaueranalyse nur dann als effizient, wenn er unter Berücksichtigung der Zensierungsmechanismen (Typ-I oder Typ-II) die erwartete Anzahl an Ausfällen maximiert oder die \textbf{erwartete Testdauer} (Expected Test Duration, ETD) bei gegebener Präzision minimiert.
Die Varianz der Schätzung wird hierbei maßgeblich durch die Anzahl der ausgefallenen Einheiten getrieben, nicht allein durch die Stichprobengröße $\sym{n}$ \cite{Nelson.2005}.


\section{Forschungsfragen und Aufbau der Arbeit}
%%%%%%%%%%%%%%%%%%%%%%%%%%%%%%%%%%%%%%%%%%%%%%%%%%%%%%%%%%%%%%%%%%%%%%%%%%%%%%%%%%%%%%%%%%%%%%%%%%%%%%%%%%%%%%%%%
%% Kapitel 4 - Parameter-Screening %%%%%%%%%%%%%%%%%%%%%%%%%%%%%%%%%%%%%%%%%%%%%%%%%%%%%%%%%%%%%%%%%%%%%%%%%%%%%%
%%%%%%%%%%%%%%%%%%%%%%%%%%%%%%%%%%%%%%%%%%%%%%%%%%%%%%%%%%%%%%%%%%%%%%%%%%%%%%%%%%%%%%%%%%%%%%%%%%%%%%%%%%%%%%%%%
\chapter{Parameter-Screening für multifaktorielle Lebensdauertests} \label{chap:screening}

Die in Kapitel~\ref{chap:ansatz} hergeleitete Notwendigkeit effizienter Testdesigns setzt voraus, dass die Anzahl der zu untersuchenden Faktoren $\sym{k}$ auf ein handhabbares Maß begrenzt ist.
Da die Komplexität und der Versuchsumfang exponentiell mit der Anzahl der Faktoren steigen (vgl. Gleichung~\ref{eq:ffvp_n}), ist eine präzise Vorselektion der Einflussgrößen entscheidend.
Klassische Ansätze der Versuchsplanung setzen hierfür oft auf experimentelle Screening-Pläne.
Im Kontext der Lebensdauererprobung führen diese jedoch zu einem Paradoxon: Um experimentell zu prüfen, ob ein Parameter die Lebensdauer beeinflusst, müssten gleichermaßen bei konventionellen Screeningdesigns (z.B. $2^{\sym{k}-\sym{p_f}}$ Designs, Plackett-Burman-Designs) sowie modernen \acp{RSD} (z.B. \ac{OMARS}-Designs) bereits zeitintensive \ac{EoL}-Tests durchgeführt werden, was den Effizienzvorteil des Screenings zunichtemacht.

Dieses Kapitel stellt daher einen effizienten methodischen Ansatz zum \textbf{heuristischen Screening} vor.
Ziel ist es, basierend auf Expertenwissen und systematischer Analyse eine qualitative Reduktion des Parameterraums vorzunehmen, \textit{bevor} physische Versuche gestartet werden \cite{Arndt.2023c}.
Dabei liegt der Fokus explizit auf der Unterscheidung zwischen bloßer Robustheit (zum Zeitpunkt $\sym{t}=0$) und echter Zuverlässigkeit (über die Zeit $\sym{t}>0$).
Aus der Perspektive der Zeitdomäne ist die differenzierte Klassifizierung der Einflussfaktoren für die Konzeption zuverlässigkeitstechnischer Untersuchungen von entscheidender Bedeutung.
Schließlich sind stochastische Degradationspfade entsprechend Abschnitt~\ref{subsec:begriffezuv} und die daraus resultierende Überlebenswahrscheinlichkeit $\sym{R}(\sym{t})$ nicht statisch zu betrachten, sondern zeitvarianten Interaktionsstrukturen und einer sich verschiebenden Effekthierarchie unterlegen.
Folglich kann das Parameterset, welches das Systemverhalten in fortgeschrittenen Phasen des Lebensdauerzyklus determiniert, signifikant von jenen Faktoren divergieren, die lediglich die initiale Performanceverteilung des Systems bei $\sym{t}=0$ dominieren.
Die Auswahl der relevanten Kovariaten ist somit inhärent dem Einfluss der Zeit ausgesetzt.
Um dieser Dynamik gerecht zu werden, synergiert der vorgestellte Ansatz teils bestens etablierte Kreativtechniken (z.B. Brainstorming, Fehlerbaum-Analysen - engl. \ac{FTA}, Ishikawa-Diagramme) mit strukturierten Bewertungswerkzeugen (z.B. \ac{DSM}, Grid-Analyse), um eine belastbare Identifikation und Priorisierung der kritischen Einflussgrößen zu gewährleisten.

\section{Identifikation potenzieller Einflussgrößen} \label{sec:screening_identifikation}
Die Basis eines jeden Screenings bildet die vollständige Erfassung aller potenziellen Einflussgrößen.
Um eine lückenlose Identifikation zu gewährleisten, ist eine strukturierte Systemanalyse unerlässlich, wie sie in der Zuverlässigkeitstechnik nach \textcite{Bertsche.2022} etabliert ist.
Ein zentrales Werkzeug hierfür ist das \textbf{Funktionsblockdiagramm}, engl. \textbf{\ac{FBD}} \cite{Lindemann.2008,Krallmann.2013,Gundlach.2004}.
Es abstrahiert das technische System auf seine Ein- und Ausgangsgrößen, klassifiziert nach den Flussgrößen \textit{Energie}, \textit{Stoff} und \textit{Signal} (vgl. Abbildung~\ref{fig:fbd_structure}) \cite{Pahl.2007}.
Innerhalb der Systemgrenzen werden Haupt- und Nebenfunktionen definiert, deren Nichterfüllung direkt zu potenziellen Ausfallmechanismen führt.
\begin{figure}[htbp]
    \centering
    \def\svgwidth{0.9\textwidth}
    % \import{plots/}{J22-arndt-fu-subfu.pdf_tex}
    \caption{Funktionsstruktur mit Haupt-, Neben- und Teilfunktionen zur Identifikation von Einflussgrößen (nach \textcite{Wallace.2007})}
    \label{fig:fbd_structure}
\end{figure}
Zur initialen Sammlung der Parameter (Informationsbeschaffung) eignen sich, aufbauend auf dem \ac{FBD}, entsprechend \textcite{Montgomery.2020b,Bruckner.2019} klassische Methoden des Qualitätsmanagements (\ac{Q7} and \ac{M7}) sowie etliche Kreativtechniken:
\begin{itemize}
    \item \textbf{Literaturrecherche:} Analyse bestehender Ausfallmodi und physikalischer Wirkmechanismen (z.B. Arrhenius, Wöhler) aus vergleichbaren Anwendungen aber auch simulativen Untersuchungen (\ac{FEA}) \cite{Breiing.1997}.
    \item \textbf{Brainstorming} bzw. \textbf{ABC-Brainstorming} ermöglichen es interdisziplinären Expertenteams, intuitives Erfahrungswissen zu explizieren \cite{Daenzer.2002,Mayers.1997}.
    \item \textbf{Delphi-Methode:} Durch mehrstufige, anonymisierte Befragungen von Experten können subjektive Einschätzungen objektiviert und Konsens über potenzielle Einflussgrößen erzielt werden \cite{Daenzer.2002}.
    \item \textbf{Morphologische Analyse:} Durch die Zerlegung des Systems in Teilfunktionen und die Variation von Lösungsprinzipien werden systematisch konstruktive Einflussparameter aufgedeckt \cite{Albers.2005,Thompson.1999}.
\end{itemize}
Das Ergebnis dieses Schrittes ist eine unsortierte, aber möglichst vollständige Liste (Parameter-Pool) aller Größen, die potenziell auf das System einwirken.

\section{Kreativmethoden zum Auswahlprozesse im Parameter-Screening} \label{sec:kreativmethoden}
{Kreativmethoden zum Auswahlprozess im Parameter-Screening} \label{sec:screening_kreativ}
Nach der Identifikation muss der Parameter-Pool zunächst strukturiert und anschließend bewertet werden, um die vitalen Einflussgrößen (\textit{vital few}) von den trivialen (\textit{trivial many}) zu trennen, vergleiche \textcite{Arndt.2023c}.
Insbesondere basierend auf Arbeiten von \textcite{Gundlach.2004,Kremer.2018b,Mayers.1997} werden im Folgenden geeignete Methoden vorgestellt, die sich für das heuristische Screening im Kontext von zeitvarianten Parametersets in der Lebensdaueranalyse eignen und darüber hinaus auf Randbedingungen der Versuchsplanung Rücksicht nehmen.
\subsection{Strukturierungsmethoden} \label{subsec:Strukturierungsmethoden}
Nachdem durch die vorangegangenen Schritte der Informationsbeschaffung eine möglichst vollständige Sammlung aller potenziellen Systemparameter, Inputs und Outputs generiert wurde, bedarf es einer systematischen Strukturierung auf Basis eines vertieften Systemverständnisses.
Dieser Schritt darf sich nicht auf eine reine Clusterung in Parametergruppen beschränken.
Vielmehr gilt es, die \textit{gegenseitigen Beeinflussungen} der Größen zu identifizieren und zu offenbaren.
Insbesondere für die Lebensdaueranalyse stellt dies einen entscheidenden Mehrwert dar, der im Einklang mit den Prinzipien des \ac{L-DoE} steht \cite{Montgomery.2020,Kremer.2021}: Phänomene wie Alterungseffekte oder die zeitvariante Änderung von Materialeigenschaften sind physikalisch oft nicht trivial durch Einzelgrößen beschreibbar, sondern resultieren maßgeblich aus Interaktionen (vgl. Abbildung~\ref{fig:degradation_scheme}).
Um diese komplexen, teils zeitabhängigen Zusammenhänge für die weitere Selektion greifbar zu machen, eignen sich primär grafische Methoden, die über die reine Auflistung hinausgehen:

\begin{itemize}
    \item \textbf{Affinitätsdiagramm:} Als Werkzeug der \textit{M7} dient es der thematischen Clusterung der oft unstrukturierten Ergebnisse aus Brainstorming-Sessions \cite{Bruckner.2019}. Es ordnet Parameter übergeordneten Kategorien zu und deckt erste logische Gruppierungen auf, womit es als ideale Vorstufe für detailliertere Analysen (z.B. Ishikawa) fungiert.

    \item \textbf{Mind-Mapping:} Diese Methode ermöglicht eine hierarchische Gliederung der Einflussgrößen ausgehend vom Untersuchungsziel (Wurzel) über Haupt- zu Nebenparametern (Äste). Der entscheidende Vorteil für das Screening liegt in der Möglichkeit, \textit{Querbeziehungen} durch Verbindungslinien zwischen den Ästen zu visualisieren, wodurch Interdependenzen abseits der direkten Hierarchie offensichtlich werden \cite{Buzan.2010}.

    \item \textbf{Erweitertes Ishikawa-Diagramm:} Das klassische Ursache-Wirkungs - Diagramm (Fischgräten-Diagramm) strukturiert Parameter traditionell nach $6M$ (Mensch, Maschine, Material, Methode, Mitwelt, Messung). Für das heuristische Screening wird es dahingehend modifiziert, dass nicht nur statische Haupteffekte, sondern explizit \textbf{Interaktionen} durch \textit{Querverbindungen} zwischen den Ästen visualisiert werden (vgl. Abbildung~\ref{fig:ishikawa_interactions}) \cite{Arndt.2022,Daenzer.2002}.

    \item \textbf{Interdependenz-Netzwerke (Vernetztes Denken):} Für hochkomplexe Systeme bietet sich die Modellierung als gerichteter Graph an. Hierbei werden Parameter als Knoten und ihre Wirkbeziehungen als Kanten dargestellt \cite{Tittmann.2019}. Dies erlaubt nicht nur die Abbildung der Wirkrichtung (positiv/negativ), sondern auch die Integration zeitlicher Dynamiken und Intensitäten, was für die Modellierung von Degradationsprozessen vorteilhaft ist.

    \item \textbf{ABC-Analyse:} Zur Reduktion der Komplexität klassifiziert die ABC-Analyse die Parameter nach dem Pareto-Prinzip in Klassen hoher (A), mittlerer (B) und geringer (C) Relevanz. Dies dient als erster Filter, um den Fokus auf die vermuteten Haupttreiber der Lebensdauer zu lenken \cite{Montgomery.2020}.
\end{itemize}

\begin{figure}[htbp]
    \centering
    \def\svgwidth{0.95\textwidth}
    % \import{plots/}{J22-arndt-Ishi.pdf_tex}
    \caption{Erweitertes Ishikawa-Diagramm zur Visualisierung von Interaktionen zwischen Einflussfaktoren (adaptiert nach \cite{Arndt.2022})}
    \label{fig:ishikawa_interactions}
\end{figure}

Die hierdurch erreichte Transparenz über die Vernetzung der Parameter bildet die notwendige Basis für den nächsten Schritt: Die Überführung der qualitativen Struktur in eine quantitative Priorisierung, um die finalen Faktoren für den Versuchsplan zu selektieren.

\subsection{Bewertungsmethoden (Decision Making)} \label{subsec:Decision}
Aufbauend auf der qualitativen Strukturierung der Einflussgrößen (vgl. Abschnitt \ref{subsec:Strukturierungsmethoden}) ist nun eine analytische Bewertung erforderlich, um die \textit{Vital Few} von den \textit{Trivial Many} zu separieren.
Ziel ist es, die gesammelten Parameter in eine Rangfolge zu bringen, die sowohl deren vermutete Relevanz für die Lebensdauer als auch deren Interaktionspotenzial widerspiegelt.
Hierfür eignen sich matrixbasierte Ansätze, die eine systematische Paarvergleichung erzwingen und subjektive Einschätzungen objektivierbar machen.
\subsubsection{Design-Structure-Matrix (\ac{DSM})} \label{subsubsec:dsm}
Die \ac{DSM} ist eine quadratische $\sym{n} \times \sym{n}$-Matrix, in der alle $\sym{n}$ identifizierten Systemparameter sowohl in den Zeilen als auch in den Spalten aufgetragen sind.
Ein Eintrag $\sym{y}_{\sym{idx_i}\sym{idx_j}}$ in der Matrix symbolisiert dabei eine gerichtete Abhängigkeit: Der Parameter in Zeile $\sym{idx_i}$ beeinflusst den Parameter in Spalte $\sym{idx_j}$.
Für das Screening im Kontext der Zuverlässigkeit wird die Matrix häufig um eine Zielgrößen-Spalte erweitert, um den direkten Einfluss auf die Lebensdauer $\sym{t}$ zu erfassen.
Die Bewertung kann auf zwei Detailebenen erfolgen:
\begin{itemize}
    \item \textbf{Binäre DSM:} Erfasst lediglich die Existenz einer Interaktion ($\sym{y}_{\sym{idx_i}\sym{idx_j}} \in \{0, 1\}$). Dies eignet sich für frühe Phasen mit geringem Detailwissen, um prinzipielle Vernetzungen aufzudecken.
    \item \textbf{Numerische DSM:} Gewichtet die Stärke des Einflusses, z.B. auf einer Skala von 0 (kein Einfluss) bis 3 (starker Einfluss). Dies erlaubt eine differenzierte Priorisierung.
\end{itemize}
Ein wesentlicher Vorteil der \ac{DSM} ist die Aufdeckung von Asymmetrien: Ein Parameter A kann Parameter B stark beeinflussen, ohne dass B signifikant auf A zurückwirkt ($\sym{y}_{\text{AB}}\neq\sym{y}_{\text{BA}}$).
Solche "aktiven" Parameter sind potenzielle Steuergrößen für den Versuchsplan.
Zudem ermöglicht die Matrix durch algorithmische Partitionierung (Clustering) das Identifizieren von Parametergruppen, die eng miteinander interagieren, was Hinweise auf physikalische Wirkmechanismen liefert \cite{Lindemann.2008}.
\begin{figure}[htbp]
    \centering
    \def\svgwidth{0.8\textwidth}
    % \import{plots/}{J22-arndt-B-DSM.pdf_tex}
    \caption{Beispielhafte Darstellung einer binären DSM zur Identifikation von Interaktionen zwischen Systemparametern (adaptiert nach \cite{Arndt.2022})}
    \label{fig:dsm_example}
\end{figure}
\subsubsection{Grid-Analyse (Portfolio-Analyse)} \label{subsubsec:grid}
Die Grid-Analyse visualisiert die Ergebnisse der numerischen \ac{DSM} in einem zweidimensionalen Portfolio-Diagramm (vgl. Abbildung~\ref{fig:grid_analysis}).
Hierfür werden für jeden Parameter $\sym{idx_i}$ zwei Kennzahlen berechnet \cite{Mayers.1997}:
\begin{enumerate}
    \item \textbf{Aktivsumme (Active Sum):} Die Summe der Zeileneinträge ($\sum_{\sym{idx_j}} \sym{y}_{\sym{idx_i}\sym{idx_j}}$). Sie ist ein Maß dafür, wie stark der Parameter das Gesamtsystem treibt.
    \item \textbf{Passivsumme (Passive Sum):} Die Summe der Spalteneinträge ($\sum_{\sym{idx_i}} \sym{y}_{\sym{idx_i}\sym{idx_j}}$). Sie beschreibt, wie stark der Parameter selbst von anderen Größen beeinflusst wird (Reaktivität).
\end{enumerate}
Die Positionierung im Diagramm erlaubt eine Klassifizierung in vier Quadranten, aus denen sich direkte Handlungsempfehlungen für das \ac{DoE} ableiten lassen:
\begin{itemize}
    \item \textbf{Aktive Parameter (hohe Aktiv-, niedrige Passivsumme):} Diese Größen sind die idealen Steuerfaktoren ($\sym{x}$) für den Versuchsplan, da sie das System dominieren, ohne selbst instabil zu sein.
    \item \textbf{Kritische Parameter (hohe Aktiv- und Passivsumme):} Diese Faktoren sind stark vernetzt. Sie sind relevant für die Lebensdauer, bergen aber aufgrund ihrer Abhängigkeiten ein hohes Risiko für unerwünschte Wechselwirkungen. Sie müssen im Versuch besonders genau überwacht werden.
    \item \textbf{Passive Parameter (niedrige Aktiv-, hohe Passivsumme):} Diese Größen eignen sich weniger als Steuerfaktoren, sondern vielmehr als Indikatoren oder Antwortgrößen ($\sym{y}$), da sie empfindlich auf Änderungen im System reagieren.
    \item \textbf{Träge Parameter (niedrige Summen):} Diese Faktoren spielen eine untergeordnete Rolle und können im Screening oft vernachlässigt oder als Konstanten fixiert werden.
\end{itemize}Um die Unsicherheit heuristischer Schätzungen abzubilden, kann die Grid-Analyse um Konfidenz-Vektoren erweitert werden.
Dabei geben Experten für jede Bewertung an, ob diese auf gesichertem Wissen (Daten, Literatur) oder Intuition beruht.
Große Differenzen zwischen \textit{gesicherter} und \textit{intuitiver} Position im Grid weisen auf Wissenslücken hin, die zwingend durch experimentelle Voruntersuchungen geschlossen werden müssen, bevor der Parameter in ein aufwendiges \ac{L-DoE} aufgenommen wird.
\begin{figure}[htbp]
    \centering
    \def\svgwidth{0.8\textwidth}
    % \import{plots/}{J22-arndt-Grid.pdf_tex}
    \caption{Grid-Analyse zur Klassifizierung von Parametern in aktive, kritische, passive und träge Faktoren}
    \label{fig:grid_analysis}
\end{figure}
\subsubsection{Finale Parameter-Diskussion}
Die Ergebnisse aus \ac{DSM} und Grid-Analyse dienen als Entscheidungsgrundlage für die finale Selektion der Versuchs-Faktoren.
In diesem Diskurs müssen die heuristischen Erkenntnisse gegen die harten Restriktionen des Versuchsplans (Budget, Machbarkeit, Messbarkeit) abgewogen werden.
Es gilt, einen Kompromiss zu finden zwischen der vollständigen Abbildung aller wirksamen Mechanismen (Modellgüte) und der Reduktion auf eine handhabbare Anzahl an Faktoren (Effizienz).
Faktoren, die als relevant identifiziert wurden, aber aus Budgetgründen nicht variiert werden können, müssen explizit als Konstanten dokumentiert oder als Noise-Faktoren in einer Robustheitsbetrachtung berücksichtigt werden.

\section{Randbedingungen in der Parameterauswahl für die Zuverlässigkeitsmodellierung} \label{sec:screening_randbedingungen}
Die Anwendung der oben genannten Methoden muss im Kontext der Lebensdaueranalyse spezifische Randbedingungen berücksichtigen, die sich fundamental von der klassischen Robustheitsoptimierung unterscheiden.

\subsection{Unterscheidung: Robustheit vs. Zuverlässigkeit}
Ein häufiges Missverständnis im Screening ist die Gleichsetzung von initialer Performance-Streuung mit Lebensdauer-Streuung.
Wie in Abbildung~\ref{fig:degradation_scheme} dargestellt, betrachtet die klassische Robustheitsanalyse nach \textcite{Klein.2014} im Verständnis von \textit{Taguchi}/\textit{Shainin} oft nur die Verteilung der Systemantwort zum Zeitpunkt $\sym{t}=0$ (Bereich A).
Die Zuverlässigkeitstechnik fokussiert jedoch auf die \textbf{Degradation} über die Zeit (Bereich B).
Ein Faktor, der die initiale Performance kaum beeinflusst (z.B. Korrosionsschutzschicht-Dicke), kann für die Lebensdauer $\sym{t}$ dominant sein.
Heuristisches Screening für \ac{L-DoE} muss daher explizit Parameter priorisieren, die \textit{zeitabhängige} Schädigungsmechanismen (Verschleiß, Alterung, Ermüdung) treiben, selbst wenn sie initial "inaktiv" erscheinen.

\begin{figure}[htbp]
    \centering
    \def\svgwidth{0.9\textwidth}
    % \import{plots/}{J22-arndt-degra.pdf_tex}
    \caption{Abgrenzung zwischen initialer Performance-Verteilung (A) und zeitabhängiger Degradation (B) im Screening-Prozess}
    \label{fig:degradation_scheme}
\end{figure}

\subsection{Anforderungen aus dem Versuchsplan (L-DoE)}
Die im heuristischen Prozess ausgewählten Faktoren müssen später im statistischen Versuchsplan (\ac{DoE}) verarbeitet werden. Daraus ergeben sich harte Restriktionen für die Auswahl:
\begin{itemize}
    \item \textbf{Einstellbarkeit (Controllability):} Nur Parameter, die im Versuch aktiv und präzise auf verschiedene Niveaus (Levels) eingestellt werden können, qualifizieren sich als Faktoren. Nicht steuerbare Größen müssen als Noise oder Co-Faktoren behandelt werden.
    \item \textbf{Beschleunigbarkeit:} Für \ac{ALT} müssen Faktoren gewählt werden, die eine physikalische Beschleunigung der Schädigung ermöglichen (z.\,B. Temperatur, Spannung), ohne den Fehlermechanismus zu verändern.
    \item \textbf{Kosten und Komplexität:} Da der Stichprobenumfang bei voll-faktoriellen Plänen mit $\sym{n} \propto \sym{m}^{\sym{k}}$ wächst, zwingt die Ökonomie zur drastischen Reduktion von $\sym{k}$. Das Screening muss daher einen "Cut-Off" definieren, der oft bei $\sym{k} \le 5 \dots 7$ Faktoren liegt.
    \item \textbf{Design Resolution (Auflösung):} Wenn viele Faktoren im Screening verbleiben, müssen fraktionelle Pläne (Teilfaktoriell) genutzt werden. Das Screening muss vorab klären, welche Interaktionen physikalisch plausibel sind, um eine geeignete Auflösung (Resolution III, IV oder V) zu wählen und Aliasing von Haupteffekten mit wichtigen Wechselwirkungen zu vermeiden.
\end{itemize}

\section{Vorgehen zum heuristischen Screening für die Zuverlässigkeitsmodellierung} \label{sec:screening_vorgehen}
Basierend auf der Analyse der Methoden und Randbedingungen wird folgendes prozedurales Vorgehen für das Parameter-Screening in der Zuverlässigkeitsmodellierung vorgeschlagen (vgl. Abbildung~\ref{fig:screening_procedure}).
Dieser Ansatz stellt eine "Firewall" vor den eigentlichen kostenintensiven Lebensdauerversuch dar.

\begin{figure}[htbp]
    \centering
    \def\svgwidth{0.95\textwidth}
    % \import{plots/}{J22-arndt-Procedure.pdf_tex}
    \caption{Methodischer Ablauf des heuristischen Screenings als Vorstufe zum L-DoE}
    \label{fig:screening_procedure}
\end{figure}

\subsection*{Phase I: Systemanalyse und Failure-Mode-Definition}
Der Prozess beginnt nicht mit den Parametern, sondern mit dem Fehlermechanismus.
Basierend auf dem \ac{FBD} (Abschnitt~\ref{sec:screening_identifikation}) wird definiert, welcher spezifische \textbf{Failure Mode} untersucht werden soll (z.\,B. "Zahnbruch" vs. "Abrieb" bei einem Riemen).
Unterschiedliche Fehlermodi werden oft von unterschiedlichen Parametersets getrieben.
Eine Vermischung führt zu Rauschen in den Daten.

\subsection*{Phase II: Heuristische Filterung}
Im zweiten Schritt wird der Parameter-Pool durch die in Abschnitt~\ref{sec:screening_kreativ} vorgestellten Methoden (z.\,B. Ishikawa + Grid-Analyse) gefiltert.
Kritisch ist hier die Bewertung der \textbf{Interaktionsdichte}.
Faktoren, die im Ishikawa-Diagramm oder der \ac{DSM} viele Vernetzungen aufweisen (hohe Aktivsumme), werden priorisiert, da sie wahrscheinlich Katalysatoren für Degradationsprozesse sind.
Das Ergebnis ist ein Ranking der Faktoren nach ihrer vermuteten Relevanz für die Lebensdauer.

\subsection*{Phase III: Test-Design-Matching}
Im letzten Schritt wird die Top-Liste der Faktoren gegen die Restriktionen des gewählten \ac{DoE}-Plans (siehe Kapitel 3 und Abschnitt~\ref{sec:screening_randbedingungen}) abgeglichen.
\begin{itemize}
    \item Ist die Anzahl $\sym{k}$ kompatibel mit dem Budget für $\sym{n}$ Versuche?
    \item Sind die Faktoren physikalisch unabhängig voneinander einstellbar (Orthogonalität)?
    \item Erlauben die gewählten Stufenabstände eine signifikante Änderung der Lebensdauer (Power $> 80\%$)?
\end{itemize}
Nur Faktoren, die diesen Filter passieren, werden in das finale \ac{L-DoE}-Modell aufgenommen.
Parameter, die als "mittelwichtig" eingestuft wurden aber aus Budgetgründen entfallen, werden als konstante Randbedingungen dokumentiert oder als Noise-Faktoren in Robustheits-Analysen ausgelagert.

\section{Zusammenfassung}
Das in diesem Kapitel vorgestellte heuristische Screening schließt die Lücke zwischen der qualitativen Systemanalyse und der quantitativen Versuchsplanung.
Indem Expertenwissen strukturiert genutzt wird (z.\,B. durch \ac{DSM} und erweiterte Ishikawa-Diagramme), kann die Anzahl der Versuchs-Faktoren $\sym{k}$ drastisch reduziert werden, ohne relevante Interaktionen zu übersehen.
Entscheidend ist dabei der Perspektivwechsel von der initialen Performance-Streuung hin zur zeitabhängigen Degradation.
Dieses Vorgehen ermöglicht es, die in Kapitel 3 diskutierten effizienten Versuchspläne (wie CCD oder OMARS) überhaupt erst anwendbar zu machen, da diese nur mit einer begrenzten Anzahl an Faktoren ($\sym{k} \approx 3 \dots 7$) wirtschaftlich operieren.
Die methodische Vorarbeit des Screenings ist somit der Hebel für die Effizienz des gesamten Lebensdauerversuchs.
%%%%%%%%%%%%%%%%%%%%%%%%%%%%%%%%%%%%%%%%%%%%%%%%%%%%%%%%%%%%%%%%%%%%%%%%%%%%%%%%%%%%%%%%%%%%%%%%%%%%%%%%%%%%%%%%%
%% Kapitel 4 - Entwurf %%%%%%%%%%%%%%%%%%%%%%%%%%%%%%%%%%%%%%%%%%%%%%%%%%%%%%%%%%%%%%%%%%%%%%%%%%%%%%%%%%%%%%%%%%
%%%%%%%%%%%%%%%%%%%%%%%%%%%%%%%%%%%%%%%%%%%%%%%%%%%%%%%%%%%%%%%%%%%%%%%%%%%%%%%%%%%%%%%%%%%%%%%%%%%%%%%%%%%%%%%%%


\chapter{Effiziente multivariate Versuchspläne für Lebensdaueruntersuchungen} \label{chap:entwurf}
%%%%%%%%%%%%%%%%%%%%%%%%%%%%%%%%%%%%%%%%%%%%%%%%%%%%%%%%%%%%%%%%%%%%%%%%%%%%%%%%%%%%%%%%%%%%%%%%%%%%%%%%%%%%%%%%%
%% Kapitel 6 - CaseStudy %%%%%%%%%%%%%%%%%%%%%%%%%%%%%%%%%%%%%%%%%%%%%%%%%%%%%%%%%%%%%%%%%%%%%%%%%%%%%%%%%%%%%%%%
%%%%%%%%%%%%%%%%%%%%%%%%%%%%%%%%%%%%%%%%%%%%%%%%%%%%%%%%%%%%%%%%%%%%%%%%%%%%%%%%%%%%%%%%%%%%%%%%%%%%%%%%%%%%%%%%%

\chapter{Fallstudie} \label{chap:casestudy}
%%%%%%%%%%%%%%%%%%%%%%%%%%%%%%%%%%%%%%%%%%%%%%%%%%%%%%%%%%%%%%%%%%%%%%%%%%%%%%%%%%%%%%%%%%%%%%%%%%%%%%%%%%%%%%%%%
%% Kapitel 7 - Zusammenfassung %%%%%%%%%%%%%%%%%%%%%%%%%%%%%%%%%%%%%%%%%%%%%%%%%%%%%%%%%%%%%%%%%%%%%%%%%%%%%%%%%%
%%%%%%%%%%%%%%%%%%%%%%%%%%%%%%%%%%%%%%%%%%%%%%%%%%%%%%%%%%%%%%%%%%%%%%%%%%%%%%%%%%%%%%%%%%%%%%%%%%%%%%%%%%%%%%%%%

\section{Key Findings}
\section{Diskussion}
\section{Ausblick}


% TODO: nocite prints the whole bib file to the bibliography. This is for testing and should be commented out in your thesis:
% \nocite{*}

% Bibliography
\setlength{\emergencystretch}{.5em}
%% Force page breaks in bibliography before splitting single entries over multiple pages
%\patchcmd{\bibsetup}{\interlinepenalty=5000}{\interlinepenalty=10000}{}{}
\printbibliography[title={Literatur}]





% Appendix, if needed:
%% This uses \section and below to have everything grouped in the virtual chapter "Anhang" in the table of contents. 
\startSMBAppendix

\addcontentsline{toc}{chapter}{Anhang}
\section{Ein Anhang}
\subsection{Unteranhang 1}

\end{document}