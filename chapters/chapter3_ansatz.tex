%%%%%%%%%%%%%%%%%%%%%%%%%%%%%%%%%%%%%%%%%%%%%%%%%%%%%%%%%%%%%%%%%%%%%%%%%%%%%%%%%%%%%%%%%%%%%%%%%%%%%%%%%%%%%%%%%
%% Kapitel 3 - Ansatz %%%%%%%%%%%%%%%%%%%%%%%%%%%%%%%%%%%%%%%%%%%%%%%%%%%%%%%%%%%%%%%%%%%%%%%%%%%%%%%%%%%%%%%%%%%
%%%%%%%%%%%%%%%%%%%%%%%%%%%%%%%%%%%%%%%%%%%%%%%%%%%%%%%%%%%%%%%%%%%%%%%%%%%%%%%%%%%%%%%%%%%%%%%%%%%%%%%%%%%%%%%%%
\chapter{Effiziente Testplanung für die multivariate Lebensdauererprobung} \label{chap:ansatz}

\section{Bewertung des Standes der Forschung und Technik} \label{sec:bewertung_stand}
Der in Kapitel~\ref{chap:stand} dargelegte Stand zur Versuchsplanungsmethodik verdeutlicht, dass sowohl die Zuverlässigkeitstechnik als auch \ac{DoE} für sich genommen etablierte Disziplinen mit jeweils umfangreichen Methodenrepertoire darstellen.
Die Schnittmenge beider Felder, die effiziente Planung multivariater Lebensdauerversuche unter Berücksichtigung nicht-normalverteilter Daten und Zensierung (Weibull-\ac{GLM}), weist jedoch spezifische Lücken auf, die den Bedarf an weiterführender Forschung begründen.

\subsection{Eignung klassischer Response Surface Designs}
Klassische Versuchspläne der \ac{RSM}, insbesondere das \ac{CCD}, haben sich als robuster Standard für die Modellierung nicht-linearer Zusammenhänge bewährt.
Ihre Stärken liegen in der sequenziellen Augmentierbarkeit, der Rotierbarkeit und der Fähigkeit zur unabhängigen Schätzung des reinen Fehlers (vgl. Abschnitt~\ref{subsubsec:rsm}).
Für die spezifischen Anforderungen der Lebensdauererprobung, die häufig eine Extrapolation der Ergebnisse aus einem hochbelasteten \ac{ALT}-Testraum in einen niedriger belasteten Anwendungsraum erfordert, ist die starre Symmetrie des klassischen \ac{CCD} jedoch nicht zwangsläufig optimal.
Es existieren zwar bereits vereinzelte Ansätze zur Modifikation von \acp{CCD}, wie die Arbeiten von \textcite{Ahn.2015, Donev.2004, G.E.P.Box.1951} und \textcite{Ardakani.2011} zeigen.
Diese fokussieren sich jedoch zumeist auf generische Optimalitätskriterien oder spezifische geometrische Restriktionen und adressieren nicht explizit die Herausforderungen der probabilistischen Lebensdauerprognose unter Extrapolation im Kontext der statistischen \textbf{Trennschärfe} (\sym{power}).
Eine systematische Methodik zur gezielten Manipulation einzelner Versuchspunktkategorien (vollfaktorielle Punkte, Zentralpunkte, Sternpunkte) in Prognoserichtung zur Steigerung der Extrapolationsgüte bei gleichzeitiger Wahrung der Modelleigenschaften und Aufwandsstrukturen fehlt bislang weitgehend - obwohl dies aus praktischer, ingenieurwissenschaftlicher Sicht im Maschinen- und Fahrzeugbau von hoher Relevanz ist.

\subsection{Limitierungen und Effizienzbegriffe moderner Versuchsplanung}
Die Anwendung computergenerierter optimaler Versuchspläne erscheint zunächst als logische Konsequenz, um maßgeschneiderte Designs für Lebensdauererprobungen zu entwerfen.
Der Begriff der \textbf{Effizienz} erfährt hierbei jedoch im Kontext multivariater Lebensdauertests (\ac{ALT}) eine signifikante Komplexitätssteigerung gegenüber der klassischen linearen Versuchsplanung, die sich in drei Dimensionen gliedert: Modellabhängigkeit, zeitliche Dynamik und strukturelle Flexibilität.
Grundlegend definiert sich die Effizienz eines Versuchsplans einerseits dennoch spezifisch über $\sym{M}(\sym{X}, \sym{theta})$ \cite{Khuri.2006,Stufken.2012}, andererseits generisch über die Leistungsfähigkeit zur Informationsbeschaffung, wenngleich sie im Grundsatz widersprüchlich zum Grundsatz der Vollständigkeit steht \cite{Thommen.2022}.
Sie trägt damit einen unmittelbaren Einfluss auf die \textbf{Effektivität} und Wirtschaftlichkeit der Lebensdauererprobung - in letzter Instanz damit auf die Verlässlichkeit der Modellierung (vgl. Abschnitt~\ref{subsec:optimal}: \sym{power}) und Prognosefähigkeit der Zuverlässigkeit $\sym{R}(\sym{t})$.

\subsubsection{Modellabhängigkeit und Lokale Optimalität}
Erstens stößt die zugrundeliegende Modellierung mittels Weibull-\acp{GLM} auf ein fundamentales Problem, das von \textcite{Khuri.2006} und \textcite{Stufken.2012} diskutiert wird.
Im Gegensatz zu klassischen linearen Modellen (unter Normalverteilungsannahme) hängt die Fisher-Informationsmatrix $\sym{FIM}$ bei \acp{GLM} nicht allein von den Einstellungen der Faktoren $\sym{X}$ ab, sondern vollständig von den wahren, aber unbekannten Modellparametern $\sym{theta}$ (insbesondere dem Formparameter $\hat{\sym{b}}$ und via der Link-Funktion dem Lageparameter $\hat{\sym{T}}$):
\begin{equation}
    \sym{M}(\sym{X}, \sym{theta}) = \sym{E} \left[ - \frac{\partial^2 \sym{L_like}(\sym{theta})}{\partial \sym{theta}^2} \right].
    \label{eq:fisher_info_depend}
\end{equation}
Daraus resultiert das Paradoxon, dass zur Konstruktion eines optimalen Plans bereits Kenntnisse über die zu ermittelnden Parameter vorliegen müssten.
Zwar existieren auf Basis von Exponential- oder Poissonverteilung diverse Algorithmen zur numerischen Bestimmung optimaler Versuchspläne für \acp{GLM} \cite{Khuri.2006}, und auch für Experimente mit binärer Antwortstruktur (logistische oder Probit-Modelle) wurden Ansätze entwickelt \cite{Yang.2011}.
Diese Verfahren setzen jedoch stets voraus, dass die wahren Parameterwerte $\sym{theta}$ bekannt sind oder verlässlich geschätzt (\textit{best guess}) werden können.
Folglich sind solche Designs, wie \textcite{Stufken.2012} ausführen, allenfalls nur \textbf{lokal optimal}.
Da die Parameter im Vorfeld einer Lebensdaueruntersuchung unbekannt sind, existieren keine universell optimalen \ac{GLM}-Designs für die Weibull-Lebensdauermodellierung.
Ein Design, das für einen angenommenen Parametersatz optimal ist, kann sich bei Abweichung der realen Werte als ineffizient erweisen („The designs may be poor if the choice of values is far from the true parameter values“, \textcite{Stufken.2012}).

\subsubsection{Zeitliche und Strukturelle Effizienz}
Zweitens muss die \textbf{zeitliche Effizienz} berücksichtigt werden.
Da die Information in Lebensdauertests stochastisch über die Zeit generiert wird und oft durch Zensierung limitiert ist, wird die Varianz der Schätzung maßgeblich durch die Anzahl der tatsächlich ausgefallenen Einheiten getrieben \cite{Nelson.2005}.
Ein Versuchsplan gilt folglich nur dann als effizient, wenn er die erwartete Testdauer bei gegebener Präzision minimiert.
Drittens bietet die Klasse der \ac{OMARS}-Designs eine Antwort auf die Forderung nach \textbf{struktureller Effizienz} und Flexibilität.
Im direkten Vergleich zum etablierten \ac{CCD}, das durch seine geometrische Konstruktion eine starre Anzahl an Versuchen fordert, ermöglichen \ac{OMARS}-Designs eine Entkopplung von Faktoren- und Versuchsanzahl (z.B. $\sym{N}=40$ statt $\sym{N} \ge 45$ für $\sym{k}=6$).
Strategisch verfolgen sie einen \textit{One-Step}-Ansatz \cite{Goos.2025}, der Screening und Optimierung integriert, um Rüstzeiten und zeitliche Drifts zu vermeiden.
Der Preis für diese Effizienz ist ein kontrolliertes Maß an \textit{Aliasing} zwischen Effekten zweiter Ordnung.
Damit qualifizieren sich diese Designs insbesondere für ressourcenbeschränkte Untersuchungen, wenngleich das \ac{CCD} seine Stärke in der Robustheit und sequenziellen Kontrollierbarkeit behält.

\subsection{Schlussfolgerung für die Testplanung}
Aus der Diskrepanz zwischen theoretischem Anspruch (optimale \ac{GLM}-Designs) und praktischer Anwendbarkeit (unbekannte Parameter) leitet sich die Notwendigkeit einer pragmatischen und zugleich statistisch fundierten Vorgehensweise ab.
Anstatt sich auf die Suche nach einem theoretisch „optimalen“ \ac{GLM}-Design zu versteifen, das aufgrund der Parameterabhängigkeit in der Praxis kaum robust umsetzbar ist, erscheint die Adaption bewährter Standard-Designs (\ac{CCD}) vielversprechender.
Dabei müssen die etablierten Optimalitätskriterien jedoch nicht als alleiniges Entwurfsziel, sondern als Benchmark-Metriken herangezogen werden:
\begin{itemize}
    \item \textbf{Modellierungsgüte (A-, D-Kriterien):} Diese Metriken sichern ab, dass die Parameterschätzung (Bestimmung von $\sym{theta}$) auch im modifizierten Design statistisch solide und präzise bleibt.
    \item \textbf{Prognosefähigkeit (G-, I-, V-Kriterien):} Diese Metriken gewinnen im Kontext der Lebensdauererprobung an primärer Bedeutung, da das Ziel die präzise Vorhersage der Zuverlässigkeit $\sym{R}(\sym{t})$ unter Einsatzbedingungen (Extrapolation) ist.
\end{itemize}
Eine effiziente Testplanung für die multivariate Lebensdauererprobung muss folglich einen Kompromiss finden: Sie muss flexibel genug sein, um Extrapolationen durch \ac{ALT} gezielt zu unterstützen (z.B. durch Anpassung des \ac{CCD}), und gleichzeitig robust genug, um Unsicherheiten der nicht-linearen \ac{GLM}-Parameterschätzung abzufangen.

\section{Forschungsfragen und Aufbau der Arbeit} \label{sec:forschungsfragen}
Das übergeordnete Ziel dieser Arbeit, die Entwicklung effizienter multivariater Lebensdauerversuche, fußt auf der Prämisse, dass etablierte Standard-Designs nicht zwingend durch vollkommen neuartige Algorithmen ersetzt, sondern vielmehr intelligent adaptiert werden müssen.
Als methodischer Anker und Referenzpunkt dient hierbei das \textbf{\ac{CCD}}.
Dieser Konsens begründet sich auf einer Synthese aus verfahrenstechnischen, physikalischen und statistischen Vorzügen, die das \ac{CCD} für die Zuverlässigkeitsmodellierung im Maschinen- und Fahrzeugbau prädestinieren.
Dies lässt sich aus praktischen Erfahrungen wie beispielsweise nach \textcite{Myers.2016,Kleppmann.2020} sowie einer Vielzahl wissenschaftlicher Publikationen ableiten und umfasst insbesondere folgende Aspekte:
\begin{enumerate}
    \item \textbf{Sequenzielle Augmentierbarkeit:} Das Design unterstützt ideal den ökonomischen Zwang zur Ressourceneffizienz. Es ermöglicht eine schrittweise Erweiterung vom Screening über lineare Modelle bis hin zur quadratischen Modellierung. Dies erlaubt den Abbruch oder die Ausweitung von Versuchen basierend auf Zwischenergebnissen und minimiert das Risiko von Fehlinvestitionen bei zeitintensiven Lebensdauertests \cite{Bertsche.2022}.
    \item \textbf{Universelle Prädiktionseigenschaften:} Da der exakte Betriebsbereich oder das Optimum der Zuverlässigkeit a priori oft unbekannt sind und Tests beschleunigt (\ac{ALT}, Offset zwischen Test- und Feldraum) stattfinden, bietet die Rotierbarkeit des Designs günstige Voraussetzungen für eine richtungsunabhängige Prädiktionsgüte \cite{Meeker.2022,Montgomery.2020}.
    \item \textbf{Wissenschaftliche Fundierung:} Das \ac{CCD} ist als breit diskutierter und gut erforschter Standard etabliert, was die Akzeptanz und Vergleichbarkeit der Ergebnisse im wissenschaftlichen und industriellen Umfeld sicherstellt, vgl. \textcite{Kleppmann.2020,Montgomery.2020,Rigdon.2022}.
    \item \textbf{Validierung physikalischer Modelle:} Durch die Variation auf fünf Faktorstufen ($-\symsub{alpha}{idx_D}, -1, 0, +1, +\symsub{alpha}{idx_D}$) ermöglicht das Design - im Gegensatz zu einfacheren Plänen - die Überprüfung der Gültigkeit physikalischer Beschleunigungsgesetze (z.B. Arrhenius) und die Detektion von Linearitätsabweichungen bei hohen Lasten \cite{Yang.2007,Modarres.2017,Meeker.2022}.
    \item \textbf{Schätzung des reinen Fehlers:} Die Integration mehrfacher Zentralpunkte erlaubt eine robuste Schätzung der natürlichen Streuung (\textit{Pure Error}) unabhängig vom Anpassungsfehler des Modells (\textit{Lack of Fit}). Dies ist für die Berechnung verlässlicher Vertrauensbereiche der Zuverlässigkeit essenziell \cite{Meeker.2022}.
    \item \textbf{Orthogonale Blockbildung:} Das Design lässt sich exzellent in orthogonale Blöcke unterteilen \cite{Montgomery.2020,Rigdon.2022}. Dies ist für Langzeitversuche von entscheidender Bedeutung, um zeitliche Trends durch beispielsweise Chargenunterschiede statistisch zu bereinigen, ohne die Parameterschätzung zu verzerren.
    \item \textbf{Robustheit gegen Datenverlust:} Aufgrund seiner symmetrischen Struktur weist das \ac{CCD} eine hohe Toleranz gegenüber fehlenden Werten (\textit{Missing Data}) auf, was die Auswertbarkeit des Versuchs auch bei unvorhergesehenen Ausfällen oder technischer Zensierung auf dem Prüfstand weitgehend sichert \cite{Montgomery.2020,Myers.2016}.
\end{enumerate}
Ausgehend von diesem Status quo zielt die vorliegende Arbeit darauf ab, dieses bewährte Grundkonstrukt nicht zu verwerfen, sondern es gezielt für die spezifischen Anforderungen der Lebensdauerprognose zu modifizieren. Dies führt zur zentralen \textbf{Forschungsfrage} dieser Arbeit:\\

\textbf{Wie können effiziente, multivariate Testdesigns (\acp{RSD}) entwickelt und geplant werden, um eine beschleunigte Lebensdauererprobung zum Zweck einer multivariaten Zuverlässigkeitsmodellierung umzusetzen?}\\

Zur Beantwortung dieser Hauptfragestellung werden folgende Teilfragestellungen abgeleitet:
\begin{enumerate}
    \item \textbf{Adaptionsbedarf und -strategie:} Wie können konventionelle Planungsstrategien und Testdesigns - allen voran \acp{CCD} - adäquat zum Zweck der beschleunigten Lebensdauererprobung (\ac{ALT}) angepasst werden?
    \item \textbf{Bewertungsmethodik:} Wie können die Auswirkungen derartiger Anpassungen (Manipulationen) an \acp{CCD} auf die Prognosegüte quantitativ charakterisiert und vergleichbar bewertet werden?
\end{enumerate}

\subsection{Aufbau der Arbeit}
Der weitere Aufbau der Arbeit orientiert sich an der logischen Abfolge von der Parameteridentifikation bis zur Design-Optimierung.
In \colorbox{yellow}{\textbf{Kapitel 4}} wird zunächst das Themenfeld des Screenings adressiert.
Hier wird ein Vorschlag zu einem heuristischen Screening-Ansatz vorgestellt, der darauf abzielt, die Effizienz im Prozess der Parameterselektion gezielt für Lebensdauer-beeinflussende Faktoren zu steigern und somit die Basis für nachfolgende \acp{RSD} zu legen.
Grundsätzlich neu ist hierbei die Behandlung mutmaßlicher Wechselwirkungen der Lebensdauer-beeinflussenden Faktoren sowie die Berücksichtigung von Änderungen in der signifikanten Aktivmenge der Einflussparameter durch den Screening-Prozess.
Darauf aufbauend widmet sich \colorbox{yellow}{\textbf{Kapitel 5}} der generischen Untersuchung und Bewertung von Versuchsplanabweichungen, ausgehend vom \ac{CCD} als Referenz.
In diesem Kontext werden ein Kostenmodell sowie ein neu entwickeltes Tool zur Bewertung der Prädiktionsgüte in Abhängigkeit der \ac{SPV} eingeführt.
Daraus abgeleitet werden konkrete Strategieempfehlungen für die Anpassung von \acp{CCD}, etwa zur Optimierung der Extrapolationsfähigkeit, vorgestellt und diskutiert.
Abschließend ist in \colorbox{yellow}{\textbf{Kapitel 6}} die Vorstellung einer Fallstudie (Case Study) vorgesehen, um die erarbeitete Methodik an einem praxisnahen Anwendungsbeispiel zu validieren und die theoretischen Erkenntnisse in den operativen Kontext zu transferieren.
% \subsubsection{Abgrenzung zu weiteren Themen in der \ac{RSM}} \label{subsubsec:abgrenzung}
% \begin{itemize}
%     \item \textbf{\ac{GLM} Mixed Models}, \textbf{Random Effects} (Generalized Linear Mixed Models) \cite[Myers 2010 Kap.7.1.2]{Myers.2010}
%     \item \textbf{Robuste Versuchspläne} (Robust Designs):
%     \item \textbf{Adaptive Versuchspläne} (Adaptive Designs):
%     \item \textbf{Bayesian Designs}, z.B. in \textcite{Englert.2012}
%     \item \textbf{Sequential Designs}:
%     \item \textbf{Optimalität für Vorhersagegüte}:
%     \item \textbf{Optimalität für Klassifikationsaufgaben}:
%     \item \textbf{Optimalität für multiple Zielgrößen}:
%     \item \textbf{Robuste Optimale Versuchspläne} (Robust Optimal Designs):
%     \item \textbf{Kombinierte Optimalitätskriterien} (Compound Optimal Designs):
%     \item \textbf{Optimale Versuchspläne für Mischmodelle} (Optimal Designs for Mixed Models):
%     \item \textbf{Optimale Versuchspläne für räumliche Modelle} (Optimal Designs for Spatial Models):
%     \item \textbf{Optimale Versuchspläne für Zeitreihenmodelle} (Optimal Designs for Time Series Models):
%     \item \textbf{Optimale Versuchspläne für nichtparametrische Modelle} (Optimal Designs for Nonparametric Models):
%     \item \textbf{Optimale Versuchspläne für Hochdimensionale Modelle} (Optimal Designs for High-Dimensional Models):
%     \item \textbf{Optimale Versuchspläne für dynamische Systeme} (Optimal Designs for Dynamic Systems):
%     \item \textbf{Optimale Versuchspläne für Netzwerke} (Optimal Designs for Networks):
%     \item \textbf{Optimale Versuchspläne für Big Data Anwendungen} (Optimal Designs for Big Data Applications):
%     \item \textbf{Optimale Versuchspläne für maschinelles Lernen} (Optimal Designs for Machine Learning): \textcite{Colak.2024}
%     \item \textbf{Optimale Versuchspläne für Künstliche Intelligenz} (Optimal Designs for Artificial Intelligence):
%     \item \textbf{Optimale Versuchspläne für Quantencomputing} (Optimal Designs for Quantum Computing):
%     \item \textbf{Optimale Versuchspläne für Blockchain-Technologien} (Optimal Designs for Blockchain Technologies):
%     \item \textbf{Optimale Versuchspläne für Internet der Dinge (IoT)} (Optimal Designs for Internet of Things (IoT)):
%     \item \textbf{Optimale Versuchspläne für Cyber-Physische Systeme} (Optimal Designs for Cyber-Physical Systems):
%     \item \textbf{Optimale Versuchspläne für Nachhaltigkeit und Umweltwissenschaften} (Optimal Designs for Sustainability and Environmental Sciences):
% \end{itemize}
% \begin{itemize}
%     \item \cite[Kap. 9.1.4]{Myers.2016}
%     \item \cite[Kap. 9.2.1]{Myers.2016}
%     \item \colorbox{yellow}{\cite[Chapter. 27]{Cui.2021}}
%     \item Monte-Carlo-Ansatz: \textcite{Escobar.1995}
%     \item \textcite[Kap.13]{Box.2007}
%     \item Evaluierung eines effizienten Versuchsplans vor Versuchsdurchführung: \textcite{Johnson.2011}
%     \item Manipulationen im CCD: \textcite{Ahn.2015,Donev.2004,G.E.P.Box.1951,Ardakani.2011}
%     \item Optimale versuchspäne  \textcite{Jones.2012}
%     \item Khuri - 2006 - Design Issues for Generalized Linear Models A Review
%     \item Stufken in Hinkelmann - 2012 - Optimal Designs for Generalized Linear Models with Applications
%     \item Yang - 2011 - OPTIMAL DESIGNS FOR GENERALIZED LINEAR MODELS WITH MULTIPLE DESIGN VARIABLES
%     \item \cite[Seite 314]{Rigdon.2022}
%     \item Monroe, E.M., Pan, R., Anderson-Cook, C. et al. (2011). A generalized linear model approach to designing accelerated life test experiments. Quality and Reliability Engineering International 27: 595-607.
% \end{itemize}