%%%%%%%%%%%%%%%%%%%%%%%%%%%%%%%%%%%%%%%%%%%%%%%%%%%%%%%%%%%%%%%%%%%%%%%%%%%%%%%%%%%%%%%%%%%%%%%%%%%%%%%%%%%%%%%%%
%% Kapitel 2 - Stand der Forschung und Technik %%%%%%%%%%%%%%%%%%%%%%%%%%%%%%%%%%%%%%%%%%%%%%%%%%%%%%%%%%%%%%%%%%
%%%%%%%%%%%%%%%%%%%%%%%%%%%%%%%%%%%%%%%%%%%%%%%%%%%%%%%%%%%%%%%%%%%%%%%%%%%%%%%%%%%%%%%%%%%%%%%%%%%%%%%%%%%%%%%%%

\chapter{Stand der Forschung und Technik} \label{chap.Std}

Dieses Kapitel stellt die für diese Arbeit erforderlichen technischen und methodischen Grundlagen bereit. Zunächst werden zentrale Begriffe und Konzepte der Zuverlässigkeitstechnik (Abschnitt~\ref{subsec.BegriffeZUV}) sowie grundlegende statistische Verfahren erläutert. Darauf aufbauend folgen die Einführung und Einordnung des \acs{DoE} sowie der multivariaten Lebensdauermodellbildung, die beide für die Entwicklung effizienter Lebensdauertests maßgeblich sind. \cite{colu92}

\section{Zuverlässigkeitstechnik und Wahrscheinlichkeitstheorie} \label{sec.Zuv}
\subsection{Begriffe und Definitionen} \label{subsec.BegriffeZUV}
\subsection{Deskriptive Statistik für Lebensdauerdaten} \label{subsec.Stat}
\subsection{Parameterschätzverfahren} \label{subsec.Schätzer}
\section{Statistische Versuchsplanung und Modellbildung} \label{sec.DoE}
\subsection{Grundbegriffe der statistischen Versuchsplanung} \label{subsec.BegriffeDoE}
\subsection{Statistische Lebensdauer-Versuchspläne} \label{subsec.Pläne}
\subsection{Statistische Modellbildung} \label{subsec.Model}
