%%%%%%%%%%%%%%%%%%%%%%%%%%%%%%%%%%%%%%%%%%%%%%%%%%%%%%%%%%%%%%%%%%%%%%%%%%%%%%%%%%%%%%%%%%%%%%%%%%%%%%%%%%%%%%%%%
%% Kapitel 2 - Stand der Forschung und Technik %%%%%%%%%%%%%%%%%%%%%%%%%%%%%%%%%%%%%%%%%%%%%%%%%%%%%%%%%%%%%%%%%%
%%%%%%%%%%%%%%%%%%%%%%%%%%%%%%%%%%%%%%%%%%%%%%%%%%%%%%%%%%%%%%%%%%%%%%%%%%%%%%%%%%%%%%%%%%%%%%%%%%%%%%%%%%%%%%%%%

\chapter{Stand der Forschung und Technik} \label{chap:stand}

Dieses Kapitel stellt die für diese Arbeit erforderlichen technischen und methodischen Grundlagen bereit. Zunächst werden zentrale Begriffe und Konzepte der Zuverlässigkeitstechnik (Abschnitt~\ref{subsec:begriffezUV}) sowie grundlegende statistische Verfahren zur Lebensdauer-Datenanalyse in Kombination mit Versuchsplänen erläutert.
Darauf aufbauend folgen die Einführung und die Einordnung von \acs{DoE} sowie multivariater Lebensdauermodellierung aus dem Stand der Technik und Forschung, die beide für die Entwicklung effizienter Lebensdauerversuchspläne maßgeblich sind.
Dies umfasst insbesondere typische Versuchspläne im Kontext der Lebensdauererprobung sowie Metriken und Indikatoren zur Bewertung der Pläne.

\section{Zuverlässigkeitstechnik und Wahrscheinlichkeitstheorie} \label{sec:zuv}

\subsection{Begriffe und Definitionen} \label{subsec:begriffezUV}

\subsection{Deskriptive Statistik für Lebensdauerdaten} \label{subsec.Stat}

\subsection{Parameterschätzverfahren} \label{subsec.Schätzer}

\section{Statistische Versuchsplanung und Modellbildung} \label{sec.DoE}

\subsection{Grundbegriffe der statistischen Versuchsplanung} \label{subsec.BegriffeDoE}

\subsection{Statistische Lebensdauer-Versuchspläne} \label{subsec.Pläne}

\subsection{Statistische Modellbildung} \label{subsec.Model}
