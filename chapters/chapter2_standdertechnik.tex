%%%%%%%%%%%%%%%%%%%%%%%%%%%%%%%%%%%%%%%%%%%%%%%%%%%%%%%%%%%%%%%%%%%%%%%%%%%%%%%%%%%%%%%%%%%%%%%%%%%%%%%%%%%%%%%%%
%% Kapitel 2 - Stand der Forschung und Technik %%%%%%%%%%%%%%%%%%%%%%%%%%%%%%%%%%%%%%%%%%%%%%%%%%%%%%%%%%%%%%%%%%
%%%%%%%%%%%%%%%%%%%%%%%%%%%%%%%%%%%%%%%%%%%%%%%%%%%%%%%%%%%%%%%%%%%%%%%%%%%%%%%%%%%%%%%%%%%%%%%%%%%%%%%%%%%%%%%%%

\chapter{Stand der Forschung und Technik} \label{chap:stand}

Dieses Kapitel stellt die für diese Arbeit erforderlichen technischen und methodischen Grundlagen bereit. Zunächst werde in Abschnitt~\ref{sec:zuv} zentrale Begriffe und Konzepte der Zuverlässigkeitstechnik sowie das grundlegende statistische Verfahren zur Lebensdauer-Datenanalyse in Kombination mit Versuchsplänen erläutert.
Darauf aufbauend folgen in Abschnitt~\ref{sec:doe} die Einführung und die Einordnung von \acs{DoE} sowie der multivariaten Lebensdauermodellierung aus dem Stand der Technik und der Wissenschaft, die beide für die Entwicklung effizienter Lebensdauerversuchspläne maßgeblich sind.
Im Kontext der Lebensdauererprobung umfasst dies insbesondere typische, statistische Versuchspläne sowie Metriken und Indikatoren zur allgemeinen Bewertung der Versuchspläne.

\section{Zuverlässigkeitstechnik und Wahrscheinlichkeitstheorie} \label{sec:zuv}
Die Zuverlässigkeitstechnik befasst sich mit der probabilistischen Beschreibung der Lebensdauer technischer Produkte und Systeme.
Ziel ist die statistische Modellierung des Ausfallverhaltens unter Berücksichtigung der Funktionalität des Produkts unter relevanten Randbedingungen.
Eine zentrale Aufgabe besteht somit in der statistischen Charakterisierung des Ausfallbegriffs mithilfe deskriptiver Statistik sowie in der Parametrisierung geeigneter Verteilungen zur Abbildung des Lebensdauerverhaltens.
Die Modellierung kann - abhängig von den Randbedingungen - auf Basis \textit{einer einzelnen} Belastungsgröße oder \textit{mehrerer} Beanspruchungsparameter erfolgen, die gemeinsam den Produktausfall determinieren.
Ein grundlegendes Verständnis des Umgangs mit zufallsverteilten Lebensdauerereignissen ist daher eine elementare Voraussetzung für die statistische Versuchsplanung im Rahmen der Zuverlässigkeitstechnik.
Weiterführende Konzepte und vertiefte methodische Ansätze zur Zuverlässigkeitstechnik sowie zur statistischen Testplanung sind allen voran in der Standardliteratur von Bertsche und Dazer \cite{Bertsche.2022} dargelegt, an deren Vorgehensweise sich die nachfolgenden Ausführungen orientieren.

\subsection{Begriffe und Definitionen} \label{subsec:begriffezuv}
Der \textbf{Ausfall} eines technischen Produkts bezeichnet den Zeitpunkt innerhalb seiner Lebensdauer, zu dem die geforderte Funktionalität unter definierten Umgebungs- und Randbedingungen nicht mehr erfüllt ist.
Die \textbf{Ausfallzeit}, welche diese Zustandsänderung zeitlich definiert, wird im Allgemeinen als kontinuierliche Zufallsvariable $\sym{tau}>0$ aufgefasst.
So ergibt sich die Wahrscheinlichkeit, dass ein Produkt im Zeitraum bis $\sym{t}$ einen Funktionsverlust erleidet, zu
\begin{equation} \label{eq:probdef}
    \sym{F}(\sym{t})=\sym{Pr}(\sym{tau}\leq \sym{t}).
\end{equation}
Diese Funktion beschreibt die \textbf{Ausfallwahrscheinlichkeit} $\sym{F}(\sym{t})$, während die \textbf{Zuverlässigkeit}
\begin{equation} \label{eq:reldef}
    \sym{R}(\sym{t})=\sym{Pr}(\sym{tau} > \sym{t}) = 1 - \sym{F}(\sym{t}) = \int_{-\infty}^{\sym{t}} \sym{f}(\sym{t}) \,d\sym{t} , \quad \sym{t} \geq 0.
\end{equation}
komplementär diejenige Wahrscheinlichkeit $\sym{R}(\sym{t}): \sym{RR}_{\geq 0} \rightarrow [0,1] \subset \sym{RR}$ quantifiziert, zu der das nicht reparierbare Produkt die realisierte Zeit $\sym{t}$ überlebt: also frei von Funktionsverlust bleibt und funktionsfähig ist \cite{Bertsche.2022,Birolini.2017,Meeker.2022,Yang.2007}.
Damit ist die Zuverlässigkeit mathematisch als reellwertige, monoton fallende und stetige Funktion definiert.
Die \textbf{Wahrscheinlichkeitsdichtefunktion} $\sym{f}(\sym{t})$ der Ausfallzeit beschreibt, wie sich die Wahrscheinlichkeiten der Ausfälle über der Zeit verteilen.
Sie folgt der Ableitung der Verteilungsfunktion $F(\sym{t})$ und kann als Maß für die Ausfallintensität pro Zeiteinheit interpretiert werden:
\begin{equation} \label{eq:pdfdef}
    \sym{f}(\sym{t}) = \frac{d}{d\sym{t}}\sym{F}(\sym{t}) = \frac{d}{d\sym{t}}\sym{Pr}(\sym{tau} \leq \sym{t}), \quad \sym{t} \geq 0.
\end{equation}
Damit beschreibt $\sym{f}(\sym{t})$ die lokale Änderungsrate der Zuverlässigkeit - also, wie schnell die Wahrscheinlichkeit der Funktionserfüllung respektive der Lebensdauer ab- oder zunimmt.


Mit Schätzung der entsprechenden Wahrscheinlichkeitsverteilung wird die Bestimmung wichtiger Zuverlässigkeitskennzahlen wie der Ausfallwahrscheinlichkeit oder einem Lebensdauerquantil ermöglicht.




Kumulative Verteilungsfunktion

Die kumulative Verteilungsfunktion

$ F(t) = \Pr(T \leq t)$
$\sym{i}$

liefert die Wahrscheinlichkeit, dass ein System bis zum Zeitpunkt $t$ ausfällt. Sie kann auch als Anteil der ausgefallenen Einheiten in einer Population interpretiert werden, sofern der zugrunde liegende Prozess stationär ist, d.h. $F(t)$ ist zeitinvariant.



Wahrscheinlichkeitsdichtefunktion

Die Wahrscheinlichkeitsdichtefunktion ergibt sich als Ableitung der cdf:

$f(t) = \frac{dF(t)}{dt}$

Sie beschreibt die Form der Ausfallverteilung über die Zeit und dient als kontinuierliches Analogon eines Histogramms der Ausfallhäufigkeiten. Die Fläche unter $f(t)$ bis $t$ entspricht der Aus


\subsection{Deskriptive Statistik für Lebensdauerdaten} \label{subsec:stat}

\subsection{Parameterschätzverfahren} \label{subsec:schätzer}

\section{Statistische Versuchsplanung und Modellbildung} \label{sec:doe}

\subsection{Grundbegriffe der statistischen Versuchsplanung} \label{subsec:begriffedoe}

\subsection{Statistische Lebensdauer-Versuchspläne} \label{subsec:pläne}

\subsection{Statistische Modellbildung} \label{subsec:model}
