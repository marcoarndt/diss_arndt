%%%%%%%%%%%%%%%%%%%%%%%%%%%%%%%%%%%%%%%%%%%%%%%%%%%%%%%%%%%%%%%%%%%%%%%%%%%%%%%%%%%%%%%%%%%%%%%%%%%%%%%%%%%%%%%%%
%% Kapitel 2 - Stand der Forschung und Technik %%%%%%%%%%%%%%%%%%%%%%%%%%%%%%%%%%%%%%%%%%%%%%%%%%%%%%%%%%%%%%%%%%
%%%%%%%%%%%%%%%%%%%%%%%%%%%%%%%%%%%%%%%%%%%%%%%%%%%%%%%%%%%%%%%%%%%%%%%%%%%%%%%%%%%%%%%%%%%%%%%%%%%%%%%%%%%%%%%%%
\chapter{Stand der Forschung und Technik} \label{chap:stand}

Dieses Kapitel stellt die für diese Arbeit erforderlichen technischen und methodischen Grundlagen bereit. Zunächst werde in Abschnitt~\ref{sec:zuv} zentrale Begriffe und Konzepte der Zuverlässigkeitstechnik sowie das grundlegende statistische Verfahren zur Lebensdauer-Datenanalyse in Kombination mit Versuchsplänen erläutert.
Darauf aufbauend folgen in Abschnitt~\ref{sec:doe} die Einführung und die Einordnung von \acs{DoE} für Lebensdaueruntersuchungen sowie der multivariaten Lebensdauermodellierung aus dem Stand der Technik und der Wissenschaft, die beide für die Entwicklung effizienter Lebensdauerversuchspläne maßgeblich sind.
Im Kontext der Lebensdauererprobung umfasst dies insbesondere typische, statistische Versuchspläne sowie Metriken und Indikatoren zur allgemeinen Bewertung der Versuchspläne.

\section{Zuverlässigkeitstechnik und Wahrscheinlichkeitstheorie} \label{sec:zuv}
Die Zuverlässigkeitstechnik befasst sich mit der probabilistischen Beschreibung der Lebensdauer technischer Produkte und Systeme sowie der strategischen und statistischen Planung von Lebensdauertests.
Ziel ist die statistische Modellierung des Ausfallverhaltens unter Berücksichtigung der Funktionalität des Produkts bei relevanten Randbedingungen.
Eine zentrale Aufgabe besteht somit in der statistischen Charakterisierung des Ausfallbegriffs mithilfe deskriptiver Statistik sowie in der Parametrisierung geeigneter Verteilungen zur Abbildung des Lebensdauerverhaltens.
Die Modellierung kann - abhängig von den Randbedingungen - auf Basis \textit{einer einzelnen} Belastungsgröße oder \textit{mehrerer} Beanspruchungsparameter erfolgen, die gemeinsam den Produktausfall determinieren.
Ein grundlegendes Verständnis des Umgangs mit zufallsverteilten Lebensdauerereignissen ist daher eine elementare Voraussetzung für die statistische Versuchsplanung im Rahmen der Zuverlässigkeitstechnik.
Weiterführende Konzepte und vertiefte methodische Ansätze zur Zuverlässigkeitstechnik sowie zur statistischen Testplanung sind allen voran in der Standardliteratur von \textcite{Bertsche.2022} dargelegt, an deren Vorgehensweise sich die nachfolgenden Ausführungen orientieren.

%%%%%%%%%%%%%%%%%%%%%%%%%%%%%%%%%%%%%%%%%%%%%%%%%%%%%%%%%%%%%%%%%%%%%%%%%%%%%%%%%%%%%%%%%%%%%%%%%%%%%%%%%%%%%%%%%%%%%%%%%%
%%%%%%%%%%%%%%%%%%%%%%%%%%%%%%%%%%%%%%%%%%%%%%%%%%%%%%%%%%%%%%%%%%%%%%%%%%%%%%%%%%%%%%%%%%%%%%%%%%%%%%%%%%%%%%%%%%%%%%%%%%
%%%%%%%%%%%%%%%%%%%%%%%%%%%%%%%%%%%%%%%%%%%%%%%%%%%%%%%%%%%%%%%%%%%%%%%%%%%%%%%%%%%%%%%%%%%%%%%%%%%%%%%%%%%%%%%%%%%%%%%%%%
%%%%%%%%%%%%%%%%%%%%%%%%%%%%%%%%%%%%%%%%%%%%%%%%%%%%%%%%%%%%%%%%%%%%%%%%%%%%%%%%%%%%%%%%%%%%%%%%%%%%%%%%%%%%%%%%%%%%%%%%%%
\subsection{Begriffe und Definitionen} \label{subsec:begriffezuv}
Der \textbf{Ausfall} eines technischen Produkts bezeichnet den Zeitpunkt innerhalb seiner Lebensdauer, zu dem die geforderte Funktionalität unter definierten Umgebungs- und Randbedingungen nicht mehr erfüllt ist - also das Lebensdauerende - engl. \textbf{\ac{EoL}}.
Als \textbf{Belastung} werden die von außen auf ein Produkt einwirkenden Einflussparameter - Kräfte und Momente im mechanischen Kontext - bezeichnet.
\textit{Einzelne} oder zeitgleich \textit{mehrere} Einflussparameter induzieren infolge der Produktgestalt daraus \textbf{Beanspruchungen}: innere Kräfte, Momente und lokale Spannungen.
Belastung und Beanspruchung sind die maßgeblichen Faktoren, welche die Lebensdauer determinieren.
Die \textbf{Ausfallzeit}, welche diese Zustandsänderung zeitlich definiert, wird im Allgemeinen als kontinuierliche Zufallsvariable $\sym{tau}>0$ aufgefasst.
So ergibt sich die Wahrscheinlichkeit, dass ein Produkt im Zeitraum bis $\sym{t}$ einen Funktionsverlust erleidet, zu
\begin{equation} \label{eq:probdef}
    \sym{F}(\sym{t})=\sym{Pr}(\sym{tau}\leq \sym{t}) = \int_{0}^{\sym{t}} \sym{f}(\sym{t}) \,d\sym{t}.
\end{equation}
Diese Funktion beschreibt die \textbf{Ausfallwahrscheinlichkeit}.
Sie definiert damit die Verteilungsfunktion - engl. \textbf{\ac{cdf}} - für stochastische \ac{EoL}-Events, während die \textbf{Zuverlässigkeit}
\begin{equation} \label{eq:reldef}
    \sym{R}(\sym{t})=\sym{Pr}(\sym{tau} > \sym{t}) = 1 - \sym{F}(\sym{t}) = \int_{\sym{t}}^{\infty} \sym{f}(\sym{t}) \,d\sym{t} , \quad \sym{t} \geq 0.
\end{equation}
komplementär diejenige Wahrscheinlichkeit $\sym{R}(\sym{t}): \sym{RR}_{\geq 0} \rightarrow [0,1] \subset \sym{RR}$ quantifiziert, zu der das nicht reparierbare Produkt die realisierte Zeit $\sym{t}$ überlebt: also frei von Funktionsverlust bleibt und funktionsfähig ist \cite{Bertsche.2022,Birolini.2017,Meeker.2022}.
Damit ist die Zuverlässigkeit mathematisch als reellwertige, monoton fallende und stetige Funktion definiert.
Gleichwohl ist $\sym{R}(\sym{t})$ keine universelle Eigenschaft, sondern vielmehr eine Funktion der Betriebsbedingungen.
Diese Bedingungen umfassen unter anderem eine oder mehrere Belastungsarten und deren Niveaus, Nutzungsverhalten sowie
spezifische Betriebsprofile. Mechanische, elektrische und thermische Belastungen treten dabei am häufigsten auf \cite{Yang.2007}.
Die \textbf{Wahrscheinlichkeitsdichtefunktion} - engl. \textbf{\ac{pdf}} - $\sym{f}(\sym{t})$ der Ausfallzeit beschreibt, wie sich die Wahrscheinlichkeiten der Ausfälle über der Zeit verteilen.
Sie folgt somit der Ableitung der \ac{cdf}:
\begin{equation} \label{eq:pdfdef}
    \sym{f}(\sym{t}) = \frac{d}{d\sym{t}}\sym{F}(\sym{t}) = \frac{d}{d\sym{t}}\sym{Pr}(\sym{tau} \leq \sym{t}), \quad \sym{t} \geq 0.
\end{equation}
Damit repräsentiert $\sym{f}(\sym{t})$ die Ausfallintensität pro Zeiteinheit und ist proportional zur lokalen Änderungsrate der Ausfallwahrscheinlichkeit.
Als vierte fundamentale Größe der Zuverlässigkeitsanalyse wird außerdem die \textbf{Ausfallrate} (auch Hazard-Funktion) $\sym{lambda}(\sym{t})$ eingeführt.
Sie quantifiziert das momentane Ausfallrisiko eines Produkts zum Zeitpunkt $\sym{t}$, bedingt dadurch, dass es bis zu diesem Zeitpunkt überlebt hat ($\sym{R}(\sym{t}) > 0$).
Mathematisch ist sie als das Verhältnis der \ac{pdf} zur Zuverlässigkeitsfunktion $\sym{R}(\sym{t})$ definiert:
\begin{equation} \label{eq:hazarddef}
    \sym{lambda}(\sym{t}) = \lim_{\Delta \sym{t} \to 0} \frac{\sym{Pr}(\sym{t} < \sym{tau} \leq \sym{t} + \Delta \sym{t} | \sym{tau} > \sym{t})}{\Delta \sym{t}} = \frac{1}{\sym{R}(\sym{t})} \left[ - \frac{d\sym{R}(\sym{t})}{d\sym{t}} \right] = \frac{\sym{f}(\sym{t})}{\sym{R}(\sym{t})} .
\end{equation}
Die Ausfallrate $\sym{lambda}(\sym{t})$ ist von zentraler Bedeutung, da ihr zeitlicher Verlauf (z.B. konstant, steigend, fallend) direkte Rückschlüsse auf zugrundeliegende Ausfallmechanismen wie Frühausfälle, Zufallsausfälle oder Verschleiß (vgl. "Badewannenkurve") zulässt \cite{Bertsche.2022,Yang.2007,Rigdon.2022}.

%%%%%%%%%%%%%%%%%%%%%%%%%%%%%%%%%%%%%%%%%%%%%%%%%%%%%%%%%%%%%%%%%%%%%%%%%%%%%%%%%%%%%%%%%%%%%%%%%%%%%%%%%%%%%%%%%%%%%%%%%%
%%%%%%%%%%%%%%%%%%%%%%%%%%%%%%%%%%%%%%%%%%%%%%%%%%%%%%%%%%%%%%%%%%%%%%%%%%%%%%%%%%%%%%%%%%%%%%%%%%%%%%%%%%%%%%%%%%%%%%%%%%
%%%%%%%%%%%%%%%%%%%%%%%%%%%%%%%%%%%%%%%%%%%%%%%%%%%%%%%%%%%%%%%%%%%%%%%%%%%%%%%%%%%%%%%%%%%%%%%%%%%%%%%%%%%%%%%%%%%%%%%%%%
%%%%%%%%%%%%%%%%%%%%%%%%%%%%%%%%%%%%%%%%%%%%%%%%%%%%%%%%%%%%%%%%%%%%%%%%%%%%%%%%%%%%%%%%%%%%%%%%%%%%%%%%%%%%%%%%%%%%%%%%%%
\subsection{Deskriptive Statistik für Lebensdauerdaten} \label{subsec:stat}
Die im vorherigen Abschnitt definierten Funktionen $\sym{F}(\sym{t})$, $\sym{R}(\sym{t})$, $\sym{f}(\sym{t})$ und $\sym{lambda}(\sym{t})$ beschreiben das stochastische Ausfallverhalten eines Produktes auf einer theoretischen Populationsebene.
Für die praktische Anwendung im Engineering müssen diese Funktionen, respektive die Parameter der ihnen zugrundeliegenden Verteilungsmodelle, auf Basis von empirisch ermittelten Lebensdauerdaten jedoch approximiert werden.

Die deskriptive Statistik stellt die notwendigen Methoden zur initialen Charakterisierung, Quantifizierung und Aufbereitung dieser Stichprobendaten bereit.
Zur Beschreibung der Lebensdauerverteilungen sind \textbf{Lageparameter} und \textbf{Streuungsmaße} notwendig, die zunächst theoretisch (für die Grundgesamtheit) definiert und anschließend aus der Stichprobe berechnet werden.

Der primäre Lageparameter ist der \textbf{Erwartungswert} $\sym{mu}$ der Zufallsvariable $\sym{tau}$.
Er repräsentiert den Schwerpunkt von $\sym{f}(\sym{t})$ und wird für kontinuierliche Lebensdauerdaten berechnet als:
\begin{equation} \label{eq:theo_mean}
    \sym{mu} = \sym{E}[\sym{tau}] = \int_{0}^{\infty} \sym{t} \cdot \sym{f}(\sym{t}) d\sym{t}.
\end{equation}
Ein weiterer Lageparameter ist das \textbf{Quantil} $\symsub{t}{q}$ der Lebensdauer.
Es definiert den Zeitpunkt, zu dem $\sym{F}(\sym{t})$ den Anteil $\sym{q}$ (respektive das \textbf{Perzentil} in Prozentpunkten) erreicht:
\begin{equation} \label{eq:quantildef}
    \sym{F}(\symsub{t}{q}) = \sym{q}, \quad \sym{q} \in [0,1].
\end{equation}
Damit gibt das $\sym{q}$~-~Quantil denjenigen Lebensdauerwert an, unterhalb dessen der Anteil $\sym{q}$ aller betrachteten Produkte ausgefallen ist.
Ein spezieller Fall ist der \textbf{Median} $\sym{t}_{0.5}$, bei dem die Ausfallwahrscheinlichkeit 50\% beträgt:
\begin{equation} \label{eq:mediandef}
    \sym{F}(\sym{t}_{0.5}) = 0.5.
\end{equation}
Der Median beschreibt somit den Zeitpunkt, zu dem die Hälfte aller Produkte ausgefallen ist.
Damit teilt er die Fläche 1 unter der \ac{pdf} in zwei gleich große Teilflächen \cite{Yang.2007,Fahrmeir.2016}.

Das primäre Streuungsmaß ist die \textbf{theoretische Varianz} $\sym{sigma_sq}$, welche die mittlere quadratische Abweichung vom Erwartungswert beschreibt:
\begin{equation} \label{eq:theo_variance}
    \sym{sigma_sq} = \sym{Var}[\sym{tau}] = \sym{E}[(\sym{tau} - \sym{mu})^2] = \int_{0}^{\infty} (\sym{t} - \sym{mu})^2 \cdot \sym{f}(\sym{t}) d\sym{t}.
\end{equation}
Da $\sym{mu}$ und $\sym{sigma_sq}$ als theoretische Parameter üblicherweise unbekannt sind, werden auch sie durch empirische Statistiken approximiert, die aus einer Stichprobe vom Umfang $\sym{n}$ (bestehend aus den Messwerten $\sym{x}_{1}, \dots, \symsub{x}{n}$) berechnet werden.
Diese werden wiederum als Realisierungen der Zufallsvariable $\sym{tau}$ aufgefasst.

Das gängige empirische Äquivalent für den Erwartungswert $\sym{mu}$ ist der \textbf{arithmetische Mittelwert} $\sym{x_bar}$:
\begin{equation} \label{eq:emp_mean}
    \sym{x_bar} = \frac{1}{\sym{n}} \sum_{\sym{i}=1}^{\sym{n}} \symsub{x}{i}.
\end{equation}
Analog wird die theoretische Varianz $\sym{sigma_sq}$ durch die \textbf{empirische Varianz} $\sym{s_sq}$ (eine erwartungstreue Kenngröße) approximiert:
\begin{equation} \label{eq:emp_variance}
    \sym{s_sq} = \frac{1}{\sym{n}-1} \sum_{\sym{i}=1}^{\sym{n}} (\symsub{x}{i} - \sym{x_bar})^2.
\end{equation}
Die \textbf{empirische Standardabweichung} $\sym{s} = \sqrt{\sym{s_sq}}$ dient entsprechend als Näherung für die theoretische Standardabweichung $\sqrt{\sym{sigma_sq}}$.\

%%%%%%%%%%%%%%%%%%%%%%%%%%%%%%%%%%%%%%%%%%%%%%%%%%%%%%%%%%%%%%%%%%%%%%%%%%%%%%%%%%%%%%%%%%%%%%%%%%%%%%%%%%%%%%%%%%%%%%%%%%
%%%%%%%%%%%%%%%%%%%%%%%%%%%%%%%%%%%%%%%%%%%%%%%%%%%%%%%%%%%%%%%%%%%%%%%%%%%%%%%%%%%%%%%%%%%%%%%%%%%%%%%%%%%%%%%%%%%%%%%%%%
%%%%%%%%%%%%%%%%%%%%%%%%%%%%%%%%%%%%%%%%%%%%%%%%%%%%%%%%%%%%%%%%%%%%%%%%%%%%%%%%%%%%%%%%%%%%%%%%%%%%%%%%%%%%%%%%%%%%%%%%%%
%%%%%%%%%%%%%%%%%%%%%%%%%%%%%%%%%%%%%%%%%%%%%%%%%%%%%%%%%%%%%%%%%%%%%%%%%%%%%%%%%%%%%%%%%%%%%%%%%%%%%%%%%%%%%%%%%%%%%%%%%%
\subsection{Parametrische Lebensdauermodelle} \label{subsec:paramleben}

Während die deskriptiven Statistiken $\sym{x_bar}$ und $\sym{s_sq}$ die zentrale Tendenz und die Streuung der vorliegenden Stichprobe quantifizieren, erlauben sie keine Extrapolation oder die Modellierung der zugrundeliegenden Funktionen $\sym{F}(\sym{t})$ und $\sym{f}(\sym{t})$ der Grundgesamtheit.
Um eine prädiktive, mathematische Beschreibung des stochastischen Ausfallverhaltens zu erhalten, müssen die in Abschnitt~\ref{subsec:begriffezuv} definierten Lebensdauerfunktionen durch geeignete parametrische Verteilungsmodelle approximiert werden.
Andernfalls können nur nichtparametrische Modellierungsansätze zur Schätzung der kumulierten Wahrscheinlichkeit in Überlebensfunktionen wie beispielsweise nach \textcite{Kaplan.1958} genutzt werden \cite{Rigdon.2022,Meeker.2022}.
Die Verteilungsmodelle hingegen bieten eine geschlossene mathematische Form für \ac{cdf} und \ac{pdf} und ermöglichen es, das komplexe Ausfallverhalten durch eine geringe Anzahl von Parametern zu charakterisieren.\

\subsubsection{Weibull-Verteilung} \label{subsubsec:weibull}
In der Zuverlässigkeitstechnik hat sich die \textbf{Weibull-Verteilung} aufgrund ihrer hohen Flexibilität als das am häufigsten verwendete Modell etabliert.
Je nach zugrundeliegendem physikalischen Ausfallmechanismus finden jedoch auch andere statistische Verteilungen Anwendung, wie beispielsweise die \textbf{Lognormal-Verteilung} (häufig bei Ermüdungs-, Korrosions- oder Diffusionsprozessen), die \textbf{Exponentialverteilung} (zur Modellierung von Zufallsausfällen ohne Alterungseffekte) oder die \textbf{Beta-Verteilung} (allgemein zur formenreichen Modellierung von $\sym{R}$ über dem festen Intervall $[0,1]$).
Für weitere Ausführungen dazu sei an dieser Stelle jedoch auf bereits ausreichend diskutierte Aufbereitungen von \textcite{Bertsche.2022,Birolini.2017,Yang.2007,Hedderich.2020,Rigdon.2022} verwiesen.

Die (zweiparametrige) Weibull-Verteilung ist das Standardmodell zur Beschreibung der Lebensdauer von technischen Produkten ohne die Berücksichtigung eines möglichen dritten Parameters - der ausfallfreien Zeit \symsub{t}{0}.
Sie wird durch den \textbf{Formparameter}  $\sym{b} > 0$ (Weibull-Modul) und die \textbf{charakteristische Lebensdauer} $\sym{T} > 0$ (Skalenparameter), welche dem $63,2$-ten Perzentil $\sym{t}_{0,632}$ entspricht, beschrieben.
Unabhängig von $\sym{b}$ gilt hier: $\sym{F}(\sym{T})=1-e^{-1} \approx 63,2 \%$.
Folgt die Lebensdauer-Zufallsvariable $\sym{tau}$ dieser Verteilung, wird dies mathematisch als $\sym{tau} \sim \sym{W}(\sym{T}, \sym{b})$ notiert.
Damit ist sie in der Lage, alle drei Phasen der "Badewannenkurve" (Frühausfälle mit $\sym{b}<1$, Zufallsausfälle $\sym{b}\approx 1$, Verschleißausfälle mit $\sym{b}>1$) durch die Wahl ihrer Parametrisierung abzubilden, vgl. \textcite{Bertsche.2022}.

Die Einheit des Skalenparameters $\sym{T}$ entspricht der Einheit des Messwertes ($\sym{t}$ in Stunden, Überrollungen, Kilometer, etc.).
Die \ac{pdf} der Weibull-Verteilung ist damit definiert als:
\begin{equation} \label{eq:weibull_pdf}
    \sym{f}(\sym{t}) = \frac{\sym{b}}{\sym{T}^{\sym{b}}} \sym{t}^{\sym{b}-1} \exp\left[ - \left(\frac{\sym{t}}{\sym{T}}\right)^{\sym{b}} \right], \quad \sym{t} > 0.
\end{equation}
Die \ac{cdf} ergibt sich durch Integration der \ac{pdf} zu:
\begin{equation} \label{eq:weibull_cdf}
    \sym{F}(\sym{t}) = 1 - \exp\left[ - \left(\frac{\sym{t}}{\sym{T}}\right)^{\sym{b}} \right], \quad \sym{t} > 0.
\end{equation}
Aus $\sym{f}(\sym{t})$ und $\sym{R}(\sym{t}) = 1 - \sym{F}(\sym{t})$ leitet sich die \textbf{Ausfallrate} $\sym{lambda}(\sym{t})$ der Weibull-Verteilung ab:
\begin{equation} \label{eq:weibull_hazard}
    \sym{lambda}(\sym{t}) = \frac{\sym{b}}{\sym{T}} \left(\frac{\sym{t}}{\sym{T}}\right)^{\sym{b}-1}, \quad \sym{t} > 0.
\end{equation}
Der Erwartungswert $\sym{mu}$ (vgl. Gleichung~\eqref{eq:theo_mean}) und die Varianz $\sym{sigma_sq}$ (vgl. Gleichung~\eqref{eq:theo_variance}) der Weibull-Verteilung lassen sich ebenfalls in geschlossener Form ausdrücken. Sie sind von der \textbf{Gamma-Funktion} $\sym{Gamma}(\cdot)$ abhängig, welche für $x > 0$ definiert ist als:
\begin{equation} \label{eq:gamma_func}
    \sym{Gamma}(x) = \int_{0}^{\infty} \sym{ups}^{x-1} \exp(-\sym{ups}) \,d\sym{ups}.
\end{equation}
Der Erwartungswert $\sym{mu}$ der Weibull-verteilten Lebensdauer $\sym{tau}$ ergibt sich zu:
\begin{equation} \label{eq:weibull_mean}
    \sym{mu} = \sym{E}[\sym{tau}] = \sym{T} \cdot \sym{Gamma}\left(1 + \frac{1}{\sym{b}}\right).
\end{equation}
Die Varianz $\sym{sigma_sq}$ ist gegeben durch:
\begin{equation} \label{eq:weibull_variance}
    \sym{sigma_sq} = \sym{Var}[\sym{tau}] = \sym{T}^2 \left[ \sym{Gamma}\left(1 + \frac{2}{\sym{b}}\right) - \sym{Gamma}^2\left(1 + \frac{1}{\sym{b}}\right) \right].
\end{equation}

Die Ausprägung der erwähnten Flexibilität der Weibull-Verteilung ist durch Abbildung~\ref{fig:abb2.1_weibull} nachvollziehbar.
Ein Spezialfall tritt ein für $\sym{b}=1$, da sich in diesem Fall die Weibull-Verteilung zur Exponentialverteilung mit dem Ausfallraten-Parameter $\sym{lambda} = 1/\sym{T}$ und dem Erwartungswert $\sym{mu} = \sym{T}$ reduziert.
Für $\sym{b}=3,6$ wird die Schiefe der \ac{pdf} annähernd eliminiert, sodass sich die Weibull-Verteilung einer Normalverteilung annähert \cite{Rinne.2008, Kececioglu.2002}.

\begin{figure}[htbp]
    \centering
    % This file was created by matlab2tikz.
%
\definecolor{mycolor1}{rgb}{0.12941,0.12941,0.12941}%
%
\begin{tikzpicture}

\begin{axis}[%
width=4.882in,
height=2.451in,
at={(0.819in,0.462in)},
scale only axis,
unbounded coords=jump,
xmin=0,
xmax=7,
xlabel style={font=\color{mycolor1}},
xlabel={$t$},
ymin=0,
ymax=3,
ylabel style={font=\color{mycolor1}},
ylabel={$f$},
axis background/.style={fill=white},
xmajorgrids,
ymajorgrids,
grid style={dotted, opacity=0.25},
legend style={legend cell align=left, align=left}
]
\addplot [color=black, dotted, line width=1.5pt]
  table[row sep=crcr]{%
0	inf\\
0.00700700700700701	5.49350929513194\\
0.014014014014014	3.75211898510536\\
0.021021021021021	2.98315871663564\\
0.028028028028028	2.52618995132104\\
0.035035035035035	2.21528197312537\\
0.042042042042042	1.98645935981812\\
0.049049049049049	1.80913686378218\\
0.0560560560560561	1.6666118973931\\
0.0630630630630631	1.5488910827451\\
0.0700700700700701	1.44958175467026\\
0.0770770770770771	1.36437972414415\\
0.0840840840840841	1.29026710086587\\
0.0910910910910911	1.22505709206378\\
0.0980980980980981	1.16712132415402\\
0.105105105105105	1.11521906741584\\
0.112112112112112	1.06838622054997\\
0.119119119119119	1.02586082012055\\
0.126126126126126	0.987031675544172\\
0.133133133133133	0.951402099135577\\
0.14014014014014	0.918563755123738\\
0.147147147147147	0.888177452843577\\
0.154154154154154	0.859958805639075\\
0.161161161161161	0.833667363170408\\
0.168168168168168	0.809098265085044\\
0.175175175175175	0.786075752888918\\
0.182182182182182	0.764448070258062\\
0.189189189189189	0.744083413907585\\
0.196196196196196	0.72486668857057\\
0.203203203203203	0.706696884011354\\
0.21021021021021	0.689484937957122\\
0.217217217217217	0.673151982072907\\
0.224224224224224	0.657627892437254\\
0.231231231231231	0.642850083985808\\
0.238238238238238	0.628762501859296\\
0.245245245245245	0.615314772763253\\
0.252252252252252	0.602461487196898\\
0.259259259259259	0.590161589364006\\
0.266266266266266	0.57837785619164\\
0.273273273273273	0.567076450482615\\
0.28028028028028	0.556226536056809\\
0.287287287287287	0.545799944974934\\
0.294294294294294	0.53577088872054\\
0.301301301301301	0.526115706643436\\
0.308308308308308	0.516812646117481\\
0.315315315315315	0.507841669796782\\
0.322322322322322	0.499184286112247\\
0.329329329329329	0.490823399770336\\
0.336336336336336	0.482743179525325\\
0.343343343343343	0.474928940916839\\
0.35035035035035	0.467367042013014\\
0.357357357357357	0.460044790489703\\
0.364364364364364	0.452950360618567\\
0.371371371371371	0.446072718940125\\
0.378378378378378	0.439401557568951\\
0.385385385385385	0.432927234222687\\
0.392392392392392	0.426640718188946\\
0.399399399399399	0.42053354154826\\
0.406406406406406	0.41459775505995\\
0.413413413413413	0.408825888193642\\
0.42042042042042	0.403210912854229\\
0.427427427427427	0.397746210404017\\
0.434434434434434	0.392425541634009\\
0.441441441441441	0.387243019377979\\
0.448448448448448	0.382193083499094\\
0.455455455455455	0.377270478010229\\
0.462462462462462	0.372470230116401\\
0.469469469469469	0.367787630991586\\
0.476476476476476	0.363218218123023\\
0.483483483483483	0.358757759074334\\
0.49049049049049	0.354402236534825\\
0.497497497497497	0.350147834536429\\
0.504504504504504	0.345990925732169\\
0.511511511511512	0.341928059640977\\
0.518518518518518	0.337955951773423\\
0.525525525525526	0.334071473561486\\
0.532532532532533	0.330271643023145\\
0.53953953953954	0.32655361609933\\
0.546546546546547	0.322914678606837\\
0.553553553553554	0.319352238756172\\
0.560560560560561	0.315863820188124\\
0.567567567567568	0.312447055487142\\
0.574574574574575	0.309099680133491\\
0.581581581581582	0.305819526859556\\
0.588588588588589	0.302604520378846\\
0.595595595595596	0.299452672458966\\
0.602602602602603	0.296362077312411\\
0.60960960960961	0.29333090728126\\
0.616616616616617	0.290357408793911\\
0.623623623623624	0.287439898573864\\
0.630630630630631	0.284576760082206\\
0.637637637637638	0.281766440177006\\
0.644644644644645	0.27900744597418\\
0.651651651651652	0.276298341895646\\
0.658658658658659	0.273637746891737\\
0.665665665665666	0.271024331825849\\
0.672672672672673	0.268456817010283\\
0.67967967967968	0.265933969883062\\
0.686686686686687	0.263454602816315\\
0.693693693693694	0.261017571047529\\
0.700700700700701	0.258621770725613\\
0.707707707707708	0.256266137064359\\
0.714714714714715	0.253949642596377\\
0.721721721721722	0.251671295521142\\
0.728728728728729	0.24943013814121\\
0.735735735735736	0.247225245381125\\
0.742742742742743	0.245055723383891\\
0.74974974974975	0.242920708180273\\
0.756756756756757	0.240819364426515\\
0.763763763763764	0.238750884206351\\
0.770770770770771	0.236714485893499\\
0.777777777777778	0.234709413071059\\
0.784784784784785	0.232734933504495\\
0.791791791791792	0.230790338165083\\
0.798798798798799	0.228874940300949\\
0.805805805805806	0.226988074552959\\
0.812812812812813	0.225129096112957\\
0.81981981981982	0.223297379921957\\
0.826826826826827	0.221492319906086\\
0.833833833833834	0.219713328248192\\
0.840840840840841	0.217959834693181\\
0.847847847847848	0.216231285885241\\
0.854854854854855	0.214527144735261\\
0.861861861861862	0.212846889816815\\
0.868868868868869	0.211190014789223\\
0.875875875875876	0.209556027846257\\
0.882882882882883	0.207944451189159\\
0.88988988988989	0.206354820522724\\
0.896896896896897	0.204786684573265\\
0.903903903903904	0.203239604627341\\
0.910910910910911	0.201713154090216\\
0.917917917917918	0.200206918063047\\
0.924924924924925	0.198720492937883\\
0.931931931931932	0.197253486009588\\
0.938938938938939	0.195805515103874\\
0.945945945945946	0.194376208220644\\
0.952952952952953	0.192965203191923\\
0.95995995995996	0.19157214735367\\
0.966966966966967	0.190196697230816\\
0.973973973973974	0.188838518234902\\
0.980980980980981	0.18749728437373\\
0.987987987987988	0.186172677972466\\
0.994994994994995	0.184864389405667\\
1.002002002002	0.183572116839733\\
1.00900900900901	0.182295565985307\\
1.01601601601602	0.181034449859173\\
1.02302302302302	0.17978848855523\\
1.03003003003003	0.178557409024133\\
1.03703703703704	0.177340944861217\\
1.04404404404404	0.17613883610234\\
1.05105105105105	0.174950829027308\\
1.05805805805806	0.17377667597053\\
1.06506506506507	0.172616135138619\\
1.07207207207207	0.171468970434619\\
1.07907907907908	0.170334951288593\\
1.08608608608609	0.169213852494285\\
1.09309309309309	0.168105454051618\\
1.1001001001001	0.167009541014775\\
1.10710710710711	0.165925903345631\\
1.11411411411411	0.164854335772317\\
1.12112112112112	0.163794637652708\\
1.12812812812813	0.162746612842634\\
1.13513513513514	0.161710069568611\\
1.14214214214214	0.160684820304936\\
1.14914914914915	0.159670681654939\\
1.15615615615616	0.158667474236257\\
1.16316316316316	0.157675022569948\\
1.17017017017017	0.156693154973307\\
1.17717717717718	0.155721703456239\\
1.18418418418418	0.154760503621038\\
1.19119119119119	0.153809394565465\\
1.1981981981982	0.152868218788971\\
1.20520520520521	0.151936822101962\\
1.21221221221221	0.151015053537989\\
1.21921921921922	0.150102765268748\\
1.22622622622623	0.149199812521785\\
1.23323323323323	0.148306053500813\\
1.24024024024024	0.147421349308526\\
1.24724724724725	0.146545563871846\\
1.25425425425425	0.145678563869484\\
1.26126126126126	0.144820218661753\\
1.26826826826827	0.143970400222535\\
1.27527527527528	0.143128983073342\\
1.28228228228228	0.142295844219378\\
1.28928928928929	0.141470863087538\\
1.2962962962963	0.140653921466283\\
1.3033033033033	0.139844903447312\\
1.31031031031031	0.139043695368972\\
1.31731731731732	0.138250185761353\\
1.32432432432432	0.137464265292998\\
1.33133133133133	0.136685826719178\\
1.33833833833834	0.135914764831679\\
1.34534534534535	0.135150976410049\\
1.35235235235235	0.134394360174253\\
1.35935935935936	0.133644816738685\\
1.36636636636637	0.132902248567504\\
1.37337337337337	0.132166559931237\\
1.38038038038038	0.131437656864611\\
1.38738738738739	0.130715447125574\\
1.39439439439439	0.129999840155466\\
1.4014014014014	0.1292907470403\\
1.40840840840841	0.128588080473121\\
1.41541541541542	0.1278917547174\\
1.42242242242242	0.127201685571441\\
1.42942942942943	0.126517790333755\\
1.43643643643644	0.125839987769386\\
1.44344344344344	0.125168198077137\\
1.45045045045045	0.124502342857695\\
1.45745745745746	0.123842345082599\\
1.46446446446446	0.123188129064043\\
1.47147147147147	0.122539620425481\\
1.47847847847848	0.121896746073014\\
1.48548548548549	0.12125943416752\\
1.49249249249249	0.120627614097528\\
1.4994994994995	0.120001216452792\\
1.50650650650651	0.11938017299855\\
1.51351351351351	0.118764416650458\\
1.52052052052052	0.118153881450161\\
1.52752752752753	0.117548502541489\\
1.53453453453453	0.116948216147272\\
1.54154154154154	0.116352959546728\\
1.54854854854855	0.115762671053436\\
1.55555555555556	0.115177289993854\\
1.56256256256256	0.114596756686384\\
1.56956956956957	0.114021012420952\\
1.57657657657658	0.113449999439097\\
1.58358358358358	0.112883660914558\\
1.59059059059059	0.11232194093433\\
1.5975975975976	0.111764784480194\\
1.6046046046046	0.111212137410689\\
1.61161161161161	0.110663946443529\\
1.61861861861862	0.110120159138441\\
1.62562562562563	0.109580723880423\\
1.63263263263263	0.109045589863393\\
1.63963963963964	0.108514707074243\\
1.64664664664665	0.107988026277258\\
1.65365365365365	0.107465498998916\\
1.66066066066066	0.10694707751304\\
1.66766766766767	0.1064327148263\\
1.67467467467467	0.105922364664057\\
1.68168168168168	0.105415981456536\\
1.68868868868869	0.104913520325318\\
1.6956956956957	0.104414937070143\\
1.7027027027027	0.103920188156026\\
1.70970970970971	0.103429230700655\\
1.71671671671672	0.102942022462092\\
1.72372372372372	0.102458521826735\\
1.73073073073073	0.101978687797574\\
1.73773773773774	0.101502479982697\\
1.74474474474474	0.101029858584057\\
1.75175175175175	0.100560784386495\\
1.75875875875876	0.100095218747008\\
1.76576576576577	0.0996331235842472\\
1.77277277277277	0.0991744613682573\\
1.77977977977978	0.0987191951104396\\
1.78678678678679	0.0982672883537337\\
1.79379379379379	0.0978187051630152\\
1.8008008008008	0.097373410115703\\
1.80780780780781	0.0969313682925702\\
1.81481481481481	0.0964925452687531\\
1.82182182182182	0.0960569071049552\\
1.82882882882883	0.0956244203388389\\
1.83583583583584	0.0951950519766016\\
1.84284284284284	0.0947687694847314\\
1.84984984984985	0.0943455407819378\\
1.85685685685686	0.0939253342312528\\
1.86386386386386	0.0935081186322985\\
1.87087087087087	0.0930938632137171\\
1.87787787787788	0.092682537625759\\
1.88488488488488	0.0922741119330258\\
1.89189189189189	0.0918685566073631\\
1.8988988988989	0.0914658425209014\\
1.90590590590591	0.0910659409392391\\
1.91291291291291	0.0906688235147672\\
1.91991991991992	0.0902744622801289\\
1.92692692692693	0.0898828296418135\\
1.93393393393393	0.0894938983738805\\
1.94094094094094	0.08910764161181\\
1.94794794794795	0.0887240328464782\\
1.95495495495495	0.0883430459182531\\
1.96196196196196	0.0879646550112087\\
1.96896896896897	0.0875888346474553\\
1.97597597597598	0.0872155596815815\\
1.98298298298298	0.0868448052952076\\
1.98998998998999	0.0864765469916453\\
1.996996996997	0.0861107605906637\\
2.004004004004	0.0857474222233572\\
2.01101101101101	0.085386508327114\\
2.01801801801802	0.085027995640683\\
2.02502502502503	0.0846718611993355\\
2.03203203203203	0.0843180823301218\\
2.03903903903904	0.0839666366472182\\
2.04604604604605	0.0836175020473642\\
2.05305305305305	0.0832706567053864\\
2.06006006006006	0.082926079069809\\
2.06706706706707	0.0825837478585463\\
2.07407407407407	0.0822436420546783\\
2.08108108108108	0.0819057409023052\\
2.08808808808809	0.0815700239024806\\
2.0950950950951	0.0812364708092208\\
2.1021021021021	0.080905061625589\\
2.10910910910911	0.0805757765998521\\
2.11611611611612	0.0802485962217095\\
2.12312312312312	0.0799235012185918\\
2.13013013013013	0.0796004725520271\\
2.13713713713714	0.0792794914140758\\
2.14414414414414	0.0789605392238287\\
2.15115115115115	0.0786435976239714\\
2.15815815815816	0.0783286484774096\\
2.16516516516517	0.078015673863957\\
2.17217217217217	0.0777046560770825\\
2.17917917917918	0.0773955776207164\\
2.18618618618619	0.0770884212061141\\
2.19319319319319	0.0767831697487765\\
2.2002002002002	0.0764798063654247\\
2.20720720720721	0.0761783143710295\\
2.21421421421421	0.0758786772758934\\
2.22122122122122	0.0755808787827842\\
2.22822822822823	0.0752849027841195\\
2.23523523523524	0.0749907333592009\\
2.24224224224224	0.0746983547714964\\
2.24924924924925	0.0744077514659709\\
2.25625625625626	0.074118908066463\\
2.26326326326326	0.0738318093731075\\
2.27027027027027	0.0735464403598027\\
2.27727727727728	0.0732627861717217\\
2.28428428428428	0.0729808321228664\\
2.29129129129129	0.0727005636936635\\
2.2982982982983	0.0724219665286019\\
2.30530530530531	0.0721450264339108\\
2.31231231231231	0.0718697293752767\\
2.31931931931932	0.0715960614755997\\
2.32632632632633	0.0713240090127878\\
2.33333333333333	0.0710535584175882\\
2.34034034034034	0.0707846962714557\\
2.34734734734735	0.0705174093044558\\
2.35435435435435	0.0702516843932045\\
2.36136136136136	0.0699875085588412\\
2.36836836836837	0.0697248689650357\\
2.37537537537538	0.069463752916029\\
2.38238238238238	0.0692041478547051\\
2.38938938938939	0.0689460413606965\\
2.3963963963964	0.0686894211485192\\
2.4034034034034	0.068434275065739\\
2.41041041041041	0.0681805910911683\\
2.41741741741742	0.0679283573330914\\
2.42442442442442	0.0676775620275194\\
2.43143143143143	0.0674281935364732\\
2.43843843843844	0.067180240346294\\
2.44544544544545	0.0669336910659814\\
2.45245245245245	0.0666885344255585\\
2.45945945945946	0.0664447592744626\\
2.46646646646647	0.0662023545799621\\
2.47347347347347	0.0659613094255984\\
2.48048048048048	0.0657216130096531\\
2.48748748748749	0.0654832546436385\\
2.49449449449449	0.0652462237508137\\
2.5015015015015	0.065010509864722\\
2.50850850850851	0.0647761026277533\\
2.51551551551552	0.0645429917897277\\
2.52252252252252	0.0643111672065018\\
2.52952952952953	0.064080618838597\\
2.53653653653654	0.0638513367498489\\
2.54354354354354	0.0636233111060771\\
2.55055055055055	0.063396532173777\\
2.55755755755756	0.0631709903188303\\
2.56456456456456	0.0629466760052367\\
2.57157157157157	0.0627235797938639\\
2.57857857857858	0.0625016923412175\\
2.58558558558559	0.0622810043982295\\
2.59259259259259	0.0620615068090653\\
2.5995995995996	0.0618431905099489\\
2.60660660660661	0.0616260465280057\\
2.61361361361361	0.0614100659801227\\
2.62062062062062	0.0611952400718268\\
2.62762762762763	0.0609815600961785\\
2.63463463463463	0.0607690174326835\\
2.64164164164164	0.0605576035462201\\
2.64864864864865	0.0603473099859822\\
2.65565565565566	0.0601381283844389\\
2.66266266266266	0.0599300504563089\\
2.66966966966967	0.0597230679975503\\
2.67667667667668	0.0595171728843651\\
2.68368368368368	0.0593123570722191\\
2.69069069069069	0.0591086125948753\\
2.6976976976977	0.0589059315634419\\
2.7047047047047	0.0587043061654342\\
2.71171171171171	0.0585037286638504\\
2.71871871871872	0.0583041913962606\\
2.72572572572573	0.0581056867739085\\
2.73273273273273	0.0579082072808277\\
2.73973973973974	0.0577117454729686\\
2.74674674674675	0.0575162939773399\\
2.75375375375375	0.057321845491161\\
2.76076076076076	0.0571283927810274\\
2.76776776776777	0.0569359286820876\\
2.77477477477477	0.056744446097232\\
2.78178178178178	0.056553937996293\\
2.78878878878879	0.056364397415257\\
2.7957957957958	0.0561758174554869\\
2.8028028028028	0.0559881912829562\\
2.80980980980981	0.0558015121274934\\
2.81681681681682	0.0556157732820374\\
2.82382382382382	0.0554309681019031\\
2.83083083083083	0.0552470900040575\\
2.83783783783784	0.0550641324664054\\
2.84484484484484	0.0548820890270858\\
2.85185185185185	0.0547009532837775\\
2.85885885885886	0.0545207188930146\\
2.86586586586587	0.0543413795695112\\
2.87287287287287	0.0541629290854957\\
2.87987987987988	0.0539853612700541\\
2.88688688688689	0.0538086700084823\\
2.89389389389389	0.0536328492416475\\
2.9009009009009	0.053457892965358\\
2.90790790790791	0.0532837952297416\\
2.91491491491491	0.053110550138633\\
2.92192192192192	0.0529381518489686\\
2.92892892892893	0.0527665945701903\\
2.93593593593594	0.0525958725636571\\
2.94294294294294	0.0524259801420642\\
2.94994994994995	0.0522569116688707\\
2.95695695695696	0.0520886615577342\\
2.96396396396396	0.0519212242719534\\
2.97097097097097	0.0517545943239183\\
2.97797797797798	0.0515887662745672\\
2.98498498498498	0.0514237347328511\\
2.99199199199199	0.0512594943552057\\
2.998998998999	0.0510960398450294\\
3.00600600600601	0.0509333659521693\\
3.01301301301301	0.0507714674724131\\
3.02002002002002	0.0506103392469882\\
3.02702702702703	0.0504499761620668\\
3.03403403403403	0.0502903731482782\\
3.04104104104104	0.0501315251802272\\
3.04804804804805	0.0499734272760182\\
3.05505505505506	0.0498160744967862\\
3.06206206206206	0.0496594619462338\\
3.06906906906907	0.0495035847701737\\
3.07607607607608	0.0493484381560778\\
3.08308308308308	0.0491940173326311\\
3.09009009009009	0.0490403175692927\\
3.0970970970971	0.0488873341758612\\
3.1041041041041	0.0487350625020463\\
3.11111111111111	0.0485834979370458\\
3.11811811811812	0.0484326359091276\\
3.12512512512513	0.0482824718852175\\
3.13213213213213	0.0481330013704922\\
3.13913913913914	0.047984219907977\\
3.14614614614615	0.047836123078149\\
3.15315315315315	0.0476887064985451\\
3.16016016016016	0.0475419658233749\\
3.16716716716717	0.047395896743139\\
3.17417417417417	0.0472504949842509\\
3.18118118118118	0.0471057563086649\\
3.18818818818819	0.0469616765135079\\
3.1951951951952	0.0468182514307159\\
3.2022022022022	0.0466754769266752\\
3.20920920920921	0.0465333489018677\\
3.21621621621622	0.046391863290521\\
3.22322322322322	0.0462510160602626\\
3.23023023023023	0.0461108032117782\\
3.23723723723724	0.0459712207784743\\
3.24424424424424	0.0458322648261455\\
3.25125125125125	0.0456939314526447\\
3.25825825825826	0.0455562167875584\\
3.26526526526527	0.0454191169918855\\
3.27227227227227	0.0452826282577198\\
3.27927927927928	0.0451467468079367\\
3.28628628628629	0.045011468895884\\
3.29329329329329	0.0448767908050752\\
3.3003003003003	0.044742708848888\\
3.30730730730731	0.0446092193702656\\
3.31431431431431	0.0444763187414213\\
3.32132132132132	0.0443440033635475\\
3.32832832832833	0.0442122696665278\\
3.33533533533534	0.0440811141086521\\
3.34234234234234	0.0439505331763357\\
3.34934934934935	0.0438205233838413\\
3.35635635635636	0.0436910812730048\\
3.36336336336336	0.0435622034129639\\
3.37037037037037	0.0434338863998899\\
3.37737737737738	0.043306126856723\\
3.38438438438438	0.0431789214329103\\
3.39139139139139	0.0430522668041475\\
3.3983983983984	0.0429261596721227\\
3.40540540540541	0.042800596764264\\
3.41241241241241	0.0426755748334899\\
3.41941941941942	0.042551090657962\\
3.42642642642643	0.0424271410408414\\
3.43343343343343	0.0423037228100477\\
3.44044044044044	0.0421808328180201\\
3.44744744744745	0.0420584679414823\\
3.45445445445445	0.0419366250812094\\
3.46146146146146	0.0418153011617978\\
3.46846846846847	0.0416944931314376\\
3.47547547547548	0.0415741979616875\\
3.48248248248248	0.0414544126472528\\
3.48948948948949	0.0413351342057652\\
3.4964964964965	0.0412163596775656\\
3.5035035035035	0.0410980861254892\\
3.51051051051051	0.0409803106346534\\
3.51751751751752	0.0408630303122472\\
3.52452452452452	0.0407462422873239\\
3.53153153153153	0.040629943710596\\
3.53853853853854	0.0405141317542318\\
3.54554554554555	0.040398803611655\\
3.55255255255255	0.0402839564973459\\
3.55955955955956	0.0401695876466458\\
3.56656656656657	0.0400556943155621\\
3.57357357357357	0.0399422737805772\\
3.58058058058058	0.0398293233384586\\
3.58758758758759	0.0397168403060706\\
3.59459459459459	0.0396048220201899\\
3.6016016016016	0.0394932658373211\\
3.60860860860861	0.0393821691335157\\
3.61561561561562	0.0392715293041927\\
3.62262262262262	0.039161343763961\\
3.62962962962963	0.0390516099464435\\
3.63663663663664	0.0389423253041044\\
3.64364364364364	0.0388334873080766\\
3.65065065065065	0.038725093447992\\
3.65765765765766	0.0386171412318143\\
3.66466466466466	0.0385096281856716\\
3.67167167167167	0.0384025518536933\\
3.67867867867868	0.0382959097978466\\
3.68568568568569	0.0381896995977761\\
3.69269269269269	0.0380839188506449\\
3.6996996996997	0.0379785651709767\\
3.70670670670671	0.0378736361905007\\
3.71371371371371	0.0377691295579974\\
3.72072072072072	0.0376650429391462\\
3.72772772772773	0.0375613740163748\\
3.73473473473473	0.03745812048871\\
3.74174174174174	0.0373552800716304\\
3.74874874874875	0.0372528504969202\\
3.75575575575576	0.0371508295125251\\
3.76276276276276	0.0370492148824094\\
3.76976976976977	0.0369480043864144\\
3.77677677677678	0.0368471958201189\\
3.78378378378378	0.0367467869947006\\
3.79079079079079	0.0366467757367994\\
3.7977977977978	0.036547159888382\\
3.8048048048048	0.0364479373066072\\
3.81181181181181	0.0363491058636944\\
3.81881881881882	0.0362506634467912\\
3.82582582582583	0.0361526079578442\\
3.83283283283283	0.0360549373134705\\
3.83983983983984	0.0359576494448302\\
3.84684684684685	0.0358607422975006\\
3.85385385385385	0.0357642138313517\\
3.86086086086086	0.035668062020423\\
3.86786786786787	0.0355722848528016\\
3.87487487487487	0.035476880330501\\
3.88188188188188	0.0353818464693421\\
3.88888888888889	0.0352871812988347\\
3.8958958958959	0.0351928828620605\\
3.9029029029029	0.0350989492155571\\
3.90990990990991	0.0350053784292037\\
3.91691691691692	0.0349121685861073\\
3.92392392392392	0.0348193177824902\\
3.93093093093093	0.0347268241275793\\
3.93793793793794	0.0346346857434956\\
3.94494494494494	0.0345429007651456\\
3.95195195195195	0.0344514673401129\\
3.95895895895896	0.0343603836285522\\
3.96596596596597	0.0342696478030829\\
3.97297297297297	0.0341792580486849\\
3.97997997997998	0.0340892125625949\\
3.98698698698699	0.033999509554204\\
3.99399399399399	0.0339101472449558\\
4.001001001001	0.0338211238682464\\
4.00800800800801	0.0337324376693248\\
4.01501501501502	0.0336440869051941\\
4.02202202202202	0.0335560698445144\\
4.02902902902903	0.0334683847675058\\
4.03603603603604	0.0333810299658535\\
4.04304304304304	0.0332940037426123\\
4.05005005005005	0.0332073044121136\\
4.05705705705706	0.0331209302998722\\
4.06406406406406	0.0330348797424947\\
4.07107107107107	0.0329491510875884\\
4.07807807807808	0.0328637426936712\\
4.08508508508509	0.0327786529300825\\
4.09209209209209	0.0326938801768949\\
4.0990990990991	0.0326094228248266\\
4.10610610610611	0.032525279275155\\
4.11311311311311	0.0324414479396308\\
4.12012012012012	0.0323579272403932\\
4.12712712712713	0.0322747156098856\\
4.13413413413413	0.0321918114907725\\
4.14114114114114	0.0321092133358571\\
4.14814814814815	0.0320269196079993\\
4.15515515515516	0.0319449287800352\\
4.16216216216216	0.0318632393346966\\
4.16916916916917	0.0317818497645319\\
4.17617617617618	0.0317007585718277\\
4.18318318318318	0.0316199642685302\\
4.19019019019019	0.0315394653761693\\
4.1971971971972	0.0314592604257811\\
4.2042042042042	0.031379347957833\\
4.21121121121121	0.0312997265221486\\
4.21821821821822	0.0312203946778334\\
4.22522522522523	0.0311413509932013\\
4.23223223223223	0.0310625940457023\\
4.23923923923924	0.0309841224218497\\
4.24624624624625	0.030905934717149\\
4.25325325325325	0.0308280295360272\\
4.26026026026026	0.0307504054917628\\
4.26726726726727	0.0306730612064162\\
4.27427427427427	0.0305959953107608\\
4.28128128128128	0.0305192064442151\\
4.28828828828829	0.0304426932547754\\
4.2952952952953	0.0303664543989486\\
4.3023023023023	0.0302904885416858\\
4.30930930930931	0.0302147943563174\\
4.31631631631632	0.0301393705244873\\
4.32332332332332	0.0300642157360889\\
4.33033033033033	0.0299893286892015\\
4.33733733733734	0.0299147080900264\\
4.34434434434434	0.0298403526528249\\
4.35135135135135	0.0297662610998562\\
4.35835835835836	0.0296924321613153\\
4.36536536536537	0.029618864575273\\
4.37237237237237	0.0295455570876146\\
4.37937937937938	0.029472508451981\\
4.38638638638639	0.0293997174297086\\
4.39339339339339	0.0293271827897713\\
4.4004004004004	0.0292549033087217\\
4.40740740740741	0.029182877770634\\
4.41441441441441	0.0291111049670464\\
4.42142142142142	0.0290395836969049\\
4.42842842842843	0.028968312766507\\
4.43543543543544	0.0288972909894458\\
4.44244244244244	0.0288265171865557\\
4.44944944944945	0.0287559901858571\\
4.45645645645646	0.0286857088225024\\
4.46346346346346	0.028615671938723\\
4.47047047047047	0.0285458783837753\\
4.47747747747748	0.0284763270138886\\
4.48448448448448	0.028407016692213\\
4.49149149149149	0.0283379462887671\\
4.4984984984985	0.0282691146803874\\
4.50550550550551	0.0282005207506772\\
4.51251251251251	0.0281321633899562\\
4.51951951951952	0.0280640414952108\\
4.52652652652653	0.0279961539700446\\
4.53353353353353	0.0279284997246292\\
4.54054054054054	0.0278610776756558\\
4.54754754754755	0.0277938867462871\\
4.55455455455455	0.0277269258661095\\
4.56156156156156	0.0276601939710858\\
4.56856856856857	0.0275936900035081\\
4.57557557557558	0.0275274129119518\\
4.58258258258258	0.0274613616512292\\
4.58958958958959	0.0273955351823439\\
4.5965965965966	0.0273299324724454\\
4.6036036036036	0.0272645524947846\\
4.61061061061061	0.027199394228669\\
4.61761761761762	0.0271344566594187\\
4.62462462462462	0.0270697387783228\\
4.63163163163163	0.0270052395825959\\
4.63863863863864	0.0269409580753352\\
4.64564564564565	0.026876893265478\\
4.65265265265265	0.0268130441677596\\
4.65965965965966	0.026749409802671\\
4.66666666666667	0.026685989196418\\
4.67367367367367	0.0266227813808796\\
4.68068068068068	0.0265597853935676\\
4.68768768768769	0.0264970002775857\\
4.69469469469469	0.0264344250815902\\
4.7017017017017	0.0263720588597493\\
4.70870870870871	0.0263099006717045\\
4.71571571571572	0.026247949582531\\
4.72272272272272	0.0261862046626995\\
4.72972972972973	0.0261246649880374\\
4.73673673673674	0.0260633296396906\\
4.74374374374374	0.0260021977040864\\
4.75075075075075	0.0259412682728955\\
4.75775775775776	0.025880540442995\\
4.76476476476476	0.0258200133164317\\
4.77177177177177	0.0257596860003858\\
4.77877877877878	0.0256995576071343\\
4.78578578578579	0.0256396272540153\\
4.79279279279279	0.0255798940633927\\
4.7997997997998	0.0255203571626205\\
4.80680680680681	0.0254610156840082\\
4.81381381381381	0.0254018687647857\\
4.82082082082082	0.0253429155470694\\
4.82782782782783	0.0252841551778276\\
4.83483483483483	0.0252255868088472\\
4.84184184184184	0.0251672095966997\\
4.84884884884885	0.025109022702708\\
4.85585585585586	0.0250510252929139\\
4.86286286286286	0.0249932165380446\\
4.86986986986987	0.0249355956134809\\
4.87687687687688	0.0248781616992249\\
4.88388388388388	0.0248209139798677\\
4.89089089089089	0.0247638516445583\\
4.8978978978979	0.0247069738869718\\
4.9049049049049	0.0246502799052786\\
4.91191191191191	0.0245937689021132\\
4.91891891891892	0.0245374400845439\\
4.92592592592593	0.0244812926640422\\
4.93293293293293	0.0244253258564528\\
4.93993993993994	0.0243695388819636\\
4.94694694694695	0.0243139309650763\\
4.95395395395395	0.0242585013345768\\
4.96096096096096	0.024203249223506\\
4.96796796796797	0.0241481738691314\\
4.97497497497497	0.0240932745129175\\
4.98198198198198	0.0240385504004984\\
4.98898898898899	0.0239840007816487\\
4.995995995996	0.023929624910256\\
5.003003003003	0.0238754220442932\\
5.01001001001001	0.0238213914457905\\
5.01701701701702	0.0237675323808086\\
5.02402402402402	0.0237138441194113\\
5.03103103103103	0.0236603259356385\\
5.03803803803804	0.0236069771074801\\
5.04504504504505	0.0235537969168488\\
5.05205205205205	0.0235007846495546\\
5.05905905905906	0.0234479395952781\\
5.06606606606607	0.0233952610475451\\
5.07307307307307	0.0233427483037009\\
5.08008008008008	0.0232904006648849\\
5.08708708708709	0.0232382174360052\\
5.09409409409409	0.0231861979257137\\
5.1011011011011	0.0231343414463816\\
5.10810810810811	0.0230826473140742\\
5.11511511511512	0.023031114848527\\
5.12212212212212	0.022979743373121\\
5.12912912912913	0.022928532214859\\
5.13613613613614	0.0228774807043419\\
5.14314314314314	0.0228265881757444\\
5.15015015015015	0.0227758539667922\\
5.15715715715716	0.0227252774187382\\
5.16416416416416	0.0226748578763399\\
5.17117117117117	0.0226245946878358\\
5.17817817817818	0.0225744872049233\\
5.18518518518519	0.0225245347827358\\
5.19219219219219	0.0224747367798204\\
5.1991991991992	0.0224250925581154\\
5.20620620620621	0.0223756014829287\\
5.21321321321321	0.0223262629229156\\
5.22022022022022	0.0222770762500572\\
5.22722722722723	0.0222280408396389\\
5.23423423423423	0.0221791560702288\\
5.24124124124124	0.0221304213236569\\
5.24824824824825	0.0220818359849935\\
5.25525525525526	0.0220333994425289\\
5.26226226226226	0.0219851110877522\\
5.26926926926927	0.0219369703153309\\
5.27627627627628	0.0218889765230907\\
5.28328328328328	0.0218411291119949\\
5.29029029029029	0.0217934274861245\\
5.2972972972973	0.0217458710526583\\
5.3043043043043	0.0216984592218527\\
5.31131131131131	0.0216511914070229\\
5.31831831831832	0.0216040670245224\\
5.32532532532533	0.0215570854937245\\
5.33233233233233	0.0215102462370024\\
5.33933933933934	0.0214635486797109\\
5.34634634634635	0.0214169922501667\\
5.35335335335335	0.0213705763796302\\
5.36036036036036	0.0213243005022868\\
5.36736736736737	0.0212781640552281\\
5.37437437437437	0.0212321664784339\\
5.38138138138138	0.0211863072147537\\
5.38838838838839	0.0211405857098892\\
5.3953953953954	0.0210950014123754\\
5.4024024024024	0.0210495537735638\\
5.40940940940941	0.0210042422476041\\
5.41641641641642	0.020959066291427\\
5.42342342342342	0.0209140253647265\\
5.43043043043043	0.0208691189299428\\
5.43743743743744	0.0208243464522453\\
5.44444444444444	0.0207797073995154\\
5.45145145145145	0.0207352012423294\\
5.45845845845846	0.0206908274539422\\
5.46546546546547	0.0206465855102705\\
5.47247247247247	0.020602474889876\\
5.47947947947948	0.0205584950739493\\
5.48648648648649	0.0205146455462936\\
5.49349349349349	0.0204709257933085\\
5.5005005005005	0.0204273353039738\\
5.50750750750751	0.020383873569834\\
5.51451451451451	0.020340540084982\\
5.52152152152152	0.0202973343460437\\
5.52852852852853	0.0202542558521626\\
5.53553553553554	0.0202113041049838\\
5.54254254254254	0.0201684786086393\\
5.54954954954955	0.0201257788697322\\
5.55655655655656	0.0200832043973221\\
5.56356356356356	0.0200407547029096\\
5.57057057057057	0.0199984293004218\\
5.57757757757758	0.0199562277061972\\
5.58458458458458	0.0199141494389713\\
5.59159159159159	0.0198721940198617\\
5.5985985985986	0.0198303609723541\\
5.60560560560561	0.0197886498222873\\
5.61261261261261	0.0197470600978395\\
5.61961961961962	0.0197055913295139\\
5.62662662662663	0.0196642430501246\\
5.63363363363363	0.0196230147947828\\
5.64064064064064	0.0195819061008827\\
5.64764764764765	0.0195409165080881\\
5.65465465465465	0.0195000455583182\\
5.66166166166166	0.0194592927957348\\
5.66866866866867	0.019418657766728\\
5.67567567567568	0.0193781400199036\\
5.68268268268268	0.0193377391060693\\
5.68968968968969	0.0192974545782216\\
5.6966966966967	0.019257285991533\\
5.7037037037037	0.0192172329033388\\
5.71071071071071	0.019177294873124\\
5.71771771771772	0.019137471462511\\
5.72472472472472	0.0190977622352466\\
5.73173173173173	0.019058166757189\\
5.73873873873874	0.0190186845962962\\
5.74574574574575	0.0189793153226126\\
5.75275275275275	0.0189400585082572\\
5.75975975975976	0.018900913727411\\
5.76676676676677	0.0188618805563052\\
5.77377377377377	0.0188229585732089\\
5.78078078078078	0.0187841473584167\\
5.78778778778779	0.0187454464942376\\
5.79479479479479	0.0187068555649823\\
5.8018018018018	0.0186683741569522\\
5.80880880880881	0.018630001858427\\
5.81581581581582	0.0185917382596536\\
5.82282282282282	0.0185535829528344\\
5.82982982982983	0.018515535532116\\
5.83683683683684	0.0184775955935776\\
5.84384384384384	0.0184397627352197\\
5.85085085085085	0.0184020365569535\\
5.85785785785786	0.0183644166605888\\
5.86486486486486	0.0183269026498237\\
5.87187187187187	0.0182894941302336\\
5.87887887887888	0.0182521907092598\\
5.88588588588589	0.018214991996199\\
5.89289289289289	0.0181778976021929\\
5.8998998998999	0.0181409071402167\\
5.90690690690691	0.0181040202250693\\
5.91391391391391	0.0180672364733623\\
5.92092092092092	0.0180305555035098\\
5.92792792792793	0.0179939769357179\\
5.93493493493493	0.0179575003919742\\
5.94194194194194	0.0179211254960382\\
5.94894894894895	0.0178848518734302\\
5.95595595595596	0.0178486791514221\\
5.96296296296296	0.0178126069590268\\
5.96996996996997	0.0177766349269884\\
5.97697697697698	0.0177407626877724\\
5.98398398398398	0.0177049898755557\\
5.99099099099099	0.017669316126217\\
5.997997997998	0.0176337410773269\\
6.00500500500501	0.0175982643681386\\
6.01201201201201	0.0175628856395781\\
6.01901901901902	0.0175276045342347\\
6.02602602602603	0.0174924206963514\\
6.03303303303303	0.0174573337718162\\
6.04004004004004	0.017422343408152\\
6.04704704704705	0.0173874492545077\\
6.05405405405405	0.0173526509616491\\
6.06106106106106	0.0173179481819497\\
6.06806806806807	0.0172833405693816\\
6.07507507507508	0.0172488277795064\\
6.08208208208208	0.0172144094694666\\
6.08908908908909	0.0171800852979764\\
6.0960960960961	0.0171458549253129\\
6.1031031031031	0.0171117180133076\\
6.11011011011011	0.0170776742253374\\
6.11711711711712	0.0170437232263161\\
6.12412412412412	0.0170098646826858\\
6.13113113113113	0.0169760982624085\\
6.13813813813814	0.0169424236349572\\
6.14514514514515	0.016908840471308\\
6.15215215215215	0.0168753484439314\\
6.15915915915916	0.0168419472267842\\
6.16616616616617	0.0168086364953012\\
6.17317317317317	0.0167754159263866\\
6.18018018018018	0.0167422851984067\\
6.18718718718719	0.016709243991181\\
6.19419419419419	0.0166762919859745\\
6.2012012012012	0.0166434288654899\\
6.20820820820821	0.0166106543138594\\
6.21521521521522	0.0165779680166367\\
6.22222222222222	0.0165453696607897\\
6.22922922922923	0.0165128589346921\\
6.23623623623624	0.0164804355281161\\
6.24324324324324	0.0164480991322246\\
6.25025025025025	0.0164158494395635\\
6.25725725725726	0.0163836861440543\\
6.26426426426426	0.0163516089409864\\
6.27127127127127	0.0163196175270097\\
6.27827827827828	0.0162877116001271\\
6.28528528528529	0.0162558908596874\\
6.29229229229229	0.0162241550063774\\
6.2992992992993	0.0161925037422154\\
6.30630630630631	0.0161609367705431\\
6.31331331331331	0.0161294537960191\\
6.32032032032032	0.0160980545246115\\
6.32732732732733	0.0160667386635907\\
6.33433433433433	0.0160355059215224\\
6.34134134134134	0.0160043560082608\\
6.34834834834835	0.0159732886349414\\
6.35535535535536	0.0159423035139741\\
6.36236236236236	0.0159114003590365\\
6.36936936936937	0.0158805788850668\\
6.37637637637638	0.0158498388082574\\
6.38338338338338	0.0158191798460477\\
6.39039039039039	0.0157886017171177\\
6.3973973973974	0.0157581041413814\\
6.4044044044044	0.01572768683998\\
6.41141141141141	0.0156973495352752\\
6.41841841841842	0.0156670919508431\\
6.42542542542543	0.0156369138114675\\
6.43243243243243	0.0156068148431331\\
6.43943943943944	0.0155767947730198\\
6.44644644644645	0.0155468533294958\\
6.45345345345345	0.0155169902421114\\
6.46046046046046	0.0154872052415927\\
6.46746746746747	0.0154574980598356\\
6.47447447447447	0.0154278684298991\\
6.48148148148148	0.0153983160859997\\
6.48848848848849	0.0153688407635049\\
6.4954954954955	0.0153394421989271\\
6.5025025025025	0.0153101201299178\\
6.50950950950951	0.0152808742952615\\
6.51651651651652	0.0152517044348694\\
6.52352352352352	0.0152226102897739\\
6.53053053053053	0.0151935916021227\\
6.53753753753754	0.0151646481151724\\
6.54454454454454	0.0151357795732834\\
6.55155155155155	0.0151069857219134\\
6.55855855855856	0.0150782663076122\\
6.56556556556557	0.0150496210780158\\
6.57257257257257	0.0150210497818406\\
6.57957957957958	0.0149925521688778\\
6.58658658658659	0.014964127989988\\
6.59359359359359	0.0149357769970953\\
6.6006006006006	0.0149074989431819\\
6.60760760760761	0.0148792935822826\\
6.61461461461461	0.0148511606694795\\
6.62162162162162	0.0148230999608961\\
6.62862862862863	0.0147951112136924\\
6.63563563563564	0.014767194186059\\
6.64264264264264	0.0147393486372125\\
6.64964964964965	0.0147115743273894\\
6.65665665665666	0.0146838710178412\\
6.66366366366366	0.0146562384708292\\
6.67067067067067	0.0146286764496193\\
6.67767767767768	0.0146011847184764\\
6.68468468468468	0.0145737630426601\\
6.69169169169169	0.0145464111884186\\
6.6986986986987	0.0145191289229843\\
6.70570570570571	0.0144919160145684\\
6.71271271271271	0.0144647722323563\\
6.71971971971972	0.0144376973465019\\
6.72672672672673	0.0144106911281234\\
6.73373373373373	0.0143837533492978\\
6.74074074074074	0.0143568837830563\\
6.74774774774775	0.0143300822033794\\
6.75475475475475	0.0143033483851917\\
6.76176176176176	0.0142766821043579\\
6.76876876876877	0.014250083137677\\
6.77577577577578	0.0142235512628784\\
6.78278278278278	0.0141970862586166\\
6.78978978978979	0.0141706879044668\\
6.7967967967968	0.0141443559809202\\
6.8038038038038	0.0141180902693792\\
6.81081081081081	0.0140918905521529\\
6.81781781781782	0.0140657566124526\\
6.82482482482482	0.0140396882343871\\
6.83183183183183	0.0140136852029583\\
6.83883883883884	0.0139877473040563\\
6.84584584584585	0.0139618743244557\\
6.85285285285285	0.0139360660518103\\
6.85985985985986	0.0139103222746492\\
6.86686686686687	0.0138846427823723\\
6.87387387387387	0.0138590273652459\\
6.88088088088088	0.0138334758143983\\
6.88788788788789	0.0138079879218153\\
6.89489489489489	0.0137825634803367\\
6.9019019019019	0.0137572022836508\\
6.90890890890891	0.0137319041262914\\
6.91591591591592	0.0137066688036325\\
6.92292292292292	0.0136814961118851\\
6.92992992992993	0.013656385848092\\
6.93693693693694	0.0136313378101247\\
6.94394394394394	0.0136063517966784\\
6.95095095095095	0.0135814276072686\\
6.95795795795796	0.0135565650422265\\
6.96496496496496	0.0135317639026952\\
6.97197197197197	0.0135070239906256\\
6.97897897897898	0.0134823451087726\\
6.98598598598599	0.0134577270606909\\
6.99299299299299	0.013433169650731\\
7	0.0134086726840355\\
};
\addlegendentry{$\mathcal{W}(1,0; 0,5)$}

\addplot [color=black, dashed, line width=1.5pt]
  table[row sep=crcr]{%
0	1\\
0.00700700700700701	0.993017484828383\\
0.014014014014014	0.986083725174888\\
0.021021021021021	0.97919838060337\\
0.028028028028028	0.972361113054784\\
0.035035035035035	0.965571586830589\\
0.042042042042042	0.958829468576262\\
0.049049049049049	0.952134427264935\\
0.0560560560560561	0.945486134181139\\
0.0630630630630631	0.938884262904666\\
0.0700700700700701	0.932328489294541\\
0.0770770770770771	0.925818491473112\\
0.0840840840840841	0.919353949810237\\
0.0910910910910911	0.912934546907601\\
0.0980980980980981	0.906559967583126\\
0.105105105105105	0.900229898855496\\
0.112112112112112	0.893944029928795\\
0.119119119119119	0.88770205217724\\
0.126126126126126	0.881503659130037\\
0.133133133133133	0.875348546456326\\
0.14014014014014	0.869236411950242\\
0.147147147147147	0.863166955516078\\
0.154154154154154	0.857139879153548\\
0.161161161161161	0.851154886943161\\
0.168168168168168	0.845211685031684\\
0.175175175175175	0.839309981617723\\
0.182182182182182	0.833449486937387\\
0.189189189189189	0.827629913250071\\
0.196196196196196	0.821850974824318\\
0.203203203203203	0.816112387923799\\
0.21021021021021	0.810413870793377\\
0.217217217217217	0.804755143645273\\
0.224224224224224	0.799135928645333\\
0.231231231231231	0.793555949899383\\
0.238238238238238	0.788014933439684\\
0.245245245245245	0.782512607211481\\
0.252252252252252	0.777048701059645\\
0.259259259259259	0.771622946715411\\
0.266266266266266	0.766235077783203\\
0.273273273273273	0.760884829727557\\
0.28028028028028	0.755571939860131\\
0.287287287287287	0.750296147326809\\
0.294294294294294	0.745057193094894\\
0.301301301301301	0.739854819940387\\
0.308308308308308	0.734688772435359\\
0.315315315315315	0.729558796935413\\
0.322322322322322	0.724464641567225\\
0.329329329329329	0.719406056216182\\
0.336336336336336	0.714382792514099\\
0.343343343343343	0.709394603827027\\
0.35035035035035	0.704441245243142\\
0.357357357357357	0.699522473560719\\
0.364364364364364	0.694638047276194\\
0.371371371371371	0.689787726572306\\
0.378378378378378	0.68497127330632\\
0.385385385385385	0.680188450998337\\
0.392392392392392	0.675439024819682\\
0.399399399399399	0.670722761581377\\
0.406406406406406	0.666039429722686\\
0.413413413413413	0.661388799299752\\
0.42042042042042	0.656770641974304\\
0.427427427427427	0.652184731002446\\
0.434434434434434	0.647630841223525\\
0.441441441441441	0.643108749049074\\
0.448448448448448	0.63861823245184\\
0.455455455455455	0.634159070954873\\
0.462462462462462	0.629731045620713\\
0.469469469469469	0.625333939040628\\
0.476476476476476	0.62096753532395\\
0.483483483483483	0.616631620087469\\
0.49049049049049	0.612325980444909\\
0.497497497497497	0.608050404996477\\
0.504504504504504	0.603804683818482\\
0.511511511511512	0.599588608453026\\
0.518518518518518	0.595401971897774\\
0.525525525525526	0.591244568595787\\
0.532532532532533	0.587116194425431\\
0.53953953953954	0.583016646690353\\
0.546546546546547	0.578945724109533\\
0.553553553553554	0.574903226807395\\
0.560560560560561	0.570888956304001\\
0.567567567567568	0.5669027155053\\
0.574574574574575	0.562944308693453\\
0.581581581581582	0.559013541517226\\
0.588588588588589	0.555110220982443\\
0.595595595595596	0.551234155442513\\
0.602602602602603	0.547385154589022\\
0.60960960960961	0.543563029442387\\
0.616616616616617	0.539767592342575\\
0.623623623623624	0.535998656939896\\
0.630630630630631	0.532256038185847\\
0.637637637637638	0.528539552324029\\
0.644644644644645	0.524849016881127\\
0.651651651651652	0.521184250657947\\
0.658658658658659	0.51754507372052\\
0.665665665665666	0.513931307391271\\
0.672672672672673	0.510342774240242\\
0.67967967967968	0.506779298076385\\
0.686686686686687	0.503240703938905\\
0.693693693693694	0.499726818088676\\
0.700700700700701	0.496237467999708\\
0.707707707707708	0.492772482350676\\
0.714714714714715	0.489331691016507\\
0.721721721721722	0.485914925060031\\
0.728728728728729	0.482522016723684\\
0.735735735735736	0.479152799421272\\
0.742742742742743	0.47580710772979\\
0.74974974974975	0.472484777381304\\
0.756756756756757	0.469185645254881\\
0.763763763763764	0.465909549368584\\
0.770770770770771	0.462656328871517\\
0.777777777777778	0.459425824035927\\
0.784784784784785	0.456217876249363\\
0.791791791791792	0.453032328006889\\
0.798798798798799	0.449869022903348\\
0.805805805805806	0.446727805625685\\
0.812812812812813	0.443608521945321\\
0.81981981981982	0.440511018710579\\
0.826826826826827	0.437435143839168\\
0.833833833833834	0.434380746310712\\
0.840840840840841	0.43134767615934\\
0.847847847847848	0.428335784466315\\
0.854854854854855	0.425344923352733\\
0.861861861861862	0.422374945972252\\
0.868868868868869	0.41942570650389\\
0.875875875875876	0.416497060144861\\
0.882882882882883	0.413588863103465\\
0.88988988988989	0.410700972592033\\
0.896896896896897	0.407833246819912\\
0.903903903903904	0.404985544986502\\
0.910910910910911	0.402157727274348\\
0.917917917917918	0.399349654842272\\
0.924924924924925	0.396561189818556\\
0.931931931931932	0.393792195294173\\
0.938938938938939	0.391042535316068\\
0.945945945945946	0.388312074880476\\
0.952952952952953	0.385600679926301\\
0.95995995995996	0.382908217328529\\
0.966966966966967	0.380234554891696\\
0.973973973973974	0.377579561343392\\
0.980980980980981	0.374943106327819\\
0.987987987987988	0.372325060399392\\
0.994994994994995	0.36972529501638\\
1.002002002002	0.367143682534598\\
1.00900900900901	0.364580096201137\\
1.01601601601602	0.362034410148143\\
1.02302302302302	0.359506499386636\\
1.03003003003003	0.356996239800374\\
1.03703703703704	0.354503508139758\\
1.04404404404404	0.35202818201578\\
1.05105105105105	0.349570139894018\\
1.05805805805806	0.347129261088664\\
1.06506506506507	0.3447054257566\\
1.07207207207207	0.342298514891516\\
1.07907907907908	0.339908410318064\\
1.08608608608609	0.337534994686058\\
1.09309309309309	0.335178151464711\\
1.1001001001001	0.332837764936914\\
1.10710710710711	0.330513720193555\\
1.11411411411411	0.328205903127876\\
1.12112112112112	0.325914200429872\\
1.12812812812813	0.323638499580725\\
1.13513513513514	0.321378688847283\\
1.14214214214214	0.319134657276573\\
1.14914914914915	0.31690629469035\\
1.15615615615616	0.314693491679694\\
1.16316316316316	0.312496139599631\\
1.17017017017017	0.310314130563805\\
1.17717717717718	0.308147357439176\\
1.18418418418418	0.305995713840764\\
1.19119119119119	0.303859094126421\\
1.1981981981982	0.301737393391649\\
1.20520520520521	0.299630507464448\\
1.21221221221221	0.297538332900198\\
1.21921921921922	0.295460766976585\\
1.22622622622623	0.293397707688553\\
1.23323323323323	0.2913490537433\\
1.24024024024024	0.289314704555302\\
1.24724724724725	0.287294560241372\\
1.25425425425425	0.285288521615764\\
1.26126126126126	0.283296490185294\\
1.26826826826827	0.281318368144509\\
1.27527527527528	0.279354058370886\\
1.28228228228228	0.277403464420058\\
1.28928928928929	0.275466490521086\\
1.2962962962963	0.27354304157175\\
1.3033033033033	0.271633023133885\\
1.31031031031031	0.269736341428741\\
1.31731731731732	0.267852903332378\\
1.32432432432432	0.265982616371098\\
1.33133133133133	0.264125388716901\\
1.33833833833834	0.262281129182976\\
1.34534534534535	0.260449747219227\\
1.35235235235235	0.258631152907825\\
1.35935935935936	0.256825256958793\\
1.36636636636637	0.255031970705624\\
1.37337337337337	0.253251206100925\\
1.38038038038038	0.251482875712095\\
1.38738738738739	0.249726892717033\\
1.39439439439439	0.247983170899876\\
1.4014014014014	0.246251624646762\\
1.40840840840841	0.24453216894163\\
1.41541541541542	0.242824719362047\\
1.42242242242242	0.241129192075058\\
1.42942942942943	0.239445503833074\\
1.43643643643644	0.237773571969784\\
1.44344344344344	0.236113314396096\\
1.45045045045045	0.234464649596104\\
1.45745745745746	0.232827496623092\\
1.46446446446446	0.231201775095551\\
1.47147147147147	0.229587405193242\\
1.47847847847848	0.227984307653268\\
1.48548548548549	0.226392403766188\\
1.49249249249249	0.224811615372152\\
1.4994994994995	0.22324186485706\\
1.50650650650651	0.221683075148756\\
1.51351351351351	0.220135169713239\\
1.52052052052052	0.21859807255091\\
1.52752752752753	0.217071708192837\\
1.53453453453453	0.215556001697052\\
1.54154154154154	0.214050878644869\\
1.54854854854855	0.212556265137233\\
1.55555555555556	0.21107208779109\\
1.56256256256256	0.209598273735784\\
1.56956956956957	0.208134750609479\\
1.57657657657658	0.206681446555608\\
1.58358358358358	0.205238290219342\\
1.59059059059059	0.203805210744088\\
1.5975975975976	0.202382137768013\\
1.6046046046046	0.200969001420584\\
1.61161161161161	0.19956573231914\\
1.61861861861862	0.198172261565487\\
1.62562562562563	0.196788520742512\\
1.63263263263263	0.195414441910827\\
1.63963963963964	0.194049957605432\\
1.64664664664665	0.1926950008324\\
1.65365365365365	0.191349505065593\\
1.66066066066066	0.190013404243392\\
1.66766766766767	0.188686632765452\\
1.67467467467467	0.187369125489485\\
1.68168168168168	0.186060817728062\\
1.68868868868869	0.184761645245433\\
1.6956956956957	0.183471544254374\\
1.7027027027027	0.182190451413058\\
1.70970970970971	0.180918303821942\\
1.71671671671672	0.179655039020682\\
1.72372372372372	0.178400594985063\\
1.73073073073073	0.177154910123954\\
1.73773773773774	0.175917923276287\\
1.74474474474474	0.174689573708051\\
1.75175175175175	0.173469801109312\\
1.75875875875876	0.172258545591248\\
1.76576576576577	0.171055747683217\\
1.77277277277277	0.169861348329827\\
1.77977977977978	0.168675288888042\\
1.78678678678679	0.167497511124305\\
1.79379379379379	0.166327957211671\\
1.8008008008008	0.165166569726977\\
1.80780780780781	0.164013291648014\\
1.81481481481481	0.162868066350735\\
1.82182182182182	0.161730837606469\\
1.82882882882883	0.160601549579164\\
1.83583583583584	0.159480146822642\\
1.84284284284284	0.158366574277881\\
1.84984984984985	0.157260777270309\\
1.85685685685686	0.156162701507119\\
1.86386386386386	0.155072293074604\\
1.87087087087087	0.153989498435514\\
1.87787787787788	0.152914264426418\\
1.88488488488488	0.151846538255104\\
1.89189189189189	0.15078626749798\\
1.8988988988989	0.149733400097504\\
1.90590590590591	0.148687884359625\\
1.91291291291291	0.147649668951249\\
1.91991991991992	0.146618702897712\\
1.92692692692693	0.145594935580286\\
1.93393393393393	0.144578316733686\\
1.94094094094094	0.143568796443607\\
1.94794794794795	0.142566325144268\\
1.95495495495495	0.141570853615987\\
1.96196196196196	0.140582332982754\\
1.96896896896897	0.139600714709841\\
1.97597597597598	0.138625950601411\\
1.98298298298298	0.137657992798157\\
1.98998998998999	0.136696793774949\\
1.996996996997	0.135742306338504\\
2.004004004004	0.134794483625066\\
2.01101101101101	0.133853279098103\\
2.01801801801802	0.13291864654603\\
2.02502502502503	0.131990540079932\\
2.03203203203203	0.131068914131314\\
2.03903903903904	0.130153723449864\\
2.04604604604605	0.129244923101233\\
2.05305305305305	0.128342468464824\\
2.06006006006006	0.127446315231606\\
2.06706706706707	0.126556419401935\\
2.07407407407407	0.125672737283395\\
2.08108108108108	0.124795225488655\\
2.08808808808809	0.123923840933335\\
2.0950950950951	0.123058540833893\\
2.1021021021021	0.122199282705524\\
2.10910910910911	0.121346024360072\\
2.11611611611612	0.120498723903962\\
2.12312312312312	0.119657339736142\\
2.13013013013013	0.118821830546039\\
2.13713713713714	0.117992155311532\\
2.14414414414414	0.117168273296938\\
2.15115115115115	0.11635014405101\\
2.15815815815816	0.115537727404954\\
2.16516516516517	0.114730983470454\\
2.17217217217217	0.113929872637717\\
2.17917917917918	0.113134355573524\\
2.18618618618619	0.112344393219301\\
2.19319319319319	0.111559946789201\\
2.2002002002002	0.110780977768201\\
2.20720720720721	0.110007447910208\\
2.21421421421421	0.109239319236184\\
2.22122122122122	0.10847655403228\\
2.22822822822823	0.107719114847985\\
2.23523523523524	0.106966964494286\\
2.24224224224224	0.106220066041843\\
2.24924924924925	0.105478382819175\\
2.25625625625626	0.104741878410863\\
2.26326326326326	0.104010516655755\\
2.27027027027027	0.103284261645199\\
2.27727727727728	0.102563077721272\\
2.28428428428428	0.101846929475035\\
2.29129129129129	0.101135781744793\\
2.2982982982983	0.100429599614367\\
2.30530530530531	0.0997283484113803\\
2.31231231231231	0.0990319937055575\\
2.31931931931932	0.098340501307033\\
2.32632632632633	0.0976538372646722\\
2.33333333333333	0.0969719678644051\\
2.34034034034034	0.0962948596275703\\
2.34734734734735	0.0956224793092721\\
2.35435435435435	0.0949547938967475\\
2.36136136136136	0.0942917706077457\\
2.36836836836837	0.0936333768889185\\
2.37537537537538	0.0929795804142219\\
2.38238238238238	0.092330349083329\\
2.38938938938939	0.091685651020054\\
2.3963963963964	0.0910454545707869\\
2.4034034034034	0.0904097283029396\\
2.41041041041041	0.0897784410034026\\
2.41741741741742	0.0891515616770122\\
2.42442442442442	0.0885290595450291\\
2.43143143143143	0.087910904043627\\
2.43843843843844	0.0872970648223918\\
2.44544544544545	0.0866875117428318\\
2.45245245245245	0.0860822148768978\\
2.45945945945946	0.0854811445055134\\
2.46646646646647	0.0848842711171165\\
2.47347347347347	0.0842915654062096\\
2.48048048048048	0.0837029982719214\\
2.48748748748749	0.0831185408165779\\
2.49449449449449	0.0825381643442835\\
2.5015015015015	0.0819618403595122\\
2.50850850850851	0.0813895405657082\\
2.51551551551552	0.0808212368638972\\
2.52252252252252	0.0802569013513062\\
2.52952952952953	0.0796965063199938\\
2.53653653653654	0.0791400242554895\\
2.54354354354354	0.0785874278354435\\
2.55055055055055	0.0780386899282841\\
2.55755755755756	0.0774937835918868\\
2.56456456456456	0.0769526820722504\\
2.57157157157157	0.0764153588021843\\
2.57857857857858	0.0758817874000035\\
2.58558558558559	0.0753519416682336\\
2.59259259259259	0.0748257955923244\\
2.5995995995996	0.0743033233393726\\
2.60660660660661	0.0737844992568539\\
2.61361361361361	0.0732692978713628\\
2.62062062062062	0.0727576938873623\\
2.62762762762763	0.0722496621859419\\
2.63463463463463	0.0717451778235844\\
2.64164164164164	0.0712442160309409\\
2.64864864864865	0.0707467522116149\\
2.65565565565566	0.0702527619409546\\
2.66266266266266	0.0697622209648539\\
2.66966966966967	0.0692751051985612\\
2.67667667667668	0.0687913907254969\\
2.68368368368368	0.0683110537960794\\
2.69069069069069	0.0678340708265592\\
2.6976976976977	0.0673604183978602\\
2.7047047047047	0.0668900732544307\\
2.71171171171171	0.0664230123031011\\
2.71871871871872	0.0659592126119502\\
2.72572572572573	0.0654986514091793\\
2.73273273273273	0.0650413060819943\\
2.73973973973974	0.064587154175495\\
2.74674674674675	0.064136173391573\\
2.75375375375375	0.0636883415878169\\
2.76076076076076	0.0632436367764248\\
2.76776776776777	0.0628020371231253\\
2.77477477477477	0.0623635209461046\\
2.78178178178178	0.061928066714943\\
2.78878878878879	0.061495653049557\\
2.7957957957958	0.0610662587191499\\
2.8028028028028	0.0606398626411696\\
2.80980980980981	0.0602164438802729\\
2.81681681681682	0.0597959816472981\\
2.82382382382382	0.0593784552982441\\
2.83083083083083	0.0589638443332569\\
2.83783783783784	0.0585521283956231\\
2.84484484484484	0.0581432872707702\\
2.85185185185185	0.0577373008852744\\
2.85885885885886	0.0573341493058747\\
2.86586586586587	0.0569338127384947\\
2.87287287287287	0.0565362715272702\\
2.87987987987988	0.0561415061535843\\
2.88688688688689	0.0557494972351095\\
2.89389389389389	0.0553602255248553\\
2.9009009009009	0.0549736719102239\\
2.90790790790791	0.0545898174120713\\
2.91491491491491	0.0542086431837757\\
2.92192192192192	0.0538301305103122\\
2.92892892892893	0.0534542608073338\\
2.93593593593594	0.0530810156202591\\
2.94294294294294	0.0527103766233658\\
2.94994994994995	0.0523423256188915\\
2.95695695695696	0.0519768445361399\\
2.96396396396396	0.0516139154305935\\
2.97097097097097	0.0512535204830328\\
2.97797797797798	0.0508956419986613\\
2.98498498498498	0.0505402624062364\\
2.99199199199199	0.0501873642572074\\
2.998998998999	0.049836930224858\\
3.00600600600601	0.0494889431034561\\
3.01301301301301	0.0491433858074089\\
3.02002002002002	0.0488002413704241\\
3.02702702702703	0.0484594929446765\\
3.03403403403403	0.0481211237999815\\
3.04104104104104	0.0477851173229728\\
3.04804804804805	0.0474514570162877\\
3.05505505505506	0.0471201264977561\\
3.06206206206206	0.046791109499597\\
3.06906906906907	0.0464643898676193\\
3.07607607607608	0.0461399515604287\\
3.08308308308308	0.0458177786486404\\
3.09009009009009	0.0454978553140965\\
3.0970970970971	0.0451801658490898\\
3.1041041041041	0.0448646946555923\\
3.11111111111111	0.0445514262444897\\
3.11811811811812	0.0442403452348204\\
3.12512512512513	0.0439314363530207\\
3.13213213213213	0.0436246844321748\\
3.13913913913914	0.0433200744112701\\
3.14614614614615	0.0430175913344579\\
3.15315315315315	0.0427172203503186\\
3.16016016016016	0.0424189467111332\\
3.16716716716717	0.0421227557721587\\
3.17417417417417	0.0418286329909093\\
3.18118118118118	0.0415365639264423\\
3.18818818818819	0.041246534238649\\
3.1951951951952	0.0409585296875511\\
3.2022022022022	0.0406725361326006\\
3.20920920920921	0.0403885395319866\\
3.21621621621622	0.0401065259419451\\
3.22322322322322	0.0398264815160746\\
3.23023023023023	0.0395483925046565\\
3.23723723723724	0.0392722452539796\\
3.24424424424424	0.0389980262056703\\
3.25125125125125	0.0387257218960261\\
3.25825825825826	0.0384553189553552\\
3.26526526526527	0.0381868041073201\\
3.27227227227227	0.0379201641682852\\
3.27927927927928	0.0376553860466699\\
3.28628628628629	0.037392456742306\\
3.29329329329329	0.0371313633457988\\
3.3003003003003	0.0368720930378939\\
3.30730730730731	0.0366146330888476\\
3.31431431431431	0.0363589708578015\\
3.32132132132132	0.0361050937921625\\
3.32832832832833	0.0358529894269861\\
3.33533533533534	0.0356026453843644\\
3.34234234234234	0.0353540493728183\\
3.34934934934935	0.0351071891866945\\
3.35635635635636	0.0348620527055656\\
3.36336336336336	0.0346186278936353\\
3.37037037037037	0.0343769027991474\\
3.37737737737738	0.0341368655537992\\
3.38438438438438	0.0338985043721583\\
3.39139139139139	0.0336618075510846\\
3.3983983983984	0.0334267634691551\\
3.40540540540541	0.0331933605860937\\
3.41241241241241	0.0329615874422044\\
3.41941941941942	0.0327314326578086\\
3.42642642642643	0.0325028849326867\\
3.43343343343343	0.0322759330455229\\
3.44044044044044	0.0320505658533544\\
3.44744744744745	0.0318267722910245\\
3.45445445445445	0.0316045413706388\\
3.46146146146146	0.0313838621810263\\
3.46846846846847	0.0311647238872034\\
3.47547547547548	0.0309471157298417\\
3.48248248248248	0.0307310270247403\\
3.48948948948949	0.0305164471623007\\
3.4964964964965	0.0303033656070061\\
3.5035035035035	0.0300917718969041\\
3.51051051051051	0.0298816556430931\\
3.51751751751752	0.0296730065292122\\
3.52452452452452	0.0294658143109345\\
3.53153153153153	0.0292600688154644\\
3.53853853853854	0.0290557599410378\\
3.54554554554555	0.0288528776564267\\
3.55255255255255	0.0286514120004459\\
3.55955955955956	0.0284513530814645\\
3.56656656656657	0.0282526910769202\\
3.57357357357357	0.0280554162328365\\
3.58058058058058	0.0278595188633447\\
3.58758758758759	0.0276649893502075\\
3.59459459459459	0.027471818142347\\
3.6016016016016	0.0272799957553762\\
3.60860860860861	0.0270895127711326\\
3.61561561561562	0.0269003598372165\\
3.62262262262262	0.0267125276665312\\
3.62962962962963	0.0265260070368274\\
3.63663663663664	0.0263407887902503\\
3.64364364364364	0.0261568638328901\\
3.65065065065065	0.025974223134335\\
3.65765765765766	0.0257928577272285\\
3.66466466466466	0.0256127587068288\\
3.67167167167167	0.0254339172305714\\
3.67867867867868	0.0252563245176353\\
3.68568568568569	0.0250799718485116\\
3.69269269269269	0.0249048505645757\\
3.6996996996997	0.0247309520676617\\
3.70670670670671	0.0245582678196407\\
3.71371371371371	0.0243867893420014\\
3.72072072072072	0.0242165082154339\\
3.72772772772773	0.024047416079416\\
3.73473473473473	0.0238795046318033\\
3.74174174174174	0.0237127656284211\\
3.74874874874875	0.0235471908826596\\
3.75575575575576	0.0233827722650725\\
3.76276276276276	0.0232195017029771\\
3.76976976976977	0.0230573711800587\\
3.77677677677678	0.0228963727359764\\
3.78378378378378	0.0227364984659724\\
3.79079079079079	0.0225777405204843\\
3.7977977977978	0.0224200911047592\\
3.8048048048048	0.0222635424784712\\
3.81181181181181	0.0221080869553413\\
3.81881881881882	0.0219537169027602\\
3.82582582582583	0.0218004247414133\\
3.83283283283283	0.0216482029449087\\
3.83983983983984	0.0214970440394076\\
3.84684684684685	0.0213469406032576\\
3.85385385385385	0.0211978852666277\\
3.86086086086086	0.0210498707111473\\
3.86786786786787	0.0209028896695461\\
3.87487487487487	0.0207569349252979\\
3.88188188188188	0.0206119993122657\\
3.88888888888889	0.0204680757143505\\
3.8958958958959	0.0203251570651412\\
3.9029029029029	0.0201832363475684\\
3.90990990990991	0.0200423065935592\\
3.91691691691692	0.0199023608836954\\
3.92392392392392	0.019763392346874\\
3.93093093093093	0.0196253941599694\\
3.93793793793794	0.0194883595474984\\
3.94494494494494	0.0193522817812881\\
3.95195195195195	0.0192171541801448\\
3.95895895895896	0.0190829701095267\\
3.96596596596597	0.0189497229812174\\
3.97297297297297	0.0188174062530031\\
3.97997997997998	0.018686013428351\\
3.98698698698699	0.0185555380560905\\
3.99399399399399	0.0184259737300964\\
4.001001001001	0.0182973140889742\\
4.00800800800801	0.0181695528157481\\
4.01501501501502	0.0180426836375506\\
4.02202202202202	0.0179167003253147\\
4.02902902902903	0.0177915966934679\\
4.03603603603604	0.0176673665996285\\
4.04304304304304	0.0175440039443041\\
4.05005005005005	0.017421502670592\\
4.05705705705706	0.0172998567638823\\
4.06406406406406	0.0171790602515617\\
4.07107107107107	0.017059107202721\\
4.07807807807808	0.0169399917278638\\
4.08508508508509	0.0168217079786169\\
4.09209209209209	0.0167042501474437\\
4.0990990990991	0.0165876124673587\\
4.10610610610611	0.0164717892116445\\
4.11311311311311	0.0163567746935705\\
4.12012012012012	0.0162425632661139\\
4.12712712712713	0.0161291493216823\\
4.13413413413413	0.0160165272918384\\
4.14114114114114	0.0159046916470265\\
4.14814814814815	0.0157936368963013\\
4.15515515515516	0.0156833575870578\\
4.16216216216216	0.0155738483047643\\
4.16916916916917	0.0154651036726958\\
4.17617617617618	0.0153571183516706\\
4.18318318318318	0.0152498870397877\\
4.19019019019019	0.015143404472167\\
4.1971971971972	0.0150376654206901\\
4.2042042042042	0.0149326646937445\\
4.21121121121121	0.0148283971359677\\
4.21821821821822	0.0147248576279951\\
4.22522522522523	0.0146220410862077\\
4.23223223223223	0.0145199424624833\\
4.23923923923924	0.014418556743948\\
4.24624624624625	0.0143178789527305\\
4.25325325325325	0.0142179041457177\\
4.26026026026026	0.0141186274143116\\
4.26726726726727	0.0140200438841888\\
4.27427427427427	0.0139221487150607\\
4.28128128128128	0.0138249371004363\\
4.28828828828829	0.0137284042673859\\
4.2952952952953	0.0136325454763068\\
4.3023023023023	0.0135373560206907\\
4.30930930930931	0.0134428312268926\\
4.31631631631632	0.0133489664539014\\
4.32332332332332	0.0132557570931116\\
4.33033033033033	0.0131631985680977\\
4.33733733733734	0.0130712863343889\\
4.34434434434434	0.0129800158792465\\
4.35135135135135	0.0128893827214418\\
4.35835835835836	0.0127993824110366\\
4.36536536536537	0.0127100105291642\\
4.37237237237237	0.0126212626878129\\
4.37937937937938	0.0125331345296103\\
4.38638638638639	0.0124456217276094\\
4.39339339339339	0.0123587199850761\\
4.4004004004004	0.0122724250352786\\
4.40740740740741	0.0121867326412772\\
4.41441441441441	0.0121016385957171\\
4.42142142142142	0.012017138720621\\
4.42842842842843	0.0119332288671849\\
4.43543543543544	0.0118499049155734\\
4.44244244244244	0.0117671627747182\\
4.44944944944945	0.0116849983821168\\
4.45645645645646	0.0116034077036334\\
4.46346346346346	0.0115223867333003\\
4.47047047047047	0.0114419314931218\\
4.47747747747748	0.0113620380328785\\
4.48448448448448	0.0112827024299334\\
4.49149149149149	0.0112039207890395\\
4.4984984984985	0.0111256892421485\\
4.50550550550551	0.0110480039482205\\
4.51251251251251	0.010970861093036\\
4.51951951951952	0.0108942568890081\\
4.52652652652653	0.0108181875749971\\
4.53353353353353	0.0107426494161253\\
4.54054054054054	0.0106676387035939\\
4.54754754754755	0.0105931517545007\\
4.55455455455455	0.0105191849116597\\
4.56156156156156	0.010445734543421\\
4.56856856856857	0.0103727970434928\\
4.57557557557558	0.0103003688307645\\
4.58258258258258	0.0102284463491305\\
4.58958958958959	0.0101570260673156\\
4.5965965965966	0.0100861044787021\\
4.6036036036036	0.010015678101157\\
4.61061061061061	0.00994574347686166\\
4.61761761761762	0.00987629717214146\\
4.62462462462462	0.00980733577729759\\
4.63163163163163	0.00973885590643946\\
4.63863863863864	0.00967085419731857\\
4.64564564564565	0.00960332731116329\\
4.65265265265265	0.00953627193251509\\
4.65965965965966	0.00946968476906564\\
4.66666666666667	0.00940356255149521\\
4.67367367367367	0.00933790203331215\\
4.68068068068068	0.00927269999069347\\
4.68768768768769	0.0092079532223266\\
4.69469469469469	0.00914365854925217\\
4.7017017017017	0.00907981281470793\\
4.70870870870871	0.00901641288397379\\
4.71571571571572	0.00895345564421788\\
4.72272272272272	0.00889093800434373\\
4.72972972972973	0.00882885689483849\\
4.73673673673674	0.00876720926762225\\
4.74374374374374	0.00870599209589834\\
4.75075075075075	0.00864520237400476\\
4.75775775775776	0.00858483711726657\\
4.76476476476476	0.00852489336184939\\
4.77177177177177	0.00846536816461386\\
4.77877877877878	0.00840625860297112\\
4.78578578578579	0.00834756177473935\\
4.79279279279279	0.00828927479800122\\
4.7997997997998	0.00823139481096247\\
4.80680680680681	0.00817391897181136\\
4.81381381381381	0.00811684445857911\\
4.82082082082082	0.00806016846900144\\
4.82782782782783	0.00800388822038084\\
4.83483483483483	0.00794800094945011\\
4.84184184184184	0.00789250391223655\\
4.84884884884885	0.00783739438392731\\
4.85585585585586	0.0077826696587356\\
4.86286286286286	0.00772832704976779\\
4.86986986986987	0.00767436388889157\\
4.87687687687688	0.00762077752660487\\
4.88388388388388	0.00756756533190584\\
4.89089089089089	0.00751472469216361\\
4.8978978978979	0.00746225301299005\\
4.9049049049049	0.0074101477181124\\
4.91191191191191	0.00735840624924676\\
4.91891891891892	0.00730702606597247\\
4.92592592592593	0.00725600464560742\\
4.93293293293293	0.00720533948308415\\
4.93993993993994	0.00715502809082686\\
4.94694694694695	0.00710506799862931\\
4.95395395395395	0.00705545675353351\\
4.96096096096096	0.00700619191970928\\
4.96796796796797	0.00695727107833465\\
4.97497497497497	0.00690869182747713\\
4.98198198198198	0.00686045178197574\\
4.98898898898899	0.00681254857332395\\
4.995995995996	0.00676497984955334\\
5.003003003003	0.00671774327511815\\
5.01001001001001	0.00667083653078061\\
5.01701701701702	0.00662425731349706\\
5.02402402402402	0.00657800333630487\\
5.03103103103103	0.00653207232821018\\
5.03803803803804	0.00648646203407635\\
5.04504504504505	0.0064411702145133\\
5.05205205205205	0.00639619464576749\\
5.05905905905906	0.0063515331196128\\
5.06606606606607	0.00630718344324208\\
5.07307307307307	0.00626314343915947\\
5.08008008008008	0.00621941094507353\\
5.08708708708709	0.00617598381379103\\
5.09409409409409	0.00613285991311158\\
5.1011011011011	0.00609003712572288\\
5.10810810810811	0.00604751334909681\\
5.11511511511512	0.00600528649538618\\
5.12212212212212	0.00596335449132224\\
5.12912912912913	0.00592171527811285\\
5.13613613613614	0.00588036681134144\\
5.14314314314314	0.00583930706086657\\
5.15015015015015	0.00579853401072234\\
5.15715715715716	0.00575804565901933\\
5.16416416416416	0.00571784001784637\\
5.17117117117117	0.00567791511317288\\
5.17817817817818	0.005638268984752\\
5.18518518518519	0.00559889968602431\\
5.19219219219219	0.00555980528402228\\
5.1991991991992	0.00552098385927536\\
5.20620620620621	0.00548243350571572\\
5.21321321321321	0.00544415233058468\\
5.22022022022022	0.00540613845433978\\
5.22722722722723	0.00536839001056249\\
5.23423423423423	0.00533090514586658\\
5.24124124124124	0.00529368201980712\\
5.24824824824825	0.0052567188047901\\
5.25525525525526	0.00522001368598273\\
5.26226226226226	0.0051835648612243\\
5.26926926926927	0.00514737054093774\\
5.27627627627628	0.00511142894804171\\
5.28328328328328	0.00507573831786337\\
5.29029029029029	0.00504029689805173\\
5.2972972972973	0.00500510294849163\\
5.3043043043043	0.00497015474121828\\
5.31131131131131	0.00493545056033244\\
5.31831831831832	0.00490098870191616\\
5.32532532532533	0.00486676747394911\\
5.33233233233233	0.00483278519622553\\
5.33933933933934	0.00479904020027171\\
5.34634634634635	0.00476553082926412\\
5.35335335335335	0.00473225543794798\\
5.36036036036036	0.00469921239255654\\
5.36736736736737	0.00466640007073086\\
5.37437437437437	0.00463381686144015\\
5.38138138138138	0.00460146116490265\\
5.38838838838839	0.00456933139250711\\
5.3953953953954	0.00453742596673478\\
5.4024024024024	0.00450574332108197\\
5.40940940940941	0.0044742818999831\\
5.41641641641642	0.00444304015873438\\
5.42342342342342	0.00441201656341792\\
5.43043043043043	0.00438120959082642\\
5.43743743743744	0.00435061772838844\\
5.44444444444444	0.00432023947409407\\
5.45145145145145	0.00429007333642119\\
5.45845845845846	0.00426011783426228\\
5.46546546546547	0.00423037149685166\\
5.47247247247247	0.00420083286369332\\
5.47947947947948	0.00417150048448916\\
5.48648648648649	0.0041423729190678\\
5.49349349349349	0.00411344873731392\\
5.5005005005005	0.00408472651909796\\
5.50750750750751	0.00405620485420645\\
5.51451451451451	0.00402788234227276\\
5.52152152152152	0.00399975759270836\\
5.52852852852853	0.00397182922463448\\
5.53553553553554	0.0039440958668144\\
5.54254254254254	0.00391655615758606\\
5.54954954954955	0.00388920874479522\\
5.55655655655656	0.0038620522857291\\
5.56356356356356	0.00383508544705043\\
5.57057057057057	0.00380830690473195\\
5.57757757757758	0.00378171534399148\\
5.58458458458458	0.00375530945922733\\
5.59159159159159	0.00372908795395415\\
5.5985985985986	0.00370304954073938\\
5.60560560560561	0.00367719294113992\\
5.61261261261261	0.00365151688563944\\
5.61961961961962	0.00362602011358605\\
5.62662662662663	0.00360070137313035\\
5.63363363363363	0.00357555942116401\\
5.64064064064064	0.00355059302325871\\
5.64764764764765	0.00352580095360557\\
5.65465465465465	0.00350118199495491\\
5.66166166166166	0.00347673493855655\\
5.66866866866867	0.00345245858410039\\
5.67567567567568	0.00342835173965753\\
5.68268268268268	0.00340441322162173\\
5.68968968968969	0.0033806418546513\\
5.6966966966967	0.0033570364716114\\
5.7037037037037	0.0033335959135167\\
5.71071071071071	0.00331031902947453\\
5.71771771771772	0.00328720467662833\\
5.72472472472472	0.00326425172010156\\
5.73173173173173	0.00324145903294198\\
5.73873873873874	0.00321882549606629\\
5.74574574574575	0.00319634999820522\\
5.75275275275275	0.00317403143584895\\
5.75975975975976	0.00315186871319295\\
5.76676676676677	0.00312986074208413\\
5.77377377377377	0.00310800644196748\\
5.78078078078078	0.00308630473983296\\
5.78778778778779	0.00306475457016285\\
5.79479479479479	0.0030433548748794\\
5.8018018018018	0.00302210460329294\\
5.80880880880881	0.00300100271205023\\
5.81581581581582	0.00298004816508328\\
5.82282282282282	0.00295923993355844\\
5.82982982982983	0.00293857699582591\\
5.83683683683684	0.00291805833736959\\
5.84384384384384	0.00289768295075725\\
5.85085085085085	0.00287744983559105\\
5.85785785785786	0.00285735799845847\\
5.86486486486486	0.00283740645288349\\
5.87187187187187	0.00281759421927819\\
5.87887887887888	0.00279792032489462\\
5.88588588588589	0.00277838380377707\\
5.89289289289289	0.00275898369671462\\
5.8998998998999	0.00273971905119406\\
5.90690690690691	0.00272058892135313\\
5.91391391391391	0.00270159236793405\\
5.92092092092092	0.00268272845823743\\
5.92792792792793	0.00266399626607646\\
5.93493493493493	0.00264539487173145\\
5.94194194194194	0.00262692336190467\\
5.94894894894895	0.00260858082967549\\
5.95595595595596	0.00259036637445589\\
5.96296296296296	0.00257227910194621\\
5.96996996996997	0.00255431812409124\\
5.97697697697698	0.00253648255903663\\
5.98398398398398	0.00251877153108562\\
5.99099099099099	0.00250118417065598\\
5.997997997998	0.00248371961423736\\
6.00500500500501	0.00246637700434891\\
6.01201201201201	0.00244915548949711\\
6.01901901901902	0.00243205422413405\\
6.02602602602603	0.00241507236861584\\
6.03303303303303	0.00239820908916143\\
6.04004004004004	0.00238146355781165\\
6.04704704704705	0.00236483495238858\\
6.05405405405405	0.00234832245645515\\
6.06106106106106	0.00233192525927511\\
6.06806806806807	0.00231564255577314\\
6.07507507507508	0.00229947354649541\\
6.08208208208208	0.00228341743757028\\
6.08908908908909	0.00226747344066931\\
6.0960960960961	0.0022516407729686\\
6.1031031031031	0.00223591865711031\\
6.11011011011011	0.00222030632116454\\
6.11711711711712	0.00220480299859137\\
6.12412412412412	0.00218940792820328\\
6.13113113113113	0.00217412035412774\\
6.13813813813814	0.00215893952577013\\
6.14514514514515	0.00214386469777683\\
6.15215215215215	0.00212889512999871\\
6.15915915915916	0.00211403008745471\\
6.16616616616617	0.00209926884029581\\
6.17317317317317	0.00208461066376914\\
6.18018018018018	0.00207005483818246\\
6.18718718718719	0.00205560064886877\\
6.19419419419419	0.00204124738615126\\
6.2012012012012	0.00202699434530843\\
6.20820820820821	0.00201284082653953\\
6.21521521521522	0.00199878613493017\\
6.22222222222222	0.0019848295804182\\
6.22922922922923	0.00197097047775986\\
6.23623623623624	0.00195720814649609\\
6.24324324324324	0.00194354191091917\\
6.25025025025025	0.00192997110003951\\
6.25725725725726	0.0019164950475527\\
6.26426426426426	0.00190311309180683\\
6.27127127127127	0.00188982457576999\\
6.27827827827828	0.00187662884699798\\
6.28528528528529	0.00186352525760232\\
6.29229229229229	0.00185051316421842\\
6.2992992992993	0.00183759192797399\\
6.30630630630631	0.00182476091445767\\
6.31331331331331	0.0018120194936879\\
6.32032032032032	0.00179936704008196\\
6.32732732732733	0.00178680293242528\\
6.33433433433433	0.00177432655384093\\
6.34134134134134	0.00176193729175933\\
6.34834834834835	0.00174963453788818\\
6.35535535535536	0.0017374176881826\\
6.36236236236236	0.00172528614281542\\
6.36936936936937	0.00171323930614784\\
6.37637637637638	0.00170127658670005\\
6.38338338338338	0.0016893973971223\\
6.39039039039039	0.001677601154166\\
6.3973973973974	0.00166588727865512\\
6.4044044044044	0.0016542551954577\\
6.41141141141141	0.00164270433345769\\
6.41841841841842	0.00163123412552684\\
6.42542542542543	0.00161984400849689\\
6.43243243243243	0.00160853342313191\\
6.43943943943944	0.00159730181410084\\
6.44644644644645	0.00158614862995023\\
6.45345345345345	0.00157507332307716\\
6.46046046046046	0.00156407534970237\\
6.46746746746747	0.00155315416984352\\
6.47447447447447	0.00154230924728873\\
6.48148148148148	0.00153154004957021\\
6.48848848848849	0.00152084604793815\\
6.4954954954955	0.00151022671733472\\
6.5025025025025	0.00149968153636835\\
6.50950950950951	0.00148920998728807\\
6.51651651651652	0.0014788115559581\\
6.52352352352352	0.00146848573183267\\
6.53053053053053	0.00145823200793084\\
6.53753753753754	0.00144804988081173\\
6.54454454454454	0.0014379388505497\\
6.55155155155155	0.00142789842070988\\
6.55855855855856	0.00141792809832375\\
6.56556556556557	0.00140802739386494\\
6.57257257257257	0.00139819582122522\\
6.57957957957958	0.00138843289769063\\
6.58658658658659	0.00137873814391773\\
6.59359359359359	0.00136911108391014\\
6.6006006006006	0.00135955124499511\\
6.60760760760761	0.00135005815780034\\
6.61461461461461	0.00134063135623093\\
6.62162162162162	0.0013312703774465\\
6.62862862862863	0.00132197476183846\\
6.63563563563564	0.00131274405300743\\
6.64264264264264	0.00130357779774085\\
6.64964964964965	0.00129447554599074\\
6.65665665665666	0.00128543685085158\\
6.66366366366366	0.00127646126853835\\
6.67067067067067	0.0012675483583648\\
6.67767767767768	0.00125869768272176\\
6.68468468468468	0.00124990880705568\\
6.69169169169169	0.00124118129984727\\
6.6986986986987	0.00123251473259036\\
6.70570570570571	0.00122390867977081\\
6.71271271271271	0.00121536271884564\\
6.71971971971972	0.00120687643022228\\
6.72672672672673	0.00119844939723798\\
6.73373373373373	0.00119008120613936\\
6.74074074074074	0.00118177144606203\\
6.74774774774775	0.00117351970901052\\
6.75475475475475	0.00116532558983816\\
6.76176176176176	0.00115718868622724\\
6.76876876876877	0.00114910859866924\\
6.77577577577578	0.00114108493044519\\
6.78278278278278	0.00113311728760626\\
6.78978978978979	0.00112520527895433\\
6.7967967967968	0.00111734851602284\\
6.8038038038038	0.00110954661305773\\
6.81081081081081	0.00110179918699844\\
6.81781781781782	0.00109410585745915\\
6.82482482482482	0.00108646624671008\\
6.83183183183183	0.00107887997965898\\
6.83883883883884	0.00107134668383266\\
6.84584584584585	0.00106386598935874\\
6.85285285285285	0.00105643752894747\\
6.85985985985986	0.00104906093787373\\
6.86686686686687	0.00104173585395908\\
6.87387387387387	0.00103446191755399\\
6.88088088088088	0.00102723877152021\\
6.88788788788789	0.0010200660612132\\
6.89489489489489	0.00101294343446472\\
6.9019019019019	0.00100587054156558\\
6.90890890890891	0.000998847035248419\\
6.91591591591592	0.000991872570670673\\
6.92292292292292	0.000984946805397654\\
6.92992992992993	0.000978069399385729\\
6.93693693693694	0.000971240014965624\\
6.94394394394394	0.000964458316825845\\
6.95095095095095	0.000957723971996217\\
6.95795795795796	0.000951036649831532\\
6.96496496496496	0.000944396021995319\\
6.97197197197197	0.000937801762443722\\
6.97897897897898	0.00093125354740949\\
6.98598598598599	0.000924751055386081\\
6.99299299299299	0.000918293967111879\\
7	0.000911881965554516\\
};
\addlegendentry{$\mathcal{W}(1,0; 1,0)$}

\addplot [color=black, dashdotted, line width=1.5pt]
  table[row sep=crcr]{%
0	0\\
0.00700700700700701	9.00921866260746e-06\\
0.014014014014014	5.46216772565062e-05\\
0.021021021021021	0.000156747997097119\\
0.028028028028028	0.000331163141010128\\
0.035035035035035	0.000591570687421838\\
0.042042042042042	0.000950332466438428\\
0.049049049049049	0.00141884071932785\\
0.0560560560560561	0.00200774024241462\\
0.0630630630630631	0.00272707393233057\\
0.0700700700700701	0.00358638423959907\\
0.0770770770770771	0.00459478705207272\\
0.0840840840840841	0.00576102724707718\\
0.0910910910910911	0.00709352145758803\\
0.0980980980980981	0.00860039156840421\\
0.105105105105105	0.0102894912714949\\
0.112112112112112	0.012168427279963\\
0.119119119119119	0.0142445763324699\\
0.126126126126126	0.0165250988098013\\
0.133133133133133	0.0190169495733888\\
0.14014014014014	0.0217268864871755\\
0.147147147147147	0.0246614769779439\\
0.154154154154154	0.0278271029116587\\
0.161161161161161	0.031229964005816\\
0.168168168168168	0.0348760799544432\\
0.175175175175175	0.0387712914093245\\
0.182182182182182	0.0429212599355234\\
0.189189189189189	0.0473314670394021\\
0.196196196196196	0.0520072123517429\\
0.203203203203203	0.0569536110362427\\
0.21021021021021	0.062175590483853\\
0.217217217217217	0.0676778863456272\\
0.224224224224224	0.0734650379504919\\
0.231231231231231	0.079541383149375\\
0.238238238238238	0.0859110526231514\\
0.245245245245245	0.0925779636887207\\
0.252252252252252	0.0995458136350622\\
0.259259259259259	0.106818072619208\\
0.266266266266266	0.11439797615063\\
0.273273273273273	0.122288517191501\\
0.28028028028028	0.130492437899554\\
0.287287287287287	0.139012221039874\\
0.294294294294294	0.147850081091708\\
0.301301301301301	0.157007955076462\\
0.308308308308308	0.166487493133203\\
0.315315315315315	0.176290048868355\\
0.322322322322322	0.186416669506774\\
0.329329329329329	0.196868085871977\\
0.336336336336336	0.207644702223994\\
0.343343343343343	0.218746585984127\\
0.35035035035035	0.230173457376746\\
0.357357357357357	0.24192467901915\\
0.364364364364364	0.253999245491548\\
0.371371371371371	0.266395772920162\\
0.378378378378378	0.279112488607545\\
0.385385385385385	0.292147220745232\\
0.392392392392392	0.305497388244935\\
0.399399399399399	0.319159990725546\\
0.406406406406406	0.333131598694293\\
0.413413413413413	0.347408343961425\\
0.42042042042042	0.361985910328834\\
0.427427427427427	0.376859524594003\\
0.434434434434434	0.39202394791161\\
0.441441441441441	0.407473467556006\\
0.448448448448448	0.423201889128619\\
0.455455455455455	0.439202529255076\\
0.462462462462462	0.455468208817509\\
0.469469469469469	0.471991246768108\\
0.476476476476476	0.488763454570438\\
0.483483483483483	0.505776131315426\\
0.49049049049049	0.523020059559159\\
0.497497497497497	0.54048550192977\\
0.504504504504504	0.558162198550636\\
0.511511511511512	0.576039365326997\\
0.518518518518518	0.594105693142704\\
0.525525525525526	0.612349348013387\\
0.532532532532533	0.630757972241589\\
0.53953953953954	0.64931868661862\\
0.546546546546547	0.668018093716783\\
0.553553553553554	0.68684228231441\\
0.560560560560561	0.705776832994659\\
0.567567567567568	0.724806824957367\\
0.574574574574575	0.743916844081357\\
0.581581581581582	0.763090992272449\\
0.588588588588589	0.782312898130112\\
0.595595595595596	0.801565728963051\\
0.602602602602603	0.82083220418125\\
0.60960960960961	0.840094610088862\\
0.616616616616617	0.859334816099081\\
0.623623623623624	0.87853429238852\\
0.630630630630631	0.897674129004875\\
0.637637637637638	0.916735056437586\\
0.644644644644645	0.935697467656947\\
0.651651651651652	0.954541441622631\\
0.658658658658659	0.973246768257866\\
0.665665665665666	0.991792974880559\\
0.672672672672673	1.01015935407755\\
0.67967967967968	1.02832499300283\\
0.686686686686687	1.04626880407501\\
0.693693693693694	1.06396955704382\\
0.700700700700701	1.08140591238924\\
0.707707707707708	1.09855645601139\\
0.714714714714715	1.11539973516282\\
0.721721721721722	1.13191429556883\\
0.728728728728729	1.14807871967541\\
0.735735735735736	1.16387166595777\\
0.742742742742743	1.17927190921644\\
0.74974974974975	1.19425838178151\\
0.756756756756757	1.20881021553939\\
0.763763763763764	1.22290678469032\\
0.770770770770771	1.23652774913871\\
0.777777777777778	1.24965309841266\\
0.784784784784785	1.26226319600306\\
0.791791791791792	1.27433882400718\\
0.798798798798799	1.28586122795649\\
0.805805805805806	1.29681216170316\\
0.812812812812813	1.3071739322352\\
0.81981981981982	1.31692944428572\\
0.826826826826827	1.32606224459776\\
0.833833833833834	1.33455656570271\\
0.840840840840841	1.34239736906703\\
0.847847847847848	1.34957038745954\\
0.854854854854855	1.35606216638939\\
0.861861861861862	1.36186010446317\\
0.868868868868869	1.36695249250874\\
0.875875875875876	1.37132855131319\\
0.882882882882883	1.37497846782226\\
0.88988988988989	1.37789342964982\\
0.896896896896897	1.38006565774758\\
0.903903903903904	1.38148843708756\\
0.910910910910911	1.38215614521282\\
0.917917917917918	1.382064278516\\
0.924924924924925	1.38120947610954\\
0.931931931931932	1.37958954115697\\
0.938938938938939	1.37720345954053\\
0.945945945945946	1.37405141574723\\
0.952952952952953	1.37013480586324\\
0.95995995995996	1.36545624757422\\
0.966966966966967	1.36001958707886\\
0.973973973973974	1.35382990283186\\
0.980980980980981	1.34689350604349\\
0.987987987987988	1.33921793787326\\
0.994994994994995	1.33081196326723\\
1.002002002002	1.32168556140023\\
1.00900900900901	1.31184991269682\\
1.01601601601602	1.30131738241816\\
1.02302302302302	1.29010150081483\\
1.03003003003003	1.2782169398599\\
1.03703703703704	1.26567948659011\\
1.04404404404404	1.25250601309754\\
1.05105105105105	1.23871444322829\\
1.05805805805806	1.22432371605899\\
1.06506506506507	1.20935374623671\\
1.07207207207207	1.19382538128148\\
1.07907907907908	1.17776035596534\\
1.08608608608609	1.16118124389516\\
1.09309309309309	1.14411140644017\\
1.1001001001001	1.12657493915813\\
1.10710710710711	1.10859661588655\\
1.11411411411411	1.09020183067735\\
1.12112112112112	1.07141653776458\\
1.12812812812813	1.05226718976526\\
1.13513513513514	1.03278067432324\\
1.14214214214214	1.01298424941463\\
1.14914914914915	0.992905477541185\\
1.15615615615616	0.972572159044953\\
1.16316316316316	0.952012264783057\\
1.17017017017017	0.931253868406201\\
1.17717717717718	0.910325078487834\\
1.18418418418418	0.889253970753147\\
1.19119119119119	0.868068520658029\\
1.1981981981982	0.846796536567784\\
1.20520520520521	0.825465593783865\\
1.21221221221221	0.804102969663953\\
1.21921921921922	0.782735580076591\\
1.22622622622623	0.761389917426122\\
1.23323323323323	0.740091990476986\\
1.24024024024024	0.718867266198515\\
1.24724724724725	0.697740613842244\\
1.25425425425425	0.676736251453496\\
1.26126126126126	0.655877695007621\\
1.26826826826827	0.635187710348924\\
1.27527527527528	0.614688268096906\\
1.28228228228228	0.594400501670255\\
1.28928928928929	0.574344668563939\\
1.2962962962963	0.554540114999016\\
1.3033033033033	0.535005244048403\\
1.31031031031031	0.515757487324934\\
1.31731731731732	0.496813280300723\\
1.32432432432432	0.478188041309265\\
1.33133133133133	0.459896154263832\\
1.33833833833834	0.441950955107874\\
1.34534534534535	0.424364721995235\\
1.35235235235235	0.407148669180243\\
1.35935935935936	0.390312944580288\\
1.36636636636637	0.373866630956345\\
1.37337337337337	0.357817750640259\\
1.38038038038038	0.342173273721527\\
1.38738738738739	0.3269391295909\\
1.39439439439439	0.312120221723479\\
1.4014014014014	0.297720445570195\\
1.40840840840841	0.2837427094137\\
1.41541541541542	0.270188958032841\\
1.42242242242242	0.257060199009145\\
1.42942942942943	0.244356531499041\\
1.43643643643644	0.232077177287161\\
1.44344344344344	0.220220513928753\\
1.45045045045045	0.208784109783285\\
1.45745745745746	0.197764760736592\\
1.46446446446446	0.187158528405464\\
1.47147147147147	0.176960779616439\\
1.47847847847848	0.167166226949645\\
1.48548548548549	0.157768970138937\\
1.49249249249249	0.148762538121144\\
1.4994994994995	0.140139931530024\\
1.50650650650651	0.131893665434476\\
1.51351351351351	0.124015812125543\\
1.52052052052052	0.116498043762807\\
1.52752752752753	0.109331674697795\\
1.53453453453453	0.1025077032999\\
1.54154154154154	0.0960168531190563\\
1.54854854854855	0.089849613228835\\
1.55555555555556	0.0839962776037344\\
1.56256256256256	0.0784469833950637\\
1.56956956956957	0.0731917479809354\\
1.57657657657658	0.0682205046773503\\
1.58358358358358	0.0635231370091133\\
1.59059059059059	0.0590895114512532\\
1.5975975975976	0.0549095085636531\\
1.6046046046046	0.0509730524536306\\
1.61161161161161	0.0472701385131656\\
1.61861861861862	0.0437908593892639\\
1.62562562562563	0.0405254291574955\\
1.63263263263263	0.0374642056799855\\
1.63963963963964	0.0345977111399854\\
1.64664664664665	0.0319166507555614\\
1.65365365365365	0.0294119296848423\\
1.66066066066066	0.0270746681446208\\
1.66766766766767	0.0248962147728615\\
1.67467467467467	0.0228681582737903\\
1.68168168168168	0.0209823373917038\\
1.68868868868869	0.019230849266411\\
1.6956956956957	0.0176060562292934\\
1.7027027027027	0.016100591104331\\
1.70970970970971	0.0147073610830831\\
1.71671671671672	0.0134195502465498\\
1.72372372372372	0.0122306208100618\\
1.73073073073073	0.0111343131698869\\
1.73773773773774	0.0101246448321064\\
1.74474474474474	0.0091959083055304\\
1.75175175175175	0.00834266804102626\\
1.75875875875876	0.00755975649963915\\
1.76576576576577	0.00684226943135043\\
1.77277277277277	0.00618556044526085\\
1.77977977977978	0.00558523495045882\\
1.78678678678679	0.00503714354487297\\
1.79379379379379	0.0045373749270568\\
1.8008008008008	0.00408224840315826\\
1.80780780780781	0.0036683060583279\\
1.81481481481481	0.00329230465856373\\
1.82182182182182	0.00295120734552027\\
1.82882882882883	0.00264217518316515\\
1.83583583583584	0.00236255861139236\\
1.84284284284284	0.00210988885783308\\
1.84984984984985	0.00188186935518255\\
1.85685685685686	0.00167636720741859\\
1.86386386386386	0.00149140474435771\\
1.87087087087087	0.00132515120010757\\
1.87787787787788	0.00117591454715846\\
1.88488488488488	0.00104213351413583\\
1.89189189189189	0.000922369811632611\\
1.8988988988989	0.000815300587074339\\
1.90590590590591	0.000719711126257076\\
1.91291291291291	0.000634487816052944\\
1.91991991991992	0.000558611379811131\\
1.92692692692693	0.000491150394202258\\
1.93393393393393	0.000431255093667208\\
1.94094094094094	0.000378151466241765\\
1.94794794794795	0.000331135642337336\\
1.95495495495495	0.000289568576065338\\
1.96196196196196	0.000252871016896624\\
1.96896896896897	0.000220518767843531\\
1.97597597597598	0.000192038224935913\\
1.98298298298298	0.000167002191527281\\
1.98998998998999	0.000145025959905383\\
1.996996996997	0.000125763651784853\\
2.004004004004	0.000108904808518776\\
2.01101101101101	9.41712212713677e-05\\
2.01801801801802	8.13139909354451e-05\\
2.02502502502503	7.01108072454176e-05\\
2.03203203203203	6.03634363188116e-05\\
2.03903903903904	5.18954057461357e-05\\
2.04604604604605	4.45498763299636e-05\\
2.05305305305305	3.81876896391257e-05\\
2.06006006006006	3.26855806829715e-05\\
2.06706706706707	2.79345452143034e-05\\
2.07407407407407	2.38383514285856e-05\\
2.08108108108108	2.03121861328495e-05\\
2.08808808808809	1.72814258022056e-05\\
2.0950950950951	1.46805233174708e-05\\
2.1021021021021	1.24520015771917e-05\\
2.10910910910911	1.05455455948369e-05\\
2.11611611611612	8.91718512137911e-06\\
2.12312312312312	7.52856026963281e-06\\
2.13013013013013	6.3462630548957e-06\\
2.13713713713714	5.34124820257118e-06\\
2.14414414414414	4.48830700391479e-06\\
2.15115115115115	3.76559842280588e-06\\
2.15815815815816	3.15423206688073e-06\\
2.16516516516517	2.6378980333298e-06\\
2.17217217217217	2.20253902142684e-06\\
2.17917917917918	1.83606046905661e-06\\
2.18618618618619	1.52807481813357e-06\\
2.19319319319319	1.26967634315919e-06\\
2.2002002002002	1.05324328782385e-06\\
2.20720720720721	8.72264346353927e-07\\
2.21421421421421	7.21186799271319e-07\\
2.22122122122122	5.95283867599807e-07\\
2.22822822822823	4.90539085704631e-07\\
2.23523523523524	4.03545711400394e-07\\
2.24224224224224	3.31419393331868e-07\\
2.24924924924925	2.71722500626245e-07\\
2.25625625625626	2.22398689204343e-07\\
2.26326326326326	1.81716433739106e-07\\
2.27027027027027	1.48220394907465e-07\\
2.27727727727728	1.20689619159559e-07\\
2.28428428428428	9.81016835926914e-08\\
2.29129129129129	7.96020025251742e-08\\
2.2982982982983	6.44776058610117e-08\\
2.30530530530531	5.21347831397238e-08\\
2.31231231231231	4.20800620681674e-08\\
2.31931931931932	3.39040570901066e-08\\
2.32632632632633	2.72677828849884e-08\\
2.33333333333333	2.18910802807025e-08\\
2.34034034034034	1.7542848555495e-08\\
2.34734734734735	1.4032819089458e-08\\
2.35435435435435	1.12046413616043e-08\\
2.36136136136136	8.9300838884735e-09\\
2.36836836836837	7.10418033776794e-09\\
2.37537537537538	5.64117515950034e-09\\
2.38238238238238	4.47114405362235e-09\\
2.38938938938939	3.53718279708151e-09\\
2.3963963963964	2.79307371072779e-09\\
2.4034034034034	2.20135265150049e-09\\
2.41041041041041	1.73171113209381e-09\\
2.41741741741742	1.35967823540832e-09\\
2.42442442442442	1.06553561561632e-09\\
2.43143143143143	8.3342624924922e-10\\
2.43843843843844	6.50623884632217e-10\\
2.44544544544545	5.06935483071365e-10\\
2.45245245245245	3.94213479679521e-10\\
2.45945945945946	3.05958528998696e-10\\
2.46646646646647	2.36996639795561e-10\\
2.47347347347347	1.83217331026989e-10\\
2.48048048048048	1.41361732125758e-10\\
2.48748748748749	1.08851470543057e-10\\
2.49449449449449	8.36507941383484e-11\\
2.5015015015015	6.41557139586848e-11\\
2.50850850850851	4.91050657851067e-11\\
2.51551551551552	3.75093121465922e-11\\
2.52252252252252	2.8593670705141e-11\\
2.52952952952953	2.17527858684036e-11\\
2.53653653653654	1.6514680144593e-11\\
2.54354354354354	1.25121486994673e-11\\
2.55055055055055	9.46011048713517e-12\\
2.55755755755756	7.137715400294e-12\\
2.56456456456456	5.37424019354142e-12\\
2.57157157157157	4.0379957405944e-12\\
2.57857857857858	3.02762222382386e-12\\
2.58558558558559	2.26527355572917e-12\\
2.59259259259259	1.69129303534522e-12\\
2.5995995995996	1.26006336587234e-12\\
2.60660660660661	9.36779348814324e-13\\
2.61361361361361	6.94943825057721e-13\\
2.62062062062062	5.14429217183107e-13\\
2.62762762762763	3.79980355464968e-13\\
2.63463463463463	2.80060787740946e-13\\
2.64164164164164	2.05965819909107e-13\\
2.64864864864865	1.51142196521671e-13\\
2.65565565565566	1.10667490065687e-13\\
2.66266266266266	8.08526338187032e-14\\
2.66966966966967	5.89391789739771e-14\\
2.67667667667668	4.28692416615734e-14\\
2.68368368368368	3.11110976495788e-14\\
2.69069069069069	2.25272759024496e-14\\
2.6976976976977	1.62750309966888e-14\\
2.7047047047047	1.17314246943165e-14\\
2.71171171171171	8.43706614884823e-15\\
2.71871871871872	6.05396472766061e-15\\
2.72572572572573	4.33403104770768e-15\\
2.73273273273273	3.09559265504356e-15\\
2.73973973973974	2.20592735198109e-15\\
2.74674674674675	1.56830366455821e-15\\
2.75375375375375	1.11238878148647e-15\\
2.76076076076076	7.87166252031862e-16\\
2.76776776776777	5.55719556598473e-16\\
2.77477477477477	3.91399404911317e-16\\
2.78178178178178	2.75014643993809e-16\\
2.78878878878879	1.92778484658609e-16\\
2.7957957957958	1.34810676167941e-16\\
2.8028028028028	9.40478532132819e-17\\
2.80980980980981	6.54528025266226e-17\\
2.81681681681682	4.54420846860573e-17\\
2.82382382382382	3.14727548868306e-17\\
2.83083083083083	2.17447115608104e-17\\
2.83783783783784	1.49868645407134e-17\\
2.84484484484484	1.03039087753061e-17\\
2.85185185185185	7.0668059740849e-18\\
2.85885885885886	4.83470644778104e-18\\
2.86586586586587	3.2994280802719e-18\\
2.87287287287287	2.24607712189873e-18\\
2.87987987987988	1.52518991592517e-18\\
2.88688688688689	1.03307598010911e-18\\
2.89389389389389	6.97984034745485e-19\\
2.9009009009009	4.70391431613367e-19\\
2.90790790790791	3.16205776618323e-19\\
2.91491491491491	2.12017871508007e-19\\
2.92192192192192	1.41795744458874e-19\\
2.92892892892893	9.45883733275363e-20\\
2.93593593593594	6.29349540684984e-20\\
2.94294294294294	4.176586201156e-20\\
2.94994994994995	2.76453506941185e-20\\
2.95695695695696	1.82511247465997e-20\\
2.96396396396396	1.20176564365235e-20\\
2.97097097097097	7.89238557294338e-21\\
2.97797797797798	5.16952804168657e-21\\
2.98498498498498	3.37709528492999e-21\\
2.99199199199199	2.20029681804604e-21\\
2.998998998999	1.42975122613484e-21\\
3.00600600600601	9.26566602936029e-22\\
3.01301301301301	5.98860220768541e-22\\
3.02002002002002	3.86013632971803e-22\\
3.02702702702703	2.48144047549169e-22\\
3.03403403403403	1.59083378109535e-22\\
3.04104104104104	1.01709412840405e-22\\
3.04804804804805	6.48497877967734e-23\\
3.05505505505506	4.12346872039589e-23\\
3.06206206206206	2.61468418396696e-23\\
3.06906906906907	1.65338432748265e-23\\
3.07607607607608	1.04261046439798e-23\\
3.08308308308308	6.55631282657055e-24\\
3.09009009009009	4.11132881661142e-24\\
3.0970970970971	2.57090092705485e-24\\
3.1041041041041	1.60311498334075e-24\\
3.11111111111111	9.96817888955656e-25\\
3.11811811811812	6.18065352790007e-25\\
3.12512512512513	3.8213428259992e-25\\
3.13213213213213	2.35589660811227e-25\\
3.13913913913914	1.44827376735878e-25\\
3.14614614614615	8.87758618392616e-26\\
3.15315315315315	5.42605888319865e-26\\
3.16016016016016	3.30685422984881e-26\\
3.16716716716717	2.00947268996757e-26\\
3.17417417417417	1.2175343776041e-26\\
3.18118118118118	7.35542847650361e-27\\
3.18818818818819	4.43055250502522e-27\\
3.1951951951952	2.66088745200845e-27\\
3.2022022022022	1.59334332388413e-27\\
3.20920920920921	9.51266043689213e-28\\
3.21621621621622	5.66239179907526e-28\\
3.22322322322322	3.36045935340691e-28\\
3.23023023023023	1.98835394847273e-28\\
3.23723723723724	1.1729528867415e-28\\
3.24424424424424	6.89850148401196e-29\\
3.25125125125125	4.04493706707208e-29\\
3.25825825825826	2.36454223412654e-29\\
3.26526526526527	1.37802203340063e-29\\
3.27227227227227	8.00634870750196e-30\\
3.27927927927928	4.63743209001815e-30\\
3.28628628628629	2.67781646950767e-30\\
3.29329329329329	1.54148638642775e-30\\
3.3003003003003	8.84605568781529e-31\\
3.30730730730731	5.06064856129971e-31\\
3.31431431431431	2.88605538076582e-31\\
3.32132132132132	1.64074295077156e-31\\
3.32832832832833	9.29842225668705e-32\\
3.33533533533534	5.25298587080811e-32\\
3.34234234234234	2.95819546126253e-32\\
3.34934934934935	1.6606058897028e-32\\
3.35635635635636	9.29224726241985e-33\\
3.36336336336336	5.18304302715396e-33\\
3.37037037037037	2.88173561553758e-33\\
3.37737737737738	1.59707055197949e-33\\
3.38438438438438	8.82246809698148e-34\\
3.39139139139139	4.85788792294529e-34\\
3.3983983983984	2.66619341743186e-34\\
3.40540540540541	1.4585386447748e-34\\
3.41241241241241	7.95283023604709e-35\\
3.41941941941942	4.32213487852079e-35\\
3.42642642642643	2.34122468606904e-35\\
3.43343343343343	1.26401255289867e-35\\
3.44044044044044	6.8017164631299e-36\\
3.44744744744745	3.64787402029008e-36\\
3.45445445445445	1.94989232390299e-36\\
3.46146146146146	1.03878634548938e-36\\
3.46846846846847	5.51546265012411e-37\\
3.47547547547548	2.91858992468538e-37\\
3.48248248248248	1.53919952884405e-37\\
3.48948948948949	8.08989280951546e-38\\
3.4964964964965	4.23752136062434e-38\\
3.5035035035035	2.21206355365107e-38\\
3.51051051051051	1.15078751153442e-38\\
3.51751751751752	5.96622720018652e-39\\
3.52452452452452	3.08252639813972e-39\\
3.53153153153153	1.58712614694334e-39\\
3.53853853853854	8.14345992872066e-40\\
3.54554554554555	4.16384544257519e-40\\
3.55255255255255	2.12160043469028e-40\\
3.55955955955956	1.07723680167816e-40\\
3.56656656656657	5.45045283577018e-41\\
3.57357357357357	2.7480398694607e-41\\
3.58058058058058	1.38063133979586e-41\\
3.58758758758759	6.91181157437151e-42\\
3.59459459459459	3.44794811063222e-42\\
3.6016016016016	1.71387608711956e-42\\
3.60860860860861	8.48873891956932e-43\\
3.61561561561562	4.18935515931383e-43\\
3.62262262262262	2.06009197541229e-43\\
3.62962962962963	1.00938452775243e-43\\
3.63663663663664	4.92779291165337e-44\\
3.64364364364364	2.39700637804838e-44\\
3.65065065065065	1.16172149477799e-44\\
3.65765765765766	5.60978358187885e-45\\
3.66466466466466	2.69896083631219e-45\\
3.67167167167167	1.2937447278637e-45\\
3.67867867867868	6.17870233869968e-46\\
3.68568568568569	2.93993517005097e-46\\
3.69269269269269	1.39368670734356e-46\\
3.6996996996997	6.58225350406004e-47\\
3.70670670670671	3.09714262971433e-47\\
3.71371371371371	1.45184438634321e-47\\
3.72072072072072	6.78025920485025e-48\\
3.72772772772773	3.15453233622871e-48\\
3.73473473473473	1.4621139464316e-48\\
3.74174174174174	6.75118625473801e-49\\
3.74874874874875	3.10546523967815e-49\\
3.75575575575576	1.42303650513622e-49\\
3.76276276276276	6.49595882560099e-50\\
3.76976976976977	2.9539517775955e-50\\
3.77677677677678	1.33810905928983e-50\\
3.78378378378378	6.03813256036985e-51\\
3.79079079079079	2.71413748540469e-51\\
3.7977977977978	1.21527384274819e-51\\
3.8048048048048	5.42031590461682e-52\\
3.81181181181181	2.40812050875804e-52\\
3.81881881881882	1.06568787922172e-52\\
3.82582582582583	4.69758930889229e-53\\
3.83283283283283	2.06256804774209e-53\\
3.83983983983984	9.02037832121913e-54\\
3.84684684684685	3.92933852957083e-54\\
3.85385385385385	1.70485466830836e-54\\
3.86086086086086	7.36755676886567e-55\\
3.86786786786787	3.17119401942489e-55\\
3.87487487487487	1.35950343724196e-55\\
3.88188188188188	5.80484933460244e-56\\
3.88888888888889	2.46859421653602e-56\\
3.8958958958959	1.04556620355057e-56\\
3.9029029029029	4.41053626191278e-57\\
3.90990990990991	1.85295229453823e-57\\
3.91691691691692	7.7529146946215e-58\\
3.92392392392392	3.23064053597086e-58\\
3.93093093093093	1.34069524741361e-58\\
3.93793793793794	5.54094997667701e-59\\
3.94494494494494	2.28058382349576e-59\\
3.95195195195195	9.34782164463748e-60\\
3.95895895895896	3.81568289643266e-60\\
3.96596596596597	1.55105257614912e-60\\
3.97297297297297	6.27867567977436e-61\\
3.97997997997998	2.53099777911156e-61\\
3.98698698698699	1.01599729121752e-61\\
3.99399399399399	4.06130207377441e-62\\
4.001001001001	1.61660855045843e-62\\
4.00800800800801	6.40775885454737e-63\\
4.01501501501502	2.52908848749199e-63\\
4.02202202202202	9.93970121441724e-64\\
4.02902902902903	3.8898148671401e-64\\
4.03603603603604	1.51574338878481e-64\\
4.04304304304304	5.88109816118622e-65\\
4.05005005005005	2.2720715803938e-65\\
4.05705705705706	8.73999655816059e-66\\
4.06406406406406	3.34750330568566e-66\\
4.07107107107107	1.27657428762638e-66\\
4.07807807807808	4.84709258394443e-67\\
4.08508508508509	1.8324051157139e-67\\
4.09209209209209	6.89701931071474e-68\\
4.0990990990991	2.58461557541236e-68\\
4.10610610610611	9.6431703429377e-69\\
4.11311311311311	3.58201894027355e-69\\
4.12012012012012	1.32469184252268e-69\\
4.12712712712713	4.87725622509627e-70\\
4.13413413413413	1.78774175177815e-70\\
4.14114114114114	6.52374934243025e-71\\
4.14814814814815	2.3699969438492e-71\\
4.15515515515516	8.57138696442666e-72\\
4.16216216216216	3.08604297116001e-72\\
4.16916916916917	1.10610189206069e-72\\
4.17617617617618	3.94662028904455e-73\\
4.18318318318318	1.40180412649427e-73\\
4.19019019019019	4.95650829762013e-74\\
4.1971971971972	1.74455868010014e-74\\
4.2042042042042	6.1123932768745e-75\\
4.21121121121121	2.13180598095919e-75\\
4.21821821821822	7.40098380987513e-76\\
4.22522522522523	2.55759237697385e-76\\
4.23223223223223	8.79767675261274e-77\\
4.23923923923924	3.01227186697169e-77\\
4.24624624624625	1.02660769018185e-77\\
4.25325325325325	3.48252087106466e-78\\
4.26026026026026	1.17586236136584e-78\\
4.26726726726727	3.95173188191209e-79\\
4.27427427427427	1.32184746236896e-79\\
4.28128128128128	4.40081148071986e-80\\
4.28828828828829	1.45826481974425e-80\\
4.2952952952953	4.80935566607271e-81\\
4.3023023023023	1.5786247342706e-81\\
4.30930930930931	5.15711793785769e-82\\
4.31631631631632	1.67674111270047e-82\\
4.32332332332332	5.42563262451094e-83\\
4.33033033033033	1.74724900523129e-83\\
4.33733733733734	5.59981768932549e-84\\
4.34434434434434	1.78608555536236e-84\\
4.35135135135135	5.66936636181511e-85\\
4.35835835835836	1.7908750511934e-85\\
4.36536536536537	5.62975051452746e-86\\
4.37237237237237	1.7611676764868e-86\\
4.37937937937938	5.48270079484873e-87\\
4.38638638638639	1.69849883391368e-87\\
4.39339339339339	5.23609365741678e-88\\
4.4004004004004	1.6062590173006e-88\\
4.40740740740741	4.90325341203613e-89\\
4.41441441441441	1.48938937243114e-89\\
4.42142142142142	4.50175515662476e-90\\
4.42842842842843	1.35394103868571e-90\\
4.43543543543544	4.05187876254449e-91\\
4.44244244244244	1.20655322024426e-91\\
4.44944944944945	3.5749042291983e-92\\
4.45645645645646	1.05391299511176e-92\\
4.46346346346346	3.09144876474898e-93\\
4.47047047047047	9.02258150576686e-94\\
4.47747747747748	2.62002573847261e-94\\
4.48448448448448	7.56973707988901e-95\\
4.49149149149149	2.17596062585029e-95\\
4.4984984984985	6.22315716993841e-96\\
4.50550550550551	1.77073879211533e-96\\
4.51251251251251	5.01275776974435e-97\\
4.51951951951952	1.41179586624472e-97\\
4.52652652652653	3.95580197201228e-98\\
4.53353353353353	1.10270438940466e-98\\
4.54054054054054	3.05801813997626e-99\\
4.54754754754755	8.43668566913119e-100\\
4.55455455455455	2.31552245117274e-100\\
4.56156156156156	6.32216598603273e-101\\
4.56856856856857	1.71718469938141e-101\\
4.57557557557558	4.6397735886041e-102\\
4.58258258258258	1.24709537020208e-102\\
4.58958958958959	3.33441828485983e-103\\
4.5965965965966	8.86854659260915e-104\\
4.6036036036036	2.34634143093298e-104\\
4.61061061061061	6.17491193716439e-105\\
4.61761761761762	1.61646215624127e-105\\
4.62462462462462	4.20910671243151e-106\\
4.63163163163163	1.09018036333008e-106\\
4.63863863863864	2.80856962556999e-107\\
4.64564564564565	7.19688970644428e-108\\
4.65265265265265	1.8343056307244e-108\\
4.65965965965966	4.65007687499714e-109\\
4.66666666666667	1.1724775185292e-109\\
4.67367367367367	2.94035146241244e-110\\
4.68068068068068	7.33396210734132e-111\\
4.68768768768769	1.81935368796286e-111\\
4.69469469469469	4.48878725211679e-112\\
4.7017017017017	1.10145985595433e-112\\
4.70870870870871	2.68800724260765e-113\\
4.71571571571572	6.52392029782119e-114\\
4.72272272272272	1.57469933992966e-114\\
4.72972972972973	3.77999966666185e-115\\
4.73673673673674	9.02371396444199e-116\\
4.74374374374374	2.14226228886941e-116\\
4.75075075075075	5.05764152022734e-117\\
4.75775775775776	1.1874240308585e-117\\
4.76476476476476	2.77230059477723e-118\\
4.77177177177177	6.43644106254576e-119\\
4.77877877877878	1.48599232055606e-119\\
4.78578578578579	3.41151207476411e-120\\
4.79279279279279	7.78809292913097e-121\\
4.7997997997998	1.76792355237959e-121\\
4.80680680680681	3.99060054331849e-122\\
4.81381381381381	8.95673493965691e-123\\
4.82082082082082	1.99890493728143e-123\\
4.82782782782783	4.43567555760846e-124\\
4.83483483483483	9.78693880301579e-125\\
4.84184184184184	2.14707734977849e-125\\
4.84884884884885	4.68334844023194e-126\\
4.85585585585586	1.01570467677364e-126\\
4.86286286286286	2.19015509412572e-127\\
4.86986986986987	4.69540387349306e-128\\
4.87687687687688	1.0008198899952e-128\\
4.88388388388388	2.12088935005329e-129\\
4.89089089089089	4.46841415423117e-130\\
4.8978978978979	9.35957997778395e-131\\
4.9049049049049	1.94904128401867e-131\\
4.91191191191191	4.03498248632773e-132\\
4.91891891891892	8.30447919068892e-133\\
4.92592592592593	1.69913336946137e-133\\
4.93293293293293	3.45605815873966e-134\\
4.93993993993994	6.98823176178142e-135\\
4.94694694694695	1.40468975877451e-135\\
4.95395395395395	2.80681998838856e-136\\
4.96096096096096	5.57524424019834e-137\\
4.96796796796797	1.10083590639448e-137\\
4.97497497497497	2.1606517909765e-138\\
4.98198198198198	4.21545670219659e-139\\
4.98898898898899	8.17516043033365e-140\\
4.995995995996	1.57591893898465e-140\\
5.003003003003	3.01961476036738e-141\\
5.01001001001001	5.75100053528907e-142\\
5.01701701701702	1.08868842270257e-142\\
5.02402402402402	2.04845415556027e-143\\
5.03103103103103	3.83094092971402e-144\\
5.03803803803804	7.12090733365278e-145\\
5.04504504504505	1.31555796883576e-145\\
5.05205205205205	2.41559166014545e-146\\
5.05905905905906	4.40828865847574e-147\\
5.06606606606607	7.99546247089878e-148\\
5.07307307307307	1.44124661268872e-148\\
5.08008008008008	2.58195242757122e-149\\
5.08708708708709	4.59692558854001e-150\\
5.09409409409409	8.13373619533479e-151\\
5.1011011011011	1.43024397818651e-151\\
5.10810810810811	2.49931887957532e-152\\
5.11511511511512	4.34029015442483e-153\\
5.12212212212212	7.49023528177422e-154\\
5.12912912912913	1.2845345913859e-154\\
5.13613613613614	2.18909110532201e-155\\
5.14314314314314	3.7071793116147e-156\\
5.15015015015015	6.23848455245973e-157\\
5.15715715715716	1.0431923553518e-157\\
5.16416416416416	1.73337883384812e-158\\
5.17117117117117	2.86193917822151e-159\\
5.17817817817818	4.69525451871806e-160\\
5.18518518518519	7.65391638630837e-161\\
5.19219219219219	1.23973289217716e-161\\
5.1991991991992	1.99519968161689e-162\\
5.20620620620621	3.19045360696683e-163\\
5.21321321321321	5.06897689405031e-164\\
5.22022022022022	8.0017316359694e-165\\
5.22722722722723	1.25498178201982e-165\\
5.23423423423423	1.95557556628388e-166\\
5.24124124124124	3.02753722595386e-167\\
5.24824824824825	4.65667632367888e-168\\
5.25525525525526	7.11587385037728e-169\\
5.26226226226226	1.08028913767393e-169\\
5.26926926926927	1.62931594047767e-170\\
5.27627627627628	2.44128270661751e-171\\
5.28328328328328	3.63389397531407e-172\\
5.29029029029029	5.37355590048973e-173\\
5.2972972972973	7.89369897498266e-174\\
5.3043043043043	1.15192059673e-174\\
5.31131131131131	1.66986606407152e-175\\
5.31831831831832	2.40464937193383e-176\\
5.32532532532533	3.43974949221216e-177\\
5.33233233233233	4.88765686845117e-178\\
5.33933933933934	6.89870129445343e-179\\
5.34634634634635	9.67209613666822e-180\\
5.35335335335335	1.34695904100453e-180\\
5.36036036036036	1.86321332243977e-181\\
5.36736736736737	2.55999465112467e-182\\
5.37437437437437	3.4936363353442e-183\\
5.38138138138138	4.73557113473585e-184\\
5.38838838838839	6.3755393407567e-185\\
5.3953953953954	8.52521326542759e-186\\
5.4024024024024	1.13222095036738e-186\\
5.40940940940941	1.49344277036684e-187\\
5.41641641641642	1.95646143860424e-188\\
5.42342342342342	2.54550029044593e-189\\
5.43043043043043	3.28918269464029e-190\\
5.43743743743744	4.2209446896153e-191\\
5.44444444444444	5.3793768755087e-192\\
5.45145145145145	6.80845780078945e-193\\
5.45845845845846	8.55763607250577e-194\\
5.46546546546547	1.06817147090262e-194\\
5.47247247247247	1.32404894879598e-195\\
5.47947947947948	1.62980938912739e-196\\
5.48648648648649	1.99220073624244e-197\\
5.49349349349349	2.41816915262257e-198\\
5.5005005005005	2.91468311811772e-199\\
5.50750750750751	3.48851734199562e-200\\
5.51451451451451	4.14599790214004e-201\\
5.52152152152152	4.89271247727853e-202\\
5.52852852852853	5.73319226993308e-203\\
5.53553553553554	6.67057509544694e-204\\
5.54254254254254	7.70626196211748e-205\\
5.54954954954955	8.83958208285677e-206\\
5.55655655655656	1.00674834128041e-206\\
5.56356356356356	1.13842672671206e-207\\
5.57057057057057	1.27813861196414e-208\\
5.57757757757758	1.42473231315405e-209\\
5.58458458458458	1.5767570180513e-210\\
5.59159159159159	1.73247181020758e-211\\
5.5985985985986	1.88986685524583e-212\\
5.60560560560561	2.04669714951726e-213\\
5.61261261261261	2.2005286040223e-214\\
5.61961961961962	2.34879555592776e-215\\
5.62662662662663	2.48886810729468e-216\\
5.63363363363363	2.61812703180503e-217\\
5.64064064064064	2.73404341321842e-218\\
5.64764764764765	2.83425972973926e-219\\
5.65465465465465	2.9166688160723e-220\\
5.66166166166166	2.97948704966501e-221\\
5.66866866866867	3.02131823639809e-222\\
5.67567567567568	3.0412050157538e-223\\
5.68268268268268	3.03866515222948e-224\\
5.68968968968969	3.01371079876784e-225\\
5.6966966966967	2.96684966606594e-226\\
5.7037037037037	2.89906795482659e-227\\
5.71071071071071	2.81179584608936e-228\\
5.71771771771772	2.70685723582223e-229\\
5.72472472472472	2.58640618533836e-230\\
5.73173173173173	2.45285318841246e-231\\
5.73873873873874	2.30878479066018e-232\\
5.74574574574575	2.15688031331677e-233\\
5.75275275275275	1.99982942494992e-234\\
5.75975975975976	1.84025408024048e-235\\
5.76676676676677	1.68063792946171e-236\\
5.77377377377377	1.52326573272128e-237\\
5.78078078078078	1.37017463553828e-238\\
5.78778778778779	1.22311842774416e-239\\
5.79479479479479	1.08354516743191e-240\\
5.8018018018018	9.52587853459487e-242\\
5.80880880880881	8.31067214714047e-243\\
5.81581581581582	7.19505183088611e-244\\
5.82282282282282	6.18147250068611e-245\\
5.82982982982983	5.26991682522375e-246\\
5.83683683683684	4.4582348932108e-247\\
5.84384384384384	3.74251074875014e-248\\
5.85085085085085	3.11743669148675e-249\\
5.85785785785786	2.57667861799972e-250\\
5.86486486486486	2.11321863872165e-251\\
5.87187187187187	1.71966446942613e-252\\
5.87887887887888	1.38851841793287e-253\\
5.88588588588589	1.11240195034523e-254\\
5.89289289289289	8.84234658490934e-256\\
5.8998998998999	6.97368843867841e-257\\
5.90690690690691	5.45682816947986e-258\\
5.91391391391391	4.23637364800425e-259\\
5.92092092092092	3.26300684717801e-260\\
5.92792792792793	2.49347467570751e-261\\
5.93493493493493	1.89037813251762e-262\\
5.94194194194194	1.42181354287189e-263\\
5.94894894894895	1.06091437698853e-264\\
5.95595595595596	7.85335505304441e-266\\
5.96296296296296	5.76714430194436e-267\\
5.96996996996997	4.20136647556126e-268\\
5.97697697697698	3.03625291137122e-269\\
5.98398398398398	2.1766891798937e-270\\
5.99099099099099	1.54795898108598e-271\\
5.997997997998	1.09199459583274e-272\\
6.00500500500501	7.64140101485499e-274\\
6.01201201201201	5.30408410978315e-275\\
6.01901901901902	3.65196126412466e-276\\
6.02602602602603	2.49409911883401e-277\\
6.03303303303303	1.68953220423651e-278\\
6.04004004004004	1.1352143863952e-279\\
6.04704704704705	7.56556501762287e-281\\
6.05405405405405	5.00092495952267e-282\\
6.06106106106106	3.27867379906776e-283\\
6.06806806806807	2.13195676907882e-284\\
6.07507507507508	1.37494173374426e-285\\
6.08208208208208	8.79446270173873e-287\\
6.08908908908909	5.57887769648117e-288\\
6.0960960960961	3.50986387475753e-289\\
6.1031031031031	2.18994393070129e-290\\
6.11011011011011	1.35509077833378e-291\\
6.11711711711712	8.31552805186104e-293\\
6.12412412412412	5.06046817367386e-294\\
6.13113113113113	3.05396708937617e-295\\
6.13813813813814	1.82769683999396e-296\\
6.14514514514515	1.08468461904426e-297\\
6.15215215215215	6.38345306865925e-299\\
6.15915915915916	3.72523843108591e-300\\
6.16616616616617	2.1557184656163e-301\\
6.17317317317317	1.23698066735233e-302\\
6.18018018018018	7.03817436238425e-304\\
6.18718718718719	3.97078801525518e-305\\
6.19419419419419	2.22129521574702e-306\\
6.2012012012012	1.23208900150813e-307\\
6.20820820820821	6.77606274952338e-309\\
6.21521521521522	3.69492484913787e-310\\
6.22222222222222	1.99765258485092e-311\\
6.22922922922923	1.07081335433518e-312\\
6.23623623623624	5.69089218513357e-314\\
6.24324324324324	2.99856198279824e-315\\
6.25025025025025	1.56640553560749e-316\\
6.25725725725726	8.11300256379453e-318\\
6.26426426426426	4.15543792747697e-319\\
6.27127127127127	2.10422558563787e-320\\
6.27827827827828	2.11460096420054e-321\\
6.28528528528529	0\\
6.29229229229229	0\\
6.2992992992993	0\\
6.30630630630631	0\\
6.31331331331331	0\\
6.32032032032032	0\\
6.32732732732733	0\\
6.33433433433433	0\\
6.34134134134134	0\\
6.34834834834835	0\\
6.35535535535536	0\\
6.36236236236236	0\\
6.36936936936937	0\\
6.37637637637638	0\\
6.38338338338338	0\\
6.39039039039039	0\\
6.3973973973974	0\\
6.4044044044044	0\\
6.41141141141141	0\\
6.41841841841842	0\\
6.42542542542543	0\\
6.43243243243243	0\\
6.43943943943944	0\\
6.44644644644645	0\\
6.45345345345345	0\\
6.46046046046046	0\\
6.46746746746747	0\\
6.47447447447447	0\\
6.48148148148148	0\\
6.48848848848849	0\\
6.4954954954955	0\\
6.5025025025025	0\\
6.50950950950951	0\\
6.51651651651652	0\\
6.52352352352352	0\\
6.53053053053053	0\\
6.53753753753754	0\\
6.54454454454454	0\\
6.55155155155155	0\\
6.55855855855856	0\\
6.56556556556557	0\\
6.57257257257257	0\\
6.57957957957958	0\\
6.58658658658659	0\\
6.59359359359359	0\\
6.6006006006006	0\\
6.60760760760761	0\\
6.61461461461461	0\\
6.62162162162162	0\\
6.62862862862863	0\\
6.63563563563564	0\\
6.64264264264264	0\\
6.64964964964965	0\\
6.65665665665666	0\\
6.66366366366366	0\\
6.67067067067067	0\\
6.67767767767768	0\\
6.68468468468468	0\\
6.69169169169169	0\\
6.6986986986987	0\\
6.70570570570571	0\\
6.71271271271271	0\\
6.71971971971972	0\\
6.72672672672673	0\\
6.73373373373373	0\\
6.74074074074074	0\\
6.74774774774775	0\\
6.75475475475475	0\\
6.76176176176176	0\\
6.76876876876877	0\\
6.77577577577578	0\\
6.78278278278278	0\\
6.78978978978979	0\\
6.7967967967968	0\\
6.8038038038038	0\\
6.81081081081081	0\\
6.81781781781782	0\\
6.82482482482482	0\\
6.83183183183183	0\\
6.83883883883884	0\\
6.84584584584585	0\\
6.85285285285285	0\\
6.85985985985986	0\\
6.86686686686687	0\\
6.87387387387387	0\\
6.88088088088088	0\\
6.88788788788789	0\\
6.89489489489489	0\\
6.9019019019019	0\\
6.90890890890891	0\\
6.91591591591592	0\\
6.92292292292292	0\\
6.92992992992993	0\\
6.93693693693694	0\\
6.94394394394394	0\\
6.95095095095095	0\\
6.95795795795796	0\\
6.96496496496496	0\\
6.97197197197197	0\\
6.97897897897898	0\\
6.98598598598599	0\\
6.99299299299299	0\\
7	0\\
};
\addlegendentry{$\mathcal{W}(1,0; 3,6)$}

\addplot [color=black, line width=1.5pt]
  table[row sep=crcr]{%
0	0\\
0.00700700700700701	3.46133292410115e-15\\
0.014014014014014	1.27232877952386e-13\\
0.021021021021021	1.04778915235462e-12\\
0.028028028028028	4.67687032337413e-12\\
0.035035035035035	1.49240740736695e-11\\
0.042042042042042	3.85149976221819e-11\\
0.049049049049049	8.58525465054697e-11\\
0.0560560560560561	1.71914023903555e-10\\
0.0630630630630631	3.17178997762423e-10\\
0.0700700700700701	5.48584298821543e-10\\
0.0770770770770771	9.0050335707581e-10\\
0.0840840840840841	1.41574766113527e-09\\
0.0910910910910911	2.14658853461687e-09\\
0.0980980980980981	3.15579772602503e-09\\
0.105105105105105	4.51770553859457e-09\\
0.112112112112112	6.31927540597362e-09\\
0.119119119119119	8.66119396246691e-09\\
0.126126126126126	1.16589757722901e-08\\
0.133133133133133	1.54440819774572e-08\\
0.14014014014014	2.01650522032061e-08\\
0.147147147147147	2.59886491266623e-08\\
0.154154154154154	3.31010151712814e-08\\
0.161161161161161	4.17088408384094e-08\\
0.168168168168168	5.20405442295252e-08\\
0.175175175175175	6.43474613495359e-08\\
0.182182182182182	7.89050468137788e-08\\
0.189189189189189	9.60140846098675e-08\\
0.196196196196196	1.16001908590796e-07\\
0.203203203203203	1.39223632398237e-07\\
0.21021021021021	1.66063388535148e-07\\
0.217217217217217	1.96935576324945e-07\\
0.224224224224224	2.32286118510867e-07\\
0.231231231231231	2.72593726264018e-07\\
0.238238238238238	3.18371172382032e-07\\
0.245245245245245	3.70166572472623e-07\\
0.252252252252252	4.28564673927528e-07\\
0.259259259259259	4.94188152502652e-07\\
0.266266266266266	5.67698916329709e-07\\
0.273273273273273	6.49799417193402e-07\\
0.28028028028028	7.41233968916216e-07\\
0.287287287287287	8.42790072700409e-07\\
0.294294294294294	9.55299749283665e-07\\
0.301301301301301	1.07964087777133e-06\\
0.308308308308308	1.21673854101415e-06\\
0.315315315315315	1.36756637740588e-06\\
0.322322322322322	1.53314793898066e-06\\
0.329329329329329	1.71455805569443e-06\\
0.336336336336336	1.91292420577971e-06\\
0.343343343343343	2.12942789206698e-06\\
0.35035035035035	2.36530602417012e-06\\
0.357357357357357	2.62185230643697e-06\\
0.364364364364364	2.9004186315698e-06\\
0.371371371371371	3.20241647982354e-06\\
0.378378378378378	3.52931832369309e-06\\
0.385385385385385	3.88265903800369e-06\\
0.392392392392392	4.26403731532129e-06\\
0.399399399399399	4.6751170866024e-06\\
0.406406406406406	5.11762894700551e-06\\
0.413413413413413	5.59337158678841e-06\\
0.42042042042042	6.10421322721789e-06\\
0.427427427427427	6.65209306142064e-06\\
0.434434434434434	7.23902270010596e-06\\
0.441441441441441	7.8670876220928e-06\\
0.448448448448448	8.53844862957538e-06\\
0.455455455455455	9.25534330806359e-06\\
0.462462462462462	1.00200874909355e-05\\
0.469469469469469	1.08350767285412e-05\\
0.476476476476476	1.17027877617985e-05\\
0.483483483483483	1.2625780000222e-05\\
0.49049049049049	1.36066970043294e-05\\
0.497497497497497	1.46482679723678e-05\\
0.504504504504504	1.57533092313079e-05\\
0.511511511511512	1.69247257320492e-05\\
0.518518518518518	1.81655125487872e-05\\
0.525525525525526	1.94787563824879e-05\\
0.532532532532533	2.08676370684208e-05\\
0.53953953953954	2.23354290876997e-05\\
0.546546546546547	2.38855030827821e-05\\
0.553553553553554	2.55213273768787e-05\\
0.560560560560561	2.72464694972253e-05\\
0.567567567567568	2.90645977021693e-05\\
0.574574574574575	3.09794825120251e-05\\
0.581581581581582	3.29949982436507e-05\\
0.588588588588589	3.51151245487012e-05\\
0.595595595595596	3.73439479555124e-05\\
0.602602602602603	3.96856634145703e-05\\
0.60960960960961	4.21445758475216e-05\\
0.616616616616617	4.47251016996807e-05\\
0.623623623623624	4.74317704959883e-05\\
0.630630630630631	5.0269226400378e-05\\
0.637637637637638	5.32422297785075e-05\\
0.644644644644645	5.6355658763809e-05\\
0.651651651651652	5.96145108268155e-05\\
0.658658658658659	6.30239043477206e-05\\
0.665665665665666	6.65890801921245e-05\\
0.672672672672673	7.03154032899263e-05\\
0.67967967967968	7.42083642173139e-05\\
0.686686686686687	7.82735807818117e-05\\
0.693693693693694	8.2516799610338e-05\\
0.700700700700701	8.69438977402281e-05\\
0.707707707707708	9.15608842131798e-05\\
0.714714714714715	9.63739016720732e-05\\
0.721721721721722	0.00010138922796062\\
0.728728728728729	0.000106613277725797\\
0.735735735735736	0.000112052604023014\\
0.742742742742743	0.000117713899923972\\
0.74974974974975	0.000123604000127165\\
0.756756756756757	0.000129729882570966\\
0.763763763763764	0.000136098670049267\\
0.770770770770771	0.000142717631829605\\
0.777777777777778	0.000149594185273732\\
0.784784784784785	0.000156735897460579\\
0.791791791791792	0.000164150486811556\\
0.798798798798799	0.00017184582471814\\
0.805805805805806	0.000179829937171693\\
0.812812812812813	0.000188111006395456\\
0.81981981981982	0.000196697372478669\\
0.826826826826827	0.000205597535012744\\
0.833833833833834	0.000214820154729453\\
0.840840840840841	0.000224374055141058\\
0.847847847847848	0.000234268224182321\\
0.854854854854855	0.000244511815854341\\
0.861861861861862	0.000255114151870155\\
0.868868868868869	0.000266084723302019\\
0.875875875875876	0.000277433192230333\\
0.882882882882883	0.000289169393394116\\
0.88988988988989	0.000301303335842976\\
0.896896896896897	0.0003138452045905\\
0.903903903903904	0.000326805362268989\\
0.910910910910911	0.000340194350785471\\
0.917917917917918	0.000354022892978908\\
0.924924924924925	0.000368301894278523\\
0.931931931931932	0.000383042444363163\\
0.938938938938939	0.000398255818821628\\
0.945945945945946	0.000413953480813866\\
0.952952952952953	0.000430147082732958\\
0.95995995995996	0.000446848467867806\\
0.966966966966967	0.000464069672066424\\
0.973973973973974	0.000481822925399753\\
0.980980980980981	0.000500120653825892\\
0.987987987987988	0.00051897548085466\\
0.994994994994995	0.000538400229212372\\
1.002002002002	0.000558407922506752\\
1.00900900900901	0.000579011786891849\\
1.01601601601602	0.000600225252732874\\
1.02302302302302	0.000622061956270823\\
1.03003003003003	0.000644535741286802\\
1.03703703703704	0.000667660660765907\\
1.04404404404404	0.000691450978560562\\
1.05105105105105	0.000715921171053179\\
1.05805805805806	0.000741085928818025\\
1.06506506506507	0.000766960158282161\\
1.07207207207207	0.000793558983385321\\
1.07907907907908	0.000820897747238595\\
1.08608608608609	0.000848992013781783\\
1.09309309309309	0.00087785756943927\\
1.1001001001001	0.000907510424774283\\
1.10710710710711	0.000937966816141369\\
1.11411411411411	0.000969243207336957\\
1.12112112112112	0.00100135629124783\\
1.12812812812813	0.00103432299149734\\
1.13513513513514	0.00106816046408926\\
1.14214214214214	0.00110288609904899\\
1.14914914914915	0.00113851752206205\\
1.15615615615616	0.00117507259610964\\
1.16316316316316	0.00121256942310107\\
1.17017017017017	0.00125102634550288\\
1.17717717717718	0.00129046194796449\\
1.18418418418418	0.00133089505894013\\
1.19119119119119	0.00137234475230691\\
1.1981981981982	0.00141483034897875\\
1.20520520520521	0.00145837141851599\\
1.21221221221221	0.00150298778073056\\
1.21921921921922	0.00154869950728627\\
1.22622622622623	0.00159552692329421\\
1.23323323323323	0.00164349060890289\\
1.24024024024024	0.00169261140088298\\
1.24724724724725	0.00174291039420629\\
1.25425425425425	0.00179440894361885\\
1.26126126126126	0.00184712866520777\\
1.26826826826827	0.00190109143796163\\
1.27527527527528	0.00195631940532414\\
1.28228228228228	0.00201283497674085\\
1.28928928928929	0.00207066082919843\\
1.2962962962963	0.00212981990875654\\
1.3033033033033	0.00219033543207172\\
1.31031031031031	0.00225223088791315\\
1.31731731731732	0.0023155300386699\\
1.32432432432432	0.00238025692184946\\
1.33133133133133	0.00244643585156706\\
1.33833833833834	0.00251409142002562\\
1.34534534534535	0.00258324849898587\\
1.35235235235235	0.00265393224122638\\
1.35935935935936	0.00272616808199312\\
1.36636636636637	0.0027999817404381\\
1.37337337337337	0.00287539922104696\\
1.38038038038038	0.00295244681505484\\
1.38738738738739	0.00303115110185033\\
1.39439439439439	0.00311153895036714\\
1.4014014014014	0.00319363752046293\\
1.40840840840841	0.00327747426428499\\
1.41541541541542	0.00336307692762237\\
1.42242242242242	0.00345047355124401\\
1.42942942942943	0.00353969247222242\\
1.43643643643644	0.00363076232524242\\
1.44344344344344	0.00372371204389464\\
1.45045045045045	0.00381857086195317\\
1.45745745745746	0.0039153683146369\\
1.46446446446446	0.00401413423985419\\
1.47147147147147	0.0041148987794302\\
1.47847847847848	0.00421769238031657\\
1.48548548548549	0.00432254579578271\\
1.49249249249249	0.00442949008658846\\
1.4994994994995	0.00453855662213728\\
1.50650650650651	0.00464977708160972\\
1.51351351351351	0.00476318345507636\\
1.52052052052052	0.00487880804458982\\
1.52752752752753	0.00499668346525524\\
1.53453453453453	0.00511684264627856\\
1.54154154154154	0.00523931883199213\\
1.54854854854855	0.00536414558285687\\
1.55555555555556	0.00549135677644053\\
1.56256256256256	0.00562098660837138\\
1.56956956956957	0.00575306959326652\\
1.57657657657658	0.00588764056563437\\
1.58358358358358	0.00602473468075057\\
1.59059059059059	0.00616438741550665\\
1.5975975975976	0.00630663456923062\\
1.6046046046046	0.00645151226447904\\
1.61161161161161	0.00659905694779963\\
1.61861861861862	0.00674930539046375\\
1.62562562562563	0.00690229468916807\\
1.63263263263263	0.00705806226670454\\
1.63963963963964	0.00721664587259804\\
1.64664664664665	0.00737808358371079\\
1.65365365365365	0.0075424138048128\\
1.66066066066066	0.00770967526911746\\
1.66766766766767	0.00787990703878161\\
1.67467467467467	0.008053148505369\\
1.68168168168168	0.00822943939027655\\
1.68868868868869	0.00840881974512226\\
1.6956956956957	0.00859132995209418\\
1.7027027027027	0.00877701072425927\\
1.70970970970971	0.00896590310583139\\
1.71671671671672	0.00915804847239745\\
1.72372372372372	0.00935348853110076\\
1.73073073073073	0.00955226532078061\\
1.73773773773774	0.0097544212120672\\
1.74474474474474	0.00995999890743066\\
1.75175175175175	0.0101690414411835\\
1.75875875875876	0.0103815921794354\\
1.76576576576577	0.0105976948199987\\
1.77277277277277	0.0108173933922446\\
1.77977977977978	0.0110407322569082\\
1.78678678678679	0.0112677561058412\\
1.79379379379379	0.0114985099617123\\
1.8008008008008	0.011733039177652\\
1.80780780780781	0.0119713894368436\\
1.81481481481481	0.0122136067520561\\
1.82182182182182	0.0124597374651205\\
1.82882882882883	0.0127098282463463\\
1.83583583583584	0.012963926093879\\
1.84284284284284	0.0132220783329945\\
1.84984984984985	0.0134843326153327\\
1.85685685685686	0.013750736918066\\
1.86386386386386	0.0140213395430029\\
1.87087087087087	0.0142961891156254\\
1.87787787787788	0.0145753345840588\\
1.88488488488488	0.014858825217971\\
1.89189189189189	0.0151467106074035\\
1.8988988988989	0.0154390406615288\\
1.90590590590591	0.015735865607335\\
1.91291291291291	0.0160372359882364\\
1.91991991991992	0.0163432026626071\\
1.92692692692693	0.0166538168022377\\
1.93393393393393	0.0169691298907128\\
1.94094094094094	0.0172891937217082\\
1.94794794794795	0.0176140603972054\\
1.95495495495495	0.0179437823256241\\
1.96196196196196	0.0182784122198679\\
1.96896896896897	0.0186180030952847\\
1.97597597597598	0.0189626082675386\\
1.98298298298298	0.0193122813503915\\
1.98998998998999	0.0196670762533943\\
1.996996996997	0.0200270471794843\\
2.004004004004	0.020392248622488\\
2.01101101101101	0.0207627353645284\\
2.01801801801802	0.0211385624733331\\
2.02502502502503	0.0215197852994436\\
2.03203203203203	0.0219064594733226\\
2.03903903903904	0.0222986409023585\\
2.04604604604605	0.0226963857677649\\
2.05305305305305	0.0230997505213729\\
2.06006006006006	0.0235087918823155\\
2.06706706706707	0.0239235668336013\\
2.07407407407407	0.0243441326185761\\
2.08108108108108	0.0247705467372705\\
2.08808808808809	0.0252028669426318\\
2.0950950950951	0.0256411512366375\\
2.1021021021021	0.0260854578662897\\
2.10910910910911	0.0265358453194878\\
2.11611611611612	0.0269923723207775\\
2.12312312312312	0.0274550978269742\\
2.13013013013013	0.0279240810226591\\
2.13713713713714	0.0283993813155461\\
2.14414414414414	0.0288810583317165\\
2.15115115115115	0.029369171910721\\
2.15815815815816	0.0298637821005454\\
2.16516516516517	0.0303649491524397\\
2.17217217217217	0.0308727335156066\\
2.17917917917918	0.0313871958317485\\
2.18618618618619	0.0319083969294714\\
2.19319319319319	0.0324363978185421\\
2.2002002002002	0.0329712596839971\\
2.20720720720721	0.0335130438801027\\
2.21421421421421	0.0340618119241609\\
2.22122122122122	0.0346176254901626\\
2.22822822822823	0.0351805464022826\\
2.23523523523524	0.0357506366282169\\
2.24224224224224	0.0363279582723584\\
2.24924924924925	0.0369125735688092\\
2.25625625625626	0.0375045448742279\\
2.26326326326326	0.038103934660509\\
2.27027027027027	0.0387108055072922\\
2.27727727727728	0.0393252200942998\\
2.28428428428428	0.0399472411934994\\
2.29129129129129	0.0405769316610903\\
2.2982982982983	0.0412143544293107\\
2.30530530530531	0.0418595724980633\\
2.31231231231231	0.0425126489263588\\
2.31931931931932	0.0431736468235717\\
2.32632632632633	0.0438426293405097\\
2.33333333333333	0.0445196596602917\\
2.34034034034034	0.0452048009890338\\
2.34734734734735	0.0458981165463406\\
2.35435435435435	0.0465996695555982\\
2.36136136136136	0.0473095232340691\\
2.36836836836837	0.0480277407827844\\
2.37537537537538	0.048754385376232\\
2.38238238238238	0.0494895201518396\\
2.38938938938939	0.0502332081992473\\
2.3963963963964	0.0509855125493709\\
2.4034034034034	0.0517464961632515\\
2.41041041041041	0.0525162219206891\\
2.41741741741742	0.05329475260866\\
2.42442442442442	0.0540821509095129\\
2.43143143143143	0.0548784793889439\\
2.43843843843844	0.0556838004837467\\
2.44544544544545	0.0564981764893365\\
2.45245245245245	0.0573216695470452\\
2.45945945945946	0.0581543416311858\\
2.46646646646647	0.0589962545358838\\
2.47347347347347	0.0598474698616732\\
2.48048048048048	0.0607080490018561\\
2.48748748748749	0.0615780531286216\\
2.49449449449449	0.0624575431789251\\
2.5015015015015	0.0633465798401225\\
2.50850850850851	0.0642452235353598\\
2.51551551551552	0.0651535344087156\\
2.52252252252252	0.0660715723100938\\
2.52952952952953	0.0669993967798644\\
2.53653653653654	0.0679370670332527\\
2.54354354354354	0.0688846419444717\\
2.55055055055055	0.0698421800305994\\
2.55755755755756	0.0708097394351961\\
2.56456456456456	0.0717873779116616\\
2.57157157157157	0.0727751528063312\\
2.57857857857858	0.0737731210413072\\
2.58558558558559	0.0747813390970251\\
2.59259259259259	0.0757998629945534\\
2.5995995995996	0.0768287482776248\\
2.60660660660661	0.0778680499943971\\
2.61361361361361	0.0789178226789446\\
2.62062062062062	0.0799781203324742\\
2.62762762762763	0.0810489964042704\\
2.63463463463463	0.0821305037723636\\
2.64164164164164	0.0832226947239229\\
2.64864864864865	0.0843256209353715\\
2.65565565565566	0.0854393334522234\\
2.66266266266266	0.0865638826686417\\
2.66966966966967	0.0876993183067165\\
2.67667667667668	0.0888456893954609\\
2.68368368368368	0.0900030442495275\\
2.69069069069069	0.0911714304476405\\
2.6976976976977	0.0923508948107461\\
2.7047047047047	0.0935414833798803\\
2.71171171171171	0.0947432413937513\\
2.71871871871872	0.0959562132660398\\
2.72572572572573	0.0971804425624147\\
2.73273273273273	0.0984159719772634\\
2.73973973973974	0.0996628433101407\\
2.74674674674675	0.100921097441931\\
2.75375375375375	0.102190774310728\\
2.76076076076076	0.103471912887431\\
2.76776776776777	0.104764551151059\\
2.77477477477477	0.106068726063782\\
2.78178178178178	0.107384473545666\\
2.78878878878879	0.10871182844915\\
2.7957957957958	0.110050824533225\\
2.8028028028028	0.111401494437351\\
2.80980980980981	0.112763869655086\\
2.81681681681682	0.114137980507444\\
2.82382382382382	0.115523856115976\\
2.83083083083083	0.116921524375575\\
2.83783783783784	0.118331011927022\\
2.84484484484484	0.119752344129242\\
2.85185185185185	0.121185545031315\\
2.85885885885886	0.122630637344207\\
2.86586586586587	0.124087642412243\\
2.87287287287287	0.125556580184322\\
2.87987987987988	0.127037469184873\\
2.88688688688689	0.128530326484554\\
2.89389389389389	0.130035167670707\\
2.9009009009009	0.13155200681756\\
2.90790790790791	0.133080856456181\\
2.91491491491491	0.134621727544204\\
2.92192192192192	0.136174629435301\\
2.92892892892893	0.137739569848438\\
2.93593593593594	0.13931655483689\\
2.94294294294294	0.14090558875704\\
2.94994994994995	0.142506674236952\\
2.95695695695696	0.144119812144734\\
2.96396396396396	0.145745001556694\\
2.97097097097097	0.147382239725283\\
2.97797797797798	0.149031522046849\\
2.98498498498498	0.150692842029192\\
2.99199199199199	0.152366191258934\\
2.998998998999	0.154051559368713\\
3.00600600600601	0.155748934004194\\
3.01301301301301	0.157458300790921\\
3.02002002002002	0.159179643301004\\
3.02702702702703	0.160912943019657\\
3.03403403403403	0.162658179311583\\
3.04104104104104	0.164415329387231\\
3.04804804804805	0.166184368268917\\
3.05505505505506	0.167965268756822\\
3.06206206206206	0.169758001394885\\
3.06906906906907	0.171562534436582\\
3.07607607607608	0.173378833810617\\
3.08308308308308	0.175206863086526\\
3.09009009009009	0.177046583440201\\
3.0970970970971	0.178897953619356\\
3.1041041041041	0.180760929908926\\
3.11111111111111	0.182635466096434\\
3.11811811811812	0.184521513437313\\
3.12512512512513	0.186419020620208\\
3.13213213213213	0.188327933732271\\
3.13913913913914	0.190248196224448\\
3.14614614614615	0.192179748876787\\
3.15315315315315	0.194122529763759\\
3.16016016016016	0.19607647421963\\
3.16716716716717	0.198041514803875\\
3.17417417417417	0.200017581266652\\
3.18118118118118	0.202004600514367\\
3.18818818818819	0.204002496575315\\
3.1951951951952	0.206011190565437\\
3.2022022022022	0.208030600654195\\
3.20920920920921	0.21006064203058\\
3.21621621621622	0.21210122686927\\
3.22322322322322	0.214152264296966\\
3.23023023023023	0.21621366035889\\
3.23723723723724	0.218285317985504\\
3.24424424424424	0.220367136959426\\
3.25125125125125	0.222459013882593\\
3.25825825825826	0.224560842143665\\
3.26526526526527	0.226672511885702\\
3.27227227227227	0.228793909974123\\
3.27927927927928	0.230924919964972\\
3.28628628628629	0.233065422073511\\
3.29329329329329	0.235215293143145\\
3.3003003003003	0.237374406614718\\
3.30730730730731	0.23954263249619\\
3.31431431431431	0.241719837332711\\
3.32132132132132	0.243905884177122\\
3.32832832832833	0.246100632560898\\
3.33533533533534	0.248303938465552\\
3.34234234234234	0.250515654294534\\
3.34934934934935	0.252735628845627\\
3.35635635635636	0.254963707283884\\
3.36336336336336	0.257199731115106\\
3.37037037037037	0.259443538159909\\
3.37737737737738	0.261694962528377\\
3.38438438438438	0.263953834595347\\
3.39139139139139	0.266219980976338\\
3.3983983983984	0.268493224504144\\
3.40540540540541	0.270773384206131\\
3.41241241241241	0.273060275282249\\
3.41941941941942	0.275353709083791\\
3.42642642642643	0.277653493092917\\
3.43343343343343	0.279959430902982\\
3.44044044044044	0.282271322199677\\
3.44744744744745	0.284588962743027\\
3.45445445445445	0.286912144350256\\
3.46146146146146	0.289240654879557\\
3.46846846846847	0.291574278214789\\
3.47547547547548	0.29391279425113\\
3.48248248248248	0.296255978881715\\
3.48948948948949	0.29860360398528\\
3.4964964964965	0.300955437414845\\
3.5035035035035	0.303311242987473\\
3.51051051051051	0.305670780475111\\
3.51751751751752	0.308033805596562\\
3.52452452452452	0.310400070010609\\
3.53153153153153	0.312769321310316\\
3.53853853853854	0.315141303018545\\
3.54554554554555	0.317515754584708\\
3.55255255255255	0.319892411382785\\
3.55955955955956	0.32227100471065\\
3.56656656656657	0.324651261790707\\
3.57357357357357	0.327032905771906\\
3.58058058058058	0.329415655733119\\
3.58758758758759	0.331799226687954\\
3.59459459459459	0.334183329591002\\
3.6016016016016	0.336567671345564\\
3.60860860860861	0.338951954812878\\
3.61561561561562	0.34133587882289\\
3.62262262262262	0.343719138186577\\
3.62962962962963	0.346101423709873\\
3.63663663663664	0.348482422209213\\
3.64364364364364	0.350861816528733\\
3.65065065065065	0.353239285559147\\
3.65765765765766	0.355614504258337\\
3.66466466466466	0.357987143673684\\
3.67167167167167	0.360356870966164\\
3.67867867867868	0.362723349436246\\
3.68568568568569	0.365086238551604\\
3.69269269269269	0.367445193976699\\
3.6996996996997	0.369799867604223\\
3.70670670670671	0.372149907588466\\
3.71371371371371	0.37449495838061\\
3.72072072072072	0.376834660765984\\
3.72772772772773	0.379168651903317\\
3.73473473473473	0.381496565365991\\
3.74174174174174	0.383818031185345\\
3.74874874874875	0.386132675896036\\
3.75575575575576	0.388440122583495\\
3.76276276276276	0.390739990933487\\
3.76976976976977	0.39303189728382\\
3.77677677677678	0.3953154546782\\
3.78378378378378	0.397590272922273\\
3.79079079079079	0.399855958641872\\
3.7977977977978	0.402112115343483\\
3.8048048048048	0.404358343476954\\
3.81181181181181	0.406594240500472\\
3.81881881881882	0.408819400947816\\
3.82582582582583	0.411033416497909\\
3.83283283283283	0.413235876046692\\
3.83983983983984	0.415426365781322\\
3.84684684684685	0.417604469256721\\
3.85385385385385	0.419769767474487\\
3.86086086086086	0.421921838964176\\
3.86786786786787	0.424060259866969\\
3.87487487487487	0.426184604021744\\
3.88188188188188	0.428294443053546\\
3.88888888888889	0.430389346464478\\
3.8958958958959	0.432468881727012\\
3.9029029029029	0.434532614379734\\
3.90990990990991	0.436580108125522\\
3.91691691691692	0.43861092493216\\
3.92392392392392	0.440624625135398\\
3.93093093093093	0.442620767544454\\
3.93793793793794	0.444598909549954\\
3.94494494494494	0.44655860723432\\
3.95195195195195	0.448499415484589\\
3.95895895895896	0.450420888107667\\
3.96596596596597	0.452322577948016\\
3.97297297297297	0.454204037007751\\
3.97997997997998	0.456064816569155\\
3.98698698698699	0.457904467319587\\
3.99399399399399	0.459722539478785\\
4.001001001001	0.461518582928528\\
4.00800800800801	0.463292147344672\\
4.01501501501502	0.465042782331508\\
4.02202202202202	0.466770037558448\\
4.02902902902903	0.468473462899009\\
4.03603603603604	0.470152608572064\\
4.04304304304304	0.471807025285349\\
4.05005005005005	0.473436264381189\\
4.05705705705706	0.475039877984414\\
4.06406406406406	0.476617419152443\\
4.07107107107107	0.478168442027486\\
4.07807807807808	0.479692501990851\\
4.08508508508509	0.481189155819294\\
4.09209209209209	0.482657961843397\\
4.0990990990991	0.484098480107912\\
4.10610610610611	0.485510272534041\\
4.11311311311311	0.486892903083598\\
4.12012012012012	0.488245937925012\\
4.12712712712713	0.489568945601111\\
4.13413413413413	0.490861497198647\\
4.14114114114114	0.492123166519495\\
4.14814814814815	0.493353530253479\\
4.15515515515516	0.494552168152761\\
4.16216216216216	0.495718663207725\\
4.16916916916917	0.496852601824312\\
4.17617617617618	0.497953574002714\\
4.18318318318318	0.49902117351738\\
4.19019019019019	0.500054998098251\\
4.1971971971972	0.501054649613156\\
4.2042042042042	0.502019734251292\\
4.21121121121121	0.502949862707709\\
4.21821821821822	0.503844650368725\\
4.22522522522523	0.504703717498183\\
4.23223223223223	0.505526689424466\\
4.23923923923924	0.506313196728191\\
4.24624624624625	0.507062875430477\\
4.25325325325325	0.507775367181717\\
4.26026026026026	0.508450319450735\\
4.26726726726727	0.509087385714258\\
4.27427427427427	0.509686225646585\\
4.28128128128128	0.510246505309359\\
4.28828828828829	0.510767897341353\\
4.2952952952953	0.511250081148133\\
4.3023023023023	0.511692743091531\\
4.30930930930931	0.512095576678788\\
4.31631631631632	0.512458282751273\\
4.32332332332332	0.51278056967266\\
4.33033033033033	0.513062153516446\\
4.33733733733734	0.513302758252702\\
4.34434434434434	0.513502115933929\\
4.35135135135135	0.513659966879903\\
4.35835835835836	0.513776059861391\\
4.36536536536537	0.51385015228261\\
4.37237237237237	0.5138820103623\\
4.37937937937938	0.513871409313296\\
4.38638638638639	0.513818133520462\\
4.39339339339339	0.51372197671686\\
4.4004004004004	0.513582742158019\\
4.40740740740741	0.51340024279419\\
4.41441441441441	0.513174301440419\\
4.42142142142142	0.512904750944348\\
4.42842842842843	0.512591434351569\\
4.43543543543544	0.512234205068419\\
4.44244244244244	0.511832927022075\\
4.44944944944945	0.5113874748178\\
4.45645645645646	0.510897733893221\\
4.46346346346346	0.510363600669481\\
4.47047047047047	0.509784982699142\\
4.47747747747748	0.509161798810686\\
4.48448448448448	0.50849397924949\\
4.49149149149149	0.507781465815115\\
4.4984984984985	0.507024211994796\\
4.50550550550551	0.506222183092966\\
4.51251251251251	0.505375356356695\\
4.51951951951952	0.5044837210969\\
4.52652652652653	0.503547278805181\\
4.53353353353353	0.502566043266159\\
4.54054054054054	0.501540040665164\\
4.54754754754755	0.500469309691156\\
4.55455455455455	0.499353901634724\\
4.56156156156156	0.498193880481052\\
4.56856856856857	0.496989322997695\\
4.57557557557558	0.495740318817063\\
4.58258258258258	0.494446970513461\\
4.58958958958959	0.493109393674569\\
4.5965965965966	0.491727716967243\\
4.6036036036036	0.490302082197495\\
4.61061061061061	0.488832644364557\\
4.61761761761762	0.487319571708889\\
4.62462462462462	0.485763045754033\\
4.63163163163163	0.484163261342183\\
4.63863863863864	0.482520426663383\\
4.64564564564565	0.480834763278224\\
4.65265265265265	0.479106506133954\\
4.65965965965966	0.477335903573891\\
4.66666666666667	0.475523217340045\\
4.67367367367367	0.473668722568862\\
4.68068068068068	0.471772707779991\\
4.68768768768769	0.469835474857995\\
4.69469469469469	0.467857339026925\\
4.7017017017017	0.465838628817681\\
4.70870870870871	0.463779686028084\\
4.71571571571572	0.461680865675596\\
4.72272272272272	0.45954253594262\\
4.72972972972973	0.457365078114332\\
4.73673673673674	0.455148886508976\\
4.74374374374374	0.452894368400595\\
4.75075075075075	0.450601943934133\\
4.75775775775776	0.448272046032889\\
4.76476476476476	0.445905120298289\\
4.77177177177177	0.443501624901934\\
4.77877877877878	0.441062030469927\\
4.78578578578579	0.438586819959456\\
4.79279279279279	0.436076488527608\\
4.7997997997998	0.433531543392453\\
4.80680680680681	0.430952503686356\\
4.81381381381381	0.42833990030156\\
4.82082082082082	0.425694275728049\\
4.82782782782783	0.423016183883706\\
4.83483483483483	0.420306189936811\\
4.84184184184184	0.417564870120909\\
4.84884884884885	0.414792811542094\\
4.85585585585586	0.411990611978752\\
4.86286286286286	0.409158879673838\\
4.86986986986987	0.406298233119744\\
4.87687687687688	0.403409300835821\\
4.88388388388388	0.400492721138647\\
4.89089089089089	0.39754914190513\\
4.8978978978979	0.394579220328521\\
4.9049049049049	0.391583622667463\\
4.91191191191191	0.388563023988161\\
4.91891891891892	0.385518107899811\\
4.92592592592593	0.38244956628338\\
4.93293293293293	0.379358099013908\\
4.93993993993994	0.37624441367643\\
4.94694694694695	0.37310922527568\\
4.95395395395395	0.369953255939738\\
4.96096096096096	0.366777234617753\\
4.96796796796797	0.363581896771928\\
4.97497497497497	0.360367984063929\\
4.98198198198198	0.357136244035908\\
4.98898898898899	0.353887429786305\\
4.995995995996	0.350622299640646\\
5.003003003003	0.347341616817516\\
5.01001001001001	0.344046149089925\\
5.01701701701702	0.340736668442275\\
5.02402402402402	0.337413950723141\\
5.03103103103103	0.334078775294098\\
5.03803803803804	0.330731924674815\\
5.04504504504505	0.327374184184666\\
5.05205205205205	0.324006341581078\\
5.05905905905906	0.320629186694882\\
5.06606606606607	0.317243511062905\\
5.07307307307307	0.31385010755807\\
5.08008008008008	0.310449770017259\\
5.08708708708709	0.307043292867201\\
5.09409409409409	0.303631470748671\\
5.1011011011011	0.300215098139253\\
5.10810810810811	0.296794968974958\\
5.11511511511512	0.293371876270986\\
5.12212212212212	0.289946611741893\\
5.12912912912913	0.286519965421473\\
5.13613613613614	0.28309272528264\\
5.14314314314314	0.279665676857593\\
5.15015015015015	0.276239602858574\\
5.15715715715716	0.272815282799509\\
5.16416416416416	0.269393492618832\\
5.17117117117117	0.265975004303783\\
5.17817817817818	0.262560585516501\\
5.18518518518519	0.2591509992222\\
5.19219219219219	0.255747003319725\\
5.1991991991992	0.252349350274801\\
5.20620620620621	0.248958786756277\\
5.21321321321321	0.24557605327565\\
5.22022022022022	0.242201883830191\\
5.22722722722723	0.238837005549957\\
5.23423423423423	0.235482138348979\\
5.24124124124124	0.232137994580942\\
5.24824824824825	0.22880527869963\\
5.25525525525526	0.225484686924435\\
5.26226226226226	0.222176906911203\\
5.26926926926927	0.218882617428724\\
5.27627627627628	0.215602488041116\\
5.28328328328328	0.21233717879639\\
5.29029029029029	0.209087339921476\\
5.2972972972973	0.205853611523956\\
5.3043043043043	0.202636623300769\\
5.31131131131131	0.199436994254148\\
5.31831831831832	0.196255332415033\\
5.32532532532533	0.193092234574198\\
5.33233233233233	0.189948286021329\\
5.33933933933934	0.186824060292283\\
5.34634634634635	0.183720118924745\\
5.35335335335335	0.180637011222495\\
5.36036036036036	0.177575274028496\\
5.36736736736737	0.174535431506992\\
5.37437437437437	0.171517994934807\\
5.38138138138138	0.168523462502031\\
5.38838838838839	0.165552319122252\\
5.3953953953954	0.162605036252498\\
5.4024024024024	0.159682071723051\\
5.40940940940941	0.156783869577265\\
5.41641641641642	0.153910859921516\\
5.42342342342342	0.151063458785417\\
5.43043043043043	0.148242067992405\\
5.43743743743744	0.145447075040795\\
5.44444444444444	0.142678852995395\\
5.45145145145145	0.139937760389772\\
5.45845845845846	0.137224141139211\\
5.46546546546547	0.134538324464447\\
5.47247247247247	0.131880624826216\\
5.47947947947948	0.129251341870632\\
5.48648648648649	0.126650760385451\\
5.49349349349349	0.1240791502672\\
5.5005005005005	0.12153676649919\\
5.50750750750751	0.119023849140397\\
5.51451451451451	0.116540623325182\\
5.52152152152152	0.114087299273819\\
5.52852852852853	0.111664072313786\\
5.53553553553554	0.109271122911752\\
5.54254254254254	0.106908616716203\\
5.54954954954955	0.104576704610615\\
5.55655655655656	0.10227552277708\\
5.56356356356356	0.100005192770295\\
5.57057057057057	0.0977658216017799\\
5.57757757757758	0.0955575018342115\\
5.58458458458458	0.0933803116857334\\
5.59159159159159	0.0912343151440891\\
5.5985985985986	0.0891195620904252\\
5.60560560560561	0.0870360884325952\\
5.61261261261261	0.0849839162477842\\
5.61961961961962	0.0829630539342663\\
5.62662662662663	0.0809734963720946\\
5.63363363363363	0.0790152250925233\\
5.64064064064064	0.0770882084559348\\
5.64764764764765	0.0751924018380571\\
5.65465465465465	0.0733277478242312\\
5.66166166166166	0.0714941764114891\\
5.66866866866867	0.069691605218192\\
5.67567567567568	0.0679199397009739\\
5.68268268268268	0.0661790733787254\\
5.68968968968969	0.064468888063344\\
5.6966966966967	0.0627892540969811\\
5.7037037037037	0.061140030595495\\
5.71071071071071	0.0595210656978266\\
5.71771771771772	0.0579321968210024\\
5.72472472472472	0.0563732509204678\\
5.73173173173173	0.0548440447554462\\
5.73873873873874	0.0533443851590185\\
5.74574574574575	0.051874069312615\\
5.75275275275275	0.050432885024605\\
5.75975975975976	0.0490206110126679\\
5.76676676676677	0.0476370171896329\\
5.77377377377377	0.0462818649524633\\
5.78078078078078	0.044954907474068\\
5.78778778778779	0.0436558899976199\\
5.79479479479479	0.0423845501330594\\
5.8018018018018	0.0411406181554596\\
5.80880880880881	0.0399238173049393\\
5.81581581581582	0.0387338640877974\\
5.82282282282282	0.0375704685785554\\
5.82982982982983	0.0364333347225922\\
5.83683683683684	0.0353221606390566\\
5.84384384384384	0.0342366389237467\\
5.85085085085085	0.0331764569516507\\
5.85785785785786	0.0321412971788447\\
5.86486486486486	0.0311308374434481\\
5.87187187187187	0.0301447512653404\\
5.87887887887888	0.0291827081443524\\
5.88588588588589	0.0282443738566433\\
5.89289289289289	0.0273294107489854\\
5.8998998998999	0.0264374780306865\\
5.90690690690691	0.0255682320628768\\
5.91391391391391	0.0247213266449045\\
5.92092092092092	0.0238964132975852\\
5.92792792792793	0.0230931415430578\\
5.93493493493493	0.0223111591810107\\
5.94194194194194	0.0215501125610435\\
5.94894894894895	0.0208096468509455\\
5.95595595595596	0.0200894063006731\\
5.96296296296296	0.0193890345018213\\
5.96996996996997	0.0187081746423931\\
5.97697697697698	0.0180464697566757\\
5.98398398398398	0.0174035629700455\\
5.99099099099099	0.0167790977385339\\
5.997997997998	0.0161727180829883\\
6.00500500500501	0.0155840688176818\\
6.01201201201201	0.0150127957732275\\
6.01901901901902	0.0144585460136662\\
6.02602602602603	0.0139209680476044\\
6.03303303303303	0.0133997120332923\\
6.04004004004004	0.0128944299775372\\
6.04704704704705	0.0124047759283609\\
6.05405405405405	0.0119304061613168\\
6.06106106106106	0.0114709793593971\\
6.06806806806807	0.0110261567864621\\
6.07507507507508	0.0105956024541423\\
6.08208208208208	0.0101789832821679\\
6.08908908908909	0.00977596925209082\\
6.0960960960961	0.00938623355437556\\
6.1031031031031	0.00900945272884368\\
6.11011011011011	0.00864530679846508\\
6.11711711711712	0.00829347939649874\\
6.12412412412412	0.0079536578869961\\
6.13113113113113	0.00762553347868609\\
6.13813813813814	0.00730880133227181\\
6.14514514514515	0.007003160661176\\
6.15215215215215	0.00670831482578102\\
6.15915915915916	0.00642397142121624\\
6.16616616616617	0.00614984235875436\\
6.17317317317317	0.00588564394088502\\
6.18018018018018	0.00563109693014143\\
6.18718718718719	0.0053859266117618\\
6.19419419419419	0.00514986285027576\\
6.2012012012012	0.00492264014011025\\
6.20820820820821	0.00470399765031643\\
6.21521521521522	0.00449367926352557\\
6.22222222222222	0.00429143360924538\\
6.22922922922923	0.00409701409161509\\
6.23623623623624	0.00391017891174215\\
6.24324324324324	0.00373069108474675\\
6.25025025025025	0.0035583184516464\\
6.25725725725726	0.00339283368621482\\
6.26426426426426	0.00323401429695466\\
6.27127127127127	0.00308164262432543\\
6.27827827827828	0.00293550583337176\\
6.28528528528529	0.00279539590189965\\
6.29229229229229	0.00266110960435046\\
6.2992992992993	0.00253244849152422\\
6.30630630630631	0.00240921886630641\\
6.31331331331331	0.00229123175555245\\
6.32032032032032	0.00217830287828632\\
6.32732732732733	0.00207025261037025\\
6.33433433433433	0.00196690594580237\\
6.34134134134134	0.00186809245480013\\
6.34834834834835	0.00177364623882721\\
6.35535535535536	0.00168340588272053\\
6.36236236236236	0.00159721440407472\\
6.36936936936937	0.00151491920003903\\
6.37637637637638	0.00143637199168212\\
6.38338338338338	0.00136142876607759\\
6.39039039039039	0.00128994971626221\\
6.3973973973974	0.00122179917921731\\
6.4044044044044	0.00115684557202075\\
6.41141141141141	0.001094961326316\\
6.41841841841842	0.00103602282124168\\
6.42542542542543	0.000979910314962515\\
6.43243243243243	0.00092650787494023\\
6.43943943943944	0.000875703307079692\\
6.44644644644645	0.000827388083882806\\
6.45345345345345	0.000781457271739447\\
6.46046046046046	0.000737809457481454\\
6.46746746746747	0.000696346674322307\\
6.47447447447447	0.000656974327301542\\
6.48148148148148	0.000619601118349454\\
6.48848848848849	0.000584138971083979\\
6.4954954954955	0.000550502955447683\\
6.5025025025025	0.000518611212289234\\
6.50950950950951	0.000488384877989681\\
6.51651651651652	0.000459748009229955\\
6.52352352352352	0.000432627507992131\\
6.53053053053053	0.000406953046882934\\
6.53753753753754	0.000382656994864035\\
6.54454454454454	0.000359674343469556\\
6.55155155155155	0.000337942633587412\\
6.55855855855856	0.000317401882876873\\
6.56556556556557	0.000297994513890959\\
6.57257257257257	0.000279665282968216\\
6.57957957957958	0.000262361209954562\\
6.58658658658659	0.000246031508811951\\
6.59359359359359	0.000230627519166854\\
6.6006006006006	0.000216102638847736\\
6.60760760760761	0.000202412257457008\\
6.61461461461461	0.000189513691019234\\
6.62162162162162	0.000177366117743917\\
6.62862862862863	0.000165930514937545\\
6.63563563563564	0.000155169597096239\\
6.64264264264264	0.000145047755207028\\
6.64964964964965	0.000135530997282441\\
6.65665665665666	0.000126586890150044\\
6.66366366366366	0.000118184502515447\\
6.67067067067067	0.000110294349314343\\
6.67767767767768	0.000102888337366308\\
6.68468468468468	9.59397123403098e-05\\
6.69169169169169	8.94230070392485e-05\\
6.6986986986987	8.33139910082731e-05\\
6.70570570570571	7.75896214692138e-05\\
6.71271271271271	7.22279955811418e-05\\
6.71971971971972	6.72083040248042e-05\\
6.72672672672673	6.2510785906624e-05\\
6.73373373373373	5.81166849759287e-05\\
6.74074074074074	5.40082071471651e-05\\
6.74774774774775	5.01684793170971e-05\\
6.75475475475475	4.65815094652848e-05\\
6.76176176176176	4.32321480245718e-05\\
6.76876876876877	4.0106050506844e-05\\
6.77577577577578	3.71896413679559e-05\\
6.78278278278278	3.44700790944407e-05\\
6.78978978978979	3.19352224934763e-05\\
6.7967967967968	2.95735981664843e-05\\
6.8038038038038	2.73743691457872e-05\\
6.81081081081081	2.53273046728468e-05\\
6.81781781781782	2.34227510958218e-05\\
6.82482482482482	2.16516038634846e-05\\
6.83183183183183	2.00052805918926e-05\\
6.83883883883884	1.84756951796934e-05\\
6.84584584584585	1.70552329474673e-05\\
6.85285285285285	1.57367267761253e-05\\
6.85985985985986	1.45134342190728e-05\\
6.86686686686687	1.33790155625987e-05\\
6.87387387387387	1.23275128087741e-05\\
6.88088088088088	1.13533295550298e-05\\
6.88788788788789	1.04512117445278e-05\\
6.89489489489489	9.6162292614423e-06\\
6.9019019019019	8.84375834532141e-06\\
6.90890890890891	8.12946479880926e-06\\
6.91591591591592	7.46928796316185e-06\\
6.92292292292292	6.8594254361855e-06\\
6.92992992992993	6.2963185074715e-06\\
6.93693693693694	5.77663828607609e-06\\
6.94394394394394	5.29727249610973e-06\\
6.95095095095095	4.85531291604758e-06\\
6.95795795795796	4.44804343795029e-06\\
6.96496496496496	4.072928723188e-06\\
6.97197197197197	3.72760343168989e-06\\
6.97897897897898	3.40986200219459e-06\\
6.98598598598599	3.11764896144608e-06\\
6.99299299299299	2.84904974076965e-06\\
7	2.60228197896641e-06\\
};
\addlegendentry{$\mathcal{W}(4,5; 6,2)$}

\end{axis}
\end{tikzpicture}%
    \caption{Weibull $\sym{f}(\sym{t})$ für ausgewählte Werte von $\sym{T}$ und $\sym{b}$. }
    \label{fig:abb2.1_weibull}
\end{figure}

%%%%%%%%%%%%%%%%%%%%%%%%%%%%%%%%%%%%%%%%%%%%%%%%%%%%%%%%%%%%%%%%%%%%%%%%%%%%%%%%%%%%%%%%%%%%%%%%%%%%%%%%%%%%%%%%%%%%%%%%%%
%%%%%%%%%%%%%%%%%%%%%%%%%%%%%%%%%%%%%%%%%%%%%%%%%%%%%%%%%%%%%%%%%%%%%%%%%%%%%%%%%%%%%%%%%%%%%%%%%%%%%%%%%%%%%%%%%%%%%%%%%%
%%%%%%%%%%%%%%%%%%%%%%%%%%%%%%%%%%%%%%%%%%%%%%%%%%%%%%%%%%%%%%%%%%%%%%%%%%%%%%%%%%%%%%%%%%%%%%%%%%%%%%%%%%%%%%%%%%%%%%%%%%
%%%%%%%%%%%%%%%%%%%%%%%%%%%%%%%%%%%%%%%%%%%%%%%%%%%%%%%%%%%%%%%%%%%%%%%%%%%%%%%%%%%%%%%%%%%%%%%%%%%%%%%%%%%%%%%%%%%%%%%%%%
\subsection{Parameterschätzverfahren} \label{subsec:schätzer}
Soll eine geschlossene mathematische Beschreibung des stochastischen Ausfallverhaltens eines Produktes gefunden werden, ist das im vorherigen Abschnitten~\ref{subsec:stat} definierte parametrische Verteilungsmodell $\sym{tau} \sim \sym{W}(\sym{T}, \sym{b})$ zu schätzen.
Die Modellparameter der Grundgesamtheit sind in der praktischen Anwendung jedoch unbekannt.
Die zentrale Problemstellung der \textbf{Parameterschätzung} besteht somit darin, aus der empirischen Stichprobe bestehend aus $\sym{n}$ Realisierungen $\sym{t}_{1}, \dots, \symsub{t}{n}$ der Zufallsvariable $\sym{tau}$ statistisch fundierte Schätzwerte $\hat{\sym{T}}$ und $\hat{\sym{b}}$ zu gewinnen.
Diese sind Voraussetzung, um das Lebensdauermodell (z.B. Gleichung~\eqref{eq:weibull_cdf}) zu quantifizieren und prädiktive Aussagen zu Quantilen oder der Zuverlässigkeit $\sym{R}(\sym{t})$ zu ermöglichen.
Eine wesentliche Komplikation hierbei sind jedoch das mögliche Auftreten von unvollständigen bzw. \textbf{zensierten} Daten sowie \textit{multivariate} Abhängigkeiten der Belastungen zur Messgröße.
Während für die Schätzung von Verteilungsparametern einfache Verfahren, wie die \textbf{Momentenmethode} oder die \textbf{Methode der kleinsten Fehlerquadrate} - engl. \ac{OLS}, die beispielsweise bei der \ac{MMR} im Wahrscheinlichkeitsnetz Anwendung findet, existieren, sind diese für die umfassende Analyse vielschichtiger Lebensdauerdaten in der Regel unzureichend und hier nur der Vollständigkeit wegen erwähnt - vgl. \cite{Bertsche.2022,Montgomery.2021}.
Das universell anwendbare und robuste Verfahren, das Herausforderungen wie zensierte Daten und multivariate Modelle inhärent behandelt, ist die \ac{MLE} \cite{Meeker.2022,Nelson.1990}.

\subsubsection{Maximum-Likelihood-Estimation} \label{subsubsec:mle}
Das Grundprinzip der \ac{MLE} besteht darin, diejenigen Parameterwerte (z.B. $\hat{\sym{T}}, \hat{\sym{b}}$) als Schätzwerte auszuwählen, welche die Wahrscheinlichkeit (engl. Likelihood) maximieren, die empirisch beobachtete Stichprobe (bestehend aus unabhängigen Ausfällen und Zensierungen) zu erhalten.
Mathematisch wird die Wahrscheinlichkeit der Realisierung von $\sym{t_vec} = (\sym{t}_{1}, \dots, \symsub{t}{n})$ einer Stichprobe durch die \textbf{Likelihood-Funktion} $\sym{L_like}$ bestimmt. Diese ist eine Funktion des unbekannten Parametervektors $\sym{theta}$, der $\sym{k}$ zu schätzende Parameter enthält (z.B. $\sym{theta} = (\sym{T}, \sym{b})$ mit $\sym{k}=2$).\

Für den vereinfachten Fall, dass die Stichprobe ausschließlich aus $\sym{n}$ exakten Ausfallereignissen (vollständige Daten) besteht, ist die Likelihood-Funktion $\sym{L_like}$ das Produkt der einzelnen Wahrscheinlichkeitsdichten $\sym{f}(\cdot)$:
\begin{equation} \label{eq:mle_likelihood_simple}
    \sym{L_like}(\sym{t_vec} | \sym{theta}) = \prod_{\sym{i}=1}^{\sym{n}} \sym{f}(\symsub{t}{i} | \sym{theta}).
\end{equation}
Zur Vereinfachung der numerischen Berechnung wird in der Anwendung die \textbf{Log-Likelihood-Funktion} $\sym{Lambda}$ verwendet. Durch die Logarithmierung wird das Produkt (Gleichung~\eqref{eq:mle_likelihood_simple}) in eine äquivalente, leichter zu maximierende Summe überführt:
\begin{equation} \label{eq:mle_loglikelihood_simple}
    \sym{Lambda} \defeq \ln\left( \sym{L_like}(\sym{theta}) \right) = \sum_{\sym{i}=1}^{\sym{n}} \ln \left[ \sym{f}(\symsub{t}{i} | \sym{theta}) \right].
\end{equation}
Wie zuvor dargelegt, ist dieser vereinfachte Ansatz für Lebensdauerdaten jedoch oft unzureichend, da er das Auftreten von zensierten Daten vernachlässigt.
Für die praktische Anwendung existiert jedoch die entsprechende Erweiterung der Likelihood-Funktion um die Differenzierung etwaiger Testausgänge als \textit{Durchläufer}.
Dazu wird die Stichprobe als Paarung von $\symsub{t}{i}, \symsub{delta}{i}$ für $\sym{t_vec}$ definiert, wobei $\symsub{t}{i}$ der beobachteten Zeit und $\symsub{delta}{i}$ einem Statusindikator ($\symsub{delta}{i}=1$ für einen exakten Ausfall; $\symsub{delta}{i}=0$ für eine Rechts-Zensierung) entspricht \cite{Meeker.2022,Kalbfleisch.2002}.
$\sym{L_like}$ für rechts-zensierte Lebensdauerdaten lautet somit:
\begin{equation} \label{eq:mle_likelihood_censored}
    \sym{L_like}(\sym{t_vec} | \sym{theta}) = \prod_{\sym{i}=1}^{\sym{n}} \left[ \sym{f}(\symsub{t}{i} | \sym{theta})^{\symsub{delta}{i}} \cdot \sym{R}(\symsub{t}{i} | \sym{theta})^{1 - \symsub{delta}{i}} \right]
\end{equation}
und definiert die Log-Likelihood Funktion als:
\begin{equation} \label{eq:mle_loglikelihood_censored}
    \sym{Lambda} \defeq \ln\left(\sym{L_like}(\sym{t_vec} | \sym{theta})\right) = \sum_{\sym{i}=1}^{\sym{n}} \left[ \symsub{delta}{i} \cdot \ln \sym{f}(\symsub{t}{i} | \sym{theta}) + (1 - \symsub{delta}{i}) \cdot \ln \sym{R}(\symsub{t}{i} | \sym{theta}) \right].
\end{equation}

Der Parametervektor $\hat{\sym{theta}}$, der den Wert von $\sym{Lambda}(\sym{theta})$ maximiert, liefert die \ac{MLE}-Werte.
Die Schätzwerte repräsentieren die (asymptotisch) effizientesten Schätzwerte für die Parameter der Grundgesamtheit.
Dies erfolgt mathematisch durch Nullsetzen $\sym{k}$ partieller Ableitungen von $\sym{Lambda}$, sofern mathematisch entsprechende Schätzwerte in geschlossener Form durch $\partial\sym{Lambda}/\partial \sym{theta} \stackrel{!}{=} 0$ identifiziert werden können \cite{Nelson.2005,Rinne.2008}.
Andernfalls werden numerische Optimierungsalgorithmen, vgl. Newton-Raphson-Verfahren, Patternsearch und vergleichbare, dafür herangezogen - siehe weiterführend \cite{Nelson.2005,Qiao.1994} sowie detaillierte Untersuchungen von \textcite{Kremer.2019b}.
An dieser Stelle sei erwähnt, dass systematische Verzerrungen (engl. \textbf{Bias}) in $\hat{\sym{theta}}$ aufgrund kleiner Stichprobenumfänge auftreten können \cite{Abernethy.2006} - jedoch auch korrigierbar sind, vgl. Arbeiten von \textcite{Hirose.1999,Ross.1996}.

Die Qualität der Parameterschätzung beeinflusst daraus nicht nur die Prädiktionsgüte zur Schätzung der Lebensdauer oder Zuverlässigkeit - sie bedingt schließlich auch die Effizienz des Schätzverfahrens.
Wird im Sinne eines effizienten Verfahrens zur multivariaten Lebensdauermodellbildung eine Methodik gesucht, ist auch die Qualität der Parameterschätzung damit entscheidend.
Vertrauensbereiche, oder engl. \acp{CI}, können eine Metrik für die Qualität der Modellierung einnehmen, da sie die Unsicherheit oder \textit{Unschärfe} in der Prädiktion bemessen.


%%%%%%%%%%%%%%%%%%%%%%%%%%%%%%%%%%%%%%%%%%%%%%%%%%%%%%%%%%%%%%%%%%%%%%%%%%%%%%%%%%%%%%%%%%%%%%%%%%%%%%%%%%%%%%%%%%%%%%%%%%
%%%%%%%%%%%%%%%%%%%%%%%%%%%%%%%%%%%%%%%%%%%%%%%%%%%%%%%%%%%%%%%%%%%%%%%%%%%%%%%%%%%%%%%%%%%%%%%%%%%%%%%%%%%%%%%%%%%%%%%%%%
%%%%%%%%%%%%%%%%%%%%%%%%%%%%%%%%%%%%%%%%%%%%%%%%%%%%%%%%%%%%%%%%%%%%%%%%%%%%%%%%%%%%%%%%%%%%%%%%%%%%%%%%%%%%%%%%%%%%%%%%%%
%%%%%%%%%%%%%%%%%%%%%%%%%%%%%%%%%%%%%%%%%%%%%%%%%%%%%%%%%%%%%%%%%%%%%%%%%%%%%%%%%%%%%%%%%%%%%%%%%%%%%%%%%%%%%%%%%%%%%%%%%%\subsubsection{Vertrauensbereiche} \label{subsubsec:ci}
Die \ac{MLE} liefert nicht nur die Punktschätzer $\hat{\sym{theta}}$, sondern auch die Quantifizierung von deren statistischer Unsicherheit (Präzision).
Obwohl verschiedene Ansätze, wie die numerisch anspruchsvolleren Berechnungen nach Likelihood-Ratio-Methode, Bootstrap-Perzentil-Methode oder Monte-Carlo-Approximation existieren, ist das gängigste Verfahren zur Berechnung von \acp{CI} die Approximation mittels asymptotischer Normalverteilung der \ac{MLE}-Schätzer $\hat{\sym{theta}}$ \cite{Bertsche.2022,Nelson.1990}.
Dies erfolgt über die \textbf{Fisher-Informationsmatrix} $\sym{FIM}$, welche die Information der Stichprobe über die Parameter $\sym{theta}$ gemäß \textcite{Kremer.2019c}  bezüglich des Rechenaufwands und resultierender Modellqualität vergleichsweise effizient quantifiziert.
So wird diese Methodik auch in gängiger Applikationen als Standard angewandt, vgl. \cite{Nelson.2005,Nelson.1990,Yang.2007}.
In der praktischen Anwendung wird die Fisher-Informationsmatrix auf Basis der resultierenden Schätzwerte $\hat{\sym{theta}} = \sym{theta}$ so als Schätzung zur Beobachtung nach $\symsub{FIM}{O}$ verwendet \cite{Nelson.1990,Lawless.2003}.
Diese ist definiert als die negative \textbf{Hesse-Matrix} $\sym{H}$ der Log-Likelihood-Funktion, ausgewertet an der Stelle der \ac{MLE}-Schätzwerte
$\hat{\sym{theta}}$:
\begin{equation} \label{eq:fim_o}
    \symsub{FIM}{O} \defeq - \sym{H}(\hat{\sym{theta}}) = - \left[ \frac{\partial^2 \sym{Lambda}(\sym{theta})}{\partial \symsub{theta}{j} \partial \symsub{theta}{l}} \right]_{\sym{theta} = \hat{\sym{theta}}}.
\end{equation}
Die Matrix $\sym{H}$ entspricht der ($\sym{k} \times \sym{k}$)-Matrix der $\sym{k}$ zweiten partiellen Ableitungen von $\sym{Lambda}$ (vgl. Gl.~\eqref{eq:mle_loglikelihood_censored}).
Eine Invertierung ${\hat{\sym{FIM}}_{\sym{O}}}^{-1}$ ergibt die geschätzte \textbf{Varianz-Kovarianz-Matrix} $\hat{\sym{V}}$:
\begin{equation} \label{eq:var_covar}
    \hat{\sym{V}} \approx {\hat{\sym{FIM}}_{\sym{O}}}^{-1}.
\end{equation}
Die Diagonalelemente dieser Matrix $\hat{\sym{V}}_{\sym{j}\sym{j}}$ entsprechen den Varianzen $\sym{Var}(\hat{\sym{theta}}_{\sym{j}})$ der einzelnen Parameterschätzwerte \cite{Nelson.1990}.
Die Nicht-Diagonalelemente $\hat{\sym{V}}_{\sym{j}\sym{l}}$ (für $\sym{j} \neq \sym{l}$) repräsentieren die \textbf{Kovarianzen} $\sym{Cov}(\hat{\sym{theta}}_{\sym{j}}, \hat{\sym{theta}}_{\sym{l}})$ \cite{Nelson.1990,Yang.2007}.
Diese Kovarianzen sind von entscheidender Bedeutung, da sie die statistische Abhängigkeit zwischen den Schätzwerten (z.B. zwischen $\hat{\sym{T}}$ und $\hat{\sym{b}}$) quantifizieren, welche für die Berechnung der \acp{CI} von abgeleiteten Funktionen wie $\hat{\sym{R}}(\sym{t})$ erforderlich sind \cite{Meeker.2022}.
Basierend auf der Annahme der asymptotischen Normalität der Schätzer wird ein zweiseitiges $(1-\sym{alpha})$-\ac{CI} für einen einzelnen Parameter $\hat{\sym{theta}}_{\sym{j}}$ direkt aus dessen Varianz approximiert durch:
\begin{equation} \label{eq:ci_normal}
    [\sym{theta}_{\sym{j},\sym{u}},\sym{theta}_{\sym{j},\sym{o}}]  = \hat{\sym{theta}}_{\sym{j}} \pm \sym{z}_{1-\sym{alpha}/2} \cdot \sqrt{\hat{\sym{V}}_{\sym{j}\sym{j}}},
\end{equation}
wobei $\sym{z}_{1-\sym{alpha}/2}$ dem $(1-\sym{alpha}/2)$-Quantil der Standardnormalverteilung entspricht.
Da Lebensdauerparameter (z.B. $\sym{T}, \sym{b}$) oft auf $\sym{RR}_{>0}$ beschränkt sind, werden \acp{CI} robust über eine Log-Transformation der Parameter berechnet, um physikalisch unmögliche (negative) Intervallgrenzen zu vermeiden \cite{Meeker.2022,Yang.2007,Nelson.1990}:
\begin{equation} \label{eq:ci_ln}
    [\sym{theta}_{\sym{j},\sym{u}}, \sym{theta}_{\sym{j},\sym{o}}] = \hat{\sym{theta}}_{\sym{j}} \exp \left( \pm \sym{z}_{1-\sym{alpha}/2} \cdot \frac{\sqrt{\hat{\sym{V}}_{\sym{j}\sym{j}}}}{\hat{\sym{theta}}_{\sym{j}}} \right).
\end{equation}

Die Berechnung dieser \acp{CI} für Schätzwerte $\sym{g_func}(\hat{\sym{theta}})$ zu Größen wie $\hat{\sym{R}}(\sym{t})$ oder $\hat{\sym{t}}_{\sym{q}}$  erfolgt mittels \textbf{Delta-Methode} \cite{Nelson.1990,Meeker.2022}.
Dieses auf einer Taylor-Reihenentwicklung basierende Verfahren (Gauß'sche Fehlerfortpflanzung) approximiert die Varianz der Funktion $\hat{\sym{g_func}}=\sym{g_func}(\hat{\sym{theta}})$ unter Einbeziehung der gesamten Varianz-Kovarianz-Matrix.
Dazu wird der \textbf{Gradientenvektor} $\sym{g_grad}$ der Funktion $\sym{g_func}$ (z.B. $\sym{g_func} = \sym{R}(\sym{t})$) bezüglich des $\sym{k}$-dimensionalen Parametervektors $\sym{theta}$ gebildet:
\begin{equation} \label{eq:delta_method_gradient}
    \sym{g_grad} \defeq \left[ \frac{\partial \sym{g_func}(\sym{theta})}{\partial \sym{theta}_{1}}, \dots, \frac{\partial \sym{g_func}(\sym{theta})}{\partial \symsub{theta}{k}} \right]^T_{\sym{theta} = \hat{\sym{theta}}}.
\end{equation}
Die approximierte Varianz $\sym{Var}(\hat{\sym{g_func}})$ der Funktion ergibt sich dann aus:
\begin{equation} \label{eq:delta_method_variance}
    \sym{Var}(\hat{\sym{g_func}}) \approx \sym{g_grad}^T \hat{\sym{V}} \sym{g_grad}.
\end{equation}
Das Vertrauensintervall für die Funktion $\hat{\sym{g_func}}$ wird anschließend unter Verwendung dieser Varianz (bzw. des Standardfehlers $\sqrt{\sym{Var}(\hat{\sym{g_func}})}$) analog zu Gleichung~\eqref{eq:ci_normal} berechnet \cite{Bain.2017b,Yang.2007,Nelson.1990}:
\begin{equation} \label{eq:ci_rel}
    [\sym{g_func}_{\sym{u}},\sym{g_func}_{\sym{o}}]  = \hat{\sym{g_func}} \pm \sym{z}_{1-\sym{alpha}/2} \sqrt{\sym{Var}(\hat{\sym{g_func}})}.
\end{equation}
Sollen auch hier nur positive Werte für $\sym{g_func}$ berücksichtigt werden, kann eine Logarithmierung in der Berechnung der Vertrauensbereiche analog zu Gl.~\ref{eq:ci_ln} erfolgen \cite{Yang.2007}.

\section{Statistische Versuchsplanung und Modellbildung} \label{sec:doe}
Multivariate Lebensdauertests erfordern definitionsgemäß die Betrachtung mehrerer $\sym{k}\geq 2$ Einflussfaktoren als Versuchsparameter.
Dementsprechend entscheidend ist das Verständnis der wesentlichen Grundlagen im Umgang mit statistischer Versuchsplanung (\ac{DoE}) für Lebensdauerdaten - auch unter dem Begriff \ac{L-DoE} zusammengefasst - sowie der darauffolgenden Lebensdauermodellbildung.
Während detaillierte Übersichten zur Historie von \ac{DoE} von anfänglichen einschlägigen Beschreibungen durch \textcite{Fisher.1935}, außerdem maßgebliche Weiterentwicklungen durch \textcite{Box.1978} oder evolutionäre Schritte durch \textcite{Taguchi.2005} ausgiebig in Werken von \textcite{Kleppmann.2016,Rigdon.2022,Montgomery.2020} beschrieben sind, wird im Folgenden auf die wesentlichen Inhalte für die Forschungsschwerpunkte eingegangen.

Der primäre Anspruch von \ac{DoE} besteht in der effizienten Planung empirischer Datenerhebungen, um Zielgrößen in Abhängigkeit erklärender Variablen zunächst robust zu modellieren und schließlich zu optimieren.
Dieses Paradigma lässt sich auch unter der Begrifflichkeit \ac{DfR} unmittelbar wiedererkennen und so auf die Analyse von Lebensdauer und Zuverlässigkeit übertragen \cite{Yang.2007,Wu.2021}.
Da das lokale oder globale Optimum der Lebensdauer- bzw. Zuverlässigkeitsfunktion eines Produktes a priori meist unbekannt ist, erfordert dessen Identifikation eine systematische Exploration des Parameterraumes.
Eine besondere Herausforderung stellt hierbei zusätzlich die Integration von \ac{ALT} dar: Die Diskrepanz zwischen dem hochbelasteten Testraum (engl. Design Space) und dem regulären Prädiktionsraum (engl. Use Space oder Field Space) kann eine Extrapolation erforderlich machen, welche die Anforderungen an die Daten- und somit auch an die Designqualität deutlich verschärft.
Das Unwissen zur tatsächlichen Lage optimaler Antwortwerte und die Möglichkeit, per se einen systematischen Offset zwischen Design- und Field-Space durch \ac{ALT} vorzufinden, stellen Teststrategien nach Best-Guess Ansätzen nachteilig.
Hier wird in der industriellen Praxis häufig fälschlicherweise ein \ac{OFAT}-Testing Ansatz gewählt - unabhängig, ob vom Vorhandensein von Lebensdauer- oder anderen Daten, welcher schlichtweg die Wahrscheinlichkeit, Optimalstellen im Parameterraum systematisch zu treffen, senkt und somit gegenüber \ac{L-DoE} nachteilig ist, vgl. \cite{Montgomery.2020,Siebertz.2017}.
Da nun die geometrische Struktur eines Versuchsplans die erreichbare Modellierungsqualität deterministisch begrenzt, ist eine präzise Bewertung der Plangüte anhand genau dieser Eigenschaft im Vorfeld unerlässlich.
Hierfür können objektive Performance-Indikatoren sowie mathematische Optimalitätskriterien dienen, vgl. \cite{Montgomery.2020,Goos.2011}.
Ergänzend zu rationalen Metriken wie der statistischen Trennschärfe (engl. Power) und dem Schätzergebnis einer Koeffizienten- bzw. Parameterschätzung sind diese Größen damit bestimmend für effiziente multivariate Lebensdauertests.

Vor diesem Hintergrund fokussiert sich dieser Abschnitt auf eine gezielte Auswahl an Grundbegriffen und Metriken für multivariate Testpläne im Kontext von \ac{L-DoE} sowie auf eine Übersicht der für Lebensdauertests geeigneten Versuchspläne, bevor abschließend die statistische Modellbildung beleuchtet wird.

Für eine grundsätzlichere Auseinandersetzung mit konventionellen Methoden und Werkzeugen von \ac{DoE} sei, mit Blick auf den Fokus der vorliegenden Arbeit, hingegen auf die einschlägige Literatur von \textcite{Kleppmann.2016}, \textcite{Siebertz.2017}, \textcite{Hinkelmann.2012} sowie vornehmlich \textcite{Montgomery.2020} und \textcite{Myers.2016} verwiesen.
Diese Werke behandeln intensiv die Inhalte grundsätzlicher statistischer Versuchsplanung, welche um Perspektiven zu \ac{L-DoE} bereits durch am \ac{IMA} entstandene Dissertationen von \textcite{Dazer.2019}, \textcite{Herzig.2021}, \textcite{Grundler.2024} und maßgeblich durch \textcite{Kremer.2021} fortschreitend ergänzt wurden.
Konsequenterweise werden Hintergründe zum Umgang mit normalverteilten Daten oder Abweichungen davon im Rahmen des \ac{DoE}, die Diskussion zu einschlägigen Vor- und Nachteilen auch unter Abgrenzung zu Alternativen wie \ac{OFAT}, die Regressionsmodellierung auf Basis der \ac{ANOVA}, konventionelle Hypothesentests sowie fundamentale Ausführungen zu \ac{ALT} in den nachfolgenden Ausführungen nicht explizit betrachtet, sondern als bekannt vorausgesetzt.

%%%%%%%%%%%%%%%%%%%%%%%%%%%%%%%%%%%%%%%%%%%%%%%%%%%%%%%%%%%%%%%%%%%%%%%%%%%%%%%%%%%%%%%%%%%%%%%%%%%%%%%%%%%%%%%%%%%%%%%%%%
%%%%%%%%%%%%%%%%%%%%%%%%%%%%%%%%%%%%%%%%%%%%%%%%%%%%%%%%%%%%%%%%%%%%%%%%%%%%%%%%%%%%%%%%%%%%%%%%%%%%%%%%%%%%%%%%%%%%%%%%%%
%%%%%%%%%%%%%%%%%%%%%%%%%%%%%%%%%%%%%%%%%%%%%%%%%%%%%%%%%%%%%%%%%%%%%%%%%%%%%%%%%%%%%%%%%%%%%%%%%%%%%%%%%%%%%%%%%%%%%%%%%%
%%%%%%%%%%%%%%%%%%%%%%%%%%%%%%%%%%%%%%%%%%%%%%%%%%%%%%%%%%%%%%%%%%%%%%%%%%%%%%%%%%%%%%%%%%%%%%%%%%%%%%%%%%%%%%%%%%%%%%%%%%
\subsection{Grundlagen zur statistischen Versuchsplanung} \label{subsec:begriffedoe}
Die Anwendung von \ac{DoE} versteht sich grundsätzlich als Verfahrenskette \cite{Coleman.1993,Montgomery.2020} entlang mehrerer Prozessschritte, die beginnend von einer spezifischen Aufgabendefinition in einer statistisch abgesicherten Testentscheidung und Datenmodellierung mündet (vgl. \ref{fig:ma_abb2.02_doe_steps}).
\begin{figure}[htbp]
    \centering
    \def\svgwidth{0.9\textwidth}
    %LaTeX with PSTricks extensions
%%Creator: Inkscape 1.3.2 (091e20e, 2023-11-25, custom)
%%Please note this file requires PSTricks extensions
\psset{xunit=.5pt,yunit=.5pt,runit=.5pt}
\begin{pspicture}(465.48268476,147.95959052)
{
\newrgbcolor{curcolor}{0.15686275 0.16078432 0.16470589}
\pscustom[linewidth=9.3718123,linecolor=curcolor]
{
\newpath
\moveto(11.89911187,142.81529572)
\lineto(12.00505852,12.85753841)
}
}
{
\newrgbcolor{curcolor}{0.15686275 0.16078432 0.16470589}
\pscustom[linestyle=none,fillstyle=solid,fillcolor=curcolor]
{
\newpath
\moveto(2.39898387,16.35425515)
\lineto(11.974196,3.95133087)
\lineto(21.52917062,16.36985151)
\curveto(15.88356034,14.38253314)(8.15716482,14.38765873)(2.39898387,16.35425515)
\closepath
}
}
{
\newrgbcolor{curcolor}{0.15686275 0.16078432 0.16470589}
\pscustom[linewidth=1.02779197,linecolor=curcolor]
{
\newpath
\moveto(2.39898387,16.35425515)
\lineto(11.974196,3.95133087)
\lineto(21.52917062,16.36985151)
\curveto(15.88356034,14.38253314)(8.15716482,14.38765873)(2.39898387,16.35425515)
\closepath
}
}
{
\newrgbcolor{curcolor}{1 1 1}
\pscustom[linestyle=none,fillstyle=solid,fillcolor=curcolor]
{
\newpath
\moveto(27.3012357,138.95873886)
\curveto(27.3012357,134.7839884)(23.91693177,131.39968447)(19.74218131,131.39968447)
\curveto(15.56743084,131.39968447)(12.18312691,134.7839884)(12.18312691,138.95873886)
\curveto(12.18312691,143.13348933)(15.56743084,146.51779326)(19.74218131,146.51779326)
\curveto(23.91693177,146.51779326)(27.3012357,143.13348933)(27.3012357,138.95873886)
\closepath
}
}
{
\newrgbcolor{curcolor}{0.15686275 0.16078432 0.16470589}
\pscustom[linewidth=2.88359956,linecolor=curcolor]
{
\newpath
\moveto(27.3012357,138.95873886)
\curveto(27.3012357,134.7839884)(23.91693177,131.39968447)(19.74218131,131.39968447)
\curveto(15.56743084,131.39968447)(12.18312691,134.7839884)(12.18312691,138.95873886)
\curveto(12.18312691,143.13348933)(15.56743084,146.51779326)(19.74218131,146.51779326)
\curveto(23.91693177,146.51779326)(27.3012357,143.13348933)(27.3012357,138.95873886)
\closepath
}
}
{
\newrgbcolor{curcolor}{0 0 0}
\pscustom[linestyle=none,fillstyle=solid,fillcolor=curcolor]
{
\newpath
\moveto(23.11718089,133.57983328)
\lineto(23.0312434,133.48608329)
\curveto(22.30207682,133.53295828)(21.41145193,133.55639578)(20.35936873,133.55639578)
\curveto(19.33332719,133.55639578)(18.25259816,133.53295828)(17.11718163,133.48608329)
\lineto(17.03124414,133.57202078)
\lineto(17.03124414,134.04858322)
\lineto(17.12499413,134.12670821)
\curveto(17.3906191,134.15795821)(17.64843156,134.18399987)(17.89843153,134.2048332)
\curveto(18.34634814,134.24129153)(18.63280644,134.28556236)(18.75780643,134.33764568)
\curveto(18.84634808,134.37931234)(18.92186891,134.4756665)(18.9843689,134.62670815)
\curveto(19.08332722,134.87149978)(19.13280638,135.861083)(19.13280638,137.59545778)
\lineto(19.13280638,139.54077004)
\curveto(19.13280638,140.1241033)(19.12238971,140.86368655)(19.10155638,141.75951977)
\lineto(19.08593139,142.3454572)
\curveto(19.08593139,142.60587383)(19.03124389,142.73608215)(18.92186891,142.73608215)
\curveto(18.87499391,142.73608215)(18.7213481,142.67618632)(18.46093146,142.55639467)
\curveto(17.74738988,142.22826971)(17.13541079,141.93920725)(16.62499419,141.68920728)
\lineto(16.5312442,141.74389477)
\curveto(16.4999942,142.04076973)(16.44530671,142.37149886)(16.36718172,142.73608215)
\lineto(16.37499422,142.79858214)
\curveto(17.81249404,143.20483209)(19.41926468,143.74910286)(21.19530613,144.43139444)
\lineto(21.35936861,144.31420695)
\curveto(21.33853527,144.00691532)(21.30988945,143.55118621)(21.27343112,142.94701962)
\curveto(21.25780612,142.71785298)(21.24218112,142.10326973)(21.22655612,141.10326985)
\lineto(21.20311862,138.03295773)
\curveto(21.20311862,135.83504133)(21.23957695,134.65535398)(21.31249361,134.49389566)
\curveto(21.36457694,134.37410401)(21.44270193,134.30118735)(21.54686858,134.27514569)
\curveto(21.73436856,134.22306236)(22.22395183,134.17618737)(23.0156184,134.13452071)
\lineto(23.11718089,134.05639572)
\closepath
}
}
{
\newrgbcolor{curcolor}{0 0 0}
\pscustom[linestyle=none,fillstyle=solid,fillcolor=curcolor]
{
\newpath
\moveto(34.0374422,144.99039502)
\lineto(34.09994219,145.04508251)
\curveto(35.00098374,145.00341585)(35.615567,144.98258252)(35.94369196,144.98258252)
\lineto(39.59994151,145.06070751)
\curveto(40.64160805,145.06070751)(41.49577461,144.96174919)(42.1624412,144.76383255)
\curveto(42.82910778,144.57112424)(43.42025354,144.23258261)(43.93587848,143.74820767)
\curveto(44.45671175,143.26904107)(44.84994086,142.7091453)(45.11556583,142.06852038)
\curveto(45.38639913,141.43310379)(45.52181578,140.75341638)(45.52181578,140.02945813)
\curveto(45.52181578,139.32633322)(45.39941996,138.6310208)(45.15462833,137.94352089)
\curveto(44.90983669,137.25602097)(44.5608784,136.63883355)(44.10775346,136.09195862)
\curveto(43.65983684,135.55029202)(43.14681607,135.11279207)(42.56869115,134.77945878)
\curveto(41.99577455,134.44612549)(41.42025379,134.22216718)(40.84212886,134.10758386)
\curveto(40.26400393,133.99820888)(39.67285817,133.94352138)(39.06869158,133.94352138)
\curveto(38.66244163,133.94352138)(37.89681672,133.95914638)(36.77181686,133.99039638)
\curveto(36.40202524,134.00081304)(36.1572336,134.00602138)(36.03744195,134.00602138)
\curveto(35.67806699,134.00602138)(35.24317122,133.98518804)(34.73275461,133.94352138)
\lineto(34.67025462,134.00602138)
\lineto(34.67025462,134.26383384)
\lineto(34.73275461,134.34195883)
\curveto(35.04004624,134.49820881)(35.21452539,134.59456297)(35.25619205,134.6310213)
\curveto(35.30306704,134.67268796)(35.33952537,135.12841707)(35.36556703,135.99820863)
\curveto(35.39681703,136.87320852)(35.41244203,137.53987511)(35.41244203,137.99820838)
\lineto(35.41244203,140.86539553)
\lineto(35.40462953,142.63883281)
\curveto(35.40462953,143.02424943)(35.3994212,143.33674939)(35.38900453,143.5763327)
\curveto(35.37858787,143.82112433)(35.3603587,143.99560348)(35.33431704,144.09977013)
\curveto(35.30827537,144.20393678)(35.27702538,144.27685344)(35.24056705,144.3185201)
\curveto(35.20931705,144.36018677)(35.15462956,144.39404093)(35.07650457,144.42008259)
\curveto(35.00358791,144.45133259)(34.87598376,144.47737425)(34.69369212,144.49820758)
\lineto(34.09994219,144.54508258)
\lineto(34.0374422,144.59977007)
\closepath
\moveto(36.92806684,134.79508378)
\curveto(37.45931678,134.69091712)(38.07129587,134.6388338)(38.76400411,134.6388338)
\curveto(39.97754563,134.6388338)(40.94629551,134.84195877)(41.67025376,135.24820872)
\curveto(42.39942033,135.65445867)(42.9436911,136.2560211)(43.30306605,137.052896)
\curveto(43.66244101,137.8497709)(43.84212849,138.74820829)(43.84212849,139.74820817)
\curveto(43.84212849,141.22216632)(43.44889937,142.36539534)(42.66244113,143.17789524)
\curveto(41.8759829,143.99039514)(40.63639972,144.39664509)(38.94369159,144.39664509)
\curveto(38.24577501,144.39664509)(37.57390009,144.34456177)(36.92806684,144.24039511)
\curveto(36.88640018,143.38622855)(36.86556685,142.422687)(36.86556685,141.34977047)
\lineto(36.86556685,138.89664577)
\lineto(36.88119185,136.59977106)
\curveto(36.88119185,136.15185444)(36.89681684,135.55029202)(36.92806684,134.79508378)
\closepath
}
}
{
\newrgbcolor{curcolor}{0 0 0}
\pscustom[linestyle=none,fillstyle=solid,fillcolor=curcolor]
{
\newpath
\moveto(53.09212735,135.15445873)
\lineto(52.84212738,134.5919588)
\curveto(52.30046078,134.24300051)(51.81608584,134.01643804)(51.38900256,133.91227139)
\curveto(50.96712761,133.80810473)(50.58952349,133.75602141)(50.2561902,133.75602141)
\curveto(49.63119027,133.75602141)(49.03744035,133.87841722)(48.47494042,134.12320886)
\curveto(47.91764882,134.3680005)(47.46191971,134.78206295)(47.10775309,135.36539621)
\curveto(46.7587948,135.94872947)(46.58431565,136.65185438)(46.58431565,137.47477095)
\curveto(46.58431565,138.02164588)(46.65202397,138.51383332)(46.78744062,138.95133327)
\curveto(46.92285727,139.39404154)(47.06348226,139.7221665)(47.20931557,139.93570814)
\curveto(47.36035722,140.14924978)(47.61296136,140.38622892)(47.96712798,140.64664556)
\curveto(48.3212946,140.90706219)(48.69629456,141.1153955)(49.09212784,141.27164548)
\curveto(49.48796113,141.42789546)(49.91504441,141.50602045)(50.37337768,141.50602045)
\curveto(50.99837761,141.50602045)(51.5478567,141.35758297)(52.02181498,141.06070801)
\curveto(52.50098159,140.76904137)(52.83691905,140.39404142)(53.02962736,139.93570814)
\curveto(53.22233566,139.47737487)(53.31868982,138.99039576)(53.31868982,138.47477082)
\curveto(53.31868982,138.31331251)(53.31087732,138.15706253)(53.29525232,138.00602088)
\lineto(53.20931483,137.92008339)
\curveto(52.85514821,137.8419584)(52.37858577,137.78987508)(51.77962751,137.76383341)
\curveto(51.18066925,137.73779175)(50.78483597,137.72477092)(50.59212766,137.72477092)
\lineto(48.08431547,137.72477092)
\curveto(48.09473213,136.64664605)(48.36556543,135.85237531)(48.89681536,135.34195871)
\curveto(49.4280653,134.83154211)(50.07910689,134.5763338)(50.84994012,134.5763338)
\curveto(51.21452341,134.5763338)(51.5634817,134.6388338)(51.89681499,134.76383378)
\curveto(52.23535662,134.88883377)(52.59212741,135.06331291)(52.96712736,135.28727122)
\closepath
\moveto(48.08431547,138.34977084)
\curveto(48.17806545,138.33414584)(48.53744041,138.31591668)(49.16244033,138.29508335)
\curveto(49.79264859,138.27425002)(50.25879436,138.26383335)(50.56087766,138.26383335)
\curveto(51.2848359,138.26383335)(51.72494002,138.27685418)(51.88119,138.30289585)
\curveto(51.88639833,138.42789583)(51.8890025,138.52424998)(51.8890025,138.59195831)
\curveto(51.8890025,139.39924988)(51.72494002,139.99820814)(51.39681506,140.38883309)
\curveto(51.0686901,140.78466637)(50.62077349,140.98258301)(50.05306522,140.98258301)
\curveto(49.43327363,140.98258301)(48.94889869,140.76122888)(48.5999404,140.3185206)
\curveto(48.25619044,139.87581232)(48.08431547,139.2195624)(48.08431547,138.34977084)
\closepath
\moveto(50.2640027,141.9435204)
\closepath
\moveto(50.17025271,133.53727143)
\closepath
}
}
{
\newrgbcolor{curcolor}{0 0 0}
\pscustom[linestyle=none,fillstyle=solid,fillcolor=curcolor]
{
\newpath
\moveto(56.82650189,140.92008302)
\lineto(58.76400165,140.92008302)
\curveto(59.4254599,140.92008302)(59.95670983,140.95654135)(60.35775145,141.02945801)
\curveto(60.7640014,141.107583)(61.24056384,141.24820798)(61.78743878,141.45133296)
\lineto(61.92806376,141.34977047)
\curveto(61.87598043,140.66747889)(61.84993877,139.80029149)(61.84993877,138.74820829)
\lineto(61.84993877,136.98258351)
\curveto(61.84993877,136.86800019)(61.86035543,136.49820857)(61.88118876,135.87320864)
\curveto(61.90202209,135.25341705)(61.92285543,134.89143793)(61.94368876,134.78727128)
\curveto(61.96973042,134.68310462)(62.00618875,134.6075838)(62.05306374,134.56070881)
\curveto(62.09993874,134.51383381)(62.1572304,134.48258382)(62.22493872,134.46695882)
\curveto(62.29264705,134.45654215)(62.57389701,134.43570882)(63.06868862,134.40445883)
\lineto(63.13900111,134.34195883)
\lineto(63.13900111,134.01383387)
\lineto(63.06868862,133.94352138)
\curveto(62.42285536,133.98518804)(61.79785544,134.00602138)(61.19368885,134.00602138)
\curveto(60.63118892,134.00602138)(60.006189,133.98518804)(59.31868908,133.94352138)
\lineto(59.24837659,134.01383387)
\lineto(59.24837659,134.34195883)
\lineto(59.31868908,134.40445883)
\curveto(59.81868902,134.43570882)(60.10254315,134.45914632)(60.17025147,134.47477132)
\curveto(60.2379598,134.49039632)(60.29525146,134.52164631)(60.34212645,134.56852131)
\curveto(60.39420978,134.62060463)(60.42806394,134.69612546)(60.44368894,134.79508378)
\curveto(60.46452227,134.89925043)(60.4853556,135.23518789)(60.50618893,135.80289615)
\curveto(60.52702226,136.37581275)(60.53743893,136.80810436)(60.53743893,137.09977099)
\lineto(60.53743893,138.6388333)
\curveto(60.53743893,138.85237494)(60.52702226,139.16747907)(60.50618893,139.58414569)
\curveto(60.4853556,140.0008123)(60.45931394,140.22737477)(60.42806394,140.2638331)
\curveto(60.39681395,140.30029143)(60.35254312,140.33154143)(60.29525146,140.35758309)
\curveto(60.24316813,140.38883309)(60.02181399,140.40445809)(59.63118904,140.40445809)
\lineto(58.87337663,140.39664559)
\lineto(58.82650164,140.17008312)
\lineto(58.74837665,140.13102062)
\lineto(56.82650189,140.13102062)
\lineto(56.82650189,136.98258351)
\curveto(56.82650189,136.81591686)(56.83691855,136.43310441)(56.85775188,135.83414615)
\curveto(56.88379355,135.24039622)(56.90723104,134.89143793)(56.92806437,134.78727128)
\curveto(56.94889771,134.68310462)(56.98275187,134.6075838)(57.02962686,134.56070881)
\curveto(57.08171019,134.51383381)(57.16243935,134.47997965)(57.27181433,134.45914632)
\curveto(57.38639765,134.43831299)(57.68066845,134.42008382)(58.15462672,134.40445883)
\lineto(58.21712672,134.34195883)
\lineto(58.21712672,134.01383387)
\lineto(58.15462672,133.94352138)
\curveto(57.98275174,133.94872972)(57.74316844,133.95914638)(57.43587681,133.97477138)
\curveto(57.02962686,133.99560471)(56.61035608,134.00602138)(56.17806447,134.00602138)
\curveto(55.61556454,134.00602138)(54.98796045,133.98518804)(54.2952522,133.94352138)
\lineto(54.23275221,134.01383387)
\lineto(54.23275221,134.34195883)
\lineto(54.2952522,134.40445883)
\curveto(54.80046047,134.43570882)(55.08691877,134.45914632)(55.15462709,134.47477132)
\curveto(55.22233542,134.49039632)(55.27962708,134.52164631)(55.32650207,134.56852131)
\curveto(55.37337707,134.62060463)(55.40462706,134.69612546)(55.42025206,134.79508378)
\curveto(55.44108539,134.89925043)(55.46191872,135.23518789)(55.48275205,135.80289615)
\curveto(55.50879372,136.37581275)(55.52181455,136.80810436)(55.52181455,137.09977099)
\lineto(55.52181455,140.11539562)
\lineto(54.51400217,140.06070813)
\lineto(54.44368968,140.12320812)
\lineto(54.44368968,140.3185206)
\lineto(54.49837717,140.39664559)
\lineto(55.52181455,140.92008302)
\lineto(55.52181455,141.38883296)
\curveto(55.52181455,141.90966623)(55.54525205,142.30289535)(55.59212704,142.56852032)
\curveto(55.63900203,142.83935362)(55.72493952,143.08674942)(55.84993951,143.31070773)
\curveto(55.98014782,143.53987437)(56.20410613,143.82633266)(56.52181442,144.17008262)
\curveto(56.83952272,144.51904091)(57.11816852,144.80289504)(57.35775182,145.02164502)
\curveto(57.59733513,145.24560332)(57.80566843,145.3940408)(57.98275174,145.46695746)
\curveto(58.15983506,145.54508245)(58.36816836,145.58414495)(58.60775167,145.58414495)
\curveto(58.80566831,145.58414495)(59.01920995,145.54508245)(59.24837659,145.46695746)
\lineto(59.24056409,144.18570762)
\lineto(59.06087661,144.11539513)
\curveto(58.76920998,144.37581176)(58.43066836,144.50602008)(58.04525174,144.50602008)
\curveto(57.77441844,144.50602008)(57.54004347,144.44872842)(57.34212682,144.3341451)
\curveto(57.14421018,144.22477012)(57.00879353,144.04247847)(56.93587687,143.78727017)
\curveto(56.86296022,143.5372702)(56.82650189,143.12060358)(56.82650189,142.53727032)
\closepath
\moveto(61.12337636,144.94352003)
\curveto(61.36295966,144.94352003)(61.5686888,144.85758254)(61.74056378,144.68570756)
\curveto(61.91243876,144.51904091)(61.99837625,144.3107076)(61.99837625,144.06070764)
\curveto(61.99837625,143.82112433)(61.91243876,143.61539519)(61.74056378,143.44352021)
\curveto(61.57389714,143.27164523)(61.36816799,143.18570774)(61.12337636,143.18570774)
\curveto(60.87858472,143.18570774)(60.67025141,143.26904107)(60.49837643,143.43570771)
\curveto(60.32650146,143.60758269)(60.24056397,143.815916)(60.24056397,144.06070764)
\curveto(60.24056397,144.30549927)(60.32650146,144.51383258)(60.49837643,144.68570756)
\curveto(60.67545975,144.85758254)(60.88379305,144.94352003)(61.12337636,144.94352003)
\closepath
}
}
{
\newrgbcolor{curcolor}{0 0 0}
\pscustom[linestyle=none,fillstyle=solid,fillcolor=curcolor]
{
\newpath
\moveto(66.20150073,141.45133296)
\lineto(66.34993821,141.34977047)
\curveto(66.31868822,141.00081218)(66.29785489,140.5424789)(66.28743822,139.97477064)
\curveto(66.70410484,140.3185206)(67.09993812,140.66747889)(67.47493807,141.02164551)
\curveto(67.58431306,141.12060383)(67.68587555,141.19352049)(67.77962554,141.24039548)
\curveto(67.87858386,141.28727048)(68.04004217,141.33674964)(68.26400048,141.38883296)
\curveto(68.48795878,141.44091629)(68.71712542,141.46695796)(68.95150039,141.46695796)
\curveto(69.34733368,141.46695796)(69.73014613,141.3862288)(70.09993775,141.22477048)
\curveto(70.4749377,141.06331217)(70.75618767,140.87060386)(70.94368765,140.64664556)
\curveto(71.13639596,140.42789558)(71.26920844,140.16747895)(71.3421251,139.86539565)
\curveto(71.41504175,139.56331236)(71.45150008,139.19091657)(71.45150008,138.74820829)
\lineto(71.45150008,137.37320846)
\curveto(71.45150008,137.2846668)(71.46452092,136.63622938)(71.49056258,135.4278962)
\curveto(71.50097924,134.93310459)(71.55566674,134.6388338)(71.65462506,134.54508381)
\curveto(71.75358338,134.45133382)(72.07650001,134.40445883)(72.62337494,134.40445883)
\lineto(72.68587493,134.34195883)
\lineto(72.68587493,134.00602138)
\lineto(72.62337494,133.94352138)
\curveto(71.96191669,133.98518804)(71.51400008,134.00602138)(71.2796251,134.00602138)
\curveto(71.14420846,134.00602138)(70.76400017,133.98518804)(70.13900025,133.94352138)
\lineto(70.04525026,134.02945887)
\curveto(70.11295858,134.70654212)(70.14681274,135.58935451)(70.14681274,136.67789605)
\lineto(70.14681274,137.70133342)
\curveto(70.14681274,138.31591668)(70.13118775,138.75862496)(70.09993775,139.02945826)
\curveto(70.07389609,139.30029156)(69.9879586,139.54508319)(69.84212528,139.76383317)
\curveto(69.69629197,139.98779147)(69.50097949,140.15966645)(69.25618785,140.2794581)
\curveto(69.01139622,140.40445809)(68.71972959,140.46695808)(68.38118796,140.46695808)
\curveto(68.11035466,140.46695808)(67.87858386,140.43831225)(67.68587555,140.38102059)
\curveto(67.49316724,140.32893726)(67.27702143,140.21695811)(67.03743813,140.04508313)
\curveto(66.79785482,139.87841648)(66.62077151,139.71956234)(66.50618819,139.56852069)
\curveto(66.39160487,139.42268737)(66.32129238,139.28466656)(66.29525072,139.15445824)
\curveto(66.27441739,139.02945826)(66.26400072,138.75862496)(66.26400072,138.34195834)
\lineto(66.26400072,136.98258351)
\curveto(66.26400072,136.86800019)(66.27441739,136.49820857)(66.29525072,135.87320864)
\curveto(66.31608405,135.25341705)(66.33691738,134.89143793)(66.35775071,134.78727128)
\curveto(66.38379238,134.68310462)(66.4202507,134.6075838)(66.4671257,134.56070881)
\curveto(66.51400069,134.51383381)(66.57129235,134.48258382)(66.63900068,134.46695882)
\curveto(66.706709,134.45654215)(66.98795897,134.43570882)(67.48275057,134.40445883)
\lineto(67.55306306,134.34195883)
\lineto(67.55306306,134.01383387)
\lineto(67.49056307,133.94352138)
\curveto(66.83952149,133.98518804)(66.2119174,134.00602138)(65.6077508,134.00602138)
\curveto(65.00879254,134.00602138)(64.38379262,133.98518804)(63.73275104,133.94352138)
\lineto(63.66243854,134.01383387)
\lineto(63.66243854,134.34195883)
\lineto(63.73275104,134.40445883)
\curveto(64.23795931,134.43570882)(64.52181344,134.45914632)(64.58431343,134.47477132)
\curveto(64.65202176,134.49039632)(64.70931342,134.52164631)(64.75618841,134.56852131)
\curveto(64.80827174,134.62060463)(64.8421259,134.69612546)(64.8577509,134.79508378)
\curveto(64.87858423,134.89925043)(64.89941756,135.23518789)(64.92025089,135.80289615)
\curveto(64.94629255,136.37581275)(64.95931338,136.80810436)(64.95931338,137.09977099)
\lineto(64.95931338,138.6388333)
\curveto(64.95931338,138.84716661)(64.94889672,139.13102074)(64.92806339,139.4903957)
\curveto(64.90723006,139.84977065)(64.88900089,140.07112479)(64.87337589,140.15445812)
\curveto(64.86295923,140.23779144)(64.82389673,140.29768727)(64.75618841,140.33414559)
\curveto(64.68848008,140.37581226)(64.55306343,140.39664559)(64.34993846,140.39664559)
\lineto(63.69368854,140.40445809)
\lineto(63.62337605,140.46695808)
\lineto(63.62337605,140.80289554)
\lineto(63.68587604,140.86539553)
\curveto(64.68066759,140.98518718)(65.51920915,141.18049966)(66.20150073,141.45133296)
\closepath
}
}
{
\newrgbcolor{curcolor}{0 0 0}
\pscustom[linestyle=none,fillstyle=solid,fillcolor=curcolor]
{
\newpath
\moveto(75.09212463,144.94352003)
\curveto(75.33691627,144.94352003)(75.54524958,144.85758254)(75.71712456,144.68570756)
\curveto(75.88899954,144.51383258)(75.97493703,144.30549927)(75.97493703,144.06070764)
\curveto(75.97493703,143.82112433)(75.88899954,143.61539519)(75.71712456,143.44352021)
\curveto(75.54524958,143.27164523)(75.33691627,143.18570774)(75.09212463,143.18570774)
\curveto(74.85254133,143.18570774)(74.64420802,143.26904107)(74.46712471,143.43570771)
\curveto(74.29524973,143.60758269)(74.20931224,143.815916)(74.20931224,144.06070764)
\curveto(74.20931224,144.30549927)(74.29524973,144.51383258)(74.46712471,144.68570756)
\curveto(74.64420802,144.85758254)(74.85254133,144.94352003)(75.09212463,144.94352003)
\closepath
\moveto(75.75618705,141.45133296)
\lineto(75.90462453,141.34977047)
\curveto(75.84733288,140.66747889)(75.81868705,139.80029149)(75.81868705,138.74820829)
\lineto(75.81868705,136.98258351)
\curveto(75.81868705,136.86800019)(75.82910371,136.49820857)(75.84993704,135.87320864)
\curveto(75.87077037,135.25341705)(75.8916037,134.89143793)(75.91243703,134.78727128)
\curveto(75.9384787,134.68310462)(75.97493703,134.6075838)(76.02181202,134.56070881)
\curveto(76.06868701,134.51383381)(76.12597867,134.48258382)(76.193687,134.46695882)
\curveto(76.26139532,134.45654215)(76.54264529,134.43570882)(77.0374369,134.40445883)
\lineto(77.10774939,134.34195883)
\lineto(77.10774939,134.01383387)
\lineto(77.04524939,133.94352138)
\curveto(76.39420781,133.98518804)(75.76660372,134.00602138)(75.16243713,134.00602138)
\curveto(74.56347887,134.00602138)(73.93847894,133.98518804)(73.28743736,133.94352138)
\lineto(73.21712487,134.01383387)
\lineto(73.21712487,134.34195883)
\lineto(73.28743736,134.40445883)
\curveto(73.79264563,134.43570882)(74.07649976,134.45914632)(74.13899975,134.47477132)
\curveto(74.20670808,134.49039632)(74.26399974,134.52164631)(74.31087473,134.56852131)
\curveto(74.36295806,134.62060463)(74.39681222,134.69612546)(74.41243722,134.79508378)
\curveto(74.43327055,134.89925043)(74.45410388,135.23518789)(74.47493721,135.80289615)
\curveto(74.50097887,136.37581275)(74.51399971,136.80810436)(74.51399971,137.09977099)
\lineto(74.51399971,138.6388333)
\curveto(74.51399971,138.84716661)(74.50358304,139.13102074)(74.48274971,139.4903957)
\curveto(74.46191638,139.84977065)(74.44368721,140.07112479)(74.42806222,140.15445812)
\curveto(74.41764555,140.23779144)(74.37858306,140.29768727)(74.31087473,140.33414559)
\curveto(74.24316641,140.37581226)(74.10774976,140.39664559)(73.90462478,140.39664559)
\lineto(73.24837486,140.40445809)
\lineto(73.17806237,140.46695808)
\lineto(73.17806237,140.80289554)
\lineto(73.24056236,140.86539553)
\curveto(74.23535391,140.98518718)(75.07389547,141.18049966)(75.75618705,141.45133296)
\closepath
\moveto(75.14681213,133.53727143)
\closepath
}
}
{
\newrgbcolor{curcolor}{0 0 0}
\pscustom[linestyle=none,fillstyle=solid,fillcolor=curcolor]
{
\newpath
\moveto(77.85774929,140.13883312)
\lineto(77.85774929,140.34195809)
\lineto(77.91243679,140.42008308)
\curveto(78.40202006,140.60237473)(78.79785334,140.77164554)(79.09993664,140.92789552)
\curveto(79.09993664,142.28727035)(79.08431164,143.07893692)(79.05306165,143.30289523)
\curveto(79.58951991,143.49039521)(80.02962403,143.69091601)(80.37337398,143.90445765)
\lineto(80.56087396,143.74820767)
\curveto(80.50879063,143.42008271)(80.44629064,142.46695783)(80.37337398,140.88883303)
\curveto(80.63379062,140.88362469)(80.91504058,140.88102053)(81.21712388,140.88102053)
\curveto(81.83170714,140.88102053)(82.27181125,140.89664553)(82.53743622,140.92789552)
\lineto(82.59212371,140.87320803)
\lineto(82.44368623,140.21695811)
\lineto(82.38118624,140.14664562)
\curveto(82.11556127,140.15185395)(81.82129047,140.15445812)(81.49837384,140.15445812)
\curveto(81.20670721,140.15445812)(80.83170726,140.15185395)(80.37337398,140.14664562)
\lineto(80.32649899,136.96695851)
\curveto(80.32649899,136.2325836)(80.34212399,135.75081283)(80.37337398,135.52164619)
\curveto(80.40983231,135.29768788)(80.5035823,135.12060457)(80.65462395,134.99039625)
\curveto(80.81087393,134.86539627)(81.04004057,134.80289628)(81.34212386,134.80289628)
\curveto(81.69108215,134.80289628)(82.01399878,134.8940421)(82.31087374,135.07633374)
\lineto(82.49837372,134.79508378)
\curveto(82.37337374,134.70654212)(82.07389461,134.44612549)(81.59993633,134.01383387)
\curveto(81.32910303,133.88883389)(81.0530614,133.8263339)(80.77181143,133.8263339)
\curveto(79.59993658,133.8263339)(79.01399915,134.39404216)(79.01399915,135.52945869)
\curveto(79.01399915,135.9461253)(79.02441582,136.30029193)(79.04524915,136.59195856)
\curveto(79.05045748,136.68050021)(79.05306165,136.77164603)(79.05306165,136.86539602)
\lineto(79.05306165,140.10758312)
\lineto(78.73274919,140.10758312)
\curveto(78.49837421,140.10758312)(78.23014508,140.09716646)(77.92806179,140.07633313)
\closepath
}
}
{
\newrgbcolor{curcolor}{0 0 0}
\pscustom[linestyle=none,fillstyle=solid,fillcolor=curcolor]
{
\newpath
\moveto(84.96712342,144.94352003)
\curveto(85.21191505,144.94352003)(85.42024836,144.85758254)(85.59212334,144.68570756)
\curveto(85.76399832,144.51383258)(85.84993581,144.30549927)(85.84993581,144.06070764)
\curveto(85.84993581,143.82112433)(85.76399832,143.61539519)(85.59212334,143.44352021)
\curveto(85.42024836,143.27164523)(85.21191505,143.18570774)(84.96712342,143.18570774)
\curveto(84.72754011,143.18570774)(84.51920681,143.26904107)(84.34212349,143.43570771)
\curveto(84.17024852,143.60758269)(84.08431103,143.815916)(84.08431103,144.06070764)
\curveto(84.08431103,144.30549927)(84.17024852,144.51383258)(84.34212349,144.68570756)
\curveto(84.51920681,144.85758254)(84.72754011,144.94352003)(84.96712342,144.94352003)
\closepath
\moveto(85.63118584,141.45133296)
\lineto(85.77962332,141.34977047)
\curveto(85.72233166,140.66747889)(85.69368583,139.80029149)(85.69368583,138.74820829)
\lineto(85.69368583,136.98258351)
\curveto(85.69368583,136.86800019)(85.70410249,136.49820857)(85.72493582,135.87320864)
\curveto(85.74576915,135.25341705)(85.76660249,134.89143793)(85.78743582,134.78727128)
\curveto(85.81347748,134.68310462)(85.84993581,134.6075838)(85.8968108,134.56070881)
\curveto(85.9436858,134.51383381)(86.00097746,134.48258382)(86.06868578,134.46695882)
\curveto(86.13639411,134.45654215)(86.41764407,134.43570882)(86.91243568,134.40445883)
\lineto(86.98274817,134.34195883)
\lineto(86.98274817,134.01383387)
\lineto(86.92024818,133.94352138)
\curveto(86.26920659,133.98518804)(85.6416025,134.00602138)(85.03743591,134.00602138)
\curveto(84.43847765,134.00602138)(83.81347773,133.98518804)(83.16243614,133.94352138)
\lineto(83.09212365,134.01383387)
\lineto(83.09212365,134.34195883)
\lineto(83.16243614,134.40445883)
\curveto(83.66764441,134.43570882)(83.95149854,134.45914632)(84.01399853,134.47477132)
\curveto(84.08170686,134.49039632)(84.13899852,134.52164631)(84.18587351,134.56852131)
\curveto(84.23795684,134.62060463)(84.271811,134.69612546)(84.287436,134.79508378)
\curveto(84.30826933,134.89925043)(84.32910266,135.23518789)(84.34993599,135.80289615)
\curveto(84.37597766,136.37581275)(84.38899849,136.80810436)(84.38899849,137.09977099)
\lineto(84.38899849,138.6388333)
\curveto(84.38899849,138.84716661)(84.37858182,139.13102074)(84.35774849,139.4903957)
\curveto(84.33691516,139.84977065)(84.318686,140.07112479)(84.303061,140.15445812)
\curveto(84.29264433,140.23779144)(84.25358184,140.29768727)(84.18587351,140.33414559)
\curveto(84.11816519,140.37581226)(83.98274854,140.39664559)(83.77962356,140.39664559)
\lineto(83.12337364,140.40445809)
\lineto(83.05306115,140.46695808)
\lineto(83.05306115,140.80289554)
\lineto(83.11556115,140.86539553)
\curveto(84.11035269,140.98518718)(84.94889425,141.18049966)(85.63118584,141.45133296)
\closepath
\moveto(85.02181091,133.53727143)
\closepath
}
}
{
\newrgbcolor{curcolor}{0 0 0}
\pscustom[linestyle=none,fillstyle=solid,fillcolor=curcolor]
{
\newpath
\moveto(87.90462306,137.58414593)
\curveto(87.90462306,138.20393752)(88.03222721,138.80029162)(88.28743551,139.37320821)
\curveto(88.54264381,139.95133314)(88.98535209,140.44872891)(89.61556034,140.86539553)
\curveto(90.25097693,141.28727048)(91.00358101,141.49820795)(91.87337257,141.49820795)
\curveto(92.96712243,141.49820795)(93.85774732,141.15185383)(94.54524724,140.45914558)
\curveto(95.23274715,139.77164566)(95.57649711,138.88622911)(95.57649711,137.80289591)
\curveto(95.57649711,136.61018772)(95.18066382,135.62841701)(94.38899726,134.85758377)
\curveto(93.60253902,134.08675053)(92.6520183,133.70133391)(91.53743511,133.70133391)
\curveto(90.80826853,133.70133391)(90.15722694,133.89664639)(89.58431035,134.28727134)
\curveto(89.01139375,134.68310462)(88.58691464,135.1648754)(88.310873,135.73258366)
\curveto(88.04003971,136.30029193)(87.90462306,136.91747935)(87.90462306,137.58414593)
\closepath
\moveto(89.38118537,138.12320837)
\curveto(89.38118537,137.39404179)(89.47493536,136.74039604)(89.66243534,136.16227111)
\curveto(89.84993532,135.58935451)(90.14681028,135.12841707)(90.55306023,134.77945878)
\curveto(90.95931018,134.43050049)(91.42285179,134.25602134)(91.94368506,134.25602134)
\curveto(92.55826831,134.25602134)(93.06608075,134.49820881)(93.46712237,134.98258375)
\curveto(93.86816399,135.47216703)(94.06868479,136.20654194)(94.06868479,137.18570848)
\curveto(94.06868479,138.29508335)(93.84993482,139.2039374)(93.41243488,139.91227065)
\curveto(92.98014326,140.62060389)(92.35514334,140.97477052)(91.53743511,140.97477052)
\curveto(90.86035186,140.97477052)(90.33170609,140.72997888)(89.9514978,140.24039561)
\curveto(89.57128952,139.75081233)(89.38118537,139.04508325)(89.38118537,138.12320837)
\closepath
\moveto(91.74837258,141.9435204)
\closepath
\moveto(91.6311851,133.53727143)
\closepath
}
}
{
\newrgbcolor{curcolor}{0 0 0}
\pscustom[linestyle=none,fillstyle=solid,fillcolor=curcolor]
{
\newpath
\moveto(98.77962171,141.45133296)
\lineto(98.9280592,141.34977047)
\curveto(98.8968092,141.00081218)(98.87597587,140.5424789)(98.8655592,139.97477064)
\curveto(99.28222582,140.3185206)(99.6780591,140.66747889)(100.05305906,141.02164551)
\curveto(100.16243404,141.12060383)(100.26399653,141.19352049)(100.35774652,141.24039548)
\curveto(100.45670484,141.28727048)(100.61816315,141.33674964)(100.84212146,141.38883296)
\curveto(101.06607977,141.44091629)(101.2952464,141.46695796)(101.52962138,141.46695796)
\curveto(101.92545466,141.46695796)(102.30826711,141.3862288)(102.67805873,141.22477048)
\curveto(103.05305869,141.06331217)(103.33430865,140.87060386)(103.52180863,140.64664556)
\curveto(103.71451694,140.42789558)(103.84732942,140.16747895)(103.92024608,139.86539565)
\curveto(103.99316274,139.56331236)(104.02962107,139.19091657)(104.02962107,138.74820829)
\lineto(104.02962107,137.37320846)
\curveto(104.02962107,137.2846668)(104.0426419,136.63622938)(104.06868356,135.4278962)
\curveto(104.07910023,134.93310459)(104.13378772,134.6388338)(104.23274604,134.54508381)
\curveto(104.33170436,134.45133382)(104.65462099,134.40445883)(105.20149592,134.40445883)
\lineto(105.26399591,134.34195883)
\lineto(105.26399591,134.00602138)
\lineto(105.20149592,133.94352138)
\curveto(104.54003767,133.98518804)(104.09212106,134.00602138)(103.85774609,134.00602138)
\curveto(103.72232944,134.00602138)(103.34212115,133.98518804)(102.71712123,133.94352138)
\lineto(102.62337124,134.02945887)
\curveto(102.69107957,134.70654212)(102.72493373,135.58935451)(102.72493373,136.67789605)
\lineto(102.72493373,137.70133342)
\curveto(102.72493373,138.31591668)(102.70930873,138.75862496)(102.67805873,139.02945826)
\curveto(102.65201707,139.30029156)(102.56607958,139.54508319)(102.42024627,139.76383317)
\curveto(102.27441295,139.98779147)(102.07910047,140.15966645)(101.83430884,140.2794581)
\curveto(101.5895172,140.40445809)(101.29785057,140.46695808)(100.95930895,140.46695808)
\curveto(100.68847565,140.46695808)(100.45670484,140.43831225)(100.26399653,140.38102059)
\curveto(100.07128822,140.32893726)(99.85514241,140.21695811)(99.61555911,140.04508313)
\curveto(99.37597581,139.87841648)(99.1988925,139.71956234)(99.08430918,139.56852069)
\curveto(98.96972586,139.42268737)(98.89941337,139.28466656)(98.8733717,139.15445824)
\curveto(98.85253837,139.02945826)(98.84212171,138.75862496)(98.84212171,138.34195834)
\lineto(98.84212171,136.98258351)
\curveto(98.84212171,136.86800019)(98.85253837,136.49820857)(98.8733717,135.87320864)
\curveto(98.89420503,135.25341705)(98.91503836,134.89143793)(98.93587169,134.78727128)
\curveto(98.96191336,134.68310462)(98.99837169,134.6075838)(99.04524668,134.56070881)
\curveto(99.09212168,134.51383381)(99.14941334,134.48258382)(99.21712166,134.46695882)
\curveto(99.28482999,134.45654215)(99.56607995,134.43570882)(100.06087156,134.40445883)
\lineto(100.13118405,134.34195883)
\lineto(100.13118405,134.01383387)
\lineto(100.06868406,133.94352138)
\curveto(99.41764247,133.98518804)(98.79003838,134.00602138)(98.18587179,134.00602138)
\curveto(97.58691353,134.00602138)(96.9619136,133.98518804)(96.31087202,133.94352138)
\lineto(96.24055953,134.01383387)
\lineto(96.24055953,134.34195883)
\lineto(96.31087202,134.40445883)
\curveto(96.81608029,134.43570882)(97.09993442,134.45914632)(97.16243441,134.47477132)
\curveto(97.23014274,134.49039632)(97.2874344,134.52164631)(97.33430939,134.56852131)
\curveto(97.38639272,134.62060463)(97.42024688,134.69612546)(97.43587188,134.79508378)
\curveto(97.45670521,134.89925043)(97.47753854,135.23518789)(97.49837187,135.80289615)
\curveto(97.52441354,136.37581275)(97.53743437,136.80810436)(97.53743437,137.09977099)
\lineto(97.53743437,138.6388333)
\curveto(97.53743437,138.84716661)(97.5270177,139.13102074)(97.50618437,139.4903957)
\curveto(97.48535104,139.84977065)(97.46712188,140.07112479)(97.45149688,140.15445812)
\curveto(97.44108021,140.23779144)(97.40201772,140.29768727)(97.33430939,140.33414559)
\curveto(97.26660107,140.37581226)(97.13118442,140.39664559)(96.92805944,140.39664559)
\lineto(96.27180952,140.40445809)
\lineto(96.20149703,140.46695808)
\lineto(96.20149703,140.80289554)
\lineto(96.26399702,140.86539553)
\curveto(97.25878857,140.98518718)(98.09733013,141.18049966)(98.77962171,141.45133296)
\closepath
}
}
{
\newrgbcolor{curcolor}{0 0 0}
\pscustom[linestyle=none,fillstyle=solid,fillcolor=curcolor]
{
}
}
{
\newrgbcolor{curcolor}{0 0 0}
\pscustom[linestyle=none,fillstyle=solid,fillcolor=curcolor]
{
\newpath
\moveto(114.67805725,144.59977007)
\lineto(114.67805725,144.92789503)
\lineto(114.74055725,144.99820752)
\curveto(115.73534879,145.11279084)(116.57389035,145.30549915)(117.25618194,145.57633245)
\lineto(117.40461942,145.48258246)
\curveto(117.34732776,144.8836242)(117.31868193,143.68310352)(117.31868193,141.8810204)
\lineto(117.31868193,136.76383354)
\curveto(117.31868193,136.23779193)(117.32649443,135.77685449)(117.34211943,135.38102121)
\curveto(117.36295276,134.99039625)(117.39680692,134.75862545)(117.44368191,134.68570879)
\curveto(117.49055691,134.61279213)(117.56607773,134.55289631)(117.67024438,134.50602131)
\curveto(117.77961937,134.45914632)(118.08430683,134.41487549)(118.58430677,134.37320883)
\lineto(118.65461926,134.31070884)
\lineto(118.65461926,134.01383387)
\lineto(118.58430677,133.94352138)
\curveto(118.02180684,133.97997971)(117.5843069,133.99820888)(117.27180693,133.99820888)
\curveto(116.99055697,133.99820888)(116.58951535,133.97997971)(116.06868208,133.94352138)
\lineto(115.98274459,134.02945887)
\curveto(115.99836959,134.53987548)(116.00618209,134.89664627)(116.00618209,135.09977124)
\curveto(116.00618209,135.1258129)(116.00878626,135.22477122)(116.01399459,135.3966462)
\curveto(115.72753629,135.1883129)(115.44107799,134.95654209)(115.1546197,134.70133379)
\curveto(114.73274475,134.32633384)(114.45670311,134.09195886)(114.3264948,133.99820888)
\curveto(114.08691149,133.87841722)(113.7457657,133.8185214)(113.30305742,133.8185214)
\curveto(112.58430751,133.8185214)(111.97232842,133.99560471)(111.46712015,134.34977133)
\curveto(110.96712021,134.70393796)(110.61034942,135.15445873)(110.39680778,135.70133367)
\curveto(110.18326614,136.25341693)(110.07649532,136.81852103)(110.07649532,137.39664596)
\curveto(110.07649532,137.98518755)(110.18587031,138.55029165)(110.40462028,139.09195825)
\curveto(110.62857859,139.63883318)(110.94368271,140.03727063)(111.34993266,140.2872706)
\curveto(111.76139095,140.53727057)(112.20670339,140.79508304)(112.68587,141.06070801)
\curveto(113.17024494,141.33154131)(113.64941155,141.46695796)(114.12336982,141.46695796)
\curveto(114.78482807,141.46695796)(115.41503633,141.30289548)(116.01399459,140.97477052)
\lineto(115.98274459,142.7169578)
\curveto(115.97232793,143.32112439)(115.95670293,143.74039517)(115.9358696,143.97477015)
\curveto(115.91503627,144.21435345)(115.8889946,144.35497843)(115.85774461,144.39664509)
\curveto(115.82649461,144.44352009)(115.77180712,144.47737425)(115.69368213,144.49820758)
\curveto(115.62076547,144.51904091)(115.30566134,144.52945758)(114.74836975,144.52945758)
\closepath
\moveto(116.01399459,139.63883318)
\curveto(115.69107796,140.00341647)(115.33170301,140.28206227)(114.93586972,140.47477058)
\curveto(114.54524477,140.66747889)(114.15201565,140.76383304)(113.75618237,140.76383304)
\curveto(113.31868242,140.76383304)(112.90982831,140.64404139)(112.52962002,140.40445809)
\curveto(112.15462006,140.17008312)(111.88378676,139.81852066)(111.71712012,139.34977072)
\curveto(111.55045347,138.88622911)(111.46712015,138.38102084)(111.46712015,137.8341459)
\curveto(111.46712015,136.92268768)(111.69628679,136.19091694)(112.15462006,135.63883367)
\curveto(112.61816167,135.08675041)(113.19889077,134.81070878)(113.89680735,134.81070878)
\curveto(114.27701564,134.81070878)(114.6129531,134.89664627)(114.90461973,135.06852124)
\curveto(115.20149469,135.24560456)(115.44368216,135.48258369)(115.63118214,135.77945866)
\curveto(115.82389045,136.07633362)(115.9358696,136.37320858)(115.96711959,136.67008355)
\curveto(115.99836959,136.96695851)(116.01399459,137.51122928)(116.01399459,138.30289585)
\closepath
}
}
{
\newrgbcolor{curcolor}{0 0 0}
\pscustom[linestyle=none,fillstyle=solid,fillcolor=curcolor]
{
\newpath
\moveto(126.09993085,135.15445873)
\lineto(125.84993088,134.5919588)
\curveto(125.30826428,134.24300051)(124.82388934,134.01643804)(124.39680606,133.91227139)
\curveto(123.97493111,133.80810473)(123.59732699,133.75602141)(123.2639937,133.75602141)
\curveto(122.63899377,133.75602141)(122.04524385,133.87841722)(121.48274391,134.12320886)
\curveto(120.92545232,134.3680005)(120.46972321,134.78206295)(120.11555658,135.36539621)
\curveto(119.76659829,135.94872947)(119.59211915,136.65185438)(119.59211915,137.47477095)
\curveto(119.59211915,138.02164588)(119.65982747,138.51383332)(119.79524412,138.95133327)
\curveto(119.93066077,139.39404154)(120.07128576,139.7221665)(120.21711907,139.93570814)
\curveto(120.36816072,140.14924978)(120.62076485,140.38622892)(120.97493148,140.64664556)
\curveto(121.3290981,140.90706219)(121.70409805,141.1153955)(122.09993134,141.27164548)
\curveto(122.49576462,141.42789546)(122.9228479,141.50602045)(123.38118118,141.50602045)
\curveto(124.0061811,141.50602045)(124.5556602,141.35758297)(125.02961848,141.06070801)
\curveto(125.50878508,140.76904137)(125.84472254,140.39404142)(126.03743085,139.93570814)
\curveto(126.23013916,139.47737487)(126.32649332,138.99039576)(126.32649332,138.47477082)
\curveto(126.32649332,138.31331251)(126.31868082,138.15706253)(126.30305582,138.00602088)
\lineto(126.21711833,137.92008339)
\curveto(125.86295171,137.8419584)(125.38638927,137.78987508)(124.78743101,137.76383341)
\curveto(124.18847275,137.73779175)(123.79263946,137.72477092)(123.59993115,137.72477092)
\lineto(121.09211896,137.72477092)
\curveto(121.10253563,136.64664605)(121.37336893,135.85237531)(121.90461886,135.34195871)
\curveto(122.4358688,134.83154211)(123.08691038,134.5763338)(123.85774362,134.5763338)
\curveto(124.22232691,134.5763338)(124.5712852,134.6388338)(124.90461849,134.76383378)
\curveto(125.24316012,134.88883377)(125.59993091,135.06331291)(125.97493086,135.28727122)
\closepath
\moveto(121.09211896,138.34977084)
\curveto(121.18586895,138.33414584)(121.54524391,138.31591668)(122.17024383,138.29508335)
\curveto(122.80045209,138.27425002)(123.26659786,138.26383335)(123.56868116,138.26383335)
\curveto(124.2926394,138.26383335)(124.73274351,138.27685418)(124.88899349,138.30289585)
\curveto(124.89420183,138.42789583)(124.89680599,138.52424998)(124.89680599,138.59195831)
\curveto(124.89680599,139.39924988)(124.73274351,139.99820814)(124.40461855,140.38883309)
\curveto(124.07649359,140.78466637)(123.62857698,140.98258301)(123.06086872,140.98258301)
\curveto(122.44107713,140.98258301)(121.95670219,140.76122888)(121.6077439,140.3185206)
\curveto(121.26399394,139.87581232)(121.09211896,139.2195624)(121.09211896,138.34977084)
\closepath
\moveto(123.27180619,141.9435204)
\closepath
\moveto(123.17805621,133.53727143)
\closepath
}
}
{
\newrgbcolor{curcolor}{0 0 0}
\pscustom[linestyle=none,fillstyle=solid,fillcolor=curcolor]
{
\newpath
\moveto(127.64680565,136.30289609)
\lineto(127.98274311,136.30289609)
\lineto(128.0530556,136.2325836)
\curveto(128.06347227,135.79508365)(128.08951393,135.4200837)(128.13118059,135.10758374)
\curveto(128.29263891,134.8627921)(128.57128471,134.66227129)(128.96711799,134.50602131)
\curveto(129.36295128,134.35497967)(129.75357623,134.27945884)(130.13899285,134.27945884)
\curveto(130.69107611,134.27945884)(131.13118023,134.42789632)(131.45930518,134.72477129)
\curveto(131.79263848,135.02164625)(131.95930512,135.37581287)(131.95930512,135.78727116)
\curveto(131.95930512,136.01122946)(131.8994093,136.20393777)(131.77961765,136.36539608)
\curveto(131.65982599,136.53206273)(131.47753435,136.66747938)(131.23274271,136.77164603)
\curveto(130.99315941,136.88102102)(130.5608678,137.00862517)(129.93586787,137.15445849)
\curveto(129.39940961,137.27945847)(129.02440965,137.37581263)(128.81086801,137.44352095)
\curveto(128.59732637,137.51643761)(128.39159723,137.63622926)(128.19368059,137.80289591)
\curveto(127.99576395,137.96956255)(127.8447223,138.17268753)(127.74055564,138.41227083)
\curveto(127.63638899,138.65706247)(127.58430566,138.9252916)(127.58430566,139.21695823)
\curveto(127.58430566,139.91487481)(127.8603473,140.47216641)(128.41243056,140.88883303)
\curveto(128.96972216,141.31070797)(129.66243041,141.52164545)(130.4905553,141.52164545)
\curveto(130.83951359,141.52164545)(131.21972188,141.47216629)(131.63118016,141.37320797)
\curveto(132.04263845,141.27945798)(132.35513841,141.18831216)(132.56868005,141.0997705)
\lineto(132.63899254,140.99039551)
\curveto(132.59732588,140.78206221)(132.57128421,140.23518727)(132.56086755,139.34977072)
\lineto(132.49055506,139.27945822)
\lineto(132.1780551,139.27945822)
\lineto(132.1077426,139.34977072)
\curveto(132.08690927,139.66747901)(132.05826344,139.88883315)(132.02180512,140.01383313)
\curveto(131.98534679,140.14404145)(131.8915968,140.28206227)(131.74055515,140.42789558)
\curveto(131.5895135,140.57893723)(131.37597186,140.70393722)(131.09993023,140.80289554)
\curveto(130.8238886,140.90706219)(130.53222197,140.95914552)(130.22493034,140.95914552)
\curveto(129.90722204,140.95914552)(129.63899291,140.91227052)(129.42024294,140.81852054)
\curveto(129.2067013,140.72477055)(129.03222215,140.5815414)(128.8968055,140.38883309)
\curveto(128.76659718,140.20133311)(128.70149302,139.97216647)(128.70149302,139.70133317)
\curveto(128.70149302,139.50341653)(128.74055552,139.32633322)(128.81868051,139.17008324)
\curveto(128.90201383,139.01383326)(129.02701382,138.88622911)(129.19368046,138.78727079)
\curveto(129.36034711,138.6935208)(129.53482626,138.62581247)(129.7171179,138.58414581)
\lineto(130.56868029,138.36539584)
\curveto(131.29784687,138.18831253)(131.81868014,138.03466671)(132.1311801,137.90445839)
\curveto(132.4488884,137.77425008)(132.69368003,137.57633343)(132.86555501,137.31070847)
\curveto(133.03742999,137.05029183)(133.12336748,136.73258354)(133.12336748,136.35758359)
\curveto(133.12336748,135.63883367)(132.82649252,135.02164625)(132.23274259,134.50602131)
\curveto(131.63899266,133.99039638)(130.87597192,133.73258391)(129.94368037,133.73258391)
\curveto(129.61555541,133.73258391)(129.2067013,133.76904224)(128.71711802,133.8419589)
\curveto(128.23274308,133.91487555)(127.81868063,133.99560471)(127.47493068,134.08414637)
\lineto(127.43586818,134.18570885)
\lineto(127.51399317,134.70914629)
\curveto(127.54003483,134.8706046)(127.55565983,135.03206292)(127.56086817,135.19352123)
\curveto(127.5660765,135.36018787)(127.57128483,135.706542)(127.57649316,136.2325836)
\closepath
}
}
{
\newrgbcolor{curcolor}{0 0 0}
\pscustom[linestyle=none,fillstyle=solid,fillcolor=curcolor]
{
}
}
{
\newrgbcolor{curcolor}{0 0 0}
\pscustom[linestyle=none,fillstyle=solid,fillcolor=curcolor]
{
\newpath
\moveto(137.87336689,144.99039502)
\lineto(137.93586689,145.04508251)
\curveto(139.22753339,145.00341585)(139.97232497,144.98258252)(140.17024161,144.98258252)
\curveto(140.34732492,144.98258252)(141.09472066,145.00341585)(142.41242883,145.04508251)
\lineto(142.47492883,144.99039502)
\lineto(142.47492883,144.59977007)
\lineto(142.41242883,144.54508258)
\curveto(142.05305388,144.53466591)(141.77440808,144.51904091)(141.57649144,144.49820758)
\curveto(141.38378313,144.47737425)(141.24315814,144.43831176)(141.15461649,144.3810201)
\curveto(141.06607483,144.32372844)(141.00617901,144.18831179)(140.97492901,143.97477015)
\curveto(140.94888735,143.76643684)(140.93586652,143.47477021)(140.93586652,143.09977025)
\lineto(140.92024152,141.30289548)
\lineto(140.92024152,138.81852078)
\curveto(140.92024152,138.00081255)(140.93586652,137.39925012)(140.96711651,137.0138335)
\curveto(141.00357484,136.63362522)(141.0764915,136.31070859)(141.18586649,136.04508362)
\curveto(141.3004498,135.78466699)(141.47232478,135.55810452)(141.70149142,135.36539621)
\curveto(141.93065806,135.17789623)(142.25878302,135.02425042)(142.6858663,134.90445876)
\curveto(143.11294958,134.78466711)(143.59732452,134.72477129)(144.13899112,134.72477129)
\curveto(144.71190772,134.72477129)(145.24576182,134.80289628)(145.74055342,134.95914626)
\curveto(146.24055336,135.12060457)(146.62076165,135.35237538)(146.88117828,135.65445867)
\curveto(147.14159492,135.95654197)(147.31086573,136.35237525)(147.38899072,136.84195853)
\curveto(147.47232404,137.33675013)(147.51399071,138.02164588)(147.51399071,138.89664577)
\lineto(147.51399071,141.30289548)
\curveto(147.51399071,141.70914543)(147.50617821,142.26643702)(147.49055321,142.97477027)
\curveto(147.48013654,143.68831185)(147.44888655,144.09716596)(147.39680322,144.20133262)
\curveto(147.34992823,144.3107076)(147.2587824,144.39143676)(147.12336575,144.44352009)
\curveto(146.99315744,144.50081175)(146.62596998,144.53466591)(146.02180339,144.54508258)
\lineto(145.9593034,144.60758257)
\lineto(145.9593034,144.97477002)
\lineto(146.02180339,145.04508251)
\curveto(146.79263663,145.00341585)(147.40982405,144.98258252)(147.87336566,144.98258252)
\curveto(148.26919895,144.98258252)(148.8837822,145.00341585)(149.71711543,145.04508251)
\lineto(149.78742792,144.98258252)
\lineto(149.78742792,144.60758257)
\lineto(149.71711543,144.54508258)
\curveto(149.22753216,144.53466591)(148.9072197,144.51383258)(148.75617805,144.48258258)
\curveto(148.61034474,144.45654092)(148.50617808,144.41487426)(148.44367809,144.3575826)
\curveto(148.38638643,144.30029094)(148.3421156,144.21174928)(148.31086561,144.09195763)
\curveto(148.28482394,143.97216598)(148.26659478,143.62841602)(148.25617811,143.06070776)
\lineto(148.23274062,141.30289548)
\lineto(148.23274062,138.92789577)
\curveto(148.23274062,137.97997922)(148.19628229,137.23258348)(148.12336563,136.68570854)
\curveto(148.05044897,136.13883361)(147.88638649,135.68831283)(147.63117819,135.33414621)
\curveto(147.38117822,134.98518792)(147.08169909,134.69091712)(146.7327408,134.45133382)
\curveto(146.38899084,134.21695885)(145.94628256,134.03727137)(145.40461597,133.91227139)
\curveto(144.86294937,133.78206307)(144.30044943,133.71695891)(143.71711617,133.71695891)
\curveto(142.9879496,133.71695891)(142.34211634,133.79768807)(141.77961641,133.95914638)
\curveto(141.22232481,134.12060469)(140.7822207,134.33414633)(140.45930407,134.5997713)
\curveto(140.13638745,134.86539627)(139.90461664,135.14404207)(139.76399166,135.4357087)
\curveto(139.62336668,135.73258366)(139.53222086,136.07893779)(139.49055419,136.47477107)
\curveto(139.44888753,136.87581269)(139.4280542,137.48518761)(139.4280542,138.30289585)
\lineto(139.4280542,141.30289548)
\lineto(139.4124292,142.82633279)
\curveto(139.40722087,143.42008271)(139.39159587,143.80289517)(139.36555421,143.97477015)
\curveto(139.34472088,144.14664512)(139.30826255,144.26122844)(139.25617922,144.3185201)
\curveto(139.20930423,144.37581176)(139.11555424,144.42268676)(138.97492926,144.45914509)
\curveto(138.83430428,144.49560341)(138.48795015,144.52424924)(137.93586689,144.54508258)
\lineto(137.87336689,144.59977007)
\closepath
\moveto(143.97492864,145.62320744)
\closepath
\moveto(143.92805365,133.53727143)
\closepath
}
}
{
\newrgbcolor{curcolor}{0 0 0}
\pscustom[linestyle=none,fillstyle=solid,fillcolor=curcolor]
{
\newpath
\moveto(152.75617756,141.45133296)
\lineto(152.90461504,141.34977047)
\curveto(152.87336504,141.00081218)(152.85253171,140.5424789)(152.84211505,139.97477064)
\curveto(153.25878166,140.3185206)(153.65461495,140.66747889)(154.0296149,141.02164551)
\curveto(154.13898989,141.12060383)(154.24055238,141.19352049)(154.33430236,141.24039548)
\curveto(154.43326069,141.28727048)(154.594719,141.33674964)(154.8186773,141.38883296)
\curveto(155.04263561,141.44091629)(155.27180225,141.46695796)(155.50617722,141.46695796)
\curveto(155.9020105,141.46695796)(156.28482296,141.3862288)(156.65461458,141.22477048)
\curveto(157.02961453,141.06331217)(157.3108645,140.87060386)(157.49836447,140.64664556)
\curveto(157.69107278,140.42789558)(157.82388527,140.16747895)(157.89680192,139.86539565)
\curveto(157.96971858,139.56331236)(158.00617691,139.19091657)(158.00617691,138.74820829)
\lineto(158.00617691,137.37320846)
\curveto(158.00617691,137.2846668)(158.01919774,136.63622938)(158.04523941,135.4278962)
\curveto(158.05565607,134.93310459)(158.11034357,134.6388338)(158.20930189,134.54508381)
\curveto(158.30826021,134.45133382)(158.63117683,134.40445883)(159.17805177,134.40445883)
\lineto(159.24055176,134.34195883)
\lineto(159.24055176,134.00602138)
\lineto(159.17805177,133.94352138)
\curveto(158.51659352,133.98518804)(158.0686769,134.00602138)(157.83430193,134.00602138)
\curveto(157.69888528,134.00602138)(157.318677,133.98518804)(156.69367707,133.94352138)
\lineto(156.59992708,134.02945887)
\curveto(156.66763541,134.70654212)(156.70148957,135.58935451)(156.70148957,136.67789605)
\lineto(156.70148957,137.70133342)
\curveto(156.70148957,138.31591668)(156.68586457,138.75862496)(156.65461458,139.02945826)
\curveto(156.62857291,139.30029156)(156.54263543,139.54508319)(156.39680211,139.76383317)
\curveto(156.25096879,139.98779147)(156.05565632,140.15966645)(155.81086468,140.2794581)
\curveto(155.56607305,140.40445809)(155.27440641,140.46695808)(154.93586479,140.46695808)
\curveto(154.66503149,140.46695808)(154.43326069,140.43831225)(154.24055238,140.38102059)
\curveto(154.04784407,140.32893726)(153.83169826,140.21695811)(153.59211496,140.04508313)
\curveto(153.35253165,139.87841648)(153.17544834,139.71956234)(153.06086502,139.56852069)
\curveto(152.9462817,139.42268737)(152.87596921,139.28466656)(152.84992755,139.15445824)
\curveto(152.82909422,139.02945826)(152.81867755,138.75862496)(152.81867755,138.34195834)
\lineto(152.81867755,136.98258351)
\curveto(152.81867755,136.86800019)(152.82909422,136.49820857)(152.84992755,135.87320864)
\curveto(152.87076088,135.25341705)(152.89159421,134.89143793)(152.91242754,134.78727128)
\curveto(152.9384692,134.68310462)(152.97492753,134.6075838)(153.02180253,134.56070881)
\curveto(153.06867752,134.51383381)(153.12596918,134.48258382)(153.1936775,134.46695882)
\curveto(153.26138583,134.45654215)(153.54263579,134.43570882)(154.0374274,134.40445883)
\lineto(154.10773989,134.34195883)
\lineto(154.10773989,134.01383387)
\lineto(154.0452399,133.94352138)
\curveto(153.39419831,133.98518804)(152.76659422,134.00602138)(152.16242763,134.00602138)
\curveto(151.56346937,134.00602138)(150.93846945,133.98518804)(150.28742786,133.94352138)
\lineto(150.21711537,134.01383387)
\lineto(150.21711537,134.34195883)
\lineto(150.28742786,134.40445883)
\curveto(150.79263613,134.43570882)(151.07649027,134.45914632)(151.13899026,134.47477132)
\curveto(151.20669858,134.49039632)(151.26399024,134.52164631)(151.31086524,134.56852131)
\curveto(151.36294856,134.62060463)(151.39680273,134.69612546)(151.41242772,134.79508378)
\curveto(151.43326106,134.89925043)(151.45409439,135.23518789)(151.47492772,135.80289615)
\curveto(151.50096938,136.37581275)(151.51399021,136.80810436)(151.51399021,137.09977099)
\lineto(151.51399021,138.6388333)
\curveto(151.51399021,138.84716661)(151.50357355,139.13102074)(151.48274022,139.4903957)
\curveto(151.46190688,139.84977065)(151.44367772,140.07112479)(151.42805272,140.15445812)
\curveto(151.41763606,140.23779144)(151.37857356,140.29768727)(151.31086524,140.33414559)
\curveto(151.24315691,140.37581226)(151.10774026,140.39664559)(150.90461529,140.39664559)
\lineto(150.24836537,140.40445809)
\lineto(150.17805288,140.46695808)
\lineto(150.17805288,140.80289554)
\lineto(150.24055287,140.86539553)
\curveto(151.23534441,140.98518718)(152.07388598,141.18049966)(152.75617756,141.45133296)
\closepath
}
}
{
\newrgbcolor{curcolor}{0 0 0}
\pscustom[linestyle=none,fillstyle=solid,fillcolor=curcolor]
{
\newpath
\moveto(159.7561767,140.13883312)
\lineto(159.7561767,140.34195809)
\lineto(159.81086419,140.42008308)
\curveto(160.30044746,140.60237473)(160.69628075,140.77164554)(160.99836404,140.92789552)
\curveto(160.99836404,142.28727035)(160.98273904,143.07893692)(160.95148905,143.30289523)
\curveto(161.48794732,143.49039521)(161.92805143,143.69091601)(162.27180139,143.90445765)
\lineto(162.45930136,143.74820767)
\curveto(162.40721804,143.42008271)(162.34471804,142.46695783)(162.27180139,140.88883303)
\curveto(162.53221802,140.88362469)(162.81346799,140.88102053)(163.11555128,140.88102053)
\curveto(163.73013454,140.88102053)(164.17023865,140.89664553)(164.43586362,140.92789552)
\lineto(164.49055111,140.87320803)
\lineto(164.34211363,140.21695811)
\lineto(164.27961364,140.14664562)
\curveto(164.01398867,140.15185395)(163.71971787,140.15445812)(163.39680125,140.15445812)
\curveto(163.10513462,140.15445812)(162.73013466,140.15185395)(162.27180139,140.14664562)
\lineto(162.22492639,136.96695851)
\curveto(162.22492639,136.2325836)(162.24055139,135.75081283)(162.27180139,135.52164619)
\curveto(162.30825971,135.29768788)(162.4020097,135.12060457)(162.55305135,134.99039625)
\curveto(162.70930133,134.86539627)(162.93846797,134.80289628)(163.24055127,134.80289628)
\curveto(163.58950956,134.80289628)(163.91242618,134.8940421)(164.20930115,135.07633374)
\lineto(164.39680112,134.79508378)
\curveto(164.27180114,134.70654212)(163.97232201,134.44612549)(163.49836373,134.01383387)
\curveto(163.22753043,133.88883389)(162.9514888,133.8263339)(162.67023884,133.8263339)
\curveto(161.49836398,133.8263339)(160.91242655,134.39404216)(160.91242655,135.52945869)
\curveto(160.91242655,135.9461253)(160.92284322,136.30029193)(160.94367655,136.59195856)
\curveto(160.94888488,136.68050021)(160.95148905,136.77164603)(160.95148905,136.86539602)
\lineto(160.95148905,140.10758312)
\lineto(160.63117659,140.10758312)
\curveto(160.39680162,140.10758312)(160.12857248,140.09716646)(159.82648919,140.07633313)
\closepath
}
}
{
\newrgbcolor{curcolor}{0 0 0}
\pscustom[linestyle=none,fillstyle=solid,fillcolor=curcolor]
{
\newpath
\moveto(171.52180024,135.15445873)
\lineto(171.27180028,134.5919588)
\curveto(170.73013368,134.24300051)(170.24575874,134.01643804)(169.81867545,133.91227139)
\curveto(169.39680051,133.80810473)(169.01919639,133.75602141)(168.68586309,133.75602141)
\curveto(168.06086317,133.75602141)(167.46711324,133.87841722)(166.90461331,134.12320886)
\curveto(166.34732172,134.3680005)(165.89159261,134.78206295)(165.53742598,135.36539621)
\curveto(165.18846769,135.94872947)(165.01398855,136.65185438)(165.01398855,137.47477095)
\curveto(165.01398855,138.02164588)(165.08169687,138.51383332)(165.21711352,138.95133327)
\curveto(165.35253017,139.39404154)(165.49315515,139.7221665)(165.63898847,139.93570814)
\curveto(165.79003012,140.14924978)(166.04263425,140.38622892)(166.39680088,140.64664556)
\curveto(166.7509675,140.90706219)(167.12596745,141.1153955)(167.52180074,141.27164548)
\curveto(167.91763402,141.42789546)(168.3447173,141.50602045)(168.80305058,141.50602045)
\curveto(169.4280505,141.50602045)(169.9775296,141.35758297)(170.45148788,141.06070801)
\curveto(170.93065448,140.76904137)(171.26659194,140.39404142)(171.45930025,139.93570814)
\curveto(171.65200856,139.47737487)(171.74836272,138.99039576)(171.74836272,138.47477082)
\curveto(171.74836272,138.31331251)(171.74055022,138.15706253)(171.72492522,138.00602088)
\lineto(171.63898773,137.92008339)
\curveto(171.28482111,137.8419584)(170.80825867,137.78987508)(170.20930041,137.76383341)
\curveto(169.61034215,137.73779175)(169.21450886,137.72477092)(169.02180055,137.72477092)
\lineto(166.51398836,137.72477092)
\curveto(166.52440503,136.64664605)(166.79523833,135.85237531)(167.32648826,135.34195871)
\curveto(167.8577382,134.83154211)(168.50877978,134.5763338)(169.27961302,134.5763338)
\curveto(169.64419631,134.5763338)(169.9931546,134.6388338)(170.32648789,134.76383378)
\curveto(170.66502952,134.88883377)(171.02180031,135.06331291)(171.39680026,135.28727122)
\closepath
\moveto(166.51398836,138.34977084)
\curveto(166.60773835,138.33414584)(166.96711331,138.31591668)(167.59211323,138.29508335)
\curveto(168.22232148,138.27425002)(168.68846726,138.26383335)(168.99055056,138.26383335)
\curveto(169.7145088,138.26383335)(170.15461291,138.27685418)(170.31086289,138.30289585)
\curveto(170.31607123,138.42789583)(170.31867539,138.52424998)(170.31867539,138.59195831)
\curveto(170.31867539,139.39924988)(170.15461291,139.99820814)(169.82648795,140.38883309)
\curveto(169.49836299,140.78466637)(169.05044638,140.98258301)(168.48273812,140.98258301)
\curveto(167.86294653,140.98258301)(167.37857159,140.76122888)(167.0296133,140.3185206)
\curveto(166.68586334,139.87581232)(166.51398836,139.2195624)(166.51398836,138.34977084)
\closepath
\moveto(168.69367559,141.9435204)
\closepath
\moveto(168.59992561,133.53727143)
\closepath
}
}
{
\newrgbcolor{curcolor}{0 0 0}
\pscustom[linestyle=none,fillstyle=solid,fillcolor=curcolor]
{
\newpath
\moveto(175.27961228,141.45133296)
\lineto(175.42804976,141.34977047)
\curveto(175.39679977,141.04768717)(175.37336227,140.51122891)(175.35773727,139.74039567)
\lineto(175.9436747,140.48258308)
\curveto(176.13638301,140.72737471)(176.30565382,140.91487469)(176.45148714,141.04508301)
\curveto(176.60252878,141.18049966)(176.77700793,141.28466631)(176.97492457,141.35758297)
\curveto(177.17284121,141.43049963)(177.37596619,141.46695796)(177.5842995,141.46695796)
\curveto(177.81346614,141.46695796)(178.02961194,141.42008296)(178.23273692,141.32633297)
\lineto(178.28742441,141.21695799)
\curveto(178.20409109,140.52424974)(178.15721609,139.91487481)(178.14679943,139.38883321)
\lineto(177.79523697,139.38883321)
\curveto(177.58690366,139.86799982)(177.2509662,140.10758312)(176.7874246,140.10758312)
\curveto(176.46450797,140.10758312)(176.183258,140.00341647)(175.9436747,139.79508316)
\curveto(175.7040914,139.59195819)(175.54263308,139.33414572)(175.45929976,139.02164576)
\curveto(175.38117477,138.71435413)(175.34211227,138.32372918)(175.34211227,137.8497709)
\lineto(175.34211227,136.98258351)
\curveto(175.34211227,136.82633353)(175.35252894,136.44352107)(175.37336227,135.83414615)
\curveto(175.3941956,135.22477122)(175.41502893,134.87320877)(175.43586226,134.77945878)
\curveto(175.46190393,134.68570879)(175.49836225,134.6153963)(175.54523725,134.56852131)
\curveto(175.59732058,134.52685464)(175.66242473,134.49820881)(175.74054972,134.48258382)
\curveto(175.82388305,134.46695882)(176.19627884,134.44091715)(176.85773709,134.40445883)
\lineto(176.92804958,134.34195883)
\lineto(176.92804958,134.01383387)
\lineto(176.85773709,133.94352138)
\curveto(176.16502884,133.98518804)(175.44107059,134.00602138)(174.68586235,134.00602138)
\curveto(174.0869041,134.00602138)(173.46190417,133.98518804)(172.81086259,133.94352138)
\lineto(172.74055009,134.01383387)
\lineto(172.74055009,134.34195883)
\lineto(172.81086259,134.40445883)
\curveto(173.31607086,134.43570882)(173.59992499,134.45914632)(173.66242498,134.47477132)
\curveto(173.73013331,134.49039632)(173.78742497,134.52164631)(173.83429996,134.56852131)
\curveto(173.88638329,134.62060463)(173.92023745,134.69612546)(173.93586245,134.79508378)
\curveto(173.95669578,134.89925043)(173.97752911,135.23518789)(173.99836244,135.80289615)
\curveto(174.0244041,136.37581275)(174.03742493,136.80810436)(174.03742493,137.09977099)
\lineto(174.03742493,138.6388333)
\curveto(174.03742493,138.84716661)(174.02700827,139.13102074)(174.00617494,139.4903957)
\curveto(173.98534161,139.84977065)(173.96711244,140.07112479)(173.95148745,140.15445812)
\curveto(173.94107078,140.23779144)(173.90200828,140.29768727)(173.83429996,140.33414559)
\curveto(173.76659163,140.37581226)(173.63117498,140.39664559)(173.42805001,140.39664559)
\lineto(172.77180009,140.40445809)
\lineto(172.7014876,140.46695808)
\lineto(172.7014876,140.80289554)
\lineto(172.76398759,140.86539553)
\curveto(173.75877914,140.98518718)(174.5973207,141.18049966)(175.27961228,141.45133296)
\closepath
}
}
{
\newrgbcolor{curcolor}{0 0 0}
\pscustom[linestyle=none,fillstyle=solid,fillcolor=curcolor]
{
\newpath
\moveto(179.38898677,136.30289609)
\lineto(179.72492423,136.30289609)
\lineto(179.79523672,136.2325836)
\curveto(179.80565339,135.79508365)(179.83169505,135.4200837)(179.87336171,135.10758374)
\curveto(180.03482003,134.8627921)(180.31346583,134.66227129)(180.70929911,134.50602131)
\curveto(181.1051324,134.35497967)(181.49575735,134.27945884)(181.88117397,134.27945884)
\curveto(182.43325723,134.27945884)(182.87336135,134.42789632)(183.2014863,134.72477129)
\curveto(183.5348196,135.02164625)(183.70148624,135.37581287)(183.70148624,135.78727116)
\curveto(183.70148624,136.01122946)(183.64159042,136.20393777)(183.52179877,136.36539608)
\curveto(183.40200711,136.53206273)(183.21971547,136.66747938)(182.97492383,136.77164603)
\curveto(182.73534053,136.88102102)(182.30304892,137.00862517)(181.67804899,137.15445849)
\curveto(181.14159073,137.27945847)(180.76659077,137.37581263)(180.55304913,137.44352095)
\curveto(180.33950749,137.51643761)(180.13377835,137.63622926)(179.93586171,137.80289591)
\curveto(179.73794507,137.96956255)(179.58690342,138.17268753)(179.48273676,138.41227083)
\curveto(179.37857011,138.65706247)(179.32648678,138.9252916)(179.32648678,139.21695823)
\curveto(179.32648678,139.91487481)(179.60252842,140.47216641)(180.15461168,140.88883303)
\curveto(180.71190328,141.31070797)(181.40461153,141.52164545)(182.23273642,141.52164545)
\curveto(182.58169471,141.52164545)(182.961903,141.47216629)(183.37336128,141.37320797)
\curveto(183.78481957,141.27945798)(184.09731953,141.18831216)(184.31086117,141.0997705)
\lineto(184.38117366,140.99039551)
\curveto(184.339507,140.78206221)(184.31346533,140.23518727)(184.30304867,139.34977072)
\lineto(184.23273618,139.27945822)
\lineto(183.92023622,139.27945822)
\lineto(183.84992372,139.34977072)
\curveto(183.82909039,139.66747901)(183.80044456,139.88883315)(183.76398624,140.01383313)
\curveto(183.72752791,140.14404145)(183.63377792,140.28206227)(183.48273627,140.42789558)
\curveto(183.33169462,140.57893723)(183.11815298,140.70393722)(182.84211135,140.80289554)
\curveto(182.56606972,140.90706219)(182.27440309,140.95914552)(181.96711146,140.95914552)
\curveto(181.64940316,140.95914552)(181.38117403,140.91227052)(181.16242406,140.81852054)
\curveto(180.94888242,140.72477055)(180.77440327,140.5815414)(180.63898662,140.38883309)
\curveto(180.5087783,140.20133311)(180.44367414,139.97216647)(180.44367414,139.70133317)
\curveto(180.44367414,139.50341653)(180.48273664,139.32633322)(180.56086163,139.17008324)
\curveto(180.64419495,139.01383326)(180.76919494,138.88622911)(180.93586158,138.78727079)
\curveto(181.10252823,138.6935208)(181.27700738,138.62581247)(181.45929902,138.58414581)
\lineto(182.31086141,138.36539584)
\curveto(183.04002799,138.18831253)(183.56086126,138.03466671)(183.87336122,137.90445839)
\curveto(184.19106952,137.77425008)(184.43586115,137.57633343)(184.60773613,137.31070847)
\curveto(184.77961111,137.05029183)(184.8655486,136.73258354)(184.8655486,136.35758359)
\curveto(184.8655486,135.63883367)(184.56867364,135.02164625)(183.97492371,134.50602131)
\curveto(183.38117378,133.99039638)(182.61815304,133.73258391)(181.68586149,133.73258391)
\curveto(181.35773653,133.73258391)(180.94888242,133.76904224)(180.45929914,133.8419589)
\curveto(179.9749242,133.91487555)(179.56086175,133.99560471)(179.2171118,134.08414637)
\lineto(179.1780493,134.18570885)
\lineto(179.25617429,134.70914629)
\curveto(179.28221595,134.8706046)(179.29784095,135.03206292)(179.30304929,135.19352123)
\curveto(179.30825762,135.36018787)(179.31346595,135.706542)(179.31867428,136.2325836)
\closepath
}
}
{
\newrgbcolor{curcolor}{0 0 0}
\pscustom[linestyle=none,fillstyle=solid,fillcolor=curcolor]
{
\newpath
\moveto(188.0842982,141.45133296)
\lineto(188.23273568,141.34977047)
\curveto(188.17544402,140.66747889)(188.14679819,139.80029149)(188.14679819,138.74820829)
\lineto(188.14679819,136.92789601)
\curveto(188.14679819,136.26122943)(188.19627736,135.79768782)(188.29523568,135.53727119)
\curveto(188.39940233,135.27685455)(188.57648564,135.07633374)(188.82648561,134.93570876)
\curveto(189.07648558,134.80029211)(189.37336054,134.73258379)(189.7171105,134.73258379)
\curveto(190.09731879,134.73258379)(190.44367291,134.80550044)(190.75617287,134.95133376)
\curveto(191.06867283,135.10237541)(191.32908947,135.31331288)(191.53742278,135.58414618)
\curveto(191.75096442,135.85497948)(191.87336024,136.03727112)(191.90461023,136.13102111)
\curveto(191.93586023,136.2247711)(191.95669356,136.4877919)(191.96711022,136.92008352)
\lineto(191.99054772,137.79508341)
\lineto(191.99054772,138.6388333)
\curveto(191.99054772,138.84716661)(191.98013106,139.13102074)(191.95929772,139.4903957)
\curveto(191.93846439,139.84977065)(191.92023523,140.07112479)(191.90461023,140.15445812)
\curveto(191.89419357,140.23779144)(191.85513107,140.29768727)(191.78742275,140.33414559)
\curveto(191.71971442,140.37581226)(191.58429777,140.39664559)(191.3811728,140.39664559)
\lineto(190.72492288,140.40445809)
\lineto(190.65461039,140.46695808)
\lineto(190.65461039,140.80289554)
\lineto(190.71711038,140.86539553)
\curveto(191.71190192,140.98518718)(192.55044349,141.18049966)(193.23273507,141.45133296)
\lineto(193.38117255,141.34977047)
\curveto(193.32388089,140.66747889)(193.29523506,139.80029149)(193.29523506,138.74820829)
\lineto(193.29523506,137.37320846)
\curveto(193.29523506,137.29508347)(193.30825589,136.65706271)(193.33429756,135.4591462)
\curveto(193.34471422,135.01643792)(193.37075588,134.74560462)(193.41242255,134.6466463)
\curveto(193.45929754,134.55289631)(193.52179753,134.48518798)(193.59992252,134.44352132)
\curveto(193.67804751,134.40706299)(193.90981832,134.38883383)(194.29523494,134.38883383)
\lineto(194.51398491,134.38883383)
\lineto(194.5842974,134.32633384)
\lineto(194.5842974,134.01383387)
\lineto(194.52179741,133.94352138)
\curveto(193.7874225,133.98518804)(193.29523506,134.00602138)(193.04523509,134.00602138)
\curveto(192.7275268,134.00602138)(192.36033934,133.98779221)(191.94367273,133.95133388)
\lineto(191.87336024,134.01383387)
\curveto(191.90461023,134.54508381)(191.92804773,134.99560459)(191.94367273,135.36539621)
\curveto(191.61554777,135.11018791)(191.24054781,134.76904211)(190.81867287,134.34195883)
\curveto(190.65721455,134.18050052)(190.42544375,134.04508387)(190.12336045,133.93570888)
\curveto(189.82127716,133.8263339)(189.4775272,133.7716464)(189.09211058,133.7716464)
\curveto(188.50877732,133.7716464)(188.05304821,133.86279223)(187.72492325,134.04508387)
\curveto(187.40200662,134.23258385)(187.17283998,134.48779215)(187.03742333,134.81070878)
\curveto(186.90200668,135.13883374)(186.83429836,135.70133367)(186.83429836,136.49820857)
\lineto(186.84211086,137.11539599)
\lineto(186.84211086,138.6388333)
\curveto(186.84211086,138.84716661)(186.83169419,139.13102074)(186.81086086,139.4903957)
\curveto(186.79002753,139.84977065)(186.77179836,140.07112479)(186.75617337,140.15445812)
\curveto(186.7457567,140.23779144)(186.70669421,140.29768727)(186.63898588,140.33414559)
\curveto(186.57127756,140.37581226)(186.43586091,140.39664559)(186.23273593,140.39664559)
\lineto(185.57648601,140.40445809)
\lineto(185.50617352,140.46695808)
\lineto(185.50617352,140.80289554)
\lineto(185.56867351,140.86539553)
\curveto(186.56346506,140.98518718)(187.40200662,141.18049966)(188.0842982,141.45133296)
\closepath
\moveto(189.95148547,141.9435204)
\closepath
\moveto(190.02961046,133.53727143)
\closepath
}
}
{
\newrgbcolor{curcolor}{0 0 0}
\pscustom[linestyle=none,fillstyle=solid,fillcolor=curcolor]
{
\newpath
\moveto(201.81085901,134.84195877)
\lineto(201.49835905,134.31852134)
\curveto(200.82648413,133.94872972)(200.07648422,133.7638339)(199.24835933,133.7638339)
\curveto(198.10252613,133.7638339)(197.21710958,134.1049797)(196.59210965,134.78727128)
\curveto(195.96710973,135.46956286)(195.65460977,136.34977109)(195.65460977,137.42789595)
\curveto(195.65460977,137.91747923)(195.7066931,138.35237501)(195.81085975,138.73258329)
\curveto(195.92023474,139.11799991)(196.05565139,139.43570821)(196.2171097,139.68570817)
\curveto(196.38377635,139.93570814)(196.56606799,140.13362479)(196.76398463,140.2794581)
\curveto(196.96190127,140.42529142)(197.29523457,140.62841639)(197.76398451,140.88883303)
\curveto(198.23794278,141.14924966)(198.59471357,141.31852047)(198.83429688,141.39664546)
\curveto(199.07388018,141.47997879)(199.40460931,141.52164545)(199.82648425,141.52164545)
\curveto(200.63377582,141.52164545)(201.29523407,141.36799963)(201.81085901,141.06070801)
\curveto(201.71190069,140.47216641)(201.63637986,139.81591649)(201.58429654,139.09195825)
\lineto(201.51398405,139.02945826)
\lineto(201.19367159,139.02945826)
\lineto(201.12335909,139.09977075)
\curveto(201.10252576,139.53727069)(201.07387993,139.83154149)(201.03742161,139.98258314)
\curveto(201.00096328,140.13362479)(200.79262997,140.2950831)(200.41242168,140.46695808)
\curveto(200.03742173,140.63883306)(199.63638011,140.72477055)(199.20929683,140.72477055)
\curveto(198.78221355,140.72477055)(198.40200526,140.63362472)(198.06867197,140.45133308)
\curveto(197.73533868,140.27424977)(197.48273454,139.97997897)(197.31085956,139.56852069)
\curveto(197.14419292,139.16227074)(197.0608596,138.67529163)(197.0608596,138.10758337)
\curveto(197.0608596,137.62841676)(197.12856792,137.15966682)(197.26398457,136.70133354)
\curveto(197.39940122,136.2482086)(197.57648453,135.87060448)(197.79523451,135.56852118)
\curveto(198.01919281,135.27164622)(198.31346361,135.03206292)(198.6780469,134.84977127)
\curveto(199.04263018,134.66747963)(199.44888013,134.5763338)(199.89679675,134.5763338)
\curveto(200.17804671,134.5763338)(200.45148418,134.61279213)(200.71710914,134.68570879)
\curveto(200.98794244,134.75862545)(201.29783824,134.8784171)(201.64679653,135.04508375)
\closepath
}
}
{
\newrgbcolor{curcolor}{0 0 0}
\pscustom[linestyle=none,fillstyle=solid,fillcolor=curcolor]
{
\newpath
\moveto(204.78742114,145.57633245)
\lineto(204.93585862,145.48258246)
\curveto(204.87856697,144.8836242)(204.84992114,143.68310352)(204.84992114,141.8810204)
\lineto(204.84992114,139.95133314)
\curveto(205.28742108,140.3107081)(205.69367103,140.66747889)(206.06867098,141.02164551)
\curveto(206.17804597,141.12060383)(206.28221263,141.19352049)(206.38117095,141.24039548)
\curveto(206.48012927,141.28727048)(206.63898341,141.33674964)(206.85773339,141.38883296)
\curveto(207.08169169,141.44091629)(207.31085833,141.46695796)(207.5452333,141.46695796)
\curveto(207.94106659,141.46695796)(208.32387904,141.3862288)(208.69367066,141.22477048)
\curveto(209.06867062,141.06331217)(209.34992058,140.87060386)(209.53742056,140.64664556)
\curveto(209.73012887,140.42789558)(209.86294135,140.16747895)(209.93585801,139.86539565)
\curveto(210.00877467,139.56331236)(210.04523299,139.19091657)(210.04523299,138.74820829)
\lineto(210.04523299,137.37320846)
\curveto(210.04523299,137.2846668)(210.05825383,136.63622938)(210.08429549,135.4278962)
\curveto(210.09992049,134.93310459)(210.15460798,134.6388338)(210.24835797,134.54508381)
\curveto(210.34731629,134.45133382)(210.67023292,134.40445883)(211.21710785,134.40445883)
\lineto(211.27960784,134.34195883)
\lineto(211.27960784,134.00602138)
\lineto(211.21710785,133.94352138)
\curveto(210.5556496,133.98518804)(210.10773299,134.00602138)(209.87335802,134.00602138)
\curveto(209.75356636,134.00602138)(209.37335808,133.98518804)(208.73273316,133.94352138)
\lineto(208.63898317,134.02945887)
\curveto(208.70669149,134.70654212)(208.74054566,135.58935451)(208.74054566,136.67789605)
\lineto(208.74054566,137.70133342)
\curveto(208.74054566,138.31591668)(208.72492066,138.75862496)(208.69367066,139.02945826)
\curveto(208.667629,139.30029156)(208.58169151,139.54508319)(208.43585819,139.76383317)
\curveto(208.29002488,139.98779147)(208.0947124,140.15966645)(207.84992077,140.2794581)
\curveto(207.61033746,140.40445809)(207.31867083,140.46695808)(206.97492087,140.46695808)
\curveto(206.62075425,140.46695808)(206.32387929,140.41747892)(206.08429598,140.3185206)
\curveto(205.84992101,140.22477061)(205.61033771,140.06331229)(205.36554607,139.83414566)
\curveto(205.12075444,139.60497902)(204.97231695,139.41227071)(204.92023363,139.25602073)
\curveto(204.87335863,139.10497908)(204.84992114,138.81070828)(204.84992114,138.37320834)
\lineto(204.84992114,136.98258351)
\curveto(204.84992114,136.86800019)(204.8603378,136.49820857)(204.88117113,135.87320864)
\curveto(204.90200446,135.25341705)(204.92283779,134.89143793)(204.94367112,134.78727128)
\curveto(204.96971279,134.68310462)(205.00617112,134.6075838)(205.05304611,134.56070881)
\curveto(205.0999211,134.51383381)(205.15721276,134.48258382)(205.22492109,134.46695882)
\curveto(205.29262941,134.45654215)(205.57387938,134.43570882)(206.06867098,134.40445883)
\lineto(206.13898348,134.34195883)
\lineto(206.13898348,134.01383387)
\lineto(206.07648348,133.94352138)
\curveto(205.4254419,133.98518804)(204.79783781,134.00602138)(204.19367122,134.00602138)
\curveto(203.59471296,134.00602138)(202.96971303,133.98518804)(202.31867145,133.94352138)
\lineto(202.24835896,134.01383387)
\lineto(202.24835896,134.34195883)
\lineto(202.31867145,134.40445883)
\curveto(202.82387972,134.43570882)(203.10773385,134.45914632)(203.17023384,134.47477132)
\curveto(203.23794217,134.49039632)(203.29523383,134.52164631)(203.34210882,134.56852131)
\curveto(203.39419215,134.62060463)(203.42804631,134.69612546)(203.44367131,134.79508378)
\curveto(203.46450464,134.89925043)(203.48533797,135.23518789)(203.5061713,135.80289615)
\curveto(203.53221296,136.37581275)(203.5452338,136.80810436)(203.5452338,137.09977099)
\lineto(203.5452338,141.14664549)
\lineto(203.5139838,142.80289529)
\curveto(203.4983588,143.38102022)(203.48012964,143.78206184)(203.45929631,144.00602014)
\curveto(203.44367131,144.22997845)(203.42023381,144.36279093)(203.38898382,144.40445759)
\curveto(203.35773382,144.44612425)(203.30304633,144.47737425)(203.22492134,144.49820758)
\curveto(203.14679635,144.51904091)(202.83169222,144.52945758)(202.27960895,144.52945758)
\lineto(202.20929646,144.59977007)
\lineto(202.20929646,144.92789503)
\lineto(202.27179645,144.99820752)
\curveto(203.266588,145.11279084)(204.10512956,145.30549915)(204.78742114,145.57633245)
\closepath
}
}
{
\newrgbcolor{curcolor}{0 0 0}
\pscustom[linestyle=none,fillstyle=solid,fillcolor=curcolor]
{
\newpath
\moveto(214.14679499,141.45133296)
\lineto(214.29523247,141.34977047)
\curveto(214.23794081,140.66747889)(214.20929498,139.80029149)(214.20929498,138.74820829)
\lineto(214.20929498,136.92789601)
\curveto(214.20929498,136.26122943)(214.25877414,135.79768782)(214.35773246,135.53727119)
\curveto(214.46189912,135.27685455)(214.63898243,135.07633374)(214.8889824,134.93570876)
\curveto(215.13898237,134.80029211)(215.43585733,134.73258379)(215.77960729,134.73258379)
\curveto(216.15981557,134.73258379)(216.5061697,134.80550044)(216.81866966,134.95133376)
\curveto(217.13116962,135.10237541)(217.39158626,135.31331288)(217.59991956,135.58414618)
\curveto(217.8134612,135.85497948)(217.93585702,136.03727112)(217.96710702,136.13102111)
\curveto(217.99835701,136.2247711)(218.01919034,136.4877919)(218.02960701,136.92008352)
\lineto(218.05304451,137.79508341)
\lineto(218.05304451,138.6388333)
\curveto(218.05304451,138.84716661)(218.04262784,139.13102074)(218.02179451,139.4903957)
\curveto(218.00096118,139.84977065)(217.98273202,140.07112479)(217.96710702,140.15445812)
\curveto(217.95669035,140.23779144)(217.91762786,140.29768727)(217.84991953,140.33414559)
\curveto(217.78221121,140.37581226)(217.64679456,140.39664559)(217.44366958,140.39664559)
\lineto(216.78741966,140.40445809)
\lineto(216.71710717,140.46695808)
\lineto(216.71710717,140.80289554)
\lineto(216.77960716,140.86539553)
\curveto(217.77439871,140.98518718)(218.61294027,141.18049966)(219.29523185,141.45133296)
\lineto(219.44366934,141.34977047)
\curveto(219.38637768,140.66747889)(219.35773185,139.80029149)(219.35773185,138.74820829)
\lineto(219.35773185,137.37320846)
\curveto(219.35773185,137.29508347)(219.37075268,136.65706271)(219.39679434,135.4591462)
\curveto(219.40721101,135.01643792)(219.43325267,134.74560462)(219.47491933,134.6466463)
\curveto(219.52179433,134.55289631)(219.58429432,134.48518798)(219.66241931,134.44352132)
\curveto(219.7405443,134.40706299)(219.9723151,134.38883383)(220.35773172,134.38883383)
\lineto(220.5764817,134.38883383)
\lineto(220.64679419,134.32633384)
\lineto(220.64679419,134.01383387)
\lineto(220.5842942,133.94352138)
\curveto(219.84991929,133.98518804)(219.35773185,134.00602138)(219.10773188,134.00602138)
\curveto(218.79002358,134.00602138)(218.42283613,133.98779221)(218.00616951,133.95133388)
\lineto(217.93585702,134.01383387)
\curveto(217.96710702,134.54508381)(217.99054451,134.99560459)(218.00616951,135.36539621)
\curveto(217.67804455,135.11018791)(217.3030446,134.76904211)(216.88116965,134.34195883)
\curveto(216.71971134,134.18050052)(216.48794053,134.04508387)(216.18585724,133.93570888)
\curveto(215.88377394,133.8263339)(215.54002398,133.7716464)(215.15460736,133.7716464)
\curveto(214.5712741,133.7716464)(214.11554499,133.86279223)(213.78742003,134.04508387)
\curveto(213.46450341,134.23258385)(213.23533677,134.48779215)(213.09992012,134.81070878)
\curveto(212.96450347,135.13883374)(212.89679514,135.70133367)(212.89679514,136.49820857)
\lineto(212.90460764,137.11539599)
\lineto(212.90460764,138.6388333)
\curveto(212.90460764,138.84716661)(212.89419098,139.13102074)(212.87335765,139.4903957)
\curveto(212.85252432,139.84977065)(212.83429515,140.07112479)(212.81867015,140.15445812)
\curveto(212.80825349,140.23779144)(212.76919099,140.29768727)(212.70148267,140.33414559)
\curveto(212.63377434,140.37581226)(212.49835769,140.39664559)(212.29523272,140.39664559)
\lineto(211.6389828,140.40445809)
\lineto(211.56867031,140.46695808)
\lineto(211.56867031,140.80289554)
\lineto(211.6311703,140.86539553)
\curveto(212.62596184,140.98518718)(213.46450341,141.18049966)(214.14679499,141.45133296)
\closepath
\moveto(216.01398226,141.9435204)
\closepath
\moveto(216.09210725,133.53727143)
\closepath
}
}
{
\newrgbcolor{curcolor}{0 0 0}
\pscustom[linestyle=none,fillstyle=solid,fillcolor=curcolor]
{
\newpath
\moveto(223.7639813,141.45133296)
\lineto(223.91241878,141.34977047)
\curveto(223.88116879,141.00081218)(223.86033546,140.5424789)(223.84991879,139.97477064)
\curveto(224.26658541,140.3185206)(224.66241869,140.66747889)(225.03741865,141.02164551)
\curveto(225.14679363,141.12060383)(225.24835612,141.19352049)(225.34210611,141.24039548)
\curveto(225.44106443,141.28727048)(225.60252274,141.33674964)(225.82648105,141.38883296)
\curveto(226.05043935,141.44091629)(226.27960599,141.46695796)(226.51398096,141.46695796)
\curveto(226.90981425,141.46695796)(227.2926267,141.3862288)(227.66241832,141.22477048)
\curveto(228.03741828,141.06331217)(228.31866824,140.87060386)(228.50616822,140.64664556)
\curveto(228.69887653,140.42789558)(228.83168901,140.16747895)(228.90460567,139.86539565)
\curveto(228.97752233,139.56331236)(229.01398066,139.19091657)(229.01398066,138.74820829)
\lineto(229.01398066,137.37320846)
\curveto(229.01398066,137.2846668)(229.02700149,136.63622938)(229.05304315,135.4278962)
\curveto(229.06345982,134.93310459)(229.11814731,134.6388338)(229.21710563,134.54508381)
\curveto(229.31606395,134.45133382)(229.63898058,134.40445883)(230.18585551,134.40445883)
\lineto(230.2483555,134.34195883)
\lineto(230.2483555,134.00602138)
\lineto(230.18585551,133.94352138)
\curveto(229.52439726,133.98518804)(229.07648065,134.00602138)(228.84210568,134.00602138)
\curveto(228.70668903,134.00602138)(228.32648074,133.98518804)(227.70148082,133.94352138)
\lineto(227.60773083,134.02945887)
\curveto(227.67543915,134.70654212)(227.70929332,135.58935451)(227.70929332,136.67789605)
\lineto(227.70929332,137.70133342)
\curveto(227.70929332,138.31591668)(227.69366832,138.75862496)(227.66241832,139.02945826)
\curveto(227.63637666,139.30029156)(227.55043917,139.54508319)(227.40460585,139.76383317)
\curveto(227.25877254,139.98779147)(227.06346006,140.15966645)(226.81866843,140.2794581)
\curveto(226.57387679,140.40445809)(226.28221016,140.46695808)(225.94366853,140.46695808)
\curveto(225.67283523,140.46695808)(225.44106443,140.43831225)(225.24835612,140.38102059)
\curveto(225.05564781,140.32893726)(224.839502,140.21695811)(224.5999187,140.04508313)
\curveto(224.3603354,139.87841648)(224.18325208,139.71956234)(224.06866877,139.56852069)
\curveto(223.95408545,139.42268737)(223.88377296,139.28466656)(223.85773129,139.15445824)
\curveto(223.83689796,139.02945826)(223.8264813,138.75862496)(223.8264813,138.34195834)
\lineto(223.8264813,136.98258351)
\curveto(223.8264813,136.86800019)(223.83689796,136.49820857)(223.85773129,135.87320864)
\curveto(223.87856462,135.25341705)(223.89939795,134.89143793)(223.92023128,134.78727128)
\curveto(223.94627295,134.68310462)(223.98273128,134.6075838)(224.02960627,134.56070881)
\curveto(224.07648126,134.51383381)(224.13377292,134.48258382)(224.20148125,134.46695882)
\curveto(224.26918957,134.45654215)(224.55043954,134.43570882)(225.04523115,134.40445883)
\lineto(225.11554364,134.34195883)
\lineto(225.11554364,134.01383387)
\lineto(225.05304364,133.94352138)
\curveto(224.40200206,133.98518804)(223.77439797,134.00602138)(223.17023138,134.00602138)
\curveto(222.57127312,134.00602138)(221.94627319,133.98518804)(221.29523161,133.94352138)
\lineto(221.22491912,134.01383387)
\lineto(221.22491912,134.34195883)
\lineto(221.29523161,134.40445883)
\curveto(221.80043988,134.43570882)(222.08429401,134.45914632)(222.146794,134.47477132)
\curveto(222.21450233,134.49039632)(222.27179399,134.52164631)(222.31866898,134.56852131)
\curveto(222.37075231,134.62060463)(222.40460647,134.69612546)(222.42023147,134.79508378)
\curveto(222.4410648,134.89925043)(222.46189813,135.23518789)(222.48273146,135.80289615)
\curveto(222.50877312,136.37581275)(222.52179396,136.80810436)(222.52179396,137.09977099)
\lineto(222.52179396,138.6388333)
\curveto(222.52179396,138.84716661)(222.51137729,139.13102074)(222.49054396,139.4903957)
\curveto(222.46971063,139.84977065)(222.45148146,140.07112479)(222.43585647,140.15445812)
\curveto(222.4254398,140.23779144)(222.38637731,140.29768727)(222.31866898,140.33414559)
\curveto(222.25096066,140.37581226)(222.11554401,140.39664559)(221.91241903,140.39664559)
\lineto(221.25616911,140.40445809)
\lineto(221.18585662,140.46695808)
\lineto(221.18585662,140.80289554)
\lineto(221.24835661,140.86539553)
\curveto(222.24314816,140.98518718)(223.08168972,141.18049966)(223.7639813,141.45133296)
\closepath
}
}
{
\newrgbcolor{curcolor}{0 0 0}
\pscustom[linestyle=none,fillstyle=solid,fillcolor=curcolor]
{
\newpath
\moveto(239.02179192,140.91227052)
\lineto(239.08429191,140.77164554)
\curveto(238.87075027,140.44352058)(238.73793779,140.22997894)(238.68585446,140.13102062)
\lineto(237.23272964,140.13102062)
\curveto(237.37856296,139.84456232)(237.45147962,139.54508319)(237.45147962,139.23258323)
\curveto(237.45147962,138.86279161)(237.36554213,138.50341665)(237.19366715,138.15445836)
\curveto(237.0270005,137.80550007)(236.79262553,137.50862511)(236.49054223,137.26383347)
\curveto(236.18845894,137.01904184)(235.84470898,136.82372936)(235.45929236,136.67789605)
\curveto(235.07908407,136.53206273)(234.64418829,136.45914607)(234.15460502,136.45914607)
\lineto(233.79523007,136.45914607)
\curveto(233.48793844,136.21435444)(233.2900218,136.03206279)(233.20148014,135.91227114)
\curveto(233.11293848,135.79247949)(233.06866766,135.67008367)(233.06866766,135.54508369)
\curveto(233.06866766,135.31591705)(233.17283431,135.15445873)(233.38116762,135.06070875)
\curveto(233.59470926,134.96695876)(234.0322092,134.92008376)(234.69366746,134.92008376)
\lineto(236.44366724,134.94352126)
\curveto(236.9957505,134.94352126)(237.41762545,134.8784171)(237.70929208,134.74820878)
\curveto(238.00616705,134.61800047)(238.24575035,134.3836255)(238.42804199,134.04508387)
\curveto(238.61033364,133.71175058)(238.70147946,133.35758396)(238.70147946,132.982584)
\curveto(238.70147946,132.40966741)(238.51397948,131.83935498)(238.13897953,131.27164671)
\curveto(237.76918791,130.69873012)(237.23272964,130.26383434)(236.52960473,129.96695937)
\curveto(235.82647982,129.67008441)(235.07647991,129.52164693)(234.27960501,129.52164693)
\curveto(233.82648006,129.52164693)(233.39939678,129.57112609)(232.99835516,129.67008441)
\curveto(232.60252188,129.76904273)(232.25356359,129.92008438)(231.95148029,130.12320935)
\curveto(231.65460533,130.321126)(231.41502203,130.58154263)(231.23273038,130.90445926)
\curveto(231.05043874,131.22737588)(230.95929292,131.56331334)(230.95929292,131.91227163)
\curveto(230.95929292,132.10497994)(230.98793875,132.30810492)(231.04523041,132.52164656)
\curveto(231.10252206,132.72997987)(231.21189705,132.95654234)(231.37335536,133.20133397)
\lineto(232.78741769,133.99039638)
\curveto(232.33429275,134.13102136)(232.04783445,134.26904218)(231.9280428,134.40445883)
\curveto(231.81345948,134.54508381)(231.75616782,134.71175045)(231.75616782,134.90445876)
\curveto(231.75616782,135.10237541)(231.82127198,135.33675038)(231.95148029,135.60758368)
\lineto(233.20929264,136.51383357)
\curveto(232.4697094,136.70654188)(231.96970946,136.99560434)(231.70929282,137.38102096)
\curveto(231.45408452,137.76643758)(231.32648037,138.18831253)(231.32648037,138.6466458)
\curveto(231.32648037,139.06852075)(231.41762619,139.46956237)(231.59991784,139.84977065)
\curveto(231.78220948,140.23518727)(232.05564695,140.5424789)(232.42023024,140.77164554)
\curveto(232.79002186,141.00081218)(233.20148014,141.18049966)(233.65460508,141.31070797)
\curveto(234.11293836,141.44612462)(234.52960498,141.51383295)(234.90460493,141.51383295)
\curveto(235.55564652,141.51383295)(236.17543811,141.30289548)(236.7639797,140.88102053)
\curveto(237.70668792,140.88102053)(238.45929199,140.89143719)(239.02179192,140.91227052)
\closepath
\moveto(232.6702302,139.09977075)
\curveto(232.6702302,138.78206245)(232.7327302,138.44091666)(232.85773018,138.07633337)
\curveto(232.98273017,137.71175008)(233.18325098,137.43570845)(233.45929261,137.24820848)
\curveto(233.74054257,137.0607085)(234.0556467,136.96695851)(234.40460499,136.96695851)
\curveto(234.8681466,136.96695851)(235.26658405,137.11800016)(235.59991734,137.42008345)
\curveto(235.93845897,137.72216675)(236.10772978,138.19352086)(236.10772978,138.83414578)
\curveto(236.10772978,139.37581238)(235.95408397,139.86279149)(235.64679234,140.2950831)
\curveto(235.33950071,140.72737471)(234.90460493,140.94352052)(234.342105,140.94352052)
\curveto(233.86814672,140.94352052)(233.46970927,140.78727054)(233.14679265,140.47477058)
\curveto(232.82908435,140.16747895)(232.6702302,139.70914567)(232.6702302,139.09977075)
\closepath
\moveto(234.95929242,133.85758389)
\curveto(234.12075086,133.85758389)(233.63637592,133.84716723)(233.5061676,133.8263339)
\curveto(233.38116762,133.80550057)(233.20148014,133.71695891)(232.96710517,133.56070893)
\curveto(232.7327302,133.40445895)(232.54002189,133.19352148)(232.38898024,132.92789651)
\curveto(232.24314692,132.65706321)(232.17023027,132.35497991)(232.17023027,132.02164662)
\curveto(232.17023027,131.43310503)(232.38898024,130.95393842)(232.82648019,130.5841468)
\curveto(233.26398013,130.21435518)(233.84731339,130.02945937)(234.57647997,130.02945937)
\curveto(235.1233549,130.02945937)(235.62856317,130.14404268)(236.09210478,130.37320932)
\curveto(236.55564639,130.60237596)(236.89679218,130.90966759)(237.11554216,131.29508421)
\curveto(237.33950046,131.68050083)(237.45147962,132.05029245)(237.45147962,132.40445907)
\curveto(237.45147962,132.7586257)(237.35772963,133.06331316)(237.17022965,133.31852146)
\curveto(236.98793801,133.56852143)(236.7483547,133.72216724)(236.45147974,133.7794589)
\curveto(236.15981311,133.83154223)(235.66241734,133.85758389)(234.95929242,133.85758389)
\closepath
}
}
{
\newrgbcolor{curcolor}{0 0 0}
\pscustom[linestyle=none,fillstyle=solid,fillcolor=curcolor]
{
\newpath
\moveto(240.09210429,136.30289609)
\lineto(240.42804175,136.30289609)
\lineto(240.49835424,136.2325836)
\curveto(240.50877091,135.79508365)(240.53481257,135.4200837)(240.57647923,135.10758374)
\curveto(240.73793754,134.8627921)(241.01658334,134.66227129)(241.41241663,134.50602131)
\curveto(241.80824991,134.35497967)(242.19887486,134.27945884)(242.58429148,134.27945884)
\curveto(243.13637475,134.27945884)(243.57647886,134.42789632)(243.90460382,134.72477129)
\curveto(244.23793711,135.02164625)(244.40460376,135.37581287)(244.40460376,135.78727116)
\curveto(244.40460376,136.01122946)(244.34470793,136.20393777)(244.22491628,136.36539608)
\curveto(244.10512463,136.53206273)(243.92283298,136.66747938)(243.67804135,136.77164603)
\curveto(243.43845804,136.88102102)(243.00616643,137.00862517)(242.38116651,137.15445849)
\curveto(241.84470824,137.27945847)(241.46970829,137.37581263)(241.25616665,137.44352095)
\curveto(241.04262501,137.51643761)(240.83689586,137.63622926)(240.63897922,137.80289591)
\curveto(240.44106258,137.96956255)(240.29002093,138.17268753)(240.18585428,138.41227083)
\curveto(240.08168762,138.65706247)(240.0296043,138.9252916)(240.0296043,139.21695823)
\curveto(240.0296043,139.91487481)(240.30564593,140.47216641)(240.8577292,140.88883303)
\curveto(241.41502079,141.31070797)(242.10772904,141.52164545)(242.93585394,141.52164545)
\curveto(243.28481223,141.52164545)(243.66502052,141.47216629)(244.0764788,141.37320797)
\curveto(244.48793708,141.27945798)(244.80043704,141.18831216)(245.01397868,141.0997705)
\lineto(245.08429117,140.99039551)
\curveto(245.04262451,140.78206221)(245.01658285,140.23518727)(245.00616618,139.34977072)
\lineto(244.93585369,139.27945822)
\lineto(244.62335373,139.27945822)
\lineto(244.55304124,139.34977072)
\curveto(244.53220791,139.66747901)(244.50356208,139.88883315)(244.46710375,140.01383313)
\curveto(244.43064542,140.14404145)(244.33689543,140.28206227)(244.18585379,140.42789558)
\curveto(244.03481214,140.57893723)(243.8212705,140.70393722)(243.54522886,140.80289554)
\curveto(243.26918723,140.90706219)(242.9775206,140.95914552)(242.67022897,140.95914552)
\curveto(242.35252068,140.95914552)(242.08429154,140.91227052)(241.86554157,140.81852054)
\curveto(241.65199993,140.72477055)(241.47752079,140.5815414)(241.34210414,140.38883309)
\curveto(241.21189582,140.20133311)(241.14679166,139.97216647)(241.14679166,139.70133317)
\curveto(241.14679166,139.50341653)(241.18585415,139.32633322)(241.26397915,139.17008324)
\curveto(241.34731247,139.01383326)(241.47231245,138.88622911)(241.6389791,138.78727079)
\curveto(241.80564575,138.6935208)(241.98012489,138.62581247)(242.16241653,138.58414581)
\lineto(243.01397893,138.36539584)
\curveto(243.74314551,138.18831253)(244.26397878,138.03466671)(244.57647874,137.90445839)
\curveto(244.89418703,137.77425008)(245.13897867,137.57633343)(245.31085365,137.31070847)
\curveto(245.48272863,137.05029183)(245.56866611,136.73258354)(245.56866611,136.35758359)
\curveto(245.56866611,135.63883367)(245.27179115,135.02164625)(244.67804122,134.50602131)
\curveto(244.0842913,133.99039638)(243.32127056,133.73258391)(242.38897901,133.73258391)
\curveto(242.06085405,133.73258391)(241.65199993,133.76904224)(241.16241666,133.8419589)
\curveto(240.67804172,133.91487555)(240.26397927,133.99560471)(239.92022931,134.08414637)
\lineto(239.88116682,134.18570885)
\lineto(239.95929181,134.70914629)
\curveto(239.98533347,134.8706046)(240.00095847,135.03206292)(240.0061668,135.19352123)
\curveto(240.01137513,135.36018787)(240.01658347,135.706542)(240.0217918,136.2325836)
\closepath
}
}
{
\newrgbcolor{curcolor}{0 0 0}
\pscustom[linestyle=none,fillstyle=solid,fillcolor=curcolor]
{
\newpath
\moveto(246.506166,133.94352138)
\lineto(246.43585351,134.00602138)
\lineto(246.36554102,134.30289634)
\curveto(247.03741593,135.01643792)(247.68064502,135.75862533)(248.29522828,136.52945856)
\lineto(250.02179057,138.6700833)
\curveto(250.56345717,139.32633322)(251.06866544,140.00341647)(251.53741538,140.70133305)
\lineto(249.69366561,140.70133305)
\curveto(249.54783229,140.70133305)(249.21189483,140.68831222)(248.68585323,140.66227055)
\curveto(248.15981163,140.64143722)(247.868145,140.59977056)(247.81085334,140.53727057)
\curveto(247.75356168,140.47997891)(247.67543669,140.14143729)(247.57647837,139.52164569)
\lineto(247.52960337,139.27164573)
\lineto(247.46710338,139.20914573)
\lineto(247.15460342,139.20914573)
\lineto(247.08429093,139.27945822)
\curveto(247.13637425,140.10237479)(247.16241592,140.57893723)(247.16241592,140.70914555)
\curveto(247.16241592,140.83414553)(247.15720759,141.01122884)(247.14679092,141.24039548)
\lineto(247.20929091,141.29508298)
\curveto(249.14679067,141.25341631)(250.69106132,141.23258298)(251.84210284,141.23258298)
\curveto(252.4566861,141.23258298)(252.94106104,141.25341631)(253.29522766,141.29508298)
\lineto(253.35772765,141.23258298)
\lineto(253.35772765,140.85758303)
\curveto(252.9983527,140.46174975)(252.43845693,139.77945816)(251.67804036,138.81070828)
\lineto(249.63116561,136.2091461)
\curveto(249.02699902,135.4435212)(248.60251991,134.88883377)(248.35772827,134.54508381)
\lineto(248.9046032,134.54508381)
\lineto(250.46710301,134.56852131)
\curveto(251.04001961,134.57372964)(251.52179038,134.59716714)(251.91241533,134.6388338)
\curveto(252.30304028,134.68050046)(252.52960276,134.71435462)(252.59210275,134.74039628)
\curveto(252.65981107,134.77164628)(252.70147774,134.79768794)(252.71710273,134.81852127)
\curveto(252.73272773,134.84456294)(252.77699856,135.00862542)(252.84991522,135.31070871)
\curveto(252.92804021,135.61279201)(252.98793603,135.89404198)(253.02960269,136.15445861)
\lineto(253.09991519,136.2169586)
\lineto(253.45147764,136.2169586)
\lineto(253.52179013,136.14664611)
\curveto(253.39679015,135.21956289)(253.32647766,134.50602131)(253.31085266,134.00602138)
\lineto(253.24835267,133.94352138)
\curveto(251.68585286,133.99039638)(250.50095717,134.01383387)(249.69366561,134.01383387)
\curveto(248.98012403,134.01383387)(247.91762416,133.99039638)(246.506166,133.94352138)
\closepath
}
}
{
\newrgbcolor{curcolor}{0 0 0}
\pscustom[linestyle=none,fillstyle=solid,fillcolor=curcolor]
{
\newpath
\moveto(256.33428979,144.94352003)
\curveto(256.57908142,144.94352003)(256.78741473,144.85758254)(256.95928971,144.68570756)
\curveto(257.13116469,144.51383258)(257.21710218,144.30549927)(257.21710218,144.06070764)
\curveto(257.21710218,143.82112433)(257.13116469,143.61539519)(256.95928971,143.44352021)
\curveto(256.78741473,143.27164523)(256.57908142,143.18570774)(256.33428979,143.18570774)
\curveto(256.09470648,143.18570774)(255.88637318,143.26904107)(255.70928986,143.43570771)
\curveto(255.53741489,143.60758269)(255.4514774,143.815916)(255.4514774,144.06070764)
\curveto(255.4514774,144.30549927)(255.53741489,144.51383258)(255.70928986,144.68570756)
\curveto(255.88637318,144.85758254)(256.09470648,144.94352003)(256.33428979,144.94352003)
\closepath
\moveto(256.99835221,141.45133296)
\lineto(257.14678969,141.34977047)
\curveto(257.08949803,140.66747889)(257.0608522,139.80029149)(257.0608522,138.74820829)
\lineto(257.0608522,136.98258351)
\curveto(257.0608522,136.86800019)(257.07126886,136.49820857)(257.09210219,135.87320864)
\curveto(257.11293552,135.25341705)(257.13376886,134.89143793)(257.15460219,134.78727128)
\curveto(257.18064385,134.68310462)(257.21710218,134.6075838)(257.26397717,134.56070881)
\curveto(257.31085217,134.51383381)(257.36814383,134.48258382)(257.43585215,134.46695882)
\curveto(257.50356048,134.45654215)(257.78481044,134.43570882)(258.27960205,134.40445883)
\lineto(258.34991454,134.34195883)
\lineto(258.34991454,134.01383387)
\lineto(258.28741455,133.94352138)
\curveto(257.63637296,133.98518804)(257.00876887,134.00602138)(256.40460228,134.00602138)
\curveto(255.80564402,134.00602138)(255.1806441,133.98518804)(254.52960251,133.94352138)
\lineto(254.45929002,134.01383387)
\lineto(254.45929002,134.34195883)
\lineto(254.52960251,134.40445883)
\curveto(255.03481078,134.43570882)(255.31866491,134.45914632)(255.3811649,134.47477132)
\curveto(255.44887323,134.49039632)(255.50616489,134.52164631)(255.55303988,134.56852131)
\curveto(255.60512321,134.62060463)(255.63897737,134.69612546)(255.65460237,134.79508378)
\curveto(255.6754357,134.89925043)(255.69626903,135.23518789)(255.71710236,135.80289615)
\curveto(255.74314403,136.37581275)(255.75616486,136.80810436)(255.75616486,137.09977099)
\lineto(255.75616486,138.6388333)
\curveto(255.75616486,138.84716661)(255.74574819,139.13102074)(255.72491486,139.4903957)
\curveto(255.70408153,139.84977065)(255.68585237,140.07112479)(255.67022737,140.15445812)
\curveto(255.6598107,140.23779144)(255.62074821,140.29768727)(255.55303988,140.33414559)
\curveto(255.48533156,140.37581226)(255.34991491,140.39664559)(255.14678993,140.39664559)
\lineto(254.49054001,140.40445809)
\lineto(254.42022752,140.46695808)
\lineto(254.42022752,140.80289554)
\lineto(254.48272752,140.86539553)
\curveto(255.47751906,140.98518718)(256.31606062,141.18049966)(256.99835221,141.45133296)
\closepath
\moveto(256.38897728,133.53727143)
\closepath
}
}
{
\newrgbcolor{curcolor}{0 0 0}
\pscustom[linestyle=none,fillstyle=solid,fillcolor=curcolor]
{
\newpath
\moveto(265.64678864,135.15445873)
\lineto(265.39678867,134.5919588)
\curveto(264.85512207,134.24300051)(264.37074713,134.01643804)(263.94366385,133.91227139)
\curveto(263.5217889,133.80810473)(263.14418478,133.75602141)(262.81085149,133.75602141)
\curveto(262.18585157,133.75602141)(261.59210164,133.87841722)(261.02960171,134.12320886)
\curveto(260.47231011,134.3680005)(260.016581,134.78206295)(259.66241438,135.36539621)
\curveto(259.31345609,135.94872947)(259.13897694,136.65185438)(259.13897694,137.47477095)
\curveto(259.13897694,138.02164588)(259.20668527,138.51383332)(259.34210192,138.95133327)
\curveto(259.47751857,139.39404154)(259.61814355,139.7221665)(259.76397686,139.93570814)
\curveto(259.91501851,140.14924978)(260.16762265,140.38622892)(260.52178927,140.64664556)
\curveto(260.87595589,140.90706219)(261.25095585,141.1153955)(261.64678913,141.27164548)
\curveto(262.04262242,141.42789546)(262.4697057,141.50602045)(262.92803897,141.50602045)
\curveto(263.5530389,141.50602045)(264.102518,141.35758297)(264.57647627,141.06070801)
\curveto(265.05564288,140.76904137)(265.39158034,140.39404142)(265.58428865,139.93570814)
\curveto(265.77699696,139.47737487)(265.87335111,138.99039576)(265.87335111,138.47477082)
\curveto(265.87335111,138.31331251)(265.86553861,138.15706253)(265.84991361,138.00602088)
\lineto(265.76397612,137.92008339)
\curveto(265.4098095,137.8419584)(264.93324706,137.78987508)(264.3342888,137.76383341)
\curveto(263.73533054,137.73779175)(263.33949726,137.72477092)(263.14678895,137.72477092)
\lineto(260.63897676,137.72477092)
\curveto(260.64939342,136.64664605)(260.92022672,135.85237531)(261.45147666,135.34195871)
\curveto(261.98272659,134.83154211)(262.63376818,134.5763338)(263.40460142,134.5763338)
\curveto(263.7691847,134.5763338)(264.11814299,134.6388338)(264.45147629,134.76383378)
\curveto(264.79001791,134.88883377)(265.1467887,135.06331291)(265.52178865,135.28727122)
\closepath
\moveto(260.63897676,138.34977084)
\curveto(260.73272674,138.33414584)(261.0921017,138.31591668)(261.71710162,138.29508335)
\curveto(262.34730988,138.27425002)(262.81345565,138.26383335)(263.11553895,138.26383335)
\curveto(263.83949719,138.26383335)(264.27960131,138.27685418)(264.43585129,138.30289585)
\curveto(264.44105962,138.42789583)(264.44366379,138.52424998)(264.44366379,138.59195831)
\curveto(264.44366379,139.39924988)(264.27960131,139.99820814)(263.95147635,140.38883309)
\curveto(263.62335139,140.78466637)(263.17543478,140.98258301)(262.60772651,140.98258301)
\curveto(261.98793492,140.98258301)(261.50355998,140.76122888)(261.15460169,140.3185206)
\curveto(260.81085174,139.87581232)(260.63897676,139.2195624)(260.63897676,138.34977084)
\closepath
\moveto(262.81866399,141.9435204)
\closepath
\moveto(262.724914,133.53727143)
\closepath
}
}
{
\newrgbcolor{curcolor}{0 0 0}
\pscustom[linestyle=none,fillstyle=solid,fillcolor=curcolor]
{
\newpath
\moveto(269.32647569,145.57633245)
\lineto(269.47491317,145.48258246)
\curveto(269.41762151,144.8836242)(269.38897568,143.68310352)(269.38897568,141.8810204)
\lineto(269.38897568,136.98258351)
\curveto(269.38897568,136.86800019)(269.39939234,136.49820857)(269.42022567,135.87320864)
\curveto(269.441059,135.25341705)(269.46189234,134.89143793)(269.48272567,134.78727128)
\curveto(269.50876733,134.68310462)(269.54522566,134.6075838)(269.59210065,134.56070881)
\curveto(269.63897565,134.51383381)(269.69626731,134.48258382)(269.76397563,134.46695882)
\curveto(269.83168396,134.45654215)(270.11293392,134.43570882)(270.60772553,134.40445883)
\lineto(270.67803802,134.34195883)
\lineto(270.67803802,134.01383387)
\lineto(270.61553803,133.94352138)
\curveto(269.96449644,133.98518804)(269.33689235,134.00602138)(268.73272576,134.00602138)
\curveto(268.1337675,134.00602138)(267.50876758,133.98518804)(266.85772599,133.94352138)
\lineto(266.7874135,134.01383387)
\lineto(266.7874135,134.34195883)
\lineto(266.85772599,134.40445883)
\curveto(267.36293426,134.43570882)(267.64678839,134.45914632)(267.70928838,134.47477132)
\curveto(267.77699671,134.49039632)(267.83428837,134.52164631)(267.88116336,134.56852131)
\curveto(267.93324669,134.62060463)(267.96710085,134.69612546)(267.98272585,134.79508378)
\curveto(268.00355918,134.89925043)(268.02439251,135.23518789)(268.04522584,135.80289615)
\curveto(268.07126751,136.37581275)(268.08428834,136.80810436)(268.08428834,137.09977099)
\lineto(268.08428834,141.14664549)
\lineto(268.05303834,142.80289529)
\curveto(268.03741334,143.38102022)(268.01918418,143.78206184)(267.99835085,144.00602014)
\curveto(267.98272585,144.22997845)(267.95928835,144.36279093)(267.92803836,144.40445759)
\curveto(267.89678836,144.44612425)(267.84210087,144.47737425)(267.76397588,144.49820758)
\curveto(267.68585089,144.51904091)(267.37074676,144.52945758)(266.81866349,144.52945758)
\lineto(266.748351,144.59977007)
\lineto(266.748351,144.92789503)
\lineto(266.810851,144.99820752)
\curveto(267.80564254,145.11279084)(268.6441841,145.30549915)(269.32647569,145.57633245)
\closepath
}
}
{
\newrgbcolor{curcolor}{0 0 0}
\pscustom[linestyle=none,fillstyle=solid,fillcolor=curcolor]
{
\newpath
\moveto(271.84991287,136.30289609)
\lineto(272.18585033,136.30289609)
\lineto(272.25616282,136.2325836)
\curveto(272.26657949,135.79508365)(272.29262115,135.4200837)(272.33428781,135.10758374)
\curveto(272.49574613,134.8627921)(272.77439193,134.66227129)(273.17022521,134.50602131)
\curveto(273.5660585,134.35497967)(273.95668345,134.27945884)(274.34210007,134.27945884)
\curveto(274.89418333,134.27945884)(275.33428744,134.42789632)(275.6624124,134.72477129)
\curveto(275.9957457,135.02164625)(276.16241234,135.37581287)(276.16241234,135.78727116)
\curveto(276.16241234,136.01122946)(276.10251652,136.20393777)(275.98272486,136.36539608)
\curveto(275.86293321,136.53206273)(275.68064157,136.66747938)(275.43584993,136.77164603)
\curveto(275.19626663,136.88102102)(274.76397501,137.00862517)(274.13897509,137.15445849)
\curveto(273.60251682,137.27945847)(273.22751687,137.37581263)(273.01397523,137.44352095)
\curveto(272.80043359,137.51643761)(272.59470445,137.63622926)(272.39678781,137.80289591)
\curveto(272.19887116,137.96956255)(272.04782952,138.17268753)(271.94366286,138.41227083)
\curveto(271.83949621,138.65706247)(271.78741288,138.9252916)(271.78741288,139.21695823)
\curveto(271.78741288,139.91487481)(272.06345451,140.47216641)(272.61553778,140.88883303)
\curveto(273.17282938,141.31070797)(273.86553763,141.52164545)(274.69366252,141.52164545)
\curveto(275.04262081,141.52164545)(275.4228291,141.47216629)(275.83428738,141.37320797)
\curveto(276.24574567,141.27945798)(276.55824563,141.18831216)(276.77178727,141.0997705)
\lineto(276.84209976,140.99039551)
\curveto(276.8004331,140.78206221)(276.77439143,140.23518727)(276.76397477,139.34977072)
\lineto(276.69366228,139.27945822)
\lineto(276.38116232,139.27945822)
\lineto(276.31084982,139.34977072)
\curveto(276.29001649,139.66747901)(276.26137066,139.88883315)(276.22491233,140.01383313)
\curveto(276.18845401,140.14404145)(276.09470402,140.28206227)(275.94366237,140.42789558)
\curveto(275.79262072,140.57893723)(275.57907908,140.70393722)(275.30303745,140.80289554)
\curveto(275.02699582,140.90706219)(274.73532918,140.95914552)(274.42803756,140.95914552)
\curveto(274.11032926,140.95914552)(273.84210013,140.91227052)(273.62335016,140.81852054)
\curveto(273.40980851,140.72477055)(273.23532937,140.5815414)(273.09991272,140.38883309)
\curveto(272.9697044,140.20133311)(272.90460024,139.97216647)(272.90460024,139.70133317)
\curveto(272.90460024,139.50341653)(272.94366274,139.32633322)(273.02178773,139.17008324)
\curveto(273.10512105,139.01383326)(273.23012104,138.88622911)(273.39678768,138.78727079)
\curveto(273.56345433,138.6935208)(273.73793347,138.62581247)(273.92022512,138.58414581)
\lineto(274.77178751,138.36539584)
\curveto(275.50095409,138.18831253)(276.02178736,138.03466671)(276.33428732,137.90445839)
\curveto(276.65199562,137.77425008)(276.89678725,137.57633343)(277.06866223,137.31070847)
\curveto(277.24053721,137.05029183)(277.3264747,136.73258354)(277.3264747,136.35758359)
\curveto(277.3264747,135.63883367)(277.02959974,135.02164625)(276.43584981,134.50602131)
\curveto(275.84209988,133.99039638)(275.07907914,133.73258391)(274.14678759,133.73258391)
\curveto(273.81866263,133.73258391)(273.40980851,133.76904224)(272.92022524,133.8419589)
\curveto(272.4358503,133.91487555)(272.02178785,133.99560471)(271.6780379,134.08414637)
\lineto(271.6389754,134.18570885)
\lineto(271.71710039,134.70914629)
\curveto(271.74314205,134.8706046)(271.75876705,135.03206292)(271.76397538,135.19352123)
\curveto(271.76918372,135.36018787)(271.77439205,135.706542)(271.77960038,136.2325836)
\closepath
}
}
{
\newrgbcolor{curcolor}{1 1 1}
\pscustom[linestyle=none,fillstyle=solid,fillcolor=curcolor]
{
\newpath
\moveto(27.3012357,95.63944434)
\curveto(27.3012357,91.46469387)(23.91693177,88.08038994)(19.74218131,88.08038994)
\curveto(15.56743084,88.08038994)(12.18312691,91.46469387)(12.18312691,95.63944434)
\curveto(12.18312691,99.8141948)(15.56743084,103.19849873)(19.74218131,103.19849873)
\curveto(23.91693177,103.19849873)(27.3012357,99.8141948)(27.3012357,95.63944434)
\closepath
}
}
{
\newrgbcolor{curcolor}{0.15686275 0.16078432 0.16470589}
\pscustom[linewidth=2.88359956,linecolor=curcolor]
{
\newpath
\moveto(27.3012357,95.63944434)
\curveto(27.3012357,91.46469387)(23.91693177,88.08038994)(19.74218131,88.08038994)
\curveto(15.56743084,88.08038994)(12.18312691,91.46469387)(12.18312691,95.63944434)
\curveto(12.18312691,99.8141948)(15.56743084,103.19849873)(19.74218131,103.19849873)
\curveto(23.91693177,103.19849873)(27.3012357,99.8141948)(27.3012357,95.63944434)
\closepath
}
}
{
\newrgbcolor{curcolor}{0 0 0}
\pscustom[linestyle=none,fillstyle=solid,fillcolor=curcolor]
{
\newpath
\moveto(23.20311838,94.09256953)
\curveto(23.20311838,92.98319467)(22.73957677,92.04309061)(21.81249355,91.27225738)
\curveto(20.89061866,90.50663247)(19.80728546,90.12382002)(18.56249395,90.12382002)
\curveto(17.9010357,90.12382002)(17.26561911,90.26704917)(16.65624419,90.55350747)
\curveto(16.5624942,91.45975735)(16.43749421,92.22538226)(16.28124423,92.85038218)
\lineto(16.32811923,92.95975717)
\lineto(16.71874418,93.07694465)
\lineto(16.81249417,93.03788216)
\curveto(17.3281191,91.72017399)(18.15363983,91.0613199)(19.28905636,91.0613199)
\curveto(19.87759796,91.0613199)(20.34895206,91.25142405)(20.70311869,91.63163233)
\curveto(21.06249364,92.01704895)(21.24218112,92.54309055)(21.24218112,93.20975714)
\curveto(21.24218112,93.87121539)(21.04686864,94.38163199)(20.65624369,94.74100695)
\curveto(20.28645207,95.08475691)(19.7890563,95.25663189)(19.16405638,95.25663189)
\curveto(18.87759808,95.25663189)(18.56770228,95.20975689)(18.23436899,95.1160069)
\lineto(18.140619,95.21756939)
\curveto(18.25520232,95.58736101)(18.35416064,95.9831943)(18.43749397,96.40506924)
\lineto(19.07811889,96.40506924)
\curveto(19.5416605,96.40506924)(19.94270211,96.56913172)(20.28124374,96.89725668)
\curveto(20.61978536,97.23058998)(20.78905618,97.62642326)(20.78905618,98.08475654)
\curveto(20.78905618,98.51704815)(20.63801453,98.86861061)(20.33593123,99.13944391)
\curveto(20.03905627,99.41027721)(19.66666048,99.54569386)(19.21874387,99.54569386)
\curveto(18.51041062,99.54569386)(18.02863985,99.36340221)(17.77343155,98.99881892)
\curveto(17.6536399,98.82694395)(17.53905658,98.57954814)(17.42968159,98.25663152)
\lineto(17.3437441,97.99881905)
\lineto(17.23436911,97.92850656)
\lineto(16.83593166,97.95975655)
\lineto(16.75780667,98.06131904)
\lineto(17.03124414,99.33475638)
\curveto(17.10936913,99.69413134)(17.16666079,100.04569379)(17.20311912,100.38944375)
\curveto(17.88020237,100.89986036)(18.73176476,101.15506866)(19.7578063,101.15506866)
\curveto(20.62759786,101.15506866)(21.32030611,100.94152702)(21.83593105,100.51444374)
\curveto(22.32551432,100.09777712)(22.57030596,99.55350636)(22.57030596,98.88163144)
\curveto(22.57030596,97.91288156)(21.99218103,97.12642332)(20.83593117,96.52225673)
\curveto(21.41926443,96.52225673)(21.9479102,96.32434009)(22.42186847,95.9285068)
\curveto(22.94270174,95.49621519)(23.20311838,94.8842361)(23.20311838,94.09256953)
\closepath
}
}
{
\newrgbcolor{curcolor}{0 0 0}
\pscustom[linestyle=none,fillstyle=solid,fillcolor=curcolor]
{
\newpath
\moveto(34.14681718,90.62422686)
\lineto(34.07650469,90.68672685)
\lineto(34.07650469,91.0929768)
\lineto(34.14681718,91.15547679)
\curveto(34.52702547,91.16068512)(34.77181711,91.17891429)(34.88119209,91.21016429)
\curveto(34.99056708,91.24662261)(35.08692123,91.31433094)(35.17025456,91.41328926)
\curveto(35.25879621,91.51745591)(35.43327536,91.85078921)(35.69369199,92.41328914)
\lineto(36.51400439,94.19453892)
\lineto(38.63900413,99.00703832)
\curveto(39.04004575,99.94974654)(39.4384832,100.89245476)(39.83431648,101.83516298)
\lineto(40.34212892,101.83516298)
\lineto(43.67806601,93.99141394)
\lineto(44.12337845,92.94453907)
\curveto(44.25358677,92.64766411)(44.40202425,92.32474748)(44.5686909,91.97578919)
\curveto(44.74056588,91.6268309)(44.85775336,91.42110176)(44.92025336,91.35860177)
\curveto(44.98796168,91.29610177)(45.06869084,91.24662261)(45.16244083,91.21016429)
\curveto(45.26139915,91.17891429)(45.47754495,91.16068512)(45.81087825,91.15547679)
\lineto(45.87337824,91.0929768)
\lineto(45.87337824,90.68672685)
\lineto(45.81087825,90.62422686)
\curveto(45.25879498,90.67110185)(44.76139921,90.69453935)(44.31869093,90.69453935)
\curveto(43.4541077,90.66849769)(42.58952448,90.64506019)(41.72494125,90.62422686)
\lineto(41.66244126,90.68672685)
\lineto(41.66244126,91.0929768)
\lineto(41.72494125,91.15547679)
\curveto(42.30827451,91.16589346)(42.6728578,91.18933095)(42.81869111,91.22578928)
\curveto(42.96452443,91.26745594)(43.03744109,91.35860177)(43.03744109,91.49922675)
\curveto(43.03744109,91.61381007)(43.00098276,91.76485172)(42.9280661,91.95235169)
\lineto(41.96712872,94.3351639)
\lineto(37.44369178,94.3351639)
\lineto(36.59212938,92.33516415)
\curveto(36.44108773,91.97578919)(36.36556691,91.71276839)(36.36556691,91.54610174)
\curveto(36.36556691,91.41589343)(36.4358794,91.31953927)(36.57650438,91.25703928)
\curveto(36.71712937,91.19453929)(37.07129599,91.16068512)(37.63900425,91.15547679)
\lineto(37.70931674,91.0929768)
\lineto(37.70931674,90.68672685)
\lineto(37.64681675,90.62422686)
\curveto(36.97494183,90.66589352)(36.38640024,90.68672685)(35.88119197,90.68672685)
\curveto(35.30827537,90.68672685)(34.73015044,90.66589352)(34.14681718,90.62422686)
\closepath
\moveto(37.70931674,95.00703882)
\lineto(41.66244126,95.00703882)
\lineto(39.701504,99.71797574)
\closepath
\moveto(40.07650395,102.30391292)
\closepath
\moveto(39.89681647,90.21797691)
\closepath
}
}
{
\newrgbcolor{curcolor}{0 0 0}
\pscustom[linestyle=none,fillstyle=solid,fillcolor=curcolor]
{
\newpath
\moveto(47.8186905,98.13203843)
\lineto(47.96712798,98.03047594)
\curveto(47.90983632,97.34818436)(47.88119049,96.48099697)(47.88119049,95.42891377)
\lineto(47.88119049,93.60860149)
\curveto(47.88119049,92.94193491)(47.93066965,92.4783933)(48.02962797,92.21797666)
\curveto(48.13379463,91.95756003)(48.31087794,91.75703922)(48.56087791,91.61641424)
\curveto(48.81087788,91.48099759)(49.10775284,91.41328926)(49.4515028,91.41328926)
\curveto(49.83171108,91.41328926)(50.17806521,91.48620592)(50.49056517,91.63203923)
\curveto(50.80306513,91.78308088)(51.06348176,91.99401836)(51.27181507,92.26485166)
\curveto(51.48535671,92.53568496)(51.60775253,92.7179766)(51.63900253,92.81172659)
\curveto(51.67025252,92.90547658)(51.69108585,93.16849738)(51.70150252,93.60078899)
\lineto(51.72494002,94.47578888)
\lineto(51.72494002,95.31953878)
\curveto(51.72494002,95.52787209)(51.71452335,95.81172622)(51.69369002,96.17110117)
\curveto(51.67285669,96.53047613)(51.65462752,96.75183027)(51.63900253,96.83516359)
\curveto(51.62858586,96.91849691)(51.58952337,96.97839274)(51.52181504,97.01485107)
\curveto(51.45410672,97.05651773)(51.31869007,97.07735106)(51.11556509,97.07735106)
\lineto(50.45931517,97.08516356)
\lineto(50.38900268,97.14766355)
\lineto(50.38900268,97.48360101)
\lineto(50.45150267,97.546101)
\curveto(51.44629422,97.66589266)(52.28483578,97.86120513)(52.96712736,98.13203843)
\lineto(53.11556484,98.03047594)
\curveto(53.05827319,97.34818436)(53.02962736,96.48099697)(53.02962736,95.42891377)
\lineto(53.02962736,94.05391393)
\curveto(53.02962736,93.97578894)(53.04264819,93.33776819)(53.06868985,92.13985167)
\curveto(53.07910652,91.69714339)(53.10514818,91.42631009)(53.14681484,91.32735177)
\curveto(53.19368984,91.23360178)(53.25618983,91.16589346)(53.33431482,91.1242268)
\curveto(53.41243981,91.08776847)(53.64421061,91.0695393)(54.02962723,91.0695393)
\lineto(54.2483772,91.0695393)
\lineto(54.3186897,91.00703931)
\lineto(54.3186897,90.69453935)
\lineto(54.2561897,90.62422686)
\curveto(53.52181479,90.66589352)(53.02962736,90.68672685)(52.77962739,90.68672685)
\curveto(52.46191909,90.68672685)(52.09473164,90.66849769)(51.67806502,90.63203936)
\lineto(51.60775253,90.69453935)
\curveto(51.63900253,91.22578928)(51.66244002,91.67631006)(51.67806502,92.04610168)
\curveto(51.34994006,91.79089338)(50.97494011,91.44974759)(50.55306516,91.02266431)
\curveto(50.39160685,90.86120599)(50.15983604,90.72578935)(49.85775275,90.61641436)
\curveto(49.55566945,90.50703937)(49.21191949,90.45235188)(48.82650287,90.45235188)
\curveto(48.24316961,90.45235188)(47.7874405,90.5434977)(47.45931554,90.72578935)
\curveto(47.13639892,90.91328932)(46.90723228,91.16849762)(46.77181563,91.49141425)
\curveto(46.63639898,91.81953921)(46.56869065,92.38203914)(46.56869065,93.17891404)
\lineto(46.57650315,93.79610147)
\lineto(46.57650315,95.31953878)
\curveto(46.57650315,95.52787209)(46.56608649,95.81172622)(46.54525315,96.17110117)
\curveto(46.52441982,96.53047613)(46.50619066,96.75183027)(46.49056566,96.83516359)
\curveto(46.480149,96.91849691)(46.4410865,96.97839274)(46.37337818,97.01485107)
\curveto(46.30566985,97.05651773)(46.1702532,97.07735106)(45.96712823,97.07735106)
\lineto(45.31087831,97.08516356)
\lineto(45.24056582,97.14766355)
\lineto(45.24056582,97.48360101)
\lineto(45.30306581,97.546101)
\curveto(46.29785735,97.66589266)(47.13639892,97.86120513)(47.8186905,98.13203843)
\closepath
\moveto(49.68587777,98.62422587)
\closepath
\moveto(49.76400276,90.21797691)
\closepath
}
}
{
\newrgbcolor{curcolor}{0 0 0}
\pscustom[linestyle=none,fillstyle=solid,fillcolor=curcolor]
{
\newpath
\moveto(55.55306454,92.98360157)
\lineto(55.889002,92.98360157)
\lineto(55.95931449,92.91328908)
\curveto(55.96973116,92.47578913)(55.99577282,92.10078918)(56.03743948,91.78828921)
\curveto(56.1988978,91.54349758)(56.4775436,91.34297677)(56.87337688,91.18672679)
\curveto(57.26921017,91.03568514)(57.65983512,90.96016432)(58.04525174,90.96016432)
\curveto(58.597335,90.96016432)(59.03743911,91.1086018)(59.36556407,91.40547676)
\curveto(59.69889737,91.70235172)(59.86556401,92.05651835)(59.86556401,92.46797663)
\curveto(59.86556401,92.69193494)(59.80566819,92.88464325)(59.68587653,93.04610156)
\curveto(59.56608488,93.21276821)(59.38379324,93.34818485)(59.1390016,93.45235151)
\curveto(58.8994183,93.5617265)(58.46712668,93.68933065)(57.84212676,93.83516396)
\curveto(57.30566849,93.96016395)(56.93066854,94.0565181)(56.7171269,94.12422643)
\curveto(56.50358526,94.19714308)(56.29785612,94.31693474)(56.09993948,94.48360138)
\curveto(55.90202283,94.65026803)(55.75098119,94.853393)(55.64681453,95.09297631)
\curveto(55.54264788,95.33776794)(55.49056455,95.60599708)(55.49056455,95.89766371)
\curveto(55.49056455,96.59558029)(55.76660618,97.15287189)(56.31868945,97.5695385)
\curveto(56.87598105,97.99141345)(57.5686893,98.20235092)(58.39681419,98.20235092)
\curveto(58.74577248,98.20235092)(59.12598077,98.15287176)(59.53743905,98.05391344)
\curveto(59.94889734,97.96016345)(60.2613973,97.86901763)(60.47493894,97.78047598)
\lineto(60.54525143,97.67110099)
\curveto(60.50358477,97.46276768)(60.4775431,96.91589275)(60.46712644,96.03047619)
\lineto(60.39681395,95.9601637)
\lineto(60.08431399,95.9601637)
\lineto(60.01400149,96.03047619)
\curveto(59.99316816,96.34818449)(59.96452233,96.56953862)(59.928064,96.69453861)
\curveto(59.89160568,96.82474693)(59.79785569,96.96276774)(59.64681404,97.10860106)
\curveto(59.49577239,97.25964271)(59.28223075,97.38464269)(59.00618912,97.48360101)
\curveto(58.73014749,97.58776767)(58.43848085,97.63985099)(58.13118923,97.63985099)
\curveto(57.81348093,97.63985099)(57.5452518,97.592976)(57.32650183,97.49922601)
\curveto(57.11296019,97.40547602)(56.93848104,97.26224687)(56.80306439,97.06953856)
\curveto(56.67285607,96.88203859)(56.60775191,96.65287195)(56.60775191,96.38203865)
\curveto(56.60775191,96.18412201)(56.64681441,96.00703869)(56.7249394,95.85078871)
\curveto(56.80827272,95.69453873)(56.93327271,95.56693458)(57.09993935,95.46797626)
\curveto(57.266606,95.37422627)(57.44108514,95.30651795)(57.62337679,95.26485129)
\lineto(58.47493918,95.04610131)
\curveto(59.20410576,94.869018)(59.72493903,94.71537219)(60.03743899,94.58516387)
\curveto(60.35514729,94.45495555)(60.59993892,94.25703891)(60.7718139,93.99141394)
\curveto(60.94368888,93.73099731)(61.02962637,93.41328901)(61.02962637,93.03828906)
\curveto(61.02962637,92.31953915)(60.73275141,91.70235172)(60.13900148,91.18672679)
\curveto(59.54525155,90.67110185)(58.78223081,90.41328938)(57.84993926,90.41328938)
\curveto(57.5218143,90.41328938)(57.11296019,90.44974771)(56.62337691,90.52266437)
\curveto(56.13900197,90.59558103)(55.72493952,90.67631018)(55.38118957,90.76485184)
\lineto(55.34212707,90.86641433)
\lineto(55.42025206,91.38985176)
\curveto(55.44629372,91.55131008)(55.46191872,91.71276839)(55.46712705,91.8742267)
\curveto(55.47233539,92.04089335)(55.47754372,92.38724747)(55.48275205,92.91328908)
\closepath
}
}
{
\newrgbcolor{curcolor}{0 0 0}
\pscustom[linestyle=none,fillstyle=solid,fillcolor=curcolor]
{
\newpath
\moveto(64.89681339,90.53828937)
\lineto(63.84212602,94.03828894)
\curveto(63.64420938,94.69453886)(63.45150107,95.30391378)(63.26400109,95.86641371)
\curveto(63.08170945,96.42891364)(62.95931363,96.78568443)(62.89681364,96.93672608)
\curveto(62.83952198,97.09297606)(62.77702199,97.20495521)(62.70931366,97.27266354)
\curveto(62.64160534,97.34037186)(62.56868868,97.38724686)(62.49056369,97.41328852)
\curveto(62.4124387,97.43933018)(62.19108456,97.47318435)(61.82650127,97.51485101)
\lineto(61.77181378,97.577351)
\lineto(61.77181378,97.89766346)
\lineto(61.84212627,97.96797595)
\curveto(62.21712622,97.93151762)(62.79785532,97.91328846)(63.58431355,97.91328846)
\curveto(64.43327178,97.91328846)(65.07129254,97.93151762)(65.49837582,97.96797595)
\lineto(65.56868831,97.89766346)
\lineto(65.56868831,97.577351)
\lineto(65.51400082,97.51485101)
\curveto(64.94629255,97.50964268)(64.60775093,97.47839268)(64.49837594,97.42110102)
\curveto(64.39420929,97.36380936)(64.34212596,97.2752677)(64.34212596,97.15547605)
\curveto(64.34212596,97.0356844)(64.41764678,96.71537194)(64.56868843,96.19453867)
\lineto(65.29525084,93.67891398)
\curveto(65.40983416,93.28828903)(65.55566748,92.83255992)(65.73275079,92.31172665)
\curveto(65.87337577,92.65026827)(66.07129241,93.10339322)(66.32650072,93.67110148)
\lineto(67.47493807,96.24141366)
\curveto(67.75097971,96.85078859)(68.00618801,97.46537185)(68.24056298,98.08516344)
\lineto(68.62337543,98.08516344)
\lineto(69.32650035,96.19453867)
\lineto(70.22493774,93.92891395)
\lineto(70.92025015,92.21016416)
\lineto(71.50618758,93.74922647)
\curveto(71.8447292,94.6294347)(72.10775,95.36380961)(72.29524998,95.9523512)
\curveto(72.48274996,96.54089279)(72.57649995,96.91849691)(72.57649995,97.08516356)
\curveto(72.57649995,97.25183021)(72.51920829,97.36120519)(72.40462497,97.41328852)
\curveto(72.29004165,97.46537185)(71.98275002,97.49922601)(71.48275008,97.51485101)
\lineto(71.41243759,97.577351)
\lineto(71.41243759,97.90547596)
\lineto(71.48275008,97.96797595)
\curveto(72.08691667,97.93151762)(72.59472911,97.91328846)(73.00618739,97.91328846)
\curveto(73.52702066,97.91328846)(74.01660393,97.93151762)(74.47493721,97.96797595)
\lineto(74.5374372,97.90547596)
\lineto(74.5374372,97.5851635)
\lineto(74.47493721,97.51485101)
\curveto(74.19368725,97.50443434)(74.00097894,97.47318435)(73.89681228,97.42110102)
\curveto(73.79785396,97.36901769)(73.68327064,97.24141354)(73.55306232,97.03828857)
\curveto(73.42806234,96.84037192)(73.23535403,96.43412197)(72.9749374,95.81953872)
\lineto(72.20149999,93.98360144)
\lineto(71.84993753,93.10860155)
\lineto(71.2952501,91.65547673)
\curveto(71.12337512,91.20235179)(70.99837514,90.829956)(70.92025015,90.53828937)
\lineto(70.18587524,90.53828937)
\lineto(69.81087529,91.51485175)
\lineto(68.02962551,95.9914137)
\lineto(67.2796256,94.3742264)
\lineto(66.41243821,92.43672663)
\lineto(65.6468133,90.53828937)
\closepath
}
}
{
\newrgbcolor{curcolor}{0 0 0}
\pscustom[linestyle=none,fillstyle=solid,fillcolor=curcolor]
{
\newpath
\moveto(76.11556201,95.9836012)
\lineto(75.81087455,96.06172619)
\lineto(75.74837455,96.13985118)
\lineto(75.74837455,97.10860106)
\curveto(76.67545777,97.79610097)(77.57649933,98.13985093)(78.45149922,98.13985093)
\curveto(79.05045748,98.13985093)(79.54785325,98.02787178)(79.94368654,97.80391347)
\curveto(80.33951982,97.57995517)(80.62597812,97.2987052)(80.80306143,96.96016358)
\curveto(80.98014474,96.62683028)(81.0686864,96.23620533)(81.0686864,95.78828872)
\lineto(81.0296239,94.21797641)
\lineto(81.0296239,91.8898517)
\curveto(81.0296239,91.57214341)(81.05045723,91.3794351)(81.0921239,91.31172677)
\curveto(81.13899889,91.24401845)(81.19108222,91.19714345)(81.24837388,91.17110179)
\curveto(81.30566554,91.15026846)(81.41243636,91.13203929)(81.56868634,91.1164143)
\lineto(82.01399878,91.0773518)
\lineto(82.07649877,91.00703931)
\lineto(82.07649877,90.69453935)
\lineto(82.01399878,90.63203936)
\curveto(81.63379049,90.66328935)(81.27962387,90.67891435)(80.95149891,90.67891435)
\curveto(80.63899895,90.67891435)(80.263999,90.66328935)(79.82649905,90.63203936)
\lineto(79.70931157,90.74141434)
\lineto(79.74056156,91.99922669)
\lineto(78.03743677,90.67110185)
\curveto(77.75097847,90.5513102)(77.43847851,90.49141437)(77.09993689,90.49141437)
\curveto(76.68327027,90.49141437)(76.32389532,90.5669352)(76.02181202,90.71797685)
\curveto(75.72493706,90.86901849)(75.49577042,91.07995597)(75.33431211,91.35078927)
\curveto(75.17806212,91.62162257)(75.09993713,91.94974753)(75.09993713,92.33516415)
\curveto(75.09993713,93.10078905)(75.33952044,93.69714315)(75.81868705,94.12422643)
\curveto(76.29785365,94.55651804)(77.60514516,94.92370549)(79.74056156,95.22578879)
\curveto(79.74056156,95.99662203)(79.56868658,96.53828863)(79.22493663,96.85078859)
\curveto(78.88118667,97.16849688)(78.42024922,97.32735103)(77.8421243,97.32735103)
\curveto(77.540041,97.32735103)(77.26399937,97.2830802)(77.0139994,97.19453855)
\curveto(76.76920776,97.10599689)(76.62858278,97.03308023)(76.59212445,96.97578857)
\curveto(76.55566612,96.92370525)(76.41504114,96.60599695)(76.1702495,96.02266369)
\closepath
\moveto(79.74056156,94.74922635)
\curveto(78.28222841,94.50443471)(77.37337435,94.24141391)(77.0139994,93.96016395)
\curveto(76.65462444,93.67891398)(76.47493696,93.2440182)(76.47493696,92.65547661)
\curveto(76.47493696,91.84297671)(76.87858275,91.43672676)(77.68587432,91.43672676)
\curveto(78.37858256,91.43672676)(79.06347831,91.83776837)(79.74056156,92.63985161)
\closepath
\moveto(78.41243673,98.62422587)
\closepath
\moveto(78.24837425,90.21797691)
\closepath
}
}
{
\newrgbcolor{curcolor}{0 0 0}
\pscustom[linestyle=none,fillstyle=solid,fillcolor=curcolor]
{
\newpath
\moveto(85.1155609,102.25703792)
\lineto(85.26399838,102.16328793)
\curveto(85.20670672,101.56432968)(85.17806089,100.36380899)(85.17806089,98.56172588)
\lineto(85.17806089,96.63203862)
\curveto(85.61556084,96.99141357)(86.02181079,97.34818436)(86.39681074,97.70235098)
\curveto(86.50618573,97.80130931)(86.61035238,97.87422596)(86.7093107,97.92110096)
\curveto(86.80826902,97.96797595)(86.96712317,98.01745511)(87.18587314,98.06953844)
\curveto(87.40983145,98.12162177)(87.63899809,98.14766343)(87.87337306,98.14766343)
\curveto(88.26920634,98.14766343)(88.6520188,98.06693427)(89.02181042,97.90547596)
\curveto(89.39681037,97.74401765)(89.67806034,97.55130934)(89.86556031,97.32735103)
\curveto(90.05826862,97.10860106)(90.19108111,96.84818442)(90.26399776,96.54610113)
\curveto(90.33691442,96.24401783)(90.37337275,95.87162204)(90.37337275,95.42891377)
\lineto(90.37337275,94.05391393)
\curveto(90.37337275,93.96537228)(90.38639358,93.31693486)(90.41243525,92.10860167)
\curveto(90.42806024,91.61381007)(90.48274774,91.31953927)(90.57649773,91.22578928)
\curveto(90.67545605,91.13203929)(90.99837267,91.0851643)(91.54524761,91.0851643)
\lineto(91.6077476,91.02266431)
\lineto(91.6077476,90.68672685)
\lineto(91.54524761,90.62422686)
\curveto(90.88378935,90.66589352)(90.43587274,90.68672685)(90.20149777,90.68672685)
\curveto(90.08170612,90.68672685)(89.70149783,90.66589352)(89.06087291,90.62422686)
\lineto(88.96712292,90.71016435)
\curveto(89.03483125,91.3872476)(89.06868541,92.27005999)(89.06868541,93.35860152)
\lineto(89.06868541,94.38203889)
\curveto(89.06868541,94.99662215)(89.05306041,95.43933043)(89.02181042,95.71016373)
\curveto(88.99576875,95.98099703)(88.90983126,96.22578867)(88.76399795,96.44453864)
\curveto(88.61816463,96.66849695)(88.42285216,96.84037192)(88.17806052,96.96016358)
\curveto(87.93847722,97.08516356)(87.64681059,97.14766355)(87.30306063,97.14766355)
\curveto(86.94889401,97.14766355)(86.65201904,97.09818439)(86.41243574,96.99922607)
\curveto(86.17806077,96.90547608)(85.93847746,96.74401777)(85.69368583,96.51485113)
\curveto(85.44889419,96.28568449)(85.30045671,96.09297618)(85.24837338,95.9367262)
\curveto(85.20149839,95.78568455)(85.17806089,95.49141376)(85.17806089,95.05391381)
\lineto(85.17806089,93.66328898)
\curveto(85.17806089,93.54870566)(85.18847756,93.17891404)(85.20931089,92.55391412)
\curveto(85.23014422,91.93412253)(85.25097755,91.57214341)(85.27181088,91.46797675)
\curveto(85.29785254,91.3638101)(85.33431087,91.28828928)(85.38118587,91.24141428)
\curveto(85.42806086,91.19453929)(85.48535252,91.16328929)(85.55306085,91.14766429)
\curveto(85.62076917,91.13724763)(85.90201914,91.1164143)(86.39681074,91.0851643)
\lineto(86.46712323,91.02266431)
\lineto(86.46712323,90.69453935)
\lineto(86.40462324,90.62422686)
\curveto(85.75358165,90.66589352)(85.12597756,90.68672685)(84.52181097,90.68672685)
\curveto(83.92285271,90.68672685)(83.29785279,90.66589352)(82.6468112,90.62422686)
\lineto(82.57649871,90.69453935)
\lineto(82.57649871,91.02266431)
\lineto(82.6468112,91.0851643)
\curveto(83.15201947,91.1164143)(83.43587361,91.13985179)(83.4983736,91.15547679)
\curveto(83.56608192,91.17110179)(83.62337358,91.20235179)(83.67024858,91.24922678)
\curveto(83.7223319,91.30131011)(83.75618607,91.37683093)(83.77181106,91.47578925)
\curveto(83.7926444,91.57995591)(83.81347773,91.91589336)(83.83431106,92.48360163)
\curveto(83.86035272,93.05651822)(83.87337355,93.48880984)(83.87337355,93.78047647)
\lineto(83.87337355,97.82735097)
\lineto(83.84212356,99.48360077)
\curveto(83.82649856,100.06172569)(83.80826939,100.46276731)(83.78743606,100.68672562)
\curveto(83.77181106,100.91068392)(83.74837357,101.04349641)(83.71712357,101.08516307)
\curveto(83.68587358,101.12682973)(83.63118608,101.15807973)(83.55306109,101.17891306)
\curveto(83.4749361,101.19974639)(83.15983197,101.21016305)(82.60774871,101.21016305)
\lineto(82.53743622,101.28047554)
\lineto(82.53743622,101.6086005)
\lineto(82.59993621,101.67891299)
\curveto(83.59472775,101.79349631)(84.43326932,101.98620462)(85.1155609,102.25703792)
\closepath
}
}
{
\newrgbcolor{curcolor}{0 0 0}
\pscustom[linestyle=none,fillstyle=solid,fillcolor=curcolor]
{
\newpath
\moveto(94.69368472,102.25703792)
\lineto(94.8421222,102.16328793)
\curveto(94.78483054,101.56432968)(94.75618471,100.36380899)(94.75618471,98.56172588)
\lineto(94.75618471,93.66328898)
\curveto(94.75618471,93.54870566)(94.76660138,93.17891404)(94.78743471,92.55391412)
\curveto(94.80826804,91.93412253)(94.82910137,91.57214341)(94.8499347,91.46797675)
\curveto(94.87597636,91.3638101)(94.91243469,91.28828928)(94.95930969,91.24141428)
\curveto(95.00618468,91.19453929)(95.06347634,91.16328929)(95.13118466,91.14766429)
\curveto(95.19889299,91.13724763)(95.48014295,91.1164143)(95.97493456,91.0851643)
\lineto(96.04524705,91.02266431)
\lineto(96.04524705,90.69453935)
\lineto(95.98274706,90.62422686)
\curveto(95.33170547,90.66589352)(94.70410138,90.68672685)(94.09993479,90.68672685)
\curveto(93.50097653,90.68672685)(92.87597661,90.66589352)(92.22493502,90.62422686)
\lineto(92.15462253,90.69453935)
\lineto(92.15462253,91.02266431)
\lineto(92.22493502,91.0851643)
\curveto(92.73014329,91.1164143)(93.01399743,91.13985179)(93.07649742,91.15547679)
\curveto(93.14420574,91.17110179)(93.2014974,91.20235179)(93.2483724,91.24922678)
\curveto(93.30045572,91.30131011)(93.33430989,91.37683093)(93.34993488,91.47578925)
\curveto(93.37076821,91.57995591)(93.39160155,91.91589336)(93.41243488,92.48360163)
\curveto(93.43847654,93.05651822)(93.45149737,93.48880984)(93.45149737,93.78047647)
\lineto(93.45149737,97.82735097)
\lineto(93.42024737,99.48360077)
\curveto(93.40462238,100.06172569)(93.38639321,100.46276731)(93.36555988,100.68672562)
\curveto(93.34993488,100.91068392)(93.32649739,101.04349641)(93.29524739,101.08516307)
\curveto(93.26399739,101.12682973)(93.2093099,101.15807973)(93.13118491,101.17891306)
\curveto(93.05305992,101.19974639)(92.73795579,101.21016305)(92.18587253,101.21016305)
\lineto(92.11556004,101.28047554)
\lineto(92.11556004,101.6086005)
\lineto(92.17806003,101.67891299)
\curveto(93.17285157,101.79349631)(94.01139314,101.98620462)(94.69368472,102.25703792)
\closepath
}
}
{
\newrgbcolor{curcolor}{0 0 0}
\pscustom[linestyle=none,fillstyle=solid,fillcolor=curcolor]
{
}
}
{
\newrgbcolor{curcolor}{0 0 0}
\pscustom[linestyle=none,fillstyle=solid,fillcolor=curcolor]
{
\newpath
\moveto(104.41243352,90.54610187)
\lineto(103.92805858,91.77266422)
\lineto(103.34993365,93.10860155)
\lineto(101.74837135,96.83516359)
\curveto(101.64420469,97.07995523)(101.55045471,97.24141354)(101.46712138,97.31953853)
\curveto(101.38899639,97.39766352)(101.30826724,97.44453852)(101.22493391,97.46016351)
\curveto(101.14160059,97.47578851)(100.97493394,97.48880934)(100.72493397,97.49922601)
\lineto(100.66243398,97.561726)
\lineto(100.66243398,97.89766346)
\lineto(100.73274647,97.96797595)
\curveto(101.46191305,97.93151762)(102.04003798,97.91328846)(102.46712126,97.91328846)
\curveto(103.07128785,97.91328846)(103.67805861,97.93151762)(104.28743353,97.96797595)
\lineto(104.34993353,97.90547596)
\lineto(104.34993353,97.561726)
\lineto(104.28743353,97.49922601)
\curveto(103.74576694,97.48360101)(103.43066281,97.44453852)(103.34212115,97.38203852)
\curveto(103.2535795,97.32474686)(103.20930867,97.24922604)(103.20930867,97.15547605)
\curveto(103.20930867,96.99922607)(103.31347532,96.66328861)(103.52180863,96.14766368)
\lineto(104.47493351,93.77266397)
\lineto(105.22493342,91.99922669)
\lineto(106.00618332,93.77266397)
\lineto(106.65462074,95.38203877)
\curveto(106.90462071,95.9914137)(107.06607903,96.42370531)(107.13899568,96.67891361)
\curveto(107.21191234,96.93412191)(107.24837067,97.10339273)(107.24837067,97.18672605)
\curveto(107.24837067,97.28568437)(107.19107901,97.35599686)(107.07649569,97.39766352)
\curveto(106.96191237,97.44453852)(106.67284991,97.47839268)(106.2093083,97.49922601)
\lineto(106.14680831,97.561726)
\lineto(106.14680831,97.89766346)
\lineto(106.2171208,97.96797595)
\curveto(106.86295405,97.93151762)(107.34732899,97.91328846)(107.67024562,97.91328846)
\curveto(107.94107892,97.91328846)(108.40462053,97.93151762)(109.06087045,97.96797595)
\lineto(109.13118294,97.89766346)
\lineto(109.13118294,97.561726)
\lineto(109.06868295,97.49922601)
\curveto(108.76139132,97.47318435)(108.55566218,97.42891352)(108.45149552,97.36641353)
\curveto(108.3525372,97.30391353)(108.25357888,97.18151772)(108.15462056,96.99922607)
\curveto(108.06087057,96.82214276)(107.78482894,96.26485116)(107.32649566,95.32735128)
\lineto(106.57649575,93.72578898)
\curveto(106.45149577,93.45495568)(106.00097499,92.39505997)(105.22493342,90.54610187)
\closepath
}
}
{
\newrgbcolor{curcolor}{0 0 0}
\pscustom[linestyle=none,fillstyle=solid,fillcolor=curcolor]
{
\newpath
\moveto(109.99837033,94.26485141)
\curveto(109.99837033,94.884643)(110.12597448,95.48099709)(110.38118278,96.05391369)
\curveto(110.63639109,96.63203862)(111.07909936,97.12943439)(111.70930762,97.546101)
\curveto(112.34472421,97.96797595)(113.09732828,98.17891343)(113.96711984,98.17891343)
\curveto(115.06086971,98.17891343)(115.9514946,97.8325593)(116.63899451,97.13985105)
\curveto(117.32649443,96.45235114)(117.67024438,95.56693458)(117.67024438,94.48360138)
\curveto(117.67024438,93.2908932)(117.2744111,92.30912248)(116.48274453,91.53828924)
\curveto(115.69628629,90.76745601)(114.74576558,90.38203939)(113.63118238,90.38203939)
\curveto(112.90201581,90.38203939)(112.25097422,90.57735186)(111.67805762,90.96797682)
\curveto(111.10514103,91.3638101)(110.68066191,91.84558087)(110.40462028,92.41328914)
\curveto(110.13378698,92.9809974)(109.99837033,93.59818482)(109.99837033,94.26485141)
\closepath
\moveto(111.47493265,94.80391384)
\curveto(111.47493265,94.07474727)(111.56868264,93.42110151)(111.75618261,92.84297658)
\curveto(111.94368259,92.27005999)(112.24055755,91.80912254)(112.6468075,91.46016425)
\curveto(113.05305745,91.11120596)(113.51659906,90.93672682)(114.03743233,90.93672682)
\curveto(114.65201559,90.93672682)(115.15982803,91.17891429)(115.56086964,91.66328923)
\curveto(115.96191126,92.1528725)(116.16243207,92.88724741)(116.16243207,93.86641396)
\curveto(116.16243207,94.97578882)(115.9436821,95.88464288)(115.50618215,96.59297612)
\curveto(115.07389054,97.30130937)(114.44889062,97.65547599)(113.63118238,97.65547599)
\curveto(112.95409913,97.65547599)(112.42545336,97.41068435)(112.04524508,96.92110108)
\curveto(111.66503679,96.43151781)(111.47493265,95.72578873)(111.47493265,94.80391384)
\closepath
\moveto(113.84211986,98.62422587)
\closepath
\moveto(113.72493237,90.21797691)
\closepath
}
}
{
\newrgbcolor{curcolor}{0 0 0}
\pscustom[linestyle=none,fillstyle=solid,fillcolor=curcolor]
{
\newpath
\moveto(120.87336899,98.13203843)
\lineto(121.02180647,98.03047594)
\curveto(120.99055648,97.68151765)(120.96972314,97.22318438)(120.95930648,96.65547611)
\curveto(121.37597309,96.99922607)(121.77180638,97.34818436)(122.14680633,97.70235098)
\curveto(122.25618132,97.80130931)(122.35774381,97.87422596)(122.4514938,97.92110096)
\curveto(122.55045212,97.96797595)(122.71191043,98.01745511)(122.93586874,98.06953844)
\curveto(123.15982704,98.12162177)(123.38899368,98.14766343)(123.62336865,98.14766343)
\curveto(124.01920194,98.14766343)(124.40201439,98.06693427)(124.77180601,97.90547596)
\curveto(125.14680596,97.74401765)(125.42805593,97.55130934)(125.61555591,97.32735103)
\curveto(125.80826421,97.10860106)(125.9410767,96.84818442)(126.01399336,96.54610113)
\curveto(126.08691001,96.24401783)(126.12336834,95.87162204)(126.12336834,95.42891377)
\lineto(126.12336834,94.05391393)
\curveto(126.12336834,93.96537228)(126.13638917,93.31693486)(126.16243084,92.10860167)
\curveto(126.1728475,91.61381007)(126.227535,91.31953927)(126.32649332,91.22578928)
\curveto(126.42545164,91.13203929)(126.74836827,91.0851643)(127.2952432,91.0851643)
\lineto(127.35774319,91.02266431)
\lineto(127.35774319,90.68672685)
\lineto(127.2952432,90.62422686)
\curveto(126.63378495,90.66589352)(126.18586833,90.68672685)(125.95149336,90.68672685)
\curveto(125.81607671,90.68672685)(125.43586843,90.66589352)(124.8108685,90.62422686)
\lineto(124.71711852,90.71016435)
\curveto(124.78482684,91.3872476)(124.818681,92.27005999)(124.818681,93.35860152)
\lineto(124.818681,94.38203889)
\curveto(124.818681,94.99662215)(124.80305601,95.43933043)(124.77180601,95.71016373)
\curveto(124.74576435,95.98099703)(124.65982686,96.22578867)(124.51399354,96.44453864)
\curveto(124.36816023,96.66849695)(124.17284775,96.84037192)(123.92805611,96.96016358)
\curveto(123.68326448,97.08516356)(123.39159785,97.14766355)(123.05305622,97.14766355)
\curveto(122.78222292,97.14766355)(122.55045212,97.11901772)(122.35774381,97.06172606)
\curveto(122.1650355,97.00964274)(121.94888969,96.89766358)(121.70930639,96.72578861)
\curveto(121.46972308,96.55912196)(121.29263977,96.40026781)(121.17805645,96.24922616)
\curveto(121.06347313,96.10339285)(120.99316064,95.96537203)(120.96711898,95.83516372)
\curveto(120.94628565,95.71016373)(120.93586898,95.43933043)(120.93586898,95.02266382)
\lineto(120.93586898,93.66328898)
\curveto(120.93586898,93.54870566)(120.94628565,93.17891404)(120.96711898,92.55391412)
\curveto(120.98795231,91.93412253)(121.00878564,91.57214341)(121.02961897,91.46797675)
\curveto(121.05566063,91.3638101)(121.09211896,91.28828928)(121.13899396,91.24141428)
\curveto(121.18586895,91.19453929)(121.24316061,91.16328929)(121.31086894,91.14766429)
\curveto(121.37857726,91.13724763)(121.65982723,91.1164143)(122.15461883,91.0851643)
\lineto(122.22493132,91.02266431)
\lineto(122.22493132,90.69453935)
\lineto(122.16243133,90.62422686)
\curveto(121.51138974,90.66589352)(120.88378566,90.68672685)(120.27961906,90.68672685)
\curveto(119.6806608,90.68672685)(119.05566088,90.66589352)(118.40461929,90.62422686)
\lineto(118.3343068,90.69453935)
\lineto(118.3343068,91.02266431)
\lineto(118.40461929,91.0851643)
\curveto(118.90982757,91.1164143)(119.1936817,91.13985179)(119.25618169,91.15547679)
\curveto(119.32389001,91.17110179)(119.38118167,91.20235179)(119.42805667,91.24922678)
\curveto(119.48014,91.30131011)(119.51399416,91.37683093)(119.52961916,91.47578925)
\curveto(119.55045249,91.57995591)(119.57128582,91.91589336)(119.59211915,92.48360163)
\curveto(119.61816081,93.05651822)(119.63118164,93.48880984)(119.63118164,93.78047647)
\lineto(119.63118164,95.31953878)
\curveto(119.63118164,95.52787209)(119.62076498,95.81172622)(119.59993165,96.17110117)
\curveto(119.57909832,96.53047613)(119.56086915,96.75183027)(119.54524415,96.83516359)
\curveto(119.53482749,96.91849691)(119.49576499,96.97839274)(119.42805667,97.01485107)
\curveto(119.36034834,97.05651773)(119.22493169,97.07735106)(119.02180672,97.07735106)
\lineto(118.3655568,97.08516356)
\lineto(118.29524431,97.14766355)
\lineto(118.29524431,97.48360101)
\lineto(118.3577443,97.546101)
\curveto(119.35253584,97.66589266)(120.19107741,97.86120513)(120.87336899,98.13203843)
\closepath
}
}
{
\newrgbcolor{curcolor}{0 0 0}
\pscustom[linestyle=none,fillstyle=solid,fillcolor=curcolor]
{
}
}
{
\newrgbcolor{curcolor}{0 0 0}
\pscustom[linestyle=none,fillstyle=solid,fillcolor=curcolor]
{
\newpath
\moveto(134.92024226,95.9523512)
\lineto(134.92024226,94.3664139)
\lineto(134.92805476,92.84297658)
\curveto(134.93326309,92.27526832)(134.94888809,91.8976642)(134.97492975,91.71016422)
\curveto(135.00097141,91.52787258)(135.03742974,91.40808093)(135.08430474,91.35078927)
\curveto(135.13638806,91.29870594)(135.23274222,91.25183095)(135.3733672,91.21016429)
\curveto(135.51399218,91.17370596)(135.86034631,91.14506013)(136.41242957,91.1242268)
\lineto(136.47492957,91.0773518)
\lineto(136.47492957,90.68672685)
\lineto(136.41242957,90.62422686)
\curveto(135.08951307,90.66589352)(134.34211733,90.68672685)(134.17024235,90.68672685)
\curveto(134.0035757,90.68672685)(133.25878413,90.66589352)(131.93586763,90.62422686)
\lineto(131.87336763,90.68672685)
\lineto(131.87336763,91.0773518)
\lineto(131.93586763,91.1242268)
\curveto(132.4254509,91.14506013)(132.74836753,91.17110179)(132.90461751,91.20235179)
\curveto(133.06607582,91.23360178)(133.17284664,91.27526844)(133.22492997,91.32735177)
\curveto(133.28222163,91.3794351)(133.32649245,91.48360175)(133.35774245,91.63985173)
\curveto(133.39420078,91.79610171)(133.41503411,92.15026834)(133.42024244,92.7023516)
\lineto(133.42805494,94.3664139)
\lineto(133.42805494,97.98360095)
\lineto(133.41242994,99.50703826)
\curveto(133.40722161,100.10078819)(133.39159661,100.48360064)(133.36555495,100.65547562)
\curveto(133.34472162,100.8273506)(133.30826329,100.94193392)(133.25617996,100.99922558)
\curveto(133.20930497,101.05651724)(133.11555498,101.10339223)(132.97493,101.13985056)
\curveto(132.83430502,101.17630889)(132.48795089,101.20495472)(131.93586763,101.22578805)
\lineto(131.87336763,101.28047554)
\lineto(131.87336763,101.6711005)
\lineto(131.93586763,101.72578799)
\curveto(134.14420069,101.69453799)(135.43326303,101.67891299)(135.80305465,101.67891299)
\curveto(137.04263783,101.67891299)(138.31347101,101.69453799)(139.61555418,101.72578799)
\lineto(139.67805417,101.6711005)
\curveto(139.59472085,101.41068386)(139.54003335,101.1138089)(139.51399169,100.78047561)
\curveto(139.48795003,100.45235065)(139.4593042,99.98620487)(139.4280542,99.38203828)
\lineto(139.36555421,99.31953829)
\lineto(138.99836676,99.31953829)
\lineto(138.94367926,99.38203828)
\curveto(138.9332626,99.41849661)(138.9098251,99.61901742)(138.87336677,99.9836007)
\curveto(138.84732511,100.270059)(138.80565845,100.5330798)(138.74836679,100.77266311)
\curveto(138.20670019,100.95495475)(137.45930444,101.04610057)(136.50617956,101.04610057)
\curveto(136.07909628,101.04610057)(135.57909634,101.01485058)(135.00617975,100.95235058)
\curveto(134.94888809,100.54610063)(134.92024226,99.55651742)(134.92024226,97.98360095)
\lineto(134.92024226,96.63203862)
\curveto(135.58170051,96.62162195)(136.03222129,96.61641362)(136.27180459,96.61641362)
\curveto(136.71451287,96.61641362)(137.09472116,96.63203862)(137.41242945,96.66328861)
\curveto(137.73013774,96.69974694)(137.92024189,96.7336011)(137.98274188,96.7648511)
\curveto(138.04524187,96.7961011)(138.09211687,96.85339276)(138.12336686,96.93672608)
\curveto(138.15982519,97.02526773)(138.19367935,97.21537188)(138.22492935,97.50703851)
\lineto(138.27180434,97.92891346)
\lineto(138.32649184,97.99141345)
\lineto(138.72492929,97.99141345)
\lineto(138.78742928,97.92110096)
\lineto(138.73274179,96.21016367)
\curveto(138.73274179,96.01224703)(138.74055429,95.72578873)(138.75617928,95.35078877)
\curveto(138.77701262,94.92370549)(138.78742928,94.59297637)(138.78742928,94.3586014)
\lineto(138.71711679,94.2961014)
\lineto(138.32649184,94.2961014)
\lineto(138.27180434,94.3586014)
\lineto(138.24055435,95.02266382)
\curveto(138.22492935,95.28308045)(138.19888769,95.47318459)(138.16242936,95.59297624)
\curveto(138.12597103,95.7127679)(138.0660752,95.79089289)(137.98274188,95.82735122)
\curveto(137.89940856,95.86380954)(137.71451275,95.89766371)(137.42805445,95.9289137)
\curveto(137.14159615,95.9601637)(136.70149204,95.9757887)(136.10774211,95.9757887)
\curveto(135.58170051,95.9757887)(135.18586723,95.9679762)(134.92024226,95.9523512)
\closepath
}
}
{
\newrgbcolor{curcolor}{0 0 0}
\pscustom[linestyle=none,fillstyle=solid,fillcolor=curcolor]
{
\newpath
\moveto(142.06867888,95.9836012)
\lineto(141.76399141,96.06172619)
\lineto(141.70149142,96.13985118)
\lineto(141.70149142,97.10860106)
\curveto(142.62857464,97.79610097)(143.5296162,98.13985093)(144.40461609,98.13985093)
\curveto(145.00357435,98.13985093)(145.50097012,98.02787178)(145.8968034,97.80391347)
\curveto(146.29263669,97.57995517)(146.57909499,97.2987052)(146.7561783,96.96016358)
\curveto(146.93326161,96.62683028)(147.02180327,96.23620533)(147.02180327,95.78828872)
\lineto(146.98274077,94.21797641)
\lineto(146.98274077,91.8898517)
\curveto(146.98274077,91.57214341)(147.0035741,91.3794351)(147.04524076,91.31172677)
\curveto(147.09211576,91.24401845)(147.14419908,91.19714345)(147.20149074,91.17110179)
\curveto(147.2587824,91.15026846)(147.36555322,91.13203929)(147.5218032,91.1164143)
\lineto(147.96711565,91.0773518)
\lineto(148.02961564,91.00703931)
\lineto(148.02961564,90.69453935)
\lineto(147.96711565,90.63203936)
\curveto(147.58690736,90.66328935)(147.23274074,90.67891435)(146.90461578,90.67891435)
\curveto(146.59211582,90.67891435)(146.21711586,90.66328935)(145.77961592,90.63203936)
\lineto(145.66242843,90.74141434)
\lineto(145.69367843,91.99922669)
\lineto(143.99055364,90.67110185)
\curveto(143.70409534,90.5513102)(143.39159538,90.49141437)(143.05305376,90.49141437)
\curveto(142.63638714,90.49141437)(142.27701218,90.5669352)(141.97492889,90.71797685)
\curveto(141.67805392,90.86901849)(141.44888729,91.07995597)(141.28742897,91.35078927)
\curveto(141.13117899,91.62162257)(141.053054,91.94974753)(141.053054,92.33516415)
\curveto(141.053054,93.10078905)(141.29263731,93.69714315)(141.77180391,94.12422643)
\curveto(142.25097052,94.55651804)(143.55826203,94.92370549)(145.69367843,95.22578879)
\curveto(145.69367843,95.99662203)(145.52180345,96.53828863)(145.17805349,96.85078859)
\curveto(144.83430354,97.16849688)(144.37336609,97.32735103)(143.79524116,97.32735103)
\curveto(143.49315787,97.32735103)(143.21711623,97.2830802)(142.96711627,97.19453855)
\curveto(142.72232463,97.10599689)(142.58169965,97.03308023)(142.54524132,96.97578857)
\curveto(142.50878299,96.92370525)(142.36815801,96.60599695)(142.12336637,96.02266369)
\closepath
\moveto(145.69367843,94.74922635)
\curveto(144.23534528,94.50443471)(143.32649122,94.24141391)(142.96711627,93.96016395)
\curveto(142.60774131,93.67891398)(142.42805383,93.2440182)(142.42805383,92.65547661)
\curveto(142.42805383,91.84297671)(142.83169962,91.43672676)(143.63899118,91.43672676)
\curveto(144.33169943,91.43672676)(145.01659518,91.83776837)(145.69367843,92.63985161)
\closepath
\moveto(144.36555359,98.62422587)
\closepath
\moveto(144.20149111,90.21797691)
\closepath
}
}
{
\newrgbcolor{curcolor}{0 0 0}
\pscustom[linestyle=none,fillstyle=solid,fillcolor=curcolor]
{
\newpath
\moveto(151.25617774,102.25703792)
\lineto(151.40461523,102.16328793)
\curveto(151.34732357,101.56432968)(151.31867774,100.36380899)(151.31867774,98.56172588)
\lineto(151.31867774,94.89766383)
\curveto(151.49576105,95.00183048)(151.84471934,95.26224712)(152.36555261,95.67891373)
\curveto(152.88638588,96.10078868)(153.43586497,96.57995529)(154.0139899,97.11641356)
\curveto(154.59732316,97.65808016)(154.89680229,97.95235095)(154.91242729,97.99922595)
\curveto(155.35513557,97.98360095)(155.62076054,97.97578845)(155.70930219,97.97578845)
\curveto(155.82909385,97.97578845)(156.07128132,97.98360095)(156.4358646,97.99922595)
\lineto(156.5061771,97.92891346)
\lineto(156.5061771,97.62422599)
\lineto(156.4436771,97.5539135)
\curveto(156.18326047,97.53828851)(155.98273966,97.50443434)(155.84211468,97.45235102)
\curveto(155.70669803,97.40026769)(155.42544806,97.22839271)(154.99836478,96.93672608)
\curveto(154.57648983,96.64505945)(154.0764899,96.262247)(153.49836497,95.78828872)
\lineto(152.74055256,95.1711013)
\curveto(153.01659419,94.84818467)(153.49836497,94.34037223)(154.18586488,93.64766398)
\curveto(154.87857313,92.96016407)(155.43846889,92.42630997)(155.86555218,92.04610168)
\curveto(156.29263546,91.6658934)(156.57128125,91.44714342)(156.70148957,91.38985176)
\curveto(156.83169789,91.33776844)(156.96971871,91.31172677)(157.11555202,91.31172677)
\lineto(157.11555202,90.97578931)
\curveto(156.92284371,90.93933099)(156.42544794,90.78047684)(155.62336471,90.49922687)
\curveto(155.08169811,90.96797682)(154.73794815,91.27787261)(154.59211483,91.42891426)
\lineto(154.0139899,92.02266419)
\lineto(153.32648999,92.7179766)
\curveto(153.11294835,92.93151824)(152.82649005,93.20235154)(152.46711509,93.5304765)
\lineto(151.31867774,94.59297637)
\lineto(151.31867774,93.66328898)
\curveto(151.31867774,93.54870566)(151.3290944,93.17891404)(151.34992773,92.55391412)
\curveto(151.37076106,91.93412253)(151.39159439,91.57214341)(151.41242772,91.46797675)
\curveto(151.43846939,91.3638101)(151.47492772,91.28828928)(151.52180271,91.24141428)
\curveto(151.56867771,91.19453929)(151.62596936,91.16328929)(151.69367769,91.14766429)
\curveto(151.76138601,91.13724763)(152.04263598,91.1164143)(152.53742759,91.0851643)
\lineto(152.60774008,91.02266431)
\lineto(152.60774008,90.69453935)
\lineto(152.54524008,90.62422686)
\curveto(151.8941985,90.66589352)(151.26659441,90.68672685)(150.66242782,90.68672685)
\curveto(150.06346956,90.68672685)(149.43846963,90.66589352)(148.78742805,90.62422686)
\lineto(148.71711556,90.69453935)
\lineto(148.71711556,91.02266431)
\lineto(148.78742805,91.0851643)
\curveto(149.29263632,91.1164143)(149.57649045,91.13985179)(149.63899044,91.15547679)
\curveto(149.70669877,91.17110179)(149.76399043,91.20235179)(149.81086542,91.24922678)
\curveto(149.86294875,91.30131011)(149.89680291,91.37683093)(149.91242791,91.47578925)
\curveto(149.93326124,91.57995591)(149.95409457,91.91589336)(149.9749279,92.48360163)
\curveto(150.00096957,93.05651822)(150.0139904,93.48880984)(150.0139904,93.78047647)
\lineto(150.0139904,97.82735097)
\lineto(149.9827404,99.48360077)
\curveto(149.9671154,100.06172569)(149.94888624,100.46276731)(149.92805291,100.68672562)
\curveto(149.91242791,100.91068392)(149.88899041,101.04349641)(149.85774042,101.08516307)
\curveto(149.82649042,101.12682973)(149.77180293,101.15807973)(149.69367794,101.17891306)
\curveto(149.61555295,101.19974639)(149.30044882,101.21016305)(148.74836555,101.21016305)
\lineto(148.67805306,101.28047554)
\lineto(148.67805306,101.6086005)
\lineto(148.74055305,101.67891299)
\curveto(149.7353446,101.79349631)(150.57388616,101.98620462)(151.25617774,102.25703792)
\closepath
}
}
{
\newrgbcolor{curcolor}{0 0 0}
\pscustom[linestyle=none,fillstyle=solid,fillcolor=curcolor]
{
\newpath
\moveto(157.67023945,96.81953859)
\lineto(157.67023945,97.02266357)
\lineto(157.72492695,97.10078856)
\curveto(158.21451022,97.2830802)(158.6103435,97.45235102)(158.9124268,97.608601)
\curveto(158.9124268,98.96797583)(158.8968018,99.7596424)(158.86555181,99.9836007)
\curveto(159.40201007,100.17110068)(159.84211418,100.37162149)(160.18586414,100.58516313)
\lineto(160.37336412,100.42891315)
\curveto(160.32128079,100.10078819)(160.2587808,99.14766331)(160.18586414,97.5695385)
\curveto(160.44628078,97.56433017)(160.72753074,97.561726)(161.02961404,97.561726)
\curveto(161.6441973,97.561726)(162.08430141,97.577351)(162.34992638,97.608601)
\lineto(162.40461387,97.5539135)
\lineto(162.25617639,96.89766358)
\lineto(162.19367639,96.82735109)
\curveto(161.92805143,96.83255943)(161.63378063,96.83516359)(161.310864,96.83516359)
\curveto(161.01919737,96.83516359)(160.64419742,96.83255943)(160.18586414,96.82735109)
\lineto(160.13898915,93.64766398)
\curveto(160.13898915,92.91328908)(160.15461415,92.4315183)(160.18586414,92.20235166)
\curveto(160.22232247,91.97839336)(160.31607246,91.80131005)(160.46711411,91.67110173)
\curveto(160.62336409,91.54610174)(160.85253073,91.48360175)(161.15461402,91.48360175)
\curveto(161.50357231,91.48360175)(161.82648894,91.57474757)(162.1233639,91.75703922)
\lineto(162.31086388,91.47578925)
\curveto(162.1858639,91.3872476)(161.88638477,91.12683096)(161.41242649,90.69453935)
\curveto(161.14159319,90.56953936)(160.86555156,90.50703937)(160.58430159,90.50703937)
\curveto(159.41242674,90.50703937)(158.82648931,91.07474764)(158.82648931,92.21016416)
\curveto(158.82648931,92.62683078)(158.83690598,92.9809974)(158.85773931,93.27266403)
\curveto(158.86294764,93.36120569)(158.86555181,93.45235151)(158.86555181,93.5461015)
\lineto(158.86555181,96.7882886)
\lineto(158.54523934,96.7882886)
\curveto(158.31086437,96.7882886)(158.04263524,96.77787193)(157.74055194,96.7570386)
\closepath
}
}
{
\newrgbcolor{curcolor}{0 0 0}
\pscustom[linestyle=none,fillstyle=solid,fillcolor=curcolor]
{
\newpath
\moveto(163.06086379,94.26485141)
\curveto(163.06086379,94.884643)(163.18846794,95.48099709)(163.44367624,96.05391369)
\curveto(163.69888454,96.63203862)(164.14159282,97.12943439)(164.77180108,97.546101)
\curveto(165.40721767,97.96797595)(166.15982174,98.17891343)(167.0296133,98.17891343)
\curveto(168.12336316,98.17891343)(169.01398805,97.8325593)(169.70148797,97.13985105)
\curveto(170.38898788,96.45235114)(170.73273784,95.56693458)(170.73273784,94.48360138)
\curveto(170.73273784,93.2908932)(170.33690456,92.30912248)(169.54523799,91.53828924)
\curveto(168.75877975,90.76745601)(167.80825904,90.38203939)(166.69367584,90.38203939)
\curveto(165.96450926,90.38203939)(165.31346768,90.57735186)(164.74055108,90.96797682)
\curveto(164.16763448,91.3638101)(163.74315537,91.84558087)(163.46711374,92.41328914)
\curveto(163.19628044,92.9809974)(163.06086379,93.59818482)(163.06086379,94.26485141)
\closepath
\moveto(164.53742611,94.80391384)
\curveto(164.53742611,94.07474727)(164.63117609,93.42110151)(164.81867607,92.84297658)
\curveto(165.00617605,92.27005999)(165.30305101,91.80912254)(165.70930096,91.46016425)
\curveto(166.11555091,91.11120596)(166.57909252,90.93672682)(167.09992579,90.93672682)
\curveto(167.71450905,90.93672682)(168.22232148,91.17891429)(168.6233631,91.66328923)
\curveto(169.02440472,92.1528725)(169.22492553,92.88724741)(169.22492553,93.86641396)
\curveto(169.22492553,94.97578882)(169.00617555,95.88464288)(168.56867561,96.59297612)
\curveto(168.136384,97.30130937)(167.51138407,97.65547599)(166.69367584,97.65547599)
\curveto(166.01659259,97.65547599)(165.48794682,97.41068435)(165.10773854,96.92110108)
\curveto(164.72753025,96.43151781)(164.53742611,95.72578873)(164.53742611,94.80391384)
\closepath
\moveto(166.90461331,98.62422587)
\closepath
\moveto(166.78742583,90.21797691)
\closepath
}
}
{
\newrgbcolor{curcolor}{0 0 0}
\pscustom[linestyle=none,fillstyle=solid,fillcolor=curcolor]
{
\newpath
\moveto(174.26398741,98.13203843)
\lineto(174.41242489,98.03047594)
\curveto(174.38117489,97.72839265)(174.3577374,97.19193438)(174.3421124,96.42110114)
\lineto(174.92804982,97.16328855)
\curveto(175.12075813,97.40808019)(175.29002895,97.59558016)(175.43586226,97.72578848)
\curveto(175.58690391,97.86120513)(175.76138306,97.96537179)(175.9592997,98.03828844)
\curveto(176.15721634,98.1112051)(176.36034131,98.14766343)(176.56867462,98.14766343)
\curveto(176.79784126,98.14766343)(177.01398707,98.10078844)(177.21711204,98.00703845)
\lineto(177.27179954,97.89766346)
\curveto(177.18846621,97.20495521)(177.14159122,96.59558029)(177.13117455,96.06953869)
\lineto(176.7796121,96.06953869)
\curveto(176.57127879,96.54870529)(176.23534133,96.7882886)(175.77179972,96.7882886)
\curveto(175.44888309,96.7882886)(175.16763313,96.68412194)(174.92804982,96.47578864)
\curveto(174.68846652,96.27266366)(174.52700821,96.01485119)(174.44367488,95.70235123)
\curveto(174.36554989,95.3950596)(174.3264874,95.00443465)(174.3264874,94.53047638)
\lineto(174.3264874,93.66328898)
\curveto(174.3264874,93.507039)(174.33690406,93.12422655)(174.3577374,92.51485162)
\curveto(174.37857073,91.9054767)(174.39940406,91.55391424)(174.42023739,91.46016425)
\curveto(174.44627905,91.36641427)(174.48273738,91.29610177)(174.52961237,91.24922678)
\curveto(174.5816957,91.20756012)(174.64679986,91.17891429)(174.72492485,91.16328929)
\curveto(174.80825817,91.14766429)(175.18065396,91.12162263)(175.84211221,91.0851643)
\lineto(175.9124247,91.02266431)
\lineto(175.9124247,90.69453935)
\lineto(175.84211221,90.62422686)
\curveto(175.14940396,90.66589352)(174.42544572,90.68672685)(173.67023748,90.68672685)
\curveto(173.07127922,90.68672685)(172.4462793,90.66589352)(171.79523771,90.62422686)
\lineto(171.72492522,90.69453935)
\lineto(171.72492522,91.02266431)
\lineto(171.79523771,91.0851643)
\curveto(172.30044598,91.1164143)(172.58430011,91.13985179)(172.64680011,91.15547679)
\curveto(172.71450843,91.17110179)(172.77180009,91.20235179)(172.81867508,91.24922678)
\curveto(172.87075841,91.30131011)(172.90461257,91.37683093)(172.92023757,91.47578925)
\curveto(172.9410709,91.57995591)(172.96190423,91.91589336)(172.98273756,92.48360163)
\curveto(173.00877923,93.05651822)(173.02180006,93.48880984)(173.02180006,93.78047647)
\lineto(173.02180006,95.31953878)
\curveto(173.02180006,95.52787209)(173.01138339,95.81172622)(172.99055006,96.17110117)
\curveto(172.96971673,96.53047613)(172.95148757,96.75183027)(172.93586257,96.83516359)
\curveto(172.92544591,96.91849691)(172.88638341,96.97839274)(172.81867508,97.01485107)
\curveto(172.75096676,97.05651773)(172.61555011,97.07735106)(172.41242513,97.07735106)
\lineto(171.75617522,97.08516356)
\lineto(171.68586272,97.14766355)
\lineto(171.68586272,97.48360101)
\lineto(171.74836272,97.546101)
\curveto(172.74315426,97.66589266)(173.58169582,97.86120513)(174.26398741,98.13203843)
\closepath
}
}
{
\newrgbcolor{curcolor}{0 0 0}
\pscustom[linestyle=none,fillstyle=solid,fillcolor=curcolor]
{
\newpath
\moveto(184.49054865,91.83516421)
\lineto(184.24054868,91.27266428)
\curveto(183.69888208,90.92370599)(183.21450714,90.69714352)(182.78742386,90.59297686)
\curveto(182.36554891,90.48881021)(181.98794479,90.43672688)(181.6546115,90.43672688)
\curveto(181.02961157,90.43672688)(180.43586165,90.5591227)(179.87336171,90.80391434)
\curveto(179.31607012,91.04870597)(178.86034101,91.46276842)(178.50617438,92.04610168)
\curveto(178.15721609,92.62943494)(177.98273695,93.33255986)(177.98273695,94.15547642)
\curveto(177.98273695,94.70235135)(178.05044527,95.19453879)(178.18586192,95.63203874)
\curveto(178.32127857,96.07474702)(178.46190356,96.40287198)(178.60773687,96.61641362)
\curveto(178.75877852,96.82995526)(179.01138265,97.0669344)(179.36554928,97.32735103)
\curveto(179.7197159,97.58776767)(180.09471585,97.79610097)(180.49054914,97.95235095)
\curveto(180.88638242,98.10860093)(181.3134657,98.18672593)(181.77179898,98.18672593)
\curveto(182.3967989,98.18672593)(182.946278,98.03828844)(183.42023628,97.74141348)
\curveto(183.89940289,97.44974685)(184.23534034,97.0747469)(184.42804865,96.61641362)
\curveto(184.62075696,96.15808034)(184.71711112,95.67110124)(184.71711112,95.1554763)
\curveto(184.71711112,94.99401799)(184.70929862,94.837768)(184.69367362,94.68672636)
\lineto(184.60773613,94.60078887)
\curveto(184.25356951,94.52266388)(183.77700707,94.47058055)(183.17804881,94.44453889)
\curveto(182.57909055,94.41849722)(182.18325726,94.40547639)(181.99054895,94.40547639)
\lineto(179.48273676,94.40547639)
\curveto(179.49315343,93.32735152)(179.76398673,92.53308079)(180.29523666,92.02266419)
\curveto(180.8264866,91.51224758)(181.47752818,91.25703928)(182.24836142,91.25703928)
\curveto(182.61294471,91.25703928)(182.961903,91.31953927)(183.29523629,91.44453926)
\curveto(183.63377792,91.56953924)(183.99054871,91.74401839)(184.36554866,91.96797669)
\closepath
\moveto(179.48273676,95.03047631)
\curveto(179.57648675,95.01485132)(179.93586171,94.99662215)(180.56086163,94.97578882)
\curveto(181.19106989,94.95495549)(181.65721566,94.94453882)(181.95929896,94.94453882)
\curveto(182.6832572,94.94453882)(183.12336131,94.95755966)(183.27961129,94.98360132)
\curveto(183.28481963,95.1086013)(183.28742379,95.20495546)(183.28742379,95.27266378)
\curveto(183.28742379,96.07995535)(183.12336131,96.67891361)(182.79523635,97.06953856)
\curveto(182.4671114,97.46537185)(182.01919478,97.66328849)(181.45148652,97.66328849)
\curveto(180.83169493,97.66328849)(180.34731999,97.44193435)(179.9983617,96.99922607)
\curveto(179.65461174,96.55651779)(179.48273676,95.90026787)(179.48273676,95.03047631)
\closepath
\moveto(181.66242399,98.62422587)
\closepath
\moveto(181.56867401,90.21797691)
\closepath
}
}
{
\newrgbcolor{curcolor}{0 0 0}
\pscustom[linestyle=none,fillstyle=solid,fillcolor=curcolor]
{
\newpath
\moveto(187.92023572,98.13203843)
\lineto(188.0686732,98.03047594)
\curveto(188.03742321,97.68151765)(188.01658988,97.22318438)(188.00617321,96.65547611)
\curveto(188.42283983,96.99922607)(188.81867311,97.34818436)(189.19367307,97.70235098)
\curveto(189.30304805,97.80130931)(189.40461054,97.87422596)(189.49836053,97.92110096)
\curveto(189.59731885,97.96797595)(189.75877716,98.01745511)(189.98273547,98.06953844)
\curveto(190.20669377,98.12162177)(190.43586041,98.14766343)(190.67023538,98.14766343)
\curveto(191.06606867,98.14766343)(191.44888112,98.06693427)(191.81867274,97.90547596)
\curveto(192.1936727,97.74401765)(192.47492266,97.55130934)(192.66242264,97.32735103)
\curveto(192.85513095,97.10860106)(192.98794343,96.84818442)(193.06086009,96.54610113)
\curveto(193.13377675,96.24401783)(193.17023508,95.87162204)(193.17023508,95.42891377)
\lineto(193.17023508,94.05391393)
\curveto(193.17023508,93.96537228)(193.18325591,93.31693486)(193.20929757,92.10860167)
\curveto(193.21971424,91.61381007)(193.27440173,91.31953927)(193.37336005,91.22578928)
\curveto(193.47231837,91.13203929)(193.795235,91.0851643)(194.34210993,91.0851643)
\lineto(194.40460992,91.02266431)
\lineto(194.40460992,90.68672685)
\lineto(194.34210993,90.62422686)
\curveto(193.68065168,90.66589352)(193.23273507,90.68672685)(192.9983601,90.68672685)
\curveto(192.86294345,90.68672685)(192.48273516,90.66589352)(191.85773524,90.62422686)
\lineto(191.76398525,90.71016435)
\curveto(191.83169357,91.3872476)(191.86554774,92.27005999)(191.86554774,93.35860152)
\lineto(191.86554774,94.38203889)
\curveto(191.86554774,94.99662215)(191.84992274,95.43933043)(191.81867274,95.71016373)
\curveto(191.79263108,95.98099703)(191.70669359,96.22578867)(191.56086027,96.44453864)
\curveto(191.41502696,96.66849695)(191.21971448,96.84037192)(190.97492285,96.96016358)
\curveto(190.73013121,97.08516356)(190.43846458,97.14766355)(190.09992295,97.14766355)
\curveto(189.82908965,97.14766355)(189.59731885,97.11901772)(189.40461054,97.06172606)
\curveto(189.21190223,97.00964274)(188.99575642,96.89766358)(188.75617312,96.72578861)
\curveto(188.51658982,96.55912196)(188.3395065,96.40026781)(188.22492319,96.24922616)
\curveto(188.11033987,96.10339285)(188.04002737,95.96537203)(188.01398571,95.83516372)
\curveto(187.99315238,95.71016373)(187.98273572,95.43933043)(187.98273572,95.02266382)
\lineto(187.98273572,93.66328898)
\curveto(187.98273572,93.54870566)(187.99315238,93.17891404)(188.01398571,92.55391412)
\curveto(188.03481904,91.93412253)(188.05565237,91.57214341)(188.0764857,91.46797675)
\curveto(188.10252737,91.3638101)(188.1389857,91.28828928)(188.18586069,91.24141428)
\curveto(188.23273568,91.19453929)(188.29002734,91.16328929)(188.35773567,91.14766429)
\curveto(188.42544399,91.13724763)(188.70669396,91.1164143)(189.20148556,91.0851643)
\lineto(189.27179806,91.02266431)
\lineto(189.27179806,90.69453935)
\lineto(189.20929806,90.62422686)
\curveto(188.55825648,90.66589352)(187.93065239,90.68672685)(187.3264858,90.68672685)
\curveto(186.72752754,90.68672685)(186.10252761,90.66589352)(185.45148603,90.62422686)
\lineto(185.38117354,90.69453935)
\lineto(185.38117354,91.02266431)
\lineto(185.45148603,91.0851643)
\curveto(185.9566943,91.1164143)(186.24054843,91.13985179)(186.30304842,91.15547679)
\curveto(186.37075675,91.17110179)(186.42804841,91.20235179)(186.4749234,91.24922678)
\curveto(186.52700673,91.30131011)(186.56086089,91.37683093)(186.57648589,91.47578925)
\curveto(186.59731922,91.57995591)(186.61815255,91.91589336)(186.63898588,92.48360163)
\curveto(186.66502754,93.05651822)(186.67804838,93.48880984)(186.67804838,93.78047647)
\lineto(186.67804838,95.31953878)
\curveto(186.67804838,95.52787209)(186.66763171,95.81172622)(186.64679838,96.17110117)
\curveto(186.62596505,96.53047613)(186.60773588,96.75183027)(186.59211089,96.83516359)
\curveto(186.58169422,96.91849691)(186.54263173,96.97839274)(186.4749234,97.01485107)
\curveto(186.40721508,97.05651773)(186.27179843,97.07735106)(186.06867345,97.07735106)
\lineto(185.41242353,97.08516356)
\lineto(185.34211104,97.14766355)
\lineto(185.34211104,97.48360101)
\lineto(185.40461103,97.546101)
\curveto(186.39940258,97.66589266)(187.23794414,97.86120513)(187.92023572,98.13203843)
\closepath
}
}
{
\newrgbcolor{curcolor}{0 0 0}
\pscustom[linestyle=none,fillstyle=solid,fillcolor=curcolor]
{
\newpath
\moveto(194.45148492,88.43672713)
\curveto(194.77440154,88.97318539)(195.00617235,89.37683118)(195.14679733,89.64766448)
\curveto(195.28742231,89.91849778)(195.40981813,90.20495608)(195.51398479,90.50703937)
\curveto(195.62335977,90.80391434)(195.69367226,91.03308097)(195.72492226,91.19453929)
\curveto(195.76138059,91.3559976)(195.81085975,91.68933089)(195.87335974,92.19453916)
\curveto(195.89940141,92.20495583)(196.04523472,92.24141416)(196.31085969,92.30391415)
\curveto(196.67544298,92.39245581)(197.0608596,92.51224746)(197.46710955,92.66328911)
\lineto(197.66242202,92.44453913)
\curveto(197.00096377,90.90808099)(196.2171097,89.60860198)(195.31085981,88.54610211)
\lineto(194.64679739,88.24141465)
\closepath
}
}
{
\newrgbcolor{curcolor}{0 0 0}
\pscustom[linestyle=none,fillstyle=solid,fillcolor=curcolor]
{
}
}
{
\newrgbcolor{curcolor}{0 0 0}
\pscustom[linestyle=none,fillstyle=solid,fillcolor=curcolor]
{
\newpath
\moveto(203.17804634,93.37422652)
\lineto(203.24054633,93.43672651)
\lineto(203.62335879,93.43672651)
\lineto(203.68585878,93.37422652)
\curveto(203.71190044,92.70755993)(203.7457546,92.29870582)(203.78742127,92.14766417)
\curveto(203.83429626,92.00183085)(203.97492124,91.84558087)(204.20929621,91.67891423)
\curveto(204.44887952,91.51745591)(204.76398365,91.38464343)(205.1546086,91.28047678)
\curveto(205.55044188,91.17631012)(205.94887933,91.1242268)(206.34992095,91.1242268)
\curveto(206.90200422,91.1242268)(207.39679582,91.22578928)(207.83429577,91.42891426)
\curveto(208.27700405,91.63203923)(208.620754,91.9289142)(208.86554564,92.31953915)
\curveto(209.11033728,92.71537243)(209.23273309,93.15808071)(209.23273309,93.64766398)
\curveto(209.23273309,93.99141394)(209.17283727,94.29089307)(209.05304562,94.54610137)
\curveto(208.9384623,94.80130968)(208.77439982,95.00443465)(208.56085818,95.1554763)
\curveto(208.35252487,95.31172628)(208.11294157,95.4263096)(207.84210827,95.49922626)
\curveto(207.57127497,95.57735125)(207.16762918,95.66068457)(206.63117092,95.74922623)
\curveto(206.11033765,95.83255955)(205.70148353,95.90808037)(205.40460857,95.9757887)
\curveto(205.11294194,96.04349702)(204.82127531,96.14766368)(204.52960867,96.28828866)
\curveto(204.23794204,96.42891364)(203.99315041,96.60599695)(203.79523377,96.81953859)
\curveto(203.60252546,97.03828857)(203.44627548,97.30391353)(203.32648382,97.6164135)
\curveto(203.2119005,97.93412179)(203.15460884,98.27266341)(203.15460884,98.63203837)
\curveto(203.15460884,99.60599658)(203.49315047,100.40287149)(204.17023372,101.02266308)
\curveto(204.84731697,101.647663)(205.75617102,101.96016296)(206.89679588,101.96016296)
\curveto(207.34992083,101.96016296)(207.84210827,101.9028713)(208.3733582,101.78828798)
\curveto(208.90981647,101.67891299)(209.39679557,101.51485051)(209.83429552,101.29610054)
\lineto(209.88898301,101.20235055)
\curveto(209.77439969,100.72318395)(209.7040872,100.04089236)(209.67804554,99.15547581)
\lineto(209.60773305,99.09297581)
\lineto(209.2014831,99.09297581)
\lineto(209.13898311,99.14766331)
\curveto(209.12856644,99.77266323)(209.11294144,100.16589235)(209.09210811,100.32735066)
\curveto(209.07127478,100.48880897)(208.83429564,100.68932978)(208.3811707,100.92891309)
\curveto(207.92804576,101.16849639)(207.43585832,101.28828804)(206.90460838,101.28828804)
\curveto(206.4619001,101.28828804)(206.04783765,101.19974639)(205.66242104,101.02266308)
\curveto(205.27700442,100.84557976)(204.98533779,100.56693396)(204.78742114,100.18672568)
\curveto(204.5895045,99.80651739)(204.49054618,99.41849661)(204.49054618,99.02266332)
\curveto(204.49054618,98.71537169)(204.55044201,98.44193423)(204.67023366,98.20235092)
\curveto(204.79002531,97.96797595)(204.94367112,97.78047598)(205.1311711,97.63985099)
\curveto(205.32387941,97.50443434)(205.54523355,97.40287186)(205.79523352,97.33516353)
\curveto(206.05044182,97.26745521)(206.51398343,97.18412188)(207.18585835,97.08516356)
\curveto(208.10773323,96.95495524)(208.77700398,96.7961011)(209.1936706,96.60860112)
\curveto(209.61554555,96.42630948)(209.95148301,96.13464284)(210.20148298,95.73360123)
\curveto(210.45669128,95.33255961)(210.58429543,94.84818467)(210.58429543,94.28047641)
\curveto(210.58429543,93.17110154)(210.13117048,92.24662249)(209.2249206,91.50703925)
\curveto(208.32387904,90.76745601)(207.23012918,90.39766439)(205.943671,90.39766439)
\curveto(204.86554613,90.39766439)(203.92023375,90.5903727)(203.10773385,90.97578931)
\lineto(203.06085886,91.0773518)
\curveto(203.11294218,91.40547676)(203.15200468,92.17110167)(203.17804634,93.37422652)
\closepath
}
}
{
\newrgbcolor{curcolor}{0 0 0}
\pscustom[linestyle=none,fillstyle=solid,fillcolor=curcolor]
{
\newpath
\moveto(211.31867034,96.81953859)
\lineto(211.31867034,97.02266357)
\lineto(211.37335783,97.10078856)
\curveto(211.8629411,97.2830802)(212.25877439,97.45235102)(212.56085768,97.608601)
\curveto(212.56085768,98.96797583)(212.54523269,99.7596424)(212.51398269,99.9836007)
\curveto(213.05044096,100.17110068)(213.49054507,100.37162149)(213.83429503,100.58516313)
\lineto(214.021795,100.42891315)
\curveto(213.96971168,100.10078819)(213.90721169,99.14766331)(213.83429503,97.5695385)
\curveto(214.09471166,97.56433017)(214.37596163,97.561726)(214.67804492,97.561726)
\curveto(215.29262818,97.561726)(215.73273229,97.577351)(215.99835726,97.608601)
\lineto(216.05304475,97.5539135)
\lineto(215.90460727,96.89766358)
\lineto(215.84210728,96.82735109)
\curveto(215.57648231,96.83255943)(215.28221152,96.83516359)(214.95929489,96.83516359)
\curveto(214.66762826,96.83516359)(214.2926283,96.83255943)(213.83429503,96.82735109)
\lineto(213.78742003,93.64766398)
\curveto(213.78742003,92.91328908)(213.80304503,92.4315183)(213.83429503,92.20235166)
\curveto(213.87075336,91.97839336)(213.96450334,91.80131005)(214.11554499,91.67110173)
\curveto(214.27179497,91.54610174)(214.50096161,91.48360175)(214.80304491,91.48360175)
\curveto(215.1520032,91.48360175)(215.47491983,91.57474757)(215.77179479,91.75703922)
\lineto(215.95929477,91.47578925)
\curveto(215.83429478,91.3872476)(215.53481565,91.12683096)(215.06085738,90.69453935)
\curveto(214.79002408,90.56953936)(214.51398244,90.50703937)(214.23273248,90.50703937)
\curveto(213.06085762,90.50703937)(212.4749202,91.07474764)(212.4749202,92.21016416)
\curveto(212.4749202,92.62683078)(212.48533686,92.9809974)(212.50617019,93.27266403)
\curveto(212.51137852,93.36120569)(212.51398269,93.45235151)(212.51398269,93.5461015)
\lineto(212.51398269,96.7882886)
\lineto(212.19367023,96.7882886)
\curveto(211.95929526,96.7882886)(211.69106613,96.77787193)(211.38898283,96.7570386)
\closepath
}
}
{
\newrgbcolor{curcolor}{0 0 0}
\pscustom[linestyle=none,fillstyle=solid,fillcolor=curcolor]
{
\newpath
\moveto(218.88116941,98.13203843)
\lineto(219.02960689,98.03047594)
\curveto(218.97231523,97.34818436)(218.9436694,96.48099697)(218.9436694,95.42891377)
\lineto(218.9436694,93.60860149)
\curveto(218.9436694,92.94193491)(218.99314856,92.4783933)(219.09210688,92.21797666)
\curveto(219.19627353,91.95756003)(219.37335684,91.75703922)(219.62335681,91.61641424)
\curveto(219.87335678,91.48099759)(220.17023175,91.41328926)(220.5139817,91.41328926)
\curveto(220.89418999,91.41328926)(221.24054411,91.48620592)(221.55304408,91.63203923)
\curveto(221.86554404,91.78308088)(222.12596067,91.99401836)(222.33429398,92.26485166)
\curveto(222.54783562,92.53568496)(222.67023144,92.7179766)(222.70148143,92.81172659)
\curveto(222.73273143,92.90547658)(222.75356476,93.16849738)(222.76398143,93.60078899)
\lineto(222.78741892,94.47578888)
\lineto(222.78741892,95.31953878)
\curveto(222.78741892,95.52787209)(222.77700226,95.81172622)(222.75616893,96.17110117)
\curveto(222.7353356,96.53047613)(222.71710643,96.75183027)(222.70148143,96.83516359)
\curveto(222.69106477,96.91849691)(222.65200227,96.97839274)(222.58429395,97.01485107)
\curveto(222.51658562,97.05651773)(222.38116897,97.07735106)(222.178044,97.07735106)
\lineto(221.52179408,97.08516356)
\lineto(221.45148159,97.14766355)
\lineto(221.45148159,97.48360101)
\lineto(221.51398158,97.546101)
\curveto(222.50877312,97.66589266)(223.34731469,97.86120513)(224.02960627,98.13203843)
\lineto(224.17804375,98.03047594)
\curveto(224.12075209,97.34818436)(224.09210626,96.48099697)(224.09210626,95.42891377)
\lineto(224.09210626,94.05391393)
\curveto(224.09210626,93.97578894)(224.10512709,93.33776819)(224.13116876,92.13985167)
\curveto(224.14158542,91.69714339)(224.16762709,91.42631009)(224.20929375,91.32735177)
\curveto(224.25616874,91.23360178)(224.31866873,91.16589346)(224.39679373,91.1242268)
\curveto(224.47491872,91.08776847)(224.70668952,91.0695393)(225.09210614,91.0695393)
\lineto(225.31085611,91.0695393)
\lineto(225.3811686,91.00703931)
\lineto(225.3811686,90.69453935)
\lineto(225.31866861,90.62422686)
\curveto(224.5842937,90.66589352)(224.09210626,90.68672685)(223.84210629,90.68672685)
\curveto(223.524398,90.68672685)(223.15721054,90.66849769)(222.74054393,90.63203936)
\lineto(222.67023144,90.69453935)
\curveto(222.70148143,91.22578928)(222.72491893,91.67631006)(222.74054393,92.04610168)
\curveto(222.41241897,91.79089338)(222.03741902,91.44974759)(221.61554407,91.02266431)
\curveto(221.45408575,90.86120599)(221.22231495,90.72578935)(220.92023165,90.61641436)
\curveto(220.61814836,90.50703937)(220.2743984,90.45235188)(219.88898178,90.45235188)
\curveto(219.30564852,90.45235188)(218.84991941,90.5434977)(218.52179445,90.72578935)
\curveto(218.19887782,90.91328932)(217.96971118,91.16849762)(217.83429453,91.49141425)
\curveto(217.69887788,91.81953921)(217.63116956,92.38203914)(217.63116956,93.17891404)
\lineto(217.63898206,93.79610147)
\lineto(217.63898206,95.31953878)
\curveto(217.63898206,95.52787209)(217.62856539,95.81172622)(217.60773206,96.17110117)
\curveto(217.58689873,96.53047613)(217.56866957,96.75183027)(217.55304457,96.83516359)
\curveto(217.5426279,96.91849691)(217.50356541,96.97839274)(217.43585708,97.01485107)
\curveto(217.36814876,97.05651773)(217.23273211,97.07735106)(217.02960713,97.07735106)
\lineto(216.37335721,97.08516356)
\lineto(216.30304472,97.14766355)
\lineto(216.30304472,97.48360101)
\lineto(216.36554472,97.546101)
\curveto(217.36033626,97.66589266)(218.19887782,97.86120513)(218.88116941,98.13203843)
\closepath
\moveto(220.74835667,98.62422587)
\closepath
\moveto(220.82648167,90.21797691)
\closepath
}
}
{
\newrgbcolor{curcolor}{0 0 0}
\pscustom[linestyle=none,fillstyle=solid,fillcolor=curcolor]
{
\newpath
\moveto(230.32648049,91.0851643)
\lineto(230.38898049,91.02266431)
\lineto(230.38898049,90.69453935)
\lineto(230.32648049,90.62422686)
\curveto(230.16502218,90.62422686)(229.86033472,90.63464352)(229.41241811,90.65547685)
\curveto(229.00616816,90.67631018)(228.58689738,90.68672685)(228.15460576,90.68672685)
\curveto(227.59210583,90.68672685)(226.96450174,90.66589352)(226.27179349,90.62422686)
\lineto(226.2092935,90.69453935)
\lineto(226.2092935,91.02266431)
\lineto(226.27179349,91.0851643)
\curveto(226.77700176,91.1164143)(227.06346006,91.13985179)(227.13116839,91.15547679)
\curveto(227.19887671,91.17110179)(227.25616837,91.20235179)(227.30304337,91.24922678)
\curveto(227.34991836,91.30131011)(227.38116836,91.37683093)(227.39679336,91.47578925)
\curveto(227.41762669,91.57995591)(227.43846002,91.91589336)(227.45929335,92.48360163)
\curveto(227.48533501,93.05651822)(227.49835584,93.48880984)(227.49835584,93.78047647)
\lineto(227.49835584,96.7961011)
\lineto(226.49054347,96.7414136)
\lineto(226.42023098,96.8039136)
\lineto(226.42023098,96.99922607)
\lineto(226.47491847,97.07735106)
\lineto(227.49835584,97.6007885)
\lineto(227.49835584,98.06953844)
\curveto(227.49835584,98.59037171)(227.52179334,98.98360083)(227.56866833,99.24922579)
\curveto(227.61554333,99.52005909)(227.70148082,99.7674549)(227.8264808,99.9914132)
\curveto(227.95668912,100.22057984)(228.18064743,100.50703814)(228.49835572,100.8507881)
\curveto(228.81606401,101.19974639)(229.09470981,101.48360052)(229.33429312,101.70235049)
\curveto(229.57387642,101.9263088)(229.78220973,102.07474628)(229.95929304,102.14766294)
\curveto(230.13637635,102.22578793)(230.34470966,102.26485042)(230.58429296,102.26485042)
\curveto(230.7822096,102.26485042)(230.99575124,102.22578793)(231.22491788,102.14766294)
\lineto(231.21710538,100.86641309)
\lineto(231.03741791,100.7961006)
\curveto(230.74575128,101.05651724)(230.40720965,101.18672556)(230.02179303,101.18672556)
\curveto(229.75095973,101.18672556)(229.51658476,101.1294339)(229.31866812,101.01485058)
\curveto(229.12075148,100.90547559)(228.98533483,100.72318395)(228.91241817,100.46797564)
\curveto(228.83950151,100.21797567)(228.80304318,99.80130906)(228.80304318,99.2179758)
\lineto(228.80304318,97.6007885)
\lineto(229.63898058,97.6007885)
\curveto(230.07127219,97.6007885)(230.48012631,97.6164135)(230.86554293,97.64766349)
\lineto(230.93585542,97.561726)
\lineto(230.79523044,96.87422609)
\lineto(230.72491794,96.7961011)
\curveto(230.56866796,96.80130943)(230.46710548,96.8039136)(230.42023048,96.8039136)
\lineto(229.40460561,96.81172609)
\lineto(228.80304318,96.81172609)
\lineto(228.80304318,93.66328898)
\curveto(228.80304318,93.49662234)(228.81345985,93.11380988)(228.83429318,92.51485162)
\curveto(228.86033484,91.9211017)(228.88377234,91.57214341)(228.90460567,91.46797675)
\curveto(228.925439,91.3638101)(228.95929316,91.28828928)(229.00616816,91.24141428)
\curveto(229.05825148,91.19453929)(229.13898064,91.16068512)(229.24835563,91.13985179)
\curveto(229.36293895,91.11901846)(229.7223139,91.1007893)(230.32648049,91.0851643)
\closepath
}
}
{
\newrgbcolor{curcolor}{0 0 0}
\pscustom[linestyle=none,fillstyle=solid,fillcolor=curcolor]
{
\newpath
\moveto(238.06085454,91.83516421)
\lineto(237.81085457,91.27266428)
\curveto(237.26918797,90.92370599)(236.78481303,90.69714352)(236.35772975,90.59297686)
\curveto(235.9358548,90.48881021)(235.55825068,90.43672688)(235.22491739,90.43672688)
\curveto(234.59991747,90.43672688)(234.00616754,90.5591227)(233.44366761,90.80391434)
\curveto(232.88637601,91.04870597)(232.4306469,91.46276842)(232.07648028,92.04610168)
\curveto(231.72752199,92.62943494)(231.55304284,93.33255986)(231.55304284,94.15547642)
\curveto(231.55304284,94.70235135)(231.62075117,95.19453879)(231.75616782,95.63203874)
\curveto(231.89158447,96.07474702)(232.03220945,96.40287198)(232.17804277,96.61641362)
\curveto(232.32908441,96.82995526)(232.58168855,97.0669344)(232.93585517,97.32735103)
\curveto(233.2900218,97.58776767)(233.66502175,97.79610097)(234.06085503,97.95235095)
\curveto(234.45668832,98.10860093)(234.8837716,98.18672593)(235.34210488,98.18672593)
\curveto(235.9671048,98.18672593)(236.5165839,98.03828844)(236.99054217,97.74141348)
\curveto(237.46970878,97.44974685)(237.80564624,97.0747469)(237.99835455,96.61641362)
\curveto(238.19106286,96.15808034)(238.28741701,95.67110124)(238.28741701,95.1554763)
\curveto(238.28741701,94.99401799)(238.27960451,94.837768)(238.26397952,94.68672636)
\lineto(238.17804203,94.60078887)
\curveto(237.8238754,94.52266388)(237.34731296,94.47058055)(236.7483547,94.44453889)
\curveto(236.14939644,94.41849722)(235.75356316,94.40547639)(235.56085485,94.40547639)
\lineto(233.05304266,94.40547639)
\curveto(233.06345932,93.32735152)(233.33429262,92.53308079)(233.86554256,92.02266419)
\curveto(234.39679249,91.51224758)(235.04783408,91.25703928)(235.81866732,91.25703928)
\curveto(236.18325061,91.25703928)(236.5322089,91.31953927)(236.86554219,91.44453926)
\curveto(237.20408381,91.56953924)(237.5608546,91.74401839)(237.93585456,91.96797669)
\closepath
\moveto(233.05304266,95.03047631)
\curveto(233.14679265,95.01485132)(233.5061676,94.99662215)(234.13116752,94.97578882)
\curveto(234.76137578,94.95495549)(235.22752156,94.94453882)(235.52960485,94.94453882)
\curveto(236.2535631,94.94453882)(236.69366721,94.95755966)(236.84991719,94.98360132)
\curveto(236.85512552,95.1086013)(236.85772969,95.20495546)(236.85772969,95.27266378)
\curveto(236.85772969,96.07995535)(236.69366721,96.67891361)(236.36554225,97.06953856)
\curveto(236.03741729,97.46537185)(235.58950068,97.66328849)(235.02179241,97.66328849)
\curveto(234.40200082,97.66328849)(233.91762588,97.44193435)(233.56866759,96.99922607)
\curveto(233.22491764,96.55651779)(233.05304266,95.90026787)(233.05304266,95.03047631)
\closepath
\moveto(235.23272989,98.62422587)
\closepath
\moveto(235.1389799,90.21797691)
\closepath
}
}
{
\newrgbcolor{curcolor}{0 0 0}
\pscustom[linestyle=none,fillstyle=solid,fillcolor=curcolor]
{
\newpath
\moveto(241.49054162,98.13203843)
\lineto(241.6389791,98.03047594)
\curveto(241.6077291,97.68151765)(241.58689577,97.22318438)(241.57647911,96.65547611)
\curveto(241.99314572,96.99922607)(242.38897901,97.34818436)(242.76397896,97.70235098)
\curveto(242.87335395,97.80130931)(242.97491643,97.87422596)(243.06866642,97.92110096)
\curveto(243.16762474,97.96797595)(243.32908306,98.01745511)(243.55304136,98.06953844)
\curveto(243.77699967,98.12162177)(244.00616631,98.14766343)(244.24054128,98.14766343)
\curveto(244.63637456,98.14766343)(245.01918702,98.06693427)(245.38897864,97.90547596)
\curveto(245.76397859,97.74401765)(246.04522856,97.55130934)(246.23272853,97.32735103)
\curveto(246.42543684,97.10860106)(246.55824933,96.84818442)(246.63116598,96.54610113)
\curveto(246.70408264,96.24401783)(246.74054097,95.87162204)(246.74054097,95.42891377)
\lineto(246.74054097,94.05391393)
\curveto(246.74054097,93.96537228)(246.7535618,93.31693486)(246.77960347,92.10860167)
\curveto(246.79002013,91.61381007)(246.84470762,91.31953927)(246.94366594,91.22578928)
\curveto(247.04262427,91.13203929)(247.36554089,91.0851643)(247.91241583,91.0851643)
\lineto(247.97491582,91.02266431)
\lineto(247.97491582,90.68672685)
\lineto(247.91241583,90.62422686)
\curveto(247.25095757,90.66589352)(246.80304096,90.68672685)(246.56866599,90.68672685)
\curveto(246.43324934,90.68672685)(246.05304105,90.66589352)(245.42804113,90.62422686)
\lineto(245.33429114,90.71016435)
\curveto(245.40199947,91.3872476)(245.43585363,92.27005999)(245.43585363,93.35860152)
\lineto(245.43585363,94.38203889)
\curveto(245.43585363,94.99662215)(245.42022863,95.43933043)(245.38897864,95.71016373)
\curveto(245.36293697,95.98099703)(245.27699948,96.22578867)(245.13116617,96.44453864)
\curveto(244.98533285,96.66849695)(244.79002038,96.84037192)(244.54522874,96.96016358)
\curveto(244.3004371,97.08516356)(244.00877047,97.14766355)(243.67022885,97.14766355)
\curveto(243.39939555,97.14766355)(243.16762474,97.11901772)(242.97491643,97.06172606)
\curveto(242.78220812,97.00964274)(242.56606232,96.89766358)(242.32647901,96.72578861)
\curveto(242.08689571,96.55912196)(241.9098124,96.40026781)(241.79522908,96.24922616)
\curveto(241.68064576,96.10339285)(241.61033327,95.96537203)(241.58429161,95.83516372)
\curveto(241.56345828,95.71016373)(241.55304161,95.43933043)(241.55304161,95.02266382)
\lineto(241.55304161,93.66328898)
\curveto(241.55304161,93.54870566)(241.56345828,93.17891404)(241.58429161,92.55391412)
\curveto(241.60512494,91.93412253)(241.62595827,91.57214341)(241.6467916,91.46797675)
\curveto(241.67283326,91.3638101)(241.70929159,91.28828928)(241.75616658,91.24141428)
\curveto(241.80304158,91.19453929)(241.86033324,91.16328929)(241.92804156,91.14766429)
\curveto(241.99574989,91.13724763)(242.27699985,91.1164143)(242.77179146,91.0851643)
\lineto(242.84210395,91.02266431)
\lineto(242.84210395,90.69453935)
\lineto(242.77960396,90.62422686)
\curveto(242.12856237,90.66589352)(241.50095828,90.68672685)(240.89679169,90.68672685)
\curveto(240.29783343,90.68672685)(239.67283351,90.66589352)(239.02179192,90.62422686)
\lineto(238.95147943,90.69453935)
\lineto(238.95147943,91.02266431)
\lineto(239.02179192,91.0851643)
\curveto(239.52700019,91.1164143)(239.81085432,91.13985179)(239.87335432,91.15547679)
\curveto(239.94106264,91.17110179)(239.9983543,91.20235179)(240.0452293,91.24922678)
\curveto(240.09731262,91.30131011)(240.13116678,91.37683093)(240.14679178,91.47578925)
\curveto(240.16762511,91.57995591)(240.18845844,91.91589336)(240.20929178,92.48360163)
\curveto(240.23533344,93.05651822)(240.24835427,93.48880984)(240.24835427,93.78047647)
\lineto(240.24835427,95.31953878)
\curveto(240.24835427,95.52787209)(240.23793761,95.81172622)(240.21710427,96.17110117)
\curveto(240.19627094,96.53047613)(240.17804178,96.75183027)(240.16241678,96.83516359)
\curveto(240.15200012,96.91849691)(240.11293762,96.97839274)(240.0452293,97.01485107)
\curveto(239.97752097,97.05651773)(239.84210432,97.07735106)(239.63897935,97.07735106)
\lineto(238.98272943,97.08516356)
\lineto(238.91241694,97.14766355)
\lineto(238.91241694,97.48360101)
\lineto(238.97491693,97.546101)
\curveto(239.96970847,97.66589266)(240.80825003,97.86120513)(241.49054162,98.13203843)
\closepath
}
}
{
\newrgbcolor{curcolor}{0 0 0}
\pscustom[linestyle=none,fillstyle=solid,fillcolor=curcolor]
{
}
}
{
\newrgbcolor{curcolor}{0 0 0}
\pscustom[linestyle=none,fillstyle=solid,fillcolor=curcolor]
{
\newpath
\moveto(254.83428997,98.13203843)
\lineto(254.98272745,98.03047594)
\curveto(254.92543579,97.34818436)(254.89678996,96.48099697)(254.89678996,95.42891377)
\lineto(254.89678996,93.60860149)
\curveto(254.89678996,92.94193491)(254.94626912,92.4783933)(255.04522745,92.21797666)
\curveto(255.1493941,91.95756003)(255.32647741,91.75703922)(255.57647738,91.61641424)
\curveto(255.82647735,91.48099759)(256.12335231,91.41328926)(256.46710227,91.41328926)
\curveto(256.84731056,91.41328926)(257.19366468,91.48620592)(257.50616464,91.63203923)
\curveto(257.8186646,91.78308088)(258.07908124,91.99401836)(258.28741455,92.26485166)
\curveto(258.50095619,92.53568496)(258.623352,92.7179766)(258.654602,92.81172659)
\curveto(258.685852,92.90547658)(258.70668533,93.16849738)(258.71710199,93.60078899)
\lineto(258.74053949,94.47578888)
\lineto(258.74053949,95.31953878)
\curveto(258.74053949,95.52787209)(258.73012282,95.81172622)(258.70928949,96.17110117)
\curveto(258.68845616,96.53047613)(258.670227,96.75183027)(258.654602,96.83516359)
\curveto(258.64418534,96.91849691)(258.60512284,96.97839274)(258.53741452,97.01485107)
\curveto(258.46970619,97.05651773)(258.33428954,97.07735106)(258.13116457,97.07735106)
\lineto(257.47491465,97.08516356)
\lineto(257.40460216,97.14766355)
\lineto(257.40460216,97.48360101)
\lineto(257.46710215,97.546101)
\curveto(258.46189369,97.66589266)(259.30043525,97.86120513)(259.98272684,98.13203843)
\lineto(260.13116432,98.03047594)
\curveto(260.07387266,97.34818436)(260.04522683,96.48099697)(260.04522683,95.42891377)
\lineto(260.04522683,94.05391393)
\curveto(260.04522683,93.97578894)(260.05824766,93.33776819)(260.08428932,92.13985167)
\curveto(260.09470599,91.69714339)(260.12074765,91.42631009)(260.16241432,91.32735177)
\curveto(260.20928931,91.23360178)(260.2717893,91.16589346)(260.34991429,91.1242268)
\curveto(260.42803928,91.08776847)(260.65981009,91.0695393)(261.04522671,91.0695393)
\lineto(261.26397668,91.0695393)
\lineto(261.33428917,91.00703931)
\lineto(261.33428917,90.69453935)
\lineto(261.27178918,90.62422686)
\curveto(260.53741427,90.66589352)(260.04522683,90.68672685)(259.79522686,90.68672685)
\curveto(259.47751857,90.68672685)(259.11033111,90.66849769)(258.6936645,90.63203936)
\lineto(258.623352,90.69453935)
\curveto(258.654602,91.22578928)(258.6780395,91.67631006)(258.6936645,92.04610168)
\curveto(258.36553954,91.79089338)(257.99053958,91.44974759)(257.56866463,91.02266431)
\curveto(257.40720632,90.86120599)(257.17543552,90.72578935)(256.87335222,90.61641436)
\curveto(256.57126892,90.50703937)(256.22751897,90.45235188)(255.84210235,90.45235188)
\curveto(255.25876909,90.45235188)(254.80303998,90.5434977)(254.47491502,90.72578935)
\curveto(254.15199839,90.91328932)(253.92283175,91.16849762)(253.7874151,91.49141425)
\curveto(253.65199845,91.81953921)(253.58429013,92.38203914)(253.58429013,93.17891404)
\lineto(253.59210263,93.79610147)
\lineto(253.59210263,95.31953878)
\curveto(253.59210263,95.52787209)(253.58168596,95.81172622)(253.56085263,96.17110117)
\curveto(253.5400193,96.53047613)(253.52179013,96.75183027)(253.50616514,96.83516359)
\curveto(253.49574847,96.91849691)(253.45668598,96.97839274)(253.38897765,97.01485107)
\curveto(253.32126933,97.05651773)(253.18585268,97.07735106)(252.9827277,97.07735106)
\lineto(252.32647778,97.08516356)
\lineto(252.25616529,97.14766355)
\lineto(252.25616529,97.48360101)
\lineto(252.31866528,97.546101)
\curveto(253.31345683,97.66589266)(254.15199839,97.86120513)(254.83428997,98.13203843)
\closepath
\moveto(256.70147724,98.62422587)
\closepath
\moveto(256.77960223,90.21797691)
\closepath
}
}
{
\newrgbcolor{curcolor}{0 0 0}
\pscustom[linestyle=none,fillstyle=solid,fillcolor=curcolor]
{
\newpath
\moveto(264.45147629,98.13203843)
\lineto(264.59991377,98.03047594)
\curveto(264.56866377,97.68151765)(264.54783044,97.22318438)(264.53741378,96.65547611)
\curveto(264.95408039,96.99922607)(265.34991368,97.34818436)(265.72491363,97.70235098)
\curveto(265.83428862,97.80130931)(265.9358511,97.87422596)(266.02960109,97.92110096)
\curveto(266.12855941,97.96797595)(266.29001773,98.01745511)(266.51397603,98.06953844)
\curveto(266.73793434,98.12162177)(266.96710098,98.14766343)(267.20147595,98.14766343)
\curveto(267.59730923,98.14766343)(267.98012168,98.06693427)(268.34991331,97.90547596)
\curveto(268.72491326,97.74401765)(269.00616322,97.55130934)(269.1936632,97.32735103)
\curveto(269.38637151,97.10860106)(269.51918399,96.84818442)(269.59210065,96.54610113)
\curveto(269.66501731,96.24401783)(269.70147564,95.87162204)(269.70147564,95.42891377)
\lineto(269.70147564,94.05391393)
\curveto(269.70147564,93.96537228)(269.71449647,93.31693486)(269.74053813,92.10860167)
\curveto(269.7509548,91.61381007)(269.80564229,91.31953927)(269.90460061,91.22578928)
\curveto(270.00355893,91.13203929)(270.32647556,91.0851643)(270.87335049,91.0851643)
\lineto(270.93585049,91.02266431)
\lineto(270.93585049,90.68672685)
\lineto(270.87335049,90.62422686)
\curveto(270.21189224,90.66589352)(269.76397563,90.68672685)(269.52960066,90.68672685)
\curveto(269.39418401,90.68672685)(269.01397572,90.66589352)(268.3889758,90.62422686)
\lineto(268.29522581,90.71016435)
\curveto(268.36293414,91.3872476)(268.3967883,92.27005999)(268.3967883,93.35860152)
\lineto(268.3967883,94.38203889)
\curveto(268.3967883,94.99662215)(268.3811633,95.43933043)(268.34991331,95.71016373)
\curveto(268.32387164,95.98099703)(268.23793415,96.22578867)(268.09210084,96.44453864)
\curveto(267.94626752,96.66849695)(267.75095505,96.84037192)(267.50616341,96.96016358)
\curveto(267.26137177,97.08516356)(266.96970514,97.14766355)(266.63116352,97.14766355)
\curveto(266.36033022,97.14766355)(266.12855941,97.11901772)(265.9358511,97.06172606)
\curveto(265.74314279,97.00964274)(265.52699699,96.89766358)(265.28741368,96.72578861)
\curveto(265.04783038,96.55912196)(264.87074707,96.40026781)(264.75616375,96.24922616)
\curveto(264.64158043,96.10339285)(264.57126794,95.96537203)(264.54522627,95.83516372)
\curveto(264.52439294,95.71016373)(264.51397628,95.43933043)(264.51397628,95.02266382)
\lineto(264.51397628,93.66328898)
\curveto(264.51397628,93.54870566)(264.52439294,93.17891404)(264.54522627,92.55391412)
\curveto(264.56605961,91.93412253)(264.58689294,91.57214341)(264.60772627,91.46797675)
\curveto(264.63376793,91.3638101)(264.67022626,91.28828928)(264.71710125,91.24141428)
\curveto(264.76397625,91.19453929)(264.82126791,91.16328929)(264.88897623,91.14766429)
\curveto(264.95668456,91.13724763)(265.23793452,91.1164143)(265.73272613,91.0851643)
\lineto(265.80303862,91.02266431)
\lineto(265.80303862,90.69453935)
\lineto(265.74053863,90.62422686)
\curveto(265.08949704,90.66589352)(264.46189295,90.68672685)(263.85772636,90.68672685)
\curveto(263.2587681,90.68672685)(262.63376818,90.66589352)(261.98272659,90.62422686)
\lineto(261.9124141,90.69453935)
\lineto(261.9124141,91.02266431)
\lineto(261.98272659,91.0851643)
\curveto(262.48793486,91.1164143)(262.77178899,91.13985179)(262.83428899,91.15547679)
\curveto(262.90199731,91.17110179)(262.95928897,91.20235179)(263.00616396,91.24922678)
\curveto(263.05824729,91.30131011)(263.09210145,91.37683093)(263.10772645,91.47578925)
\curveto(263.12855978,91.57995591)(263.14939311,91.91589336)(263.17022644,92.48360163)
\curveto(263.19626811,93.05651822)(263.20928894,93.48880984)(263.20928894,93.78047647)
\lineto(263.20928894,95.31953878)
\curveto(263.20928894,95.52787209)(263.19887227,95.81172622)(263.17803894,96.17110117)
\curveto(263.15720561,96.53047613)(263.13897645,96.75183027)(263.12335145,96.83516359)
\curveto(263.11293478,96.91849691)(263.07387229,96.97839274)(263.00616396,97.01485107)
\curveto(262.93845564,97.05651773)(262.80303899,97.07735106)(262.59991401,97.07735106)
\lineto(261.9436641,97.08516356)
\lineto(261.8733516,97.14766355)
\lineto(261.8733516,97.48360101)
\lineto(261.9358516,97.546101)
\curveto(262.93064314,97.66589266)(263.7691847,97.86120513)(264.45147629,98.13203843)
\closepath
}
}
{
\newrgbcolor{curcolor}{0 0 0}
\pscustom[linestyle=none,fillstyle=solid,fillcolor=curcolor]
{
\newpath
\moveto(276.34991232,101.28047554)
\lineto(276.34991232,101.6086005)
\lineto(276.41241231,101.67891299)
\curveto(277.40720386,101.79349631)(278.24574542,101.98620462)(278.928037,102.25703792)
\lineto(279.07647448,102.16328793)
\curveto(279.01918282,101.56432968)(278.99053699,100.36380899)(278.99053699,98.56172588)
\lineto(278.99053699,93.44453901)
\curveto(278.99053699,92.91849741)(278.99834949,92.45755996)(279.01397449,92.06172668)
\curveto(279.03480782,91.67110173)(279.06866198,91.43933092)(279.11553698,91.36641427)
\curveto(279.16241197,91.29349761)(279.2379328,91.23360178)(279.34209945,91.18672679)
\curveto(279.45147444,91.13985179)(279.7561619,91.09558097)(280.25616184,91.0539143)
\lineto(280.32647433,90.99141431)
\lineto(280.32647433,90.69453935)
\lineto(280.25616184,90.62422686)
\curveto(279.69366191,90.66068519)(279.25616196,90.67891435)(278.943662,90.67891435)
\curveto(278.66241203,90.67891435)(278.26137042,90.66068519)(277.74053715,90.62422686)
\lineto(277.65459966,90.71016435)
\curveto(277.67022466,91.22058095)(277.67803716,91.57735174)(277.67803716,91.78047671)
\curveto(277.67803716,91.80651838)(277.68064132,91.9054767)(277.68584965,92.07735168)
\curveto(277.39939136,91.86901837)(277.11293306,91.63724757)(276.82647476,91.38203926)
\curveto(276.40459981,91.00703931)(276.12855818,90.77266434)(275.99834986,90.67891435)
\curveto(275.75876656,90.5591227)(275.41762077,90.49922687)(274.97491249,90.49922687)
\curveto(274.25616258,90.49922687)(273.64418349,90.67631018)(273.13897521,91.03047681)
\curveto(272.63897528,91.38464343)(272.28220449,91.83516421)(272.06866285,92.38203914)
\curveto(271.85512121,92.93412241)(271.74835039,93.4992265)(271.74835039,94.07735143)
\curveto(271.74835039,94.66589303)(271.85772537,95.23099712)(272.07647535,95.77266372)
\curveto(272.30043365,96.31953866)(272.61553778,96.71797611)(273.02178773,96.96797608)
\curveto(273.43324601,97.21797604)(273.87855846,97.47578851)(274.35772506,97.74141348)
\curveto(274.8421,98.01224678)(275.32126661,98.14766343)(275.79522489,98.14766343)
\curveto(276.45668314,98.14766343)(277.08689139,97.98360095)(277.68584965,97.65547599)
\lineto(277.65459966,99.39766328)
\curveto(277.64418299,100.00182987)(277.62855799,100.42110065)(277.60772466,100.65547562)
\curveto(277.58689133,100.89505892)(277.56084967,101.03568391)(277.52959967,101.07735057)
\curveto(277.49834968,101.12422556)(277.44366218,101.15807973)(277.36553719,101.17891306)
\curveto(277.29262054,101.19974639)(276.97751641,101.21016305)(276.42022481,101.21016305)
\closepath
\moveto(277.68584965,96.31953866)
\curveto(277.36293303,96.68412194)(277.00355807,96.96276774)(276.60772479,97.15547605)
\curveto(276.21709984,97.34818436)(275.82387072,97.44453852)(275.42803743,97.44453852)
\curveto(274.99053749,97.44453852)(274.58168337,97.32474686)(274.20147508,97.08516356)
\curveto(273.82647513,96.85078859)(273.55564183,96.49922613)(273.38897518,96.03047619)
\curveto(273.22230854,95.56693458)(273.13897521,95.06172631)(273.13897521,94.51485138)
\curveto(273.13897521,93.60339316)(273.36814185,92.87162241)(273.82647513,92.31953915)
\curveto(274.29001674,91.76745588)(274.87074583,91.49141425)(275.56866242,91.49141425)
\curveto(275.9488707,91.49141425)(276.28480816,91.57735174)(276.57647479,91.74922672)
\curveto(276.87334975,91.92631003)(277.11553722,92.16328917)(277.3030372,92.46016413)
\curveto(277.49574551,92.75703909)(277.60772466,93.05391406)(277.63897466,93.35078902)
\curveto(277.67022466,93.64766398)(277.68584965,94.19193475)(277.68584965,94.98360132)
\closepath
}
}
{
\newrgbcolor{curcolor}{0 0 0}
\pscustom[linestyle=none,fillstyle=solid,fillcolor=curcolor]
{
}
}
{
\newrgbcolor{curcolor}{0 0 0}
\pscustom[linestyle=none,fillstyle=solid,fillcolor=curcolor]
{
\newpath
\moveto(285.32647371,101.72578799)
\curveto(286.63897355,101.68412133)(287.47491095,101.663288)(287.8342859,101.663288)
\curveto(288.32386918,101.663288)(288.87334827,101.67630883)(289.4827232,101.70235049)
\curveto(290.12855645,101.72318382)(290.5269939,101.73360049)(290.67803555,101.73360049)
\curveto(291.40720213,101.73360049)(292.00876455,101.65807966)(292.48272283,101.50703802)
\curveto(292.96188944,101.35599637)(293.35772272,101.07995473)(293.67022268,100.67891312)
\curveto(293.98272264,100.2778715)(294.13897263,99.79349656)(294.13897263,99.2257883)
\curveto(294.13897263,98.75703835)(294.02959764,98.30391341)(293.81084767,97.86641346)
\curveto(293.59209769,97.42891352)(293.28741023,97.05651773)(292.89678528,96.7492261)
\curveto(292.50616033,96.44193447)(292.11032704,96.22058033)(291.70928543,96.08516368)
\curveto(291.30824381,95.95495537)(290.89157719,95.88985121)(290.45928558,95.88985121)
\curveto(290.13636895,95.88985121)(289.776994,95.92630954)(289.38116071,95.99922619)
\lineto(289.24834823,96.47578864)
\lineto(289.31866072,96.55391363)
\curveto(289.70928567,96.46537197)(290.02699397,96.42110114)(290.2717856,96.42110114)
\curveto(290.95407718,96.42110114)(291.50095212,96.64505945)(291.9124104,97.09297606)
\curveto(292.32907702,97.546101)(292.53741032,98.12422593)(292.53741032,98.82735085)
\curveto(292.53741032,99.53568409)(292.32126452,100.10078819)(291.8889729,100.52266314)
\curveto(291.46188962,100.94453808)(290.8134522,101.15547556)(289.94366064,101.15547556)
\curveto(289.46970237,101.15547556)(288.9488691,101.0669339)(288.38116084,100.88985059)
\curveto(288.33428584,100.48880897)(288.31084834,99.52005909)(288.31084834,97.98360095)
\lineto(288.31084834,94.3664139)
\lineto(288.31866084,92.84297658)
\curveto(288.32386918,92.25443499)(288.33949417,91.8742267)(288.36553584,91.70235172)
\curveto(288.3915775,91.53047675)(288.42803583,91.41589343)(288.47491082,91.35860177)
\curveto(288.52699415,91.30131011)(288.62334831,91.25183095)(288.76397329,91.21016429)
\curveto(288.90459827,91.17370596)(289.25095239,91.14506013)(289.80303566,91.1242268)
\lineto(289.86553565,91.0773518)
\lineto(289.86553565,90.68672685)
\lineto(289.80303566,90.62422686)
\curveto(288.47491082,90.66589352)(287.72751508,90.68672685)(287.56084844,90.68672685)
\curveto(287.38376513,90.68672685)(286.63897355,90.66589352)(285.32647371,90.62422686)
\lineto(285.26397372,90.68672685)
\lineto(285.26397372,91.0773518)
\lineto(285.32647371,91.1242268)
\curveto(285.81084865,91.14506013)(286.13376528,91.17110179)(286.29522359,91.20235179)
\curveto(286.45668191,91.23360178)(286.56345273,91.27266428)(286.61553605,91.31953927)
\curveto(286.67282771,91.3716226)(286.71709854,91.47839342)(286.74834854,91.63985173)
\curveto(286.78480687,91.80131005)(286.8056402,92.160685)(286.81084853,92.7179766)
\lineto(286.81866103,94.3664139)
\lineto(286.81866103,97.98360095)
\lineto(286.80303603,99.50703826)
\curveto(286.7978277,100.10078819)(286.7822027,100.48360064)(286.75616104,100.65547562)
\curveto(286.73532771,100.8273506)(286.69886938,100.94193392)(286.64678605,100.99922558)
\curveto(286.59991106,101.05651724)(286.50616107,101.10339223)(286.36553608,101.13985056)
\curveto(286.2249111,101.17630889)(285.87855698,101.20495472)(285.32647371,101.22578805)
\lineto(285.26397372,101.28047554)
\lineto(285.26397372,101.6711005)
\closepath
}
}
{
\newrgbcolor{curcolor}{0 0 0}
\pscustom[linestyle=none,fillstyle=solid,fillcolor=curcolor]
{
\newpath
\moveto(295.67803494,95.9836012)
\lineto(295.37334747,96.06172619)
\lineto(295.31084748,96.13985118)
\lineto(295.31084748,97.10860106)
\curveto(296.2379307,97.79610097)(297.13897226,98.13985093)(298.01397215,98.13985093)
\curveto(298.61293041,98.13985093)(299.11032618,98.02787178)(299.50615946,97.80391347)
\curveto(299.90199275,97.57995517)(300.18845105,97.2987052)(300.36553436,96.96016358)
\curveto(300.54261767,96.62683028)(300.63115933,96.23620533)(300.63115933,95.78828872)
\lineto(300.59209683,94.21797641)
\lineto(300.59209683,91.8898517)
\curveto(300.59209683,91.57214341)(300.61293016,91.3794351)(300.65459682,91.31172677)
\curveto(300.70147182,91.24401845)(300.75355514,91.19714345)(300.8108468,91.17110179)
\curveto(300.86813846,91.15026846)(300.97490928,91.13203929)(301.13115926,91.1164143)
\lineto(301.57647171,91.0773518)
\lineto(301.6389717,91.00703931)
\lineto(301.6389717,90.69453935)
\lineto(301.57647171,90.63203936)
\curveto(301.19626342,90.66328935)(300.8420968,90.67891435)(300.51397184,90.67891435)
\curveto(300.20147188,90.67891435)(299.82647192,90.66328935)(299.38897198,90.63203936)
\lineto(299.27178449,90.74141434)
\lineto(299.30303449,91.99922669)
\lineto(297.5999097,90.67110185)
\curveto(297.3134514,90.5513102)(297.00095144,90.49141437)(296.66240981,90.49141437)
\curveto(296.2457432,90.49141437)(295.88636824,90.5669352)(295.58428495,90.71797685)
\curveto(295.28740998,90.86901849)(295.05824335,91.07995597)(294.89678503,91.35078927)
\curveto(294.74053505,91.62162257)(294.66241006,91.94974753)(294.66241006,92.33516415)
\curveto(294.66241006,93.10078905)(294.90199336,93.69714315)(295.38115997,94.12422643)
\curveto(295.86032658,94.55651804)(297.16761809,94.92370549)(299.30303449,95.22578879)
\curveto(299.30303449,95.99662203)(299.13115951,96.53828863)(298.78740955,96.85078859)
\curveto(298.44365959,97.16849688)(297.98272215,97.32735103)(297.40459722,97.32735103)
\curveto(297.10251393,97.32735103)(296.82647229,97.2830802)(296.57647233,97.19453855)
\curveto(296.33168069,97.10599689)(296.19105571,97.03308023)(296.15459738,96.97578857)
\curveto(296.11813905,96.92370525)(295.97751407,96.60599695)(295.73272243,96.02266369)
\closepath
\moveto(299.30303449,94.74922635)
\curveto(297.84470134,94.50443471)(296.93584728,94.24141391)(296.57647233,93.96016395)
\curveto(296.21709737,93.67891398)(296.03740989,93.2440182)(296.03740989,92.65547661)
\curveto(296.03740989,91.84297671)(296.44105568,91.43672676)(297.24834724,91.43672676)
\curveto(297.94105549,91.43672676)(298.62595124,91.83776837)(299.30303449,92.63985161)
\closepath
\moveto(297.97490965,98.62422587)
\closepath
\moveto(297.81084717,90.21797691)
\closepath
}
}
{
\newrgbcolor{curcolor}{0 0 0}
\pscustom[linestyle=none,fillstyle=solid,fillcolor=curcolor]
{
\newpath
\moveto(305.02178378,98.13203843)
\lineto(305.17022127,98.03047594)
\curveto(305.13897127,97.72839265)(305.11553377,97.19193438)(305.09990877,96.42110114)
\lineto(305.6858462,97.16328855)
\curveto(305.87855451,97.40808019)(306.04782532,97.59558016)(306.19365864,97.72578848)
\curveto(306.34470029,97.86120513)(306.51917943,97.96537179)(306.71709607,98.03828844)
\curveto(306.91501272,98.1112051)(307.11813769,98.14766343)(307.326471,98.14766343)
\curveto(307.55563764,98.14766343)(307.77178344,98.10078844)(307.97490842,98.00703845)
\lineto(308.02959591,97.89766346)
\curveto(307.94626259,97.20495521)(307.8993876,96.59558029)(307.88897093,96.06953869)
\lineto(307.53740847,96.06953869)
\curveto(307.32907517,96.54870529)(306.99313771,96.7882886)(306.5295961,96.7882886)
\curveto(306.20667947,96.7882886)(305.92542951,96.68412194)(305.6858462,96.47578864)
\curveto(305.4462629,96.27266366)(305.28480458,96.01485119)(305.20147126,95.70235123)
\curveto(305.12334627,95.3950596)(305.08428378,95.00443465)(305.08428378,94.53047638)
\lineto(305.08428378,93.66328898)
\curveto(305.08428378,93.507039)(305.09470044,93.12422655)(305.11553377,92.51485162)
\curveto(305.1363671,91.9054767)(305.15720043,91.55391424)(305.17803376,91.46016425)
\curveto(305.20407543,91.36641427)(305.24053376,91.29610177)(305.28740875,91.24922678)
\curveto(305.33949208,91.20756012)(305.40459624,91.17891429)(305.48272123,91.16328929)
\curveto(305.56605455,91.14766429)(305.93845034,91.12162263)(306.59990859,91.0851643)
\lineto(306.67022108,91.02266431)
\lineto(306.67022108,90.69453935)
\lineto(306.59990859,90.62422686)
\curveto(305.90720034,90.66589352)(305.1832421,90.68672685)(304.42803386,90.68672685)
\curveto(303.8290756,90.68672685)(303.20407567,90.66589352)(302.55303409,90.62422686)
\lineto(302.4827216,90.69453935)
\lineto(302.4827216,91.02266431)
\lineto(302.55303409,91.0851643)
\curveto(303.05824236,91.1164143)(303.34209649,91.13985179)(303.40459648,91.15547679)
\curveto(303.47230481,91.17110179)(303.52959647,91.20235179)(303.57647146,91.24922678)
\curveto(303.62855479,91.30131011)(303.66240895,91.37683093)(303.67803395,91.47578925)
\curveto(303.69886728,91.57995591)(303.71970061,91.91589336)(303.74053394,92.48360163)
\curveto(303.76657561,93.05651822)(303.77959644,93.48880984)(303.77959644,93.78047647)
\lineto(303.77959644,95.31953878)
\curveto(303.77959644,95.52787209)(303.76917977,95.81172622)(303.74834644,96.17110117)
\curveto(303.72751311,96.53047613)(303.70928395,96.75183027)(303.69365895,96.83516359)
\curveto(303.68324228,96.91849691)(303.64417979,96.97839274)(303.57647146,97.01485107)
\curveto(303.50876314,97.05651773)(303.37334649,97.07735106)(303.17022151,97.07735106)
\lineto(302.51397159,97.08516356)
\lineto(302.4436591,97.14766355)
\lineto(302.4436591,97.48360101)
\lineto(302.50615909,97.546101)
\curveto(303.50095064,97.66589266)(304.3394922,97.86120513)(305.02178378,98.13203843)
\closepath
}
}
{
\newrgbcolor{curcolor}{0 0 0}
\pscustom[linestyle=none,fillstyle=solid,fillcolor=curcolor]
{
\newpath
\moveto(309.99834567,95.9836012)
\lineto(309.69365821,96.06172619)
\lineto(309.63115822,96.13985118)
\lineto(309.63115822,97.10860106)
\curveto(310.55824143,97.79610097)(311.45928299,98.13985093)(312.33428288,98.13985093)
\curveto(312.93324114,98.13985093)(313.43063691,98.02787178)(313.8264702,97.80391347)
\curveto(314.22230348,97.57995517)(314.50876178,97.2987052)(314.68584509,96.96016358)
\curveto(314.8629284,96.62683028)(314.95147006,96.23620533)(314.95147006,95.78828872)
\lineto(314.91240756,94.21797641)
\lineto(314.91240756,91.8898517)
\curveto(314.91240756,91.57214341)(314.93324089,91.3794351)(314.97490756,91.31172677)
\curveto(315.02178255,91.24401845)(315.07386588,91.19714345)(315.13115754,91.17110179)
\curveto(315.1884492,91.15026846)(315.29522002,91.13203929)(315.45147,91.1164143)
\lineto(315.89678244,91.0773518)
\lineto(315.95928244,91.00703931)
\lineto(315.95928244,90.69453935)
\lineto(315.89678244,90.63203936)
\curveto(315.51657416,90.66328935)(315.16240753,90.67891435)(314.83428257,90.67891435)
\curveto(314.52178261,90.67891435)(314.14678266,90.66328935)(313.70928271,90.63203936)
\lineto(313.59209523,90.74141434)
\lineto(313.62334522,91.99922669)
\lineto(311.92022043,90.67110185)
\curveto(311.63376214,90.5513102)(311.32126217,90.49141437)(310.98272055,90.49141437)
\curveto(310.56605393,90.49141437)(310.20667898,90.5669352)(309.90459568,90.71797685)
\curveto(309.60772072,90.86901849)(309.37855408,91.07995597)(309.21709577,91.35078927)
\curveto(309.06084579,91.62162257)(308.9827208,91.94974753)(308.9827208,92.33516415)
\curveto(308.9827208,93.10078905)(309.2223041,93.69714315)(309.70147071,94.12422643)
\curveto(310.18063731,94.55651804)(311.48792882,94.92370549)(313.62334522,95.22578879)
\curveto(313.62334522,95.99662203)(313.45147024,96.53828863)(313.10772029,96.85078859)
\curveto(312.76397033,97.16849688)(312.30303289,97.32735103)(311.72490796,97.32735103)
\curveto(311.42282466,97.32735103)(311.14678303,97.2830802)(310.89678306,97.19453855)
\curveto(310.65199142,97.10599689)(310.51136644,97.03308023)(310.47490811,96.97578857)
\curveto(310.43844978,96.92370525)(310.2978248,96.60599695)(310.05303316,96.02266369)
\closepath
\moveto(313.62334522,94.74922635)
\curveto(312.16501207,94.50443471)(311.25615802,94.24141391)(310.89678306,93.96016395)
\curveto(310.5374081,93.67891398)(310.35772063,93.2440182)(310.35772063,92.65547661)
\curveto(310.35772063,91.84297671)(310.76136641,91.43672676)(311.56865798,91.43672676)
\curveto(312.26136622,91.43672676)(312.94626197,91.83776837)(313.62334522,92.63985161)
\closepath
\moveto(312.29522039,98.62422587)
\closepath
\moveto(312.13115791,90.21797691)
\closepath
}
}
{
\newrgbcolor{curcolor}{0 0 0}
\pscustom[linestyle=none,fillstyle=solid,fillcolor=curcolor]
{
\newpath
\moveto(319.21709453,98.13203843)
\lineto(319.36553202,98.03047594)
\curveto(319.33428202,97.68672599)(319.31344869,97.2752677)(319.30303202,96.7961011)
\lineto(320.04521943,97.47578851)
\curveto(320.25355274,97.66849682)(320.39157356,97.79089264)(320.45928188,97.84297597)
\curveto(320.52699021,97.90026763)(320.68584435,97.96537179)(320.93584432,98.03828844)
\curveto(321.18584429,98.1112051)(321.44886509,98.14766343)(321.72490672,98.14766343)
\curveto(322.24573999,98.14766343)(322.7014691,98.01745511)(323.09209406,97.75703848)
\curveto(323.48271901,97.50183018)(323.77178147,97.14766355)(323.95928145,96.69453861)
\curveto(324.66240636,97.36120519)(325.07126048,97.73099681)(325.1858438,97.80391347)
\curveto(325.30563545,97.87683013)(325.49313543,97.94974679)(325.74834373,98.02266345)
\curveto(326.00355203,98.10078844)(326.25876033,98.13985093)(326.51396863,98.13985093)
\curveto(326.92542692,98.13985093)(327.30303104,98.05130928)(327.64678099,97.87422596)
\curveto(327.99053095,97.69714265)(328.26396842,97.46537185)(328.46709339,97.17891355)
\curveto(328.67021837,96.89245525)(328.78740585,96.58255946)(328.81865585,96.24922616)
\curveto(328.85511418,95.91589287)(328.87334334,95.45755959)(328.87334334,94.87422633)
\lineto(328.87334334,94.05391393)
\curveto(328.87334334,93.96537228)(328.88896834,93.31693486)(328.92021834,92.10860167)
\curveto(328.930635,91.61381007)(328.9853225,91.31953927)(329.08428082,91.22578928)
\curveto(329.18323914,91.13203929)(329.5035516,91.0851643)(330.0452182,91.0851643)
\lineto(330.10771819,91.02266431)
\lineto(330.10771819,90.68672685)
\lineto(330.0452182,90.62422686)
\curveto(329.38375995,90.66589352)(328.93584334,90.68672685)(328.70146836,90.68672685)
\curveto(328.57126005,90.68672685)(328.19365593,90.66589352)(327.568656,90.62422686)
\lineto(327.46709352,90.71016435)
\curveto(327.53480184,91.3872476)(327.568656,92.27005999)(327.568656,93.35860152)
\lineto(327.568656,94.2961014)
\curveto(327.568656,95.1242263)(327.52959351,95.70235123)(327.45146852,96.03047619)
\curveto(327.37334353,96.36380948)(327.19105188,96.63464278)(326.90459359,96.84297609)
\curveto(326.61813529,97.05651773)(326.2769895,97.16328855)(325.88115621,97.16328855)
\curveto(325.59990625,97.16328855)(325.33688545,97.10599689)(325.09209381,96.99141357)
\curveto(324.84730217,96.88203859)(324.62594803,96.71537194)(324.42803139,96.49141363)
\curveto(324.23532308,96.27266366)(324.12855226,96.06693452)(324.10771893,95.87422621)
\curveto(324.0868856,95.68672623)(324.07646893,95.33776794)(324.07646893,94.82735134)
\lineto(324.07646893,93.85078896)
\curveto(324.07646893,93.62162232)(324.0868856,93.19193487)(324.10771893,92.56172662)
\curveto(324.12855226,91.93151836)(324.14938559,91.56433091)(324.17021892,91.46016425)
\curveto(324.19626059,91.3559976)(324.23271891,91.28047678)(324.27959391,91.23360178)
\curveto(324.33167724,91.19193512)(324.3889689,91.16328929)(324.45146889,91.14766429)
\curveto(324.51917721,91.13724763)(324.80042718,91.1164143)(325.29521878,91.0851643)
\lineto(325.36553128,91.02266431)
\lineto(325.36553128,90.69453935)
\lineto(325.30303128,90.62422686)
\curveto(324.6519897,90.66589352)(324.02438561,90.68672685)(323.42021902,90.68672685)
\curveto(322.86292742,90.68672685)(322.23792749,90.66589352)(321.54521925,90.62422686)
\lineto(321.47490675,90.69453935)
\lineto(321.47490675,91.02266431)
\lineto(321.54521925,91.0851643)
\curveto(322.05042752,91.1164143)(322.33428165,91.13985179)(322.39678164,91.15547679)
\curveto(322.46448997,91.17110179)(322.52178163,91.20235179)(322.56865662,91.24922678)
\curveto(322.62073995,91.30131011)(322.65459411,91.37683093)(322.67021911,91.47578925)
\curveto(322.69105244,91.57995591)(322.71188577,91.91589336)(322.7327191,92.48360163)
\curveto(322.75876076,93.05651822)(322.7717816,93.48880984)(322.7717816,93.78047647)
\lineto(322.7717816,94.67110136)
\curveto(322.7717816,95.25964295)(322.73011493,95.7127679)(322.64678161,96.03047619)
\curveto(322.56865662,96.34818449)(322.39678164,96.61641362)(322.13115667,96.83516359)
\curveto(321.86553171,97.05391356)(321.53219841,97.16328855)(321.1311568,97.16328855)
\curveto(320.81865684,97.16328855)(320.53219854,97.10078856)(320.2717819,96.97578857)
\curveto(320.0165736,96.85599692)(319.80563613,96.70495527)(319.63896948,96.52266363)
\curveto(319.47751117,96.34558032)(319.37594868,96.17891367)(319.33428202,96.02266369)
\curveto(319.29782369,95.86641371)(319.27959453,95.52787209)(319.27959453,95.00703882)
\lineto(319.27959453,93.85078896)
\curveto(319.27959453,93.62162232)(319.29001119,93.19193487)(319.31084452,92.56172662)
\curveto(319.33167785,91.93151836)(319.35251118,91.56433091)(319.37334451,91.46016425)
\curveto(319.39938618,91.3559976)(319.43584451,91.28047678)(319.4827195,91.23360178)
\curveto(319.53480283,91.19193512)(319.59209449,91.16328929)(319.65459448,91.14766429)
\curveto(319.7223028,91.13724763)(320.00355277,91.1164143)(320.49834438,91.0851643)
\lineto(320.56865687,91.02266431)
\lineto(320.56865687,90.69453935)
\lineto(320.50615687,90.62422686)
\curveto(319.85511529,90.66589352)(319.2275112,90.68672685)(318.62334461,90.68672685)
\curveto(318.02438635,90.68672685)(317.39938642,90.66589352)(316.74834484,90.62422686)
\lineto(316.67803235,90.69453935)
\lineto(316.67803235,91.02266431)
\lineto(316.74834484,91.0851643)
\curveto(317.25355311,91.1164143)(317.53740724,91.13985179)(317.59990723,91.15547679)
\curveto(317.66761556,91.17110179)(317.72490722,91.20235179)(317.77178221,91.24922678)
\curveto(317.82386554,91.30131011)(317.8577197,91.37683093)(317.8733447,91.47578925)
\curveto(317.89417803,91.57995591)(317.91501136,91.91589336)(317.93584469,92.48360163)
\curveto(317.96188635,93.05651822)(317.97490719,93.48880984)(317.97490719,93.78047647)
\lineto(317.97490719,95.31953878)
\curveto(317.97490719,95.52787209)(317.96449052,95.81172622)(317.94365719,96.17110117)
\curveto(317.92282386,96.53047613)(317.9045947,96.75183027)(317.8889697,96.83516359)
\curveto(317.87855303,96.91849691)(317.83949054,96.97839274)(317.77178221,97.01485107)
\curveto(317.70407389,97.05651773)(317.56865724,97.07735106)(317.36553226,97.07735106)
\lineto(316.70928234,97.08516356)
\lineto(316.63896985,97.14766355)
\lineto(316.63896985,97.48360101)
\lineto(316.70146984,97.546101)
\curveto(317.69626139,97.66589266)(318.53480295,97.86120513)(319.21709453,98.13203843)
\closepath
}
}
{
\newrgbcolor{curcolor}{0 0 0}
\pscustom[linestyle=none,fillstyle=solid,fillcolor=curcolor]
{
\newpath
\moveto(337.37334229,91.83516421)
\lineto(337.12334233,91.27266428)
\curveto(336.58167573,90.92370599)(336.09730079,90.69714352)(335.6702175,90.59297686)
\curveto(335.24834256,90.48881021)(334.87073844,90.43672688)(334.53740514,90.43672688)
\curveto(333.91240522,90.43672688)(333.31865529,90.5591227)(332.75615536,90.80391434)
\curveto(332.19886377,91.04870597)(331.74313466,91.46276842)(331.38896803,92.04610168)
\curveto(331.04000974,92.62943494)(330.8655306,93.33255986)(330.8655306,94.15547642)
\curveto(330.8655306,94.70235135)(330.93323892,95.19453879)(331.06865557,95.63203874)
\curveto(331.20407222,96.07474702)(331.3446972,96.40287198)(331.49053052,96.61641362)
\curveto(331.64157217,96.82995526)(331.8941763,97.0669344)(332.24834293,97.32735103)
\curveto(332.60250955,97.58776767)(332.9775095,97.79610097)(333.37334279,97.95235095)
\curveto(333.76917607,98.10860093)(334.19625935,98.18672593)(334.65459263,98.18672593)
\curveto(335.27959255,98.18672593)(335.82907165,98.03828844)(336.30302993,97.74141348)
\curveto(336.78219653,97.44974685)(337.11813399,97.0747469)(337.3108423,96.61641362)
\curveto(337.50355061,96.15808034)(337.59990477,95.67110124)(337.59990477,95.1554763)
\curveto(337.59990477,94.99401799)(337.59209227,94.837768)(337.57646727,94.68672636)
\lineto(337.49052978,94.60078887)
\curveto(337.13636316,94.52266388)(336.65980072,94.47058055)(336.06084246,94.44453889)
\curveto(335.4618842,94.41849722)(335.06605091,94.40547639)(334.8733426,94.40547639)
\lineto(332.36553041,94.40547639)
\curveto(332.37594708,93.32735152)(332.64678038,92.53308079)(333.17803031,92.02266419)
\curveto(333.70928025,91.51224758)(334.36032183,91.25703928)(335.13115507,91.25703928)
\curveto(335.49573836,91.25703928)(335.84469665,91.31953927)(336.17802994,91.44453926)
\curveto(336.51657157,91.56953924)(336.87334236,91.74401839)(337.24834231,91.96797669)
\closepath
\moveto(332.36553041,95.03047631)
\curveto(332.4592804,95.01485132)(332.81865536,94.99662215)(333.44365528,94.97578882)
\curveto(334.07386353,94.95495549)(334.54000931,94.94453882)(334.84209261,94.94453882)
\curveto(335.56605085,94.94453882)(336.00615496,94.95755966)(336.16240494,94.98360132)
\curveto(336.16761328,95.1086013)(336.17021744,95.20495546)(336.17021744,95.27266378)
\curveto(336.17021744,96.07995535)(336.00615496,96.67891361)(335.67803,97.06953856)
\curveto(335.34990504,97.46537185)(334.90198843,97.66328849)(334.33428017,97.66328849)
\curveto(333.71448858,97.66328849)(333.23011364,97.44193435)(332.88115535,96.99922607)
\curveto(332.53740539,96.55651779)(332.36553041,95.90026787)(332.36553041,95.03047631)
\closepath
\moveto(334.54521764,98.62422587)
\closepath
\moveto(334.45146765,90.21797691)
\closepath
}
}
{
\newrgbcolor{curcolor}{0 0 0}
\pscustom[linestyle=none,fillstyle=solid,fillcolor=curcolor]
{
\newpath
\moveto(338.49052966,96.81953859)
\lineto(338.49052966,97.02266357)
\lineto(338.54521715,97.10078856)
\curveto(339.03480042,97.2830802)(339.43063371,97.45235102)(339.732717,97.608601)
\curveto(339.732717,98.96797583)(339.71709201,99.7596424)(339.68584201,99.9836007)
\curveto(340.22230028,100.17110068)(340.66240439,100.37162149)(341.00615435,100.58516313)
\lineto(341.19365432,100.42891315)
\curveto(341.141571,100.10078819)(341.079071,99.14766331)(341.00615435,97.5695385)
\curveto(341.26657098,97.56433017)(341.54782095,97.561726)(341.84990424,97.561726)
\curveto(342.4644875,97.561726)(342.90459161,97.577351)(343.17021658,97.608601)
\lineto(343.22490407,97.5539135)
\lineto(343.07646659,96.89766358)
\lineto(343.0139666,96.82735109)
\curveto(342.74834163,96.83255943)(342.45407083,96.83516359)(342.13115421,96.83516359)
\curveto(341.83948758,96.83516359)(341.46448762,96.83255943)(341.00615435,96.82735109)
\lineto(340.95927935,93.64766398)
\curveto(340.95927935,92.91328908)(340.97490435,92.4315183)(341.00615435,92.20235166)
\curveto(341.04261268,91.97839336)(341.13636266,91.80131005)(341.28740431,91.67110173)
\curveto(341.44365429,91.54610174)(341.67282093,91.48360175)(341.97490423,91.48360175)
\curveto(342.32386252,91.48360175)(342.64677914,91.57474757)(342.94365411,91.75703922)
\lineto(343.13115408,91.47578925)
\curveto(343.0061541,91.3872476)(342.70667497,91.12683096)(342.2327167,90.69453935)
\curveto(341.9618834,90.56953936)(341.68584176,90.50703937)(341.4045918,90.50703937)
\curveto(340.23271694,90.50703937)(339.64677951,91.07474764)(339.64677951,92.21016416)
\curveto(339.64677951,92.62683078)(339.65719618,92.9809974)(339.67802951,93.27266403)
\curveto(339.68323784,93.36120569)(339.68584201,93.45235151)(339.68584201,93.5461015)
\lineto(339.68584201,96.7882886)
\lineto(339.36552955,96.7882886)
\curveto(339.13115458,96.7882886)(338.86292544,96.77787193)(338.56084215,96.7570386)
\closepath
}
}
{
\newrgbcolor{curcolor}{0 0 0}
\pscustom[linestyle=none,fillstyle=solid,fillcolor=curcolor]
{
\newpath
\moveto(350.25615321,91.83516421)
\lineto(350.00615324,91.27266428)
\curveto(349.46448664,90.92370599)(348.9801117,90.69714352)(348.55302842,90.59297686)
\curveto(348.13115347,90.48881021)(347.75354935,90.43672688)(347.42021606,90.43672688)
\curveto(346.79521613,90.43672688)(346.20146621,90.5591227)(345.63896628,90.80391434)
\curveto(345.08167468,91.04870597)(344.62594557,91.46276842)(344.27177894,92.04610168)
\curveto(343.92282065,92.62943494)(343.74834151,93.33255986)(343.74834151,94.15547642)
\curveto(343.74834151,94.70235135)(343.81604983,95.19453879)(343.95146648,95.63203874)
\curveto(344.08688313,96.07474702)(344.22750812,96.40287198)(344.37334143,96.61641362)
\curveto(344.52438308,96.82995526)(344.77698722,97.0669344)(345.13115384,97.32735103)
\curveto(345.48532046,97.58776767)(345.86032041,97.79610097)(346.2561537,97.95235095)
\curveto(346.65198698,98.10860093)(347.07907026,98.18672593)(347.53740354,98.18672593)
\curveto(348.16240346,98.18672593)(348.71188256,98.03828844)(349.18584084,97.74141348)
\curveto(349.66500745,97.44974685)(350.0009449,97.0747469)(350.19365321,96.61641362)
\curveto(350.38636152,96.15808034)(350.48271568,95.67110124)(350.48271568,95.1554763)
\curveto(350.48271568,94.99401799)(350.47490318,94.837768)(350.45927818,94.68672636)
\lineto(350.37334069,94.60078887)
\curveto(350.01917407,94.52266388)(349.54261163,94.47058055)(348.94365337,94.44453889)
\curveto(348.34469511,94.41849722)(347.94886182,94.40547639)(347.75615351,94.40547639)
\lineto(345.24834132,94.40547639)
\curveto(345.25875799,93.32735152)(345.52959129,92.53308079)(346.06084122,92.02266419)
\curveto(346.59209116,91.51224758)(347.24313274,91.25703928)(348.01396598,91.25703928)
\curveto(348.37854927,91.25703928)(348.72750756,91.31953927)(349.06084085,91.44453926)
\curveto(349.39938248,91.56953924)(349.75615327,91.74401839)(350.13115322,91.96797669)
\closepath
\moveto(345.24834132,95.03047631)
\curveto(345.34209131,95.01485132)(345.70146627,94.99662215)(346.32646619,94.97578882)
\curveto(346.95667445,94.95495549)(347.42282022,94.94453882)(347.72490352,94.94453882)
\curveto(348.44886176,94.94453882)(348.88896587,94.95755966)(349.04521586,94.98360132)
\curveto(349.05042419,95.1086013)(349.05302835,95.20495546)(349.05302835,95.27266378)
\curveto(349.05302835,96.07995535)(348.88896587,96.67891361)(348.56084092,97.06953856)
\curveto(348.23271596,97.46537185)(347.78479934,97.66328849)(347.21709108,97.66328849)
\curveto(346.59729949,97.66328849)(346.11292455,97.44193435)(345.76396626,96.99922607)
\curveto(345.4202163,96.55651779)(345.24834132,95.90026787)(345.24834132,95.03047631)
\closepath
\moveto(347.42802855,98.62422587)
\closepath
\moveto(347.33427857,90.21797691)
\closepath
}
}
{
\newrgbcolor{curcolor}{0 0 0}
\pscustom[linestyle=none,fillstyle=solid,fillcolor=curcolor]
{
\newpath
\moveto(354.01396524,98.13203843)
\lineto(354.16240272,98.03047594)
\curveto(354.13115273,97.72839265)(354.10771523,97.19193438)(354.09209023,96.42110114)
\lineto(354.67802766,97.16328855)
\curveto(354.87073597,97.40808019)(355.04000678,97.59558016)(355.1858401,97.72578848)
\curveto(355.33688175,97.86120513)(355.51136089,97.96537179)(355.70927753,98.03828844)
\curveto(355.90719418,98.1112051)(356.11031915,98.14766343)(356.31865246,98.14766343)
\curveto(356.5478191,98.14766343)(356.7639649,98.10078844)(356.96708988,98.00703845)
\lineto(357.02177737,97.89766346)
\curveto(356.93844405,97.20495521)(356.89156905,96.59558029)(356.88115239,96.06953869)
\lineto(356.52958993,96.06953869)
\curveto(356.32125662,96.54870529)(355.98531917,96.7882886)(355.52177756,96.7882886)
\curveto(355.19886093,96.7882886)(354.91761096,96.68412194)(354.67802766,96.47578864)
\curveto(354.43844436,96.27266366)(354.27698604,96.01485119)(354.19365272,95.70235123)
\curveto(354.11552773,95.3950596)(354.07646524,95.00443465)(354.07646524,94.53047638)
\lineto(354.07646524,93.66328898)
\curveto(354.07646524,93.507039)(354.0868819,93.12422655)(354.10771523,92.51485162)
\curveto(354.12854856,91.9054767)(354.14938189,91.55391424)(354.17021522,91.46016425)
\curveto(354.19625689,91.36641427)(354.23271522,91.29610177)(354.27959021,91.24922678)
\curveto(354.33167354,91.20756012)(354.3967777,91.17891429)(354.47490269,91.16328929)
\curveto(354.55823601,91.14766429)(354.9306318,91.12162263)(355.59209005,91.0851643)
\lineto(355.66240254,91.02266431)
\lineto(355.66240254,90.69453935)
\lineto(355.59209005,90.62422686)
\curveto(354.8993818,90.66589352)(354.17542356,90.68672685)(353.42021532,90.68672685)
\curveto(352.82125706,90.68672685)(352.19625713,90.66589352)(351.54521555,90.62422686)
\lineto(351.47490306,90.69453935)
\lineto(351.47490306,91.02266431)
\lineto(351.54521555,91.0851643)
\curveto(352.05042382,91.1164143)(352.33427795,91.13985179)(352.39677794,91.15547679)
\curveto(352.46448627,91.17110179)(352.52177793,91.20235179)(352.56865292,91.24922678)
\curveto(352.62073625,91.30131011)(352.65459041,91.37683093)(352.67021541,91.47578925)
\curveto(352.69104874,91.57995591)(352.71188207,91.91589336)(352.7327154,92.48360163)
\curveto(352.75875706,93.05651822)(352.7717779,93.48880984)(352.7717779,93.78047647)
\lineto(352.7717779,95.31953878)
\curveto(352.7717779,95.52787209)(352.76136123,95.81172622)(352.7405279,96.17110117)
\curveto(352.71969457,96.53047613)(352.7014654,96.75183027)(352.68584041,96.83516359)
\curveto(352.67542374,96.91849691)(352.63636125,96.97839274)(352.56865292,97.01485107)
\curveto(352.5009446,97.05651773)(352.36552795,97.07735106)(352.16240297,97.07735106)
\lineto(351.50615305,97.08516356)
\lineto(351.43584056,97.14766355)
\lineto(351.43584056,97.48360101)
\lineto(351.49834055,97.546101)
\curveto(352.4931321,97.66589266)(353.33167366,97.86120513)(354.01396524,98.13203843)
\closepath
}
}
{
\newrgbcolor{curcolor}{0 0 0}
\pscustom[linestyle=none,fillstyle=solid,fillcolor=curcolor]
{
\newpath
\moveto(357.23271485,101.21016305)
\lineto(357.16240235,101.28047554)
\lineto(357.16240235,101.6086005)
\lineto(357.22490235,101.67891299)
\curveto(358.21969389,101.79349631)(359.05823545,101.98620462)(359.74052704,102.25703792)
\lineto(359.88896452,102.16328793)
\curveto(359.83167286,101.56432968)(359.80302703,100.36380899)(359.80302703,98.56172588)
\lineto(359.80302703,96.56953862)
\curveto(359.90198535,96.69974694)(360.08688116,96.89766358)(360.35771446,97.16328855)
\curveto(360.62854776,97.43412185)(360.84729773,97.62422599)(361.01396438,97.73360098)
\curveto(361.18583936,97.84297597)(361.41761016,97.93933012)(361.70927679,98.02266345)
\curveto(362.00615176,98.10599677)(362.31604755,98.14766343)(362.63896418,98.14766343)
\curveto(363.22750577,98.14766343)(363.74313071,98.00183011)(364.18583899,97.71016348)
\curveto(364.6337556,97.41849685)(364.96448473,97.0278719)(365.17802637,96.53828863)
\curveto(365.39677634,96.05391369)(365.50615133,95.53568459)(365.50615133,94.98360132)
\curveto(365.50615133,94.45755972)(365.4150055,93.95755978)(365.23271386,93.4836015)
\curveto(365.05563055,93.00964323)(364.84469307,92.61901828)(364.59990144,92.31172665)
\curveto(364.45406812,92.13464334)(364.19625565,91.91068503)(363.82646403,91.63985173)
\curveto(363.10250579,91.11381013)(362.59208919,90.78568517)(362.29521422,90.65547685)
\curveto(362.00354759,90.5200602)(361.6233393,90.45235188)(361.15458936,90.45235188)
\curveto(360.79000607,90.45235188)(360.45927695,90.50443521)(360.16240198,90.60860186)
\curveto(359.86552702,90.70756018)(359.55823539,90.86641433)(359.2405271,91.0851643)
\lineto(358.60771468,90.49141437)
\lineto(358.34990221,90.59297686)
\curveto(358.44886053,91.54089341)(358.49833969,92.61901828)(358.49833969,93.82735146)
\lineto(358.49833969,97.96797595)
\lineto(358.46708969,99.46797577)
\curveto(358.4514647,100.04089236)(358.43323553,100.44193398)(358.4124022,100.67110062)
\curveto(358.3967772,100.90547559)(358.37333971,101.04349641)(358.34208971,101.08516307)
\curveto(358.31083971,101.12682973)(358.25615222,101.15807973)(358.17802723,101.17891306)
\curveto(358.10511057,101.19974639)(357.79000644,101.21016305)(357.23271485,101.21016305)
\closepath
\moveto(359.80302703,95.63203874)
\lineto(359.80302703,92.7023516)
\curveto(359.80302703,92.45755996)(359.81344369,92.29610165)(359.83427703,92.21797666)
\curveto(359.86031869,92.14506)(359.96448534,92.03047668)(360.14677699,91.8742267)
\curveto(360.33427696,91.72318506)(360.57386027,91.59037257)(360.8655269,91.47578925)
\curveto(361.15719353,91.36641427)(361.45927682,91.31172677)(361.77177679,91.31172677)
\curveto(362.51136003,91.31172677)(363.08948496,91.59818507)(363.50615157,92.17110167)
\curveto(363.92802652,92.74401826)(364.13896399,93.47058067)(364.13896399,94.3507889)
\curveto(364.13896399,94.90287216)(364.05042234,95.38985127)(363.87333903,95.81172622)
\curveto(363.70146405,96.2388095)(363.44104741,96.56433029)(363.09208912,96.7882886)
\curveto(362.74313083,97.0122469)(362.34990171,97.12422606)(361.91240177,97.12422606)
\curveto(361.39677683,97.12422606)(360.94104772,96.97058024)(360.54521444,96.66328861)
\curveto(360.15458949,96.36120532)(359.90719368,96.01745536)(359.80302703,95.63203874)
\closepath
}
}
{
\newrgbcolor{curcolor}{0 0 0}
\pscustom[linestyle=none,fillstyle=solid,fillcolor=curcolor]
{
\newpath
\moveto(373.09208789,91.83516421)
\lineto(372.84208792,91.27266428)
\curveto(372.30042132,90.92370599)(371.81604638,90.69714352)(371.3889631,90.59297686)
\curveto(370.96708815,90.48881021)(370.58948403,90.43672688)(370.25615074,90.43672688)
\curveto(369.63115082,90.43672688)(369.03740089,90.5591227)(368.47490096,90.80391434)
\curveto(367.91760936,91.04870597)(367.46188025,91.46276842)(367.10771363,92.04610168)
\curveto(366.75875534,92.62943494)(366.58427619,93.33255986)(366.58427619,94.15547642)
\curveto(366.58427619,94.70235135)(366.65198452,95.19453879)(366.78740117,95.63203874)
\curveto(366.92281782,96.07474702)(367.0634428,96.40287198)(367.20927612,96.61641362)
\curveto(367.36031776,96.82995526)(367.6129219,97.0669344)(367.96708852,97.32735103)
\curveto(368.32125515,97.58776767)(368.6962551,97.79610097)(369.09208838,97.95235095)
\curveto(369.48792167,98.10860093)(369.91500495,98.18672593)(370.37333823,98.18672593)
\curveto(370.99833815,98.18672593)(371.54781725,98.03828844)(372.02177552,97.74141348)
\curveto(372.50094213,97.44974685)(372.83687959,97.0747469)(373.0295879,96.61641362)
\curveto(373.22229621,96.15808034)(373.31865036,95.67110124)(373.31865036,95.1554763)
\curveto(373.31865036,94.99401799)(373.31083786,94.837768)(373.29521287,94.68672636)
\lineto(373.20927538,94.60078887)
\curveto(372.85510875,94.52266388)(372.37854631,94.47058055)(371.77958805,94.44453889)
\curveto(371.18062979,94.41849722)(370.78479651,94.40547639)(370.5920882,94.40547639)
\lineto(368.08427601,94.40547639)
\curveto(368.09469267,93.32735152)(368.36552597,92.53308079)(368.89677591,92.02266419)
\curveto(369.42802584,91.51224758)(370.07906743,91.25703928)(370.84990067,91.25703928)
\curveto(371.21448396,91.25703928)(371.56344225,91.31953927)(371.89677554,91.44453926)
\curveto(372.23531716,91.56953924)(372.59208795,91.74401839)(372.96708791,91.96797669)
\closepath
\moveto(368.08427601,95.03047631)
\curveto(368.178026,95.01485132)(368.53740095,94.99662215)(369.16240087,94.97578882)
\curveto(369.79260913,94.95495549)(370.25875491,94.94453882)(370.5608382,94.94453882)
\curveto(371.28479645,94.94453882)(371.72490056,94.95755966)(371.88115054,94.98360132)
\curveto(371.88635887,95.1086013)(371.88896304,95.20495546)(371.88896304,95.27266378)
\curveto(371.88896304,96.07995535)(371.72490056,96.67891361)(371.3967756,97.06953856)
\curveto(371.06865064,97.46537185)(370.62073403,97.66328849)(370.05302577,97.66328849)
\curveto(369.43323417,97.66328849)(368.94885923,97.44193435)(368.59990094,96.99922607)
\curveto(368.25615099,96.55651779)(368.08427601,95.90026787)(368.08427601,95.03047631)
\closepath
\moveto(370.26396324,98.62422587)
\closepath
\moveto(370.17021325,90.21797691)
\closepath
}
}
{
\newrgbcolor{curcolor}{0 0 0}
\pscustom[linestyle=none,fillstyle=solid,fillcolor=curcolor]
{
\newpath
\moveto(376.84989993,98.13203843)
\lineto(376.99833741,98.03047594)
\curveto(376.96708741,97.72839265)(376.94364992,97.19193438)(376.92802492,96.42110114)
\lineto(377.51396235,97.16328855)
\curveto(377.70667065,97.40808019)(377.87594147,97.59558016)(378.02177478,97.72578848)
\curveto(378.17281643,97.86120513)(378.34729558,97.96537179)(378.54521222,98.03828844)
\curveto(378.74312886,98.1112051)(378.94625384,98.14766343)(379.15458714,98.14766343)
\curveto(379.38375378,98.14766343)(379.59989959,98.10078844)(379.80302456,98.00703845)
\lineto(379.85771206,97.89766346)
\curveto(379.77437873,97.20495521)(379.72750374,96.59558029)(379.71708707,96.06953869)
\lineto(379.36552462,96.06953869)
\curveto(379.15719131,96.54870529)(378.82125385,96.7882886)(378.35771224,96.7882886)
\curveto(378.03479561,96.7882886)(377.75354565,96.68412194)(377.51396235,96.47578864)
\curveto(377.27437904,96.27266366)(377.11292073,96.01485119)(377.0295874,95.70235123)
\curveto(376.95146241,95.3950596)(376.91239992,95.00443465)(376.91239992,94.53047638)
\lineto(376.91239992,93.66328898)
\curveto(376.91239992,93.507039)(376.92281658,93.12422655)(376.94364992,92.51485162)
\curveto(376.96448325,91.9054767)(376.98531658,91.55391424)(377.00614991,91.46016425)
\curveto(377.03219157,91.36641427)(377.0686499,91.29610177)(377.11552489,91.24922678)
\curveto(377.16760822,91.20756012)(377.23271238,91.17891429)(377.31083737,91.16328929)
\curveto(377.39417069,91.14766429)(377.76656648,91.12162263)(378.42802473,91.0851643)
\lineto(378.49833722,91.02266431)
\lineto(378.49833722,90.69453935)
\lineto(378.42802473,90.62422686)
\curveto(377.73531648,90.66589352)(377.01135824,90.68672685)(376.25615,90.68672685)
\curveto(375.65719174,90.68672685)(375.03219182,90.66589352)(374.38115023,90.62422686)
\lineto(374.31083774,90.69453935)
\lineto(374.31083774,91.02266431)
\lineto(374.38115023,91.0851643)
\curveto(374.8863585,91.1164143)(375.17021263,91.13985179)(375.23271263,91.15547679)
\curveto(375.30042095,91.17110179)(375.35771261,91.20235179)(375.40458761,91.24922678)
\curveto(375.45667093,91.30131011)(375.49052509,91.37683093)(375.50615009,91.47578925)
\curveto(375.52698342,91.57995591)(375.54781675,91.91589336)(375.56865009,92.48360163)
\curveto(375.59469175,93.05651822)(375.60771258,93.48880984)(375.60771258,93.78047647)
\lineto(375.60771258,95.31953878)
\curveto(375.60771258,95.52787209)(375.59729591,95.81172622)(375.57646258,96.17110117)
\curveto(375.55562925,96.53047613)(375.53740009,96.75183027)(375.52177509,96.83516359)
\curveto(375.51135843,96.91849691)(375.47229593,96.97839274)(375.40458761,97.01485107)
\curveto(375.33687928,97.05651773)(375.20146263,97.07735106)(374.99833766,97.07735106)
\lineto(374.34208774,97.08516356)
\lineto(374.27177524,97.14766355)
\lineto(374.27177524,97.48360101)
\lineto(374.33427524,97.546101)
\curveto(375.32906678,97.66589266)(376.16760834,97.86120513)(376.84989993,98.13203843)
\closepath
}
}
{
\newrgbcolor{curcolor}{0 0 0}
\pscustom[linestyle=none,fillstyle=solid,fillcolor=curcolor]
{
\newpath
\moveto(387.07646117,91.83516421)
\lineto(386.8264612,91.27266428)
\curveto(386.2847946,90.92370599)(385.80041966,90.69714352)(385.37333638,90.59297686)
\curveto(384.95146143,90.48881021)(384.57385731,90.43672688)(384.24052402,90.43672688)
\curveto(383.61552409,90.43672688)(383.02177417,90.5591227)(382.45927424,90.80391434)
\curveto(381.90198264,91.04870597)(381.44625353,91.46276842)(381.0920869,92.04610168)
\curveto(380.74312861,92.62943494)(380.56864947,93.33255986)(380.56864947,94.15547642)
\curveto(380.56864947,94.70235135)(380.63635779,95.19453879)(380.77177444,95.63203874)
\curveto(380.90719109,96.07474702)(381.04781608,96.40287198)(381.19364939,96.61641362)
\curveto(381.34469104,96.82995526)(381.59729518,97.0669344)(381.9514618,97.32735103)
\curveto(382.30562842,97.58776767)(382.68062837,97.79610097)(383.07646166,97.95235095)
\curveto(383.47229494,98.10860093)(383.89937822,98.18672593)(384.3577115,98.18672593)
\curveto(384.98271142,98.18672593)(385.53219052,98.03828844)(386.0061488,97.74141348)
\curveto(386.48531541,97.44974685)(386.82125286,97.0747469)(387.01396117,96.61641362)
\curveto(387.20666948,96.15808034)(387.30302364,95.67110124)(387.30302364,95.1554763)
\curveto(387.30302364,94.99401799)(387.29521114,94.837768)(387.27958614,94.68672636)
\lineto(387.19364865,94.60078887)
\curveto(386.83948203,94.52266388)(386.36291959,94.47058055)(385.76396133,94.44453889)
\curveto(385.16500307,94.41849722)(384.76916978,94.40547639)(384.57646147,94.40547639)
\lineto(382.06864928,94.40547639)
\curveto(382.07906595,93.32735152)(382.34989925,92.53308079)(382.88114918,92.02266419)
\curveto(383.41239912,91.51224758)(384.0634407,91.25703928)(384.83427394,91.25703928)
\curveto(385.19885723,91.25703928)(385.54781552,91.31953927)(385.88114881,91.44453926)
\curveto(386.21969044,91.56953924)(386.57646123,91.74401839)(386.95146118,91.96797669)
\closepath
\moveto(382.06864928,95.03047631)
\curveto(382.16239927,95.01485132)(382.52177423,94.99662215)(383.14677415,94.97578882)
\curveto(383.77698241,94.95495549)(384.24312818,94.94453882)(384.54521148,94.94453882)
\curveto(385.26916972,94.94453882)(385.70927383,94.95755966)(385.86552382,94.98360132)
\curveto(385.87073215,95.1086013)(385.87333631,95.20495546)(385.87333631,95.27266378)
\curveto(385.87333631,96.07995535)(385.70927383,96.67891361)(385.38114888,97.06953856)
\curveto(385.05302392,97.46537185)(384.6051073,97.66328849)(384.03739904,97.66328849)
\curveto(383.41760745,97.66328849)(382.93323251,97.44193435)(382.58427422,96.99922607)
\curveto(382.24052426,96.55651779)(382.06864928,95.90026787)(382.06864928,95.03047631)
\closepath
\moveto(384.24833651,98.62422587)
\closepath
\moveto(384.15458653,90.21797691)
\closepath
}
}
{
\newrgbcolor{curcolor}{0 0 0}
\pscustom[linestyle=none,fillstyle=solid,fillcolor=curcolor]
{
\newpath
\moveto(390.0842733,101.6242255)
\curveto(390.32906493,101.6242255)(390.53739824,101.53828801)(390.70927322,101.36641303)
\curveto(390.8811482,101.19453805)(390.96708569,100.98620475)(390.96708569,100.74141311)
\curveto(390.96708569,100.50182981)(390.8811482,100.29610066)(390.70927322,100.12422569)
\curveto(390.53739824,99.95235071)(390.32906493,99.86641322)(390.0842733,99.86641322)
\curveto(389.84468999,99.86641322)(389.63635668,99.94974654)(389.45927337,100.11641319)
\curveto(389.28739839,100.28828817)(389.2014609,100.49662147)(389.2014609,100.74141311)
\curveto(389.2014609,100.98620475)(389.28739839,101.19453805)(389.45927337,101.36641303)
\curveto(389.63635668,101.53828801)(389.84468999,101.6242255)(390.0842733,101.6242255)
\closepath
\moveto(390.74833571,98.13203843)
\lineto(390.8967732,98.03047594)
\curveto(390.83948154,97.34818436)(390.81083571,96.48099697)(390.81083571,95.42891377)
\lineto(390.81083571,93.66328898)
\curveto(390.81083571,93.54870566)(390.82125237,93.17891404)(390.8420857,92.55391412)
\curveto(390.86291903,91.93412253)(390.88375236,91.57214341)(390.90458569,91.46797675)
\curveto(390.93062736,91.3638101)(390.96708569,91.28828928)(391.01396068,91.24141428)
\curveto(391.06083567,91.19453929)(391.11812733,91.16328929)(391.18583566,91.14766429)
\curveto(391.25354398,91.13724763)(391.53479395,91.1164143)(392.02958556,91.0851643)
\lineto(392.09989805,91.02266431)
\lineto(392.09989805,90.69453935)
\lineto(392.03739805,90.62422686)
\curveto(391.38635647,90.66589352)(390.75875238,90.68672685)(390.15458579,90.68672685)
\curveto(389.55562753,90.68672685)(388.9306276,90.66589352)(388.27958602,90.62422686)
\lineto(388.20927353,90.69453935)
\lineto(388.20927353,91.02266431)
\lineto(388.27958602,91.0851643)
\curveto(388.78479429,91.1164143)(389.06864842,91.13985179)(389.13114841,91.15547679)
\curveto(389.19885674,91.17110179)(389.2561484,91.20235179)(389.30302339,91.24922678)
\curveto(389.35510672,91.30131011)(389.38896088,91.37683093)(389.40458588,91.47578925)
\curveto(389.42541921,91.57995591)(389.44625254,91.91589336)(389.46708587,92.48360163)
\curveto(389.49312753,93.05651822)(389.50614837,93.48880984)(389.50614837,93.78047647)
\lineto(389.50614837,95.31953878)
\curveto(389.50614837,95.52787209)(389.4957317,95.81172622)(389.47489837,96.17110117)
\curveto(389.45406504,96.53047613)(389.43583588,96.75183027)(389.42021088,96.83516359)
\curveto(389.40979421,96.91849691)(389.37073172,96.97839274)(389.30302339,97.01485107)
\curveto(389.23531507,97.05651773)(389.09989842,97.07735106)(388.89677344,97.07735106)
\lineto(388.24052352,97.08516356)
\lineto(388.17021103,97.14766355)
\lineto(388.17021103,97.48360101)
\lineto(388.23271102,97.546101)
\curveto(389.22750257,97.66589266)(390.06604413,97.86120513)(390.74833571,98.13203843)
\closepath
\moveto(390.13896079,90.21797691)
\closepath
}
}
{
\newrgbcolor{curcolor}{0 0 0}
\pscustom[linestyle=none,fillstyle=solid,fillcolor=curcolor]
{
\newpath
\moveto(399.27177216,91.52266425)
\lineto(398.9592722,90.99922681)
\curveto(398.28739728,90.62943519)(397.53739738,90.44453938)(396.70927248,90.44453938)
\curveto(395.56343929,90.44453938)(394.67802273,90.78568517)(394.05302281,91.46797675)
\curveto(393.42802288,92.15026834)(393.11552292,93.03047656)(393.11552292,94.10860143)
\curveto(393.11552292,94.5981847)(393.16760625,95.03308048)(393.2717729,95.41328877)
\curveto(393.38114789,95.79870539)(393.51656454,96.11641368)(393.67802285,96.36641365)
\curveto(393.8446895,96.61641362)(394.02698114,96.81433026)(394.22489778,96.96016358)
\curveto(394.42281443,97.10599689)(394.75614772,97.30912187)(395.22489766,97.5695385)
\curveto(395.69885594,97.82995514)(396.05562673,97.99922595)(396.29521003,98.07735094)
\curveto(396.53479333,98.16068426)(396.86552246,98.20235092)(397.28739741,98.20235092)
\curveto(398.09468897,98.20235092)(398.75614723,98.04870511)(399.27177216,97.74141348)
\curveto(399.17281384,97.15287189)(399.09729302,96.49662197)(399.04520969,95.77266372)
\lineto(398.9748972,95.71016373)
\lineto(398.65458474,95.71016373)
\lineto(398.58427225,95.78047622)
\curveto(398.56343892,96.21797617)(398.53479309,96.51224696)(398.49833476,96.66328861)
\curveto(398.46187643,96.81433026)(398.25354312,96.97578857)(397.87333483,97.14766355)
\curveto(397.49833488,97.31953853)(397.09729326,97.40547602)(396.67020998,97.40547602)
\curveto(396.2431267,97.40547602)(395.86291842,97.3143302)(395.52958512,97.13203856)
\curveto(395.19625183,96.95495524)(394.9436477,96.66068445)(394.77177272,96.24922616)
\curveto(394.60510607,95.84297621)(394.52177275,95.35599711)(394.52177275,94.78828884)
\curveto(394.52177275,94.30912224)(394.58948107,93.84037229)(394.72489772,93.38203902)
\curveto(394.86031437,92.92891407)(395.03739768,92.55130995)(395.25614766,92.24922666)
\curveto(395.48010596,91.95235169)(395.77437676,91.71276839)(396.13896005,91.53047675)
\curveto(396.50354334,91.3481851)(396.90979329,91.25703928)(397.3577099,91.25703928)
\curveto(397.63895986,91.25703928)(397.91239733,91.29349761)(398.1780223,91.36641427)
\curveto(398.4488556,91.43933092)(398.75875139,91.55912258)(399.10770968,91.72578922)
\closepath
}
}
{
\newrgbcolor{curcolor}{0 0 0}
\pscustom[linestyle=none,fillstyle=solid,fillcolor=curcolor]
{
\newpath
\moveto(402.2483343,102.25703792)
\lineto(402.39677178,102.16328793)
\curveto(402.33948012,101.56432968)(402.31083429,100.36380899)(402.31083429,98.56172588)
\lineto(402.31083429,96.63203862)
\curveto(402.74833423,96.99141357)(403.15458418,97.34818436)(403.52958414,97.70235098)
\curveto(403.63895912,97.80130931)(403.74312578,97.87422596)(403.8420841,97.92110096)
\curveto(403.94104242,97.96797595)(404.09989657,98.01745511)(404.31864654,98.06953844)
\curveto(404.54260485,98.12162177)(404.77177148,98.14766343)(405.00614646,98.14766343)
\curveto(405.40197974,98.14766343)(405.78479219,98.06693427)(406.15458381,97.90547596)
\curveto(406.52958377,97.74401765)(406.81083373,97.55130934)(406.99833371,97.32735103)
\curveto(407.19104202,97.10860106)(407.3238545,96.84818442)(407.39677116,96.54610113)
\curveto(407.46968782,96.24401783)(407.50614615,95.87162204)(407.50614615,95.42891377)
\lineto(407.50614615,94.05391393)
\curveto(407.50614615,93.96537228)(407.51916698,93.31693486)(407.54520864,92.10860167)
\curveto(407.56083364,91.61381007)(407.61552113,91.31953927)(407.70927112,91.22578928)
\curveto(407.80822944,91.13203929)(408.13114607,91.0851643)(408.678021,91.0851643)
\lineto(408.74052099,91.02266431)
\lineto(408.74052099,90.68672685)
\lineto(408.678021,90.62422686)
\curveto(408.01656275,90.66589352)(407.56864614,90.68672685)(407.33427117,90.68672685)
\curveto(407.21447952,90.68672685)(406.83427123,90.66589352)(406.19364631,90.62422686)
\lineto(406.09989632,90.71016435)
\curveto(406.16760465,91.3872476)(406.20145881,92.27005999)(406.20145881,93.35860152)
\lineto(406.20145881,94.38203889)
\curveto(406.20145881,94.99662215)(406.18583381,95.43933043)(406.15458381,95.71016373)
\curveto(406.12854215,95.98099703)(406.04260466,96.22578867)(405.89677135,96.44453864)
\curveto(405.75093803,96.66849695)(405.55562555,96.84037192)(405.31083392,96.96016358)
\curveto(405.07125061,97.08516356)(404.77958398,97.14766355)(404.43583403,97.14766355)
\curveto(404.0816674,97.14766355)(403.78479244,97.09818439)(403.54520914,96.99922607)
\curveto(403.31083416,96.90547608)(403.07125086,96.74401777)(402.82645922,96.51485113)
\curveto(402.58166759,96.28568449)(402.43323011,96.09297618)(402.38114678,95.9367262)
\curveto(402.33427178,95.78568455)(402.31083429,95.49141376)(402.31083429,95.05391381)
\lineto(402.31083429,93.66328898)
\curveto(402.31083429,93.54870566)(402.32125095,93.17891404)(402.34208428,92.55391412)
\curveto(402.36291761,91.93412253)(402.38375095,91.57214341)(402.40458428,91.46797675)
\curveto(402.43062594,91.3638101)(402.46708427,91.28828928)(402.51395926,91.24141428)
\curveto(402.56083426,91.19453929)(402.61812592,91.16328929)(402.68583424,91.14766429)
\curveto(402.75354257,91.13724763)(403.03479253,91.1164143)(403.52958414,91.0851643)
\lineto(403.59989663,91.02266431)
\lineto(403.59989663,90.69453935)
\lineto(403.53739664,90.62422686)
\curveto(402.88635505,90.66589352)(402.25875096,90.68672685)(401.65458437,90.68672685)
\curveto(401.05562611,90.68672685)(400.43062619,90.66589352)(399.7795846,90.62422686)
\lineto(399.70927211,90.69453935)
\lineto(399.70927211,91.02266431)
\lineto(399.7795846,91.0851643)
\curveto(400.28479287,91.1164143)(400.568647,91.13985179)(400.63114699,91.15547679)
\curveto(400.69885532,91.17110179)(400.75614698,91.20235179)(400.80302197,91.24922678)
\curveto(400.8551053,91.30131011)(400.88895946,91.37683093)(400.90458446,91.47578925)
\curveto(400.92541779,91.57995591)(400.94625112,91.91589336)(400.96708445,92.48360163)
\curveto(400.99312612,93.05651822)(401.00614695,93.48880984)(401.00614695,93.78047647)
\lineto(401.00614695,97.82735097)
\lineto(400.97489695,99.48360077)
\curveto(400.95927195,100.06172569)(400.94104279,100.46276731)(400.92020946,100.68672562)
\curveto(400.90458446,100.91068392)(400.88114696,101.04349641)(400.84989697,101.08516307)
\curveto(400.81864697,101.12682973)(400.76395948,101.15807973)(400.68583449,101.17891306)
\curveto(400.6077095,101.19974639)(400.29260537,101.21016305)(399.7405221,101.21016305)
\lineto(399.67020961,101.28047554)
\lineto(399.67020961,101.6086005)
\lineto(399.73270961,101.67891299)
\curveto(400.72750115,101.79349631)(401.56604271,101.98620462)(402.2483343,102.25703792)
\closepath
}
}
{
\newrgbcolor{curcolor}{1 1 1}
\pscustom[linestyle=none,fillstyle=solid,fillcolor=curcolor]
{
\newpath
\moveto(27.3012357,73.97979101)
\curveto(27.3012357,69.80504054)(23.91693177,66.42073661)(19.74218131,66.42073661)
\curveto(15.56743084,66.42073661)(12.18312691,69.80504054)(12.18312691,73.97979101)
\curveto(12.18312691,78.15454147)(15.56743084,81.53884541)(19.74218131,81.53884541)
\curveto(23.91693177,81.53884541)(27.3012357,78.15454147)(27.3012357,73.97979101)
\closepath
}
}
{
\newrgbcolor{curcolor}{0.15686275 0.16078432 0.16470589}
\pscustom[linewidth=2.88359956,linecolor=curcolor]
{
\newpath
\moveto(27.3012357,73.97979101)
\curveto(27.3012357,69.80504054)(23.91693177,66.42073661)(19.74218131,66.42073661)
\curveto(15.56743084,66.42073661)(12.18312691,69.80504054)(12.18312691,73.97979101)
\curveto(12.18312691,78.15454147)(15.56743084,81.53884541)(19.74218131,81.53884541)
\curveto(23.91693177,81.53884541)(27.3012357,78.15454147)(27.3012357,73.97979101)
\closepath
}
}
{
\newrgbcolor{curcolor}{0 0 0}
\pscustom[linestyle=none,fillstyle=solid,fillcolor=curcolor]
{
\newpath
\moveto(23.42186835,68.60869792)
\lineto(23.33593086,68.51494793)
\curveto(22.58072262,68.55661459)(21.73957689,68.57744792)(20.81249367,68.57744792)
\curveto(20.28645207,68.57744792)(19.41926468,68.55661459)(18.21093149,68.51494793)
\lineto(18.1328065,68.60088542)
\lineto(18.1328065,69.08526036)
\lineto(18.21874399,69.17119785)
\lineto(19.05468139,69.23369784)
\curveto(19.47134801,69.26494784)(19.71874381,69.3222395)(19.7968688,69.40557282)
\curveto(19.90103545,69.52015614)(19.96093128,69.80140611)(19.97655628,70.24932272)
\lineto(20.00780627,71.27276009)
\lineto(18.9765564,71.27276009)
\lineto(17.41405659,71.25713509)
\curveto(17.13280663,71.25713509)(16.71874418,71.2493226)(16.17186925,71.2336976)
\lineto(16.06249426,71.88213502)
\curveto(16.2760359,72.54359327)(16.84374416,73.72848896)(17.76561905,75.43682208)
\curveto(18.7135356,77.1920302)(19.45051467,78.39775921)(19.97655628,79.05400913)
\lineto(21.93749353,79.44463408)
\lineto(22.07811852,79.3118216)
\curveto(21.98436853,78.30661339)(21.93749353,76.66859276)(21.93749353,74.39775971)
\lineto(21.93749353,72.56182243)
\lineto(22.1874935,72.56182243)
\curveto(22.43228514,72.56182243)(22.79686843,72.57484327)(23.28124337,72.60088493)
\lineto(23.37499336,72.51494744)
\curveto(23.35936836,72.00453084)(23.35936836,71.61911422)(23.37499336,71.35869758)
\lineto(23.28124337,71.26494759)
\curveto(23.19791005,71.27015593)(23.06509756,71.27276009)(22.88280592,71.27276009)
\lineto(21.93749353,71.27276009)
\curveto(21.93749353,70.1269269)(21.99478519,69.49932281)(22.10936851,69.38994782)
\curveto(22.2031185,69.30140617)(22.40363931,69.24411451)(22.71093094,69.21807285)
\curveto(22.94530591,69.19723952)(23.15624338,69.17380202)(23.34374336,69.14776035)
\lineto(23.42186835,69.07744786)
\closepath
\moveto(20.10155626,77.99932176)
\curveto(19.40884801,76.8899469)(18.89582724,76.02275951)(18.56249395,75.39775958)
\curveto(18.33332731,74.9706763)(17.88280653,74.02536392)(17.21093162,72.56182243)
\lineto(20.01561877,72.56182243)
\closepath
}
}
{
\newrgbcolor{curcolor}{0 0 0}
\pscustom[linestyle=none,fillstyle=solid,fillcolor=curcolor]
{
\newpath
\moveto(34.14681718,68.96457353)
\lineto(34.07650469,69.02707352)
\lineto(34.07650469,69.43332347)
\lineto(34.14681718,69.49582346)
\curveto(34.52702547,69.5010318)(34.77181711,69.51926096)(34.88119209,69.55051096)
\curveto(34.99056708,69.58696928)(35.08692123,69.65467761)(35.17025456,69.75363593)
\curveto(35.25879621,69.85780258)(35.43327536,70.19113588)(35.69369199,70.75363581)
\lineto(36.51400439,72.53488559)
\lineto(38.63900413,77.34738499)
\curveto(39.04004575,78.29009321)(39.4384832,79.23280143)(39.83431648,80.17550965)
\lineto(40.34212892,80.17550965)
\lineto(43.67806601,72.33176061)
\lineto(44.12337845,71.28488574)
\curveto(44.25358677,70.98801078)(44.40202425,70.66509415)(44.5686909,70.31613586)
\curveto(44.74056588,69.96717757)(44.85775336,69.76144843)(44.92025336,69.69894844)
\curveto(44.98796168,69.63644845)(45.06869084,69.58696928)(45.16244083,69.55051096)
\curveto(45.26139915,69.51926096)(45.47754495,69.5010318)(45.81087825,69.49582346)
\lineto(45.87337824,69.43332347)
\lineto(45.87337824,69.02707352)
\lineto(45.81087825,68.96457353)
\curveto(45.25879498,69.01144852)(44.76139921,69.03488602)(44.31869093,69.03488602)
\curveto(43.4541077,69.00884436)(42.58952448,68.98540686)(41.72494125,68.96457353)
\lineto(41.66244126,69.02707352)
\lineto(41.66244126,69.43332347)
\lineto(41.72494125,69.49582346)
\curveto(42.30827451,69.50624013)(42.6728578,69.52967763)(42.81869111,69.56613595)
\curveto(42.96452443,69.60780262)(43.03744109,69.69894844)(43.03744109,69.83957342)
\curveto(43.03744109,69.95415674)(43.00098276,70.10519839)(42.9280661,70.29269836)
\lineto(41.96712872,72.67551057)
\lineto(37.44369178,72.67551057)
\lineto(36.59212938,70.67551082)
\curveto(36.44108773,70.31613586)(36.36556691,70.05311506)(36.36556691,69.88644841)
\curveto(36.36556691,69.7562401)(36.4358794,69.65988594)(36.57650438,69.59738595)
\curveto(36.71712937,69.53488596)(37.07129599,69.5010318)(37.63900425,69.49582346)
\lineto(37.70931674,69.43332347)
\lineto(37.70931674,69.02707352)
\lineto(37.64681675,68.96457353)
\curveto(36.97494183,69.00624019)(36.38640024,69.02707352)(35.88119197,69.02707352)
\curveto(35.30827537,69.02707352)(34.73015044,69.00624019)(34.14681718,68.96457353)
\closepath
\moveto(37.70931674,73.34738549)
\lineto(41.66244126,73.34738549)
\lineto(39.701504,78.05832241)
\closepath
\moveto(40.07650395,80.64425959)
\closepath
\moveto(39.89681647,68.55832358)
\closepath
}
}
{
\newrgbcolor{curcolor}{0 0 0}
\pscustom[linestyle=none,fillstyle=solid,fillcolor=curcolor]
{
\newpath
\moveto(47.8186905,76.4723851)
\lineto(47.96712798,76.37082262)
\curveto(47.90983632,75.68853103)(47.88119049,74.82134364)(47.88119049,73.76926044)
\lineto(47.88119049,71.94894816)
\curveto(47.88119049,71.28228158)(47.93066965,70.81873997)(48.02962797,70.55832333)
\curveto(48.13379463,70.2979067)(48.31087794,70.09738589)(48.56087791,69.95676091)
\curveto(48.81087788,69.82134426)(49.10775284,69.75363593)(49.4515028,69.75363593)
\curveto(49.83171108,69.75363593)(50.17806521,69.82655259)(50.49056517,69.9723859)
\curveto(50.80306513,70.12342755)(51.06348176,70.33436503)(51.27181507,70.60519833)
\curveto(51.48535671,70.87603163)(51.60775253,71.05832327)(51.63900253,71.15207326)
\curveto(51.67025252,71.24582325)(51.69108585,71.50884405)(51.70150252,71.94113566)
\lineto(51.72494002,72.81613555)
\lineto(51.72494002,73.65988545)
\curveto(51.72494002,73.86821876)(51.71452335,74.15207289)(51.69369002,74.51144784)
\curveto(51.67285669,74.8708228)(51.65462752,75.09217694)(51.63900253,75.17551026)
\curveto(51.62858586,75.25884359)(51.58952337,75.31873941)(51.52181504,75.35519774)
\curveto(51.45410672,75.3968644)(51.31869007,75.41769773)(51.11556509,75.41769773)
\lineto(50.45931517,75.42551023)
\lineto(50.38900268,75.48801022)
\lineto(50.38900268,75.82394768)
\lineto(50.45150267,75.88644767)
\curveto(51.44629422,76.00623933)(52.28483578,76.2015518)(52.96712736,76.4723851)
\lineto(53.11556484,76.37082262)
\curveto(53.05827319,75.68853103)(53.02962736,74.82134364)(53.02962736,73.76926044)
\lineto(53.02962736,72.39426061)
\curveto(53.02962736,72.31613561)(53.04264819,71.67811486)(53.06868985,70.48019834)
\curveto(53.07910652,70.03749006)(53.10514818,69.76665676)(53.14681484,69.66769844)
\curveto(53.19368984,69.57394845)(53.25618983,69.50624013)(53.33431482,69.46457347)
\curveto(53.41243981,69.42811514)(53.64421061,69.40988597)(54.02962723,69.40988597)
\lineto(54.2483772,69.40988597)
\lineto(54.3186897,69.34738598)
\lineto(54.3186897,69.03488602)
\lineto(54.2561897,68.96457353)
\curveto(53.52181479,69.00624019)(53.02962736,69.02707352)(52.77962739,69.02707352)
\curveto(52.46191909,69.02707352)(52.09473164,69.00884436)(51.67806502,68.97238603)
\lineto(51.60775253,69.03488602)
\curveto(51.63900253,69.56613595)(51.66244002,70.01665673)(51.67806502,70.38644835)
\curveto(51.34994006,70.13124005)(50.97494011,69.79009426)(50.55306516,69.36301098)
\curveto(50.39160685,69.20155267)(50.15983604,69.06613602)(49.85775275,68.95676103)
\curveto(49.55566945,68.84738604)(49.21191949,68.79269855)(48.82650287,68.79269855)
\curveto(48.24316961,68.79269855)(47.7874405,68.88384437)(47.45931554,69.06613602)
\curveto(47.13639892,69.25363599)(46.90723228,69.50884429)(46.77181563,69.83176092)
\curveto(46.63639898,70.15988588)(46.56869065,70.72238581)(46.56869065,71.51926071)
\lineto(46.57650315,72.13644814)
\lineto(46.57650315,73.65988545)
\curveto(46.57650315,73.86821876)(46.56608649,74.15207289)(46.54525315,74.51144784)
\curveto(46.52441982,74.8708228)(46.50619066,75.09217694)(46.49056566,75.17551026)
\curveto(46.480149,75.25884359)(46.4410865,75.31873941)(46.37337818,75.35519774)
\curveto(46.30566985,75.3968644)(46.1702532,75.41769773)(45.96712823,75.41769773)
\lineto(45.31087831,75.42551023)
\lineto(45.24056582,75.48801022)
\lineto(45.24056582,75.82394768)
\lineto(45.30306581,75.88644767)
\curveto(46.29785735,76.00623933)(47.13639892,76.2015518)(47.8186905,76.4723851)
\closepath
\moveto(49.68587777,76.96457254)
\closepath
\moveto(49.76400276,68.55832358)
\closepath
}
}
{
\newrgbcolor{curcolor}{0 0 0}
\pscustom[linestyle=none,fillstyle=solid,fillcolor=curcolor]
{
\newpath
\moveto(55.55306454,71.32394824)
\lineto(55.889002,71.32394824)
\lineto(55.95931449,71.25363575)
\curveto(55.96973116,70.8161358)(55.99577282,70.44113585)(56.03743948,70.12863588)
\curveto(56.1988978,69.88384425)(56.4775436,69.68332344)(56.87337688,69.52707346)
\curveto(57.26921017,69.37603181)(57.65983512,69.30051099)(58.04525174,69.30051099)
\curveto(58.597335,69.30051099)(59.03743911,69.44894847)(59.36556407,69.74582343)
\curveto(59.69889737,70.0426984)(59.86556401,70.39686502)(59.86556401,70.8083233)
\curveto(59.86556401,71.03228161)(59.80566819,71.22498992)(59.68587653,71.38644823)
\curveto(59.56608488,71.55311488)(59.38379324,71.68853153)(59.1390016,71.79269818)
\curveto(58.8994183,71.90207317)(58.46712668,72.02967732)(57.84212676,72.17551063)
\curveto(57.30566849,72.30051062)(56.93066854,72.39686477)(56.7171269,72.4645731)
\curveto(56.50358526,72.53748975)(56.29785612,72.65728141)(56.09993948,72.82394805)
\curveto(55.90202283,72.9906147)(55.75098119,73.19373967)(55.64681453,73.43332298)
\curveto(55.54264788,73.67811461)(55.49056455,73.94634375)(55.49056455,74.23801038)
\curveto(55.49056455,74.93592696)(55.76660618,75.49321856)(56.31868945,75.90988517)
\curveto(56.87598105,76.33176012)(57.5686893,76.54269759)(58.39681419,76.54269759)
\curveto(58.74577248,76.54269759)(59.12598077,76.49321843)(59.53743905,76.39426011)
\curveto(59.94889734,76.30051012)(60.2613973,76.2093643)(60.47493894,76.12082265)
\lineto(60.54525143,76.01144766)
\curveto(60.50358477,75.80311435)(60.4775431,75.25623942)(60.46712644,74.37082286)
\lineto(60.39681395,74.30051037)
\lineto(60.08431399,74.30051037)
\lineto(60.01400149,74.37082286)
\curveto(59.99316816,74.68853116)(59.96452233,74.9098853)(59.928064,75.03488528)
\curveto(59.89160568,75.1650936)(59.79785569,75.30311441)(59.64681404,75.44894773)
\curveto(59.49577239,75.59998938)(59.28223075,75.72498936)(59.00618912,75.82394768)
\curveto(58.73014749,75.92811434)(58.43848085,75.98019766)(58.13118923,75.98019766)
\curveto(57.81348093,75.98019766)(57.5452518,75.93332267)(57.32650183,75.83957268)
\curveto(57.11296019,75.74582269)(56.93848104,75.60259354)(56.80306439,75.40988523)
\curveto(56.67285607,75.22238526)(56.60775191,74.99321862)(56.60775191,74.72238532)
\curveto(56.60775191,74.52446868)(56.64681441,74.34738536)(56.7249394,74.19113538)
\curveto(56.80827272,74.0348854)(56.93327271,73.90728125)(57.09993935,73.80832293)
\curveto(57.266606,73.71457294)(57.44108514,73.64686462)(57.62337679,73.60519796)
\lineto(58.47493918,73.38644798)
\curveto(59.20410576,73.20936467)(59.72493903,73.05571886)(60.03743899,72.92551054)
\curveto(60.35514729,72.79530222)(60.59993892,72.59738558)(60.7718139,72.33176061)
\curveto(60.94368888,72.07134398)(61.02962637,71.75363568)(61.02962637,71.37863573)
\curveto(61.02962637,70.65988582)(60.73275141,70.0426984)(60.13900148,69.52707346)
\curveto(59.54525155,69.01144852)(58.78223081,68.75363605)(57.84993926,68.75363605)
\curveto(57.5218143,68.75363605)(57.11296019,68.79009438)(56.62337691,68.86301104)
\curveto(56.13900197,68.9359277)(55.72493952,69.01665686)(55.38118957,69.10519851)
\lineto(55.34212707,69.206761)
\lineto(55.42025206,69.73019843)
\curveto(55.44629372,69.89165675)(55.46191872,70.05311506)(55.46712705,70.21457337)
\curveto(55.47233539,70.38124002)(55.47754372,70.72759414)(55.48275205,71.25363575)
\closepath
}
}
{
\newrgbcolor{curcolor}{0 0 0}
\pscustom[linestyle=none,fillstyle=solid,fillcolor=curcolor]
{
\newpath
\moveto(64.89681339,68.87863604)
\lineto(63.84212602,72.37863561)
\curveto(63.64420938,73.03488553)(63.45150107,73.64426045)(63.26400109,74.20676038)
\curveto(63.08170945,74.76926031)(62.95931363,75.1260311)(62.89681364,75.27707275)
\curveto(62.83952198,75.43332273)(62.77702199,75.54530188)(62.70931366,75.61301021)
\curveto(62.64160534,75.68071853)(62.56868868,75.72759353)(62.49056369,75.75363519)
\curveto(62.4124387,75.77967685)(62.19108456,75.81353102)(61.82650127,75.85519768)
\lineto(61.77181378,75.91769767)
\lineto(61.77181378,76.23801013)
\lineto(61.84212627,76.30832262)
\curveto(62.21712622,76.27186429)(62.79785532,76.25363513)(63.58431355,76.25363513)
\curveto(64.43327178,76.25363513)(65.07129254,76.27186429)(65.49837582,76.30832262)
\lineto(65.56868831,76.23801013)
\lineto(65.56868831,75.91769767)
\lineto(65.51400082,75.85519768)
\curveto(64.94629255,75.84998935)(64.60775093,75.81873935)(64.49837594,75.76144769)
\curveto(64.39420929,75.70415603)(64.34212596,75.61561437)(64.34212596,75.49582272)
\curveto(64.34212596,75.37603107)(64.41764678,75.05571861)(64.56868843,74.53488534)
\lineto(65.29525084,72.01926065)
\curveto(65.40983416,71.6286357)(65.55566748,71.17290659)(65.73275079,70.65207332)
\curveto(65.87337577,70.99061495)(66.07129241,71.44373989)(66.32650072,72.01144815)
\lineto(67.47493807,74.58176034)
\curveto(67.75097971,75.19113526)(68.00618801,75.80571852)(68.24056298,76.42551011)
\lineto(68.62337543,76.42551011)
\lineto(69.32650035,74.53488534)
\lineto(70.22493774,72.26926062)
\lineto(70.92025015,70.55051083)
\lineto(71.50618758,72.08957314)
\curveto(71.8447292,72.96978137)(72.10775,73.70415628)(72.29524998,74.29269787)
\curveto(72.48274996,74.88123947)(72.57649995,75.25884359)(72.57649995,75.42551023)
\curveto(72.57649995,75.59217688)(72.51920829,75.70155186)(72.40462497,75.75363519)
\curveto(72.29004165,75.80571852)(71.98275002,75.83957268)(71.48275008,75.85519768)
\lineto(71.41243759,75.91769767)
\lineto(71.41243759,76.24582263)
\lineto(71.48275008,76.30832262)
\curveto(72.08691667,76.27186429)(72.59472911,76.25363513)(73.00618739,76.25363513)
\curveto(73.52702066,76.25363513)(74.01660393,76.27186429)(74.47493721,76.30832262)
\lineto(74.5374372,76.24582263)
\lineto(74.5374372,75.92551017)
\lineto(74.47493721,75.85519768)
\curveto(74.19368725,75.84478101)(74.00097894,75.81353102)(73.89681228,75.76144769)
\curveto(73.79785396,75.70936436)(73.68327064,75.58176021)(73.55306232,75.37863524)
\curveto(73.42806234,75.1807186)(73.23535403,74.77446865)(72.9749374,74.15988539)
\lineto(72.20149999,72.32394811)
\lineto(71.84993753,71.44894822)
\lineto(71.2952501,69.9958234)
\curveto(71.12337512,69.54269846)(70.99837514,69.17030267)(70.92025015,68.87863604)
\lineto(70.18587524,68.87863604)
\lineto(69.81087529,69.85519842)
\lineto(68.02962551,74.33176037)
\lineto(67.2796256,72.71457307)
\lineto(66.41243821,70.7770733)
\lineto(65.6468133,68.87863604)
\closepath
}
}
{
\newrgbcolor{curcolor}{0 0 0}
\pscustom[linestyle=none,fillstyle=solid,fillcolor=curcolor]
{
\newpath
\moveto(76.11556201,74.32394787)
\lineto(75.81087455,74.40207286)
\lineto(75.74837455,74.48019785)
\lineto(75.74837455,75.44894773)
\curveto(76.67545777,76.13644764)(77.57649933,76.4801976)(78.45149922,76.4801976)
\curveto(79.05045748,76.4801976)(79.54785325,76.36821845)(79.94368654,76.14426014)
\curveto(80.33951982,75.92030184)(80.62597812,75.63905187)(80.80306143,75.30051025)
\curveto(80.98014474,74.96717695)(81.0686864,74.576552)(81.0686864,74.12863539)
\lineto(81.0296239,72.55832309)
\lineto(81.0296239,70.23019837)
\curveto(81.0296239,69.91249008)(81.05045723,69.71978177)(81.0921239,69.65207344)
\curveto(81.13899889,69.58436512)(81.19108222,69.53749012)(81.24837388,69.51144846)
\curveto(81.30566554,69.49061513)(81.41243636,69.47238597)(81.56868634,69.45676097)
\lineto(82.01399878,69.41769847)
\lineto(82.07649877,69.34738598)
\lineto(82.07649877,69.03488602)
\lineto(82.01399878,68.97238603)
\curveto(81.63379049,69.00363602)(81.27962387,69.01926102)(80.95149891,69.01926102)
\curveto(80.63899895,69.01926102)(80.263999,69.00363602)(79.82649905,68.97238603)
\lineto(79.70931157,69.08176101)
\lineto(79.74056156,70.33957336)
\lineto(78.03743677,69.01144852)
\curveto(77.75097847,68.89165687)(77.43847851,68.83176104)(77.09993689,68.83176104)
\curveto(76.68327027,68.83176104)(76.32389532,68.90728187)(76.02181202,69.05832352)
\curveto(75.72493706,69.20936516)(75.49577042,69.42030264)(75.33431211,69.69113594)
\curveto(75.17806212,69.96196924)(75.09993713,70.2900942)(75.09993713,70.67551082)
\curveto(75.09993713,71.44113572)(75.33952044,72.03748982)(75.81868705,72.4645731)
\curveto(76.29785365,72.89686471)(77.60514516,73.26405216)(79.74056156,73.56613546)
\curveto(79.74056156,74.3369687)(79.56868658,74.8786353)(79.22493663,75.19113526)
\curveto(78.88118667,75.50884355)(78.42024922,75.6676977)(77.8421243,75.6676977)
\curveto(77.540041,75.6676977)(77.26399937,75.62342687)(77.0139994,75.53488522)
\curveto(76.76920776,75.44634356)(76.62858278,75.3734269)(76.59212445,75.31613525)
\curveto(76.55566612,75.26405192)(76.41504114,74.94634362)(76.1702495,74.36301036)
\closepath
\moveto(79.74056156,73.08957302)
\curveto(78.28222841,72.84478138)(77.37337435,72.58176058)(77.0139994,72.30051062)
\curveto(76.65462444,72.01926065)(76.47493696,71.58436487)(76.47493696,70.99582328)
\curveto(76.47493696,70.18332338)(76.87858275,69.77707343)(77.68587432,69.77707343)
\curveto(78.37858256,69.77707343)(79.06347831,70.17811505)(79.74056156,70.98019828)
\closepath
\moveto(78.41243673,76.96457254)
\closepath
\moveto(78.24837425,68.55832358)
\closepath
}
}
{
\newrgbcolor{curcolor}{0 0 0}
\pscustom[linestyle=none,fillstyle=solid,fillcolor=curcolor]
{
\newpath
\moveto(85.1155609,80.59738459)
\lineto(85.26399838,80.50363461)
\curveto(85.20670672,79.90467635)(85.17806089,78.70415566)(85.17806089,76.90207255)
\lineto(85.17806089,74.97238529)
\curveto(85.61556084,75.33176024)(86.02181079,75.68853103)(86.39681074,76.04269766)
\curveto(86.50618573,76.14165598)(86.61035238,76.21457263)(86.7093107,76.26144763)
\curveto(86.80826902,76.30832262)(86.96712317,76.35780178)(87.18587314,76.40988511)
\curveto(87.40983145,76.46196844)(87.63899809,76.4880101)(87.87337306,76.4880101)
\curveto(88.26920634,76.4880101)(88.6520188,76.40728094)(89.02181042,76.24582263)
\curveto(89.39681037,76.08436432)(89.67806034,75.89165601)(89.86556031,75.6676977)
\curveto(90.05826862,75.44894773)(90.19108111,75.18853109)(90.26399776,74.8864478)
\curveto(90.33691442,74.5843645)(90.37337275,74.21196871)(90.37337275,73.76926044)
\lineto(90.37337275,72.39426061)
\curveto(90.37337275,72.30571895)(90.38639358,71.65728153)(90.41243525,70.44894835)
\curveto(90.42806024,69.95415674)(90.48274774,69.65988594)(90.57649773,69.56613595)
\curveto(90.67545605,69.47238597)(90.99837267,69.42551097)(91.54524761,69.42551097)
\lineto(91.6077476,69.36301098)
\lineto(91.6077476,69.02707352)
\lineto(91.54524761,68.96457353)
\curveto(90.88378935,69.00624019)(90.43587274,69.02707352)(90.20149777,69.02707352)
\curveto(90.08170612,69.02707352)(89.70149783,69.00624019)(89.06087291,68.96457353)
\lineto(88.96712292,69.05051102)
\curveto(89.03483125,69.72759427)(89.06868541,70.61040666)(89.06868541,71.69894819)
\lineto(89.06868541,72.72238556)
\curveto(89.06868541,73.33696882)(89.05306041,73.7796771)(89.02181042,74.0505104)
\curveto(88.99576875,74.3213437)(88.90983126,74.56613534)(88.76399795,74.78488531)
\curveto(88.61816463,75.00884362)(88.42285216,75.1807186)(88.17806052,75.30051025)
\curveto(87.93847722,75.42551023)(87.64681059,75.48801022)(87.30306063,75.48801022)
\curveto(86.94889401,75.48801022)(86.65201904,75.43853106)(86.41243574,75.33957274)
\curveto(86.17806077,75.24582275)(85.93847746,75.08436444)(85.69368583,74.8551978)
\curveto(85.44889419,74.62603116)(85.30045671,74.43332285)(85.24837338,74.27707287)
\curveto(85.20149839,74.12603123)(85.17806089,73.83176043)(85.17806089,73.39426048)
\lineto(85.17806089,72.00363565)
\curveto(85.17806089,71.88905233)(85.18847756,71.51926071)(85.20931089,70.89426079)
\curveto(85.23014422,70.2744692)(85.25097755,69.91249008)(85.27181088,69.80832342)
\curveto(85.29785254,69.70415677)(85.33431087,69.62863595)(85.38118587,69.58176095)
\curveto(85.42806086,69.53488596)(85.48535252,69.50363596)(85.55306085,69.48801096)
\curveto(85.62076917,69.4775943)(85.90201914,69.45676097)(86.39681074,69.42551097)
\lineto(86.46712323,69.36301098)
\lineto(86.46712323,69.03488602)
\lineto(86.40462324,68.96457353)
\curveto(85.75358165,69.00624019)(85.12597756,69.02707352)(84.52181097,69.02707352)
\curveto(83.92285271,69.02707352)(83.29785279,69.00624019)(82.6468112,68.96457353)
\lineto(82.57649871,69.03488602)
\lineto(82.57649871,69.36301098)
\lineto(82.6468112,69.42551097)
\curveto(83.15201947,69.45676097)(83.43587361,69.48019846)(83.4983736,69.49582346)
\curveto(83.56608192,69.51144846)(83.62337358,69.54269846)(83.67024858,69.58957345)
\curveto(83.7223319,69.64165678)(83.75618607,69.7171776)(83.77181106,69.81613592)
\curveto(83.7926444,69.92030258)(83.81347773,70.25624004)(83.83431106,70.8239483)
\curveto(83.86035272,71.39686489)(83.87337355,71.82915651)(83.87337355,72.12082314)
\lineto(83.87337355,76.16769764)
\lineto(83.84212356,77.82394744)
\curveto(83.82649856,78.40207236)(83.80826939,78.80311398)(83.78743606,79.02707229)
\curveto(83.77181106,79.25103059)(83.74837357,79.38384308)(83.71712357,79.42550974)
\curveto(83.68587358,79.4671764)(83.63118608,79.4984264)(83.55306109,79.51925973)
\curveto(83.4749361,79.54009306)(83.15983197,79.55050972)(82.60774871,79.55050972)
\lineto(82.53743622,79.62082221)
\lineto(82.53743622,79.94894717)
\lineto(82.59993621,80.01925967)
\curveto(83.59472775,80.13384298)(84.43326932,80.32655129)(85.1155609,80.59738459)
\closepath
}
}
{
\newrgbcolor{curcolor}{0 0 0}
\pscustom[linestyle=none,fillstyle=solid,fillcolor=curcolor]
{
\newpath
\moveto(94.69368472,80.59738459)
\lineto(94.8421222,80.50363461)
\curveto(94.78483054,79.90467635)(94.75618471,78.70415566)(94.75618471,76.90207255)
\lineto(94.75618471,72.00363565)
\curveto(94.75618471,71.88905233)(94.76660138,71.51926071)(94.78743471,70.89426079)
\curveto(94.80826804,70.2744692)(94.82910137,69.91249008)(94.8499347,69.80832342)
\curveto(94.87597636,69.70415677)(94.91243469,69.62863595)(94.95930969,69.58176095)
\curveto(95.00618468,69.53488596)(95.06347634,69.50363596)(95.13118466,69.48801096)
\curveto(95.19889299,69.4775943)(95.48014295,69.45676097)(95.97493456,69.42551097)
\lineto(96.04524705,69.36301098)
\lineto(96.04524705,69.03488602)
\lineto(95.98274706,68.96457353)
\curveto(95.33170547,69.00624019)(94.70410138,69.02707352)(94.09993479,69.02707352)
\curveto(93.50097653,69.02707352)(92.87597661,69.00624019)(92.22493502,68.96457353)
\lineto(92.15462253,69.03488602)
\lineto(92.15462253,69.36301098)
\lineto(92.22493502,69.42551097)
\curveto(92.73014329,69.45676097)(93.01399743,69.48019846)(93.07649742,69.49582346)
\curveto(93.14420574,69.51144846)(93.2014974,69.54269846)(93.2483724,69.58957345)
\curveto(93.30045572,69.64165678)(93.33430989,69.7171776)(93.34993488,69.81613592)
\curveto(93.37076821,69.92030258)(93.39160155,70.25624004)(93.41243488,70.8239483)
\curveto(93.43847654,71.39686489)(93.45149737,71.82915651)(93.45149737,72.12082314)
\lineto(93.45149737,76.16769764)
\lineto(93.42024737,77.82394744)
\curveto(93.40462238,78.40207236)(93.38639321,78.80311398)(93.36555988,79.02707229)
\curveto(93.34993488,79.25103059)(93.32649739,79.38384308)(93.29524739,79.42550974)
\curveto(93.26399739,79.4671764)(93.2093099,79.4984264)(93.13118491,79.51925973)
\curveto(93.05305992,79.54009306)(92.73795579,79.55050972)(92.18587253,79.55050972)
\lineto(92.11556004,79.62082221)
\lineto(92.11556004,79.94894717)
\lineto(92.17806003,80.01925967)
\curveto(93.17285157,80.13384298)(94.01139314,80.32655129)(94.69368472,80.59738459)
\closepath
}
}
{
\newrgbcolor{curcolor}{0 0 0}
\pscustom[linestyle=none,fillstyle=solid,fillcolor=curcolor]
{
}
}
{
\newrgbcolor{curcolor}{0 0 0}
\pscustom[linestyle=none,fillstyle=solid,fillcolor=curcolor]
{
\newpath
\moveto(105.68587086,79.62082221)
\lineto(105.68587086,79.94894717)
\lineto(105.74837085,80.01925967)
\curveto(106.7431624,80.13384298)(107.58170396,80.32655129)(108.26399554,80.59738459)
\lineto(108.41243303,80.50363461)
\curveto(108.35514137,79.90467635)(108.32649554,78.70415566)(108.32649554,76.90207255)
\lineto(108.32649554,71.78488568)
\curveto(108.32649554,71.25884408)(108.33430804,70.79790664)(108.34993303,70.40207335)
\curveto(108.37076636,70.0114484)(108.40462053,69.77967759)(108.45149552,69.70676094)
\curveto(108.49837052,69.63384428)(108.57389134,69.57394845)(108.67805799,69.52707346)
\curveto(108.78743298,69.48019846)(109.09212044,69.43592764)(109.59212038,69.39426098)
\lineto(109.66243287,69.33176098)
\lineto(109.66243287,69.03488602)
\lineto(109.59212038,68.96457353)
\curveto(109.02962045,69.00103186)(108.5921205,69.01926102)(108.27962054,69.01926102)
\curveto(107.99837058,69.01926102)(107.59732896,69.00103186)(107.07649569,68.96457353)
\lineto(106.9905582,69.05051102)
\curveto(107.0061832,69.56092762)(107.0139957,69.91769841)(107.0139957,70.12082339)
\curveto(107.0139957,70.14686505)(107.01659987,70.24582337)(107.0218082,70.41769835)
\curveto(106.7353499,70.20936504)(106.4488916,69.97759424)(106.1624333,69.72238593)
\curveto(105.74055836,69.34738598)(105.46451672,69.11301101)(105.33430841,69.01926102)
\curveto(105.0947251,68.89946937)(104.75357931,68.83957354)(104.31087103,68.83957354)
\curveto(103.59212112,68.83957354)(102.98014203,69.01665686)(102.47493376,69.37082348)
\curveto(101.97493382,69.7249901)(101.61816303,70.17551088)(101.40462139,70.72238581)
\curveto(101.19107975,71.27446908)(101.08430893,71.83957317)(101.08430893,72.4176981)
\curveto(101.08430893,73.0062397)(101.19368392,73.57134379)(101.41243389,74.11301039)
\curveto(101.6363922,74.65988533)(101.95149632,75.05832278)(102.35774627,75.30832275)
\curveto(102.76920456,75.55832272)(103.214517,75.81613518)(103.69368361,76.08176015)
\curveto(104.17805855,76.35259345)(104.65722516,76.4880101)(105.13118343,76.4880101)
\curveto(105.79264168,76.4880101)(106.42284994,76.32394762)(107.0218082,75.99582266)
\lineto(106.9905582,77.73800995)
\curveto(106.98014154,78.34217654)(106.96451654,78.76144732)(106.94368321,78.99582229)
\curveto(106.92284988,79.2354056)(106.89680821,79.37603058)(106.86555822,79.41769724)
\curveto(106.83430822,79.46457223)(106.77962073,79.4984264)(106.70149574,79.51925973)
\curveto(106.62857908,79.54009306)(106.31347495,79.55050972)(105.75618335,79.55050972)
\closepath
\moveto(107.0218082,74.65988533)
\curveto(106.69889157,75.02446861)(106.33951662,75.30311441)(105.94368333,75.49582272)
\curveto(105.55305838,75.68853103)(105.15982926,75.78488519)(104.76399598,75.78488519)
\curveto(104.32649603,75.78488519)(103.91764191,75.66509354)(103.53743363,75.42551023)
\curveto(103.16243367,75.19113526)(102.89160037,74.8395728)(102.72493373,74.37082286)
\curveto(102.55826708,73.90728125)(102.47493376,73.40207298)(102.47493376,72.85519805)
\curveto(102.47493376,71.94373983)(102.7041004,71.21196908)(103.16243367,70.65988582)
\curveto(103.62597528,70.10780255)(104.20670438,69.83176092)(104.90462096,69.83176092)
\curveto(105.28482925,69.83176092)(105.6207667,69.91769841)(105.91243333,70.08957339)
\curveto(106.2093083,70.2666567)(106.45149577,70.50363584)(106.63899575,70.8005108)
\curveto(106.83170405,71.09738577)(106.94368321,71.39426073)(106.9749332,71.69113569)
\curveto(107.0061832,71.98801066)(107.0218082,72.53228142)(107.0218082,73.32394799)
\closepath
}
}
{
\newrgbcolor{curcolor}{0 0 0}
\pscustom[linestyle=none,fillstyle=solid,fillcolor=curcolor]
{
\newpath
\moveto(117.10774445,70.17551088)
\lineto(116.85774449,69.61301095)
\curveto(116.31607789,69.26405266)(115.83170294,69.03749019)(115.40461966,68.93332353)
\curveto(114.98274472,68.82915688)(114.6051406,68.77707355)(114.2718073,68.77707355)
\curveto(113.64680738,68.77707355)(113.05305745,68.89946937)(112.49055752,69.14426101)
\curveto(111.93326593,69.38905264)(111.47753682,69.80311509)(111.12337019,70.38644835)
\curveto(110.7744119,70.96978161)(110.59993276,71.67290653)(110.59993276,72.49582309)
\curveto(110.59993276,73.04269803)(110.66764108,73.53488546)(110.80305773,73.97238541)
\curveto(110.93847438,74.41509369)(111.07909936,74.74321865)(111.22493268,74.95676029)
\curveto(111.37597433,75.17030193)(111.62857846,75.40728107)(111.98274509,75.6676977)
\curveto(112.33691171,75.92811434)(112.71191166,76.13644764)(113.10774495,76.29269762)
\curveto(113.50357823,76.44894761)(113.93066151,76.5270726)(114.38899479,76.5270726)
\curveto(115.01399471,76.5270726)(115.56347381,76.37863511)(116.03743209,76.08176015)
\curveto(116.51659869,75.79009352)(116.85253615,75.41509357)(117.04524446,74.95676029)
\curveto(117.23795277,74.49842701)(117.33430693,74.01144791)(117.33430693,73.49582297)
\curveto(117.33430693,73.33436466)(117.32649443,73.17811468)(117.31086943,73.02707303)
\lineto(117.22493194,72.94113554)
\curveto(116.87076532,72.86301055)(116.39420288,72.81092722)(115.79524462,72.78488556)
\curveto(115.19628636,72.75884389)(114.80045307,72.74582306)(114.60774476,72.74582306)
\lineto(112.09993257,72.74582306)
\curveto(112.11034924,71.66769819)(112.38118254,70.87342746)(112.91243247,70.36301086)
\curveto(113.44368241,69.85259425)(114.09472399,69.59738595)(114.86555723,69.59738595)
\curveto(115.23014052,69.59738595)(115.57909881,69.65988594)(115.9124321,69.78488593)
\curveto(116.25097373,69.90988591)(116.60774452,70.08436506)(116.98274447,70.30832336)
\closepath
\moveto(112.09993257,73.37082298)
\curveto(112.19368256,73.35519799)(112.55305752,73.33696882)(113.17805744,73.31613549)
\curveto(113.80826569,73.29530216)(114.27441147,73.2848855)(114.57649477,73.2848855)
\curveto(115.30045301,73.2848855)(115.74055712,73.29790633)(115.8968071,73.32394799)
\curveto(115.90201544,73.44894798)(115.9046196,73.54530213)(115.9046196,73.61301046)
\curveto(115.9046196,74.42030202)(115.74055712,75.01926028)(115.41243216,75.40988523)
\curveto(115.0843072,75.80571852)(114.63639059,76.00363516)(114.06868233,76.00363516)
\curveto(113.44889074,76.00363516)(112.9645158,75.78228102)(112.61555751,75.33957274)
\curveto(112.27180755,74.89686446)(112.09993257,74.24061454)(112.09993257,73.37082298)
\closepath
\moveto(114.2796198,76.96457254)
\closepath
\moveto(114.18586981,68.55832358)
\closepath
}
}
{
\newrgbcolor{curcolor}{0 0 0}
\pscustom[linestyle=none,fillstyle=solid,fillcolor=curcolor]
{
\newpath
\moveto(118.65461926,71.32394824)
\lineto(118.99055672,71.32394824)
\lineto(119.06086921,71.25363575)
\curveto(119.07128588,70.8161358)(119.09732754,70.44113585)(119.1389942,70.12863588)
\curveto(119.30045252,69.88384425)(119.57909832,69.68332344)(119.9749316,69.52707346)
\curveto(120.37076489,69.37603181)(120.76138984,69.30051099)(121.14680646,69.30051099)
\curveto(121.69888972,69.30051099)(122.13899383,69.44894847)(122.46711879,69.74582343)
\curveto(122.80045209,70.0426984)(122.96711873,70.39686502)(122.96711873,70.8083233)
\curveto(122.96711873,71.03228161)(122.90722291,71.22498992)(122.78743125,71.38644823)
\curveto(122.6676396,71.55311488)(122.48534796,71.68853153)(122.24055632,71.79269818)
\curveto(122.00097302,71.90207317)(121.5686814,72.02967732)(120.94368148,72.17551063)
\curveto(120.40722321,72.30051062)(120.03222326,72.39686477)(119.81868162,72.4645731)
\curveto(119.60513998,72.53748975)(119.39941084,72.65728141)(119.2014942,72.82394805)
\curveto(119.00357755,72.9906147)(118.85253591,73.19373967)(118.74836925,73.43332298)
\curveto(118.6442026,73.67811461)(118.59211927,73.94634375)(118.59211927,74.23801038)
\curveto(118.59211927,74.93592696)(118.8681609,75.49321856)(119.42024417,75.90988517)
\curveto(119.97753577,76.33176012)(120.67024401,76.54269759)(121.49836891,76.54269759)
\curveto(121.8473272,76.54269759)(122.22753549,76.49321843)(122.63899377,76.39426011)
\curveto(123.05045205,76.30051012)(123.36295202,76.2093643)(123.57649366,76.12082265)
\lineto(123.64680615,76.01144766)
\curveto(123.60513949,75.80311435)(123.57909782,75.25623942)(123.56868116,74.37082286)
\lineto(123.49836867,74.30051037)
\lineto(123.1858687,74.30051037)
\lineto(123.11555621,74.37082286)
\curveto(123.09472288,74.68853116)(123.06607705,74.9098853)(123.02961872,75.03488528)
\curveto(122.9931604,75.1650936)(122.89941041,75.30311441)(122.74836876,75.44894773)
\curveto(122.59732711,75.59998938)(122.38378547,75.72498936)(122.10774384,75.82394768)
\curveto(121.83170221,75.92811434)(121.54003557,75.98019766)(121.23274395,75.98019766)
\curveto(120.91503565,75.98019766)(120.64680652,75.93332267)(120.42805654,75.83957268)
\curveto(120.2145149,75.74582269)(120.04003576,75.60259354)(119.90461911,75.40988523)
\curveto(119.77441079,75.22238526)(119.70930663,74.99321862)(119.70930663,74.72238532)
\curveto(119.70930663,74.52446868)(119.74836913,74.34738536)(119.82649412,74.19113538)
\curveto(119.90982744,74.0348854)(120.03482743,73.90728125)(120.20149407,73.80832293)
\curveto(120.36816072,73.71457294)(120.54263986,73.64686462)(120.72493151,73.60519796)
\lineto(121.5764939,73.38644798)
\curveto(122.30566048,73.20936467)(122.82649375,73.05571886)(123.13899371,72.92551054)
\curveto(123.456702,72.79530222)(123.70149364,72.59738558)(123.87336862,72.33176061)
\curveto(124.0452436,72.07134398)(124.13118109,71.75363568)(124.13118109,71.37863573)
\curveto(124.13118109,70.65988582)(123.83430612,70.0426984)(123.2405562,69.52707346)
\curveto(122.64680627,69.01144852)(121.88378553,68.75363605)(120.95149398,68.75363605)
\curveto(120.62336902,68.75363605)(120.2145149,68.79009438)(119.72493163,68.86301104)
\curveto(119.24055669,68.9359277)(118.82649424,69.01665686)(118.48274428,69.10519851)
\lineto(118.44368179,69.206761)
\lineto(118.52180678,69.73019843)
\curveto(118.54784844,69.89165675)(118.56347344,70.05311506)(118.56868177,70.21457337)
\curveto(118.57389011,70.38124002)(118.57909844,70.72759414)(118.58430677,71.25363575)
\closepath
}
}
{
\newrgbcolor{curcolor}{0 0 0}
\pscustom[linestyle=none,fillstyle=solid,fillcolor=curcolor]
{
}
}
{
\newrgbcolor{curcolor}{0 0 0}
\pscustom[linestyle=none,fillstyle=solid,fillcolor=curcolor]
{
\newpath
\moveto(134.02180487,68.88644854)
\curveto(133.94367988,69.11561518)(133.61815909,69.99061507)(133.04524249,71.51144821)
\lineto(130.38899282,78.0817599)
\curveto(130.08690952,78.83175981)(129.89680538,79.25623893)(129.81868039,79.35519725)
\curveto(129.7405554,79.4593639)(129.42805544,79.52967639)(128.8811805,79.56613472)
\lineto(128.81086801,79.63644721)
\lineto(128.81086801,79.99582217)
\lineto(128.8811805,80.06613466)
\curveto(129.6780554,80.024468)(130.35513865,80.00363467)(130.91243025,80.00363467)
\curveto(131.43326352,80.00363467)(132.1233676,80.024468)(132.9827425,80.06613466)
\lineto(133.04524249,80.00363467)
\lineto(133.04524249,79.57394722)
\lineto(132.9827425,79.51144723)
\curveto(132.35253424,79.51144723)(131.97493012,79.4906139)(131.84993014,79.44894724)
\curveto(131.73013848,79.41248891)(131.67024266,79.33957225)(131.67024266,79.23019726)
\curveto(131.67024266,79.16769727)(131.75357598,78.88905147)(131.92024263,78.39425987)
\curveto(132.08690927,77.90467659)(132.23534676,77.49061414)(132.36555507,77.15207252)
\lineto(134.81086727,70.97238578)
\lineto(136.84992952,76.09738515)
\curveto(136.9905545,76.44634344)(137.19628364,77.00884337)(137.46711694,77.78488494)
\curveto(137.73795024,78.56092651)(137.87336689,79.02707229)(137.87336689,79.18332227)
\curveto(137.87336689,79.31353059)(137.80826274,79.39686391)(137.67805442,79.43332224)
\curveto(137.55305443,79.46978057)(137.17284615,79.51405139)(136.53742956,79.56613472)
\lineto(136.46711707,79.62863471)
\lineto(136.46711707,80.00363467)
\lineto(136.53742956,80.06613466)
\curveto(137.44888778,80.024468)(138.02961687,80.00363467)(138.27961684,80.00363467)
\curveto(138.48274182,80.00363467)(139.00878342,80.024468)(139.85774165,80.06613466)
\lineto(139.92024164,80.00363467)
\lineto(139.92024164,79.62863471)
\lineto(139.85774165,79.56613472)
\curveto(139.4827417,79.55050972)(139.25617922,79.52186389)(139.17805423,79.48019723)
\curveto(139.09992924,79.43853057)(139.02961675,79.35780141)(138.96711676,79.23800976)
\curveto(138.9098251,79.11821811)(138.72492929,78.70675983)(138.41242933,78.00363491)
\lineto(135.97492963,72.00363565)
\curveto(135.4280547,70.66509415)(135.04003391,69.62603178)(134.81086727,68.88644854)
\closepath
}
}
{
\newrgbcolor{curcolor}{0 0 0}
\pscustom[linestyle=none,fillstyle=solid,fillcolor=curcolor]
{
\newpath
\moveto(145.97492839,70.17551088)
\lineto(145.72492843,69.61301095)
\curveto(145.18326183,69.26405266)(144.69888689,69.03749019)(144.2718036,68.93332353)
\curveto(143.84992866,68.82915688)(143.47232454,68.77707355)(143.13899124,68.77707355)
\curveto(142.51399132,68.77707355)(141.92024139,68.89946937)(141.35774146,69.14426101)
\curveto(140.80044987,69.38905264)(140.34472076,69.80311509)(139.99055413,70.38644835)
\curveto(139.64159584,70.96978161)(139.4671167,71.67290653)(139.4671167,72.49582309)
\curveto(139.4671167,73.04269803)(139.53482502,73.53488546)(139.67024167,73.97238541)
\curveto(139.80565832,74.41509369)(139.9462833,74.74321865)(140.09211662,74.95676029)
\curveto(140.24315827,75.17030193)(140.4957624,75.40728107)(140.84992903,75.6676977)
\curveto(141.20409565,75.92811434)(141.5790956,76.13644764)(141.97492889,76.29269762)
\curveto(142.37076217,76.44894761)(142.79784545,76.5270726)(143.25617873,76.5270726)
\curveto(143.88117865,76.5270726)(144.43065775,76.37863511)(144.90461603,76.08176015)
\curveto(145.38378263,75.79009352)(145.71972009,75.41509357)(145.9124284,74.95676029)
\curveto(146.10513671,74.49842701)(146.20149087,74.01144791)(146.20149087,73.49582297)
\curveto(146.20149087,73.33436466)(146.19367837,73.17811468)(146.17805337,73.02707303)
\lineto(146.09211588,72.94113554)
\curveto(145.73794926,72.86301055)(145.26138682,72.81092722)(144.66242856,72.78488556)
\curveto(144.0634703,72.75884389)(143.66763701,72.74582306)(143.4749287,72.74582306)
\lineto(140.96711651,72.74582306)
\curveto(140.97753318,71.66769819)(141.24836648,70.87342746)(141.77961641,70.36301086)
\curveto(142.31086635,69.85259425)(142.96190793,69.59738595)(143.73274117,69.59738595)
\curveto(144.09732446,69.59738595)(144.44628275,69.65988594)(144.77961604,69.78488593)
\curveto(145.11815767,69.90988591)(145.47492846,70.08436506)(145.84992841,70.30832336)
\closepath
\moveto(140.96711651,73.37082298)
\curveto(141.0608665,73.35519799)(141.42024146,73.33696882)(142.04524138,73.31613549)
\curveto(142.67544963,73.29530216)(143.14159541,73.2848855)(143.44367871,73.2848855)
\curveto(144.16763695,73.2848855)(144.60774106,73.29790633)(144.76399104,73.32394799)
\curveto(144.76919938,73.44894798)(144.77180354,73.54530213)(144.77180354,73.61301046)
\curveto(144.77180354,74.42030202)(144.60774106,75.01926028)(144.2796161,75.40988523)
\curveto(143.95149114,75.80571852)(143.50357453,76.00363516)(142.93586627,76.00363516)
\curveto(142.31607468,76.00363516)(141.83169974,75.78228102)(141.48274145,75.33957274)
\curveto(141.13899149,74.89686446)(140.96711651,74.24061454)(140.96711651,73.37082298)
\closepath
\moveto(143.14680374,76.96457254)
\closepath
\moveto(143.05305376,68.55832358)
\closepath
}
}
{
\newrgbcolor{curcolor}{0 0 0}
\pscustom[linestyle=none,fillstyle=solid,fillcolor=curcolor]
{
\newpath
\moveto(149.73274043,76.4723851)
\lineto(149.88117791,76.37082262)
\curveto(149.84992792,76.06873932)(149.82649042,75.53228105)(149.81086542,74.76144781)
\lineto(150.39680285,75.50363522)
\curveto(150.58951116,75.74842686)(150.75878197,75.93592684)(150.90461529,76.06613515)
\curveto(151.05565693,76.2015518)(151.23013608,76.30571846)(151.42805272,76.37863511)
\curveto(151.62596936,76.45155177)(151.82909434,76.4880101)(152.03742765,76.4880101)
\curveto(152.26659429,76.4880101)(152.48274009,76.44113511)(152.68586507,76.34738512)
\lineto(152.74055256,76.23801013)
\curveto(152.65721924,75.54530188)(152.61034424,74.93592696)(152.59992758,74.40988536)
\lineto(152.24836512,74.40988536)
\curveto(152.04003181,74.88905196)(151.70409436,75.12863527)(151.24055275,75.12863527)
\curveto(150.91763612,75.12863527)(150.63638615,75.02446861)(150.39680285,74.81613531)
\curveto(150.15721955,74.61301033)(149.99576123,74.35519786)(149.91242791,74.0426979)
\curveto(149.83430292,73.73540627)(149.79524042,73.34478132)(149.79524042,72.87082305)
\lineto(149.79524042,72.00363565)
\curveto(149.79524042,71.84738567)(149.80565709,71.46457322)(149.82649042,70.8551983)
\curveto(149.84732375,70.24582337)(149.86815708,69.89426091)(149.88899041,69.80051093)
\curveto(149.91503208,69.70676094)(149.9514904,69.63644845)(149.9983654,69.58957345)
\curveto(150.05044873,69.54790679)(150.11555288,69.51926096)(150.19367787,69.50363596)
\curveto(150.2770112,69.48801096)(150.64940699,69.4619693)(151.31086524,69.42551097)
\lineto(151.38117773,69.36301098)
\lineto(151.38117773,69.03488602)
\lineto(151.31086524,68.96457353)
\curveto(150.61815699,69.00624019)(149.89419874,69.02707352)(149.1389905,69.02707352)
\curveto(148.54003225,69.02707352)(147.91503232,69.00624019)(147.26399074,68.96457353)
\lineto(147.19367824,69.03488602)
\lineto(147.19367824,69.36301098)
\lineto(147.26399074,69.42551097)
\curveto(147.76919901,69.45676097)(148.05305314,69.48019846)(148.11555313,69.49582346)
\curveto(148.18326146,69.51144846)(148.24055312,69.54269846)(148.28742811,69.58957345)
\curveto(148.33951144,69.64165678)(148.3733656,69.7171776)(148.3889906,69.81613592)
\curveto(148.40982393,69.92030258)(148.43065726,70.25624004)(148.45149059,70.8239483)
\curveto(148.47753225,71.39686489)(148.49055308,71.82915651)(148.49055308,72.12082314)
\lineto(148.49055308,73.65988545)
\curveto(148.49055308,73.86821876)(148.48013642,74.15207289)(148.45930309,74.51144784)
\curveto(148.43846976,74.8708228)(148.42024059,75.09217694)(148.4046156,75.17551026)
\curveto(148.39419893,75.25884359)(148.35513643,75.31873941)(148.28742811,75.35519774)
\curveto(148.21971978,75.3968644)(148.08430313,75.41769773)(147.88117816,75.41769773)
\lineto(147.22492824,75.42551023)
\lineto(147.15461575,75.48801022)
\lineto(147.15461575,75.82394768)
\lineto(147.21711574,75.88644767)
\curveto(148.21190729,76.00623933)(149.05044885,76.2015518)(149.73274043,76.4723851)
\closepath
}
}
{
\newrgbcolor{curcolor}{0 0 0}
\pscustom[linestyle=none,fillstyle=solid,fillcolor=curcolor]
{
\newpath
\moveto(153.84211492,71.32394824)
\lineto(154.17805238,71.32394824)
\lineto(154.24836487,71.25363575)
\curveto(154.25878154,70.8161358)(154.2848232,70.44113585)(154.32648986,70.12863588)
\curveto(154.48794818,69.88384425)(154.76659398,69.68332344)(155.16242726,69.52707346)
\curveto(155.55826055,69.37603181)(155.9488855,69.30051099)(156.33430212,69.30051099)
\curveto(156.88638538,69.30051099)(157.3264895,69.44894847)(157.65461445,69.74582343)
\curveto(157.98794775,70.0426984)(158.15461439,70.39686502)(158.15461439,70.8083233)
\curveto(158.15461439,71.03228161)(158.09471857,71.22498992)(157.97492692,71.38644823)
\curveto(157.85513526,71.55311488)(157.67284362,71.68853153)(157.42805198,71.79269818)
\curveto(157.18846868,71.90207317)(156.75617707,72.02967732)(156.13117714,72.17551063)
\curveto(155.59471888,72.30051062)(155.21971892,72.39686477)(155.00617728,72.4645731)
\curveto(154.79263564,72.53748975)(154.5869065,72.65728141)(154.38898986,72.82394805)
\curveto(154.19107322,72.9906147)(154.04003157,73.19373967)(153.93586491,73.43332298)
\curveto(153.83169826,73.67811461)(153.77961493,73.94634375)(153.77961493,74.23801038)
\curveto(153.77961493,74.93592696)(154.05565657,75.49321856)(154.60773983,75.90988517)
\curveto(155.16503143,76.33176012)(155.85773968,76.54269759)(156.68586457,76.54269759)
\curveto(157.03482286,76.54269759)(157.41503115,76.49321843)(157.82648943,76.39426011)
\curveto(158.23794772,76.30051012)(158.55044768,76.2093643)(158.76398932,76.12082265)
\lineto(158.83430181,76.01144766)
\curveto(158.79263515,75.80311435)(158.76659348,75.25623942)(158.75617682,74.37082286)
\lineto(158.68586433,74.30051037)
\lineto(158.37336437,74.30051037)
\lineto(158.30305187,74.37082286)
\curveto(158.28221854,74.68853116)(158.25357271,74.9098853)(158.21711439,75.03488528)
\curveto(158.18065606,75.1650936)(158.08690607,75.30311441)(157.93586442,75.44894773)
\curveto(157.78482277,75.59998938)(157.57128113,75.72498936)(157.2952395,75.82394768)
\curveto(157.01919787,75.92811434)(156.72753124,75.98019766)(156.42023961,75.98019766)
\curveto(156.10253131,75.98019766)(155.83430218,75.93332267)(155.61555221,75.83957268)
\curveto(155.40201057,75.74582269)(155.22753142,75.60259354)(155.09211477,75.40988523)
\curveto(154.96190645,75.22238526)(154.89680229,74.99321862)(154.89680229,74.72238532)
\curveto(154.89680229,74.52446868)(154.93586479,74.34738536)(155.01398978,74.19113538)
\curveto(155.0973231,74.0348854)(155.22232309,73.90728125)(155.38898973,73.80832293)
\curveto(155.55565638,73.71457294)(155.73013553,73.64686462)(155.91242717,73.60519796)
\lineto(156.76398956,73.38644798)
\curveto(157.49315614,73.20936467)(158.01398941,73.05571886)(158.32648937,72.92551054)
\curveto(158.64419767,72.79530222)(158.8889893,72.59738558)(159.06086428,72.33176061)
\curveto(159.23273926,72.07134398)(159.31867675,71.75363568)(159.31867675,71.37863573)
\curveto(159.31867675,70.65988582)(159.02180179,70.0426984)(158.42805186,69.52707346)
\curveto(157.83430193,69.01144852)(157.07128119,68.75363605)(156.13898964,68.75363605)
\curveto(155.81086468,68.75363605)(155.40201057,68.79009438)(154.91242729,68.86301104)
\curveto(154.42805235,68.9359277)(154.0139899,69.01665686)(153.67023995,69.10519851)
\lineto(153.63117745,69.206761)
\lineto(153.70930244,69.73019843)
\curveto(153.7353441,69.89165675)(153.7509691,70.05311506)(153.75617744,70.21457337)
\curveto(153.76138577,70.38124002)(153.7665941,70.72759414)(153.77180243,71.25363575)
\closepath
}
}
{
\newrgbcolor{curcolor}{0 0 0}
\pscustom[linestyle=none,fillstyle=solid,fillcolor=curcolor]
{
\newpath
\moveto(162.53742635,76.4723851)
\lineto(162.68586383,76.37082262)
\curveto(162.62857217,75.68853103)(162.59992634,74.82134364)(162.59992634,73.76926044)
\lineto(162.59992634,71.94894816)
\curveto(162.59992634,71.28228158)(162.64940551,70.81873997)(162.74836383,70.55832333)
\curveto(162.85253048,70.2979067)(163.02961379,70.09738589)(163.27961376,69.95676091)
\curveto(163.52961373,69.82134426)(163.82648869,69.75363593)(164.17023865,69.75363593)
\curveto(164.55044694,69.75363593)(164.89680106,69.82655259)(165.20930102,69.9723859)
\curveto(165.52180098,70.12342755)(165.78221762,70.33436503)(165.99055093,70.60519833)
\curveto(166.20409257,70.87603163)(166.32648839,71.05832327)(166.35773838,71.15207326)
\curveto(166.38898838,71.24582325)(166.40982171,71.50884405)(166.42023837,71.94113566)
\lineto(166.44367587,72.81613555)
\lineto(166.44367587,73.65988545)
\curveto(166.44367587,73.86821876)(166.43325921,74.15207289)(166.41242587,74.51144784)
\curveto(166.39159254,74.8708228)(166.37336338,75.09217694)(166.35773838,75.17551026)
\curveto(166.34732172,75.25884359)(166.30825922,75.31873941)(166.2405509,75.35519774)
\curveto(166.17284257,75.3968644)(166.03742592,75.41769773)(165.83430095,75.41769773)
\lineto(165.17805103,75.42551023)
\lineto(165.10773854,75.48801022)
\lineto(165.10773854,75.82394768)
\lineto(165.17023853,75.88644767)
\curveto(166.16503007,76.00623933)(167.00357164,76.2015518)(167.68586322,76.4723851)
\lineto(167.8343007,76.37082262)
\curveto(167.77700904,75.68853103)(167.74836321,74.82134364)(167.74836321,73.76926044)
\lineto(167.74836321,72.39426061)
\curveto(167.74836321,72.31613561)(167.76138404,71.67811486)(167.78742571,70.48019834)
\curveto(167.79784237,70.03749006)(167.82388403,69.76665676)(167.8655507,69.66769844)
\curveto(167.91242569,69.57394845)(167.97492568,69.50624013)(168.05305067,69.46457347)
\curveto(168.13117566,69.42811514)(168.36294647,69.40988597)(168.74836309,69.40988597)
\lineto(168.96711306,69.40988597)
\lineto(169.03742555,69.34738598)
\lineto(169.03742555,69.03488602)
\lineto(168.97492556,68.96457353)
\curveto(168.24055065,69.00624019)(167.74836321,69.02707352)(167.49836324,69.02707352)
\curveto(167.18065495,69.02707352)(166.81346749,69.00884436)(166.39680088,68.97238603)
\lineto(166.32648839,69.03488602)
\curveto(166.35773838,69.56613595)(166.38117588,70.01665673)(166.39680088,70.38644835)
\curveto(166.06867592,70.13124005)(165.69367596,69.79009426)(165.27180102,69.36301098)
\curveto(165.1103427,69.20155267)(164.8785719,69.06613602)(164.5764886,68.95676103)
\curveto(164.27440531,68.84738604)(163.93065535,68.79269855)(163.54523873,68.79269855)
\curveto(162.96190547,68.79269855)(162.50617636,68.88384437)(162.1780514,69.06613602)
\curveto(161.85513477,69.25363599)(161.62596813,69.50884429)(161.49055148,69.83176092)
\curveto(161.35513483,70.15988588)(161.28742651,70.72238581)(161.28742651,71.51926071)
\lineto(161.29523901,72.13644814)
\lineto(161.29523901,73.65988545)
\curveto(161.29523901,73.86821876)(161.28482234,74.15207289)(161.26398901,74.51144784)
\curveto(161.24315568,74.8708228)(161.22492651,75.09217694)(161.20930152,75.17551026)
\curveto(161.19888485,75.25884359)(161.15982236,75.31873941)(161.09211403,75.35519774)
\curveto(161.02440571,75.3968644)(160.88898906,75.41769773)(160.68586408,75.41769773)
\lineto(160.02961416,75.42551023)
\lineto(159.95930167,75.48801022)
\lineto(159.95930167,75.82394768)
\lineto(160.02180166,75.88644767)
\curveto(161.01659321,76.00623933)(161.85513477,76.2015518)(162.53742635,76.4723851)
\closepath
\moveto(164.40461362,76.96457254)
\closepath
\moveto(164.48273861,68.55832358)
\closepath
}
}
{
\newrgbcolor{curcolor}{0 0 0}
\pscustom[linestyle=none,fillstyle=solid,fillcolor=curcolor]
{
\newpath
\moveto(176.26398716,69.86301092)
\lineto(175.9514872,69.33957348)
\curveto(175.27961228,68.96978186)(174.52961237,68.78488605)(173.70148748,68.78488605)
\curveto(172.55565428,68.78488605)(171.67023773,69.12603184)(171.0452378,69.80832342)
\curveto(170.42023788,70.49061501)(170.10773792,71.37082323)(170.10773792,72.4489481)
\curveto(170.10773792,72.93853137)(170.15982125,73.37342715)(170.2639879,73.75363544)
\curveto(170.37336289,74.13905206)(170.50877954,74.45676035)(170.67023785,74.70676032)
\curveto(170.8369045,74.95676029)(171.01919614,75.15467693)(171.21711278,75.30051025)
\curveto(171.41502942,75.44634356)(171.74836272,75.64946854)(172.21711266,75.90988517)
\curveto(172.69107093,76.17030181)(173.04784172,76.33957262)(173.28742503,76.41769761)
\curveto(173.52700833,76.50103093)(173.85773746,76.54269759)(174.2796124,76.54269759)
\curveto(175.08690397,76.54269759)(175.74836222,76.38905178)(176.26398716,76.08176015)
\curveto(176.16502884,75.49321856)(176.08950801,74.83696864)(176.03742469,74.11301039)
\lineto(175.9671122,74.0505104)
\lineto(175.64679974,74.0505104)
\lineto(175.57648724,74.12082289)
\curveto(175.55565391,74.55832284)(175.52700808,74.85259364)(175.49054976,75.00363528)
\curveto(175.45409143,75.15467693)(175.24575812,75.31613525)(174.86554983,75.48801022)
\curveto(174.49054988,75.6598852)(174.08950826,75.74582269)(173.66242498,75.74582269)
\curveto(173.2353417,75.74582269)(172.85513341,75.65467687)(172.52180012,75.47238523)
\curveto(172.18846683,75.29530191)(171.93586269,75.00103112)(171.76398771,74.58957283)
\curveto(171.59732107,74.18332288)(171.51398775,73.69634378)(171.51398775,73.12863551)
\curveto(171.51398775,72.64946891)(171.58169607,72.18071897)(171.71711272,71.72238569)
\curveto(171.85252937,71.26926074)(172.02961268,70.89165662)(172.24836266,70.58957333)
\curveto(172.47232096,70.29269836)(172.76659176,70.05311506)(173.13117505,69.87082342)
\curveto(173.49575833,69.68853177)(173.90200828,69.59738595)(174.3499249,69.59738595)
\curveto(174.63117486,69.59738595)(174.90461233,69.63384428)(175.17023729,69.70676094)
\curveto(175.44107059,69.77967759)(175.75096639,69.89946925)(176.09992468,70.06613589)
\closepath
}
}
{
\newrgbcolor{curcolor}{0 0 0}
\pscustom[linestyle=none,fillstyle=solid,fillcolor=curcolor]
{
\newpath
\moveto(179.24054929,80.59738459)
\lineto(179.38898677,80.50363461)
\curveto(179.33169512,79.90467635)(179.30304929,78.70415566)(179.30304929,76.90207255)
\lineto(179.30304929,74.97238529)
\curveto(179.74054923,75.33176024)(180.14679918,75.68853103)(180.52179914,76.04269766)
\curveto(180.63117412,76.14165598)(180.73534078,76.21457263)(180.8342991,76.26144763)
\curveto(180.93325742,76.30832262)(181.09211156,76.35780178)(181.31086154,76.40988511)
\curveto(181.53481984,76.46196844)(181.76398648,76.4880101)(181.99836145,76.4880101)
\curveto(182.39419474,76.4880101)(182.77700719,76.40728094)(183.14679881,76.24582263)
\curveto(183.52179877,76.08436432)(183.80304873,75.89165601)(183.99054871,75.6676977)
\curveto(184.18325702,75.44894773)(184.3160695,75.18853109)(184.38898616,74.8864478)
\curveto(184.46190282,74.5843645)(184.49836114,74.21196871)(184.49836114,73.76926044)
\lineto(184.49836114,72.39426061)
\curveto(184.49836114,72.30571895)(184.51138198,71.65728153)(184.53742364,70.44894835)
\curveto(184.55304864,69.95415674)(184.60773613,69.65988594)(184.70148612,69.56613595)
\curveto(184.80044444,69.47238597)(185.12336107,69.42551097)(185.670236,69.42551097)
\lineto(185.73273599,69.36301098)
\lineto(185.73273599,69.02707352)
\lineto(185.670236,68.96457353)
\curveto(185.00877775,69.00624019)(184.56086114,69.02707352)(184.32648617,69.02707352)
\curveto(184.20669451,69.02707352)(183.82648623,69.00624019)(183.18586131,68.96457353)
\lineto(183.09211132,69.05051102)
\curveto(183.15981964,69.72759427)(183.19367381,70.61040666)(183.19367381,71.69894819)
\lineto(183.19367381,72.72238556)
\curveto(183.19367381,73.33696882)(183.17804881,73.7796771)(183.14679881,74.0505104)
\curveto(183.12075715,74.3213437)(183.03481966,74.56613534)(182.88898634,74.78488531)
\curveto(182.74315303,75.00884362)(182.54784055,75.1807186)(182.30304892,75.30051025)
\curveto(182.06346561,75.42551023)(181.77179898,75.48801022)(181.42804902,75.48801022)
\curveto(181.0738824,75.48801022)(180.77700744,75.43853106)(180.53742413,75.33957274)
\curveto(180.30304916,75.24582275)(180.06346586,75.08436444)(179.81867422,74.8551978)
\curveto(179.57388259,74.62603116)(179.4254451,74.43332285)(179.37336178,74.27707287)
\curveto(179.32648678,74.12603123)(179.30304929,73.83176043)(179.30304929,73.39426048)
\lineto(179.30304929,72.00363565)
\curveto(179.30304929,71.88905233)(179.31346595,71.51926071)(179.33429928,70.89426079)
\curveto(179.35513261,70.2744692)(179.37596594,69.91249008)(179.39679927,69.80832342)
\curveto(179.42284094,69.70415677)(179.45929927,69.62863595)(179.50617426,69.58176095)
\curveto(179.55304925,69.53488596)(179.61034091,69.50363596)(179.67804924,69.48801096)
\curveto(179.74575756,69.4775943)(180.02700753,69.45676097)(180.52179914,69.42551097)
\lineto(180.59211163,69.36301098)
\lineto(180.59211163,69.03488602)
\lineto(180.52961163,68.96457353)
\curveto(179.87857005,69.00624019)(179.25096596,69.02707352)(178.64679937,69.02707352)
\curveto(178.04784111,69.02707352)(177.42284118,69.00624019)(176.7717996,68.96457353)
\lineto(176.70148711,69.03488602)
\lineto(176.70148711,69.36301098)
\lineto(176.7717996,69.42551097)
\curveto(177.27700787,69.45676097)(177.560862,69.48019846)(177.62336199,69.49582346)
\curveto(177.69107032,69.51144846)(177.74836198,69.54269846)(177.79523697,69.58957345)
\curveto(177.8473203,69.64165678)(177.88117446,69.7171776)(177.89679946,69.81613592)
\curveto(177.91763279,69.92030258)(177.93846612,70.25624004)(177.95929945,70.8239483)
\curveto(177.98534111,71.39686489)(177.99836195,71.82915651)(177.99836195,72.12082314)
\lineto(177.99836195,76.16769764)
\lineto(177.96711195,77.82394744)
\curveto(177.95148695,78.40207236)(177.93325779,78.80311398)(177.91242446,79.02707229)
\curveto(177.89679946,79.25103059)(177.87336196,79.38384308)(177.84211197,79.42550974)
\curveto(177.81086197,79.4671764)(177.75617448,79.4984264)(177.67804949,79.51925973)
\curveto(177.5999245,79.54009306)(177.28482037,79.55050972)(176.7327371,79.55050972)
\lineto(176.66242461,79.62082221)
\lineto(176.66242461,79.94894717)
\lineto(176.7249246,80.01925967)
\curveto(177.71971615,80.13384298)(178.55825771,80.32655129)(179.24054929,80.59738459)
\closepath
}
}
{
\newrgbcolor{curcolor}{0 0 0}
\pscustom[linestyle=none,fillstyle=solid,fillcolor=curcolor]
{
\newpath
\moveto(186.68586087,71.32394824)
\lineto(187.02179833,71.32394824)
\lineto(187.09211082,71.25363575)
\curveto(187.10252749,70.8161358)(187.12856915,70.44113585)(187.17023582,70.12863588)
\curveto(187.33169413,69.88384425)(187.61033993,69.68332344)(188.00617321,69.52707346)
\curveto(188.4020065,69.37603181)(188.79263145,69.30051099)(189.17804807,69.30051099)
\curveto(189.73013133,69.30051099)(190.17023545,69.44894847)(190.4983604,69.74582343)
\curveto(190.8316937,70.0426984)(190.99836034,70.39686502)(190.99836034,70.8083233)
\curveto(190.99836034,71.03228161)(190.93846452,71.22498992)(190.81867287,71.38644823)
\curveto(190.69888121,71.55311488)(190.51658957,71.68853153)(190.27179793,71.79269818)
\curveto(190.03221463,71.90207317)(189.59992302,72.02967732)(188.97492309,72.17551063)
\curveto(188.43846483,72.30051062)(188.06346487,72.39686477)(187.84992323,72.4645731)
\curveto(187.63638159,72.53748975)(187.43065245,72.65728141)(187.23273581,72.82394805)
\curveto(187.03481917,72.9906147)(186.88377752,73.19373967)(186.77961086,73.43332298)
\curveto(186.67544421,73.67811461)(186.62336088,73.94634375)(186.62336088,74.23801038)
\curveto(186.62336088,74.93592696)(186.89940252,75.49321856)(187.45148578,75.90988517)
\curveto(188.00877738,76.33176012)(188.70148563,76.54269759)(189.52961052,76.54269759)
\curveto(189.87856881,76.54269759)(190.2587771,76.49321843)(190.67023538,76.39426011)
\curveto(191.08169367,76.30051012)(191.39419363,76.2093643)(191.60773527,76.12082265)
\lineto(191.67804776,76.01144766)
\curveto(191.6363811,75.80311435)(191.61033943,75.25623942)(191.59992277,74.37082286)
\lineto(191.52961028,74.30051037)
\lineto(191.21711032,74.30051037)
\lineto(191.14679782,74.37082286)
\curveto(191.12596449,74.68853116)(191.09731866,74.9098853)(191.06086034,75.03488528)
\curveto(191.02440201,75.1650936)(190.93065202,75.30311441)(190.77961037,75.44894773)
\curveto(190.62856872,75.59998938)(190.41502708,75.72498936)(190.13898545,75.82394768)
\curveto(189.86294382,75.92811434)(189.57127719,75.98019766)(189.26398556,75.98019766)
\curveto(188.94627726,75.98019766)(188.67804813,75.93332267)(188.45929816,75.83957268)
\curveto(188.24575652,75.74582269)(188.07127737,75.60259354)(187.93586072,75.40988523)
\curveto(187.8056524,75.22238526)(187.74054824,74.99321862)(187.74054824,74.72238532)
\curveto(187.74054824,74.52446868)(187.77961074,74.34738536)(187.85773573,74.19113538)
\curveto(187.94106905,74.0348854)(188.06606904,73.90728125)(188.23273568,73.80832293)
\curveto(188.39940233,73.71457294)(188.57388148,73.64686462)(188.75617312,73.60519796)
\lineto(189.60773551,73.38644798)
\curveto(190.33690209,73.20936467)(190.85773536,73.05571886)(191.17023532,72.92551054)
\curveto(191.48794362,72.79530222)(191.73273525,72.59738558)(191.90461023,72.33176061)
\curveto(192.07648521,72.07134398)(192.1624227,71.75363568)(192.1624227,71.37863573)
\curveto(192.1624227,70.65988582)(191.86554774,70.0426984)(191.27179781,69.52707346)
\curveto(190.67804788,69.01144852)(189.91502714,68.75363605)(188.98273559,68.75363605)
\curveto(188.65461063,68.75363605)(188.24575652,68.79009438)(187.75617324,68.86301104)
\curveto(187.2717983,68.9359277)(186.85773585,69.01665686)(186.5139859,69.10519851)
\lineto(186.4749234,69.206761)
\lineto(186.55304839,69.73019843)
\curveto(186.57909005,69.89165675)(186.59471505,70.05311506)(186.59992339,70.21457337)
\curveto(186.60513172,70.38124002)(186.61034005,70.72759414)(186.61554838,71.25363575)
\closepath
}
}
{
\newrgbcolor{curcolor}{0 0 0}
\pscustom[linestyle=none,fillstyle=solid,fillcolor=curcolor]
{
\newpath
\moveto(195.56867228,76.56613509)
\lineto(195.71710976,76.4723851)
\curveto(195.66502643,75.87863518)(195.6363806,75.35259357)(195.63117227,74.8942603)
\curveto(196.33950552,75.54009355)(196.83169296,75.96717683)(197.10773459,76.17551014)
\curveto(197.38898456,76.38384345)(197.87075533,76.4880101)(198.55304691,76.4880101)
\curveto(198.98013019,76.4880101)(199.36815098,76.42811427)(199.71710927,76.30832262)
\curveto(200.06606756,76.1937393)(200.38898419,76.00103099)(200.68585915,75.73019769)
\curveto(200.98273411,75.46457273)(201.21710908,75.12863527)(201.38898406,74.72238532)
\curveto(201.56085904,74.3213437)(201.64679653,73.85519793)(201.64679653,73.32394799)
\curveto(201.64679653,72.94894804)(201.5869007,72.56873975)(201.46710905,72.18332313)
\curveto(201.3473174,71.79790651)(201.18846325,71.44113572)(200.99054661,71.11301076)
\curveto(200.7978383,70.7848858)(200.64158832,70.56613583)(200.52179667,70.45676084)
\curveto(200.40721335,70.35259419)(200.16763005,70.19374004)(199.80304676,69.9801984)
\curveto(199.4749218,69.78228176)(199.149401,69.56874012)(198.82648438,69.33957348)
\curveto(198.60252607,69.17811517)(198.44627609,69.07394851)(198.35773444,69.02707352)
\curveto(198.26919278,68.98019853)(198.1207553,68.93332353)(197.91242199,68.88644854)
\curveto(197.70408868,68.83957354)(197.48273454,68.81613605)(197.24835957,68.81613605)
\curveto(196.70669297,68.81613605)(196.16763054,68.95415686)(195.63117227,69.2301985)
\lineto(195.63117227,67.5661362)
\curveto(195.63117227,67.45155288)(195.64158894,67.08436543)(195.66242227,66.46457384)
\curveto(195.6832556,65.83957391)(195.70408893,65.47499063)(195.72492226,65.37082397)
\curveto(195.75096392,65.26665732)(195.78742225,65.19113649)(195.83429725,65.1442615)
\curveto(195.88117224,65.09738651)(195.9384639,65.06874068)(196.00617223,65.05832401)
\curveto(196.07388055,65.04269901)(196.35513052,65.01926151)(196.84992212,64.98801152)
\lineto(196.92023461,64.92551153)
\lineto(196.92023461,64.59738657)
\lineto(196.85773462,64.52707408)
\curveto(196.20669303,64.56874074)(195.58169311,64.58957407)(194.98273485,64.58957407)
\curveto(194.38377659,64.58957407)(193.75877667,64.56874074)(193.10773508,64.52707408)
\lineto(193.03742259,64.59738657)
\lineto(193.03742259,64.92551153)
\lineto(193.10773508,64.98801152)
\curveto(193.60773502,65.01926151)(193.89158915,65.04269901)(193.95929748,65.05832401)
\curveto(194.0270058,65.07394901)(194.08429746,65.10780317)(194.13117246,65.1598865)
\curveto(194.18325578,65.20676149)(194.21710995,65.28488648)(194.23273494,65.39426147)
\curveto(194.25356828,65.50363645)(194.27440161,65.85259475)(194.29523494,66.44113634)
\curveto(194.3212766,67.02967793)(194.33429743,67.47238621)(194.33429743,67.76926118)
\lineto(194.33429743,73.75363544)
\curveto(194.33429743,74.02446874)(194.3212766,74.3369687)(194.29523494,74.69113532)
\curveto(194.27440161,75.05051028)(194.25617244,75.25884359)(194.24054744,75.31613525)
\curveto(194.22492245,75.3734269)(194.18585995,75.4203019)(194.12335996,75.45676023)
\curveto(194.0660683,75.49321856)(193.93325581,75.51144772)(193.72492251,75.51144772)
\lineto(193.06867259,75.51926022)
\lineto(192.9983601,75.58176021)
\lineto(192.9983601,75.91769767)
\lineto(193.06086009,75.98019766)
\curveto(194.05565163,76.09998932)(194.89158903,76.29530179)(195.56867228,76.56613509)
\closepath
\moveto(195.63117227,70.57394833)
\curveto(195.87075558,70.32915669)(196.16763054,70.11561505)(196.52179716,69.93332341)
\curveto(196.87596379,69.7562401)(197.26658874,69.66769844)(197.69367202,69.66769844)
\curveto(198.19888029,69.66769844)(198.64419273,69.79530259)(199.02960935,70.05051089)
\curveto(199.4202343,70.3057192)(199.7223176,70.67811498)(199.93585924,71.16769826)
\curveto(200.14940088,71.66248986)(200.2561717,72.20936479)(200.2561717,72.80832305)
\curveto(200.2561717,73.29269799)(200.15981755,73.74321877)(199.96710924,74.15988539)
\curveto(199.77960926,74.576552)(199.48533846,74.90467696)(199.08429685,75.14426027)
\curveto(198.68846356,75.38384357)(198.26658861,75.50363522)(197.818672,75.50363522)
\curveto(197.54263037,75.50363522)(197.27440124,75.45415606)(197.0139846,75.35519774)
\curveto(196.75356797,75.26144775)(196.50877633,75.12082277)(196.27960969,74.93332279)
\curveto(196.05565139,74.74582282)(195.89419307,74.56613534)(195.79523475,74.39426036)
\curveto(195.69627643,74.22759371)(195.6441931,74.08696873)(195.63898477,73.97238541)
\curveto(195.63377644,73.86301042)(195.63117227,73.74061461)(195.63117227,73.60519796)
\closepath
}
}
{
\newrgbcolor{curcolor}{0 0 0}
\pscustom[linestyle=none,fillstyle=solid,fillcolor=curcolor]
{
\newpath
\moveto(205.21710859,80.59738459)
\lineto(205.36554607,80.50363461)
\curveto(205.30825441,79.90467635)(205.27960858,78.70415566)(205.27960858,76.90207255)
\lineto(205.27960858,72.00363565)
\curveto(205.27960858,71.88905233)(205.29002525,71.51926071)(205.31085858,70.89426079)
\curveto(205.33169191,70.2744692)(205.35252524,69.91249008)(205.37335857,69.80832342)
\curveto(205.39940023,69.70415677)(205.43585856,69.62863595)(205.48273356,69.58176095)
\curveto(205.52960855,69.53488596)(205.58690021,69.50363596)(205.65460854,69.48801096)
\curveto(205.72231686,69.4775943)(206.00356683,69.45676097)(206.49835843,69.42551097)
\lineto(206.56867092,69.36301098)
\lineto(206.56867092,69.03488602)
\lineto(206.50617093,68.96457353)
\curveto(205.85512934,69.00624019)(205.22752526,69.02707352)(204.62335866,69.02707352)
\curveto(204.0244004,69.02707352)(203.39940048,69.00624019)(202.74835889,68.96457353)
\lineto(202.6780464,69.03488602)
\lineto(202.6780464,69.36301098)
\lineto(202.74835889,69.42551097)
\curveto(203.25356717,69.45676097)(203.5374213,69.48019846)(203.59992129,69.49582346)
\curveto(203.66762961,69.51144846)(203.72492127,69.54269846)(203.77179627,69.58957345)
\curveto(203.8238796,69.64165678)(203.85773376,69.7171776)(203.87335876,69.81613592)
\curveto(203.89419209,69.92030258)(203.91502542,70.25624004)(203.93585875,70.8239483)
\curveto(203.96190041,71.39686489)(203.97492124,71.82915651)(203.97492124,72.12082314)
\lineto(203.97492124,76.16769764)
\lineto(203.94367125,77.82394744)
\curveto(203.92804625,78.40207236)(203.90981708,78.80311398)(203.88898375,79.02707229)
\curveto(203.87335876,79.25103059)(203.84992126,79.38384308)(203.81867126,79.42550974)
\curveto(203.78742127,79.4671764)(203.73273377,79.4984264)(203.65460878,79.51925973)
\curveto(203.57648379,79.54009306)(203.26137966,79.55050972)(202.7092964,79.55050972)
\lineto(202.63898391,79.62082221)
\lineto(202.63898391,79.94894717)
\lineto(202.7014839,80.01925967)
\curveto(203.69627544,80.13384298)(204.53481701,80.32655129)(205.21710859,80.59738459)
\closepath
}
}
{
\newrgbcolor{curcolor}{0 0 0}
\pscustom[linestyle=none,fillstyle=solid,fillcolor=curcolor]
{
\newpath
\moveto(208.60773317,74.32394787)
\lineto(208.30304571,74.40207286)
\lineto(208.24054572,74.48019785)
\lineto(208.24054572,75.44894773)
\curveto(209.16762894,76.13644764)(210.06867049,76.4801976)(210.94367038,76.4801976)
\curveto(211.54262864,76.4801976)(212.04002442,76.36821845)(212.4358577,76.14426014)
\curveto(212.83169098,75.92030184)(213.11814928,75.63905187)(213.29523259,75.30051025)
\curveto(213.47231591,74.96717695)(213.56085756,74.576552)(213.56085756,74.12863539)
\lineto(213.52179507,72.55832309)
\lineto(213.52179507,70.23019837)
\curveto(213.52179507,69.91249008)(213.5426284,69.71978177)(213.58429506,69.65207344)
\curveto(213.63117005,69.58436512)(213.68325338,69.53749012)(213.74054504,69.51144846)
\curveto(213.7978367,69.49061513)(213.90460752,69.47238597)(214.0608575,69.45676097)
\lineto(214.50616994,69.41769847)
\lineto(214.56866994,69.34738598)
\lineto(214.56866994,69.03488602)
\lineto(214.50616994,68.97238603)
\curveto(214.12596166,69.00363602)(213.77179504,69.01926102)(213.44367008,69.01926102)
\curveto(213.13117011,69.01926102)(212.75617016,69.00363602)(212.31867021,68.97238603)
\lineto(212.20148273,69.08176101)
\lineto(212.23273272,70.33957336)
\lineto(210.52960793,69.01144852)
\curveto(210.24314964,68.89165687)(209.93064968,68.83176104)(209.59210805,68.83176104)
\curveto(209.17544144,68.83176104)(208.81606648,68.90728187)(208.51398318,69.05832352)
\curveto(208.21710822,69.20936516)(207.98794158,69.42030264)(207.82648327,69.69113594)
\curveto(207.67023329,69.96196924)(207.5921083,70.2900942)(207.5921083,70.67551082)
\curveto(207.5921083,71.44113572)(207.8316916,72.03748982)(208.31085821,72.4645731)
\curveto(208.79002482,72.89686471)(210.09731632,73.26405216)(212.23273272,73.56613546)
\curveto(212.23273272,74.3369687)(212.06085775,74.8786353)(211.71710779,75.19113526)
\curveto(211.37335783,75.50884355)(210.91242039,75.6676977)(210.33429546,75.6676977)
\curveto(210.03221216,75.6676977)(209.75617053,75.62342687)(209.50617056,75.53488522)
\curveto(209.26137892,75.44634356)(209.12075394,75.3734269)(209.08429561,75.31613525)
\curveto(209.04783728,75.26405192)(208.9072123,74.94634362)(208.66242067,74.36301036)
\closepath
\moveto(212.23273272,73.08957302)
\curveto(210.77439957,72.84478138)(209.86554552,72.58176058)(209.50617056,72.30051062)
\curveto(209.14679561,72.01926065)(208.96710813,71.58436487)(208.96710813,70.99582328)
\curveto(208.96710813,70.18332338)(209.37075391,69.77707343)(210.17804548,69.77707343)
\curveto(210.87075373,69.77707343)(211.55564948,70.17811505)(212.23273272,70.98019828)
\closepath
\moveto(210.90460789,76.96457254)
\closepath
\moveto(210.74054541,68.55832358)
\closepath
}
}
{
\newrgbcolor{curcolor}{0 0 0}
\pscustom[linestyle=none,fillstyle=solid,fillcolor=curcolor]
{
\newpath
\moveto(217.62335706,76.4723851)
\lineto(217.77179454,76.37082262)
\curveto(217.74054455,76.02186432)(217.71971122,75.56353105)(217.70929455,74.99582278)
\curveto(218.12596116,75.33957274)(218.52179445,75.68853103)(218.8967944,76.04269766)
\curveto(219.00616939,76.14165598)(219.10773188,76.21457263)(219.20148187,76.26144763)
\curveto(219.30044019,76.30832262)(219.4618985,76.35780178)(219.68585681,76.40988511)
\curveto(219.90981511,76.46196844)(220.13898175,76.4880101)(220.37335672,76.4880101)
\curveto(220.76919001,76.4880101)(221.15200246,76.40728094)(221.52179408,76.24582263)
\curveto(221.89679403,76.08436432)(222.178044,75.89165601)(222.36554398,75.6676977)
\curveto(222.55825229,75.44894773)(222.69106477,75.18853109)(222.76398143,74.8864478)
\curveto(222.83689808,74.5843645)(222.87335641,74.21196871)(222.87335641,73.76926044)
\lineto(222.87335641,72.39426061)
\curveto(222.87335641,72.30571895)(222.88637724,71.65728153)(222.91241891,70.44894835)
\curveto(222.92283557,69.95415674)(222.97752307,69.65988594)(223.07648139,69.56613595)
\curveto(223.17543971,69.47238597)(223.49835634,69.42551097)(224.04523127,69.42551097)
\lineto(224.10773126,69.36301098)
\lineto(224.10773126,69.02707352)
\lineto(224.04523127,68.96457353)
\curveto(223.38377302,69.00624019)(222.93585641,69.02707352)(222.70148143,69.02707352)
\curveto(222.56606478,69.02707352)(222.1858565,69.00624019)(221.56085657,68.96457353)
\lineto(221.46710659,69.05051102)
\curveto(221.53481491,69.72759427)(221.56866907,70.61040666)(221.56866907,71.69894819)
\lineto(221.56866907,72.72238556)
\curveto(221.56866907,73.33696882)(221.55304408,73.7796771)(221.52179408,74.0505104)
\curveto(221.49575242,74.3213437)(221.40981493,74.56613534)(221.26398161,74.78488531)
\curveto(221.1181483,75.00884362)(220.92283582,75.1807186)(220.67804418,75.30051025)
\curveto(220.43325255,75.42551023)(220.14158592,75.48801022)(219.80304429,75.48801022)
\curveto(219.53221099,75.48801022)(219.30044019,75.45936439)(219.10773188,75.40207273)
\curveto(218.91502357,75.34998941)(218.69887776,75.23801025)(218.45929446,75.06613528)
\curveto(218.21971115,74.89946863)(218.04262784,74.74061448)(217.92804452,74.58957283)
\curveto(217.8134612,74.44373952)(217.74314871,74.3057187)(217.71710705,74.17551039)
\curveto(217.69627372,74.0505104)(217.68585705,73.7796771)(217.68585705,73.36301049)
\lineto(217.68585705,72.00363565)
\curveto(217.68585705,71.88905233)(217.69627372,71.51926071)(217.71710705,70.89426079)
\curveto(217.73794038,70.2744692)(217.75877371,69.91249008)(217.77960704,69.80832342)
\curveto(217.8056487,69.70415677)(217.84210703,69.62863595)(217.88898203,69.58176095)
\curveto(217.93585702,69.53488596)(217.99314868,69.50363596)(218.06085701,69.48801096)
\curveto(218.12856533,69.4775943)(218.4098153,69.45676097)(218.9046069,69.42551097)
\lineto(218.97491939,69.36301098)
\lineto(218.97491939,69.03488602)
\lineto(218.9124194,68.96457353)
\curveto(218.26137781,69.00624019)(217.63377373,69.02707352)(217.02960713,69.02707352)
\curveto(216.43064887,69.02707352)(215.80564895,69.00624019)(215.15460736,68.96457353)
\lineto(215.08429487,69.03488602)
\lineto(215.08429487,69.36301098)
\lineto(215.15460736,69.42551097)
\curveto(215.65981564,69.45676097)(215.94366977,69.48019846)(216.00616976,69.49582346)
\curveto(216.07387808,69.51144846)(216.13116974,69.54269846)(216.17804474,69.58957345)
\curveto(216.23012807,69.64165678)(216.26398223,69.7171776)(216.27960723,69.81613592)
\curveto(216.30044056,69.92030258)(216.32127389,70.25624004)(216.34210722,70.8239483)
\curveto(216.36814888,71.39686489)(216.38116971,71.82915651)(216.38116971,72.12082314)
\lineto(216.38116971,73.65988545)
\curveto(216.38116971,73.86821876)(216.37075305,74.15207289)(216.34991972,74.51144784)
\curveto(216.32908639,74.8708228)(216.31085722,75.09217694)(216.29523222,75.17551026)
\curveto(216.28481556,75.25884359)(216.24575306,75.31873941)(216.17804474,75.35519774)
\curveto(216.11033641,75.3968644)(215.97491976,75.41769773)(215.77179479,75.41769773)
\lineto(215.11554487,75.42551023)
\lineto(215.04523238,75.48801022)
\lineto(215.04523238,75.82394768)
\lineto(215.10773237,75.88644767)
\curveto(216.10252391,76.00623933)(216.94106548,76.2015518)(217.62335706,76.4723851)
\closepath
}
}
{
\newrgbcolor{curcolor}{0 0 0}
\pscustom[linestyle=none,fillstyle=solid,fillcolor=curcolor]
{
\newpath
\moveto(225.05304364,71.32394824)
\lineto(225.3889811,71.32394824)
\lineto(225.45929359,71.25363575)
\curveto(225.46971026,70.8161358)(225.49575192,70.44113585)(225.53741858,70.12863588)
\curveto(225.6988769,69.88384425)(225.9775227,69.68332344)(226.37335598,69.52707346)
\curveto(226.76918927,69.37603181)(227.15981422,69.30051099)(227.54523084,69.30051099)
\curveto(228.0973141,69.30051099)(228.53741821,69.44894847)(228.86554317,69.74582343)
\curveto(229.19887647,70.0426984)(229.36554311,70.39686502)(229.36554311,70.8083233)
\curveto(229.36554311,71.03228161)(229.30564729,71.22498992)(229.18585563,71.38644823)
\curveto(229.06606398,71.55311488)(228.88377234,71.68853153)(228.6389807,71.79269818)
\curveto(228.3993974,71.90207317)(227.96710578,72.02967732)(227.34210586,72.17551063)
\curveto(226.80564759,72.30051062)(226.43064764,72.39686477)(226.217106,72.4645731)
\curveto(226.00356436,72.53748975)(225.79783522,72.65728141)(225.59991858,72.82394805)
\curveto(225.40200193,72.9906147)(225.25096029,73.19373967)(225.14679363,73.43332298)
\curveto(225.04262698,73.67811461)(224.99054365,73.94634375)(224.99054365,74.23801038)
\curveto(224.99054365,74.93592696)(225.26658528,75.49321856)(225.81866855,75.90988517)
\curveto(226.37596015,76.33176012)(227.0686684,76.54269759)(227.89679329,76.54269759)
\curveto(228.24575158,76.54269759)(228.62595987,76.49321843)(229.03741815,76.39426011)
\curveto(229.44887644,76.30051012)(229.7613764,76.2093643)(229.97491804,76.12082265)
\lineto(230.04523053,76.01144766)
\curveto(230.00356387,75.80311435)(229.9775222,75.25623942)(229.96710554,74.37082286)
\lineto(229.89679305,74.30051037)
\lineto(229.58429309,74.30051037)
\lineto(229.51398059,74.37082286)
\curveto(229.49314726,74.68853116)(229.46450143,74.9098853)(229.4280431,75.03488528)
\curveto(229.39158478,75.1650936)(229.29783479,75.30311441)(229.14679314,75.44894773)
\curveto(228.99575149,75.59998938)(228.78220985,75.72498936)(228.50616822,75.82394768)
\curveto(228.23012659,75.92811434)(227.93845996,75.98019766)(227.63116833,75.98019766)
\curveto(227.31346003,75.98019766)(227.0452309,75.93332267)(226.82648093,75.83957268)
\curveto(226.61293929,75.74582269)(226.43846014,75.60259354)(226.30304349,75.40988523)
\curveto(226.17283517,75.22238526)(226.10773101,74.99321862)(226.10773101,74.72238532)
\curveto(226.10773101,74.52446868)(226.14679351,74.34738536)(226.2249185,74.19113538)
\curveto(226.30825182,74.0348854)(226.43325181,73.90728125)(226.59991845,73.80832293)
\curveto(226.7665851,73.71457294)(226.94106424,73.64686462)(227.12335589,73.60519796)
\lineto(227.97491828,73.38644798)
\curveto(228.70408486,73.20936467)(229.22491813,73.05571886)(229.53741809,72.92551054)
\curveto(229.85512639,72.79530222)(230.09991802,72.59738558)(230.271793,72.33176061)
\curveto(230.44366798,72.07134398)(230.52960547,71.75363568)(230.52960547,71.37863573)
\curveto(230.52960547,70.65988582)(230.23273051,70.0426984)(229.63898058,69.52707346)
\curveto(229.04523065,69.01144852)(228.28220991,68.75363605)(227.34991836,68.75363605)
\curveto(227.0217934,68.75363605)(226.61293929,68.79009438)(226.12335601,68.86301104)
\curveto(225.63898107,68.9359277)(225.22491862,69.01665686)(224.88116867,69.10519851)
\lineto(224.84210617,69.206761)
\lineto(224.92023116,69.73019843)
\curveto(224.94627282,69.89165675)(224.96189782,70.05311506)(224.96710615,70.21457337)
\curveto(224.97231449,70.38124002)(224.97752282,70.72759414)(224.98273115,71.25363575)
\closepath
}
}
{
\newrgbcolor{curcolor}{1 1 1}
\pscustom[linestyle=none,fillstyle=solid,fillcolor=curcolor]
{
\newpath
\moveto(27.3012357,30.66049902)
\curveto(27.3012357,26.48574855)(23.91693177,23.10144462)(19.74218131,23.10144462)
\curveto(15.56743084,23.10144462)(12.18312691,26.48574855)(12.18312691,30.66049902)
\curveto(12.18312691,34.83524948)(15.56743084,38.21955341)(19.74218131,38.21955341)
\curveto(23.91693177,38.21955341)(27.3012357,34.83524948)(27.3012357,30.66049902)
\closepath
}
}
{
\newrgbcolor{curcolor}{0.15686275 0.16078432 0.16470589}
\pscustom[linewidth=2.88359956,linecolor=curcolor]
{
\newpath
\moveto(27.3012357,30.66049902)
\curveto(27.3012357,26.48574855)(23.91693177,23.10144462)(19.74218131,23.10144462)
\curveto(15.56743084,23.10144462)(12.18312691,26.48574855)(12.18312691,30.66049902)
\curveto(12.18312691,34.83524948)(15.56743084,38.21955341)(19.74218131,38.21955341)
\curveto(23.91693177,38.21955341)(27.3012357,34.83524948)(27.3012357,30.66049902)
\closepath
}
}
{
\newrgbcolor{curcolor}{0 0 0}
\pscustom[linestyle=none,fillstyle=solid,fillcolor=curcolor]
{
\newpath
\moveto(23.18358713,29.03549922)
\curveto(23.18358713,27.5354994)(22.6523372,26.41310371)(21.58983733,25.66831213)
\curveto(21.08983739,25.31935384)(20.46483747,25.1448747)(19.71483756,25.1448747)
\curveto(18.65233769,25.1448747)(17.82160863,25.53029132)(17.22265037,26.30112455)
\curveto(16.60806711,27.08237446)(16.30077548,28.18393682)(16.30077548,29.60581165)
\curveto(16.30077548,30.98081148)(16.54817128,32.12143634)(17.04296289,33.02768623)
\curveto(17.56900449,33.98601944)(18.2565044,34.74643601)(19.10546263,35.30893594)
\curveto(19.98046253,35.88706087)(20.95702491,36.17612334)(22.03514977,36.17612334)
\lineto(22.6523372,35.6683109)
\lineto(22.41014973,35.40268593)
\curveto(22.22264975,35.43393593)(22.04556644,35.44956093)(21.87889979,35.44956093)
\curveto(21.07681656,35.44956093)(20.37369164,35.15529013)(19.76952505,34.56674854)
\curveto(19.49869175,34.30112357)(19.26171261,33.96518611)(19.05858764,33.55893616)
\curveto(18.72525435,32.88185291)(18.50129604,31.95216552)(18.38671272,30.769874)
\curveto(19.0846293,31.42091559)(19.50390008,31.77508221)(19.64452507,31.83237387)
\curveto(19.97785836,31.97299886)(20.36587914,32.04331135)(20.80858742,32.04331135)
\curveto(21.48046234,32.04331135)(22.03514977,31.79331138)(22.47264972,31.29331144)
\curveto(22.94660799,30.75164484)(23.18358713,29.99904077)(23.18358713,29.03549922)
\closepath
\moveto(21.30077486,28.17612432)
\curveto(21.30077486,28.98341589)(21.14973321,29.60320748)(20.84764992,30.03549909)
\curveto(20.57160829,30.44174904)(20.19400417,30.64487402)(19.71483756,30.64487402)
\curveto(19.09504597,30.64487402)(18.68358769,30.37664489)(18.48046271,29.84018662)
\curveto(18.38671272,29.59018665)(18.33983773,29.24383252)(18.33983773,28.80112425)
\curveto(18.33983773,27.65529105)(18.55598353,26.84539532)(18.98827515,26.37143705)
\curveto(19.23306678,26.10060375)(19.54035841,25.9651871)(19.91015003,25.9651871)
\curveto(20.83723325,25.9651871)(21.30077486,26.70216617)(21.30077486,28.17612432)
\closepath
}
}
{
\newrgbcolor{curcolor}{0 0 0}
\pscustom[linestyle=none,fillstyle=solid,fillcolor=curcolor]
{
\newpath
\moveto(34.42025465,28.3952812)
\lineto(34.48275464,28.45778119)
\lineto(34.86556709,28.45778119)
\lineto(34.92806709,28.3952812)
\curveto(34.95410875,27.72861461)(34.98796291,27.3197605)(35.02962957,27.16871885)
\curveto(35.07650457,27.02288553)(35.21712955,26.86663555)(35.45150452,26.69996891)
\curveto(35.69108783,26.53851059)(36.00619195,26.40569811)(36.39681691,26.30153146)
\curveto(36.79265019,26.1973648)(37.19108764,26.14528147)(37.59212926,26.14528147)
\curveto(38.14421252,26.14528147)(38.63900413,26.24684396)(39.07650408,26.44996894)
\curveto(39.51921235,26.65309391)(39.86296231,26.94996888)(40.10775395,27.34059383)
\curveto(40.35254559,27.73642711)(40.4749414,28.17913539)(40.4749414,28.66871866)
\curveto(40.4749414,29.01246862)(40.41504558,29.31194775)(40.29525393,29.56715605)
\curveto(40.18067061,29.82236435)(40.01660813,30.02548933)(39.80306649,30.17653098)
\curveto(39.59473318,30.33278096)(39.35514987,30.44736428)(39.08431657,30.52028094)
\curveto(38.81348327,30.59840593)(38.40983749,30.68173925)(37.87337922,30.7702809)
\curveto(37.35254595,30.85361423)(36.94369184,30.92913505)(36.64681688,30.99684338)
\curveto(36.35515024,31.0645517)(36.06348361,31.16871836)(35.77181698,31.30934334)
\curveto(35.48015035,31.44996832)(35.23535872,31.62705163)(35.03744207,31.84059327)
\curveto(34.84473376,32.05934325)(34.68848378,32.32496821)(34.56869213,32.63746817)
\curveto(34.45410881,32.95517647)(34.39681715,33.29371809)(34.39681715,33.65309305)
\curveto(34.39681715,34.62705126)(34.73535878,35.42392616)(35.41244203,36.04371775)
\curveto(36.08952528,36.66871768)(36.99837933,36.98121764)(38.13900419,36.98121764)
\curveto(38.59212914,36.98121764)(39.08431657,36.92392598)(39.61556651,36.80934266)
\curveto(40.15202478,36.69996767)(40.63900388,36.53590519)(41.07650383,36.31715522)
\lineto(41.13119132,36.22340523)
\curveto(41.016608,35.74423862)(40.94629551,35.06194704)(40.92025385,34.17653048)
\lineto(40.84994136,34.11403049)
\lineto(40.44369141,34.11403049)
\lineto(40.38119141,34.16871799)
\curveto(40.37077475,34.79371791)(40.35514975,35.18694703)(40.33431642,35.34840534)
\curveto(40.31348309,35.50986365)(40.07650395,35.71038446)(39.62337901,35.94996777)
\curveto(39.17025406,36.18955107)(38.67806662,36.30934272)(38.14681669,36.30934272)
\curveto(37.70410841,36.30934272)(37.29004596,36.22080107)(36.90462934,36.04371775)
\curveto(36.51921272,35.86663444)(36.22754609,35.58798864)(36.02962945,35.20778036)
\curveto(35.83171281,34.82757207)(35.73275449,34.43955129)(35.73275449,34.043718)
\curveto(35.73275449,33.73642637)(35.79265031,33.46298891)(35.91244197,33.2234056)
\curveto(36.03223362,32.98903063)(36.18587943,32.80153065)(36.37337941,32.66090567)
\curveto(36.56608772,32.52548902)(36.78744186,32.42392653)(37.03744183,32.35621821)
\curveto(37.29265013,32.28850988)(37.75619174,32.20517656)(38.42806666,32.10621824)
\curveto(39.34994154,31.97600992)(40.01921229,31.81715578)(40.43587891,31.6296558)
\curveto(40.85775386,31.44736415)(41.19369131,31.15569752)(41.44369128,30.75465591)
\curveto(41.69889959,30.35361429)(41.82650374,29.86923935)(41.82650374,29.30153109)
\curveto(41.82650374,28.19215622)(41.37337879,27.26767717)(40.4671289,26.52809393)
\curveto(39.56608735,25.78851069)(38.47233748,25.41871906)(37.18587931,25.41871906)
\curveto(36.10775444,25.41871906)(35.16244206,25.61142737)(34.34994216,25.99684399)
\lineto(34.30306716,26.09840648)
\curveto(34.35515049,26.42653144)(34.39421299,27.19215635)(34.42025465,28.3952812)
\closepath
}
}
{
\newrgbcolor{curcolor}{0 0 0}
\pscustom[linestyle=none,fillstyle=solid,fillcolor=curcolor]
{
\newpath
\moveto(42.56087865,31.84059327)
\lineto(42.56087865,32.04371825)
\lineto(42.61556614,32.12184324)
\curveto(43.10514941,32.30413488)(43.5009827,32.47340569)(43.80306599,32.62965568)
\curveto(43.80306599,33.98903051)(43.78744099,34.78069708)(43.756191,35.00465538)
\curveto(44.29264927,35.19215536)(44.73275338,35.39267617)(45.07650334,35.60621781)
\lineto(45.26400331,35.44996783)
\curveto(45.21191999,35.12184287)(45.14941999,34.16871799)(45.07650334,32.59059318)
\curveto(45.33691997,32.58538485)(45.61816994,32.58278068)(45.92025323,32.58278068)
\curveto(46.53483649,32.58278068)(46.9749406,32.59840568)(47.24056557,32.62965568)
\lineto(47.29525306,32.57496818)
\lineto(47.14681558,31.91871826)
\lineto(47.08431559,31.84840577)
\curveto(46.81869062,31.8536141)(46.52441982,31.85621827)(46.2015032,31.85621827)
\curveto(45.90983657,31.85621827)(45.53483661,31.8536141)(45.07650334,31.84840577)
\lineto(45.02962834,28.66871866)
\curveto(45.02962834,27.93434375)(45.04525334,27.45257298)(45.07650334,27.22340634)
\curveto(45.11296166,26.99944804)(45.20671165,26.82236472)(45.3577533,26.69215641)
\curveto(45.51400328,26.56715642)(45.74316992,26.50465643)(46.04525322,26.50465643)
\curveto(46.39421151,26.50465643)(46.71712813,26.59580225)(47.0140031,26.7780939)
\lineto(47.20150307,26.49684393)
\curveto(47.07650309,26.40830228)(46.77702396,26.14788564)(46.30306568,25.71559403)
\curveto(46.03223238,25.59059404)(45.75619075,25.52809405)(45.47494079,25.52809405)
\curveto(44.30306593,25.52809405)(43.7171285,26.09580231)(43.7171285,27.23121884)
\curveto(43.7171285,27.64788546)(43.72754517,28.00205208)(43.7483785,28.29371871)
\curveto(43.75358683,28.38226037)(43.756191,28.47340619)(43.756191,28.56715618)
\lineto(43.756191,31.80934328)
\lineto(43.43587854,31.80934328)
\curveto(43.20150357,31.80934328)(42.93327443,31.79892661)(42.63119114,31.77809328)
\closepath
}
}
{
\newrgbcolor{curcolor}{0 0 0}
\pscustom[linestyle=none,fillstyle=solid,fillcolor=curcolor]
{
\newpath
\moveto(49.07650284,31.00465588)
\lineto(48.77181538,31.08278087)
\lineto(48.70931539,31.16090586)
\lineto(48.70931539,32.12965574)
\curveto(49.63639861,32.81715565)(50.53744016,33.16090561)(51.41244005,33.16090561)
\curveto(52.01139831,33.16090561)(52.50879409,33.04892646)(52.90462737,32.82496815)
\curveto(53.30046066,32.60100985)(53.58691895,32.31975988)(53.76400226,31.98121826)
\curveto(53.94108558,31.64788496)(54.02962723,31.25726001)(54.02962723,30.8093434)
\lineto(53.99056474,29.23903109)
\lineto(53.99056474,26.91090638)
\curveto(53.99056474,26.59319809)(54.01139807,26.40048978)(54.05306473,26.33278145)
\curveto(54.09993972,26.26507313)(54.15202305,26.21819813)(54.20931471,26.19215647)
\curveto(54.26660637,26.17132314)(54.37337719,26.15309397)(54.52962717,26.13746898)
\lineto(54.97493962,26.09840648)
\lineto(55.03743961,26.02809399)
\lineto(55.03743961,25.71559403)
\lineto(54.97493962,25.65309404)
\curveto(54.59473133,25.68434403)(54.24056471,25.69996903)(53.91243975,25.69996903)
\curveto(53.59993978,25.69996903)(53.22493983,25.68434403)(52.78743989,25.65309404)
\lineto(52.6702524,25.76246902)
\lineto(52.7015024,27.02028137)
\lineto(50.99837761,25.69215653)
\curveto(50.71191931,25.57236488)(50.39941935,25.51246905)(50.06087772,25.51246905)
\curveto(49.64421111,25.51246905)(49.28483615,25.58798988)(48.98275285,25.73903152)
\curveto(48.68587789,25.89007317)(48.45671125,26.10101065)(48.29525294,26.37184395)
\curveto(48.13900296,26.64267725)(48.06087797,26.97080221)(48.06087797,27.35621883)
\curveto(48.06087797,28.12184373)(48.30046127,28.71819782)(48.77962788,29.1452811)
\curveto(49.25879449,29.57757272)(50.56608599,29.94476017)(52.7015024,30.24684347)
\curveto(52.7015024,31.01767671)(52.52962742,31.55934331)(52.18587746,31.87184327)
\curveto(51.8421275,32.18955156)(51.38119006,32.34840571)(50.80306513,32.34840571)
\curveto(50.50098183,32.34840571)(50.2249402,32.30413488)(49.97494023,32.21559323)
\curveto(49.7301486,32.12705157)(49.58952361,32.05413491)(49.55306528,31.99684325)
\curveto(49.51660696,31.94475993)(49.37598197,31.62705163)(49.13119034,31.04371837)
\closepath
\moveto(52.7015024,29.77028103)
\curveto(51.24316924,29.52548939)(50.33431519,29.26246859)(49.97494023,28.98121862)
\curveto(49.61556528,28.69996866)(49.4358778,28.26507288)(49.4358778,27.67653129)
\curveto(49.4358778,26.86403139)(49.83952358,26.45778144)(50.64681515,26.45778144)
\curveto(51.3395234,26.45778144)(52.02441915,26.85882305)(52.7015024,27.66090629)
\closepath
\moveto(51.37337756,33.64528055)
\closepath
\moveto(51.20931508,25.23903159)
\closepath
}
}
{
\newrgbcolor{curcolor}{0 0 0}
\pscustom[linestyle=none,fillstyle=solid,fillcolor=curcolor]
{
\newpath
\moveto(55.77962702,31.84059327)
\lineto(55.77962702,32.04371825)
\lineto(55.83431451,32.12184324)
\curveto(56.32389778,32.30413488)(56.71973107,32.47340569)(57.02181436,32.62965568)
\curveto(57.02181436,33.98903051)(57.00618936,34.78069708)(56.97493937,35.00465538)
\curveto(57.51139764,35.19215536)(57.95150175,35.39267617)(58.29525171,35.60621781)
\lineto(58.48275168,35.44996783)
\curveto(58.43066836,35.12184287)(58.36816836,34.16871799)(58.29525171,32.59059318)
\curveto(58.55566834,32.58538485)(58.83691831,32.58278068)(59.1390016,32.58278068)
\curveto(59.75358486,32.58278068)(60.19368897,32.59840568)(60.45931394,32.62965568)
\lineto(60.51400143,32.57496818)
\lineto(60.36556395,31.91871826)
\lineto(60.30306396,31.84840577)
\curveto(60.03743899,31.8536141)(59.74316819,31.85621827)(59.42025157,31.85621827)
\curveto(59.12858494,31.85621827)(58.75358498,31.8536141)(58.29525171,31.84840577)
\lineto(58.24837671,28.66871866)
\curveto(58.24837671,27.93434375)(58.26400171,27.45257298)(58.29525171,27.22340634)
\curveto(58.33171003,26.99944804)(58.42546002,26.82236472)(58.57650167,26.69215641)
\curveto(58.73275165,26.56715642)(58.96191829,26.50465643)(59.26400159,26.50465643)
\curveto(59.61295988,26.50465643)(59.9358765,26.59580225)(60.23275147,26.7780939)
\lineto(60.42025144,26.49684393)
\curveto(60.29525146,26.40830228)(59.99577233,26.14788564)(59.52181405,25.71559403)
\curveto(59.25098075,25.59059404)(58.97493912,25.52809405)(58.69368916,25.52809405)
\curveto(57.5218143,25.52809405)(56.93587687,26.09580231)(56.93587687,27.23121884)
\curveto(56.93587687,27.64788546)(56.94629354,28.00205208)(56.96712687,28.29371871)
\curveto(56.9723352,28.38226037)(56.97493937,28.47340619)(56.97493937,28.56715618)
\lineto(56.97493937,31.80934328)
\lineto(56.65462691,31.80934328)
\curveto(56.42025194,31.80934328)(56.1520228,31.79892661)(55.84993951,31.77809328)
\closepath
}
}
{
\newrgbcolor{curcolor}{0 0 0}
\pscustom[linestyle=none,fillstyle=solid,fillcolor=curcolor]
{
\newpath
\moveto(62.88900114,36.64528018)
\curveto(63.13379278,36.64528018)(63.34212608,36.55934269)(63.51400106,36.38746771)
\curveto(63.68587604,36.21559273)(63.77181353,36.00725943)(63.77181353,35.76246779)
\curveto(63.77181353,35.52288449)(63.68587604,35.31715534)(63.51400106,35.14528036)
\curveto(63.34212608,34.97340539)(63.13379278,34.8874679)(62.88900114,34.8874679)
\curveto(62.64941784,34.8874679)(62.44108453,34.97080122)(62.26400122,35.13746787)
\curveto(62.09212624,35.30934284)(62.00618875,35.51767615)(62.00618875,35.76246779)
\curveto(62.00618875,36.00725943)(62.09212624,36.21559273)(62.26400122,36.38746771)
\curveto(62.44108453,36.55934269)(62.64941784,36.64528018)(62.88900114,36.64528018)
\closepath
\moveto(63.55306356,33.15309311)
\lineto(63.70150104,33.05153062)
\curveto(63.64420938,32.36923904)(63.61556355,31.50205165)(63.61556355,30.44996844)
\lineto(63.61556355,28.68434366)
\curveto(63.61556355,28.56976034)(63.62598022,28.19996872)(63.64681355,27.5749688)
\curveto(63.66764688,26.95517721)(63.68848021,26.59319809)(63.70931354,26.48903143)
\curveto(63.7353552,26.38486478)(63.77181353,26.30934395)(63.81868852,26.26246896)
\curveto(63.86556352,26.21559397)(63.92285518,26.18434397)(63.9905635,26.16871897)
\curveto(64.05827183,26.15830231)(64.33952179,26.13746898)(64.8343134,26.10621898)
\lineto(64.90462589,26.04371899)
\lineto(64.90462589,25.71559403)
\lineto(64.8421259,25.64528154)
\curveto(64.19108431,25.6869482)(63.56348022,25.70778153)(62.95931363,25.70778153)
\curveto(62.36035537,25.70778153)(61.73535545,25.6869482)(61.08431386,25.64528154)
\lineto(61.01400137,25.71559403)
\lineto(61.01400137,26.04371899)
\lineto(61.08431386,26.10621898)
\curveto(61.58952213,26.13746898)(61.87337626,26.16090647)(61.93587626,26.17653147)
\curveto(62.00358458,26.19215647)(62.06087624,26.22340647)(62.10775124,26.27028146)
\curveto(62.15983456,26.32236479)(62.19368873,26.39788561)(62.20931372,26.49684393)
\curveto(62.23014705,26.60101059)(62.25098038,26.93694804)(62.27181372,27.50465631)
\curveto(62.29785538,28.0775729)(62.31087621,28.50986452)(62.31087621,28.80153115)
\lineto(62.31087621,30.34059346)
\curveto(62.31087621,30.54892677)(62.30045955,30.8327809)(62.27962621,31.19215585)
\curveto(62.25879288,31.55153081)(62.24056372,31.77288495)(62.22493872,31.85621827)
\curveto(62.21452206,31.93955159)(62.17545956,31.99944742)(62.10775124,32.03590575)
\curveto(62.04004291,32.07757241)(61.90462626,32.09840574)(61.70150129,32.09840574)
\lineto(61.04525137,32.10621824)
\lineto(60.97493888,32.16871823)
\lineto(60.97493888,32.50465569)
\lineto(61.03743887,32.56715568)
\curveto(62.03223041,32.68694733)(62.87077198,32.88225981)(63.55306356,33.15309311)
\closepath
\moveto(62.94368863,25.23903159)
\closepath
}
}
{
\newrgbcolor{curcolor}{0 0 0}
\pscustom[linestyle=none,fillstyle=solid,fillcolor=curcolor]
{
\newpath
\moveto(66.08431325,28.00465625)
\lineto(66.4202507,28.00465625)
\lineto(66.4905632,27.93434375)
\curveto(66.50097986,27.49684381)(66.52702152,27.12184385)(66.56868819,26.80934389)
\curveto(66.7301465,26.56455226)(67.0087923,26.36403145)(67.40462558,26.20778147)
\curveto(67.80045887,26.05673982)(68.19108382,25.98121899)(68.57650044,25.98121899)
\curveto(69.1285837,25.98121899)(69.56868782,26.12965648)(69.89681278,26.42653144)
\curveto(70.23014607,26.7234064)(70.39681271,27.07757303)(70.39681271,27.48903131)
\curveto(70.39681271,27.71298961)(70.33691689,27.90569792)(70.21712524,28.06715624)
\curveto(70.09733358,28.23382288)(69.91504194,28.36923953)(69.6702503,28.47340619)
\curveto(69.430667,28.58278117)(68.99837539,28.71038533)(68.37337546,28.85621864)
\curveto(67.8369172,28.98121862)(67.46191724,29.07757278)(67.2483756,29.1452811)
\curveto(67.03483396,29.21819776)(66.82910482,29.33798941)(66.63118818,29.50465606)
\curveto(66.43327154,29.67132271)(66.28222989,29.87444768)(66.17806323,30.11403099)
\curveto(66.07389658,30.35882262)(66.02181325,30.62705176)(66.02181325,30.91871839)
\curveto(66.02181325,31.61663497)(66.29785489,32.17392656)(66.84993815,32.59059318)
\curveto(67.40722975,33.01246813)(68.099938,33.2234056)(68.92806289,33.2234056)
\curveto(69.27702119,33.2234056)(69.65722947,33.17392644)(70.06868775,33.07496812)
\curveto(70.48014604,32.98121813)(70.792646,32.89007231)(71.00618764,32.80153065)
\lineto(71.07650013,32.69215567)
\curveto(71.03483347,32.48382236)(71.00879181,31.93694743)(70.99837514,31.05153087)
\lineto(70.92806265,30.98121838)
\lineto(70.61556269,30.98121838)
\lineto(70.5452502,31.05153087)
\curveto(70.52441686,31.36923916)(70.49577103,31.5905933)(70.45931271,31.71559329)
\curveto(70.42285438,31.84580161)(70.32910439,31.98382242)(70.17806274,32.12965574)
\curveto(70.02702109,32.28069738)(69.81347945,32.40569737)(69.53743782,32.50465569)
\curveto(69.26139619,32.60882234)(68.96972956,32.66090567)(68.66243793,32.66090567)
\curveto(68.34472963,32.66090567)(68.0765005,32.61403068)(67.85775053,32.52028069)
\curveto(67.64420889,32.4265307)(67.46972974,32.28330155)(67.33431309,32.09059324)
\curveto(67.20410477,31.90309326)(67.13900062,31.67392663)(67.13900062,31.40309333)
\curveto(67.13900062,31.20517668)(67.17806311,31.02809337)(67.2561881,30.87184339)
\curveto(67.33952142,30.71559341)(67.46452141,30.58798926)(67.63118805,30.48903094)
\curveto(67.7978547,30.39528095)(67.97233385,30.32757263)(68.15462549,30.28590596)
\lineto(69.00618789,30.06715599)
\curveto(69.73535446,29.89007268)(70.25618773,29.73642687)(70.56868769,29.60621855)
\curveto(70.88639599,29.47601023)(71.13118762,29.27809359)(71.3030626,29.01246862)
\curveto(71.47493758,28.75205199)(71.56087507,28.43434369)(71.56087507,28.05934374)
\curveto(71.56087507,27.34059383)(71.26400011,26.7234064)(70.67025018,26.20778147)
\curveto(70.07650025,25.69215653)(69.31347951,25.43434406)(68.38118796,25.43434406)
\curveto(68.053063,25.43434406)(67.64420889,25.47080239)(67.15462561,25.54371905)
\curveto(66.67025067,25.61663571)(66.25618822,25.69736486)(65.91243827,25.78590652)
\lineto(65.87337577,25.88746901)
\lineto(65.95150076,26.41090644)
\curveto(65.97754243,26.57236476)(65.99316742,26.73382307)(65.99837576,26.89528138)
\curveto(66.00358409,27.06194803)(66.00879242,27.40830215)(66.01400075,27.93434375)
\closepath
}
}
{
\newrgbcolor{curcolor}{0 0 0}
\pscustom[linestyle=none,fillstyle=solid,fillcolor=curcolor]
{
\newpath
\moveto(72.43587496,31.84059327)
\lineto(72.43587496,32.04371825)
\lineto(72.49056246,32.12184324)
\curveto(72.98014573,32.30413488)(73.37597901,32.47340569)(73.67806231,32.62965568)
\curveto(73.67806231,33.98903051)(73.66243731,34.78069708)(73.63118732,35.00465538)
\curveto(74.16764558,35.19215536)(74.60774969,35.39267617)(74.95149965,35.60621781)
\lineto(75.13899963,35.44996783)
\curveto(75.0869163,35.12184287)(75.02441631,34.16871799)(74.95149965,32.59059318)
\curveto(75.21191629,32.58538485)(75.49316625,32.58278068)(75.79524955,32.58278068)
\curveto(76.40983281,32.58278068)(76.84993692,32.59840568)(77.11556189,32.62965568)
\lineto(77.17024938,32.57496818)
\lineto(77.0218119,31.91871826)
\lineto(76.9593119,31.84840577)
\curveto(76.69368694,31.8536141)(76.39941614,31.85621827)(76.07649951,31.85621827)
\curveto(75.78483288,31.85621827)(75.40983293,31.8536141)(74.95149965,31.84840577)
\lineto(74.90462466,28.66871866)
\curveto(74.90462466,27.93434375)(74.92024966,27.45257298)(74.95149965,27.22340634)
\curveto(74.98795798,26.99944804)(75.08170797,26.82236472)(75.23274962,26.69215641)
\curveto(75.3889996,26.56715642)(75.61816624,26.50465643)(75.92024953,26.50465643)
\curveto(76.26920782,26.50465643)(76.59212445,26.59580225)(76.88899941,26.7780939)
\lineto(77.07649939,26.49684393)
\curveto(76.95149941,26.40830228)(76.65202028,26.14788564)(76.178062,25.71559403)
\curveto(75.9072287,25.59059404)(75.63118707,25.52809405)(75.3499371,25.52809405)
\curveto(74.17806225,25.52809405)(73.59212482,26.09580231)(73.59212482,27.23121884)
\curveto(73.59212482,27.64788546)(73.60254149,28.00205208)(73.62337482,28.29371871)
\curveto(73.62858315,28.38226037)(73.63118732,28.47340619)(73.63118732,28.56715618)
\lineto(73.63118732,31.80934328)
\lineto(73.31087485,31.80934328)
\curveto(73.07649988,31.80934328)(72.80827075,31.79892661)(72.50618745,31.77809328)
\closepath
}
}
{
\newrgbcolor{curcolor}{0 0 0}
\pscustom[linestyle=none,fillstyle=solid,fillcolor=curcolor]
{
\newpath
\moveto(79.54524909,36.64528018)
\curveto(79.79004072,36.64528018)(79.99837403,36.55934269)(80.17024901,36.38746771)
\curveto(80.34212399,36.21559273)(80.42806148,36.00725943)(80.42806148,35.76246779)
\curveto(80.42806148,35.52288449)(80.34212399,35.31715534)(80.17024901,35.14528036)
\curveto(79.99837403,34.97340539)(79.79004072,34.8874679)(79.54524909,34.8874679)
\curveto(79.30566578,34.8874679)(79.09733247,34.97080122)(78.92024916,35.13746787)
\curveto(78.74837418,35.30934284)(78.66243669,35.51767615)(78.66243669,35.76246779)
\curveto(78.66243669,36.00725943)(78.74837418,36.21559273)(78.92024916,36.38746771)
\curveto(79.09733247,36.55934269)(79.30566578,36.64528018)(79.54524909,36.64528018)
\closepath
\moveto(80.2093115,33.15309311)
\lineto(80.35774899,33.05153062)
\curveto(80.30045733,32.36923904)(80.2718115,31.50205165)(80.2718115,30.44996844)
\lineto(80.2718115,28.68434366)
\curveto(80.2718115,28.56976034)(80.28222816,28.19996872)(80.30306149,27.5749688)
\curveto(80.32389482,26.95517721)(80.34472815,26.59319809)(80.36556148,26.48903143)
\curveto(80.39160315,26.38486478)(80.42806148,26.30934395)(80.47493647,26.26246896)
\curveto(80.52181147,26.21559397)(80.57910312,26.18434397)(80.64681145,26.16871897)
\curveto(80.71451977,26.15830231)(80.99576974,26.13746898)(81.49056135,26.10621898)
\lineto(81.56087384,26.04371899)
\lineto(81.56087384,25.71559403)
\lineto(81.49837384,25.64528154)
\curveto(80.84733226,25.6869482)(80.21972817,25.70778153)(79.61556158,25.70778153)
\curveto(79.01660332,25.70778153)(78.39160339,25.6869482)(77.74056181,25.64528154)
\lineto(77.67024932,25.71559403)
\lineto(77.67024932,26.04371899)
\lineto(77.74056181,26.10621898)
\curveto(78.24577008,26.13746898)(78.52962421,26.16090647)(78.5921242,26.17653147)
\curveto(78.65983253,26.19215647)(78.71712419,26.22340647)(78.76399918,26.27028146)
\curveto(78.81608251,26.32236479)(78.84993667,26.39788561)(78.86556167,26.49684393)
\curveto(78.886395,26.60101059)(78.90722833,26.93694804)(78.92806166,27.50465631)
\curveto(78.95410333,28.0775729)(78.96712416,28.50986452)(78.96712416,28.80153115)
\lineto(78.96712416,30.34059346)
\curveto(78.96712416,30.54892677)(78.95670749,30.8327809)(78.93587416,31.19215585)
\curveto(78.91504083,31.55153081)(78.89681167,31.77288495)(78.88118667,31.85621827)
\curveto(78.87077,31.93955159)(78.83170751,31.99944742)(78.76399918,32.03590575)
\curveto(78.69629086,32.07757241)(78.56087421,32.09840574)(78.35774923,32.09840574)
\lineto(77.70149931,32.10621824)
\lineto(77.63118682,32.16871823)
\lineto(77.63118682,32.50465569)
\lineto(77.69368681,32.56715568)
\curveto(78.68847836,32.68694733)(79.52701992,32.88225981)(80.2093115,33.15309311)
\closepath
\moveto(79.59993658,25.23903159)
\closepath
}
}
{
\newrgbcolor{curcolor}{0 0 0}
\pscustom[linestyle=none,fillstyle=solid,fillcolor=curcolor]
{
\newpath
\moveto(82.74056119,28.00465625)
\lineto(83.07649865,28.00465625)
\lineto(83.14681114,27.93434375)
\curveto(83.15722781,27.49684381)(83.18326947,27.12184385)(83.22493613,26.80934389)
\curveto(83.38639445,26.56455226)(83.66504024,26.36403145)(84.06087353,26.20778147)
\curveto(84.45670681,26.05673982)(84.84733177,25.98121899)(85.23274838,25.98121899)
\curveto(85.78483165,25.98121899)(86.22493576,26.12965648)(86.55306072,26.42653144)
\curveto(86.88639401,26.7234064)(87.05306066,27.07757303)(87.05306066,27.48903131)
\curveto(87.05306066,27.71298961)(86.99316483,27.90569792)(86.87337318,28.06715624)
\curveto(86.75358153,28.23382288)(86.57128989,28.36923953)(86.32649825,28.47340619)
\curveto(86.08691495,28.58278117)(85.65462333,28.71038533)(85.02962341,28.85621864)
\curveto(84.49316514,28.98121862)(84.11816519,29.07757278)(83.90462355,29.1452811)
\curveto(83.69108191,29.21819776)(83.48535277,29.33798941)(83.28743612,29.50465606)
\curveto(83.08951948,29.67132271)(82.93847783,29.87444768)(82.83431118,30.11403099)
\curveto(82.73014453,30.35882262)(82.6780612,30.62705176)(82.6780612,30.91871839)
\curveto(82.6780612,31.61663497)(82.95410283,32.17392656)(83.5061861,32.59059318)
\curveto(84.0634777,33.01246813)(84.75618594,33.2234056)(85.58431084,33.2234056)
\curveto(85.93326913,33.2234056)(86.31347742,33.17392644)(86.7249357,33.07496812)
\curveto(87.13639398,32.98121813)(87.44889394,32.89007231)(87.66243558,32.80153065)
\lineto(87.73274808,32.69215567)
\curveto(87.69108141,32.48382236)(87.66503975,31.93694743)(87.65462309,31.05153087)
\lineto(87.58431059,30.98121838)
\lineto(87.27181063,30.98121838)
\lineto(87.20149814,31.05153087)
\curveto(87.18066481,31.36923916)(87.15201898,31.5905933)(87.11556065,31.71559329)
\curveto(87.07910232,31.84580161)(86.98535234,31.98382242)(86.83431069,32.12965574)
\curveto(86.68326904,32.28069738)(86.4697274,32.40569737)(86.19368577,32.50465569)
\curveto(85.91764413,32.60882234)(85.6259775,32.66090567)(85.31868587,32.66090567)
\curveto(85.00097758,32.66090567)(84.73274845,32.61403068)(84.51399847,32.52028069)
\curveto(84.30045683,32.4265307)(84.12597769,32.28330155)(83.99056104,32.09059324)
\curveto(83.86035272,31.90309326)(83.79524856,31.67392663)(83.79524856,31.40309333)
\curveto(83.79524856,31.20517668)(83.83431106,31.02809337)(83.91243605,30.87184339)
\curveto(83.99576937,30.71559341)(84.12076935,30.58798926)(84.287436,30.48903094)
\curveto(84.45410265,30.39528095)(84.62858179,30.32757263)(84.81087344,30.28590596)
\lineto(85.66243583,30.06715599)
\curveto(86.39160241,29.89007268)(86.91243568,29.73642687)(87.22493564,29.60621855)
\curveto(87.54264393,29.47601023)(87.78743557,29.27809359)(87.95931055,29.01246862)
\curveto(88.13118553,28.75205199)(88.21712302,28.43434369)(88.21712302,28.05934374)
\curveto(88.21712302,27.34059383)(87.92024805,26.7234064)(87.32649813,26.20778147)
\curveto(86.7327482,25.69215653)(85.96972746,25.43434406)(85.03743591,25.43434406)
\curveto(84.70931095,25.43434406)(84.30045683,25.47080239)(83.81087356,25.54371905)
\curveto(83.32649862,25.61663571)(82.91243617,25.69736486)(82.56868621,25.78590652)
\lineto(82.52962372,25.88746901)
\lineto(82.60774871,26.41090644)
\curveto(82.63379037,26.57236476)(82.64941537,26.73382307)(82.6546237,26.89528138)
\curveto(82.65983204,27.06194803)(82.66504037,27.40830215)(82.6702487,27.93434375)
\closepath
}
}
{
\newrgbcolor{curcolor}{0 0 0}
\pscustom[linestyle=none,fillstyle=solid,fillcolor=curcolor]
{
\newpath
\moveto(95.51399712,26.54371893)
\lineto(95.20149716,26.02028149)
\curveto(94.52962224,25.65048987)(93.77962233,25.46559406)(92.95149743,25.46559406)
\curveto(91.80566424,25.46559406)(90.92024768,25.80673985)(90.29524776,26.48903143)
\curveto(89.67024784,27.17132301)(89.35774788,28.05153124)(89.35774788,29.12965611)
\curveto(89.35774788,29.61923938)(89.4098312,30.05413516)(89.51399786,30.43434345)
\curveto(89.62337284,30.81976006)(89.75878949,31.13746836)(89.92024781,31.38746833)
\curveto(90.08691445,31.6374683)(90.2692061,31.83538494)(90.46712274,31.98121826)
\curveto(90.66503938,32.12705157)(90.99837267,32.33017655)(91.46712262,32.59059318)
\curveto(91.94108089,32.85100981)(92.29785168,33.02028063)(92.53743498,33.09840562)
\curveto(92.77701829,33.18173894)(93.10774741,33.2234056)(93.52962236,33.2234056)
\curveto(94.33691393,33.2234056)(94.99837218,33.06975979)(95.51399712,32.76246816)
\curveto(95.4150388,32.17392656)(95.33951797,31.51767665)(95.28743464,30.7937184)
\lineto(95.21712215,30.73121841)
\lineto(94.89680969,30.73121841)
\lineto(94.8264972,30.8015309)
\curveto(94.80566387,31.23903085)(94.77701804,31.53330164)(94.74055971,31.68434329)
\curveto(94.70410138,31.83538494)(94.49576808,31.99684325)(94.11555979,32.16871823)
\curveto(93.74055984,32.34059321)(93.33951822,32.4265307)(92.91243494,32.4265307)
\curveto(92.48535166,32.4265307)(92.10514337,32.33538488)(91.77181008,32.15309323)
\curveto(91.43847679,31.97600992)(91.18587265,31.68173913)(91.01399767,31.27028084)
\curveto(90.84733103,30.86403089)(90.7639977,30.37705179)(90.7639977,29.80934352)
\curveto(90.7639977,29.33017692)(90.83170603,28.86142697)(90.96712268,28.4030937)
\curveto(91.10253933,27.94996875)(91.27962264,27.57236463)(91.49837261,27.27028134)
\curveto(91.72233092,26.97340637)(92.01660171,26.73382307)(92.381185,26.55153142)
\curveto(92.74576829,26.36923978)(93.15201824,26.27809396)(93.59993485,26.27809396)
\curveto(93.88118482,26.27809396)(94.15462228,26.31455229)(94.42024725,26.38746894)
\curveto(94.69108055,26.4603856)(95.00097635,26.58017725)(95.34993464,26.7468439)
\closepath
}
}
{
\newrgbcolor{curcolor}{0 0 0}
\pscustom[linestyle=none,fillstyle=solid,fillcolor=curcolor]
{
\newpath
\moveto(98.49055925,37.2780926)
\lineto(98.63899673,37.18434261)
\curveto(98.58170507,36.58538435)(98.55305924,35.38486367)(98.55305924,33.58278056)
\lineto(98.55305924,31.6530933)
\curveto(98.99055919,32.01246825)(99.39680914,32.36923904)(99.77180909,32.72340566)
\curveto(99.88118408,32.82236398)(99.98535073,32.89528064)(100.08430905,32.94215564)
\curveto(100.18326737,32.98903063)(100.34212152,33.03850979)(100.56087149,33.09059312)
\curveto(100.7848298,33.14267645)(101.01399644,33.16871811)(101.24837141,33.16871811)
\curveto(101.64420469,33.16871811)(102.02701715,33.08798895)(102.39680877,32.92653064)
\curveto(102.77180872,32.76507233)(103.05305869,32.57236402)(103.24055866,32.34840571)
\curveto(103.43326697,32.12965574)(103.56607946,31.8692391)(103.63899611,31.56715581)
\curveto(103.71191277,31.26507251)(103.7483711,30.89267672)(103.7483711,30.44996844)
\lineto(103.7483711,29.07496861)
\curveto(103.7483711,28.98642696)(103.76139193,28.33798954)(103.7874336,27.12965635)
\curveto(103.80305859,26.63486475)(103.85774609,26.34059395)(103.95149608,26.24684396)
\curveto(104.0504544,26.15309397)(104.37337102,26.10621898)(104.92024596,26.10621898)
\lineto(104.98274595,26.04371899)
\lineto(104.98274595,25.70778153)
\lineto(104.92024596,25.64528154)
\curveto(104.25878771,25.6869482)(103.81087109,25.70778153)(103.57649612,25.70778153)
\curveto(103.45670447,25.70778153)(103.07649618,25.6869482)(102.43587126,25.64528154)
\lineto(102.34212127,25.73121903)
\curveto(102.4098296,26.40830228)(102.44368376,27.29111467)(102.44368376,28.3796562)
\lineto(102.44368376,29.40309357)
\curveto(102.44368376,30.01767683)(102.42805876,30.46038511)(102.39680877,30.73121841)
\curveto(102.3707671,31.00205171)(102.28482962,31.24684335)(102.1389963,31.46559332)
\curveto(101.99316298,31.68955162)(101.79785051,31.8614266)(101.55305887,31.98121826)
\curveto(101.31347557,32.10621824)(101.02180894,32.16871823)(100.67805898,32.16871823)
\curveto(100.32389236,32.16871823)(100.02701739,32.11923907)(99.78743409,32.02028075)
\curveto(99.55305912,31.92653076)(99.31347581,31.76507245)(99.06868418,31.53590581)
\curveto(98.82389254,31.30673917)(98.67545506,31.11403086)(98.62337173,30.95778088)
\curveto(98.57649674,30.80673923)(98.55305924,30.51246844)(98.55305924,30.07496849)
\lineto(98.55305924,28.68434366)
\curveto(98.55305924,28.56976034)(98.56347591,28.19996872)(98.58430924,27.5749688)
\curveto(98.60514257,26.95517721)(98.6259759,26.59319809)(98.64680923,26.48903143)
\curveto(98.67285089,26.38486478)(98.70930922,26.30934395)(98.75618422,26.26246896)
\curveto(98.80305921,26.21559397)(98.86035087,26.18434397)(98.9280592,26.16871897)
\curveto(98.99576752,26.15830231)(99.27701749,26.13746898)(99.77180909,26.10621898)
\lineto(99.84212158,26.04371899)
\lineto(99.84212158,25.71559403)
\lineto(99.77962159,25.64528154)
\curveto(99.12858,25.6869482)(98.50097592,25.70778153)(97.89680932,25.70778153)
\curveto(97.29785106,25.70778153)(96.67285114,25.6869482)(96.02180955,25.64528154)
\lineto(95.95149706,25.71559403)
\lineto(95.95149706,26.04371899)
\lineto(96.02180955,26.10621898)
\curveto(96.52701783,26.13746898)(96.81087196,26.16090647)(96.87337195,26.17653147)
\curveto(96.94108027,26.19215647)(96.99837193,26.22340647)(97.04524693,26.27028146)
\curveto(97.09733025,26.32236479)(97.13118442,26.39788561)(97.14680942,26.49684393)
\curveto(97.16764275,26.60101059)(97.18847608,26.93694804)(97.20930941,27.50465631)
\curveto(97.23535107,28.0775729)(97.2483719,28.50986452)(97.2483719,28.80153115)
\lineto(97.2483719,32.84840565)
\lineto(97.21712191,34.50465544)
\curveto(97.20149691,35.08278037)(97.18326774,35.48382199)(97.16243441,35.7077803)
\curveto(97.14680942,35.9317386)(97.12337192,36.06455108)(97.09212192,36.10621775)
\curveto(97.06087193,36.14788441)(97.00618443,36.1791344)(96.92805944,36.19996773)
\curveto(96.84993445,36.22080107)(96.53483032,36.23121773)(95.98274706,36.23121773)
\lineto(95.91243457,36.30153022)
\lineto(95.91243457,36.62965518)
\lineto(95.97493456,36.69996767)
\curveto(96.9697261,36.81455099)(97.80826767,37.0072593)(98.49055925,37.2780926)
\closepath
}
}
{
\newrgbcolor{curcolor}{0 0 0}
\pscustom[linestyle=none,fillstyle=solid,fillcolor=curcolor]
{
\newpath
\moveto(112.05305758,26.85621889)
\lineto(111.80305761,26.29371896)
\curveto(111.26139101,25.94476067)(110.77701607,25.71819819)(110.34993279,25.61403154)
\curveto(109.92805784,25.50986489)(109.55045372,25.45778156)(109.21712043,25.45778156)
\curveto(108.5921205,25.45778156)(107.99837058,25.58017738)(107.43587065,25.82496901)
\curveto(106.87857905,26.06976065)(106.42284994,26.4838231)(106.06868332,27.06715636)
\curveto(105.71972503,27.65048962)(105.54524588,28.35361454)(105.54524588,29.1765311)
\curveto(105.54524588,29.72340603)(105.6129542,30.21559347)(105.74837085,30.65309342)
\curveto(105.8837875,31.0958017)(106.02441249,31.42392666)(106.1702458,31.6374683)
\curveto(106.32128745,31.85100994)(106.57389159,32.08798908)(106.92805821,32.34840571)
\curveto(107.28222483,32.60882234)(107.65722479,32.81715565)(108.05305807,32.97340563)
\curveto(108.44889136,33.12965561)(108.87597464,33.2077806)(109.33430791,33.2077806)
\curveto(109.95930784,33.2077806)(110.50878693,33.05934312)(110.98274521,32.76246816)
\curveto(111.46191182,32.47080153)(111.79784928,32.09580157)(111.99055759,31.6374683)
\curveto(112.18326589,31.17913502)(112.27962005,30.69215591)(112.27962005,30.17653098)
\curveto(112.27962005,30.01507266)(112.27180755,29.85882268)(112.25618255,29.70778104)
\lineto(112.17024506,29.62184355)
\curveto(111.81607844,29.54371856)(111.339516,29.49163523)(110.74055774,29.46559357)
\curveto(110.14159948,29.4395519)(109.7457662,29.42653107)(109.55305789,29.42653107)
\lineto(107.04524569,29.42653107)
\curveto(107.05566236,28.3484062)(107.32649566,27.55413547)(107.85774559,27.04371886)
\curveto(108.38899553,26.53330226)(109.04003712,26.27809396)(109.81087035,26.27809396)
\curveto(110.17545364,26.27809396)(110.52441193,26.34059395)(110.85774522,26.46559394)
\curveto(111.19628685,26.59059392)(111.55305764,26.76507306)(111.92805759,26.98903137)
\closepath
\moveto(107.04524569,30.05153099)
\curveto(107.13899568,30.03590599)(107.49837064,30.01767683)(108.12337056,29.9968435)
\curveto(108.75357882,29.97601017)(109.21972459,29.9655935)(109.52180789,29.9655935)
\curveto(110.24576613,29.9655935)(110.68587025,29.97861434)(110.84212023,30.004656)
\curveto(110.84732856,30.12965598)(110.84993273,30.22601014)(110.84993273,30.29371846)
\curveto(110.84993273,31.10101003)(110.68587025,31.69996829)(110.35774529,32.09059324)
\curveto(110.02962033,32.48642653)(109.58170372,32.68434317)(109.01399545,32.68434317)
\curveto(108.39420386,32.68434317)(107.90982892,32.46298903)(107.56087063,32.02028075)
\curveto(107.21712067,31.57757247)(107.04524569,30.92132255)(107.04524569,30.05153099)
\closepath
\moveto(109.22493293,33.64528055)
\closepath
\moveto(109.13118294,25.23903159)
\closepath
}
}
{
\newrgbcolor{curcolor}{0 0 0}
\pscustom[linestyle=none,fillstyle=solid,fillcolor=curcolor]
{
}
}
{
\newrgbcolor{curcolor}{0 0 0}
\pscustom[linestyle=none,fillstyle=solid,fillcolor=curcolor]
{
\newpath
\moveto(117.04524446,36.69215517)
\lineto(117.10774445,36.74684267)
\curveto(118.00878601,36.70517601)(118.62336927,36.68434268)(118.95149423,36.68434268)
\lineto(122.60774378,36.76246767)
\curveto(123.64941031,36.76246767)(124.50357688,36.66350934)(125.17024346,36.4655927)
\curveto(125.83691004,36.27288439)(126.4280558,35.93434277)(126.94368074,35.44996783)
\curveto(127.46451401,34.97080122)(127.85774313,34.41090546)(128.1233681,33.77028053)
\curveto(128.3942014,33.13486395)(128.52961805,32.45517653)(128.52961805,31.73121829)
\curveto(128.52961805,31.02809337)(128.40722223,30.33278096)(128.16243059,29.64528104)
\curveto(127.91763895,28.95778113)(127.56868066,28.3405937)(127.11555572,27.79371877)
\curveto(126.66763911,27.25205217)(126.15461834,26.81455223)(125.57649341,26.48121893)
\curveto(125.00357681,26.14788564)(124.42805605,25.92392734)(123.84993112,25.80934402)
\curveto(123.27180619,25.69996903)(122.68066043,25.64528154)(122.07649384,25.64528154)
\curveto(121.67024389,25.64528154)(120.90461899,25.66090653)(119.77961912,25.69215653)
\curveto(119.4098275,25.7025732)(119.16503587,25.70778153)(119.04524422,25.70778153)
\curveto(118.68586926,25.70778153)(118.25097348,25.6869482)(117.74055688,25.64528154)
\lineto(117.67805688,25.70778153)
\lineto(117.67805688,25.965594)
\lineto(117.74055688,26.04371899)
\curveto(118.0478485,26.19996897)(118.22232765,26.29632312)(118.26399431,26.33278145)
\curveto(118.31086931,26.37444811)(118.34732763,26.83017722)(118.3733693,27.69996878)
\curveto(118.40461929,28.57496868)(118.42024429,29.24163526)(118.42024429,29.69996854)
\lineto(118.42024429,32.56715568)
\lineto(118.41243179,34.34059296)
\curveto(118.41243179,34.72600958)(118.40722346,35.03850954)(118.3968068,35.27809285)
\curveto(118.38639013,35.52288449)(118.36816097,35.69736363)(118.3421193,35.80153028)
\curveto(118.31607764,35.90569694)(118.28482764,35.9786136)(118.24836931,36.02028026)
\curveto(118.21711932,36.06194692)(118.16243182,36.09580108)(118.08430683,36.12184274)
\curveto(118.01139018,36.15309274)(117.88378603,36.1791344)(117.70149438,36.19996773)
\lineto(117.10774445,36.24684273)
\lineto(117.04524446,36.30153022)
\closepath
\moveto(119.93586911,26.49684393)
\curveto(120.46711904,26.39267728)(121.07909813,26.34059395)(121.77180638,26.34059395)
\curveto(122.9853479,26.34059395)(123.95409778,26.54371893)(124.67805602,26.94996888)
\curveto(125.4072226,27.35621883)(125.95149336,27.95778125)(126.31086832,28.75465615)
\curveto(126.67024328,29.55153105)(126.84993075,30.44996844)(126.84993075,31.44996832)
\curveto(126.84993075,32.92392647)(126.45670163,34.0671555)(125.6702434,34.8796554)
\curveto(124.88378516,35.6921553)(123.64420198,36.09840525)(121.95149386,36.09840525)
\curveto(121.25357728,36.09840525)(120.58170236,36.04632192)(119.93586911,35.94215527)
\curveto(119.89420244,35.08798871)(119.87336911,34.12444716)(119.87336911,33.05153062)
\lineto(119.87336911,30.59840593)
\lineto(119.88899411,28.30153121)
\curveto(119.88899411,27.8536146)(119.90461911,27.25205217)(119.93586911,26.49684393)
\closepath
}
}
{
\newrgbcolor{curcolor}{0 0 0}
\pscustom[linestyle=none,fillstyle=solid,fillcolor=curcolor]
{
\newpath
\moveto(130.84993026,31.00465588)
\lineto(130.5452428,31.08278087)
\lineto(130.48274281,31.16090586)
\lineto(130.48274281,32.12965574)
\curveto(131.40982602,32.81715565)(132.31086758,33.16090561)(133.18586747,33.16090561)
\curveto(133.78482573,33.16090561)(134.2822215,33.04892646)(134.67805479,32.82496815)
\curveto(135.07388807,32.60100985)(135.36034637,32.31975988)(135.53742968,31.98121826)
\curveto(135.71451299,31.64788496)(135.80305465,31.25726001)(135.80305465,30.8093434)
\lineto(135.76399215,29.23903109)
\lineto(135.76399215,26.91090638)
\curveto(135.76399215,26.59319809)(135.78482548,26.40048978)(135.82649215,26.33278145)
\curveto(135.87336714,26.26507313)(135.92545047,26.21819813)(135.98274213,26.19215647)
\curveto(136.04003379,26.17132314)(136.14680461,26.15309397)(136.30305459,26.13746898)
\lineto(136.74836703,26.09840648)
\lineto(136.81086702,26.02809399)
\lineto(136.81086702,25.71559403)
\lineto(136.74836703,25.65309404)
\curveto(136.36815875,25.68434403)(136.01399212,25.69996903)(135.68586716,25.69996903)
\curveto(135.3733672,25.69996903)(134.99836725,25.68434403)(134.5608673,25.65309404)
\lineto(134.44367982,25.76246902)
\lineto(134.47492981,27.02028137)
\lineto(132.77180502,25.69215653)
\curveto(132.48534672,25.57236488)(132.17284676,25.51246905)(131.83430514,25.51246905)
\curveto(131.41763852,25.51246905)(131.05826357,25.58798988)(130.75618027,25.73903152)
\curveto(130.45930531,25.89007317)(130.23013867,26.10101065)(130.06868036,26.37184395)
\curveto(129.91243038,26.64267725)(129.83430538,26.97080221)(129.83430538,27.35621883)
\curveto(129.83430538,28.12184373)(130.07388869,28.71819782)(130.5530553,29.1452811)
\curveto(131.0322219,29.57757272)(132.33951341,29.94476017)(134.47492981,30.24684347)
\curveto(134.47492981,31.01767671)(134.30305483,31.55934331)(133.95930488,31.87184327)
\curveto(133.61555492,32.18955156)(133.15461748,32.34840571)(132.57649255,32.34840571)
\curveto(132.27440925,32.34840571)(131.99836762,32.30413488)(131.74836765,32.21559323)
\curveto(131.50357601,32.12705157)(131.36295103,32.05413491)(131.3264927,31.99684325)
\curveto(131.29003437,31.94475993)(131.14940939,31.62705163)(130.90461775,31.04371837)
\closepath
\moveto(134.47492981,29.77028103)
\curveto(133.01659666,29.52548939)(132.1077426,29.26246859)(131.74836765,28.98121862)
\curveto(131.38899269,28.69996866)(131.20930522,28.26507288)(131.20930522,27.67653129)
\curveto(131.20930522,26.86403139)(131.612951,26.45778144)(132.42024257,26.45778144)
\curveto(133.11295081,26.45778144)(133.79784656,26.85882305)(134.47492981,27.66090629)
\closepath
\moveto(133.14680498,33.64528055)
\closepath
\moveto(132.9827425,25.23903159)
\closepath
}
}
{
\newrgbcolor{curcolor}{0 0 0}
\pscustom[linestyle=none,fillstyle=solid,fillcolor=curcolor]
{
\newpath
\moveto(137.55305443,31.84059327)
\lineto(137.55305443,32.04371825)
\lineto(137.60774193,32.12184324)
\curveto(138.0973252,32.30413488)(138.49315848,32.47340569)(138.79524178,32.62965568)
\curveto(138.79524178,33.98903051)(138.77961678,34.78069708)(138.74836679,35.00465538)
\curveto(139.28482505,35.19215536)(139.72492917,35.39267617)(140.06867912,35.60621781)
\lineto(140.2561791,35.44996783)
\curveto(140.20409577,35.12184287)(140.14159578,34.16871799)(140.06867912,32.59059318)
\curveto(140.32909576,32.58538485)(140.61034572,32.58278068)(140.91242902,32.58278068)
\curveto(141.52701228,32.58278068)(141.96711639,32.59840568)(142.23274136,32.62965568)
\lineto(142.28742885,32.57496818)
\lineto(142.13899137,31.91871826)
\lineto(142.07649138,31.84840577)
\curveto(141.81086641,31.8536141)(141.51659561,31.85621827)(141.19367898,31.85621827)
\curveto(140.90201235,31.85621827)(140.5270124,31.8536141)(140.06867912,31.84840577)
\lineto(140.02180413,28.66871866)
\curveto(140.02180413,27.93434375)(140.03742913,27.45257298)(140.06867912,27.22340634)
\curveto(140.10513745,26.99944804)(140.19888744,26.82236472)(140.34992909,26.69215641)
\curveto(140.50617907,26.56715642)(140.73534571,26.50465643)(141.037429,26.50465643)
\curveto(141.38638729,26.50465643)(141.70930392,26.59580225)(142.00617888,26.7780939)
\lineto(142.19367886,26.49684393)
\curveto(142.06867888,26.40830228)(141.76919975,26.14788564)(141.29524147,25.71559403)
\curveto(141.02440817,25.59059404)(140.74836654,25.52809405)(140.46711657,25.52809405)
\curveto(139.29524172,25.52809405)(138.70930429,26.09580231)(138.70930429,27.23121884)
\curveto(138.70930429,27.64788546)(138.71972096,28.00205208)(138.74055429,28.29371871)
\curveto(138.74576262,28.38226037)(138.74836679,28.47340619)(138.74836679,28.56715618)
\lineto(138.74836679,31.80934328)
\lineto(138.42805433,31.80934328)
\curveto(138.19367935,31.80934328)(137.92545022,31.79892661)(137.62336692,31.77809328)
\closepath
}
}
{
\newrgbcolor{curcolor}{0 0 0}
\pscustom[linestyle=none,fillstyle=solid,fillcolor=curcolor]
{
\newpath
\moveto(149.31867798,26.85621889)
\lineto(149.06867801,26.29371896)
\curveto(148.52701141,25.94476067)(148.04263647,25.71819819)(147.61555319,25.61403154)
\curveto(147.19367824,25.50986489)(146.81607412,25.45778156)(146.48274083,25.45778156)
\curveto(145.85774091,25.45778156)(145.26399098,25.58017738)(144.70149105,25.82496901)
\curveto(144.14419945,26.06976065)(143.68847034,26.4838231)(143.33430372,27.06715636)
\curveto(142.98534543,27.65048962)(142.81086628,28.35361454)(142.81086628,29.1765311)
\curveto(142.81086628,29.72340603)(142.87857461,30.21559347)(143.01399126,30.65309342)
\curveto(143.14940791,31.0958017)(143.29003289,31.42392666)(143.43586621,31.6374683)
\curveto(143.58690786,31.85100994)(143.83951199,32.08798908)(144.19367861,32.34840571)
\curveto(144.54784524,32.60882234)(144.92284519,32.81715565)(145.31867848,32.97340563)
\curveto(145.71451176,33.12965561)(146.14159504,33.2077806)(146.59992832,33.2077806)
\curveto(147.22492824,33.2077806)(147.77440734,33.05934312)(148.24836561,32.76246816)
\curveto(148.72753222,32.47080153)(149.06346968,32.09580157)(149.25617799,31.6374683)
\curveto(149.4488863,31.17913502)(149.54524045,30.69215591)(149.54524045,30.17653098)
\curveto(149.54524045,30.01507266)(149.53742796,29.85882268)(149.52180296,29.70778104)
\lineto(149.43586547,29.62184355)
\curveto(149.08169885,29.54371856)(148.6051364,29.49163523)(148.00617814,29.46559357)
\curveto(147.40721988,29.4395519)(147.0113866,29.42653107)(146.81867829,29.42653107)
\lineto(144.3108661,29.42653107)
\curveto(144.32128277,28.3484062)(144.59211607,27.55413547)(145.123366,27.04371886)
\curveto(145.65461593,26.53330226)(146.30565752,26.27809396)(147.07649076,26.27809396)
\curveto(147.44107405,26.27809396)(147.79003234,26.34059395)(148.12336563,26.46559394)
\curveto(148.46190725,26.59059392)(148.81867804,26.76507306)(149.193678,26.98903137)
\closepath
\moveto(144.3108661,30.05153099)
\curveto(144.40461609,30.03590599)(144.76399104,30.01767683)(145.38899097,29.9968435)
\curveto(146.01919922,29.97601017)(146.485345,29.9655935)(146.78742829,29.9655935)
\curveto(147.51138654,29.9655935)(147.95149065,29.97861434)(148.10774063,30.004656)
\curveto(148.11294896,30.12965598)(148.11555313,30.22601014)(148.11555313,30.29371846)
\curveto(148.11555313,31.10101003)(147.95149065,31.69996829)(147.62336569,32.09059324)
\curveto(147.29524073,32.48642653)(146.84732412,32.68434317)(146.27961586,32.68434317)
\curveto(145.65982427,32.68434317)(145.17544933,32.46298903)(144.82649104,32.02028075)
\curveto(144.48274108,31.57757247)(144.3108661,30.92132255)(144.3108661,30.05153099)
\closepath
\moveto(146.49055333,33.64528055)
\closepath
\moveto(146.39680334,25.23903159)
\closepath
}
}
{
\newrgbcolor{curcolor}{0 0 0}
\pscustom[linestyle=none,fillstyle=solid,fillcolor=curcolor]
{
\newpath
\moveto(152.74836506,33.15309311)
\lineto(152.89680254,33.05153062)
\curveto(152.86555255,32.70257233)(152.84471921,32.24423906)(152.83430255,31.67653079)
\curveto(153.25096916,32.02028075)(153.64680245,32.36923904)(154.0218024,32.72340566)
\curveto(154.13117739,32.82236398)(154.23273988,32.89528064)(154.32648986,32.94215564)
\curveto(154.42544819,32.98903063)(154.5869065,33.03850979)(154.81086481,33.09059312)
\curveto(155.03482311,33.14267645)(155.26398975,33.16871811)(155.49836472,33.16871811)
\curveto(155.89419801,33.16871811)(156.27701046,33.08798895)(156.64680208,32.92653064)
\curveto(157.02180203,32.76507233)(157.303052,32.57236402)(157.49055197,32.34840571)
\curveto(157.68326028,32.12965574)(157.81607277,31.8692391)(157.88898943,31.56715581)
\curveto(157.96190608,31.26507251)(157.99836441,30.89267672)(157.99836441,30.44996844)
\lineto(157.99836441,29.07496861)
\curveto(157.99836441,28.98642696)(158.01138524,28.33798954)(158.03742691,27.12965635)
\curveto(158.04784357,26.63486475)(158.10253107,26.34059395)(158.20148939,26.24684396)
\curveto(158.30044771,26.15309397)(158.62336434,26.10621898)(159.17023927,26.10621898)
\lineto(159.23273926,26.04371899)
\lineto(159.23273926,25.70778153)
\lineto(159.17023927,25.64528154)
\curveto(158.50878102,25.6869482)(158.0608644,25.70778153)(157.82648943,25.70778153)
\curveto(157.69107278,25.70778153)(157.3108645,25.6869482)(156.68586457,25.64528154)
\lineto(156.59211459,25.73121903)
\curveto(156.65982291,26.40830228)(156.69367707,27.29111467)(156.69367707,28.3796562)
\lineto(156.69367707,29.40309357)
\curveto(156.69367707,30.01767683)(156.67805208,30.46038511)(156.64680208,30.73121841)
\curveto(156.62076042,31.00205171)(156.53482293,31.24684335)(156.38898961,31.46559332)
\curveto(156.2431563,31.68955162)(156.04784382,31.8614266)(155.80305218,31.98121826)
\curveto(155.55826055,32.10621824)(155.26659392,32.16871823)(154.92805229,32.16871823)
\curveto(154.65721899,32.16871823)(154.42544819,32.1400724)(154.23273988,32.08278074)
\curveto(154.04003157,32.03069742)(153.82388576,31.91871826)(153.58430246,31.74684328)
\curveto(153.34471915,31.58017664)(153.16763584,31.42132249)(153.05305252,31.27028084)
\curveto(152.9384692,31.12444753)(152.86815671,30.98642671)(152.84211505,30.85621839)
\curveto(152.82128172,30.73121841)(152.81086505,30.46038511)(152.81086505,30.04371849)
\lineto(152.81086505,28.68434366)
\curveto(152.81086505,28.56976034)(152.82128172,28.19996872)(152.84211505,27.5749688)
\curveto(152.86294838,26.95517721)(152.88378171,26.59319809)(152.90461504,26.48903143)
\curveto(152.9306567,26.38486478)(152.96711503,26.30934395)(153.01399003,26.26246896)
\curveto(153.06086502,26.21559397)(153.11815668,26.18434397)(153.18586501,26.16871897)
\curveto(153.25357333,26.15830231)(153.5348233,26.13746898)(154.0296149,26.10621898)
\lineto(154.09992739,26.04371899)
\lineto(154.09992739,25.71559403)
\lineto(154.0374274,25.64528154)
\curveto(153.38638581,25.6869482)(152.75878172,25.70778153)(152.15461513,25.70778153)
\curveto(151.55565687,25.70778153)(150.93065695,25.6869482)(150.27961536,25.64528154)
\lineto(150.20930287,25.71559403)
\lineto(150.20930287,26.04371899)
\lineto(150.27961536,26.10621898)
\curveto(150.78482364,26.13746898)(151.06867777,26.16090647)(151.13117776,26.17653147)
\curveto(151.19888608,26.19215647)(151.25617774,26.22340647)(151.30305274,26.27028146)
\curveto(151.35513606,26.32236479)(151.38899023,26.39788561)(151.40461523,26.49684393)
\curveto(151.42544856,26.60101059)(151.44628189,26.93694804)(151.46711522,27.50465631)
\curveto(151.49315688,28.0775729)(151.50617771,28.50986452)(151.50617771,28.80153115)
\lineto(151.50617771,30.34059346)
\curveto(151.50617771,30.54892677)(151.49576105,30.8327809)(151.47492772,31.19215585)
\curveto(151.45409439,31.55153081)(151.43586522,31.77288495)(151.42024022,31.85621827)
\curveto(151.40982356,31.93955159)(151.37076106,31.99944742)(151.30305274,32.03590575)
\curveto(151.23534441,32.07757241)(151.09992776,32.09840574)(150.89680279,32.09840574)
\lineto(150.24055287,32.10621824)
\lineto(150.17024038,32.16871823)
\lineto(150.17024038,32.50465569)
\lineto(150.23274037,32.56715568)
\curveto(151.22753191,32.68694733)(152.06607348,32.88225981)(152.74836506,33.15309311)
\closepath
}
}
{
\newrgbcolor{curcolor}{0 0 0}
\pscustom[linestyle=none,fillstyle=solid,fillcolor=curcolor]
{
\newpath
\moveto(161.04523904,31.00465588)
\lineto(160.74055157,31.08278087)
\lineto(160.67805158,31.16090586)
\lineto(160.67805158,32.12965574)
\curveto(161.6051348,32.81715565)(162.50617636,33.16090561)(163.38117625,33.16090561)
\curveto(163.98013451,33.16090561)(164.47753028,33.04892646)(164.87336356,32.82496815)
\curveto(165.26919685,32.60100985)(165.55565515,32.31975988)(165.73273846,31.98121826)
\curveto(165.90982177,31.64788496)(165.99836343,31.25726001)(165.99836343,30.8093434)
\lineto(165.95930093,29.23903109)
\lineto(165.95930093,26.91090638)
\curveto(165.95930093,26.59319809)(165.98013426,26.40048978)(166.02180092,26.33278145)
\curveto(166.06867592,26.26507313)(166.12075924,26.21819813)(166.1780509,26.19215647)
\curveto(166.23534256,26.17132314)(166.34211338,26.15309397)(166.49836336,26.13746898)
\lineto(166.94367581,26.09840648)
\lineto(167.0061758,26.02809399)
\lineto(167.0061758,25.71559403)
\lineto(166.94367581,25.65309404)
\curveto(166.56346752,25.68434403)(166.2093009,25.69996903)(165.88117594,25.69996903)
\curveto(165.56867598,25.69996903)(165.19367603,25.68434403)(164.75617608,25.65309404)
\lineto(164.63898859,25.76246902)
\lineto(164.67023859,27.02028137)
\lineto(162.9671138,25.69215653)
\curveto(162.6806555,25.57236488)(162.36815554,25.51246905)(162.02961392,25.51246905)
\curveto(161.6129473,25.51246905)(161.25357234,25.58798988)(160.95148905,25.73903152)
\curveto(160.65461408,25.89007317)(160.42544745,26.10101065)(160.26398913,26.37184395)
\curveto(160.10773915,26.64267725)(160.02961416,26.97080221)(160.02961416,27.35621883)
\curveto(160.02961416,28.12184373)(160.26919747,28.71819782)(160.74836407,29.1452811)
\curveto(161.22753068,29.57757272)(162.53482219,29.94476017)(164.67023859,30.24684347)
\curveto(164.67023859,31.01767671)(164.49836361,31.55934331)(164.15461365,31.87184327)
\curveto(163.8108637,32.18955156)(163.34992625,32.34840571)(162.77180132,32.34840571)
\curveto(162.46971803,32.34840571)(162.19367639,32.30413488)(161.94367643,32.21559323)
\curveto(161.69888479,32.12705157)(161.55825981,32.05413491)(161.52180148,31.99684325)
\curveto(161.48534315,31.94475993)(161.34471817,31.62705163)(161.09992653,31.04371837)
\closepath
\moveto(164.67023859,29.77028103)
\curveto(163.21190544,29.52548939)(162.30305138,29.26246859)(161.94367643,28.98121862)
\curveto(161.58430147,28.69996866)(161.40461399,28.26507288)(161.40461399,27.67653129)
\curveto(161.40461399,26.86403139)(161.80825978,26.45778144)(162.61555134,26.45778144)
\curveto(163.30825959,26.45778144)(163.99315534,26.85882305)(164.67023859,27.66090629)
\closepath
\moveto(163.34211375,33.64528055)
\closepath
\moveto(163.17805127,25.23903159)
\closepath
}
}
{
\newrgbcolor{curcolor}{0 0 0}
\pscustom[linestyle=none,fillstyle=solid,fillcolor=curcolor]
{
\newpath
\moveto(170.06086292,33.15309311)
\lineto(170.20930041,33.05153062)
\curveto(170.17805041,32.70257233)(170.15721708,32.24423906)(170.14680041,31.67653079)
\curveto(170.56346703,32.02028075)(170.95930031,32.36923904)(171.33430027,32.72340566)
\curveto(171.44367525,32.82236398)(171.54523774,32.89528064)(171.63898773,32.94215564)
\curveto(171.73794605,32.98903063)(171.89940436,33.03850979)(172.12336267,33.09059312)
\curveto(172.34732098,33.14267645)(172.57648761,33.16871811)(172.81086259,33.16871811)
\curveto(173.20669587,33.16871811)(173.58950832,33.08798895)(173.95929994,32.92653064)
\curveto(174.3342999,32.76507233)(174.61554986,32.57236402)(174.80304984,32.34840571)
\curveto(174.99575815,32.12965574)(175.12857063,31.8692391)(175.20148729,31.56715581)
\curveto(175.27440395,31.26507251)(175.31086228,30.89267672)(175.31086228,30.44996844)
\lineto(175.31086228,29.07496861)
\curveto(175.31086228,28.98642696)(175.32388311,28.33798954)(175.34992477,27.12965635)
\curveto(175.36034144,26.63486475)(175.41502893,26.34059395)(175.51398725,26.24684396)
\curveto(175.61294557,26.15309397)(175.9358622,26.10621898)(176.48273713,26.10621898)
\lineto(176.54523713,26.04371899)
\lineto(176.54523713,25.70778153)
\lineto(176.48273713,25.64528154)
\curveto(175.82127888,25.6869482)(175.37336227,25.70778153)(175.1389873,25.70778153)
\curveto(175.00357065,25.70778153)(174.62336236,25.6869482)(173.99836244,25.64528154)
\lineto(173.90461245,25.73121903)
\curveto(173.97232078,26.40830228)(174.00617494,27.29111467)(174.00617494,28.3796562)
\lineto(174.00617494,29.40309357)
\curveto(174.00617494,30.01767683)(173.99054994,30.46038511)(173.95929994,30.73121841)
\curveto(173.93325828,31.00205171)(173.84732079,31.24684335)(173.70148748,31.46559332)
\curveto(173.55565416,31.68955162)(173.36034168,31.8614266)(173.11555005,31.98121826)
\curveto(172.87075841,32.10621824)(172.57909178,32.16871823)(172.24055016,32.16871823)
\curveto(171.96971686,32.16871823)(171.73794605,32.1400724)(171.54523774,32.08278074)
\curveto(171.35252943,32.03069742)(171.13638363,31.91871826)(170.89680032,31.74684328)
\curveto(170.65721702,31.58017664)(170.48013371,31.42132249)(170.36555039,31.27028084)
\curveto(170.25096707,31.12444753)(170.18065458,30.98642671)(170.15461291,30.85621839)
\curveto(170.13377958,30.73121841)(170.12336292,30.46038511)(170.12336292,30.04371849)
\lineto(170.12336292,28.68434366)
\curveto(170.12336292,28.56976034)(170.13377958,28.19996872)(170.15461291,27.5749688)
\curveto(170.17544624,26.95517721)(170.19627957,26.59319809)(170.21711291,26.48903143)
\curveto(170.24315457,26.38486478)(170.2796129,26.30934395)(170.32648789,26.26246896)
\curveto(170.37336289,26.21559397)(170.43065455,26.18434397)(170.49836287,26.16871897)
\curveto(170.5660712,26.15830231)(170.84732116,26.13746898)(171.34211277,26.10621898)
\lineto(171.41242526,26.04371899)
\lineto(171.41242526,25.71559403)
\lineto(171.34992527,25.64528154)
\curveto(170.69888368,25.6869482)(170.07127959,25.70778153)(169.467113,25.70778153)
\curveto(168.86815474,25.70778153)(168.24315482,25.6869482)(167.59211323,25.64528154)
\lineto(167.52180074,25.71559403)
\lineto(167.52180074,26.04371899)
\lineto(167.59211323,26.10621898)
\curveto(168.0973215,26.13746898)(168.38117563,26.16090647)(168.44367562,26.17653147)
\curveto(168.51138395,26.19215647)(168.56867561,26.22340647)(168.6155506,26.27028146)
\curveto(168.66763393,26.32236479)(168.70148809,26.39788561)(168.71711309,26.49684393)
\curveto(168.73794642,26.60101059)(168.75877975,26.93694804)(168.77961308,27.50465631)
\curveto(168.80565475,28.0775729)(168.81867558,28.50986452)(168.81867558,28.80153115)
\lineto(168.81867558,30.34059346)
\curveto(168.81867558,30.54892677)(168.80825891,30.8327809)(168.78742558,31.19215585)
\curveto(168.76659225,31.55153081)(168.74836309,31.77288495)(168.73273809,31.85621827)
\curveto(168.72232142,31.93955159)(168.68325893,31.99944742)(168.6155506,32.03590575)
\curveto(168.54784228,32.07757241)(168.41242563,32.09840574)(168.20930065,32.09840574)
\lineto(167.55305073,32.10621824)
\lineto(167.48273824,32.16871823)
\lineto(167.48273824,32.50465569)
\lineto(167.54523824,32.56715568)
\curveto(168.54002978,32.68694733)(169.37857134,32.88225981)(170.06086292,33.15309311)
\closepath
}
}
{
\newrgbcolor{curcolor}{0 0 0}
\pscustom[linestyle=none,fillstyle=solid,fillcolor=curcolor]
{
\newpath
\moveto(178.3577369,31.00465588)
\lineto(178.05304944,31.08278087)
\lineto(177.99054945,31.16090586)
\lineto(177.99054945,32.12965574)
\curveto(178.91763267,32.81715565)(179.81867422,33.16090561)(180.69367411,33.16090561)
\curveto(181.29263237,33.16090561)(181.79002815,33.04892646)(182.18586143,32.82496815)
\curveto(182.58169471,32.60100985)(182.86815301,32.31975988)(183.04523632,31.98121826)
\curveto(183.22231964,31.64788496)(183.31086129,31.25726001)(183.31086129,30.8093434)
\lineto(183.2717988,29.23903109)
\lineto(183.2717988,26.91090638)
\curveto(183.2717988,26.59319809)(183.29263213,26.40048978)(183.33429879,26.33278145)
\curveto(183.38117378,26.26507313)(183.43325711,26.21819813)(183.49054877,26.19215647)
\curveto(183.54784043,26.17132314)(183.65461125,26.15309397)(183.81086123,26.13746898)
\lineto(184.25617367,26.09840648)
\lineto(184.31867367,26.02809399)
\lineto(184.31867367,25.71559403)
\lineto(184.25617367,25.65309404)
\curveto(183.87596539,25.68434403)(183.52179877,25.69996903)(183.19367381,25.69996903)
\curveto(182.88117384,25.69996903)(182.50617389,25.68434403)(182.06867394,25.65309404)
\lineto(181.95148646,25.76246902)
\lineto(181.98273645,27.02028137)
\lineto(180.27961166,25.69215653)
\curveto(179.99315337,25.57236488)(179.68065341,25.51246905)(179.34211178,25.51246905)
\curveto(178.92544517,25.51246905)(178.56607021,25.58798988)(178.26398691,25.73903152)
\curveto(177.96711195,25.89007317)(177.73794531,26.10101065)(177.576487,26.37184395)
\curveto(177.42023702,26.64267725)(177.34211203,26.97080221)(177.34211203,27.35621883)
\curveto(177.34211203,28.12184373)(177.58169533,28.71819782)(178.06086194,29.1452811)
\curveto(178.54002855,29.57757272)(179.84732005,29.94476017)(181.98273645,30.24684347)
\curveto(181.98273645,31.01767671)(181.81086148,31.55934331)(181.46711152,31.87184327)
\curveto(181.12336156,32.18955156)(180.66242412,32.34840571)(180.08429919,32.34840571)
\curveto(179.78221589,32.34840571)(179.50617426,32.30413488)(179.25617429,32.21559323)
\curveto(179.01138265,32.12705157)(178.87075767,32.05413491)(178.83429934,31.99684325)
\curveto(178.79784101,31.94475993)(178.65721603,31.62705163)(178.4124244,31.04371837)
\closepath
\moveto(181.98273645,29.77028103)
\curveto(180.5244033,29.52548939)(179.61554925,29.26246859)(179.25617429,28.98121862)
\curveto(178.89679934,28.69996866)(178.71711186,28.26507288)(178.71711186,27.67653129)
\curveto(178.71711186,26.86403139)(179.12075764,26.45778144)(179.92804921,26.45778144)
\curveto(180.62075746,26.45778144)(181.30565321,26.85882305)(181.98273645,27.66090629)
\closepath
\moveto(180.65461162,33.64528055)
\closepath
\moveto(180.49054914,25.23903159)
\closepath
}
}
{
\newrgbcolor{curcolor}{0 0 0}
\pscustom[linestyle=none,fillstyle=solid,fillcolor=curcolor]
{
\newpath
\moveto(187.62336076,37.2780926)
\lineto(187.77179824,37.18434261)
\curveto(187.71450658,36.58538435)(187.68586075,35.38486367)(187.68586075,33.58278056)
\lineto(187.68586075,28.68434366)
\curveto(187.68586075,28.56976034)(187.69627742,28.19996872)(187.71711075,27.5749688)
\curveto(187.73794408,26.95517721)(187.75877741,26.59319809)(187.77961074,26.48903143)
\curveto(187.8056524,26.38486478)(187.84211073,26.30934395)(187.88898573,26.26246896)
\curveto(187.93586072,26.21559397)(187.99315238,26.18434397)(188.06086071,26.16871897)
\curveto(188.12856903,26.15830231)(188.409819,26.13746898)(188.9046106,26.10621898)
\lineto(188.97492309,26.04371899)
\lineto(188.97492309,25.71559403)
\lineto(188.9124231,25.64528154)
\curveto(188.26138151,25.6869482)(187.63377742,25.70778153)(187.02961083,25.70778153)
\curveto(186.43065257,25.70778153)(185.80565265,25.6869482)(185.15461106,25.64528154)
\lineto(185.08429857,25.71559403)
\lineto(185.08429857,26.04371899)
\lineto(185.15461106,26.10621898)
\curveto(185.65981933,26.13746898)(185.94367347,26.16090647)(186.00617346,26.17653147)
\curveto(186.07388178,26.19215647)(186.13117344,26.22340647)(186.17804844,26.27028146)
\curveto(186.23013176,26.32236479)(186.26398593,26.39788561)(186.27961093,26.49684393)
\curveto(186.30044426,26.60101059)(186.32127759,26.93694804)(186.34211092,27.50465631)
\curveto(186.36815258,28.0775729)(186.38117341,28.50986452)(186.38117341,28.80153115)
\lineto(186.38117341,32.84840565)
\lineto(186.34992342,34.50465544)
\curveto(186.33429842,35.08278037)(186.31606925,35.48382199)(186.29523592,35.7077803)
\curveto(186.27961093,35.9317386)(186.25617343,36.06455108)(186.22492343,36.10621775)
\curveto(186.19367344,36.14788441)(186.13898594,36.1791344)(186.06086095,36.19996773)
\curveto(185.98273596,36.22080107)(185.66763183,36.23121773)(185.11554857,36.23121773)
\lineto(185.04523608,36.30153022)
\lineto(185.04523608,36.62965518)
\lineto(185.10773607,36.69996767)
\curveto(186.10252761,36.81455099)(186.94106918,37.0072593)(187.62336076,37.2780926)
\closepath
}
}
{
\newrgbcolor{curcolor}{0 0 0}
\pscustom[linestyle=none,fillstyle=solid,fillcolor=curcolor]
{
\newpath
\moveto(189.709298,21.46559455)
\curveto(189.88638131,21.9030945)(190.0191938,22.30413611)(190.10773545,22.6687194)
\lineto(190.29523543,22.6687194)
\curveto(190.55044373,22.39267777)(190.82648536,22.25465695)(191.12336033,22.25465695)
\curveto(191.49315195,22.25465695)(191.81346441,22.43955276)(192.08429771,22.80934439)
\curveto(192.36033934,23.17392767)(192.74836013,23.98382341)(193.24836007,25.23903159)
\curveto(193.05044342,25.79632318)(192.84992261,26.32236479)(192.64679764,26.81715639)
\lineto(191.40461029,29.89528101)
\curveto(191.27440198,30.21298931)(191.08169367,30.66611425)(190.82648536,31.25465584)
\curveto(190.57127706,31.84840577)(190.41763125,32.17392656)(190.36554792,32.23121822)
\curveto(190.31867293,32.29371822)(190.25617293,32.35100988)(190.17804794,32.4030932)
\curveto(190.09992295,32.45517653)(189.88898548,32.50465569)(189.54523552,32.55153068)
\lineto(189.48273553,32.61403068)
\lineto(189.48273553,32.92653064)
\lineto(189.55304802,32.98903063)
\curveto(190.15200628,32.9525723)(190.76658954,32.93434314)(191.39679779,32.93434314)
\curveto(192.13638104,32.93434314)(192.70669347,32.9525723)(193.10773508,32.98903063)
\lineto(193.17023508,32.92653064)
\lineto(193.17023508,32.61403068)
\lineto(193.11554758,32.55153068)
\curveto(193.05825592,32.54632235)(192.89679761,32.53330152)(192.63117264,32.51246819)
\curveto(192.37075601,32.49684319)(192.20669353,32.45517653)(192.1389852,32.3874682)
\curveto(192.07648521,32.32496821)(192.04523521,32.25205156)(192.04523521,32.16871823)
\curveto(192.04523521,32.08017658)(192.12596437,31.79111411)(192.28742268,31.30153084)
\curveto(192.448881,30.8171559)(192.60513098,30.38746845)(192.75617263,30.0124685)
\lineto(193.24836007,28.79371865)
\curveto(193.57127669,27.99163541)(193.82127666,27.40048965)(193.99835997,27.02028137)
\lineto(194.41242242,27.93434375)
\curveto(194.55825574,28.26767705)(194.76658905,28.77288532)(195.03742235,29.44996857)
\lineto(195.68585977,31.10621836)
\curveto(195.85252641,31.53330164)(195.9462764,31.79892661)(195.96710973,31.90309326)
\curveto(195.99315139,32.00725992)(196.00617223,32.08278074)(196.00617223,32.12965574)
\curveto(196.00617223,32.24423906)(195.96450556,32.32236405)(195.88117224,32.36403071)
\curveto(195.79783892,32.40569737)(195.61294311,32.4421557)(195.32648481,32.47340569)
\lineto(194.99835985,32.52028069)
\lineto(194.94367236,32.58278068)
\lineto(194.94367236,32.92653064)
\lineto(195.00617235,32.98903063)
\curveto(195.4436723,32.9525723)(195.9619014,32.93434314)(196.56085966,32.93434314)
\curveto(197.10773459,32.93434314)(197.54523454,32.9525723)(197.8733595,32.98903063)
\lineto(197.93585949,32.92653064)
\lineto(197.93585949,32.58278068)
\lineto(197.8733595,32.52028069)
\curveto(197.68065119,32.51507236)(197.52700537,32.48903069)(197.41242205,32.4421557)
\curveto(197.29783873,32.3952807)(197.18846375,32.31194738)(197.08429709,32.19215573)
\curveto(196.98013044,32.07236408)(196.79263046,31.73382245)(196.52179716,31.17653085)
\curveto(196.25096386,30.62444759)(195.98013056,30.05153099)(195.70929726,29.45778107)
\lineto(194.94367236,27.77809377)
\lineto(193.13898508,23.55934429)
\curveto(192.9827351,23.19476101)(192.78742262,22.80413605)(192.55304765,22.38746944)
\curveto(192.32388101,21.97080282)(192.04002688,21.67132369)(191.70148526,21.48903205)
\curveto(191.36815196,21.30153207)(191.01398534,21.20778208)(190.63898539,21.20778208)
\curveto(190.31086043,21.20778208)(190.00096463,21.29371957)(189.709298,21.46559455)
\closepath
\moveto(193.88898499,33.64528055)
\closepath
\moveto(195.72492226,25.23903159)
\closepath
}
}
{
\newrgbcolor{curcolor}{0 0 0}
\pscustom[linestyle=none,fillstyle=solid,fillcolor=curcolor]
{
\newpath
\moveto(199.04523435,28.00465625)
\lineto(199.38117181,28.00465625)
\lineto(199.4514843,27.93434375)
\curveto(199.46190097,27.49684381)(199.48794263,27.12184385)(199.52960929,26.80934389)
\curveto(199.6910676,26.56455226)(199.9697134,26.36403145)(200.36554669,26.20778147)
\curveto(200.76137997,26.05673982)(201.15200492,25.98121899)(201.53742154,25.98121899)
\curveto(202.08950481,25.98121899)(202.52960892,26.12965648)(202.85773388,26.42653144)
\curveto(203.19106717,26.7234064)(203.35773382,27.07757303)(203.35773382,27.48903131)
\curveto(203.35773382,27.71298961)(203.29783799,27.90569792)(203.17804634,28.06715624)
\curveto(203.05825469,28.23382288)(202.87596305,28.36923953)(202.63117141,28.47340619)
\curveto(202.39158811,28.58278117)(201.95929649,28.71038533)(201.33429657,28.85621864)
\curveto(200.7978383,28.98121862)(200.42283835,29.07757278)(200.20929671,29.1452811)
\curveto(199.99575507,29.21819776)(199.79002593,29.33798941)(199.59210928,29.50465606)
\curveto(199.39419264,29.67132271)(199.24315099,29.87444768)(199.13898434,30.11403099)
\curveto(199.03481769,30.35882262)(198.98273436,30.62705176)(198.98273436,30.91871839)
\curveto(198.98273436,31.61663497)(199.25877599,32.17392656)(199.81085926,32.59059318)
\curveto(200.36815085,33.01246813)(201.0608591,33.2234056)(201.888984,33.2234056)
\curveto(202.23794229,33.2234056)(202.61815058,33.17392644)(203.02960886,33.07496812)
\curveto(203.44106714,32.98121813)(203.7535671,32.89007231)(203.96710874,32.80153065)
\lineto(204.03742124,32.69215567)
\curveto(203.99575457,32.48382236)(203.96971291,31.93694743)(203.95929625,31.05153087)
\lineto(203.88898375,30.98121838)
\lineto(203.57648379,30.98121838)
\lineto(203.5061713,31.05153087)
\curveto(203.48533797,31.36923916)(203.45669214,31.5905933)(203.42023381,31.71559329)
\curveto(203.38377548,31.84580161)(203.29002549,31.98382242)(203.13898385,32.12965574)
\curveto(202.9879422,32.28069738)(202.77440056,32.40569737)(202.49835893,32.50465569)
\curveto(202.22231729,32.60882234)(201.93065066,32.66090567)(201.62335903,32.66090567)
\curveto(201.30565074,32.66090567)(201.03742161,32.61403068)(200.81867163,32.52028069)
\curveto(200.60512999,32.4265307)(200.43065085,32.28330155)(200.2952342,32.09059324)
\curveto(200.16502588,31.90309326)(200.09992172,31.67392663)(200.09992172,31.40309333)
\curveto(200.09992172,31.20517668)(200.13898422,31.02809337)(200.21710921,30.87184339)
\curveto(200.30044253,30.71559341)(200.42544251,30.58798926)(200.59210916,30.48903094)
\curveto(200.75877581,30.39528095)(200.93325495,30.32757263)(201.1155466,30.28590596)
\lineto(201.96710899,30.06715599)
\curveto(202.69627557,29.89007268)(203.21710884,29.73642687)(203.5296088,29.60621855)
\curveto(203.84731709,29.47601023)(204.09210873,29.27809359)(204.26398371,29.01246862)
\curveto(204.43585869,28.75205199)(204.52179618,28.43434369)(204.52179618,28.05934374)
\curveto(204.52179618,27.34059383)(204.22492121,26.7234064)(203.63117129,26.20778147)
\curveto(203.03742136,25.69215653)(202.27440062,25.43434406)(201.34210907,25.43434406)
\curveto(201.01398411,25.43434406)(200.60512999,25.47080239)(200.11554672,25.54371905)
\curveto(199.63117178,25.61663571)(199.21710933,25.69736486)(198.87335937,25.78590652)
\lineto(198.83429688,25.88746901)
\lineto(198.91242187,26.41090644)
\curveto(198.93846353,26.57236476)(198.95408853,26.73382307)(198.95929686,26.89528138)
\curveto(198.96450519,27.06194803)(198.96971353,27.40830215)(198.97492186,27.93434375)
\closepath
}
}
{
\newrgbcolor{curcolor}{0 0 0}
\pscustom[linestyle=none,fillstyle=solid,fillcolor=curcolor]
{
\newpath
\moveto(211.94367026,26.85621889)
\lineto(211.69367029,26.29371896)
\curveto(211.15200369,25.94476067)(210.66762875,25.71819819)(210.24054547,25.61403154)
\curveto(209.81867052,25.50986489)(209.4410664,25.45778156)(209.10773311,25.45778156)
\curveto(208.48273319,25.45778156)(207.88898326,25.58017738)(207.32648333,25.82496901)
\curveto(206.76919173,26.06976065)(206.31346262,26.4838231)(205.959296,27.06715636)
\curveto(205.61033771,27.65048962)(205.43585856,28.35361454)(205.43585856,29.1765311)
\curveto(205.43585856,29.72340603)(205.50356689,30.21559347)(205.63898354,30.65309342)
\curveto(205.77440019,31.0958017)(205.91502517,31.42392666)(206.06085849,31.6374683)
\curveto(206.21190013,31.85100994)(206.46450427,32.08798908)(206.81867089,32.34840571)
\curveto(207.17283752,32.60882234)(207.54783747,32.81715565)(207.94367075,32.97340563)
\curveto(208.33950404,33.12965561)(208.76658732,33.2077806)(209.2249206,33.2077806)
\curveto(209.84992052,33.2077806)(210.39939962,33.05934312)(210.87335789,32.76246816)
\curveto(211.3525245,32.47080153)(211.68846196,32.09580157)(211.88117027,31.6374683)
\curveto(212.07387858,31.17913502)(212.17023273,30.69215591)(212.17023273,30.17653098)
\curveto(212.17023273,30.01507266)(212.16242023,29.85882268)(212.14679524,29.70778104)
\lineto(212.06085775,29.62184355)
\curveto(211.70669112,29.54371856)(211.23012868,29.49163523)(210.63117042,29.46559357)
\curveto(210.03221216,29.4395519)(209.63637888,29.42653107)(209.44367057,29.42653107)
\lineto(206.93585838,29.42653107)
\curveto(206.94627504,28.3484062)(207.21710834,27.55413547)(207.74835828,27.04371886)
\curveto(208.27960821,26.53330226)(208.9306498,26.27809396)(209.70148304,26.27809396)
\curveto(210.06606633,26.27809396)(210.41502462,26.34059395)(210.74835791,26.46559394)
\curveto(211.08689953,26.59059392)(211.44367032,26.76507306)(211.81867028,26.98903137)
\closepath
\moveto(206.93585838,30.05153099)
\curveto(207.02960837,30.03590599)(207.38898332,30.01767683)(208.01398325,29.9968435)
\curveto(208.6441915,29.97601017)(209.11033728,29.9655935)(209.41242057,29.9655935)
\curveto(210.13637882,29.9655935)(210.57648293,29.97861434)(210.73273291,30.004656)
\curveto(210.73794124,30.12965598)(210.74054541,30.22601014)(210.74054541,30.29371846)
\curveto(210.74054541,31.10101003)(210.57648293,31.69996829)(210.24835797,32.09059324)
\curveto(209.92023301,32.48642653)(209.4723164,32.68434317)(208.90460814,32.68434317)
\curveto(208.28481655,32.68434317)(207.8004416,32.46298903)(207.45148331,32.02028075)
\curveto(207.10773336,31.57757247)(206.93585838,30.92132255)(206.93585838,30.05153099)
\closepath
\moveto(209.11554561,33.64528055)
\closepath
\moveto(209.02179562,25.23903159)
\closepath
}
}
{
\newrgbcolor{curcolor}{1 1 1}
\pscustom[linestyle=none,fillstyle=solid,fillcolor=curcolor]
{
\newpath
\moveto(27.3012357,9.00085355)
\curveto(27.3012357,4.82610309)(23.91693177,1.44179916)(19.74218131,1.44179916)
\curveto(15.56743084,1.44179916)(12.18312691,4.82610309)(12.18312691,9.00085355)
\curveto(12.18312691,13.17560402)(15.56743084,16.55990795)(19.74218131,16.55990795)
\curveto(23.91693177,16.55990795)(27.3012357,13.17560402)(27.3012357,9.00085355)
\closepath
}
}
{
\newrgbcolor{curcolor}{0.15686275 0.16078432 0.16470589}
\pscustom[linewidth=2.88359956,linecolor=curcolor]
{
\newpath
\moveto(27.3012357,9.00085355)
\curveto(27.3012357,4.82610309)(23.91693177,1.44179916)(19.74218131,1.44179916)
\curveto(15.56743084,1.44179916)(12.18312691,4.82610309)(12.18312691,9.00085355)
\curveto(12.18312691,13.17560402)(15.56743084,16.55990795)(19.74218131,16.55990795)
\curveto(23.91693177,16.55990795)(27.3012357,13.17560402)(27.3012357,9.00085355)
\closepath
}
}
{
\newrgbcolor{curcolor}{0 0 0}
\pscustom[linestyle=none,fillstyle=solid,fillcolor=curcolor]
{
\newpath
\moveto(23.29296212,13.58288424)
\curveto(22.08462893,11.37975951)(21.10025405,9.42142642)(20.33983748,7.70788496)
\curveto(19.65233757,6.15580182)(19.11327513,4.83548948)(18.72265018,3.74694795)
\lineto(18.5976502,3.66101046)
\curveto(18.36327523,3.68705213)(18.10546276,3.70007296)(17.82421279,3.70007296)
\curveto(17.54817116,3.70007296)(17.24608786,3.68705213)(16.9179629,3.66101046)
\lineto(16.83983791,3.83288544)
\lineto(20.207025,9.63757223)
\curveto(20.48306663,10.10632217)(20.69660827,10.48913462)(20.84764992,10.78600958)
\lineto(21.58983733,12.23132191)
\lineto(21.53514983,12.37194689)
\lineto(19.55858758,12.37194689)
\lineto(18.16015025,12.34850939)
\curveto(17.70702531,12.34330106)(17.42056701,12.31986356)(17.30077536,12.2781969)
\curveto(17.2122337,12.2469469)(17.13931704,12.16882191)(17.08202538,12.04382193)
\curveto(17.02994206,11.91882194)(16.95181707,11.42663451)(16.84765041,10.56725961)
\lineto(16.76171292,10.48132212)
\lineto(16.27733798,10.49694712)
\lineto(16.19140049,10.58288461)
\curveto(16.22265049,11.16100954)(16.23827549,11.71569697)(16.23827549,12.2469469)
\curveto(16.23827549,12.86153016)(16.22265049,13.53080091)(16.19140049,14.25475916)
\lineto(16.29296298,14.34069665)
\curveto(17.53775449,14.28340499)(18.75390018,14.25475916)(19.94140003,14.25475916)
\curveto(21.48827484,14.25475916)(22.57681637,14.26777999)(23.20702463,14.29382165)
\lineto(23.29296212,14.22350916)
\closepath
}
}
{
\newrgbcolor{curcolor}{0 0 0}
\pscustom[linestyle=none,fillstyle=solid,fillcolor=curcolor]
{
\newpath
\moveto(34.42025465,6.73563573)
\lineto(34.48275464,6.79813573)
\lineto(34.86556709,6.79813573)
\lineto(34.92806709,6.73563573)
\curveto(34.95410875,6.06896915)(34.98796291,5.66011503)(35.02962957,5.50907339)
\curveto(35.07650457,5.36324007)(35.21712955,5.20699009)(35.45150452,5.04032344)
\curveto(35.69108783,4.87886513)(36.00619195,4.74605265)(36.39681691,4.64188599)
\curveto(36.79265019,4.53771934)(37.19108764,4.48563601)(37.59212926,4.48563601)
\curveto(38.14421252,4.48563601)(38.63900413,4.5871985)(39.07650408,4.79032347)
\curveto(39.51921235,4.99344845)(39.86296231,5.29032341)(40.10775395,5.68094836)
\curveto(40.35254559,6.07678165)(40.4749414,6.51948993)(40.4749414,7.0090732)
\curveto(40.4749414,7.35282316)(40.41504558,7.65230229)(40.29525393,7.90751059)
\curveto(40.18067061,8.16271889)(40.01660813,8.36584387)(39.80306649,8.51688551)
\curveto(39.59473318,8.6731355)(39.35514987,8.78771881)(39.08431657,8.86063547)
\curveto(38.81348327,8.93876046)(38.40983749,9.02209379)(37.87337922,9.11063544)
\curveto(37.35254595,9.19396876)(36.94369184,9.26948959)(36.64681688,9.33719791)
\curveto(36.35515024,9.40490624)(36.06348361,9.50907289)(35.77181698,9.64969788)
\curveto(35.48015035,9.79032286)(35.23535872,9.96740617)(35.03744207,10.18094781)
\curveto(34.84473376,10.39969778)(34.68848378,10.66532275)(34.56869213,10.97782271)
\curveto(34.45410881,11.29553101)(34.39681715,11.63407263)(34.39681715,11.99344759)
\curveto(34.39681715,12.9674058)(34.73535878,13.7642807)(35.41244203,14.38407229)
\curveto(36.08952528,15.00907221)(36.99837933,15.32157218)(38.13900419,15.32157218)
\curveto(38.59212914,15.32157218)(39.08431657,15.26428052)(39.61556651,15.1496972)
\curveto(40.15202478,15.04032221)(40.63900388,14.87625973)(41.07650383,14.65750976)
\lineto(41.13119132,14.56375977)
\curveto(41.016608,14.08459316)(40.94629551,13.40230158)(40.92025385,12.51688502)
\lineto(40.84994136,12.45438503)
\lineto(40.44369141,12.45438503)
\lineto(40.38119141,12.50907252)
\curveto(40.37077475,13.13407245)(40.35514975,13.52730156)(40.33431642,13.68875988)
\curveto(40.31348309,13.85021819)(40.07650395,14.050739)(39.62337901,14.2903223)
\curveto(39.17025406,14.52990561)(38.67806662,14.64969726)(38.14681669,14.64969726)
\curveto(37.70410841,14.64969726)(37.29004596,14.5611556)(36.90462934,14.38407229)
\curveto(36.51921272,14.20698898)(36.22754609,13.92834318)(36.02962945,13.54813489)
\curveto(35.83171281,13.16792661)(35.73275449,12.77990582)(35.73275449,12.38407254)
\curveto(35.73275449,12.07678091)(35.79265031,11.80334344)(35.91244197,11.56376014)
\curveto(36.03223362,11.32938517)(36.18587943,11.14188519)(36.37337941,11.00126021)
\curveto(36.56608772,10.86584356)(36.78744186,10.76428107)(37.03744183,10.69657275)
\curveto(37.29265013,10.62886442)(37.75619174,10.5455311)(38.42806666,10.44657278)
\curveto(39.34994154,10.31636446)(40.01921229,10.15751031)(40.43587891,9.97001034)
\curveto(40.85775386,9.78771869)(41.19369131,9.49605206)(41.44369128,9.09501044)
\curveto(41.69889959,8.69396883)(41.82650374,8.20959389)(41.82650374,7.64188562)
\curveto(41.82650374,6.53251076)(41.37337879,5.60803171)(40.4671289,4.86844846)
\curveto(39.56608735,4.12886522)(38.47233748,3.7590736)(37.18587931,3.7590736)
\curveto(36.10775444,3.7590736)(35.16244206,3.95178191)(34.34994216,4.33719853)
\lineto(34.30306716,4.43876102)
\curveto(34.35515049,4.76688598)(34.39421299,5.53251088)(34.42025465,6.73563573)
\closepath
}
}
{
\newrgbcolor{curcolor}{0 0 0}
\pscustom[linestyle=none,fillstyle=solid,fillcolor=curcolor]
{
\newpath
\moveto(48.98275285,4.88407346)
\lineto(48.67025289,4.36063603)
\curveto(47.99837798,3.99084441)(47.24837807,3.8059486)(46.42025317,3.8059486)
\curveto(45.27441998,3.8059486)(44.38900342,4.14709439)(43.7640035,4.82938597)
\curveto(43.13900357,5.51167755)(42.82650361,6.39188578)(42.82650361,7.47001064)
\curveto(42.82650361,7.95959392)(42.87858694,8.3944897)(42.98275359,8.77469798)
\curveto(43.09212858,9.1601146)(43.22754523,9.4778229)(43.38900354,9.72782287)
\curveto(43.55567019,9.97782283)(43.73796183,10.17573948)(43.93587848,10.32157279)
\curveto(44.13379512,10.46740611)(44.46712841,10.67053108)(44.93587835,10.93094772)
\curveto(45.40983663,11.19136435)(45.76660742,11.36063516)(46.00619072,11.43876015)
\curveto(46.24577403,11.52209348)(46.57650315,11.56376014)(46.9983781,11.56376014)
\curveto(47.80566967,11.56376014)(48.46712792,11.41011432)(48.98275285,11.1028227)
\curveto(48.88379453,10.5142811)(48.80827371,9.85803118)(48.75619038,9.13407294)
\lineto(48.68587789,9.07157295)
\lineto(48.36556543,9.07157295)
\lineto(48.29525294,9.14188544)
\curveto(48.27441961,9.57938538)(48.24577378,9.87365618)(48.20931545,10.02469783)
\curveto(48.17285712,10.17573948)(47.96452381,10.33719779)(47.58431553,10.50907277)
\curveto(47.20931557,10.68094775)(46.80827396,10.76688524)(46.38119068,10.76688524)
\curveto(45.95410739,10.76688524)(45.57389911,10.67573942)(45.24056582,10.49344777)
\curveto(44.90723252,10.31636446)(44.65462839,10.02209366)(44.48275341,9.61063538)
\curveto(44.31608676,9.20438543)(44.23275344,8.71740632)(44.23275344,8.14969806)
\curveto(44.23275344,7.67053145)(44.30046176,7.20178151)(44.43587841,6.74344823)
\curveto(44.57129506,6.29032329)(44.74837838,5.91271917)(44.96712835,5.61063587)
\curveto(45.19108666,5.31376091)(45.48535745,5.07417761)(45.84994074,4.89188596)
\curveto(46.21452403,4.70959432)(46.62077398,4.6184485)(47.06869059,4.6184485)
\curveto(47.34994056,4.6184485)(47.62337802,4.65490682)(47.88900299,4.72782348)
\curveto(48.15983629,4.80074014)(48.46973208,4.92053179)(48.81869037,5.08719844)
\closepath
}
}
{
\newrgbcolor{curcolor}{0 0 0}
\pscustom[linestyle=none,fillstyle=solid,fillcolor=curcolor]
{
\newpath
\moveto(51.95931499,15.61844714)
\lineto(52.10775247,15.52469715)
\curveto(52.05046081,14.92573889)(52.02181498,13.72521821)(52.02181498,11.92313509)
\lineto(52.02181498,9.99344783)
\curveto(52.45931493,10.35282279)(52.86556488,10.70959358)(53.24056483,11.0637602)
\curveto(53.34993982,11.16271852)(53.45410647,11.23563518)(53.55306479,11.28251017)
\curveto(53.65202311,11.32938517)(53.81087726,11.37886433)(54.02962723,11.43094766)
\curveto(54.25358554,11.48303098)(54.48275218,11.50907265)(54.71712715,11.50907265)
\curveto(55.11296043,11.50907265)(55.49577288,11.42834349)(55.86556451,11.26688518)
\curveto(56.24056446,11.10542686)(56.52181442,10.91271855)(56.7093144,10.68876025)
\curveto(56.90202271,10.47001027)(57.03483519,10.20959364)(57.10775185,9.90751034)
\curveto(57.18066851,9.60542705)(57.21712684,9.23303126)(57.21712684,8.79032298)
\lineto(57.21712684,7.41532315)
\curveto(57.21712684,7.3267815)(57.23014767,6.67834408)(57.25618933,5.47001089)
\curveto(57.27181433,4.97521929)(57.32650183,4.68094849)(57.42025181,4.5871985)
\curveto(57.51921014,4.49344851)(57.84212676,4.44657352)(58.38900169,4.44657352)
\lineto(58.45150169,4.38407352)
\lineto(58.45150169,4.04813607)
\lineto(58.38900169,3.98563607)
\curveto(57.72754344,4.02730274)(57.27962683,4.04813607)(57.04525186,4.04813607)
\curveto(56.92546021,4.04813607)(56.54525192,4.02730274)(55.904627,3.98563607)
\lineto(55.81087701,4.07157356)
\curveto(55.87858534,4.74865681)(55.9124395,5.6314692)(55.9124395,6.72001074)
\lineto(55.9124395,7.74344811)
\curveto(55.9124395,8.35803137)(55.8968145,8.80073965)(55.86556451,9.07157295)
\curveto(55.83952284,9.34240625)(55.75358535,9.58719788)(55.60775204,9.80594786)
\curveto(55.46191872,10.02990616)(55.26660625,10.20178114)(55.02181461,10.32157279)
\curveto(54.78223131,10.44657278)(54.49056468,10.50907277)(54.14681472,10.50907277)
\curveto(53.79264809,10.50907277)(53.49577313,10.45959361)(53.25618983,10.36063529)
\curveto(53.02181486,10.2668853)(52.78223155,10.10542699)(52.53743992,9.87626035)
\curveto(52.29264828,9.64709371)(52.1442108,9.4543854)(52.09212747,9.29813542)
\curveto(52.04525248,9.14709377)(52.02181498,8.85282297)(52.02181498,8.41532303)
\lineto(52.02181498,7.0246982)
\curveto(52.02181498,6.91011488)(52.03223164,6.54032326)(52.05306498,5.91532334)
\curveto(52.07389831,5.29553175)(52.09473164,4.93355262)(52.11556497,4.82938597)
\curveto(52.14160663,4.72521932)(52.17806496,4.64969849)(52.22493995,4.6028235)
\curveto(52.27181495,4.5559485)(52.32910661,4.52469851)(52.39681493,4.50907351)
\curveto(52.46452326,4.49865684)(52.74577322,4.47782351)(53.24056483,4.44657352)
\lineto(53.31087732,4.38407352)
\lineto(53.31087732,4.05594857)
\lineto(53.24837733,3.98563607)
\curveto(52.59733574,4.02730274)(51.96973165,4.04813607)(51.36556506,4.04813607)
\curveto(50.7666068,4.04813607)(50.14160688,4.02730274)(49.49056529,3.98563607)
\lineto(49.4202528,4.05594857)
\lineto(49.4202528,4.38407352)
\lineto(49.49056529,4.44657352)
\curveto(49.99577356,4.47782351)(50.27962769,4.50126101)(50.34212769,4.51688601)
\curveto(50.40983601,4.53251101)(50.46712767,4.563761)(50.51400267,4.610636)
\curveto(50.56608599,4.66271932)(50.59994015,4.73824015)(50.61556515,4.83719847)
\curveto(50.63639848,4.94136512)(50.65723181,5.27730258)(50.67806515,5.84501084)
\curveto(50.70410681,6.41792744)(50.71712764,6.85021905)(50.71712764,7.14188568)
\lineto(50.71712764,11.18876019)
\lineto(50.68587764,12.84500998)
\curveto(50.67025265,13.42313491)(50.65202348,13.82417653)(50.63119015,14.04813483)
\curveto(50.61556515,14.27209314)(50.59212766,14.40490562)(50.56087766,14.44657228)
\curveto(50.52962766,14.48823895)(50.47494017,14.51948894)(50.39681518,14.54032227)
\curveto(50.31869019,14.5611556)(50.00358606,14.57157227)(49.4515028,14.57157227)
\lineto(49.38119031,14.64188476)
\lineto(49.38119031,14.97000972)
\lineto(49.4436903,15.04032221)
\curveto(50.43848184,15.15490553)(51.2770234,15.34761384)(51.95931499,15.61844714)
\closepath
}
}
{
\newrgbcolor{curcolor}{0 0 0}
\pscustom[linestyle=none,fillstyle=solid,fillcolor=curcolor]
{
\newpath
\moveto(61.53743881,15.61844714)
\lineto(61.68587629,15.52469715)
\curveto(61.62858463,14.92573889)(61.5999388,13.72521821)(61.5999388,11.92313509)
\lineto(61.5999388,7.0246982)
\curveto(61.5999388,6.91011488)(61.61035546,6.54032326)(61.63118879,5.91532334)
\curveto(61.65202213,5.29553175)(61.67285546,4.93355262)(61.69368879,4.82938597)
\curveto(61.71973045,4.72521932)(61.75618878,4.64969849)(61.80306377,4.6028235)
\curveto(61.84993877,4.5559485)(61.90723043,4.52469851)(61.97493875,4.50907351)
\curveto(62.04264708,4.49865684)(62.32389704,4.47782351)(62.81868865,4.44657352)
\lineto(62.88900114,4.38407352)
\lineto(62.88900114,4.05594857)
\lineto(62.82650115,3.98563607)
\curveto(62.17545956,4.02730274)(61.54785547,4.04813607)(60.94368888,4.04813607)
\curveto(60.34473062,4.04813607)(59.7197307,4.02730274)(59.06868911,3.98563607)
\lineto(58.99837662,4.05594857)
\lineto(58.99837662,4.38407352)
\lineto(59.06868911,4.44657352)
\curveto(59.57389738,4.47782351)(59.85775151,4.50126101)(59.92025151,4.51688601)
\curveto(59.98795983,4.53251101)(60.04525149,4.563761)(60.09212648,4.610636)
\curveto(60.14420981,4.66271932)(60.17806397,4.73824015)(60.19368897,4.83719847)
\curveto(60.2145223,4.94136512)(60.23535563,5.27730258)(60.25618896,5.84501084)
\curveto(60.28223063,6.41792744)(60.29525146,6.85021905)(60.29525146,7.14188568)
\lineto(60.29525146,11.18876019)
\lineto(60.26400146,12.84500998)
\curveto(60.24837647,13.42313491)(60.2301473,13.82417653)(60.20931397,14.04813483)
\curveto(60.19368897,14.27209314)(60.17025147,14.40490562)(60.13900148,14.44657228)
\curveto(60.10775148,14.48823895)(60.05306399,14.51948894)(59.974939,14.54032227)
\curveto(59.89681401,14.5611556)(59.58170988,14.57157227)(59.02962662,14.57157227)
\lineto(58.95931412,14.64188476)
\lineto(58.95931412,14.97000972)
\lineto(59.02181412,15.04032221)
\curveto(60.01660566,15.15490553)(60.85514722,15.34761384)(61.53743881,15.61844714)
\closepath
}
}
{
\newrgbcolor{curcolor}{0 0 0}
\pscustom[linestyle=none,fillstyle=solid,fillcolor=curcolor]
{
\newpath
\moveto(65.97493826,11.49344765)
\lineto(66.12337574,11.39188516)
\curveto(66.06608408,10.70959358)(66.03743825,9.84240618)(66.03743825,8.79032298)
\lineto(66.03743825,6.97001071)
\curveto(66.03743825,6.30334412)(66.08691741,5.83980251)(66.18587573,5.57938588)
\curveto(66.29004239,5.31896924)(66.4671257,5.11844843)(66.71712567,4.97782345)
\curveto(66.96712564,4.8424068)(67.2640006,4.77469848)(67.60775056,4.77469848)
\curveto(67.98795884,4.77469848)(68.33431297,4.84761513)(68.64681293,4.99344845)
\curveto(68.95931289,5.1444901)(69.21972953,5.35542757)(69.42806283,5.62626087)
\curveto(69.64160447,5.89709417)(69.76400029,6.07938582)(69.79525029,6.1731358)
\curveto(69.82650028,6.26688579)(69.84733361,6.52990659)(69.85775028,6.96219821)
\lineto(69.88118778,7.8371981)
\lineto(69.88118778,8.68094799)
\curveto(69.88118778,8.8892813)(69.87077111,9.17313543)(69.84993778,9.53251039)
\curveto(69.82910445,9.89188535)(69.81087529,10.11323948)(69.79525029,10.19657281)
\curveto(69.78483362,10.27990613)(69.74577113,10.33980196)(69.6780628,10.37626029)
\curveto(69.61035448,10.41792695)(69.47493783,10.43876028)(69.27181285,10.43876028)
\lineto(68.61556293,10.44657278)
\lineto(68.54525044,10.50907277)
\lineto(68.54525044,10.84501023)
\lineto(68.60775043,10.90751022)
\curveto(69.60254198,11.02730187)(70.44108354,11.22261435)(71.12337512,11.49344765)
\lineto(71.27181261,11.39188516)
\curveto(71.21452095,10.70959358)(71.18587512,9.84240618)(71.18587512,8.79032298)
\lineto(71.18587512,7.41532315)
\curveto(71.18587512,7.33719816)(71.19889595,6.69917741)(71.22493761,5.50126089)
\curveto(71.23535428,5.05855261)(71.26139594,4.78771931)(71.3030626,4.68876099)
\curveto(71.3499376,4.595011)(71.41243759,4.52730267)(71.49056258,4.48563601)
\curveto(71.56868757,4.44917768)(71.80045837,4.43094852)(72.18587499,4.43094852)
\lineto(72.40462497,4.43094852)
\lineto(72.47493746,4.36844853)
\lineto(72.47493746,4.05594857)
\lineto(72.41243747,3.98563607)
\curveto(71.67806256,4.02730274)(71.18587512,4.04813607)(70.93587515,4.04813607)
\curveto(70.61816685,4.04813607)(70.2509794,4.0299069)(69.83431278,3.99344857)
\lineto(69.76400029,4.05594857)
\curveto(69.79525029,4.5871985)(69.81868779,5.03771928)(69.83431278,5.4075109)
\curveto(69.50618782,5.1523026)(69.13118787,4.81115681)(68.70931292,4.38407352)
\curveto(68.54785461,4.22261521)(68.3160838,4.08719856)(68.01400051,3.97782357)
\curveto(67.71191721,3.86844859)(67.36816725,3.81376109)(66.98275063,3.81376109)
\curveto(66.39941737,3.81376109)(65.94368826,3.90490692)(65.6155633,4.08719856)
\curveto(65.29264668,4.27469854)(65.06348004,4.52990684)(64.92806339,4.85282347)
\curveto(64.79264674,5.18094843)(64.72493841,5.74344836)(64.72493841,6.54032326)
\lineto(64.73275091,7.15751068)
\lineto(64.73275091,8.68094799)
\curveto(64.73275091,8.8892813)(64.72233425,9.17313543)(64.70150092,9.53251039)
\curveto(64.68066759,9.89188535)(64.66243842,10.11323948)(64.64681342,10.19657281)
\curveto(64.63639676,10.27990613)(64.59733426,10.33980196)(64.52962594,10.37626029)
\curveto(64.46191761,10.41792695)(64.32650096,10.43876028)(64.12337599,10.43876028)
\lineto(63.46712607,10.44657278)
\lineto(63.39681358,10.50907277)
\lineto(63.39681358,10.84501023)
\lineto(63.45931357,10.90751022)
\curveto(64.45410511,11.02730187)(65.29264668,11.22261435)(65.97493826,11.49344765)
\closepath
\moveto(67.84212553,11.98563509)
\closepath
\moveto(67.92025052,3.57938612)
\closepath
}
}
{
\newrgbcolor{curcolor}{0 0 0}
\pscustom[linestyle=none,fillstyle=solid,fillcolor=curcolor]
{
\newpath
\moveto(73.70931231,6.34501078)
\lineto(74.04524976,6.34501078)
\lineto(74.11556226,6.27469829)
\curveto(74.12597892,5.83719835)(74.15202058,5.46219839)(74.19368725,5.14969843)
\curveto(74.35514556,4.90490679)(74.63379136,4.70438599)(75.02962464,4.548136)
\curveto(75.42545793,4.39709436)(75.81608288,4.32157353)(76.2014995,4.32157353)
\curveto(76.75358276,4.32157353)(77.19368688,4.47001101)(77.52181184,4.76688598)
\curveto(77.85514513,5.06376094)(78.02181177,5.41792756)(78.02181177,5.82938585)
\curveto(78.02181177,6.05334415)(77.96191595,6.24605246)(77.8421243,6.40751078)
\curveto(77.72233264,6.57417742)(77.540041,6.70959407)(77.29524936,6.81376072)
\curveto(77.05566606,6.92313571)(76.62337445,7.05073986)(75.99837452,7.19657318)
\curveto(75.46191626,7.32157316)(75.0869163,7.41792732)(74.87337466,7.48563564)
\curveto(74.65983302,7.5585523)(74.45410388,7.67834395)(74.25618724,7.8450106)
\curveto(74.0582706,8.01167724)(73.90722895,8.21480222)(73.80306229,8.45438552)
\curveto(73.69889564,8.69917716)(73.64681231,8.96740629)(73.64681231,9.25907292)
\curveto(73.64681231,9.9569895)(73.92285395,10.5142811)(74.47493721,10.93094772)
\curveto(75.03222881,11.35282267)(75.72493706,11.56376014)(76.55306195,11.56376014)
\curveto(76.90202025,11.56376014)(77.28222853,11.51428098)(77.69368681,11.41532266)
\curveto(78.1051451,11.32157267)(78.41764506,11.23042685)(78.6311867,11.14188519)
\lineto(78.70149919,11.0325102)
\curveto(78.65983253,10.8241769)(78.63379086,10.27730196)(78.6233742,9.39188541)
\lineto(78.55306171,9.32157292)
\lineto(78.24056175,9.32157292)
\lineto(78.17024926,9.39188541)
\curveto(78.14941592,9.7095937)(78.12077009,9.93094784)(78.08431177,10.05594783)
\curveto(78.04785344,10.18615614)(77.95410345,10.32417696)(77.8030618,10.47001027)
\curveto(77.65202015,10.62105192)(77.43847851,10.74605191)(77.16243688,10.84501023)
\curveto(76.88639525,10.94917688)(76.59472862,11.00126021)(76.28743699,11.00126021)
\curveto(75.96972869,11.00126021)(75.70149956,10.95438521)(75.48274959,10.86063523)
\curveto(75.26920795,10.76688524)(75.0947288,10.62365609)(74.95931215,10.43094778)
\curveto(74.82910383,10.2434478)(74.76399968,10.01428116)(74.76399968,9.74344786)
\curveto(74.76399968,9.54553122)(74.80306217,9.36844791)(74.88118716,9.21219793)
\curveto(74.96452048,9.05594795)(75.08952047,8.9283438)(75.25618711,8.82938548)
\curveto(75.42285376,8.73563549)(75.59733291,8.66792716)(75.77962455,8.6262605)
\lineto(76.63118695,8.40751053)
\curveto(77.36035352,8.23042722)(77.88118679,8.0767814)(78.19368675,7.94657309)
\curveto(78.51139505,7.81636477)(78.75618668,7.61844813)(78.92806166,7.35282316)
\curveto(79.09993664,7.09240652)(79.18587413,6.77469823)(79.18587413,6.39969828)
\curveto(79.18587413,5.68094836)(78.88899917,5.06376094)(78.29524924,4.548136)
\curveto(77.70149931,4.03251107)(76.93847857,3.7746986)(76.00618702,3.7746986)
\curveto(75.67806206,3.7746986)(75.26920795,3.81115693)(74.77962467,3.88407359)
\curveto(74.29524973,3.95699024)(73.88118728,4.0377194)(73.53743733,4.12626106)
\lineto(73.49837483,4.22782354)
\lineto(73.57649982,4.75126098)
\curveto(73.60254149,4.91271929)(73.61816648,5.07417761)(73.62337482,5.23563592)
\curveto(73.62858315,5.40230257)(73.63379148,5.74865669)(73.63899981,6.27469829)
\closepath
}
}
{
\newrgbcolor{curcolor}{0 0 0}
\pscustom[linestyle=none,fillstyle=solid,fillcolor=curcolor]
{
\newpath
\moveto(80.49056147,6.34501078)
\lineto(80.82649893,6.34501078)
\lineto(80.89681142,6.27469829)
\curveto(80.90722808,5.83719835)(80.93326975,5.46219839)(80.97493641,5.14969843)
\curveto(81.13639472,4.90490679)(81.41504052,4.70438599)(81.81087381,4.548136)
\curveto(82.20670709,4.39709436)(82.59733204,4.32157353)(82.98274866,4.32157353)
\curveto(83.53483193,4.32157353)(83.97493604,4.47001101)(84.303061,4.76688598)
\curveto(84.63639429,5.06376094)(84.80306094,5.41792756)(84.80306094,5.82938585)
\curveto(84.80306094,6.05334415)(84.74316511,6.24605246)(84.62337346,6.40751078)
\curveto(84.50358181,6.57417742)(84.32129016,6.70959407)(84.07649853,6.81376072)
\curveto(83.83691522,6.92313571)(83.40462361,7.05073986)(82.77962369,7.19657318)
\curveto(82.24316542,7.32157316)(81.86816547,7.41792732)(81.65462383,7.48563564)
\curveto(81.44108219,7.5585523)(81.23535304,7.67834395)(81.0374364,7.8450106)
\curveto(80.83951976,8.01167724)(80.68847811,8.21480222)(80.58431146,8.45438552)
\curveto(80.4801448,8.69917716)(80.42806148,8.96740629)(80.42806148,9.25907292)
\curveto(80.42806148,9.9569895)(80.70410311,10.5142811)(81.25618637,10.93094772)
\curveto(81.81347797,11.35282267)(82.50618622,11.56376014)(83.33431112,11.56376014)
\curveto(83.68326941,11.56376014)(84.0634777,11.51428098)(84.47493598,11.41532266)
\curveto(84.88639426,11.32157267)(85.19889422,11.23042685)(85.41243586,11.14188519)
\lineto(85.48274835,11.0325102)
\curveto(85.44108169,10.8241769)(85.41504003,10.27730196)(85.40462336,9.39188541)
\lineto(85.33431087,9.32157292)
\lineto(85.02181091,9.32157292)
\lineto(84.95149842,9.39188541)
\curveto(84.93066509,9.7095937)(84.90201926,9.93094784)(84.86556093,10.05594783)
\curveto(84.8291026,10.18615614)(84.73535261,10.32417696)(84.58431096,10.47001027)
\curveto(84.43326932,10.62105192)(84.21972768,10.74605191)(83.94368604,10.84501023)
\curveto(83.66764441,10.94917688)(83.37597778,11.00126021)(83.06868615,11.00126021)
\curveto(82.75097786,11.00126021)(82.48274872,10.95438521)(82.26399875,10.86063523)
\curveto(82.05045711,10.76688524)(81.87597797,10.62365609)(81.74056132,10.43094778)
\curveto(81.610353,10.2434478)(81.54524884,10.01428116)(81.54524884,9.74344786)
\curveto(81.54524884,9.54553122)(81.58431133,9.36844791)(81.66243632,9.21219793)
\curveto(81.74576965,9.05594795)(81.87076963,8.9283438)(82.03743628,8.82938548)
\curveto(82.20410292,8.73563549)(82.37858207,8.66792716)(82.56087371,8.6262605)
\lineto(83.41243611,8.40751053)
\curveto(84.14160269,8.23042722)(84.66243595,8.0767814)(84.97493592,7.94657309)
\curveto(85.29264421,7.81636477)(85.53743585,7.61844813)(85.70931083,7.35282316)
\curveto(85.8811858,7.09240652)(85.96712329,6.77469823)(85.96712329,6.39969828)
\curveto(85.96712329,5.68094836)(85.67024833,5.06376094)(85.0764984,4.548136)
\curveto(84.48274848,4.03251107)(83.71972774,3.7746986)(82.78743619,3.7746986)
\curveto(82.45931123,3.7746986)(82.05045711,3.81115693)(81.56087384,3.88407359)
\curveto(81.0764989,3.95699024)(80.66243645,4.0377194)(80.31868649,4.12626106)
\lineto(80.279624,4.22782354)
\lineto(80.35774899,4.75126098)
\curveto(80.38379065,4.91271929)(80.39941565,5.07417761)(80.40462398,5.23563592)
\curveto(80.40983231,5.40230257)(80.41504065,5.74865669)(80.42024898,6.27469829)
\closepath
}
}
{
\newrgbcolor{curcolor}{0 0 0}
\pscustom[linestyle=none,fillstyle=solid,fillcolor=curcolor]
{
\newpath
\moveto(90.98274768,4.44657352)
\lineto(91.04524767,4.38407352)
\lineto(91.04524767,4.05594857)
\lineto(90.98274768,3.98563607)
\curveto(90.82128936,3.98563607)(90.5166019,3.99605274)(90.06868529,4.01688607)
\curveto(89.66243534,4.0377194)(89.24316456,4.04813607)(88.81087294,4.04813607)
\curveto(88.24837301,4.04813607)(87.62076892,4.02730274)(86.92806068,3.98563607)
\lineto(86.86556068,4.05594857)
\lineto(86.86556068,4.38407352)
\lineto(86.92806068,4.44657352)
\curveto(87.43326895,4.47782351)(87.71972724,4.50126101)(87.78743557,4.51688601)
\curveto(87.85514389,4.53251101)(87.91243555,4.563761)(87.95931055,4.610636)
\curveto(88.00618554,4.66271932)(88.03743554,4.73824015)(88.05306054,4.83719847)
\curveto(88.07389387,4.94136512)(88.0947272,5.27730258)(88.11556053,5.84501084)
\curveto(88.14160219,6.41792744)(88.15462302,6.85021905)(88.15462302,7.14188568)
\lineto(88.15462302,10.15751031)
\lineto(87.14681065,10.10282282)
\lineto(87.07649816,10.16532281)
\lineto(87.07649816,10.36063529)
\lineto(87.13118565,10.43876028)
\lineto(88.15462302,10.96219771)
\lineto(88.15462302,11.43094766)
\curveto(88.15462302,11.95178092)(88.17806052,12.34501004)(88.22493552,12.61063501)
\curveto(88.27181051,12.88146831)(88.357748,13.12886411)(88.48274798,13.35282242)
\curveto(88.6129563,13.58198906)(88.83691461,13.86844736)(89.1546229,14.21219731)
\curveto(89.4723312,14.5611556)(89.75097699,14.84500973)(89.9905603,15.06375971)
\curveto(90.2301436,15.28771801)(90.43847691,15.4361555)(90.61556022,15.50907215)
\curveto(90.79264353,15.58719714)(91.00097684,15.62625964)(91.24056014,15.62625964)
\curveto(91.43847679,15.62625964)(91.65201843,15.58719714)(91.88118506,15.50907215)
\lineto(91.87337257,14.22782231)
\lineto(91.69368509,14.15750982)
\curveto(91.40201846,14.41792645)(91.06347683,14.54813477)(90.67806021,14.54813477)
\curveto(90.40722691,14.54813477)(90.17285194,14.49084311)(89.9749353,14.37625979)
\curveto(89.77701866,14.26688481)(89.64160201,14.08459316)(89.56868535,13.82938486)
\curveto(89.49576869,13.57938489)(89.45931036,13.16271828)(89.45931036,12.57938501)
\lineto(89.45931036,10.96219771)
\lineto(90.29524776,10.96219771)
\curveto(90.72753937,10.96219771)(91.13639349,10.97782271)(91.52181011,11.00907271)
\lineto(91.5921226,10.92313522)
\lineto(91.45149762,10.2356353)
\lineto(91.38118513,10.15751031)
\curveto(91.22493515,10.16271865)(91.12337266,10.16532281)(91.07649766,10.16532281)
\lineto(90.06087279,10.17313531)
\lineto(89.45931036,10.17313531)
\lineto(89.45931036,7.0246982)
\curveto(89.45931036,6.85803155)(89.46972703,6.4752191)(89.49056036,5.87626084)
\curveto(89.51660202,5.28251091)(89.54003952,4.93355262)(89.56087285,4.82938597)
\curveto(89.58170618,4.72521932)(89.61556034,4.64969849)(89.66243534,4.6028235)
\curveto(89.71451867,4.5559485)(89.79524782,4.52209434)(89.90462281,4.50126101)
\curveto(90.01920613,4.48042768)(90.37858108,4.46219851)(90.98274768,4.44657352)
\closepath
}
}
{
\newrgbcolor{curcolor}{0 0 0}
\pscustom[linestyle=none,fillstyle=solid,fillcolor=curcolor]
{
\newpath
\moveto(92.34212251,7.62626062)
\curveto(92.34212251,8.24605222)(92.46972666,8.84240631)(92.72493496,9.4153229)
\curveto(92.98014326,9.99344783)(93.42285154,10.4908436)(94.0530598,10.90751022)
\curveto(94.68847639,11.32938517)(95.44108046,11.54032264)(96.31087202,11.54032264)
\curveto(97.40462188,11.54032264)(98.29524677,11.19396852)(98.98274669,10.50126027)
\curveto(99.6702466,9.81376036)(100.01399656,8.9283438)(100.01399656,7.8450106)
\curveto(100.01399656,6.65230241)(99.61816328,5.6705317)(98.82649671,4.89969846)
\curveto(98.04003847,4.12886522)(97.08951776,3.7434486)(95.97493456,3.7434486)
\curveto(95.24576798,3.7434486)(94.5947264,3.93876108)(94.0218098,4.32938603)
\curveto(93.4488932,4.72521932)(93.02441409,5.20699009)(92.74837246,5.77469835)
\curveto(92.47753916,6.34240662)(92.34212251,6.95959404)(92.34212251,7.62626062)
\closepath
\moveto(93.81868483,8.16532306)
\curveto(93.81868483,7.43615648)(93.91243481,6.78251073)(94.09993479,6.2043858)
\curveto(94.28743477,5.6314692)(94.58430973,5.17053176)(94.99055968,4.82157347)
\curveto(95.39680963,4.47261518)(95.86035124,4.29813604)(96.38118451,4.29813604)
\curveto(96.99576777,4.29813604)(97.5035802,4.54032351)(97.90462182,5.02469845)
\curveto(98.30566344,5.51428172)(98.50618425,6.24865663)(98.50618425,7.22782317)
\curveto(98.50618425,8.33719804)(98.28743427,9.24605209)(97.84993433,9.95438534)
\curveto(97.41764272,10.66271858)(96.79264279,11.01688521)(95.97493456,11.01688521)
\curveto(95.29785131,11.01688521)(94.76920554,10.77209357)(94.38899726,10.2825103)
\curveto(94.00878897,9.79292702)(93.81868483,9.08719794)(93.81868483,8.16532306)
\closepath
\moveto(96.18587203,11.98563509)
\closepath
\moveto(96.06868455,3.57938612)
\closepath
}
}
{
\newrgbcolor{curcolor}{0 0 0}
\pscustom[linestyle=none,fillstyle=solid,fillcolor=curcolor]
{
\newpath
\moveto(103.46712114,15.61844714)
\lineto(103.61555862,15.52469715)
\curveto(103.55826696,14.92573889)(103.52962113,13.72521821)(103.52962113,11.92313509)
\lineto(103.52962113,7.0246982)
\curveto(103.52962113,6.91011488)(103.54003779,6.54032326)(103.56087112,5.91532334)
\curveto(103.58170446,5.29553175)(103.60253779,4.93355262)(103.62337112,4.82938597)
\curveto(103.64941278,4.72521932)(103.68587111,4.64969849)(103.7327461,4.6028235)
\curveto(103.7796211,4.5559485)(103.83691276,4.52469851)(103.90462108,4.50907351)
\curveto(103.97232941,4.49865684)(104.25357937,4.47782351)(104.74837098,4.44657352)
\lineto(104.81868347,4.38407352)
\lineto(104.81868347,4.05594857)
\lineto(104.75618348,3.98563607)
\curveto(104.10514189,4.02730274)(103.4775378,4.04813607)(102.87337121,4.04813607)
\curveto(102.27441295,4.04813607)(101.64941303,4.02730274)(100.99837144,3.98563607)
\lineto(100.92805895,4.05594857)
\lineto(100.92805895,4.38407352)
\lineto(100.99837144,4.44657352)
\curveto(101.50357971,4.47782351)(101.78743384,4.50126101)(101.84993384,4.51688601)
\curveto(101.91764216,4.53251101)(101.97493382,4.563761)(102.02180881,4.610636)
\curveto(102.07389214,4.66271932)(102.1077463,4.73824015)(102.1233713,4.83719847)
\curveto(102.14420463,4.94136512)(102.16503796,5.27730258)(102.18587129,5.84501084)
\curveto(102.21191296,6.41792744)(102.22493379,6.85021905)(102.22493379,7.14188568)
\lineto(102.22493379,11.18876019)
\lineto(102.19368379,12.84500998)
\curveto(102.1780588,13.42313491)(102.15982963,13.82417653)(102.1389963,14.04813483)
\curveto(102.1233713,14.27209314)(102.0999338,14.40490562)(102.06868381,14.44657228)
\curveto(102.03743381,14.48823895)(101.98274632,14.51948894)(101.90462133,14.54032227)
\curveto(101.82649634,14.5611556)(101.51139221,14.57157227)(100.95930895,14.57157227)
\lineto(100.88899645,14.64188476)
\lineto(100.88899645,14.97000972)
\lineto(100.95149645,15.04032221)
\curveto(101.94628799,15.15490553)(102.78482955,15.34761384)(103.46712114,15.61844714)
\closepath
}
}
{
\newrgbcolor{curcolor}{0 0 0}
\pscustom[linestyle=none,fillstyle=solid,fillcolor=curcolor]
{
\newpath
\moveto(113.81868236,10.95438521)
\lineto(113.88118235,10.81376023)
\curveto(113.66764071,10.48563527)(113.53482823,10.27209363)(113.4827449,10.17313531)
\lineto(112.02962008,10.17313531)
\curveto(112.1754534,9.88667701)(112.24837005,9.58719788)(112.24837005,9.27469792)
\curveto(112.24837005,8.9049063)(112.16243256,8.54553134)(111.99055759,8.19657305)
\curveto(111.82389094,7.84761476)(111.58951597,7.5507398)(111.28743267,7.30594816)
\curveto(110.98534938,7.06115653)(110.64159942,6.86584405)(110.2561828,6.72001074)
\curveto(109.87597451,6.57417742)(109.44107873,6.50126076)(108.95149546,6.50126076)
\lineto(108.5921205,6.50126076)
\curveto(108.28482888,6.25646913)(108.08691223,6.07417748)(107.99837058,5.95438583)
\curveto(107.90982892,5.83459418)(107.86555809,5.71219836)(107.86555809,5.58719838)
\curveto(107.86555809,5.35803174)(107.96972475,5.19657342)(108.17805806,5.10282344)
\curveto(108.3915997,5.00907345)(108.82909964,4.96219845)(109.49055789,4.96219845)
\lineto(111.24055768,4.98563595)
\curveto(111.79264094,4.98563595)(112.21451589,4.92053179)(112.50618252,4.79032347)
\curveto(112.80305748,4.66011516)(113.04264079,4.42574019)(113.22493243,4.08719856)
\curveto(113.40722408,3.75386527)(113.4983699,3.39969865)(113.4983699,3.02469869)
\curveto(113.4983699,2.4517821)(113.31086992,1.88146967)(112.93586997,1.3137614)
\curveto(112.56607835,0.74084481)(112.02962008,0.30594903)(111.32649517,0.00907406)
\curveto(110.62337025,-0.2878009)(109.87337035,-0.43623838)(109.07649544,-0.43623838)
\curveto(108.6233705,-0.43623838)(108.19628722,-0.38675922)(107.7952456,-0.2878009)
\curveto(107.39941232,-0.18884258)(107.05045403,-0.03780093)(106.74837073,0.16532404)
\curveto(106.45149577,0.36324069)(106.21191246,0.62365732)(106.02962082,0.94657395)
\curveto(105.84732918,1.26949058)(105.75618335,1.60542803)(105.75618335,1.95438632)
\curveto(105.75618335,2.14709463)(105.78482918,2.35021961)(105.84212084,2.56376125)
\curveto(105.8994125,2.77209456)(106.00878749,2.99865703)(106.1702458,3.24344867)
\lineto(107.58430813,4.03251107)
\curveto(107.13118318,4.17313605)(106.84472489,4.31115687)(106.72493323,4.44657352)
\curveto(106.61034992,4.5871985)(106.55305826,4.75386515)(106.55305826,4.94657346)
\curveto(106.55305826,5.1444901)(106.61816241,5.37886507)(106.74837073,5.64969837)
\lineto(108.00618308,6.55594826)
\curveto(107.26659983,6.74865657)(106.7665999,7.03771903)(106.50618326,7.42313565)
\curveto(106.25097496,7.80855227)(106.12337081,8.23042722)(106.12337081,8.68876049)
\curveto(106.12337081,9.11063544)(106.21451663,9.51167706)(106.39680827,9.89188535)
\curveto(106.57909992,10.27730196)(106.85253739,10.58459359)(107.21712067,10.81376023)
\curveto(107.58691229,11.04292687)(107.99837058,11.22261435)(108.45149552,11.35282267)
\curveto(108.9098288,11.48823932)(109.32649541,11.55594764)(109.70149537,11.55594764)
\curveto(110.35253695,11.55594764)(110.97232854,11.34501017)(111.56087014,10.92313522)
\curveto(112.50357836,10.92313522)(113.25618243,10.93355188)(113.81868236,10.95438521)
\closepath
\moveto(107.46712064,9.14188544)
\curveto(107.46712064,8.82417714)(107.52962064,8.48303135)(107.65462062,8.11844806)
\curveto(107.7796206,7.75386478)(107.98014141,7.47782314)(108.25618305,7.29032317)
\curveto(108.53743301,7.10282319)(108.85253714,7.0090732)(109.20149543,7.0090732)
\curveto(109.66503704,7.0090732)(110.06347449,7.16011485)(110.39680778,7.46219815)
\curveto(110.73534941,7.76428144)(110.90462022,8.23563555)(110.90462022,8.87626047)
\curveto(110.90462022,9.41792707)(110.7509744,9.90490618)(110.44368278,10.33719779)
\curveto(110.13639115,10.7694894)(109.70149537,10.98563521)(109.13899544,10.98563521)
\curveto(108.66503716,10.98563521)(108.26659971,10.82938523)(107.94368308,10.51688527)
\curveto(107.62597479,10.20959364)(107.46712064,9.75126036)(107.46712064,9.14188544)
\closepath
\moveto(109.75618286,3.89969858)
\curveto(108.9176413,3.89969858)(108.43326636,3.88928192)(108.30305804,3.86844859)
\curveto(108.17805806,3.84761526)(107.99837058,3.7590736)(107.76399561,3.60282362)
\curveto(107.52962064,3.44657364)(107.33691233,3.23563617)(107.18587068,2.9700112)
\curveto(107.04003736,2.6991779)(106.9671207,2.3970946)(106.9671207,2.06376131)
\curveto(106.9671207,1.47521972)(107.18587068,0.99605311)(107.62337062,0.62626149)
\curveto(108.06087057,0.25646987)(108.64420383,0.07157406)(109.37337041,0.07157406)
\curveto(109.92024534,0.07157406)(110.42545361,0.18615738)(110.88899522,0.41532401)
\curveto(111.35253683,0.64449065)(111.69368262,0.95178228)(111.91243259,1.3371989)
\curveto(112.1363909,1.72261552)(112.24837005,2.09240714)(112.24837005,2.44657376)
\curveto(112.24837005,2.80074039)(112.15462006,3.10542785)(111.96712009,3.36063615)
\curveto(111.78482844,3.61063612)(111.54524514,3.76428193)(111.24837018,3.82157359)
\curveto(110.95670355,3.87365692)(110.45930777,3.89969858)(109.75618286,3.89969858)
\closepath
}
}
{
\newrgbcolor{curcolor}{0 0 0}
\pscustom[linestyle=none,fillstyle=solid,fillcolor=curcolor]
{
\newpath
\moveto(121.00618147,5.19657342)
\lineto(120.7561815,4.63407349)
\curveto(120.2145149,4.2851152)(119.73013996,4.05855273)(119.30305668,3.95438608)
\curveto(118.88118174,3.85021942)(118.50357762,3.7981361)(118.17024432,3.7981361)
\curveto(117.5452444,3.7981361)(116.95149447,3.92053192)(116.38899454,4.16532355)
\curveto(115.83170294,4.41011519)(115.37597383,4.82417764)(115.02180721,5.4075109)
\curveto(114.67284892,5.99084416)(114.49836978,6.69396907)(114.49836978,7.51688564)
\curveto(114.49836978,8.06376057)(114.5660781,8.55594801)(114.70149475,8.99344796)
\curveto(114.8369114,9.43615623)(114.97753638,9.76428119)(115.1233697,9.97782283)
\curveto(115.27441135,10.19136448)(115.52701548,10.42834361)(115.88118211,10.68876025)
\curveto(116.23534873,10.94917688)(116.61034868,11.15751019)(117.00618197,11.31376017)
\curveto(117.40201525,11.47001015)(117.82909853,11.54813514)(118.28743181,11.54813514)
\curveto(118.91243173,11.54813514)(119.46191083,11.39969766)(119.93586911,11.1028227)
\curveto(120.41503571,10.81115607)(120.75097317,10.43615611)(120.94368148,9.97782283)
\curveto(121.13638979,9.51948956)(121.23274395,9.03251045)(121.23274395,8.51688551)
\curveto(121.23274395,8.3554272)(121.22493145,8.19917722)(121.20930645,8.04813557)
\lineto(121.12336896,7.96219808)
\curveto(120.76920234,7.88407309)(120.29263989,7.83198977)(119.69368164,7.8059481)
\curveto(119.09472338,7.77990644)(118.69889009,7.76688561)(118.50618178,7.76688561)
\lineto(115.99836959,7.76688561)
\curveto(116.00878626,6.68876074)(116.27961956,5.89449001)(116.81086949,5.3840734)
\curveto(117.34211943,4.8736568)(117.99316101,4.6184485)(118.76399425,4.6184485)
\curveto(119.12857754,4.6184485)(119.47753583,4.68094849)(119.81086912,4.80594847)
\curveto(120.14941075,4.93094846)(120.50618154,5.1054276)(120.88118149,5.32938591)
\closepath
\moveto(115.99836959,8.39188553)
\curveto(116.09211958,8.37626053)(116.45149454,8.35803137)(117.07649446,8.33719804)
\curveto(117.70670271,8.31636471)(118.17284849,8.30594804)(118.47493179,8.30594804)
\curveto(119.19889003,8.30594804)(119.63899414,8.31896887)(119.79524412,8.34501054)
\curveto(119.80045246,8.47001052)(119.80305662,8.56636468)(119.80305662,8.634073)
\curveto(119.80305662,9.44136457)(119.63899414,10.04032283)(119.31086918,10.43094778)
\curveto(118.98274422,10.82678106)(118.53482761,11.02469771)(117.96711935,11.02469771)
\curveto(117.34732776,11.02469771)(116.86295282,10.80334357)(116.51399453,10.36063529)
\curveto(116.17024457,9.91792701)(115.99836959,9.26167709)(115.99836959,8.39188553)
\closepath
\moveto(118.17805682,11.98563509)
\closepath
\moveto(118.08430683,3.57938612)
\closepath
}
}
{
\newrgbcolor{curcolor}{0 0 0}
\pscustom[linestyle=none,fillstyle=solid,fillcolor=curcolor]
{
\newpath
\moveto(124.76399351,11.49344765)
\lineto(124.91243099,11.39188516)
\curveto(124.881181,11.08980186)(124.8577435,10.5533436)(124.8421185,9.78251036)
\lineto(125.42805593,10.52469777)
\curveto(125.62076424,10.7694894)(125.79003505,10.95698938)(125.93586837,11.0871977)
\curveto(126.08691001,11.22261435)(126.26138916,11.326781)(126.4593058,11.39969766)
\curveto(126.65722244,11.47261432)(126.86034742,11.50907265)(127.06868073,11.50907265)
\curveto(127.29784736,11.50907265)(127.51399317,11.46219765)(127.71711815,11.36844766)
\lineto(127.77180564,11.25907268)
\curveto(127.68847232,10.56636443)(127.64159732,9.9569895)(127.63118066,9.4309479)
\lineto(127.2796182,9.4309479)
\curveto(127.07128489,9.91011451)(126.73534743,10.14969781)(126.27180582,10.14969781)
\curveto(125.9488892,10.14969781)(125.66763923,10.04553116)(125.42805593,9.83719785)
\curveto(125.18847262,9.63407288)(125.02701431,9.37626041)(124.94368099,9.06376045)
\curveto(124.865556,8.75646882)(124.8264935,8.36584387)(124.8264935,7.89188559)
\lineto(124.8264935,7.0246982)
\curveto(124.8264935,6.86844822)(124.83691017,6.48563577)(124.8577435,5.87626084)
\curveto(124.87857683,5.26688592)(124.89941016,4.91532346)(124.92024349,4.82157347)
\curveto(124.94628515,4.72782348)(124.98274348,4.65751099)(125.02961848,4.610636)
\curveto(125.0817018,4.56896934)(125.14680596,4.54032351)(125.22493095,4.52469851)
\curveto(125.30826428,4.50907351)(125.68066006,4.48303185)(126.34211832,4.44657352)
\lineto(126.41243081,4.38407352)
\lineto(126.41243081,4.05594857)
\lineto(126.34211832,3.98563607)
\curveto(125.64941007,4.02730274)(124.92545182,4.04813607)(124.17024358,4.04813607)
\curveto(123.57128532,4.04813607)(122.9462854,4.02730274)(122.29524381,3.98563607)
\lineto(122.22493132,4.05594857)
\lineto(122.22493132,4.38407352)
\lineto(122.29524381,4.44657352)
\curveto(122.80045209,4.47782351)(123.08430622,4.50126101)(123.14680621,4.51688601)
\curveto(123.21451453,4.53251101)(123.27180619,4.563761)(123.31868119,4.610636)
\curveto(123.37076452,4.66271932)(123.40461868,4.73824015)(123.42024368,4.83719847)
\curveto(123.44107701,4.94136512)(123.46191034,5.27730258)(123.48274367,5.84501084)
\curveto(123.50878533,6.41792744)(123.52180616,6.85021905)(123.52180616,7.14188568)
\lineto(123.52180616,8.68094799)
\curveto(123.52180616,8.8892813)(123.5113895,9.17313543)(123.49055617,9.53251039)
\curveto(123.46972284,9.89188535)(123.45149367,10.11323948)(123.43586867,10.19657281)
\curveto(123.42545201,10.27990613)(123.38638951,10.33980196)(123.31868119,10.37626029)
\curveto(123.25097286,10.41792695)(123.11555621,10.43876028)(122.91243124,10.43876028)
\lineto(122.25618132,10.44657278)
\lineto(122.18586883,10.50907277)
\lineto(122.18586883,10.84501023)
\lineto(122.24836882,10.90751022)
\curveto(123.24316036,11.02730187)(124.08170193,11.22261435)(124.76399351,11.49344765)
\closepath
}
}
{
\newrgbcolor{curcolor}{0 0 0}
\pscustom[linestyle=none,fillstyle=solid,fillcolor=curcolor]
{
\newpath
\moveto(130.78743027,11.49344765)
\lineto(130.93586775,11.39188516)
\curveto(130.87857609,10.70959358)(130.84993026,9.84240618)(130.84993026,8.79032298)
\lineto(130.84993026,6.97001071)
\curveto(130.84993026,6.30334412)(130.89940942,5.83980251)(130.99836774,5.57938588)
\curveto(131.1025344,5.31896924)(131.27961771,5.11844843)(131.52961768,4.97782345)
\curveto(131.77961765,4.8424068)(132.07649261,4.77469848)(132.42024257,4.77469848)
\curveto(132.80045085,4.77469848)(133.14680498,4.84761513)(133.45930494,4.99344845)
\curveto(133.7718049,5.1444901)(134.03222153,5.35542757)(134.24055484,5.62626087)
\curveto(134.45409648,5.89709417)(134.5764923,6.07938582)(134.6077423,6.1731358)
\curveto(134.63899229,6.26688579)(134.65982562,6.52990659)(134.67024229,6.96219821)
\lineto(134.69367979,7.8371981)
\lineto(134.69367979,8.68094799)
\curveto(134.69367979,8.8892813)(134.68326312,9.17313543)(134.66242979,9.53251039)
\curveto(134.64159646,9.89188535)(134.62336729,10.11323948)(134.6077423,10.19657281)
\curveto(134.59732563,10.27990613)(134.55826314,10.33980196)(134.49055481,10.37626029)
\curveto(134.42284649,10.41792695)(134.28742984,10.43876028)(134.08430486,10.43876028)
\lineto(133.42805494,10.44657278)
\lineto(133.35774245,10.50907277)
\lineto(133.35774245,10.84501023)
\lineto(133.42024244,10.90751022)
\curveto(134.41503399,11.02730187)(135.25357555,11.22261435)(135.93586713,11.49344765)
\lineto(136.08430461,11.39188516)
\curveto(136.02701295,10.70959358)(135.99836712,9.84240618)(135.99836712,8.79032298)
\lineto(135.99836712,7.41532315)
\curveto(135.99836712,7.33719816)(136.01138796,6.69917741)(136.03742962,5.50126089)
\curveto(136.04784629,5.05855261)(136.07388795,4.78771931)(136.11555461,4.68876099)
\curveto(136.1624296,4.595011)(136.2249296,4.52730267)(136.30305459,4.48563601)
\curveto(136.38117958,4.44917768)(136.61295038,4.43094852)(136.998367,4.43094852)
\lineto(137.21711697,4.43094852)
\lineto(137.28742947,4.36844853)
\lineto(137.28742947,4.05594857)
\lineto(137.22492947,3.98563607)
\curveto(136.49055456,4.02730274)(135.99836712,4.04813607)(135.74836716,4.04813607)
\curveto(135.43065886,4.04813607)(135.06347141,4.0299069)(134.64680479,3.99344857)
\lineto(134.5764923,4.05594857)
\curveto(134.6077423,4.5871985)(134.63117979,5.03771928)(134.64680479,5.4075109)
\curveto(134.31867983,5.1523026)(133.94367988,4.81115681)(133.52180493,4.38407352)
\curveto(133.36034662,4.22261521)(133.12857581,4.08719856)(132.82649252,3.97782357)
\curveto(132.52440922,3.86844859)(132.18065926,3.81376109)(131.79524264,3.81376109)
\curveto(131.21190938,3.81376109)(130.75618027,3.90490692)(130.42805531,4.08719856)
\curveto(130.10513868,4.27469854)(129.87597205,4.52990684)(129.7405554,4.85282347)
\curveto(129.60513875,5.18094843)(129.53743042,5.74344836)(129.53743042,6.54032326)
\lineto(129.54524292,7.15751068)
\lineto(129.54524292,8.68094799)
\curveto(129.54524292,8.8892813)(129.53482626,9.17313543)(129.51399292,9.53251039)
\curveto(129.49315959,9.89188535)(129.47493043,10.11323948)(129.45930543,10.19657281)
\curveto(129.44888877,10.27990613)(129.40982627,10.33980196)(129.34211795,10.37626029)
\curveto(129.27440962,10.41792695)(129.13899297,10.43876028)(128.935868,10.43876028)
\lineto(128.27961808,10.44657278)
\lineto(128.20930559,10.50907277)
\lineto(128.20930559,10.84501023)
\lineto(128.27180558,10.90751022)
\curveto(129.26659712,11.02730187)(130.10513868,11.22261435)(130.78743027,11.49344765)
\closepath
\moveto(132.65461754,11.98563509)
\closepath
\moveto(132.73274253,3.57938612)
\closepath
}
}
{
\newrgbcolor{curcolor}{0 0 0}
\pscustom[linestyle=none,fillstyle=solid,fillcolor=curcolor]
{
\newpath
\moveto(140.40461658,11.49344765)
\lineto(140.55305406,11.39188516)
\curveto(140.52180407,11.04292687)(140.50097074,10.58459359)(140.49055407,10.01688533)
\curveto(140.90722069,10.36063529)(141.30305397,10.70959358)(141.67805392,11.0637602)
\curveto(141.78742891,11.16271852)(141.8889914,11.23563518)(141.98274139,11.28251017)
\curveto(142.08169971,11.32938517)(142.24315802,11.37886433)(142.46711633,11.43094766)
\curveto(142.69107463,11.48303098)(142.92024127,11.50907265)(143.15461624,11.50907265)
\curveto(143.55044953,11.50907265)(143.93326198,11.42834349)(144.3030536,11.26688518)
\curveto(144.67805355,11.10542686)(144.95930352,10.91271855)(145.1468035,10.68876025)
\curveto(145.33951181,10.47001027)(145.47232429,10.20959364)(145.54524095,9.90751034)
\curveto(145.61815761,9.60542705)(145.65461593,9.23303126)(145.65461593,8.79032298)
\lineto(145.65461593,7.41532315)
\curveto(145.65461593,7.3267815)(145.66763677,6.67834408)(145.69367843,5.47001089)
\curveto(145.70409509,4.97521929)(145.75878259,4.68094849)(145.85774091,4.5871985)
\curveto(145.95669923,4.49344851)(146.27961586,4.44657352)(146.82649079,4.44657352)
\lineto(146.88899078,4.38407352)
\lineto(146.88899078,4.04813607)
\lineto(146.82649079,3.98563607)
\curveto(146.16503254,4.02730274)(145.71711593,4.04813607)(145.48274096,4.04813607)
\curveto(145.34732431,4.04813607)(144.96711602,4.02730274)(144.3421161,3.98563607)
\lineto(144.24836611,4.07157356)
\curveto(144.31607443,4.74865681)(144.3499286,5.6314692)(144.3499286,6.72001074)
\lineto(144.3499286,7.74344811)
\curveto(144.3499286,8.35803137)(144.3343036,8.80073965)(144.3030536,9.07157295)
\curveto(144.27701194,9.34240625)(144.19107445,9.58719788)(144.04524113,9.80594786)
\curveto(143.89940782,10.02990616)(143.70409534,10.20178114)(143.4593037,10.32157279)
\curveto(143.21451207,10.44657278)(142.92284544,10.50907277)(142.58430381,10.50907277)
\curveto(142.31347051,10.50907277)(142.08169971,10.48042694)(141.8889914,10.42313528)
\curveto(141.69628309,10.37105195)(141.48013728,10.2590728)(141.24055398,10.08719782)
\curveto(141.00097067,9.92053118)(140.82388736,9.76167703)(140.70930404,9.61063538)
\curveto(140.59472072,9.46480206)(140.52440823,9.32678125)(140.49836657,9.19657293)
\curveto(140.47753324,9.07157295)(140.46711657,8.80073965)(140.46711657,8.38407303)
\lineto(140.46711657,7.0246982)
\curveto(140.46711657,6.91011488)(140.47753324,6.54032326)(140.49836657,5.91532334)
\curveto(140.5191999,5.29553175)(140.54003323,4.93355262)(140.56086656,4.82938597)
\curveto(140.58690823,4.72521932)(140.62336655,4.64969849)(140.67024155,4.6028235)
\curveto(140.71711654,4.5559485)(140.7744082,4.52469851)(140.84211653,4.50907351)
\curveto(140.90982485,4.49865684)(141.19107482,4.47782351)(141.68586642,4.44657352)
\lineto(141.75617891,4.38407352)
\lineto(141.75617891,4.05594857)
\lineto(141.69367892,3.98563607)
\curveto(141.04263734,4.02730274)(140.41503325,4.04813607)(139.81086665,4.04813607)
\curveto(139.2119084,4.04813607)(138.58690847,4.02730274)(137.93586689,3.98563607)
\lineto(137.86555439,4.05594857)
\lineto(137.86555439,4.38407352)
\lineto(137.93586689,4.44657352)
\curveto(138.44107516,4.47782351)(138.72492929,4.50126101)(138.78742928,4.51688601)
\curveto(138.85513761,4.53251101)(138.91242927,4.563761)(138.95930426,4.610636)
\curveto(139.01138759,4.66271932)(139.04524175,4.73824015)(139.06086675,4.83719847)
\curveto(139.08170008,4.94136512)(139.10253341,5.27730258)(139.12336674,5.84501084)
\curveto(139.1494084,6.41792744)(139.16242923,6.85021905)(139.16242923,7.14188568)
\lineto(139.16242923,8.68094799)
\curveto(139.16242923,8.8892813)(139.15201257,9.17313543)(139.13117924,9.53251039)
\curveto(139.11034591,9.89188535)(139.09211674,10.11323948)(139.07649175,10.19657281)
\curveto(139.06607508,10.27990613)(139.02701258,10.33980196)(138.95930426,10.37626029)
\curveto(138.89159593,10.41792695)(138.75617928,10.43876028)(138.55305431,10.43876028)
\lineto(137.89680439,10.44657278)
\lineto(137.8264919,10.50907277)
\lineto(137.8264919,10.84501023)
\lineto(137.88899189,10.90751022)
\curveto(138.88378344,11.02730187)(139.722325,11.22261435)(140.40461658,11.49344765)
\closepath
}
}
{
\newrgbcolor{curcolor}{0 0 0}
\pscustom[linestyle=none,fillstyle=solid,fillcolor=curcolor]
{
\newpath
\moveto(155.6624272,10.95438521)
\lineto(155.72492719,10.81376023)
\curveto(155.51138555,10.48563527)(155.37857307,10.27209363)(155.32648974,10.17313531)
\lineto(153.87336492,10.17313531)
\curveto(154.01919824,9.88667701)(154.09211489,9.58719788)(154.09211489,9.27469792)
\curveto(154.09211489,8.9049063)(154.0061774,8.54553134)(153.83430243,8.19657305)
\curveto(153.66763578,7.84761476)(153.43326081,7.5507398)(153.13117751,7.30594816)
\curveto(152.82909422,7.06115653)(152.48534426,6.86584405)(152.09992764,6.72001074)
\curveto(151.71971935,6.57417742)(151.28482357,6.50126076)(150.7952403,6.50126076)
\lineto(150.43586534,6.50126076)
\curveto(150.12857372,6.25646913)(149.93065707,6.07417748)(149.84211542,5.95438583)
\curveto(149.75357376,5.83459418)(149.70930293,5.71219836)(149.70930293,5.58719838)
\curveto(149.70930293,5.35803174)(149.81346959,5.19657342)(150.0218029,5.10282344)
\curveto(150.23534454,5.00907345)(150.67284448,4.96219845)(151.33430273,4.96219845)
\lineto(153.08430252,4.98563595)
\curveto(153.63638578,4.98563595)(154.05826073,4.92053179)(154.34992736,4.79032347)
\curveto(154.64680233,4.66011516)(154.88638563,4.42574019)(155.06867727,4.08719856)
\curveto(155.25096892,3.75386527)(155.34211474,3.39969865)(155.34211474,3.02469869)
\curveto(155.34211474,2.4517821)(155.15461476,1.88146967)(154.77961481,1.3137614)
\curveto(154.40982319,0.74084481)(153.87336492,0.30594903)(153.17024001,0.00907406)
\curveto(152.46711509,-0.2878009)(151.71711519,-0.43623838)(150.92024028,-0.43623838)
\curveto(150.46711534,-0.43623838)(150.04003206,-0.38675922)(149.63899044,-0.2878009)
\curveto(149.24315716,-0.18884258)(148.89419887,-0.03780093)(148.59211557,0.16532404)
\curveto(148.29524061,0.36324069)(148.0556573,0.62365732)(147.87336566,0.94657395)
\curveto(147.69107402,1.26949058)(147.59992819,1.60542803)(147.59992819,1.95438632)
\curveto(147.59992819,2.14709463)(147.62857402,2.35021961)(147.68586568,2.56376125)
\curveto(147.74315734,2.77209456)(147.85253233,2.99865703)(148.01399064,3.24344867)
\lineto(149.42805297,4.03251107)
\curveto(148.97492802,4.17313605)(148.68846973,4.31115687)(148.56867807,4.44657352)
\curveto(148.45409476,4.5871985)(148.3968031,4.75386515)(148.3968031,4.94657346)
\curveto(148.3968031,5.1444901)(148.46190725,5.37886507)(148.59211557,5.64969837)
\lineto(149.84992792,6.55594826)
\curveto(149.11034467,6.74865657)(148.61034474,7.03771903)(148.3499281,7.42313565)
\curveto(148.0947198,7.80855227)(147.96711565,8.23042722)(147.96711565,8.68876049)
\curveto(147.96711565,9.11063544)(148.05826147,9.51167706)(148.24055312,9.89188535)
\curveto(148.42284476,10.27730196)(148.69628223,10.58459359)(149.06086551,10.81376023)
\curveto(149.43065714,11.04292687)(149.84211542,11.22261435)(150.29524036,11.35282267)
\curveto(150.75357364,11.48823932)(151.17024025,11.55594764)(151.54524021,11.55594764)
\curveto(152.19628179,11.55594764)(152.81607338,11.34501017)(153.40461498,10.92313522)
\curveto(154.3473232,10.92313522)(155.09992727,10.93355188)(155.6624272,10.95438521)
\closepath
\moveto(149.31086548,9.14188544)
\curveto(149.31086548,8.82417714)(149.37336548,8.48303135)(149.49836546,8.11844806)
\curveto(149.62336544,7.75386478)(149.82388625,7.47782314)(150.09992789,7.29032317)
\curveto(150.38117785,7.10282319)(150.69628198,7.0090732)(151.04524027,7.0090732)
\curveto(151.50878188,7.0090732)(151.90721933,7.16011485)(152.24055262,7.46219815)
\curveto(152.57909425,7.76428144)(152.74836506,8.23563555)(152.74836506,8.87626047)
\curveto(152.74836506,9.41792707)(152.59471925,9.90490618)(152.28742762,10.33719779)
\curveto(151.98013599,10.7694894)(151.54524021,10.98563521)(150.98274028,10.98563521)
\curveto(150.508782,10.98563521)(150.11034455,10.82938523)(149.78742792,10.51688527)
\curveto(149.46971963,10.20959364)(149.31086548,9.75126036)(149.31086548,9.14188544)
\closepath
\moveto(151.5999277,3.89969858)
\curveto(150.76138614,3.89969858)(150.2770112,3.88928192)(150.14680288,3.86844859)
\curveto(150.0218029,3.84761526)(149.84211542,3.7590736)(149.60774045,3.60282362)
\curveto(149.37336548,3.44657364)(149.18065717,3.23563617)(149.02961552,2.9700112)
\curveto(148.8837822,2.6991779)(148.81086555,2.3970946)(148.81086555,2.06376131)
\curveto(148.81086555,1.47521972)(149.02961552,0.99605311)(149.46711546,0.62626149)
\curveto(149.90461541,0.25646987)(150.48794867,0.07157406)(151.21711525,0.07157406)
\curveto(151.76399018,0.07157406)(152.26919845,0.18615738)(152.73274006,0.41532401)
\curveto(153.19628167,0.64449065)(153.53742746,0.95178228)(153.75617744,1.3371989)
\curveto(153.98013574,1.72261552)(154.09211489,2.09240714)(154.09211489,2.44657376)
\curveto(154.09211489,2.80074039)(153.99836491,3.10542785)(153.81086493,3.36063615)
\curveto(153.62857328,3.61063612)(153.38898998,3.76428193)(153.09211502,3.82157359)
\curveto(152.80044839,3.87365692)(152.30305261,3.89969858)(151.5999277,3.89969858)
\closepath
}
}
{
\newrgbcolor{curcolor}{0 0 0}
\pscustom[linestyle=none,fillstyle=solid,fillcolor=curcolor]
{
\newpath
\moveto(162.84992631,5.19657342)
\lineto(162.59992634,4.63407349)
\curveto(162.05825974,4.2851152)(161.5738848,4.05855273)(161.14680152,3.95438608)
\curveto(160.72492658,3.85021942)(160.34732246,3.7981361)(160.01398916,3.7981361)
\curveto(159.38898924,3.7981361)(158.79523931,3.92053192)(158.23273938,4.16532355)
\curveto(157.67544779,4.41011519)(157.21971867,4.82417764)(156.86555205,5.4075109)
\curveto(156.51659376,5.99084416)(156.34211462,6.69396907)(156.34211462,7.51688564)
\curveto(156.34211462,8.06376057)(156.40982294,8.55594801)(156.54523959,8.99344796)
\curveto(156.68065624,9.43615623)(156.82128122,9.76428119)(156.96711454,9.97782283)
\curveto(157.11815619,10.19136448)(157.37076032,10.42834361)(157.72492695,10.68876025)
\curveto(158.07909357,10.94917688)(158.45409352,11.15751019)(158.84992681,11.31376017)
\curveto(159.24576009,11.47001015)(159.67284337,11.54813514)(160.13117665,11.54813514)
\curveto(160.75617657,11.54813514)(161.30565567,11.39969766)(161.77961395,11.1028227)
\curveto(162.25878055,10.81115607)(162.59471801,10.43615611)(162.78742632,9.97782283)
\curveto(162.98013463,9.51948956)(163.07648879,9.03251045)(163.07648879,8.51688551)
\curveto(163.07648879,8.3554272)(163.06867629,8.19917722)(163.05305129,8.04813557)
\lineto(162.9671138,7.96219808)
\curveto(162.61294718,7.88407309)(162.13638474,7.83198977)(161.53742648,7.8059481)
\curveto(160.93846822,7.77990644)(160.54263493,7.76688561)(160.34992662,7.76688561)
\lineto(157.84211443,7.76688561)
\curveto(157.8525311,6.68876074)(158.1233644,5.89449001)(158.65461433,5.3840734)
\curveto(159.18586427,4.8736568)(159.83690585,4.6184485)(160.60773909,4.6184485)
\curveto(160.97232238,4.6184485)(161.32128067,4.68094849)(161.65461396,4.80594847)
\curveto(161.99315559,4.93094846)(162.34992638,5.1054276)(162.72492633,5.32938591)
\closepath
\moveto(157.84211443,8.39188553)
\curveto(157.93586442,8.37626053)(158.29523938,8.35803137)(158.9202393,8.33719804)
\curveto(159.55044755,8.31636471)(160.01659333,8.30594804)(160.31867663,8.30594804)
\curveto(161.04263487,8.30594804)(161.48273898,8.31896887)(161.63898896,8.34501054)
\curveto(161.6441973,8.47001052)(161.64680146,8.56636468)(161.64680146,8.634073)
\curveto(161.64680146,9.44136457)(161.48273898,10.04032283)(161.15461402,10.43094778)
\curveto(160.82648906,10.82678106)(160.37857245,11.02469771)(159.81086419,11.02469771)
\curveto(159.1910726,11.02469771)(158.70669766,10.80334357)(158.35773937,10.36063529)
\curveto(158.01398941,9.91792701)(157.84211443,9.26167709)(157.84211443,8.39188553)
\closepath
\moveto(160.02180166,11.98563509)
\closepath
\moveto(159.92805167,3.57938612)
\closepath
}
}
{
\newrgbcolor{curcolor}{0 0 0}
\pscustom[linestyle=none,fillstyle=solid,fillcolor=curcolor]
{
\newpath
\moveto(166.27961339,11.49344765)
\lineto(166.42805087,11.39188516)
\curveto(166.39680088,11.04292687)(166.37596755,10.58459359)(166.36555088,10.01688533)
\curveto(166.7822175,10.36063529)(167.17805078,10.70959358)(167.55305073,11.0637602)
\curveto(167.66242572,11.16271852)(167.76398821,11.23563518)(167.8577382,11.28251017)
\curveto(167.95669652,11.32938517)(168.11815483,11.37886433)(168.34211314,11.43094766)
\curveto(168.56607144,11.48303098)(168.79523808,11.50907265)(169.02961305,11.50907265)
\curveto(169.42544634,11.50907265)(169.80825879,11.42834349)(170.17805041,11.26688518)
\curveto(170.55305036,11.10542686)(170.83430033,10.91271855)(171.02180031,10.68876025)
\curveto(171.21450862,10.47001027)(171.3473211,10.20959364)(171.42023776,9.90751034)
\curveto(171.49315441,9.60542705)(171.52961274,9.23303126)(171.52961274,8.79032298)
\lineto(171.52961274,7.41532315)
\curveto(171.52961274,7.3267815)(171.54263358,6.67834408)(171.56867524,5.47001089)
\curveto(171.5790919,4.97521929)(171.6337794,4.68094849)(171.73273772,4.5871985)
\curveto(171.83169604,4.49344851)(172.15461267,4.44657352)(172.7014876,4.44657352)
\lineto(172.76398759,4.38407352)
\lineto(172.76398759,4.04813607)
\lineto(172.7014876,3.98563607)
\curveto(172.04002935,4.02730274)(171.59211274,4.04813607)(171.35773776,4.04813607)
\curveto(171.22232112,4.04813607)(170.84211283,4.02730274)(170.21711291,3.98563607)
\lineto(170.12336292,4.07157356)
\curveto(170.19107124,4.74865681)(170.2249254,5.6314692)(170.2249254,6.72001074)
\lineto(170.2249254,7.74344811)
\curveto(170.2249254,8.35803137)(170.20930041,8.80073965)(170.17805041,9.07157295)
\curveto(170.15200875,9.34240625)(170.06607126,9.58719788)(169.92023794,9.80594786)
\curveto(169.77440463,10.02990616)(169.57909215,10.20178114)(169.33430051,10.32157279)
\curveto(169.08950888,10.44657278)(168.79784225,10.50907277)(168.45930062,10.50907277)
\curveto(168.18846732,10.50907277)(167.95669652,10.48042694)(167.76398821,10.42313528)
\curveto(167.5712799,10.37105195)(167.35513409,10.2590728)(167.11555079,10.08719782)
\curveto(166.87596748,9.92053118)(166.69888417,9.76167703)(166.58430085,9.61063538)
\curveto(166.46971753,9.46480206)(166.39940504,9.32678125)(166.37336338,9.19657293)
\curveto(166.35253005,9.07157295)(166.34211338,8.80073965)(166.34211338,8.38407303)
\lineto(166.34211338,7.0246982)
\curveto(166.34211338,6.91011488)(166.35253005,6.54032326)(166.37336338,5.91532334)
\curveto(166.39419671,5.29553175)(166.41503004,4.93355262)(166.43586337,4.82938597)
\curveto(166.46190504,4.72521932)(166.49836336,4.64969849)(166.54523836,4.6028235)
\curveto(166.59211335,4.5559485)(166.64940501,4.52469851)(166.71711334,4.50907351)
\curveto(166.78482166,4.49865684)(167.06607163,4.47782351)(167.56086323,4.44657352)
\lineto(167.63117572,4.38407352)
\lineto(167.63117572,4.05594857)
\lineto(167.56867573,3.98563607)
\curveto(166.91763415,4.02730274)(166.29003006,4.04813607)(165.68586346,4.04813607)
\curveto(165.0869052,4.04813607)(164.46190528,4.02730274)(163.8108637,3.98563607)
\lineto(163.7405512,4.05594857)
\lineto(163.7405512,4.38407352)
\lineto(163.8108637,4.44657352)
\curveto(164.31607197,4.47782351)(164.5999261,4.50126101)(164.66242609,4.51688601)
\curveto(164.73013442,4.53251101)(164.78742608,4.563761)(164.83430107,4.610636)
\curveto(164.8863844,4.66271932)(164.92023856,4.73824015)(164.93586356,4.83719847)
\curveto(164.95669689,4.94136512)(164.97753022,5.27730258)(164.99836355,5.84501084)
\curveto(165.02440521,6.41792744)(165.03742604,6.85021905)(165.03742604,7.14188568)
\lineto(165.03742604,8.68094799)
\curveto(165.03742604,8.8892813)(165.02700938,9.17313543)(165.00617605,9.53251039)
\curveto(164.98534272,9.89188535)(164.96711355,10.11323948)(164.95148855,10.19657281)
\curveto(164.94107189,10.27990613)(164.90200939,10.33980196)(164.83430107,10.37626029)
\curveto(164.76659274,10.41792695)(164.63117609,10.43876028)(164.42805112,10.43876028)
\lineto(163.7718012,10.44657278)
\lineto(163.70148871,10.50907277)
\lineto(163.70148871,10.84501023)
\lineto(163.7639887,10.90751022)
\curveto(164.75878025,11.02730187)(165.59732181,11.22261435)(166.27961339,11.49344765)
\closepath
}
}
{
\newrgbcolor{curcolor}{0 0 0}
\pscustom[linestyle=none,fillstyle=solid,fillcolor=curcolor]
{
}
}
{
\newrgbcolor{curcolor}{0 0 0}
\pscustom[linestyle=none,fillstyle=solid,fillcolor=curcolor]
{
\newpath
\moveto(179.62336175,11.49344765)
\lineto(179.77179923,11.39188516)
\curveto(179.71450757,10.70959358)(179.68586174,9.84240618)(179.68586174,8.79032298)
\lineto(179.68586174,6.97001071)
\curveto(179.68586174,6.30334412)(179.7353409,5.83980251)(179.83429922,5.57938588)
\curveto(179.93846587,5.31896924)(180.11554919,5.11844843)(180.36554915,4.97782345)
\curveto(180.61554912,4.8424068)(180.91242409,4.77469848)(181.25617404,4.77469848)
\curveto(181.63638233,4.77469848)(181.98273645,4.84761513)(182.29523642,4.99344845)
\curveto(182.60773638,5.1444901)(182.86815301,5.35542757)(183.07648632,5.62626087)
\curveto(183.29002796,5.89709417)(183.41242378,6.07938582)(183.44367377,6.1731358)
\curveto(183.47492377,6.26688579)(183.4957571,6.52990659)(183.50617377,6.96219821)
\lineto(183.52961126,7.8371981)
\lineto(183.52961126,8.68094799)
\curveto(183.52961126,8.8892813)(183.5191946,9.17313543)(183.49836127,9.53251039)
\curveto(183.47752794,9.89188535)(183.45929877,10.11323948)(183.44367377,10.19657281)
\curveto(183.43325711,10.27990613)(183.39419461,10.33980196)(183.32648629,10.37626029)
\curveto(183.25877796,10.41792695)(183.12336131,10.43876028)(182.92023634,10.43876028)
\lineto(182.26398642,10.44657278)
\lineto(182.19367393,10.50907277)
\lineto(182.19367393,10.84501023)
\lineto(182.25617392,10.90751022)
\curveto(183.25096547,11.02730187)(184.08950703,11.22261435)(184.77179861,11.49344765)
\lineto(184.92023609,11.39188516)
\curveto(184.86294443,10.70959358)(184.8342986,9.84240618)(184.8342986,8.79032298)
\lineto(184.8342986,7.41532315)
\curveto(184.8342986,7.33719816)(184.84731944,6.69917741)(184.8733611,5.50126089)
\curveto(184.88377776,5.05855261)(184.90981943,4.78771931)(184.95148609,4.68876099)
\curveto(184.99836108,4.595011)(185.06086108,4.52730267)(185.13898607,4.48563601)
\curveto(185.21711106,4.44917768)(185.44888186,4.43094852)(185.83429848,4.43094852)
\lineto(186.05304845,4.43094852)
\lineto(186.12336094,4.36844853)
\lineto(186.12336094,4.05594857)
\lineto(186.06086095,3.98563607)
\curveto(185.32648604,4.02730274)(184.8342986,4.04813607)(184.58429863,4.04813607)
\curveto(184.26659034,4.04813607)(183.89940289,4.0299069)(183.48273627,3.99344857)
\lineto(183.41242378,4.05594857)
\curveto(183.44367377,4.5871985)(183.46711127,5.03771928)(183.48273627,5.4075109)
\curveto(183.15461131,5.1523026)(182.77961136,4.81115681)(182.35773641,4.38407352)
\curveto(182.1962781,4.22261521)(181.96450729,4.08719856)(181.66242399,3.97782357)
\curveto(181.3603407,3.86844859)(181.01659074,3.81376109)(180.63117412,3.81376109)
\curveto(180.04784086,3.81376109)(179.59211175,3.90490692)(179.26398679,4.08719856)
\curveto(178.94107016,4.27469854)(178.71190352,4.52990684)(178.57648687,4.85282347)
\curveto(178.44107022,5.18094843)(178.3733619,5.74344836)(178.3733619,6.54032326)
\lineto(178.3811744,7.15751068)
\lineto(178.3811744,8.68094799)
\curveto(178.3811744,8.8892813)(178.37075773,9.17313543)(178.3499244,9.53251039)
\curveto(178.32909107,9.89188535)(178.31086191,10.11323948)(178.29523691,10.19657281)
\curveto(178.28482024,10.27990613)(178.24575775,10.33980196)(178.17804942,10.37626029)
\curveto(178.1103411,10.41792695)(177.97492445,10.43876028)(177.77179947,10.43876028)
\lineto(177.11554956,10.44657278)
\lineto(177.04523706,10.50907277)
\lineto(177.04523706,10.84501023)
\lineto(177.10773706,10.90751022)
\curveto(178.1025286,11.02730187)(178.94107016,11.22261435)(179.62336175,11.49344765)
\closepath
\moveto(181.49054902,11.98563509)
\closepath
\moveto(181.56867401,3.57938612)
\closepath
}
}
{
\newrgbcolor{curcolor}{0 0 0}
\pscustom[linestyle=none,fillstyle=solid,fillcolor=curcolor]
{
\newpath
\moveto(189.24054806,11.49344765)
\lineto(189.38898554,11.39188516)
\curveto(189.35773555,11.04292687)(189.33690221,10.58459359)(189.32648555,10.01688533)
\curveto(189.74315216,10.36063529)(190.13898545,10.70959358)(190.5139854,11.0637602)
\curveto(190.62336039,11.16271852)(190.72492288,11.23563518)(190.81867287,11.28251017)
\curveto(190.91763119,11.32938517)(191.0790895,11.37886433)(191.30304781,11.43094766)
\curveto(191.52700611,11.48303098)(191.75617275,11.50907265)(191.99054772,11.50907265)
\curveto(192.38638101,11.50907265)(192.76919346,11.42834349)(193.13898508,11.26688518)
\curveto(193.51398503,11.10542686)(193.795235,10.91271855)(193.98273498,10.68876025)
\curveto(194.17544328,10.47001027)(194.30825577,10.20959364)(194.38117243,9.90751034)
\curveto(194.45408908,9.60542705)(194.49054741,9.23303126)(194.49054741,8.79032298)
\lineto(194.49054741,7.41532315)
\curveto(194.49054741,7.3267815)(194.50356824,6.67834408)(194.52960991,5.47001089)
\curveto(194.54002657,4.97521929)(194.59471407,4.68094849)(194.69367239,4.5871985)
\curveto(194.79263071,4.49344851)(195.11554734,4.44657352)(195.66242227,4.44657352)
\lineto(195.72492226,4.38407352)
\lineto(195.72492226,4.04813607)
\lineto(195.66242227,3.98563607)
\curveto(195.00096402,4.02730274)(194.5530474,4.04813607)(194.31867243,4.04813607)
\curveto(194.18325578,4.04813607)(193.8030475,4.02730274)(193.17804757,3.98563607)
\lineto(193.08429759,4.07157356)
\curveto(193.15200591,4.74865681)(193.18586007,5.6314692)(193.18586007,6.72001074)
\lineto(193.18586007,7.74344811)
\curveto(193.18586007,8.35803137)(193.17023508,8.80073965)(193.13898508,9.07157295)
\curveto(193.11294342,9.34240625)(193.02700593,9.58719788)(192.88117261,9.80594786)
\curveto(192.7353393,10.02990616)(192.54002682,10.20178114)(192.29523518,10.32157279)
\curveto(192.05044355,10.44657278)(191.75877692,10.50907277)(191.42023529,10.50907277)
\curveto(191.14940199,10.50907277)(190.91763119,10.48042694)(190.72492288,10.42313528)
\curveto(190.53221457,10.37105195)(190.31606876,10.2590728)(190.07648546,10.08719782)
\curveto(189.83690215,9.92053118)(189.65981884,9.76167703)(189.54523552,9.61063538)
\curveto(189.4306522,9.46480206)(189.36033971,9.32678125)(189.33429805,9.19657293)
\curveto(189.31346472,9.07157295)(189.30304805,8.80073965)(189.30304805,8.38407303)
\lineto(189.30304805,7.0246982)
\curveto(189.30304805,6.91011488)(189.31346472,6.54032326)(189.33429805,5.91532334)
\curveto(189.35513138,5.29553175)(189.37596471,4.93355262)(189.39679804,4.82938597)
\curveto(189.4228397,4.72521932)(189.45929803,4.64969849)(189.50617303,4.6028235)
\curveto(189.55304802,4.5559485)(189.61033968,4.52469851)(189.67804801,4.50907351)
\curveto(189.74575633,4.49865684)(190.0270063,4.47782351)(190.5217979,4.44657352)
\lineto(190.59211039,4.38407352)
\lineto(190.59211039,4.05594857)
\lineto(190.5296104,3.98563607)
\curveto(189.87856881,4.02730274)(189.25096473,4.04813607)(188.64679813,4.04813607)
\curveto(188.04783987,4.04813607)(187.42283995,4.02730274)(186.77179836,3.98563607)
\lineto(186.70148587,4.05594857)
\lineto(186.70148587,4.38407352)
\lineto(186.77179836,4.44657352)
\curveto(187.27700664,4.47782351)(187.56086077,4.50126101)(187.62336076,4.51688601)
\curveto(187.69106908,4.53251101)(187.74836074,4.563761)(187.79523574,4.610636)
\curveto(187.84731907,4.66271932)(187.88117323,4.73824015)(187.89679823,4.83719847)
\curveto(187.91763156,4.94136512)(187.93846489,5.27730258)(187.95929822,5.84501084)
\curveto(187.98533988,6.41792744)(187.99836071,6.85021905)(187.99836071,7.14188568)
\lineto(187.99836071,8.68094799)
\curveto(187.99836071,8.8892813)(187.98794405,9.17313543)(187.96711072,9.53251039)
\curveto(187.94627739,9.89188535)(187.92804822,10.11323948)(187.91242322,10.19657281)
\curveto(187.90200656,10.27990613)(187.86294406,10.33980196)(187.79523574,10.37626029)
\curveto(187.72752741,10.41792695)(187.59211076,10.43876028)(187.38898579,10.43876028)
\lineto(186.73273587,10.44657278)
\lineto(186.66242338,10.50907277)
\lineto(186.66242338,10.84501023)
\lineto(186.72492337,10.90751022)
\curveto(187.71971491,11.02730187)(188.55825648,11.22261435)(189.24054806,11.49344765)
\closepath
}
}
{
\newrgbcolor{curcolor}{0 0 0}
\pscustom[linestyle=none,fillstyle=solid,fillcolor=curcolor]
{
\newpath
\moveto(201.13898409,14.64188476)
\lineto(201.13898409,14.97000972)
\lineto(201.20148409,15.04032221)
\curveto(202.19627563,15.15490553)(203.03481719,15.34761384)(203.71710877,15.61844714)
\lineto(203.86554626,15.52469715)
\curveto(203.8082546,14.92573889)(203.77960877,13.72521821)(203.77960877,11.92313509)
\lineto(203.77960877,6.80594823)
\curveto(203.77960877,6.27990662)(203.78742127,5.81896918)(203.80304626,5.4231359)
\curveto(203.8238796,5.03251094)(203.85773376,4.80074014)(203.90460875,4.72782348)
\curveto(203.95148375,4.65490682)(204.02700457,4.595011)(204.13117122,4.548136)
\curveto(204.24054621,4.50126101)(204.54523367,4.45699018)(205.04523361,4.41532352)
\lineto(205.1155461,4.35282353)
\lineto(205.1155461,4.05594857)
\lineto(205.04523361,3.98563607)
\curveto(204.48273368,4.0220944)(204.04523373,4.04032357)(203.73273377,4.04032357)
\curveto(203.45148381,4.04032357)(203.05044219,4.0220944)(202.52960892,3.98563607)
\lineto(202.44367143,4.07157356)
\curveto(202.45929643,4.58199017)(202.46710893,4.93876096)(202.46710893,5.14188593)
\curveto(202.46710893,5.16792759)(202.4697131,5.26688592)(202.47492143,5.43876089)
\curveto(202.18846313,5.23042759)(201.90200483,4.99865678)(201.61554653,4.74344848)
\curveto(201.19367159,4.36844853)(200.91762995,4.13407356)(200.78742164,4.04032357)
\curveto(200.54783833,3.92053192)(200.20669254,3.86063609)(199.76398426,3.86063609)
\curveto(199.04523435,3.86063609)(198.43325526,4.0377194)(197.92804699,4.39188602)
\curveto(197.42804705,4.74605265)(197.07127626,5.19657342)(196.85773462,5.74344836)
\curveto(196.64419298,6.29553162)(196.53742216,6.86063572)(196.53742216,7.43876065)
\curveto(196.53742216,8.02730224)(196.64679715,8.59240634)(196.86554712,9.13407294)
\curveto(197.08950543,9.68094787)(197.40460955,10.07938532)(197.8108595,10.32938529)
\curveto(198.22231779,10.57938526)(198.66763023,10.83719773)(199.14679684,11.1028227)
\curveto(199.63117178,11.373656)(200.11033839,11.50907265)(200.58429666,11.50907265)
\curveto(201.24575491,11.50907265)(201.87596317,11.34501017)(202.47492143,11.01688521)
\lineto(202.44367143,12.75907249)
\curveto(202.43325477,13.36323908)(202.41762977,13.78250987)(202.39679644,14.01688484)
\curveto(202.37596311,14.25646814)(202.34992144,14.39709312)(202.31867145,14.43875978)
\curveto(202.28742145,14.48563478)(202.23273396,14.51948894)(202.15460897,14.54032227)
\curveto(202.08169231,14.5611556)(201.76658818,14.57157227)(201.20929658,14.57157227)
\closepath
\moveto(202.47492143,9.68094787)
\curveto(202.1520048,10.04553116)(201.79262985,10.32417696)(201.39679656,10.51688527)
\curveto(201.00617161,10.70959358)(200.61294249,10.80594773)(200.21710921,10.80594773)
\curveto(199.77960926,10.80594773)(199.37075514,10.68615608)(198.99054686,10.44657278)
\curveto(198.6155469,10.21219781)(198.3447136,9.86063535)(198.17804696,9.39188541)
\curveto(198.01138031,8.9283438)(197.92804699,8.42313553)(197.92804699,7.87626059)
\curveto(197.92804699,6.96480237)(198.15721363,6.23303163)(198.6155469,5.68094836)
\curveto(199.07908851,5.1288651)(199.65981761,4.85282347)(200.35773419,4.85282347)
\curveto(200.73794248,4.85282347)(201.07387993,4.93876096)(201.36554656,5.11063593)
\curveto(201.66242153,5.28771925)(201.904609,5.52469838)(202.09210898,5.82157335)
\curveto(202.28481728,6.11844831)(202.39679644,6.41532327)(202.42804643,6.71219824)
\curveto(202.45929643,7.0090732)(202.47492143,7.55334397)(202.47492143,8.34501054)
\closepath
}
}
{
\newrgbcolor{curcolor}{0 0 0}
\pscustom[linestyle=none,fillstyle=solid,fillcolor=curcolor]
{
}
}
{
\newrgbcolor{curcolor}{0 0 0}
\pscustom[linestyle=none,fillstyle=solid,fillcolor=curcolor]
{
\newpath
\moveto(210.22492047,15.03250971)
\lineto(210.28742046,15.0871972)
\curveto(212.07387858,15.05594721)(213.2171076,15.04032221)(213.71710754,15.04032221)
\curveto(215.42544066,15.04032221)(216.94627381,15.05594721)(218.27960698,15.0871972)
\lineto(218.31866947,15.03250971)
\curveto(218.18846116,14.53250977)(218.092107,13.7486557)(218.02960701,12.6809475)
\lineto(217.98273202,12.61844751)
\lineto(217.59991956,12.61844751)
\lineto(217.52960707,12.6809475)
\curveto(217.5348154,12.7278225)(217.53741957,12.76948916)(217.53741957,12.80594749)
\curveto(217.53741957,13.24865576)(217.50356541,13.67053071)(217.43585708,14.07157233)
\curveto(217.10773212,14.21740565)(216.65200301,14.31115563)(216.06866975,14.3528223)
\curveto(215.49054482,14.39969729)(215.05564904,14.42313479)(214.76398241,14.42313479)
\curveto(214.35252413,14.42313479)(213.82648253,14.39448896)(213.18585761,14.3371973)
\lineto(213.15460761,13.08719745)
\lineto(213.13117011,11.65751013)
\lineto(213.10773262,10.07157282)
\curveto(213.67023255,10.04032283)(214.18846165,10.02469783)(214.66241993,10.02469783)
\curveto(215.14158653,10.02469783)(215.54262815,10.04032283)(215.86554478,10.07157282)
\curveto(216.1884614,10.10282282)(216.37596138,10.13407282)(216.42804471,10.16532281)
\curveto(216.48533637,10.20178114)(216.52700303,10.2668853)(216.55304469,10.36063529)
\curveto(216.58429469,10.45959361)(216.61033635,10.63146859)(216.63116968,10.87626022)
\lineto(216.66241968,11.36063516)
\lineto(216.72491967,11.42313516)
\lineto(217.08429463,11.42313516)
\lineto(217.14679462,11.36063516)
\curveto(217.13637795,10.7929269)(217.12335712,10.22261447)(217.10773212,9.64969788)
\lineto(217.14679462,7.89188559)
\lineto(217.08429463,7.8293856)
\lineto(216.72491967,7.8293856)
\lineto(216.66241968,7.89188559)
\lineto(216.60773219,8.36063553)
\curveto(216.56085719,8.72521882)(216.52700303,8.94136463)(216.5061697,9.00907295)
\curveto(216.48533637,9.08198961)(216.43585721,9.1444896)(216.35773222,9.19657293)
\curveto(216.28481556,9.24865626)(216.09210725,9.29032292)(215.77960729,9.32157292)
\curveto(215.46710733,9.35282291)(215.03741988,9.36844791)(214.49054495,9.36844791)
\curveto(214.18846165,9.36844791)(213.72752421,9.35542708)(213.10773262,9.32938541)
\curveto(213.08169095,8.89188547)(213.06867012,8.16271889)(213.06867012,7.14188568)
\curveto(213.06867012,6.04813582)(213.08169095,5.24344842)(213.10773262,4.72782348)
\lineto(214.54523244,4.72782348)
\curveto(215.81085728,4.72782348)(216.60252385,4.74344848)(216.92023215,4.77469848)
\curveto(217.23794044,4.80594847)(217.54002374,4.8814693)(217.82648204,5.00126095)
\curveto(217.87335703,5.1210526)(217.94887785,5.39709423)(218.05304451,5.82938585)
\curveto(218.16241949,6.26688579)(218.22491949,6.54032326)(218.24054448,6.64969825)
\lineto(218.31866947,6.70438574)
\lineto(218.69366943,6.70438574)
\lineto(218.75616942,6.64969825)
\curveto(218.70929443,6.42573994)(218.66241943,6.03511499)(218.61554444,5.47782339)
\curveto(218.57387778,4.92053179)(218.55304445,4.44396935)(218.55304445,4.04813607)
\lineto(218.49835695,3.98563607)
\curveto(217.50356541,4.01688607)(216.20148224,4.03251107)(214.59210743,4.03251107)
\lineto(212.4671077,4.04813607)
\curveto(211.97231609,4.04813607)(211.45669115,4.02730274)(210.92023289,3.98563607)
\lineto(210.85773289,4.04813607)
\lineto(210.85773289,4.30594853)
\lineto(210.92023289,4.38407352)
\curveto(211.22752452,4.54032351)(211.40200366,4.63667766)(211.44367032,4.67313599)
\curveto(211.49054532,4.71480265)(211.52700365,5.17053176)(211.55304531,6.04032332)
\curveto(211.5842953,6.91532321)(211.5999203,7.5819898)(211.5999203,8.04032307)
\lineto(211.5999203,10.90751022)
\lineto(211.5921078,12.6809475)
\curveto(211.5921078,13.06636412)(211.58689947,13.37886408)(211.57648281,13.61844739)
\curveto(211.56606614,13.86323902)(211.54783698,14.03771817)(211.52179531,14.14188482)
\curveto(211.49575365,14.24605148)(211.46450365,14.31896813)(211.42804532,14.36063479)
\curveto(211.39679533,14.40230146)(211.34210783,14.43615562)(211.26398284,14.46219728)
\curveto(211.19106619,14.49344728)(211.06346204,14.51948894)(210.88117039,14.54032227)
\lineto(210.28742046,14.58719727)
\lineto(210.22492047,14.64188476)
\closepath
\moveto(214.68585742,15.66532213)
\closepath
\moveto(214.65460743,3.57938612)
\closepath
}
}
{
\newrgbcolor{curcolor}{0 0 0}
\pscustom[linestyle=none,fillstyle=solid,fillcolor=curcolor]
{
\newpath
\moveto(222.30304398,11.49344765)
\lineto(222.45148146,11.39188516)
\curveto(222.42023147,11.0481352)(222.39939814,10.63667692)(222.38898147,10.15751031)
\lineto(223.13116888,10.83719773)
\curveto(223.33950219,11.02990604)(223.47752301,11.15230186)(223.54523133,11.20438518)
\curveto(223.61293966,11.26167684)(223.7717938,11.326781)(224.02179377,11.39969766)
\curveto(224.27179374,11.47261432)(224.53481454,11.50907265)(224.81085617,11.50907265)
\curveto(225.33168944,11.50907265)(225.78741855,11.37886433)(226.17804351,11.11844769)
\curveto(226.56866846,10.86323939)(226.85773092,10.50907277)(227.0452309,10.05594783)
\curveto(227.74835581,10.72261441)(228.15720993,11.09240603)(228.27179325,11.16532269)
\curveto(228.3915849,11.23823935)(228.57908488,11.311156)(228.83429318,11.38407266)
\curveto(229.08950148,11.46219765)(229.34470978,11.50126015)(229.59991808,11.50126015)
\curveto(230.01137637,11.50126015)(230.38898049,11.41271849)(230.73273044,11.23563518)
\curveto(231.0764804,11.05855187)(231.34991787,10.82678106)(231.55304284,10.54032277)
\curveto(231.75616782,10.25386447)(231.8733553,9.94396867)(231.9046053,9.61063538)
\curveto(231.94106363,9.27730209)(231.95929279,8.81896881)(231.95929279,8.23563555)
\lineto(231.95929279,7.41532315)
\curveto(231.95929279,7.3267815)(231.97491779,6.67834408)(232.00616779,5.47001089)
\curveto(232.01658445,4.97521929)(232.07127195,4.68094849)(232.17023027,4.5871985)
\curveto(232.26918859,4.49344851)(232.58950105,4.44657352)(233.13116765,4.44657352)
\lineto(233.19366764,4.38407352)
\lineto(233.19366764,4.04813607)
\lineto(233.13116765,3.98563607)
\curveto(232.4697094,4.02730274)(232.02179278,4.04813607)(231.78741781,4.04813607)
\curveto(231.6572095,4.04813607)(231.27960538,4.02730274)(230.65460545,3.98563607)
\lineto(230.55304297,4.07157356)
\curveto(230.62075129,4.74865681)(230.65460545,5.6314692)(230.65460545,6.72001074)
\lineto(230.65460545,7.65751062)
\curveto(230.65460545,8.48563552)(230.61554296,9.06376045)(230.53741797,9.39188541)
\curveto(230.45929298,9.7252187)(230.27700133,9.996052)(229.99054304,10.20438531)
\curveto(229.70408474,10.41792695)(229.36293895,10.52469777)(228.96710566,10.52469777)
\curveto(228.6858557,10.52469777)(228.4228349,10.46740611)(228.17804326,10.35282279)
\curveto(227.93325162,10.2434478)(227.71189748,10.07678116)(227.51398084,9.85282285)
\curveto(227.32127253,9.63407288)(227.21450171,9.42834374)(227.19366838,9.23563543)
\curveto(227.17283505,9.04813545)(227.16241838,8.69917716)(227.16241838,8.18876056)
\lineto(227.16241838,7.21219818)
\curveto(227.16241838,6.98303154)(227.17283505,6.55334409)(227.19366838,5.92313583)
\curveto(227.21450171,5.29292758)(227.23533504,4.92574012)(227.25616837,4.82157347)
\curveto(227.28221004,4.71740682)(227.31866836,4.64188599)(227.36554336,4.595011)
\curveto(227.41762669,4.55334434)(227.47491835,4.52469851)(227.53741834,4.50907351)
\curveto(227.60512666,4.49865684)(227.88637663,4.47782351)(228.38116823,4.44657352)
\lineto(228.45148073,4.38407352)
\lineto(228.45148073,4.05594857)
\lineto(228.38898073,3.98563607)
\curveto(227.73793915,4.02730274)(227.11033506,4.04813607)(226.50616846,4.04813607)
\curveto(225.94887687,4.04813607)(225.32387694,4.02730274)(224.6311687,3.98563607)
\lineto(224.5608562,4.05594857)
\lineto(224.5608562,4.38407352)
\lineto(224.6311687,4.44657352)
\curveto(225.13637697,4.47782351)(225.4202311,4.50126101)(225.48273109,4.51688601)
\curveto(225.55043942,4.53251101)(225.60773108,4.563761)(225.65460607,4.610636)
\curveto(225.7066894,4.66271932)(225.74054356,4.73824015)(225.75616856,4.83719847)
\curveto(225.77700189,4.94136512)(225.79783522,5.27730258)(225.81866855,5.84501084)
\curveto(225.84471021,6.41792744)(225.85773104,6.85021905)(225.85773104,7.14188568)
\lineto(225.85773104,8.03251057)
\curveto(225.85773104,8.62105217)(225.81606438,9.07417711)(225.73273106,9.39188541)
\curveto(225.65460607,9.7095937)(225.48273109,9.97782283)(225.21710612,10.19657281)
\curveto(224.95148116,10.41532278)(224.61814786,10.52469777)(224.21710625,10.52469777)
\curveto(223.90460629,10.52469777)(223.61814799,10.46219778)(223.35773135,10.33719779)
\curveto(223.10252305,10.21740614)(222.89158558,10.06636449)(222.72491893,9.88407285)
\curveto(222.56346062,9.70698953)(222.46189813,9.54032289)(222.42023147,9.38407291)
\curveto(222.38377314,9.22782293)(222.36554398,8.8892813)(222.36554398,8.36844803)
\lineto(222.36554398,7.21219818)
\curveto(222.36554398,6.98303154)(222.37596064,6.55334409)(222.39679397,5.92313583)
\curveto(222.4176273,5.29292758)(222.43846063,4.92574012)(222.45929396,4.82157347)
\curveto(222.48533563,4.71740682)(222.52179396,4.64188599)(222.56866895,4.595011)
\curveto(222.62075228,4.55334434)(222.67804394,4.52469851)(222.74054393,4.50907351)
\curveto(222.80825225,4.49865684)(223.08950222,4.47782351)(223.58429383,4.44657352)
\lineto(223.65460632,4.38407352)
\lineto(223.65460632,4.05594857)
\lineto(223.59210632,3.98563607)
\curveto(222.94106474,4.02730274)(222.31346065,4.04813607)(221.70929406,4.04813607)
\curveto(221.1103358,4.04813607)(220.48533587,4.02730274)(219.83429429,3.98563607)
\lineto(219.7639818,4.05594857)
\lineto(219.7639818,4.38407352)
\lineto(219.83429429,4.44657352)
\curveto(220.33950256,4.47782351)(220.62335669,4.50126101)(220.68585668,4.51688601)
\curveto(220.75356501,4.53251101)(220.81085667,4.563761)(220.85773166,4.610636)
\curveto(220.90981499,4.66271932)(220.94366915,4.73824015)(220.95929415,4.83719847)
\curveto(220.98012748,4.94136512)(221.00096081,5.27730258)(221.02179414,5.84501084)
\curveto(221.0478358,6.41792744)(221.06085664,6.85021905)(221.06085664,7.14188568)
\lineto(221.06085664,8.68094799)
\curveto(221.06085664,8.8892813)(221.05043997,9.17313543)(221.02960664,9.53251039)
\curveto(221.00877331,9.89188535)(220.99054415,10.11323948)(220.97491915,10.19657281)
\curveto(220.96450248,10.27990613)(220.92543999,10.33980196)(220.85773166,10.37626029)
\curveto(220.79002334,10.41792695)(220.65460669,10.43876028)(220.45148171,10.43876028)
\lineto(219.79523179,10.44657278)
\lineto(219.7249193,10.50907277)
\lineto(219.7249193,10.84501023)
\lineto(219.78741929,10.90751022)
\curveto(220.78221084,11.02730187)(221.6207524,11.22261435)(222.30304398,11.49344765)
\closepath
}
}
{
\newrgbcolor{curcolor}{0 0 0}
\pscustom[linestyle=none,fillstyle=solid,fillcolor=curcolor]
{
\newpath
\moveto(236.44366724,11.58719764)
\lineto(236.59210472,11.49344765)
\curveto(236.54002139,10.89969772)(236.51137556,10.37365612)(236.50616723,9.91532284)
\curveto(237.21450048,10.5611561)(237.70668792,10.98823938)(237.98272955,11.19657268)
\curveto(238.26397952,11.40490599)(238.74575029,11.50907265)(239.42804187,11.50907265)
\curveto(239.85512515,11.50907265)(240.24314594,11.44917682)(240.59210423,11.32938517)
\curveto(240.94106252,11.21480185)(241.26397915,11.02209354)(241.56085411,10.75126024)
\curveto(241.85772907,10.48563527)(242.09210404,10.14969781)(242.26397902,9.74344786)
\curveto(242.435854,9.34240625)(242.52179149,8.87626047)(242.52179149,8.34501054)
\curveto(242.52179149,7.97001058)(242.46189566,7.5898023)(242.34210401,7.20438568)
\curveto(242.22231236,6.81896906)(242.06345821,6.46219827)(241.86554157,6.13407331)
\curveto(241.67283326,5.80594835)(241.51658328,5.58719838)(241.39679163,5.47782339)
\curveto(241.28220831,5.37365674)(241.04262501,5.21480259)(240.67804172,5.00126095)
\curveto(240.34991676,4.80334431)(240.02439596,4.58980267)(239.70147934,4.36063603)
\curveto(239.47752103,4.19917771)(239.32127105,4.09501106)(239.2327294,4.04813607)
\curveto(239.14418774,4.00126107)(238.99575026,3.95438608)(238.78741695,3.90751108)
\curveto(238.57908364,3.86063609)(238.3577295,3.83719859)(238.12335453,3.83719859)
\curveto(237.58168793,3.83719859)(237.0426255,3.97521941)(236.50616723,4.25126104)
\lineto(236.50616723,2.58719875)
\curveto(236.50616723,2.47261543)(236.5165839,2.10542797)(236.53741723,1.48563638)
\curveto(236.55825056,0.86063646)(236.57908389,0.49605317)(236.59991722,0.39188652)
\curveto(236.62595888,0.28771986)(236.66241721,0.21219904)(236.70929221,0.16532404)
\curveto(236.7561672,0.11844905)(236.81345886,0.08980322)(236.88116719,0.07938656)
\curveto(236.94887551,0.06376156)(237.23012548,0.04032406)(237.72491708,0.00907406)
\lineto(237.79522957,-0.05342593)
\lineto(237.79522957,-0.38155089)
\lineto(237.73272958,-0.45186338)
\curveto(237.08168799,-0.41019672)(236.45668807,-0.38936339)(235.85772981,-0.38936339)
\curveto(235.25877155,-0.38936339)(234.63377163,-0.41019672)(233.98273004,-0.45186338)
\lineto(233.91241755,-0.38155089)
\lineto(233.91241755,-0.05342593)
\lineto(233.98273004,0.00907406)
\curveto(234.48272998,0.04032406)(234.76658411,0.06376156)(234.83429244,0.07938656)
\curveto(234.90200076,0.09501155)(234.95929242,0.12886572)(235.00616742,0.18094904)
\curveto(235.05825074,0.22782404)(235.09210491,0.30594903)(235.1077299,0.41532401)
\curveto(235.12856324,0.524699)(235.14939657,0.87365729)(235.1702299,1.46219888)
\curveto(235.19627156,2.05074048)(235.20929239,2.49344876)(235.20929239,2.79032372)
\lineto(235.20929239,8.77469798)
\curveto(235.20929239,9.04553128)(235.19627156,9.35803124)(235.1702299,9.71219787)
\curveto(235.14939657,10.07157282)(235.1311674,10.27990613)(235.1155424,10.33719779)
\curveto(235.09991741,10.39448945)(235.06085491,10.44136444)(234.99835492,10.47782277)
\curveto(234.94106326,10.5142811)(234.80825077,10.53251027)(234.59991747,10.53251027)
\lineto(233.94366755,10.54032277)
\lineto(233.87335506,10.60282276)
\lineto(233.87335506,10.93876022)
\lineto(233.93585505,11.00126021)
\curveto(234.93064659,11.12105186)(235.76658399,11.31636434)(236.44366724,11.58719764)
\closepath
\moveto(236.50616723,5.59501088)
\curveto(236.74575054,5.35021924)(237.0426255,5.1366776)(237.39679212,4.95438595)
\curveto(237.75095875,4.77730264)(238.1415837,4.68876099)(238.56866698,4.68876099)
\curveto(239.07387525,4.68876099)(239.51918769,4.81636514)(239.90460431,5.07157344)
\curveto(240.29522926,5.32678174)(240.59731256,5.69917753)(240.8108542,6.1887608)
\curveto(241.02439584,6.68355241)(241.13116666,7.23042734)(241.13116666,7.8293856)
\curveto(241.13116666,8.31376054)(241.03481251,8.76428132)(240.8421042,9.18094793)
\curveto(240.65460422,9.59761455)(240.36033342,9.92573951)(239.95929181,10.16532281)
\curveto(239.56345852,10.40490612)(239.14158357,10.52469777)(238.69366696,10.52469777)
\curveto(238.41762533,10.52469777)(238.1493962,10.47521861)(237.88897956,10.37626029)
\curveto(237.62856293,10.2825103)(237.38377129,10.14188531)(237.15460465,9.95438534)
\curveto(236.93064635,9.76688536)(236.76918803,9.58719788)(236.67022971,9.4153229)
\curveto(236.57127139,9.24865626)(236.51918806,9.10803128)(236.51397973,8.99344796)
\curveto(236.5087714,8.88407297)(236.50616723,8.76167715)(236.50616723,8.6262605)
\closepath
}
}
{
\newrgbcolor{curcolor}{0 0 0}
\pscustom[linestyle=none,fillstyle=solid,fillcolor=curcolor]
{
\newpath
\moveto(247.67022836,4.44657352)
\lineto(247.73272835,4.38407352)
\lineto(247.73272835,4.05594857)
\lineto(247.67022836,3.98563607)
\curveto(247.50877004,3.98563607)(247.20408258,3.99605274)(246.75616597,4.01688607)
\curveto(246.34991602,4.0377194)(245.93064524,4.04813607)(245.49835362,4.04813607)
\curveto(244.93585369,4.04813607)(244.3082496,4.02730274)(243.61554136,3.98563607)
\lineto(243.55304136,4.05594857)
\lineto(243.55304136,4.38407352)
\lineto(243.61554136,4.44657352)
\curveto(244.12074963,4.47782351)(244.40720792,4.50126101)(244.47491625,4.51688601)
\curveto(244.54262457,4.53251101)(244.59991623,4.563761)(244.64679123,4.610636)
\curveto(244.69366622,4.66271932)(244.72491622,4.73824015)(244.74054122,4.83719847)
\curveto(244.76137455,4.94136512)(244.78220788,5.27730258)(244.80304121,5.84501084)
\curveto(244.82908287,6.41792744)(244.8421037,6.85021905)(244.8421037,7.14188568)
\lineto(244.8421037,10.15751031)
\lineto(243.83429133,10.10282282)
\lineto(243.76397884,10.16532281)
\lineto(243.76397884,10.36063529)
\lineto(243.81866633,10.43876028)
\lineto(244.8421037,10.96219771)
\lineto(244.8421037,11.43094766)
\curveto(244.8421037,11.95178092)(244.8655412,12.34501004)(244.9124162,12.61063501)
\curveto(244.95929119,12.88146831)(245.04522868,13.12886411)(245.17022866,13.35282242)
\curveto(245.30043698,13.58198906)(245.52439529,13.86844736)(245.84210358,14.21219731)
\curveto(246.15981187,14.5611556)(246.43845767,14.84500973)(246.67804098,15.06375971)
\curveto(246.91762428,15.28771801)(247.12595759,15.4361555)(247.3030409,15.50907215)
\curveto(247.48012421,15.58719714)(247.68845752,15.62625964)(247.92804082,15.62625964)
\curveto(248.12595747,15.62625964)(248.33949911,15.58719714)(248.56866574,15.50907215)
\lineto(248.56085325,14.22782231)
\lineto(248.38116577,14.15750982)
\curveto(248.08949914,14.41792645)(247.75095751,14.54813477)(247.36554089,14.54813477)
\curveto(247.09470759,14.54813477)(246.86033262,14.49084311)(246.66241598,14.37625979)
\curveto(246.46449934,14.26688481)(246.32908269,14.08459316)(246.25616603,13.82938486)
\curveto(246.18324937,13.57938489)(246.14679104,13.16271828)(246.14679104,12.57938501)
\lineto(246.14679104,10.96219771)
\lineto(246.98272844,10.96219771)
\curveto(247.41502005,10.96219771)(247.82387417,10.97782271)(248.20929079,11.00907271)
\lineto(248.27960328,10.92313522)
\lineto(248.1389783,10.2356353)
\lineto(248.06866581,10.15751031)
\curveto(247.91241583,10.16271865)(247.81085334,10.16532281)(247.76397834,10.16532281)
\lineto(246.74835347,10.17313531)
\lineto(246.14679104,10.17313531)
\lineto(246.14679104,7.0246982)
\curveto(246.14679104,6.85803155)(246.15720771,6.4752191)(246.17804104,5.87626084)
\curveto(246.2040827,5.28251091)(246.2275202,4.93355262)(246.24835353,4.82938597)
\curveto(246.26918686,4.72521932)(246.30304102,4.64969849)(246.34991602,4.6028235)
\curveto(246.40199935,4.5559485)(246.4827285,4.52209434)(246.59210349,4.50126101)
\curveto(246.70668681,4.48042768)(247.06606176,4.46219851)(247.67022836,4.44657352)
\closepath
}
}
{
\newrgbcolor{curcolor}{0 0 0}
\pscustom[linestyle=none,fillstyle=solid,fillcolor=curcolor]
{
\newpath
\moveto(255.4046024,5.19657342)
\lineto(255.15460243,4.63407349)
\curveto(254.61293583,4.2851152)(254.12856089,4.05855273)(253.70147761,3.95438608)
\curveto(253.27960266,3.85021942)(252.90199854,3.7981361)(252.56866525,3.7981361)
\curveto(251.94366533,3.7981361)(251.3499154,3.92053192)(250.78741547,4.16532355)
\curveto(250.23012387,4.41011519)(249.77439476,4.82417764)(249.42022814,5.4075109)
\curveto(249.07126985,5.99084416)(248.8967907,6.69396907)(248.8967907,7.51688564)
\curveto(248.8967907,8.06376057)(248.96449903,8.55594801)(249.09991568,8.99344796)
\curveto(249.23533233,9.43615623)(249.37595731,9.76428119)(249.52179063,9.97782283)
\curveto(249.67283228,10.19136448)(249.92543641,10.42834361)(250.27960303,10.68876025)
\curveto(250.63376966,10.94917688)(251.00876961,11.15751019)(251.40460289,11.31376017)
\curveto(251.80043618,11.47001015)(252.22751946,11.54813514)(252.68585274,11.54813514)
\curveto(253.31085266,11.54813514)(253.86033176,11.39969766)(254.33429003,11.1028227)
\curveto(254.81345664,10.81115607)(255.1493941,10.43615611)(255.34210241,9.97782283)
\curveto(255.53481072,9.51948956)(255.63116487,9.03251045)(255.63116487,8.51688551)
\curveto(255.63116487,8.3554272)(255.62335237,8.19917722)(255.60772738,8.04813557)
\lineto(255.52178989,7.96219808)
\curveto(255.16762326,7.88407309)(254.69106082,7.83198977)(254.09210256,7.8059481)
\curveto(253.4931443,7.77990644)(253.09731102,7.76688561)(252.90460271,7.76688561)
\lineto(250.39679052,7.76688561)
\curveto(250.40720718,6.68876074)(250.67804048,5.89449001)(251.20929042,5.3840734)
\curveto(251.74054035,4.8736568)(252.39158194,4.6184485)(253.16241518,4.6184485)
\curveto(253.52699847,4.6184485)(253.87595676,4.68094849)(254.20929005,4.80594847)
\curveto(254.54783167,4.93094846)(254.90460246,5.1054276)(255.27960242,5.32938591)
\closepath
\moveto(250.39679052,8.39188553)
\curveto(250.49054051,8.37626053)(250.84991546,8.35803137)(251.47491539,8.33719804)
\curveto(252.10512364,8.31636471)(252.57126942,8.30594804)(252.87335271,8.30594804)
\curveto(253.59731096,8.30594804)(254.03741507,8.31896887)(254.19366505,8.34501054)
\curveto(254.19887338,8.47001052)(254.20147755,8.56636468)(254.20147755,8.634073)
\curveto(254.20147755,9.44136457)(254.03741507,10.04032283)(253.70929011,10.43094778)
\curveto(253.38116515,10.82678106)(252.93324854,11.02469771)(252.36554028,11.02469771)
\curveto(251.74574869,11.02469771)(251.26137375,10.80334357)(250.91241546,10.36063529)
\curveto(250.5686655,9.91792701)(250.39679052,9.26167709)(250.39679052,8.39188553)
\closepath
\moveto(252.57647775,11.98563509)
\closepath
\moveto(252.48272776,3.57938612)
\closepath
}
}
{
\newrgbcolor{curcolor}{0 0 0}
\pscustom[linestyle=none,fillstyle=solid,fillcolor=curcolor]
{
\newpath
\moveto(258.81866448,15.61844714)
\lineto(258.96710196,15.52469715)
\curveto(258.9098103,14.92573889)(258.88116447,13.72521821)(258.88116447,11.92313509)
\lineto(258.88116447,9.99344783)
\curveto(259.31866442,10.35282279)(259.72491437,10.70959358)(260.09991432,11.0637602)
\curveto(260.20928931,11.16271852)(260.31345596,11.23563518)(260.41241428,11.28251017)
\curveto(260.51137261,11.32938517)(260.67022675,11.37886433)(260.88897673,11.43094766)
\curveto(261.11293503,11.48303098)(261.34210167,11.50907265)(261.57647664,11.50907265)
\curveto(261.97230993,11.50907265)(262.35512238,11.42834349)(262.724914,11.26688518)
\curveto(263.09991395,11.10542686)(263.38116392,10.91271855)(263.5686639,10.68876025)
\curveto(263.7613722,10.47001027)(263.89418469,10.20959364)(263.96710135,9.90751034)
\curveto(264.040018,9.60542705)(264.07647633,9.23303126)(264.07647633,8.79032298)
\lineto(264.07647633,7.41532315)
\curveto(264.07647633,7.3267815)(264.08949716,6.67834408)(264.11553883,5.47001089)
\curveto(264.13116383,4.97521929)(264.18585132,4.68094849)(264.27960131,4.5871985)
\curveto(264.37855963,4.49344851)(264.70147626,4.44657352)(265.24835119,4.44657352)
\lineto(265.31085118,4.38407352)
\lineto(265.31085118,4.04813607)
\lineto(265.24835119,3.98563607)
\curveto(264.58689294,4.02730274)(264.13897632,4.04813607)(263.90460135,4.04813607)
\curveto(263.7848097,4.04813607)(263.40460142,4.02730274)(262.76397649,3.98563607)
\lineto(262.67022651,4.07157356)
\curveto(262.73793483,4.74865681)(262.77178899,5.6314692)(262.77178899,6.72001074)
\lineto(262.77178899,7.74344811)
\curveto(262.77178899,8.35803137)(262.756164,8.80073965)(262.724914,9.07157295)
\curveto(262.69887234,9.34240625)(262.61293485,9.58719788)(262.46710153,9.80594786)
\curveto(262.32126822,10.02990616)(262.12595574,10.20178114)(261.8811641,10.32157279)
\curveto(261.6415808,10.44657278)(261.34991417,10.50907277)(261.00616421,10.50907277)
\curveto(260.65199759,10.50907277)(260.35512262,10.45959361)(260.11553932,10.36063529)
\curveto(259.88116435,10.2668853)(259.64158105,10.10542699)(259.39678941,9.87626035)
\curveto(259.15199777,9.64709371)(259.00356029,9.4543854)(258.95147696,9.29813542)
\curveto(258.90460197,9.14709377)(258.88116447,8.85282297)(258.88116447,8.41532303)
\lineto(258.88116447,7.0246982)
\curveto(258.88116447,6.91011488)(258.89158114,6.54032326)(258.91241447,5.91532334)
\curveto(258.9332478,5.29553175)(258.95408113,4.93355262)(258.97491446,4.82938597)
\curveto(259.00095612,4.72521932)(259.03741445,4.64969849)(259.08428945,4.6028235)
\curveto(259.13116444,4.5559485)(259.1884561,4.52469851)(259.25616443,4.50907351)
\curveto(259.32387275,4.49865684)(259.60512272,4.47782351)(260.09991432,4.44657352)
\lineto(260.17022681,4.38407352)
\lineto(260.17022681,4.05594857)
\lineto(260.10772682,3.98563607)
\curveto(259.45668524,4.02730274)(258.82908115,4.04813607)(258.22491455,4.04813607)
\curveto(257.62595629,4.04813607)(257.00095637,4.02730274)(256.34991479,3.98563607)
\lineto(256.27960229,4.05594857)
\lineto(256.27960229,4.38407352)
\lineto(256.34991479,4.44657352)
\curveto(256.85512306,4.47782351)(257.13897719,4.50126101)(257.20147718,4.51688601)
\curveto(257.26918551,4.53251101)(257.32647716,4.563761)(257.37335216,4.610636)
\curveto(257.42543549,4.66271932)(257.45928965,4.73824015)(257.47491465,4.83719847)
\curveto(257.49574798,4.94136512)(257.51658131,5.27730258)(257.53741464,5.84501084)
\curveto(257.5634563,6.41792744)(257.57647713,6.85021905)(257.57647713,7.14188568)
\lineto(257.57647713,11.18876019)
\lineto(257.54522714,12.84500998)
\curveto(257.52960214,13.42313491)(257.51137298,13.82417653)(257.49053964,14.04813483)
\curveto(257.47491465,14.27209314)(257.45147715,14.40490562)(257.42022715,14.44657228)
\curveto(257.38897716,14.48823895)(257.33428966,14.51948894)(257.25616467,14.54032227)
\curveto(257.17803968,14.5611556)(256.86293556,14.57157227)(256.31085229,14.57157227)
\lineto(256.2405398,14.64188476)
\lineto(256.2405398,14.97000972)
\lineto(256.30303979,15.04032221)
\curveto(257.29783133,15.15490553)(258.1363729,15.34761384)(258.81866448,15.61844714)
\closepath
}
}
{
\newrgbcolor{curcolor}{0 0 0}
\pscustom[linestyle=none,fillstyle=solid,fillcolor=curcolor]
{
\newpath
\moveto(268.3967883,15.61844714)
\lineto(268.54522578,15.52469715)
\curveto(268.48793412,14.92573889)(268.45928829,13.72521821)(268.45928829,11.92313509)
\lineto(268.45928829,7.0246982)
\curveto(268.45928829,6.91011488)(268.46970496,6.54032326)(268.49053829,5.91532334)
\curveto(268.51137162,5.29553175)(268.53220495,4.93355262)(268.55303828,4.82938597)
\curveto(268.57907994,4.72521932)(268.61553827,4.64969849)(268.66241327,4.6028235)
\curveto(268.70928826,4.5559485)(268.76657992,4.52469851)(268.83428825,4.50907351)
\curveto(268.90199657,4.49865684)(269.18324654,4.47782351)(269.67803814,4.44657352)
\lineto(269.74835063,4.38407352)
\lineto(269.74835063,4.05594857)
\lineto(269.68585064,3.98563607)
\curveto(269.03480905,4.02730274)(268.40720497,4.04813607)(267.80303837,4.04813607)
\curveto(267.20408011,4.04813607)(266.57908019,4.02730274)(265.9280386,3.98563607)
\lineto(265.85772611,4.05594857)
\lineto(265.85772611,4.38407352)
\lineto(265.9280386,4.44657352)
\curveto(266.43324688,4.47782351)(266.71710101,4.50126101)(266.779601,4.51688601)
\curveto(266.84730932,4.53251101)(266.90460098,4.563761)(266.95147598,4.610636)
\curveto(267.0035593,4.66271932)(267.03741347,4.73824015)(267.05303847,4.83719847)
\curveto(267.0738718,4.94136512)(267.09470513,5.27730258)(267.11553846,5.84501084)
\curveto(267.14158012,6.41792744)(267.15460095,6.85021905)(267.15460095,7.14188568)
\lineto(267.15460095,11.18876019)
\lineto(267.12335096,12.84500998)
\curveto(267.10772596,13.42313491)(267.08949679,13.82417653)(267.06866346,14.04813483)
\curveto(267.05303847,14.27209314)(267.02960097,14.40490562)(266.99835097,14.44657228)
\curveto(266.96710098,14.48823895)(266.91241348,14.51948894)(266.83428849,14.54032227)
\curveto(266.7561635,14.5611556)(266.44105937,14.57157227)(265.88897611,14.57157227)
\lineto(265.81866362,14.64188476)
\lineto(265.81866362,14.97000972)
\lineto(265.88116361,15.04032221)
\curveto(266.87595515,15.15490553)(267.71449672,15.34761384)(268.3967883,15.61844714)
\closepath
}
}
{
\newrgbcolor{curcolor}{0 0 0}
\pscustom[linestyle=none,fillstyle=solid,fillcolor=curcolor]
{
\newpath
\moveto(272.83428775,11.49344765)
\lineto(272.98272523,11.39188516)
\curveto(272.92543357,10.70959358)(272.89678774,9.84240618)(272.89678774,8.79032298)
\lineto(272.89678774,6.97001071)
\curveto(272.89678774,6.30334412)(272.94626691,5.83980251)(273.04522523,5.57938588)
\curveto(273.14939188,5.31896924)(273.32647519,5.11844843)(273.57647516,4.97782345)
\curveto(273.82647513,4.8424068)(274.12335009,4.77469848)(274.46710005,4.77469848)
\curveto(274.84730834,4.77469848)(275.19366246,4.84761513)(275.50616242,4.99344845)
\curveto(275.81866238,5.1444901)(276.07907902,5.35542757)(276.28741233,5.62626087)
\curveto(276.50095397,5.89709417)(276.62334979,6.07938582)(276.65459978,6.1731358)
\curveto(276.68584978,6.26688579)(276.70668311,6.52990659)(276.71709977,6.96219821)
\lineto(276.74053727,7.8371981)
\lineto(276.74053727,8.68094799)
\curveto(276.74053727,8.8892813)(276.73012061,9.17313543)(276.70928727,9.53251039)
\curveto(276.68845394,9.89188535)(276.67022478,10.11323948)(276.65459978,10.19657281)
\curveto(276.64418312,10.27990613)(276.60512062,10.33980196)(276.5374123,10.37626029)
\curveto(276.46970397,10.41792695)(276.33428732,10.43876028)(276.13116235,10.43876028)
\lineto(275.47491243,10.44657278)
\lineto(275.40459994,10.50907277)
\lineto(275.40459994,10.84501023)
\lineto(275.46709993,10.90751022)
\curveto(276.46189147,11.02730187)(277.30043304,11.22261435)(277.98272462,11.49344765)
\lineto(278.1311621,11.39188516)
\curveto(278.07387044,10.70959358)(278.04522461,9.84240618)(278.04522461,8.79032298)
\lineto(278.04522461,7.41532315)
\curveto(278.04522461,7.33719816)(278.05824544,6.69917741)(278.08428711,5.50126089)
\curveto(278.09470377,5.05855261)(278.12074543,4.78771931)(278.1624121,4.68876099)
\curveto(278.20928709,4.595011)(278.27178708,4.52730267)(278.34991207,4.48563601)
\curveto(278.42803706,4.44917768)(278.65980787,4.43094852)(279.04522449,4.43094852)
\lineto(279.26397446,4.43094852)
\lineto(279.33428695,4.36844853)
\lineto(279.33428695,4.05594857)
\lineto(279.27178696,3.98563607)
\curveto(278.53741205,4.02730274)(278.04522461,4.04813607)(277.79522464,4.04813607)
\curveto(277.47751635,4.04813607)(277.11032889,4.0299069)(276.69366228,3.99344857)
\lineto(276.62334979,4.05594857)
\curveto(276.65459978,4.5871985)(276.67803728,5.03771928)(276.69366228,5.4075109)
\curveto(276.36553732,5.1523026)(275.99053736,4.81115681)(275.56866242,4.38407352)
\curveto(275.4072041,4.22261521)(275.1754333,4.08719856)(274.87335,3.97782357)
\curveto(274.57126671,3.86844859)(274.22751675,3.81376109)(273.84210013,3.81376109)
\curveto(273.25876687,3.81376109)(272.80303776,3.90490692)(272.4749128,4.08719856)
\curveto(272.15199617,4.27469854)(271.92282953,4.52990684)(271.78741288,4.85282347)
\curveto(271.65199623,5.18094843)(271.58428791,5.74344836)(271.58428791,6.54032326)
\lineto(271.59210041,7.15751068)
\lineto(271.59210041,8.68094799)
\curveto(271.59210041,8.8892813)(271.58168374,9.17313543)(271.56085041,9.53251039)
\curveto(271.54001708,9.89188535)(271.52178791,10.11323948)(271.50616292,10.19657281)
\curveto(271.49574625,10.27990613)(271.45668376,10.33980196)(271.38897543,10.37626029)
\curveto(271.32126711,10.41792695)(271.18585046,10.43876028)(270.98272548,10.43876028)
\lineto(270.32647556,10.44657278)
\lineto(270.25616307,10.50907277)
\lineto(270.25616307,10.84501023)
\lineto(270.31866306,10.90751022)
\curveto(271.31345461,11.02730187)(272.15199617,11.22261435)(272.83428775,11.49344765)
\closepath
\moveto(274.70147502,11.98563509)
\closepath
\moveto(274.77960001,3.57938612)
\closepath
}
}
{
\newrgbcolor{curcolor}{0 0 0}
\pscustom[linestyle=none,fillstyle=solid,fillcolor=curcolor]
{
\newpath
\moveto(282.45147407,11.49344765)
\lineto(282.59991155,11.39188516)
\curveto(282.56866155,11.04292687)(282.54782822,10.58459359)(282.53741156,10.01688533)
\curveto(282.95407817,10.36063529)(283.34991146,10.70959358)(283.72491141,11.0637602)
\curveto(283.8342864,11.16271852)(283.93584888,11.23563518)(284.02959887,11.28251017)
\curveto(284.12855719,11.32938517)(284.29001551,11.37886433)(284.51397381,11.43094766)
\curveto(284.73793212,11.48303098)(284.96709876,11.50907265)(285.20147373,11.50907265)
\curveto(285.59730701,11.50907265)(285.98011946,11.42834349)(286.34991109,11.26688518)
\curveto(286.72491104,11.10542686)(287.00616101,10.91271855)(287.19366098,10.68876025)
\curveto(287.38636929,10.47001027)(287.51918178,10.20959364)(287.59209843,9.90751034)
\curveto(287.66501509,9.60542705)(287.70147342,9.23303126)(287.70147342,8.79032298)
\lineto(287.70147342,7.41532315)
\curveto(287.70147342,7.3267815)(287.71449425,6.67834408)(287.74053591,5.47001089)
\curveto(287.75095258,4.97521929)(287.80564007,4.68094849)(287.90459839,4.5871985)
\curveto(288.00355672,4.49344851)(288.32647334,4.44657352)(288.87334827,4.44657352)
\lineto(288.93584827,4.38407352)
\lineto(288.93584827,4.04813607)
\lineto(288.87334827,3.98563607)
\curveto(288.21189002,4.02730274)(287.76397341,4.04813607)(287.52959844,4.04813607)
\curveto(287.39418179,4.04813607)(287.0139735,4.02730274)(286.38897358,3.98563607)
\lineto(286.29522359,4.07157356)
\curveto(286.36293192,4.74865681)(286.39678608,5.6314692)(286.39678608,6.72001074)
\lineto(286.39678608,7.74344811)
\curveto(286.39678608,8.35803137)(286.38116108,8.80073965)(286.34991109,9.07157295)
\curveto(286.32386942,9.34240625)(286.23793193,9.58719788)(286.09209862,9.80594786)
\curveto(285.9462653,10.02990616)(285.75095283,10.20178114)(285.50616119,10.32157279)
\curveto(285.26136955,10.44657278)(284.96970292,10.50907277)(284.6311613,10.50907277)
\curveto(284.360328,10.50907277)(284.12855719,10.48042694)(283.93584888,10.42313528)
\curveto(283.74314057,10.37105195)(283.52699477,10.2590728)(283.28741146,10.08719782)
\curveto(283.04782816,9.92053118)(282.87074485,9.76167703)(282.75616153,9.61063538)
\curveto(282.64157821,9.46480206)(282.57126572,9.32678125)(282.54522406,9.19657293)
\curveto(282.52439072,9.07157295)(282.51397406,8.80073965)(282.51397406,8.38407303)
\lineto(282.51397406,7.0246982)
\curveto(282.51397406,6.91011488)(282.52439072,6.54032326)(282.54522406,5.91532334)
\curveto(282.56605739,5.29553175)(282.58689072,4.93355262)(282.60772405,4.82938597)
\curveto(282.63376571,4.72521932)(282.67022404,4.64969849)(282.71709903,4.6028235)
\curveto(282.76397403,4.5559485)(282.82126569,4.52469851)(282.88897401,4.50907351)
\curveto(282.95668234,4.49865684)(283.2379323,4.47782351)(283.73272391,4.44657352)
\lineto(283.8030364,4.38407352)
\lineto(283.8030364,4.05594857)
\lineto(283.74053641,3.98563607)
\curveto(283.08949482,4.02730274)(282.46189073,4.04813607)(281.85772414,4.04813607)
\curveto(281.25876588,4.04813607)(280.63376596,4.02730274)(279.98272437,3.98563607)
\lineto(279.91241188,4.05594857)
\lineto(279.91241188,4.38407352)
\lineto(279.98272437,4.44657352)
\curveto(280.48793264,4.47782351)(280.77178677,4.50126101)(280.83428677,4.51688601)
\curveto(280.90199509,4.53251101)(280.95928675,4.563761)(281.00616174,4.610636)
\curveto(281.05824507,4.66271932)(281.09209923,4.73824015)(281.10772423,4.83719847)
\curveto(281.12855756,4.94136512)(281.14939089,5.27730258)(281.17022422,5.84501084)
\curveto(281.19626589,6.41792744)(281.20928672,6.85021905)(281.20928672,7.14188568)
\lineto(281.20928672,8.68094799)
\curveto(281.20928672,8.8892813)(281.19887005,9.17313543)(281.17803672,9.53251039)
\curveto(281.15720339,9.89188535)(281.13897423,10.11323948)(281.12334923,10.19657281)
\curveto(281.11293257,10.27990613)(281.07387007,10.33980196)(281.00616174,10.37626029)
\curveto(280.93845342,10.41792695)(280.80303677,10.43876028)(280.5999118,10.43876028)
\lineto(279.94366188,10.44657278)
\lineto(279.87334938,10.50907277)
\lineto(279.87334938,10.84501023)
\lineto(279.93584938,10.90751022)
\curveto(280.93064092,11.02730187)(281.76918248,11.22261435)(282.45147407,11.49344765)
\closepath
}
}
{
\newrgbcolor{curcolor}{0 0 0}
\pscustom[linestyle=none,fillstyle=solid,fillcolor=curcolor]
{
\newpath
\moveto(297.70928469,10.95438521)
\lineto(297.77178468,10.81376023)
\curveto(297.55824304,10.48563527)(297.42543055,10.27209363)(297.37334723,10.17313531)
\lineto(295.92022241,10.17313531)
\curveto(296.06605572,9.88667701)(296.13897238,9.58719788)(296.13897238,9.27469792)
\curveto(296.13897238,8.9049063)(296.05303489,8.54553134)(295.88115991,8.19657305)
\curveto(295.71449326,7.84761476)(295.48011829,7.5507398)(295.178035,7.30594816)
\curveto(294.8759517,7.06115653)(294.53220174,6.86584405)(294.14678512,6.72001074)
\curveto(293.76657684,6.57417742)(293.33168106,6.50126076)(292.84209779,6.50126076)
\lineto(292.48272283,6.50126076)
\curveto(292.1754312,6.25646913)(291.97751456,6.07417748)(291.8889729,5.95438583)
\curveto(291.80043125,5.83459418)(291.75616042,5.71219836)(291.75616042,5.58719838)
\curveto(291.75616042,5.35803174)(291.86032707,5.19657342)(292.06866038,5.10282344)
\curveto(292.28220202,5.00907345)(292.71970197,4.96219845)(293.38116022,4.96219845)
\lineto(295.13116,4.98563595)
\curveto(295.68324327,4.98563595)(296.10511822,4.92053179)(296.39678485,4.79032347)
\curveto(296.69365981,4.66011516)(296.93324311,4.42574019)(297.11553476,4.08719856)
\curveto(297.2978264,3.75386527)(297.38897222,3.39969865)(297.38897222,3.02469869)
\curveto(297.38897222,2.4517821)(297.20147225,1.88146967)(296.82647229,1.3137614)
\curveto(296.45668067,0.74084481)(295.92022241,0.30594903)(295.21709749,0.00907406)
\curveto(294.51397258,-0.2878009)(293.76397267,-0.43623838)(292.96709777,-0.43623838)
\curveto(292.51397283,-0.43623838)(292.08688955,-0.38675922)(291.68584793,-0.2878009)
\curveto(291.29001464,-0.18884258)(290.94105635,-0.03780093)(290.63897306,0.16532404)
\curveto(290.34209809,0.36324069)(290.10251479,0.62365732)(289.92022315,0.94657395)
\curveto(289.7379315,1.26949058)(289.64678568,1.60542803)(289.64678568,1.95438632)
\curveto(289.64678568,2.14709463)(289.67543151,2.35021961)(289.73272317,2.56376125)
\curveto(289.79001483,2.77209456)(289.89938982,2.99865703)(290.06084813,3.24344867)
\lineto(291.47491045,4.03251107)
\curveto(291.02178551,4.17313605)(290.73532721,4.31115687)(290.61553556,4.44657352)
\curveto(290.50095224,4.5871985)(290.44366058,4.75386515)(290.44366058,4.94657346)
\curveto(290.44366058,5.1444901)(290.50876474,5.37886507)(290.63897306,5.64969837)
\lineto(291.8967854,6.55594826)
\curveto(291.15720216,6.74865657)(290.65720222,7.03771903)(290.39678559,7.42313565)
\curveto(290.14157729,7.80855227)(290.01397313,8.23042722)(290.01397313,8.68876049)
\curveto(290.01397313,9.11063544)(290.10511896,9.51167706)(290.2874106,9.89188535)
\curveto(290.46970224,10.27730196)(290.74313971,10.58459359)(291.107723,10.81376023)
\curveto(291.47751462,11.04292687)(291.8889729,11.22261435)(292.34209785,11.35282267)
\curveto(292.80043112,11.48823932)(293.21709774,11.55594764)(293.59209769,11.55594764)
\curveto(294.24313928,11.55594764)(294.86293087,11.34501017)(295.45147246,10.92313522)
\curveto(296.39418068,10.92313522)(297.14678475,10.93355188)(297.70928469,10.95438521)
\closepath
\moveto(291.35772297,9.14188544)
\curveto(291.35772297,8.82417714)(291.42022296,8.48303135)(291.54522295,8.11844806)
\curveto(291.67022293,7.75386478)(291.87074374,7.47782314)(292.14678537,7.29032317)
\curveto(292.42803534,7.10282319)(292.74313946,7.0090732)(293.09209775,7.0090732)
\curveto(293.55563936,7.0090732)(293.95407682,7.16011485)(294.28741011,7.46219815)
\curveto(294.62595173,7.76428144)(294.79522254,8.23563555)(294.79522254,8.87626047)
\curveto(294.79522254,9.41792707)(294.64157673,9.90490618)(294.3342851,10.33719779)
\curveto(294.02699347,10.7694894)(293.59209769,10.98563521)(293.02959776,10.98563521)
\curveto(292.55563949,10.98563521)(292.15720204,10.82938523)(291.83428541,10.51688527)
\curveto(291.51657712,10.20959364)(291.35772297,9.75126036)(291.35772297,9.14188544)
\closepath
\moveto(293.64678519,3.89969858)
\curveto(292.80824362,3.89969858)(292.32386868,3.88928192)(292.19366037,3.86844859)
\curveto(292.06866038,3.84761526)(291.8889729,3.7590736)(291.65459793,3.60282362)
\curveto(291.42022296,3.44657364)(291.22751465,3.23563617)(291.076473,2.9700112)
\curveto(290.93063969,2.6991779)(290.85772303,2.3970946)(290.85772303,2.06376131)
\curveto(290.85772303,1.47521972)(291.076473,0.99605311)(291.51397295,0.62626149)
\curveto(291.9514729,0.25646987)(292.53480616,0.07157406)(293.26397273,0.07157406)
\curveto(293.81084767,0.07157406)(294.31605594,0.18615738)(294.77959755,0.41532401)
\curveto(295.24313916,0.64449065)(295.58428495,0.95178228)(295.80303492,1.3371989)
\curveto(296.02699323,1.72261552)(296.13897238,2.09240714)(296.13897238,2.44657376)
\curveto(296.13897238,2.80074039)(296.04522239,3.10542785)(295.85772241,3.36063615)
\curveto(295.67543077,3.61063612)(295.43584747,3.76428193)(295.1389725,3.82157359)
\curveto(294.84730587,3.87365692)(294.3499101,3.89969858)(293.64678519,3.89969858)
\closepath
}
}
{
\newrgbcolor{curcolor}{0 0 0}
\pscustom[linestyle=none,fillstyle=solid,fillcolor=curcolor]
{
\newpath
\moveto(304.8967838,5.19657342)
\lineto(304.64678383,4.63407349)
\curveto(304.10511723,4.2851152)(303.62074229,4.05855273)(303.19365901,3.95438608)
\curveto(302.77178406,3.85021942)(302.39417994,3.7981361)(302.06084665,3.7981361)
\curveto(301.43584673,3.7981361)(300.8420968,3.92053192)(300.27959687,4.16532355)
\curveto(299.72230527,4.41011519)(299.26657616,4.82417764)(298.91240954,5.4075109)
\curveto(298.56345125,5.99084416)(298.3889721,6.69396907)(298.3889721,7.51688564)
\curveto(298.3889721,8.06376057)(298.45668043,8.55594801)(298.59209708,8.99344796)
\curveto(298.72751373,9.43615623)(298.86813871,9.76428119)(299.01397202,9.97782283)
\curveto(299.16501367,10.19136448)(299.41761781,10.42834361)(299.77178443,10.68876025)
\curveto(300.12595105,10.94917688)(300.50095101,11.15751019)(300.89678429,11.31376017)
\curveto(301.29261758,11.47001015)(301.71970086,11.54813514)(302.17803413,11.54813514)
\curveto(302.80303406,11.54813514)(303.35251316,11.39969766)(303.82647143,11.1028227)
\curveto(304.30563804,10.81115607)(304.6415755,10.43615611)(304.83428381,9.97782283)
\curveto(305.02699212,9.51948956)(305.12334627,9.03251045)(305.12334627,8.51688551)
\curveto(305.12334627,8.3554272)(305.11553377,8.19917722)(305.09990877,8.04813557)
\lineto(305.01397128,7.96219808)
\curveto(304.65980466,7.88407309)(304.18324222,7.83198977)(303.58428396,7.8059481)
\curveto(302.9853257,7.77990644)(302.58949242,7.76688561)(302.39678411,7.76688561)
\lineto(299.88897192,7.76688561)
\curveto(299.89938858,6.68876074)(300.17022188,5.89449001)(300.70147182,5.3840734)
\curveto(301.23272175,4.8736568)(301.88376334,4.6184485)(302.65459658,4.6184485)
\curveto(303.01917986,4.6184485)(303.36813815,4.68094849)(303.70147145,4.80594847)
\curveto(304.04001307,4.93094846)(304.39678386,5.1054276)(304.77178381,5.32938591)
\closepath
\moveto(299.88897192,8.39188553)
\curveto(299.98272191,8.37626053)(300.34209686,8.35803137)(300.96709678,8.33719804)
\curveto(301.59730504,8.31636471)(302.06345082,8.30594804)(302.36553411,8.30594804)
\curveto(303.08949236,8.30594804)(303.52959647,8.31896887)(303.68584645,8.34501054)
\curveto(303.69105478,8.47001052)(303.69365895,8.56636468)(303.69365895,8.634073)
\curveto(303.69365895,9.44136457)(303.52959647,10.04032283)(303.20147151,10.43094778)
\curveto(302.87334655,10.82678106)(302.42542994,11.02469771)(301.85772167,11.02469771)
\curveto(301.23793008,11.02469771)(300.75355514,10.80334357)(300.40459685,10.36063529)
\curveto(300.0608469,9.91792701)(299.88897192,9.26167709)(299.88897192,8.39188553)
\closepath
\moveto(302.06865915,11.98563509)
\closepath
\moveto(301.97490916,3.57938612)
\closepath
}
}
{
\newrgbcolor{curcolor}{0 0 0}
\pscustom[linestyle=none,fillstyle=solid,fillcolor=curcolor]
{
\newpath
\moveto(308.32647088,11.49344765)
\lineto(308.47490836,11.39188516)
\curveto(308.44365836,11.04292687)(308.42282503,10.58459359)(308.41240837,10.01688533)
\curveto(308.82907498,10.36063529)(309.22490827,10.70959358)(309.59990822,11.0637602)
\curveto(309.70928321,11.16271852)(309.81084569,11.23563518)(309.90459568,11.28251017)
\curveto(310.003554,11.32938517)(310.16501232,11.37886433)(310.38897062,11.43094766)
\curveto(310.61292893,11.48303098)(310.84209557,11.50907265)(311.07647054,11.50907265)
\curveto(311.47230382,11.50907265)(311.85511627,11.42834349)(312.2249079,11.26688518)
\curveto(312.59990785,11.10542686)(312.88115781,10.91271855)(313.06865779,10.68876025)
\curveto(313.2613661,10.47001027)(313.39417858,10.20959364)(313.46709524,9.90751034)
\curveto(313.5400119,9.60542705)(313.57647023,9.23303126)(313.57647023,8.79032298)
\lineto(313.57647023,7.41532315)
\curveto(313.57647023,7.3267815)(313.58949106,6.67834408)(313.61553272,5.47001089)
\curveto(313.62594939,4.97521929)(313.68063688,4.68094849)(313.7795952,4.5871985)
\curveto(313.87855352,4.49344851)(314.20147015,4.44657352)(314.74834508,4.44657352)
\lineto(314.81084508,4.38407352)
\lineto(314.81084508,4.04813607)
\lineto(314.74834508,3.98563607)
\curveto(314.08688683,4.02730274)(313.63897022,4.04813607)(313.40459525,4.04813607)
\curveto(313.2691786,4.04813607)(312.88897031,4.02730274)(312.26397039,3.98563607)
\lineto(312.1702204,4.07157356)
\curveto(312.23792873,4.74865681)(312.27178289,5.6314692)(312.27178289,6.72001074)
\lineto(312.27178289,7.74344811)
\curveto(312.27178289,8.35803137)(312.25615789,8.80073965)(312.2249079,9.07157295)
\curveto(312.19886623,9.34240625)(312.11292874,9.58719788)(311.96709543,9.80594786)
\curveto(311.82126211,10.02990616)(311.62594964,10.20178114)(311.381158,10.32157279)
\curveto(311.13636636,10.44657278)(310.84469973,10.50907277)(310.50615811,10.50907277)
\curveto(310.23532481,10.50907277)(310.003554,10.48042694)(309.81084569,10.42313528)
\curveto(309.61813738,10.37105195)(309.40199158,10.2590728)(309.16240827,10.08719782)
\curveto(308.92282497,9.92053118)(308.74574166,9.76167703)(308.63115834,9.61063538)
\curveto(308.51657502,9.46480206)(308.44626253,9.32678125)(308.42022086,9.19657293)
\curveto(308.39938753,9.07157295)(308.38897087,8.80073965)(308.38897087,8.38407303)
\lineto(308.38897087,7.0246982)
\curveto(308.38897087,6.91011488)(308.39938753,6.54032326)(308.42022086,5.91532334)
\curveto(308.4410542,5.29553175)(308.46188753,4.93355262)(308.48272086,4.82938597)
\curveto(308.50876252,4.72521932)(308.54522085,4.64969849)(308.59209584,4.6028235)
\curveto(308.63897084,4.5559485)(308.6962625,4.52469851)(308.76397082,4.50907351)
\curveto(308.83167915,4.49865684)(309.11292911,4.47782351)(309.60772072,4.44657352)
\lineto(309.67803321,4.38407352)
\lineto(309.67803321,4.05594857)
\lineto(309.61553322,3.98563607)
\curveto(308.96449163,4.02730274)(308.33688754,4.04813607)(307.73272095,4.04813607)
\curveto(307.13376269,4.04813607)(306.50876277,4.02730274)(305.85772118,3.98563607)
\lineto(305.78740869,4.05594857)
\lineto(305.78740869,4.38407352)
\lineto(305.85772118,4.44657352)
\curveto(306.36292945,4.47782351)(306.64678358,4.50126101)(306.70928358,4.51688601)
\curveto(306.7769919,4.53251101)(306.83428356,4.563761)(306.88115855,4.610636)
\curveto(306.93324188,4.66271932)(306.96709604,4.73824015)(306.98272104,4.83719847)
\curveto(307.00355437,4.94136512)(307.0243877,5.27730258)(307.04522103,5.84501084)
\curveto(307.0712627,6.41792744)(307.08428353,6.85021905)(307.08428353,7.14188568)
\lineto(307.08428353,8.68094799)
\curveto(307.08428353,8.8892813)(307.07386686,9.17313543)(307.05303353,9.53251039)
\curveto(307.0322002,9.89188535)(307.01397104,10.11323948)(306.99834604,10.19657281)
\curveto(306.98792937,10.27990613)(306.94886688,10.33980196)(306.88115855,10.37626029)
\curveto(306.81345023,10.41792695)(306.67803358,10.43876028)(306.4749086,10.43876028)
\lineto(305.81865869,10.44657278)
\lineto(305.74834619,10.50907277)
\lineto(305.74834619,10.84501023)
\lineto(305.81084619,10.90751022)
\curveto(306.80563773,11.02730187)(307.64417929,11.22261435)(308.32647088,11.49344765)
\closepath
}
}
{
\newrgbcolor{curcolor}{0 0 0}
\pscustom[linestyle=none,fillstyle=solid,fillcolor=curcolor]
{
\newpath
\moveto(314.96709506,2.20438629)
\lineto(319.18584454,15.61844714)
\lineto(319.26396953,15.68094713)
\lineto(320.04521943,15.68094713)
\lineto(320.09209443,15.60282214)
\lineto(315.87334495,2.19657379)
\lineto(315.78740746,2.1262613)
\lineto(315.02178255,2.1262613)
\closepath
}
}
{
\newrgbcolor{curcolor}{0 0 0}
\pscustom[linestyle=none,fillstyle=solid,fillcolor=curcolor]
{
\newpath
\moveto(320.46709438,9.4309479)
\curveto(320.46709438,10.21219781)(320.57907353,10.94136438)(320.80303184,11.61844763)
\curveto(321.02699014,12.29553088)(321.40719843,12.92053081)(321.9436567,13.4934474)
\curveto(322.4853233,14.07157233)(323.13636488,14.51688478)(323.89678146,14.82938474)
\curveto(324.65719803,15.14709303)(325.51657292,15.30594718)(326.47490614,15.30594718)
\curveto(328.25094759,15.30594718)(329.66500991,14.82157224)(330.71709312,13.85282236)
\curveto(331.76917632,12.88928081)(332.29521792,11.57157264)(332.29521792,9.89969784)
\curveto(332.29521792,8.72261466)(332.03740545,7.66532312)(331.52178052,6.72782324)
\curveto(331.00615558,5.79032335)(330.26657234,5.06115677)(329.30303079,4.54032351)
\curveto(328.34469757,4.01949024)(327.26657271,3.7590736)(326.06865619,3.7590736)
\curveto(325.21969796,3.7590736)(324.43844806,3.89969858)(323.72490648,4.18094855)
\curveto(323.0113649,4.46740685)(322.40198997,4.89188596)(321.8967817,5.45438589)
\curveto(321.39678176,6.01688582)(321.03219848,6.66792741)(320.80303184,7.40751065)
\curveto(320.57907353,8.14709389)(320.46709438,8.82157298)(320.46709438,9.4309479)
\closepath
\moveto(322.09209418,9.94657284)
\curveto(322.09209418,9.31115625)(322.19365667,8.6262605)(322.39678164,7.89188559)
\curveto(322.60511495,7.16271902)(322.91240658,6.53511493)(323.31865653,6.00907332)
\curveto(323.73011481,5.48824006)(324.20146892,5.10021927)(324.73271885,4.84501097)
\curveto(325.26917712,4.58980267)(325.87855205,4.46219851)(326.56084363,4.46219851)
\curveto(327.3681352,4.46219851)(328.08688511,4.64709433)(328.71709336,5.01688595)
\curveto(329.34730162,5.38667757)(329.82646823,5.95959416)(330.15459318,6.73563573)
\curveto(330.48271814,7.51688564)(330.64678062,8.39188553)(330.64678062,9.36063541)
\curveto(330.64678062,10.41271861)(330.45928065,11.36323933)(330.08428069,12.21219756)
\curveto(329.70928074,13.06115579)(329.17542664,13.67834321)(328.48271839,14.06375983)
\curveto(327.79001014,14.44917645)(327.02178107,14.64188476)(326.17803118,14.64188476)
\curveto(324.91240633,14.64188476)(323.91501062,14.24605148)(323.18584404,13.45438491)
\curveto(322.45667747,12.66271834)(322.09209418,11.49344765)(322.09209418,9.94657284)
\closepath
\moveto(326.26396866,15.66532213)
\closepath
\moveto(326.30303116,3.57938612)
\closepath
}
}
{
\newrgbcolor{curcolor}{0 0 0}
\pscustom[linestyle=none,fillstyle=solid,fillcolor=curcolor]
{
\newpath
\moveto(335.56084252,11.58719764)
\lineto(335.70928,11.49344765)
\curveto(335.65719667,10.89969772)(335.62855084,10.37365612)(335.62334251,9.91532284)
\curveto(336.33167576,10.5611561)(336.8238632,10.98823938)(337.09990483,11.19657268)
\curveto(337.38115479,11.40490599)(337.86292557,11.50907265)(338.54521715,11.50907265)
\curveto(338.97230043,11.50907265)(339.36032122,11.44917682)(339.70927951,11.32938517)
\curveto(340.0582378,11.21480185)(340.38115442,11.02209354)(340.67802939,10.75126024)
\curveto(340.97490435,10.48563527)(341.20927932,10.14969781)(341.3811543,9.74344786)
\curveto(341.55302928,9.34240625)(341.63896677,8.87626047)(341.63896677,8.34501054)
\curveto(341.63896677,7.97001058)(341.57907094,7.5898023)(341.45927929,7.20438568)
\curveto(341.33948764,6.81896906)(341.18063349,6.46219827)(340.98271685,6.13407331)
\curveto(340.79000854,5.80594835)(340.63375856,5.58719838)(340.51396691,5.47782339)
\curveto(340.39938359,5.37365674)(340.15980028,5.21480259)(339.795217,5.00126095)
\curveto(339.46709204,4.80334431)(339.14157124,4.58980267)(338.81865462,4.36063603)
\curveto(338.59469631,4.19917771)(338.43844633,4.09501106)(338.34990467,4.04813607)
\curveto(338.26136302,4.00126107)(338.11292554,3.95438608)(337.90459223,3.90751108)
\curveto(337.69625892,3.86063609)(337.47490478,3.83719859)(337.24052981,3.83719859)
\curveto(336.69886321,3.83719859)(336.15980078,3.97521941)(335.62334251,4.25126104)
\lineto(335.62334251,2.58719875)
\curveto(335.62334251,2.47261543)(335.63375918,2.10542797)(335.65459251,1.48563638)
\curveto(335.67542584,0.86063646)(335.69625917,0.49605317)(335.7170925,0.39188652)
\curveto(335.74313416,0.28771986)(335.77959249,0.21219904)(335.82646749,0.16532404)
\curveto(335.87334248,0.11844905)(335.93063414,0.08980322)(335.99834246,0.07938656)
\curveto(336.06605079,0.06376156)(336.34730075,0.04032406)(336.84209236,0.00907406)
\lineto(336.91240485,-0.05342593)
\lineto(336.91240485,-0.38155089)
\lineto(336.84990486,-0.45186338)
\curveto(336.19886327,-0.41019672)(335.57386335,-0.38936339)(334.97490509,-0.38936339)
\curveto(334.37594683,-0.38936339)(333.75094691,-0.41019672)(333.09990532,-0.45186338)
\lineto(333.02959283,-0.38155089)
\lineto(333.02959283,-0.05342593)
\lineto(333.09990532,0.00907406)
\curveto(333.59990526,0.04032406)(333.88375939,0.06376156)(333.95146772,0.07938656)
\curveto(334.01917604,0.09501155)(334.0764677,0.12886572)(334.1233427,0.18094904)
\curveto(334.17542602,0.22782404)(334.20928018,0.30594903)(334.22490518,0.41532401)
\curveto(334.24573851,0.524699)(334.26657184,0.87365729)(334.28740518,1.46219888)
\curveto(334.31344684,2.05074048)(334.32646767,2.49344876)(334.32646767,2.79032372)
\lineto(334.32646767,8.77469798)
\curveto(334.32646767,9.04553128)(334.31344684,9.35803124)(334.28740518,9.71219787)
\curveto(334.26657184,10.07157282)(334.24834268,10.27990613)(334.23271768,10.33719779)
\curveto(334.21709268,10.39448945)(334.17803019,10.44136444)(334.1155302,10.47782277)
\curveto(334.05823854,10.5142811)(333.92542605,10.53251027)(333.71709275,10.53251027)
\lineto(333.06084283,10.54032277)
\lineto(332.99053034,10.60282276)
\lineto(332.99053034,10.93876022)
\lineto(333.05303033,11.00126021)
\curveto(334.04782187,11.12105186)(334.88375927,11.31636434)(335.56084252,11.58719764)
\closepath
\moveto(335.62334251,5.59501088)
\curveto(335.86292581,5.35021924)(336.15980078,5.1366776)(336.5139674,4.95438595)
\curveto(336.86813402,4.77730264)(337.25875898,4.68876099)(337.68584226,4.68876099)
\curveto(338.19105053,4.68876099)(338.63636297,4.81636514)(339.02177959,5.07157344)
\curveto(339.41240454,5.32678174)(339.71448784,5.69917753)(339.92802948,6.1887608)
\curveto(340.14157112,6.68355241)(340.24834194,7.23042734)(340.24834194,7.8293856)
\curveto(340.24834194,8.31376054)(340.15198779,8.76428132)(339.95927948,9.18094793)
\curveto(339.7717795,9.59761455)(339.4775087,9.92573951)(339.07646708,10.16532281)
\curveto(338.6806338,10.40490612)(338.25875885,10.52469777)(337.81084224,10.52469777)
\curveto(337.53480061,10.52469777)(337.26657147,10.47521861)(337.00615484,10.37626029)
\curveto(336.74573821,10.2825103)(336.50094657,10.14188531)(336.27177993,9.95438534)
\curveto(336.04782162,9.76688536)(335.88636331,9.58719788)(335.78740499,9.4153229)
\curveto(335.68844667,9.24865626)(335.63636334,9.10803128)(335.63115501,8.99344796)
\curveto(335.62594668,8.88407297)(335.62334251,8.76167715)(335.62334251,8.6262605)
\closepath
}
}
{
\newrgbcolor{curcolor}{0 0 0}
\pscustom[linestyle=none,fillstyle=solid,fillcolor=curcolor]
{
\newpath
\moveto(342.64677914,10.18094781)
\lineto(342.64677914,10.38407278)
\lineto(342.70146664,10.46219778)
\curveto(343.19104991,10.64448942)(343.5868832,10.81376023)(343.88896649,10.97001021)
\curveto(343.88896649,12.32938504)(343.87334149,13.12105161)(343.8420915,13.34500992)
\curveto(344.37854976,13.5325099)(344.81865388,13.73303071)(345.16240383,13.94657235)
\lineto(345.34990381,13.79032236)
\curveto(345.29782048,13.46219741)(345.23532049,12.50907252)(345.16240383,10.93094772)
\curveto(345.42282047,10.92573938)(345.70407043,10.92313522)(346.00615373,10.92313522)
\curveto(346.62073699,10.92313522)(347.0608411,10.93876022)(347.32646607,10.97001021)
\lineto(347.38115356,10.91532272)
\lineto(347.23271608,10.2590728)
\lineto(347.17021609,10.18876031)
\curveto(346.90459112,10.19396864)(346.61032032,10.19657281)(346.2874037,10.19657281)
\curveto(345.99573706,10.19657281)(345.62073711,10.19396864)(345.16240383,10.18876031)
\lineto(345.11552884,7.0090732)
\curveto(345.11552884,6.27469829)(345.13115384,5.79292752)(345.16240383,5.56376088)
\curveto(345.19886216,5.33980257)(345.29261215,5.16271926)(345.4436538,5.03251094)
\curveto(345.59990378,4.90751096)(345.82907042,4.84501097)(346.13115371,4.84501097)
\curveto(346.48011201,4.84501097)(346.80302863,4.93615679)(347.0999036,5.11844843)
\lineto(347.28740357,4.83719847)
\curveto(347.16240359,4.74865681)(346.86292446,4.48824018)(346.38896618,4.05594857)
\curveto(346.11813288,3.93094858)(345.84209125,3.86844859)(345.56084129,3.86844859)
\curveto(344.38896643,3.86844859)(343.803029,4.43615685)(343.803029,5.57157338)
\curveto(343.803029,5.98823999)(343.81344567,6.34240662)(343.834279,6.63407325)
\curveto(343.83948733,6.7226149)(343.8420915,6.81376072)(343.8420915,6.90751071)
\lineto(343.8420915,10.14969781)
\lineto(343.52177904,10.14969781)
\curveto(343.28740407,10.14969781)(343.01917493,10.13928115)(342.71709164,10.11844782)
\closepath
}
}
{
\newrgbcolor{curcolor}{0 0 0}
\pscustom[linestyle=none,fillstyle=solid,fillcolor=curcolor]
{
\newpath
\moveto(349.75615327,14.98563472)
\curveto(350.0009449,14.98563472)(350.20927821,14.89969723)(350.38115319,14.72782225)
\curveto(350.55302817,14.55594727)(350.63896566,14.34761396)(350.63896566,14.10282233)
\curveto(350.63896566,13.86323902)(350.55302817,13.65750988)(350.38115319,13.4856349)
\curveto(350.20927821,13.31375992)(350.0009449,13.22782243)(349.75615327,13.22782243)
\curveto(349.51656996,13.22782243)(349.30823666,13.31115576)(349.13115334,13.4778224)
\curveto(348.95927837,13.64969738)(348.87334088,13.85803069)(348.87334088,14.10282233)
\curveto(348.87334088,14.34761396)(348.95927837,14.55594727)(349.13115334,14.72782225)
\curveto(349.30823666,14.89969723)(349.51656996,14.98563472)(349.75615327,14.98563472)
\closepath
\moveto(350.42021569,11.49344765)
\lineto(350.56865317,11.39188516)
\curveto(350.51136151,10.70959358)(350.48271568,9.84240618)(350.48271568,8.79032298)
\lineto(350.48271568,7.0246982)
\curveto(350.48271568,6.91011488)(350.49313234,6.54032326)(350.51396567,5.91532334)
\curveto(350.53479901,5.29553175)(350.55563234,4.93355262)(350.57646567,4.82938597)
\curveto(350.60250733,4.72521932)(350.63896566,4.64969849)(350.68584065,4.6028235)
\curveto(350.73271565,4.5559485)(350.79000731,4.52469851)(350.85771563,4.50907351)
\curveto(350.92542396,4.49865684)(351.20667392,4.47782351)(351.70146553,4.44657352)
\lineto(351.77177802,4.38407352)
\lineto(351.77177802,4.05594857)
\lineto(351.70927803,3.98563607)
\curveto(351.05823644,4.02730274)(350.43063235,4.04813607)(349.82646576,4.04813607)
\curveto(349.2275075,4.04813607)(348.60250758,4.02730274)(347.95146599,3.98563607)
\lineto(347.8811535,4.05594857)
\lineto(347.8811535,4.38407352)
\lineto(347.95146599,4.44657352)
\curveto(348.45667426,4.47782351)(348.74052839,4.50126101)(348.80302839,4.51688601)
\curveto(348.87073671,4.53251101)(348.92802837,4.563761)(348.97490336,4.610636)
\curveto(349.02698669,4.66271932)(349.06084085,4.73824015)(349.07646585,4.83719847)
\curveto(349.09729918,4.94136512)(349.11813251,5.27730258)(349.13896584,5.84501084)
\curveto(349.16500751,6.41792744)(349.17802834,6.85021905)(349.17802834,7.14188568)
\lineto(349.17802834,8.68094799)
\curveto(349.17802834,8.8892813)(349.16761167,9.17313543)(349.14677834,9.53251039)
\curveto(349.12594501,9.89188535)(349.10771585,10.11323948)(349.09209085,10.19657281)
\curveto(349.08167418,10.27990613)(349.04261169,10.33980196)(348.97490336,10.37626029)
\curveto(348.90719504,10.41792695)(348.77177839,10.43876028)(348.56865341,10.43876028)
\lineto(347.9124035,10.44657278)
\lineto(347.842091,10.50907277)
\lineto(347.842091,10.84501023)
\lineto(347.904591,10.90751022)
\curveto(348.89938254,11.02730187)(349.7379241,11.22261435)(350.42021569,11.49344765)
\closepath
\moveto(349.81084076,3.57938612)
\closepath
}
}
{
\newrgbcolor{curcolor}{0 0 0}
\pscustom[linestyle=none,fillstyle=solid,fillcolor=curcolor]
{
\newpath
\moveto(355.03740262,11.49344765)
\lineto(355.1858401,11.39188516)
\curveto(355.1545901,11.0481352)(355.13375677,10.63667692)(355.12334011,10.15751031)
\lineto(355.86552751,10.83719773)
\curveto(356.07386082,11.02990604)(356.21188164,11.15230186)(356.27958996,11.20438518)
\curveto(356.34729829,11.26167684)(356.50615244,11.326781)(356.7561524,11.39969766)
\curveto(357.00615237,11.47261432)(357.26917317,11.50907265)(357.54521481,11.50907265)
\curveto(358.06604808,11.50907265)(358.52177719,11.37886433)(358.91240214,11.11844769)
\curveto(359.30302709,10.86323939)(359.59208956,10.50907277)(359.77958953,10.05594783)
\curveto(360.48271445,10.72261441)(360.89156856,11.09240603)(361.00615188,11.16532269)
\curveto(361.12594353,11.23823935)(361.31344351,11.311156)(361.56865181,11.38407266)
\curveto(361.82386011,11.46219765)(362.07906842,11.50126015)(362.33427672,11.50126015)
\curveto(362.745735,11.50126015)(363.12333912,11.41271849)(363.46708908,11.23563518)
\curveto(363.81083903,11.05855187)(364.0842765,10.82678106)(364.28740148,10.54032277)
\curveto(364.49052645,10.25386447)(364.60771394,9.94396867)(364.63896393,9.61063538)
\curveto(364.67542226,9.27730209)(364.69365143,8.81896881)(364.69365143,8.23563555)
\lineto(364.69365143,7.41532315)
\curveto(364.69365143,7.3267815)(364.70927642,6.67834408)(364.74052642,5.47001089)
\curveto(364.75094309,4.97521929)(364.80563058,4.68094849)(364.9045889,4.5871985)
\curveto(365.00354722,4.49344851)(365.32385968,4.44657352)(365.86552628,4.44657352)
\lineto(365.92802627,4.38407352)
\lineto(365.92802627,4.04813607)
\lineto(365.86552628,3.98563607)
\curveto(365.20406803,4.02730274)(364.75615142,4.04813607)(364.52177645,4.04813607)
\curveto(364.39156813,4.04813607)(364.01396401,4.02730274)(363.38896409,3.98563607)
\lineto(363.2874016,4.07157356)
\curveto(363.35510992,4.74865681)(363.38896409,5.6314692)(363.38896409,6.72001074)
\lineto(363.38896409,7.65751062)
\curveto(363.38896409,8.48563552)(363.34990159,9.06376045)(363.2717766,9.39188541)
\curveto(363.19365161,9.7252187)(363.01135997,9.996052)(362.72490167,10.20438531)
\curveto(362.43844337,10.41792695)(362.09729758,10.52469777)(361.70146429,10.52469777)
\curveto(361.42021433,10.52469777)(361.15719353,10.46740611)(360.91240189,10.35282279)
\curveto(360.66761026,10.2434478)(360.44625612,10.07678116)(360.24833947,9.85282285)
\curveto(360.05563116,9.63407288)(359.94886034,9.42834374)(359.92802701,9.23563543)
\curveto(359.90719368,9.04813545)(359.89677702,8.69917716)(359.89677702,8.18876056)
\lineto(359.89677702,7.21219818)
\curveto(359.89677702,6.98303154)(359.90719368,6.55334409)(359.92802701,5.92313583)
\curveto(359.94886034,5.29292758)(359.96969368,4.92574012)(359.99052701,4.82157347)
\curveto(360.01656867,4.71740682)(360.053027,4.64188599)(360.09990199,4.595011)
\curveto(360.15198532,4.55334434)(360.20927698,4.52469851)(360.27177697,4.50907351)
\curveto(360.3394853,4.49865684)(360.62073526,4.47782351)(361.11552687,4.44657352)
\lineto(361.18583936,4.38407352)
\lineto(361.18583936,4.05594857)
\lineto(361.12333937,3.98563607)
\curveto(360.47229778,4.02730274)(359.84469369,4.04813607)(359.2405271,4.04813607)
\curveto(358.6832355,4.04813607)(358.05823558,4.02730274)(357.36552733,3.98563607)
\lineto(357.29521484,4.05594857)
\lineto(357.29521484,4.38407352)
\lineto(357.36552733,4.44657352)
\curveto(357.8707356,4.47782351)(358.15458973,4.50126101)(358.21708972,4.51688601)
\curveto(358.28479805,4.53251101)(358.34208971,4.563761)(358.3889647,4.610636)
\curveto(358.44104803,4.66271932)(358.47490219,4.73824015)(358.49052719,4.83719847)
\curveto(358.51136052,4.94136512)(358.53219385,5.27730258)(358.55302718,5.84501084)
\curveto(358.57906885,6.41792744)(358.59208968,6.85021905)(358.59208968,7.14188568)
\lineto(358.59208968,8.03251057)
\curveto(358.59208968,8.62105217)(358.55042302,9.07417711)(358.46708969,9.39188541)
\curveto(358.3889647,9.7095937)(358.21708972,9.97782283)(357.95146476,10.19657281)
\curveto(357.68583979,10.41532278)(357.3525065,10.52469777)(356.95146488,10.52469777)
\curveto(356.63896492,10.52469777)(356.35250662,10.46219778)(356.09208999,10.33719779)
\curveto(355.83688168,10.21740614)(355.62594421,10.06636449)(355.45927756,9.88407285)
\curveto(355.29781925,9.70698953)(355.19625676,9.54032289)(355.1545901,9.38407291)
\curveto(355.11813177,9.22782293)(355.09990261,8.8892813)(355.09990261,8.36844803)
\lineto(355.09990261,7.21219818)
\curveto(355.09990261,6.98303154)(355.11031927,6.55334409)(355.13115261,5.92313583)
\curveto(355.15198594,5.29292758)(355.17281927,4.92574012)(355.1936526,4.82157347)
\curveto(355.21969426,4.71740682)(355.25615259,4.64188599)(355.30302758,4.595011)
\curveto(355.35511091,4.55334434)(355.41240257,4.52469851)(355.47490256,4.50907351)
\curveto(355.54261089,4.49865684)(355.82386085,4.47782351)(356.31865246,4.44657352)
\lineto(356.38896495,4.38407352)
\lineto(356.38896495,4.05594857)
\lineto(356.32646496,3.98563607)
\curveto(355.67542337,4.02730274)(355.04781928,4.04813607)(354.44365269,4.04813607)
\curveto(353.84469443,4.04813607)(353.21969451,4.02730274)(352.56865292,3.98563607)
\lineto(352.49834043,4.05594857)
\lineto(352.49834043,4.38407352)
\lineto(352.56865292,4.44657352)
\curveto(353.07386119,4.47782351)(353.35771532,4.50126101)(353.42021532,4.51688601)
\curveto(353.48792364,4.53251101)(353.5452153,4.563761)(353.59209029,4.610636)
\curveto(353.64417362,4.66271932)(353.67802778,4.73824015)(353.69365278,4.83719847)
\curveto(353.71448611,4.94136512)(353.73531944,5.27730258)(353.75615277,5.84501084)
\curveto(353.78219444,6.41792744)(353.79521527,6.85021905)(353.79521527,7.14188568)
\lineto(353.79521527,8.68094799)
\curveto(353.79521527,8.8892813)(353.7847986,9.17313543)(353.76396527,9.53251039)
\curveto(353.74313194,9.89188535)(353.72490278,10.11323948)(353.70927778,10.19657281)
\curveto(353.69886112,10.27990613)(353.65979862,10.33980196)(353.59209029,10.37626029)
\curveto(353.52438197,10.41792695)(353.38896532,10.43876028)(353.18584034,10.43876028)
\lineto(352.52959043,10.44657278)
\lineto(352.45927793,10.50907277)
\lineto(352.45927793,10.84501023)
\lineto(352.52177793,10.90751022)
\curveto(353.51656947,11.02730187)(354.35511103,11.22261435)(355.03740262,11.49344765)
\closepath
}
}
{
\newrgbcolor{curcolor}{0 0 0}
\pscustom[linestyle=none,fillstyle=solid,fillcolor=curcolor]
{
\newpath
\moveto(368.53740095,14.98563472)
\curveto(368.78219259,14.98563472)(368.9905259,14.89969723)(369.16240087,14.72782225)
\curveto(369.33427585,14.55594727)(369.42021334,14.34761396)(369.42021334,14.10282233)
\curveto(369.42021334,13.86323902)(369.33427585,13.65750988)(369.16240087,13.4856349)
\curveto(368.9905259,13.31375992)(368.78219259,13.22782243)(368.53740095,13.22782243)
\curveto(368.29781765,13.22782243)(368.08948434,13.31115576)(367.91240103,13.4778224)
\curveto(367.74052605,13.64969738)(367.65458856,13.85803069)(367.65458856,14.10282233)
\curveto(367.65458856,14.34761396)(367.74052605,14.55594727)(367.91240103,14.72782225)
\curveto(368.08948434,14.89969723)(368.29781765,14.98563472)(368.53740095,14.98563472)
\closepath
\moveto(369.20146337,11.49344765)
\lineto(369.34990085,11.39188516)
\curveto(369.29260919,10.70959358)(369.26396336,9.84240618)(369.26396336,8.79032298)
\lineto(369.26396336,7.0246982)
\curveto(369.26396336,6.91011488)(369.27438003,6.54032326)(369.29521336,5.91532334)
\curveto(369.31604669,5.29553175)(369.33688002,4.93355262)(369.35771335,4.82938597)
\curveto(369.38375501,4.72521932)(369.42021334,4.64969849)(369.46708834,4.6028235)
\curveto(369.51396333,4.5559485)(369.57125499,4.52469851)(369.63896332,4.50907351)
\curveto(369.70667164,4.49865684)(369.98792161,4.47782351)(370.48271321,4.44657352)
\lineto(370.5530257,4.38407352)
\lineto(370.5530257,4.05594857)
\lineto(370.49052571,3.98563607)
\curveto(369.83948412,4.02730274)(369.21188004,4.04813607)(368.60771344,4.04813607)
\curveto(368.00875518,4.04813607)(367.38375526,4.02730274)(366.73271367,3.98563607)
\lineto(366.66240118,4.05594857)
\lineto(366.66240118,4.38407352)
\lineto(366.73271367,4.44657352)
\curveto(367.23792195,4.47782351)(367.52177608,4.50126101)(367.58427607,4.51688601)
\curveto(367.65198439,4.53251101)(367.70927605,4.563761)(367.75615105,4.610636)
\curveto(367.80823438,4.66271932)(367.84208854,4.73824015)(367.85771354,4.83719847)
\curveto(367.87854687,4.94136512)(367.8993802,5.27730258)(367.92021353,5.84501084)
\curveto(367.94625519,6.41792744)(367.95927602,6.85021905)(367.95927602,7.14188568)
\lineto(367.95927602,8.68094799)
\curveto(367.95927602,8.8892813)(367.94885936,9.17313543)(367.92802603,9.53251039)
\curveto(367.9071927,9.89188535)(367.88896353,10.11323948)(367.87333853,10.19657281)
\curveto(367.86292187,10.27990613)(367.82385937,10.33980196)(367.75615105,10.37626029)
\curveto(367.68844272,10.41792695)(367.55302607,10.43876028)(367.3499011,10.43876028)
\lineto(366.69365118,10.44657278)
\lineto(366.62333869,10.50907277)
\lineto(366.62333869,10.84501023)
\lineto(366.68583868,10.90751022)
\curveto(367.68063022,11.02730187)(368.51917179,11.22261435)(369.20146337,11.49344765)
\closepath
\moveto(368.59208845,3.57938612)
\closepath
}
}
{
\newrgbcolor{curcolor}{0 0 0}
\pscustom[linestyle=none,fillstyle=solid,fillcolor=curcolor]
{
\newpath
\moveto(377.8498998,5.19657342)
\lineto(377.59989983,4.63407349)
\curveto(377.05823323,4.2851152)(376.57385829,4.05855273)(376.14677501,3.95438608)
\curveto(375.72490007,3.85021942)(375.34729595,3.7981361)(375.01396265,3.7981361)
\curveto(374.38896273,3.7981361)(373.7952128,3.92053192)(373.23271287,4.16532355)
\curveto(372.67542128,4.41011519)(372.21969216,4.82417764)(371.86552554,5.4075109)
\curveto(371.51656725,5.99084416)(371.34208811,6.69396907)(371.34208811,7.51688564)
\curveto(371.34208811,8.06376057)(371.40979643,8.55594801)(371.54521308,8.99344796)
\curveto(371.68062973,9.43615623)(371.82125471,9.76428119)(371.96708803,9.97782283)
\curveto(372.11812968,10.19136448)(372.37073381,10.42834361)(372.72490044,10.68876025)
\curveto(373.07906706,10.94917688)(373.45406701,11.15751019)(373.8499003,11.31376017)
\curveto(374.24573358,11.47001015)(374.67281686,11.54813514)(375.13115014,11.54813514)
\curveto(375.75615006,11.54813514)(376.30562916,11.39969766)(376.77958744,11.1028227)
\curveto(377.25875404,10.81115607)(377.5946915,10.43615611)(377.78739981,9.97782283)
\curveto(377.98010812,9.51948956)(378.07646228,9.03251045)(378.07646228,8.51688551)
\curveto(378.07646228,8.3554272)(378.06864978,8.19917722)(378.05302478,8.04813557)
\lineto(377.96708729,7.96219808)
\curveto(377.61292067,7.88407309)(377.13635823,7.83198977)(376.53739997,7.8059481)
\curveto(375.93844171,7.77990644)(375.54260842,7.76688561)(375.34990011,7.76688561)
\lineto(372.84208792,7.76688561)
\curveto(372.85250459,6.68876074)(373.12333789,5.89449001)(373.65458782,5.3840734)
\curveto(374.18583776,4.8736568)(374.83687934,4.6184485)(375.60771258,4.6184485)
\curveto(375.97229587,4.6184485)(376.32125416,4.68094849)(376.65458745,4.80594847)
\curveto(376.99312908,4.93094846)(377.34989987,5.1054276)(377.72489982,5.32938591)
\closepath
\moveto(372.84208792,8.39188553)
\curveto(372.93583791,8.37626053)(373.29521287,8.35803137)(373.92021279,8.33719804)
\curveto(374.55042104,8.31636471)(375.01656682,8.30594804)(375.31865012,8.30594804)
\curveto(376.04260836,8.30594804)(376.48271247,8.31896887)(376.63896245,8.34501054)
\curveto(376.64417079,8.47001052)(376.64677495,8.56636468)(376.64677495,8.634073)
\curveto(376.64677495,9.44136457)(376.48271247,10.04032283)(376.15458751,10.43094778)
\curveto(375.82646255,10.82678106)(375.37854594,11.02469771)(374.81083768,11.02469771)
\curveto(374.19104609,11.02469771)(373.70667115,10.80334357)(373.35771286,10.36063529)
\curveto(373.0139629,9.91792701)(372.84208792,9.26167709)(372.84208792,8.39188553)
\closepath
\moveto(375.02177515,11.98563509)
\closepath
\moveto(374.92802516,3.57938612)
\closepath
}
}
{
\newrgbcolor{curcolor}{0 0 0}
\pscustom[linestyle=none,fillstyle=solid,fillcolor=curcolor]
{
\newpath
\moveto(381.60771184,11.49344765)
\lineto(381.75614932,11.39188516)
\curveto(381.72489933,11.08980186)(381.70146183,10.5533436)(381.68583683,9.78251036)
\lineto(382.27177426,10.52469777)
\curveto(382.46448257,10.7694894)(382.63375338,10.95698938)(382.7795867,11.0871977)
\curveto(382.93062834,11.22261435)(383.10510749,11.326781)(383.30302413,11.39969766)
\curveto(383.50094077,11.47261432)(383.70406575,11.50907265)(383.91239906,11.50907265)
\curveto(384.14156569,11.50907265)(384.3577115,11.46219765)(384.56083648,11.36844766)
\lineto(384.61552397,11.25907268)
\curveto(384.53219065,10.56636443)(384.48531565,9.9569895)(384.47489899,9.4309479)
\lineto(384.12333653,9.4309479)
\curveto(383.91500322,9.91011451)(383.57906576,10.14969781)(383.11552415,10.14969781)
\curveto(382.79260753,10.14969781)(382.51135756,10.04553116)(382.27177426,9.83719785)
\curveto(382.03219095,9.63407288)(381.87073264,9.37626041)(381.78739932,9.06376045)
\curveto(381.70927433,8.75646882)(381.67021183,8.36584387)(381.67021183,7.89188559)
\lineto(381.67021183,7.0246982)
\curveto(381.67021183,6.86844822)(381.6806285,6.48563577)(381.70146183,5.87626084)
\curveto(381.72229516,5.26688592)(381.74312849,4.91532346)(381.76396182,4.82157347)
\curveto(381.79000348,4.72782348)(381.82646181,4.65751099)(381.87333681,4.610636)
\curveto(381.92542013,4.56896934)(381.99052429,4.54032351)(382.06864928,4.52469851)
\curveto(382.15198261,4.50907351)(382.52437839,4.48303185)(383.18583665,4.44657352)
\lineto(383.25614914,4.38407352)
\lineto(383.25614914,4.05594857)
\lineto(383.18583665,3.98563607)
\curveto(382.4931284,4.02730274)(381.76917015,4.04813607)(381.01396191,4.04813607)
\curveto(380.41500365,4.04813607)(379.79000373,4.02730274)(379.13896214,3.98563607)
\lineto(379.06864965,4.05594857)
\lineto(379.06864965,4.38407352)
\lineto(379.13896214,4.44657352)
\curveto(379.64417042,4.47782351)(379.92802455,4.50126101)(379.99052454,4.51688601)
\curveto(380.05823286,4.53251101)(380.11552452,4.563761)(380.16239952,4.610636)
\curveto(380.21448285,4.66271932)(380.24833701,4.73824015)(380.26396201,4.83719847)
\curveto(380.28479534,4.94136512)(380.30562867,5.27730258)(380.326462,5.84501084)
\curveto(380.35250366,6.41792744)(380.36552449,6.85021905)(380.36552449,7.14188568)
\lineto(380.36552449,8.68094799)
\curveto(380.36552449,8.8892813)(380.35510783,9.17313543)(380.3342745,9.53251039)
\curveto(380.31344117,9.89188535)(380.295212,10.11323948)(380.279587,10.19657281)
\curveto(380.26917034,10.27990613)(380.23010784,10.33980196)(380.16239952,10.37626029)
\curveto(380.09469119,10.41792695)(379.95927454,10.43876028)(379.75614957,10.43876028)
\lineto(379.09989965,10.44657278)
\lineto(379.02958716,10.50907277)
\lineto(379.02958716,10.84501023)
\lineto(379.09208715,10.90751022)
\curveto(380.08687869,11.02730187)(380.92542026,11.22261435)(381.60771184,11.49344765)
\closepath
}
}
{
\newrgbcolor{curcolor}{0 0 0}
\pscustom[linestyle=none,fillstyle=solid,fillcolor=curcolor]
{
\newpath
\moveto(387.6311486,11.49344765)
\lineto(387.77958608,11.39188516)
\curveto(387.72229442,10.70959358)(387.69364859,9.84240618)(387.69364859,8.79032298)
\lineto(387.69364859,6.97001071)
\curveto(387.69364859,6.30334412)(387.74312775,5.83980251)(387.84208607,5.57938588)
\curveto(387.94625273,5.31896924)(388.12333604,5.11844843)(388.37333601,4.97782345)
\curveto(388.62333598,4.8424068)(388.92021094,4.77469848)(389.2639609,4.77469848)
\curveto(389.64416918,4.77469848)(389.99052331,4.84761513)(390.30302327,4.99344845)
\curveto(390.61552323,5.1444901)(390.87593986,5.35542757)(391.08427317,5.62626087)
\curveto(391.29781481,5.89709417)(391.42021063,6.07938582)(391.45146063,6.1731358)
\curveto(391.48271062,6.26688579)(391.50354395,6.52990659)(391.51396062,6.96219821)
\lineto(391.53739812,7.8371981)
\lineto(391.53739812,8.68094799)
\curveto(391.53739812,8.8892813)(391.52698145,9.17313543)(391.50614812,9.53251039)
\curveto(391.48531479,9.89188535)(391.46708562,10.11323948)(391.45146063,10.19657281)
\curveto(391.44104396,10.27990613)(391.40198147,10.33980196)(391.33427314,10.37626029)
\curveto(391.26656482,10.41792695)(391.13114817,10.43876028)(390.92802319,10.43876028)
\lineto(390.27177327,10.44657278)
\lineto(390.20146078,10.50907277)
\lineto(390.20146078,10.84501023)
\lineto(390.26396077,10.90751022)
\curveto(391.25875232,11.02730187)(392.09729388,11.22261435)(392.77958546,11.49344765)
\lineto(392.92802294,11.39188516)
\curveto(392.87073128,10.70959358)(392.84208546,9.84240618)(392.84208546,8.79032298)
\lineto(392.84208546,7.41532315)
\curveto(392.84208546,7.33719816)(392.85510629,6.69917741)(392.88114795,5.50126089)
\curveto(392.89156462,5.05855261)(392.91760628,4.78771931)(392.95927294,4.68876099)
\curveto(393.00614793,4.595011)(393.06864793,4.52730267)(393.14677292,4.48563601)
\curveto(393.22489791,4.44917768)(393.45666871,4.43094852)(393.84208533,4.43094852)
\lineto(394.0608353,4.43094852)
\lineto(394.1311478,4.36844853)
\lineto(394.1311478,4.05594857)
\lineto(394.0686478,3.98563607)
\curveto(393.33427289,4.02730274)(392.84208546,4.04813607)(392.59208549,4.04813607)
\curveto(392.27437719,4.04813607)(391.90718974,4.0299069)(391.49052312,3.99344857)
\lineto(391.42021063,4.05594857)
\curveto(391.45146063,4.5871985)(391.47489812,5.03771928)(391.49052312,5.4075109)
\curveto(391.16239816,5.1523026)(390.78739821,4.81115681)(390.36552326,4.38407352)
\curveto(390.20406495,4.22261521)(389.97229414,4.08719856)(389.67021085,3.97782357)
\curveto(389.36812755,3.86844859)(389.02437759,3.81376109)(388.63896097,3.81376109)
\curveto(388.05562771,3.81376109)(387.5998986,3.90490692)(387.27177364,4.08719856)
\curveto(386.94885702,4.27469854)(386.71969038,4.52990684)(386.58427373,4.85282347)
\curveto(386.44885708,5.18094843)(386.38114875,5.74344836)(386.38114875,6.54032326)
\lineto(386.38896125,7.15751068)
\lineto(386.38896125,8.68094799)
\curveto(386.38896125,8.8892813)(386.37854459,9.17313543)(386.35771125,9.53251039)
\curveto(386.33687792,9.89188535)(386.31864876,10.11323948)(386.30302376,10.19657281)
\curveto(386.2926071,10.27990613)(386.2535446,10.33980196)(386.18583628,10.37626029)
\curveto(386.11812795,10.41792695)(385.9827113,10.43876028)(385.77958633,10.43876028)
\lineto(385.12333641,10.44657278)
\lineto(385.05302392,10.50907277)
\lineto(385.05302392,10.84501023)
\lineto(385.11552391,10.90751022)
\curveto(386.11031545,11.02730187)(386.94885702,11.22261435)(387.6311486,11.49344765)
\closepath
\moveto(389.49833587,11.98563509)
\closepath
\moveto(389.57646086,3.57938612)
\closepath
}
}
{
\newrgbcolor{curcolor}{0 0 0}
\pscustom[linestyle=none,fillstyle=solid,fillcolor=curcolor]
{
\newpath
\moveto(397.24833491,11.49344765)
\lineto(397.39677239,11.39188516)
\curveto(397.3655224,11.04292687)(397.34468907,10.58459359)(397.3342724,10.01688533)
\curveto(397.75093902,10.36063529)(398.1467723,10.70959358)(398.52177225,11.0637602)
\curveto(398.63114724,11.16271852)(398.73270973,11.23563518)(398.82645972,11.28251017)
\curveto(398.92541804,11.32938517)(399.08687635,11.37886433)(399.31083466,11.43094766)
\curveto(399.53479296,11.48303098)(399.7639596,11.50907265)(399.99833457,11.50907265)
\curveto(400.39416786,11.50907265)(400.77698031,11.42834349)(401.14677193,11.26688518)
\curveto(401.52177188,11.10542686)(401.80302185,10.91271855)(401.99052183,10.68876025)
\curveto(402.18323014,10.47001027)(402.31604262,10.20959364)(402.38895928,9.90751034)
\curveto(402.46187594,9.60542705)(402.49833426,9.23303126)(402.49833426,8.79032298)
\lineto(402.49833426,7.41532315)
\curveto(402.49833426,7.3267815)(402.5113551,6.67834408)(402.53739676,5.47001089)
\curveto(402.54781343,4.97521929)(402.60250092,4.68094849)(402.70145924,4.5871985)
\curveto(402.80041756,4.49344851)(403.12333419,4.44657352)(403.67020912,4.44657352)
\lineto(403.73270911,4.38407352)
\lineto(403.73270911,4.04813607)
\lineto(403.67020912,3.98563607)
\curveto(403.00875087,4.02730274)(402.56083426,4.04813607)(402.32645929,4.04813607)
\curveto(402.19104264,4.04813607)(401.81083435,4.02730274)(401.18583443,3.98563607)
\lineto(401.09208444,4.07157356)
\curveto(401.15979276,4.74865681)(401.19364693,5.6314692)(401.19364693,6.72001074)
\lineto(401.19364693,7.74344811)
\curveto(401.19364693,8.35803137)(401.17802193,8.80073965)(401.14677193,9.07157295)
\curveto(401.12073027,9.34240625)(401.03479278,9.58719788)(400.88895946,9.80594786)
\curveto(400.74312615,10.02990616)(400.54781367,10.20178114)(400.30302204,10.32157279)
\curveto(400.0582304,10.44657278)(399.76656377,10.50907277)(399.42802214,10.50907277)
\curveto(399.15718884,10.50907277)(398.92541804,10.48042694)(398.73270973,10.42313528)
\curveto(398.54000142,10.37105195)(398.32385561,10.2590728)(398.08427231,10.08719782)
\curveto(397.844689,9.92053118)(397.66760569,9.76167703)(397.55302237,9.61063538)
\curveto(397.43843906,9.46480206)(397.36812656,9.32678125)(397.3420849,9.19657293)
\curveto(397.32125157,9.07157295)(397.3108349,8.80073965)(397.3108349,8.38407303)
\lineto(397.3108349,7.0246982)
\curveto(397.3108349,6.91011488)(397.32125157,6.54032326)(397.3420849,5.91532334)
\curveto(397.36291823,5.29553175)(397.38375156,4.93355262)(397.40458489,4.82938597)
\curveto(397.43062656,4.72521932)(397.46708488,4.64969849)(397.51395988,4.6028235)
\curveto(397.56083487,4.5559485)(397.61812653,4.52469851)(397.68583486,4.50907351)
\curveto(397.75354318,4.49865684)(398.03479315,4.47782351)(398.52958475,4.44657352)
\lineto(398.59989725,4.38407352)
\lineto(398.59989725,4.05594857)
\lineto(398.53739725,3.98563607)
\curveto(397.88635567,4.02730274)(397.25875158,4.04813607)(396.65458499,4.04813607)
\curveto(396.05562673,4.04813607)(395.4306268,4.02730274)(394.77958522,3.98563607)
\lineto(394.70927272,4.05594857)
\lineto(394.70927272,4.38407352)
\lineto(394.77958522,4.44657352)
\curveto(395.28479349,4.47782351)(395.56864762,4.50126101)(395.63114761,4.51688601)
\curveto(395.69885594,4.53251101)(395.7561476,4.563761)(395.80302259,4.610636)
\curveto(395.85510592,4.66271932)(395.88896008,4.73824015)(395.90458508,4.83719847)
\curveto(395.92541841,4.94136512)(395.94625174,5.27730258)(395.96708507,5.84501084)
\curveto(395.99312673,6.41792744)(396.00614757,6.85021905)(396.00614757,7.14188568)
\lineto(396.00614757,8.68094799)
\curveto(396.00614757,8.8892813)(395.9957309,9.17313543)(395.97489757,9.53251039)
\curveto(395.95406424,9.89188535)(395.93583507,10.11323948)(395.92021008,10.19657281)
\curveto(395.90979341,10.27990613)(395.87073092,10.33980196)(395.80302259,10.37626029)
\curveto(395.73531427,10.41792695)(395.59989762,10.43876028)(395.39677264,10.43876028)
\lineto(394.74052272,10.44657278)
\lineto(394.67021023,10.50907277)
\lineto(394.67021023,10.84501023)
\lineto(394.73271022,10.90751022)
\curveto(395.72750177,11.02730187)(396.56604333,11.22261435)(397.24833491,11.49344765)
\closepath
}
}
{
\newrgbcolor{curcolor}{0 0 0}
\pscustom[linestyle=none,fillstyle=solid,fillcolor=curcolor]
{
\newpath
\moveto(412.50614553,10.95438521)
\lineto(412.56864552,10.81376023)
\curveto(412.35510388,10.48563527)(412.2222914,10.27209363)(412.17020807,10.17313531)
\lineto(410.71708325,10.17313531)
\curveto(410.86291657,9.88667701)(410.93583322,9.58719788)(410.93583322,9.27469792)
\curveto(410.93583322,8.9049063)(410.84989573,8.54553134)(410.67802076,8.19657305)
\curveto(410.51135411,7.84761476)(410.27697914,7.5507398)(409.97489584,7.30594816)
\curveto(409.67281255,7.06115653)(409.32906259,6.86584405)(408.94364597,6.72001074)
\curveto(408.56343768,6.57417742)(408.1285419,6.50126076)(407.63895863,6.50126076)
\lineto(407.27958367,6.50126076)
\curveto(406.97229205,6.25646913)(406.7743754,6.07417748)(406.68583375,5.95438583)
\curveto(406.59729209,5.83459418)(406.55302126,5.71219836)(406.55302126,5.58719838)
\curveto(406.55302126,5.35803174)(406.65718792,5.19657342)(406.86552123,5.10282344)
\curveto(407.07906287,5.00907345)(407.51656281,4.96219845)(408.17802106,4.96219845)
\lineto(409.92802085,4.98563595)
\curveto(410.48010411,4.98563595)(410.90197906,4.92053179)(411.19364569,4.79032347)
\curveto(411.49052066,4.66011516)(411.73010396,4.42574019)(411.9123956,4.08719856)
\curveto(412.09468725,3.75386527)(412.18583307,3.39969865)(412.18583307,3.02469869)
\curveto(412.18583307,2.4517821)(411.99833309,1.88146967)(411.62333314,1.3137614)
\curveto(411.25354152,0.74084481)(410.71708325,0.30594903)(410.01395834,0.00907406)
\curveto(409.31083342,-0.2878009)(408.56083352,-0.43623838)(407.76395862,-0.43623838)
\curveto(407.31083367,-0.43623838)(406.88375039,-0.38675922)(406.48270877,-0.2878009)
\curveto(406.08687549,-0.18884258)(405.7379172,-0.03780093)(405.4358339,0.16532404)
\curveto(405.13895894,0.36324069)(404.89937564,0.62365732)(404.71708399,0.94657395)
\curveto(404.53479235,1.26949058)(404.44364652,1.60542803)(404.44364652,1.95438632)
\curveto(404.44364652,2.14709463)(404.47229235,2.35021961)(404.52958401,2.56376125)
\curveto(404.58687567,2.77209456)(404.69625066,2.99865703)(404.85770897,3.24344867)
\lineto(406.2717713,4.03251107)
\curveto(405.81864636,4.17313605)(405.53218806,4.31115687)(405.41239641,4.44657352)
\curveto(405.29781309,4.5871985)(405.24052143,4.75386515)(405.24052143,4.94657346)
\curveto(405.24052143,5.1444901)(405.30562559,5.37886507)(405.4358339,5.64969837)
\lineto(406.69364625,6.55594826)
\curveto(405.95406301,6.74865657)(405.45406307,7.03771903)(405.19364643,7.42313565)
\curveto(404.93843813,7.80855227)(404.81083398,8.23042722)(404.81083398,8.68876049)
\curveto(404.81083398,9.11063544)(404.9019798,9.51167706)(405.08427145,9.89188535)
\curveto(405.26656309,10.27730196)(405.54000056,10.58459359)(405.90458384,10.81376023)
\curveto(406.27437547,11.04292687)(406.68583375,11.22261435)(407.13895869,11.35282267)
\curveto(407.59729197,11.48823932)(408.01395858,11.55594764)(408.38895854,11.55594764)
\curveto(409.04000012,11.55594764)(409.65979171,11.34501017)(410.24833331,10.92313522)
\curveto(411.19104153,10.92313522)(411.9436456,10.93355188)(412.50614553,10.95438521)
\closepath
\moveto(406.15458381,9.14188544)
\curveto(406.15458381,8.82417714)(406.21708381,8.48303135)(406.34208379,8.11844806)
\curveto(406.46708378,7.75386478)(406.66760458,7.47782314)(406.94364622,7.29032317)
\curveto(407.22489618,7.10282319)(407.54000031,7.0090732)(407.8889586,7.0090732)
\curveto(408.35250021,7.0090732)(408.75093766,7.16011485)(409.08427095,7.46219815)
\curveto(409.42281258,7.76428144)(409.59208339,8.23563555)(409.59208339,8.87626047)
\curveto(409.59208339,9.41792707)(409.43843758,9.90490618)(409.13114595,10.33719779)
\curveto(408.82385432,10.7694894)(408.38895854,10.98563521)(407.82645861,10.98563521)
\curveto(407.35250033,10.98563521)(406.95406288,10.82938523)(406.63114625,10.51688527)
\curveto(406.31343796,10.20959364)(406.15458381,9.75126036)(406.15458381,9.14188544)
\closepath
\moveto(408.44364603,3.89969858)
\curveto(407.60510447,3.89969858)(407.12072953,3.88928192)(406.99052121,3.86844859)
\curveto(406.86552123,3.84761526)(406.68583375,3.7590736)(406.45145878,3.60282362)
\curveto(406.21708381,3.44657364)(406.0243755,3.23563617)(405.87333385,2.9700112)
\curveto(405.72750053,2.6991779)(405.65458388,2.3970946)(405.65458388,2.06376131)
\curveto(405.65458388,1.47521972)(405.87333385,0.99605311)(406.31083379,0.62626149)
\curveto(406.74833374,0.25646987)(407.331667,0.07157406)(408.06083358,0.07157406)
\curveto(408.60770851,0.07157406)(409.11291678,0.18615738)(409.57645839,0.41532401)
\curveto(410.04,0.64449065)(410.38114579,0.95178228)(410.59989577,1.3371989)
\curveto(410.82385407,1.72261552)(410.93583322,2.09240714)(410.93583322,2.44657376)
\curveto(410.93583322,2.80074039)(410.84208324,3.10542785)(410.65458326,3.36063615)
\curveto(410.47229161,3.61063612)(410.23270831,3.76428193)(409.93583335,3.82157359)
\curveto(409.64416672,3.87365692)(409.14677094,3.89969858)(408.44364603,3.89969858)
\closepath
}
}
{
\newrgbcolor{curcolor}{1 1 1}
\pscustom[linestyle=none,fillstyle=solid,fillcolor=curcolor]
{
\newpath
\moveto(27.3012357,52.32015635)
\curveto(27.3012357,48.14540588)(23.91693177,44.76110195)(19.74218131,44.76110195)
\curveto(15.56743084,44.76110195)(12.18312691,48.14540588)(12.18312691,52.32015635)
\curveto(12.18312691,56.49490681)(15.56743084,59.87921074)(19.74218131,59.87921074)
\curveto(23.91693177,59.87921074)(27.3012357,56.49490681)(27.3012357,52.32015635)
\closepath
}
}
{
\newrgbcolor{curcolor}{0.15686275 0.16078432 0.16470589}
\pscustom[linewidth=2.88359956,linecolor=curcolor]
{
\newpath
\moveto(27.3012357,52.32015635)
\curveto(27.3012357,48.14540588)(23.91693177,44.76110195)(19.74218131,44.76110195)
\curveto(15.56743084,44.76110195)(12.18312691,48.14540588)(12.18312691,52.32015635)
\curveto(12.18312691,56.49490681)(15.56743084,59.87921074)(19.74218131,59.87921074)
\curveto(23.91693177,59.87921074)(27.3012357,56.49490681)(27.3012357,52.32015635)
\closepath
}
}
{
\newrgbcolor{curcolor}{0 0 0}
\pscustom[linestyle=none,fillstyle=solid,fillcolor=curcolor]
{
\newpath
\moveto(23.18358713,50.95687526)
\curveto(23.18358713,49.8318754)(22.74869135,48.87093802)(21.87889979,48.07406312)
\curveto(20.9882749,47.25114655)(19.89452504,46.83968827)(18.5976502,46.83968827)
\curveto(18.0403586,46.83968827)(17.44140034,46.96208409)(16.80077542,47.20687573)
\curveto(16.76952542,47.50375069)(16.60285878,48.23291727)(16.30077548,49.39437546)
\lineto(16.34765047,49.50375044)
\lineto(16.72265043,49.63656293)
\lineto(16.83202541,49.58968793)
\curveto(17.35285868,48.38656308)(18.18879608,47.78500066)(19.3398376,47.78500066)
\curveto(20.15754584,47.78500066)(20.7330666,48.14177144)(21.06639989,48.85531302)
\curveto(21.21744154,49.17822965)(21.29296236,49.5532296)(21.29296236,49.98031288)
\curveto(21.29296236,50.67822947)(21.10546239,51.22770856)(20.73046243,51.62875018)
\curveto(20.35546248,52.03500013)(19.83723338,52.23812511)(19.17577512,52.23812511)
\curveto(18.5820252,52.23812511)(17.98567111,52.01937513)(17.38671285,51.58187519)
\lineto(17.08202538,51.75375017)
\curveto(17.14973371,52.75895838)(17.18358787,53.57927077)(17.18358787,54.21468736)
\lineto(17.18358787,55.29281223)
\curveto(17.18358787,56.06364547)(17.14452538,56.8735412)(17.06640039,57.72249943)
\lineto(17.16796287,57.80062442)
\curveto(18.42837938,57.74333276)(19.48827509,57.71468693)(20.34764998,57.71468693)
\curveto(20.90494158,57.71468693)(21.69660815,57.74333276)(22.72264969,57.80062442)
\lineto(22.79296218,57.67562444)
\curveto(22.73046219,57.1599995)(22.69921219,56.78760371)(22.69921219,56.55843707)
\curveto(22.69921219,56.42822876)(22.70702469,56.24854128)(22.72264969,56.01937464)
\lineto(22.6445247,55.91781215)
\curveto(22.00389978,55.94906215)(21.2356707,55.96468715)(20.33983748,55.96468715)
\curveto(19.26171261,55.96468715)(18.49087938,55.94385382)(18.02733777,55.90218715)
\lineto(17.96483777,53.34749997)
\curveto(18.72525435,53.70687493)(19.43098343,53.8865624)(20.08202501,53.8865624)
\curveto(21.0351499,53.8865624)(21.79035814,53.62093744)(22.34764973,53.0896875)
\curveto(22.90494133,52.55843757)(23.18358713,51.84750015)(23.18358713,50.95687526)
\closepath
}
}
{
\newrgbcolor{curcolor}{0 0 0}
\pscustom[linestyle=none,fillstyle=solid,fillcolor=curcolor]
{
\newpath
\moveto(34.0374422,58.3518125)
\lineto(34.09994219,58.4065)
\curveto(35.00098374,58.36483334)(35.615567,58.344)(35.94369196,58.344)
\lineto(39.59994151,58.42212499)
\curveto(40.64160805,58.42212499)(41.49577461,58.32316667)(42.1624412,58.12525003)
\curveto(42.82910778,57.93254172)(43.42025354,57.5940001)(43.93587848,57.10962516)
\curveto(44.45671175,56.63045855)(44.84994086,56.07056278)(45.11556583,55.42993786)
\curveto(45.38639913,54.79452128)(45.52181578,54.11483386)(45.52181578,53.39087562)
\curveto(45.52181578,52.6877507)(45.39941996,51.99243829)(45.15462833,51.30493837)
\curveto(44.90983669,50.61743846)(44.5608784,50.00025103)(44.10775346,49.4533761)
\curveto(43.65983684,48.9117095)(43.14681607,48.47420955)(42.56869115,48.14087626)
\curveto(41.99577455,47.80754297)(41.42025379,47.58358466)(40.84212886,47.46900135)
\curveto(40.26400393,47.35962636)(39.67285817,47.30493887)(39.06869158,47.30493887)
\curveto(38.66244163,47.30493887)(37.89681672,47.32056386)(36.77181686,47.35181386)
\curveto(36.40202524,47.36223053)(36.1572336,47.36743886)(36.03744195,47.36743886)
\curveto(35.67806699,47.36743886)(35.24317122,47.34660553)(34.73275461,47.30493887)
\lineto(34.67025462,47.36743886)
\lineto(34.67025462,47.62525133)
\lineto(34.73275461,47.70337632)
\curveto(35.04004624,47.8596263)(35.21452539,47.95598045)(35.25619205,47.99243878)
\curveto(35.30306704,48.03410544)(35.33952537,48.48983455)(35.36556703,49.35962611)
\curveto(35.39681703,50.234626)(35.41244203,50.90129259)(35.41244203,51.35962587)
\lineto(35.41244203,54.22681301)
\lineto(35.40462953,56.00025029)
\curveto(35.40462953,56.38566691)(35.3994212,56.69816687)(35.38900453,56.93775018)
\curveto(35.37858787,57.18254181)(35.3603587,57.35702096)(35.33431704,57.46118761)
\curveto(35.30827537,57.56535427)(35.27702538,57.63827092)(35.24056705,57.67993759)
\curveto(35.20931705,57.72160425)(35.15462956,57.75545841)(35.07650457,57.78150007)
\curveto(35.00358791,57.81275007)(34.87598376,57.83879173)(34.69369212,57.85962506)
\lineto(34.09994219,57.90650006)
\lineto(34.0374422,57.96118755)
\closepath
\moveto(36.92806684,48.15650126)
\curveto(37.45931678,48.05233461)(38.07129587,48.00025128)(38.76400411,48.00025128)
\curveto(39.97754563,48.00025128)(40.94629551,48.20337625)(41.67025376,48.6096262)
\curveto(42.39942033,49.01587615)(42.9436911,49.61743858)(43.30306605,50.41431348)
\curveto(43.66244101,51.21118838)(43.84212849,52.10962577)(43.84212849,53.10962565)
\curveto(43.84212849,54.5835838)(43.44889937,55.72681283)(42.66244113,56.53931273)
\curveto(41.8759829,57.35181263)(40.63639972,57.75806258)(38.94369159,57.75806258)
\curveto(38.24577501,57.75806258)(37.57390009,57.70597925)(36.92806684,57.6018126)
\curveto(36.88640018,56.74764603)(36.86556685,55.78410449)(36.86556685,54.71118795)
\lineto(36.86556685,52.25806325)
\lineto(36.88119185,49.96118854)
\curveto(36.88119185,49.51327193)(36.89681684,48.9117095)(36.92806684,48.15650126)
\closepath
}
}
{
\newrgbcolor{curcolor}{0 0 0}
\pscustom[linestyle=none,fillstyle=solid,fillcolor=curcolor]
{
\newpath
\moveto(48.88900287,54.81275044)
\lineto(49.03744035,54.71118795)
\curveto(48.98014869,54.02889637)(48.95150286,53.16170898)(48.95150286,52.10962577)
\lineto(48.95150286,50.2893135)
\curveto(48.95150286,49.62264691)(49.00098202,49.1591053)(49.09994034,48.89868867)
\curveto(49.20410699,48.63827203)(49.38119031,48.43775123)(49.63119027,48.29712624)
\curveto(49.88119024,48.16170959)(50.17806521,48.09400127)(50.52181516,48.09400127)
\curveto(50.90202345,48.09400127)(51.24837757,48.16691793)(51.56087754,48.31275124)
\curveto(51.8733775,48.46379289)(52.13379413,48.67473036)(52.34212744,48.94556366)
\curveto(52.55566908,49.21639696)(52.6780649,49.39868861)(52.70931489,49.4924386)
\curveto(52.74056489,49.58618858)(52.76139822,49.84920939)(52.77181489,50.281501)
\lineto(52.79525238,51.15650089)
\lineto(52.79525238,52.00025079)
\curveto(52.79525238,52.20858409)(52.78483572,52.49243823)(52.76400239,52.85181318)
\curveto(52.74316906,53.21118814)(52.72493989,53.43254228)(52.70931489,53.5158756)
\curveto(52.69889823,53.59920892)(52.65983573,53.65910475)(52.59212741,53.69556308)
\curveto(52.52441908,53.73722974)(52.38900243,53.75806307)(52.18587746,53.75806307)
\lineto(51.52962754,53.76587557)
\lineto(51.45931505,53.82837556)
\lineto(51.45931505,54.16431302)
\lineto(51.52181504,54.22681301)
\curveto(52.51660659,54.34660466)(53.35514815,54.54191714)(54.03743973,54.81275044)
\lineto(54.18587721,54.71118795)
\curveto(54.12858555,54.02889637)(54.09993972,53.16170898)(54.09993972,52.10962577)
\lineto(54.09993972,50.73462594)
\curveto(54.09993972,50.65650095)(54.11296056,50.0184802)(54.13900222,48.82056368)
\curveto(54.14941888,48.3778554)(54.17546055,48.1070221)(54.21712721,48.00806378)
\curveto(54.2640022,47.91431379)(54.3265022,47.84660547)(54.40462719,47.8049388)
\curveto(54.48275218,47.76848048)(54.71452298,47.75025131)(55.0999396,47.75025131)
\lineto(55.31868957,47.75025131)
\lineto(55.38900206,47.68775132)
\lineto(55.38900206,47.37525136)
\lineto(55.32650207,47.30493887)
\curveto(54.59212716,47.34660553)(54.09993972,47.36743886)(53.84993975,47.36743886)
\curveto(53.53223146,47.36743886)(53.16504401,47.34920969)(52.74837739,47.31275136)
\lineto(52.6780649,47.37525136)
\curveto(52.70931489,47.90650129)(52.73275239,48.35702207)(52.74837739,48.72681369)
\curveto(52.42025243,48.47160539)(52.04525248,48.1304596)(51.62337753,47.70337632)
\curveto(51.46191922,47.541918)(51.23014841,47.40650135)(50.92806511,47.29712637)
\curveto(50.62598182,47.18775138)(50.28223186,47.13306389)(49.89681524,47.13306389)
\curveto(49.31348198,47.13306389)(48.85775287,47.22420971)(48.52962791,47.40650135)
\curveto(48.20671128,47.59400133)(47.97754464,47.84920963)(47.84212799,48.17212626)
\curveto(47.70671134,48.50025122)(47.63900302,49.06275115)(47.63900302,49.85962605)
\lineto(47.64681552,50.47681347)
\lineto(47.64681552,52.00025079)
\curveto(47.64681552,52.20858409)(47.63639885,52.49243823)(47.61556552,52.85181318)
\curveto(47.59473219,53.21118814)(47.57650303,53.43254228)(47.56087803,53.5158756)
\curveto(47.55046136,53.59920892)(47.51139887,53.65910475)(47.44369054,53.69556308)
\curveto(47.37598222,53.73722974)(47.24056557,53.75806307)(47.03744059,53.75806307)
\lineto(46.38119068,53.76587557)
\lineto(46.31087818,53.82837556)
\lineto(46.31087818,54.16431302)
\lineto(46.37337818,54.22681301)
\curveto(47.36816972,54.34660466)(48.20671128,54.54191714)(48.88900287,54.81275044)
\closepath
\moveto(50.75619014,55.30493788)
\closepath
\moveto(50.83431513,46.89868892)
\closepath
}
}
{
\newrgbcolor{curcolor}{0 0 0}
\pscustom[linestyle=none,fillstyle=solid,fillcolor=curcolor]
{
\newpath
\moveto(58.83431414,54.81275044)
\lineto(58.98275162,54.71118795)
\curveto(58.95150163,54.40910466)(58.92806413,53.87264639)(58.91243913,53.10181315)
\lineto(59.49837656,53.84400056)
\curveto(59.69108487,54.0887922)(59.86035568,54.27629217)(60.006189,54.40650049)
\curveto(60.15723064,54.54191714)(60.33170979,54.64608379)(60.52962643,54.71900045)
\curveto(60.72754307,54.79191711)(60.93066805,54.82837544)(61.13900136,54.82837544)
\curveto(61.36816799,54.82837544)(61.5843138,54.78150044)(61.78743878,54.68775046)
\lineto(61.84212627,54.57837547)
\curveto(61.75879295,53.88566722)(61.71191795,53.2762923)(61.70150129,52.75025069)
\lineto(61.34993883,52.75025069)
\curveto(61.14160552,53.2294173)(60.80566806,53.46900061)(60.34212645,53.46900061)
\curveto(60.01920983,53.46900061)(59.73795986,53.36483395)(59.49837656,53.15650064)
\curveto(59.25879325,52.95337567)(59.09733494,52.6955632)(59.01400162,52.38306324)
\curveto(58.93587663,52.07577161)(58.89681413,51.68514666)(58.89681413,51.21118838)
\lineto(58.89681413,50.34400099)
\curveto(58.89681413,50.18775101)(58.9072308,49.80493856)(58.92806413,49.19556363)
\curveto(58.94889746,48.58618871)(58.96973079,48.23462625)(58.99056412,48.14087626)
\curveto(59.01660578,48.04712627)(59.05306411,47.97681378)(59.09993911,47.92993879)
\curveto(59.15202243,47.88827213)(59.21712659,47.8596263)(59.29525158,47.8440013)
\curveto(59.37858491,47.8283763)(59.75098069,47.80233464)(60.41243894,47.76587631)
\lineto(60.48275144,47.70337632)
\lineto(60.48275144,47.37525136)
\lineto(60.41243894,47.30493887)
\curveto(59.7197307,47.34660553)(58.99577245,47.36743886)(58.24056421,47.36743886)
\curveto(57.64160595,47.36743886)(57.01660603,47.34660553)(56.36556444,47.30493887)
\lineto(56.29525195,47.37525136)
\lineto(56.29525195,47.70337632)
\lineto(56.36556444,47.76587631)
\curveto(56.87077271,47.7971263)(57.15462685,47.8205638)(57.21712684,47.8361888)
\curveto(57.28483516,47.8518138)(57.34212682,47.88306379)(57.38900182,47.92993879)
\curveto(57.44108514,47.98202212)(57.47493931,48.05754294)(57.49056431,48.15650126)
\curveto(57.51139764,48.26066791)(57.53223097,48.59660537)(57.5530643,49.16431364)
\curveto(57.57910596,49.73723023)(57.59212679,50.16952185)(57.59212679,50.46118848)
\lineto(57.59212679,52.00025079)
\curveto(57.59212679,52.20858409)(57.58171013,52.49243823)(57.5608768,52.85181318)
\curveto(57.54004347,53.21118814)(57.5218143,53.43254228)(57.5061893,53.5158756)
\curveto(57.49577264,53.59920892)(57.45671014,53.65910475)(57.38900182,53.69556308)
\curveto(57.32129349,53.73722974)(57.18587684,53.75806307)(56.98275187,53.75806307)
\lineto(56.32650195,53.76587557)
\lineto(56.25618946,53.82837556)
\lineto(56.25618946,54.16431302)
\lineto(56.31868945,54.22681301)
\curveto(57.31348099,54.34660466)(58.15202256,54.54191714)(58.83431414,54.81275044)
\closepath
}
}
{
\newrgbcolor{curcolor}{0 0 0}
\pscustom[linestyle=none,fillstyle=solid,fillcolor=curcolor]
{
\newpath
\moveto(68.93587539,48.20337625)
\lineto(68.62337543,47.67993882)
\curveto(67.95150052,47.3101472)(67.20150061,47.12525139)(66.37337571,47.12525139)
\curveto(65.22754252,47.12525139)(64.34212596,47.46639718)(63.71712604,48.14868876)
\curveto(63.09212611,48.83098034)(62.77962615,49.71118857)(62.77962615,50.78931344)
\curveto(62.77962615,51.27889671)(62.83170948,51.71379249)(62.93587613,52.09400078)
\curveto(63.04525112,52.47941739)(63.18066777,52.79712569)(63.34212608,53.04712566)
\curveto(63.50879273,53.29712563)(63.69108437,53.49504227)(63.88900102,53.64087558)
\curveto(64.08691766,53.7867089)(64.42025095,53.98983387)(64.88900089,54.25025051)
\curveto(65.36295917,54.51066714)(65.71972996,54.67993796)(65.95931326,54.75806295)
\curveto(66.19889656,54.84139627)(66.52962569,54.88306293)(66.95150064,54.88306293)
\curveto(67.75879221,54.88306293)(68.42025046,54.72941712)(68.93587539,54.42212549)
\curveto(68.83691707,53.83358389)(68.76139625,53.17733397)(68.70931292,52.45337573)
\lineto(68.63900043,52.39087574)
\lineto(68.31868797,52.39087574)
\lineto(68.24837548,52.46118823)
\curveto(68.22754215,52.89868818)(68.19889632,53.19295897)(68.16243799,53.34400062)
\curveto(68.12597966,53.49504227)(67.91764635,53.65650058)(67.53743807,53.82837556)
\curveto(67.16243811,54.00025054)(66.7613965,54.08618803)(66.33431321,54.08618803)
\curveto(65.90722993,54.08618803)(65.52702165,53.99504221)(65.19368836,53.81275056)
\curveto(64.86035506,53.63566725)(64.60775093,53.34139645)(64.43587595,52.92993817)
\curveto(64.2692093,52.52368822)(64.18587598,52.03670912)(64.18587598,51.46900085)
\curveto(64.18587598,50.98983424)(64.2535843,50.5210843)(64.38900095,50.06275103)
\curveto(64.5244176,49.60962608)(64.70150092,49.23202196)(64.92025089,48.92993867)
\curveto(65.14420919,48.6330637)(65.43847999,48.3934804)(65.80306328,48.21118875)
\curveto(66.16764657,48.02889711)(66.57389652,47.93775129)(67.02181313,47.93775129)
\curveto(67.3030631,47.93775129)(67.57650056,47.97420962)(67.84212553,48.04712627)
\curveto(68.11295883,48.12004293)(68.42285462,48.23983458)(68.77181291,48.40650123)
\closepath
}
}
{
\newrgbcolor{curcolor}{0 0 0}
\pscustom[linestyle=none,fillstyle=solid,fillcolor=curcolor]
{
\newpath
\moveto(71.91243753,58.93774993)
\lineto(72.06087501,58.84399994)
\curveto(72.00358335,58.24504168)(71.97493752,57.044521)(71.97493752,55.24243789)
\lineto(71.97493752,53.31275062)
\curveto(72.41243747,53.67212558)(72.81868742,54.02889637)(73.19368737,54.38306299)
\curveto(73.30306236,54.48202131)(73.40722901,54.55493797)(73.50618733,54.60181297)
\curveto(73.60514565,54.64868796)(73.7639998,54.69816712)(73.98274977,54.75025045)
\curveto(74.20670808,54.80233377)(74.43587472,54.82837544)(74.67024969,54.82837544)
\curveto(75.06608297,54.82837544)(75.44889542,54.74764628)(75.81868705,54.58618797)
\curveto(76.193687,54.42472965)(76.47493696,54.23202134)(76.66243694,54.00806304)
\curveto(76.85514525,53.78931307)(76.98795773,53.52889643)(77.06087439,53.22681314)
\curveto(77.13379105,52.92472984)(77.17024938,52.55233405)(77.17024938,52.10962577)
\lineto(77.17024938,50.73462594)
\curveto(77.17024938,50.64608429)(77.18327021,49.99764687)(77.20931187,48.78931368)
\curveto(77.22493687,48.29452208)(77.27962437,48.00025128)(77.37337435,47.90650129)
\curveto(77.47233267,47.8127513)(77.7952493,47.76587631)(78.34212423,47.76587631)
\lineto(78.40462423,47.70337632)
\lineto(78.40462423,47.36743886)
\lineto(78.34212423,47.30493887)
\curveto(77.68066598,47.34660553)(77.23274937,47.36743886)(76.9983744,47.36743886)
\curveto(76.87858275,47.36743886)(76.49837446,47.34660553)(75.85774954,47.30493887)
\lineto(75.76399955,47.39087636)
\curveto(75.83170788,48.0679596)(75.86556204,48.950772)(75.86556204,50.03931353)
\lineto(75.86556204,51.0627509)
\curveto(75.86556204,51.67733416)(75.84993704,52.12004244)(75.81868705,52.39087574)
\curveto(75.79264538,52.66170904)(75.70670789,52.90650067)(75.56087458,53.12525065)
\curveto(75.41504126,53.34920895)(75.21972879,53.52108393)(74.97493715,53.64087558)
\curveto(74.73535385,53.76587557)(74.44368721,53.82837556)(74.09993726,53.82837556)
\curveto(73.74577063,53.82837556)(73.44889567,53.7788964)(73.20931237,53.67993808)
\curveto(72.9749374,53.58618809)(72.73535409,53.42472978)(72.49056246,53.19556314)
\curveto(72.24577082,52.9663965)(72.09733334,52.77368819)(72.04525001,52.61743821)
\curveto(71.99837502,52.46639656)(71.97493752,52.17212577)(71.97493752,51.73462582)
\lineto(71.97493752,50.34400099)
\curveto(71.97493752,50.22941767)(71.98535418,49.85962605)(72.00618752,49.23462613)
\curveto(72.02702085,48.61483454)(72.04785418,48.25285542)(72.06868751,48.14868876)
\curveto(72.09472917,48.04452211)(72.1311875,47.96900128)(72.17806249,47.92212629)
\curveto(72.22493749,47.8752513)(72.28222915,47.8440013)(72.34993747,47.8283763)
\curveto(72.4176458,47.81795964)(72.69889576,47.7971263)(73.19368737,47.76587631)
\lineto(73.26399986,47.70337632)
\lineto(73.26399986,47.37525136)
\lineto(73.20149987,47.30493887)
\curveto(72.55045828,47.34660553)(71.92285419,47.36743886)(71.3186876,47.36743886)
\curveto(70.71972934,47.36743886)(70.09472942,47.34660553)(69.44368783,47.30493887)
\lineto(69.37337534,47.37525136)
\lineto(69.37337534,47.70337632)
\lineto(69.44368783,47.76587631)
\curveto(69.9488961,47.7971263)(70.23275023,47.8205638)(70.29525023,47.8361888)
\curveto(70.36295855,47.8518138)(70.42025021,47.88306379)(70.46712521,47.92993879)
\curveto(70.51920853,47.98202212)(70.55306269,48.05754294)(70.56868769,48.15650126)
\curveto(70.58952102,48.26066791)(70.61035435,48.59660537)(70.63118768,49.16431364)
\curveto(70.65722935,49.73723023)(70.67025018,50.16952185)(70.67025018,50.46118848)
\lineto(70.67025018,54.50806298)
\lineto(70.63900018,56.16431277)
\curveto(70.62337519,56.7424377)(70.60514602,57.14347932)(70.58431269,57.36743762)
\curveto(70.56868769,57.59139593)(70.5452502,57.72420841)(70.5140002,57.76587508)
\curveto(70.4827502,57.80754174)(70.42806271,57.83879173)(70.34993772,57.85962506)
\curveto(70.27181273,57.88045839)(69.9567086,57.89087506)(69.40462534,57.89087506)
\lineto(69.33431284,57.96118755)
\lineto(69.33431284,58.28931251)
\lineto(69.39681284,58.359625)
\curveto(70.39160438,58.47420832)(71.23014594,58.66691663)(71.91243753,58.93774993)
\closepath
}
}
{
\newrgbcolor{curcolor}{0 0 0}
\pscustom[linestyle=none,fillstyle=solid,fillcolor=curcolor]
{
\newpath
\moveto(83.06868615,47.76587631)
\lineto(83.13118614,47.70337632)
\lineto(83.13118614,47.37525136)
\lineto(83.06868615,47.30493887)
\curveto(82.90722784,47.30493887)(82.60254038,47.31535553)(82.15462376,47.33618886)
\curveto(81.74837381,47.35702219)(81.32910303,47.36743886)(80.89681142,47.36743886)
\curveto(80.33431149,47.36743886)(79.7067074,47.34660553)(79.01399915,47.30493887)
\lineto(78.95149916,47.37525136)
\lineto(78.95149916,47.70337632)
\lineto(79.01399915,47.76587631)
\curveto(79.51920742,47.7971263)(79.80566572,47.8205638)(79.87337405,47.8361888)
\curveto(79.94108237,47.8518138)(79.99837403,47.88306379)(80.04524902,47.92993879)
\curveto(80.09212402,47.98202212)(80.12337401,48.05754294)(80.13899901,48.15650126)
\curveto(80.15983234,48.26066791)(80.18066567,48.59660537)(80.201499,49.16431364)
\curveto(80.22754067,49.73723023)(80.2405615,50.16952185)(80.2405615,50.46118848)
\lineto(80.2405615,53.4768131)
\lineto(79.23274912,53.42212561)
\lineto(79.16243663,53.4846256)
\lineto(79.16243663,53.67993808)
\lineto(79.21712413,53.75806307)
\lineto(80.2405615,54.28150051)
\lineto(80.2405615,54.75025045)
\curveto(80.2405615,55.27108372)(80.263999,55.66431283)(80.31087399,55.9299378)
\curveto(80.35774899,56.2007711)(80.44368648,56.4481669)(80.56868646,56.67212521)
\curveto(80.69889478,56.90129185)(80.92285308,57.18775015)(81.24056138,57.5315001)
\curveto(81.55826967,57.88045839)(81.83691547,58.16431253)(82.07649877,58.3830625)
\curveto(82.31608208,58.60702081)(82.52441539,58.75545829)(82.7014987,58.82837494)
\curveto(82.87858201,58.90649994)(83.08691532,58.94556243)(83.32649862,58.94556243)
\curveto(83.52441526,58.94556243)(83.7379569,58.90649994)(83.96712354,58.82837494)
\lineto(83.95931104,57.5471251)
\lineto(83.77962356,57.47681261)
\curveto(83.48795693,57.73722925)(83.14941531,57.86743756)(82.76399869,57.86743756)
\curveto(82.49316539,57.86743756)(82.25879042,57.8101459)(82.06087378,57.69556258)
\curveto(81.86295713,57.5861876)(81.72754048,57.40389595)(81.65462383,57.14868765)
\curveto(81.58170717,56.89868768)(81.54524884,56.48202107)(81.54524884,55.89868781)
\lineto(81.54524884,54.28150051)
\lineto(82.38118624,54.28150051)
\curveto(82.81347785,54.28150051)(83.22233197,54.2971255)(83.60774858,54.3283755)
\lineto(83.67806108,54.24243801)
\lineto(83.53743609,53.55493809)
\lineto(83.4671236,53.4768131)
\curveto(83.31087362,53.48202144)(83.20931113,53.4846256)(83.16243614,53.4846256)
\lineto(82.14681127,53.4924381)
\lineto(81.54524884,53.4924381)
\lineto(81.54524884,50.34400099)
\curveto(81.54524884,50.17733434)(81.5556655,49.79452189)(81.57649884,49.19556363)
\curveto(81.6025405,48.60181371)(81.625978,48.25285542)(81.64681133,48.14868876)
\curveto(81.66764466,48.04452211)(81.70149882,47.96900128)(81.74837381,47.92212629)
\curveto(81.80045714,47.8752513)(81.8811863,47.84139713)(81.99056128,47.8205638)
\curveto(82.1051446,47.79973047)(82.46451956,47.78150131)(83.06868615,47.76587631)
\closepath
}
}
{
\newrgbcolor{curcolor}{0 0 0}
\pscustom[linestyle=none,fillstyle=solid,fillcolor=curcolor]
{
\newpath
\moveto(86.59993572,54.81275044)
\lineto(86.7483732,54.71118795)
\curveto(86.69108154,54.02889637)(86.66243571,53.16170898)(86.66243571,52.10962577)
\lineto(86.66243571,50.2893135)
\curveto(86.66243571,49.62264691)(86.71191487,49.1591053)(86.81087319,48.89868867)
\curveto(86.91503984,48.63827203)(87.09212316,48.43775123)(87.34212312,48.29712624)
\curveto(87.59212309,48.16170959)(87.88899806,48.09400127)(88.23274801,48.09400127)
\curveto(88.6129563,48.09400127)(88.95931043,48.16691793)(89.27181039,48.31275124)
\curveto(89.58431035,48.46379289)(89.84472698,48.67473036)(90.05306029,48.94556366)
\curveto(90.26660193,49.21639696)(90.38899775,49.39868861)(90.42024774,49.4924386)
\curveto(90.45149774,49.58618858)(90.47233107,49.84920939)(90.48274774,50.281501)
\lineto(90.50618523,51.15650089)
\lineto(90.50618523,52.00025079)
\curveto(90.50618523,52.20858409)(90.49576857,52.49243823)(90.47493524,52.85181318)
\curveto(90.45410191,53.21118814)(90.43587274,53.43254228)(90.42024774,53.5158756)
\curveto(90.40983108,53.59920892)(90.37076858,53.65910475)(90.30306026,53.69556308)
\curveto(90.23535193,53.73722974)(90.09993528,53.75806307)(89.89681031,53.75806307)
\lineto(89.24056039,53.76587557)
\lineto(89.1702479,53.82837556)
\lineto(89.1702479,54.16431302)
\lineto(89.23274789,54.22681301)
\curveto(90.22753944,54.34660466)(91.066081,54.54191714)(91.74837258,54.81275044)
\lineto(91.89681006,54.71118795)
\curveto(91.8395184,54.02889637)(91.81087257,53.16170898)(91.81087257,52.10962577)
\lineto(91.81087257,50.73462594)
\curveto(91.81087257,50.65650095)(91.82389341,50.0184802)(91.84993507,48.82056368)
\curveto(91.86035173,48.3778554)(91.8863934,48.1070221)(91.92806006,48.00806378)
\curveto(91.97493505,47.91431379)(92.03743505,47.84660547)(92.11556004,47.8049388)
\curveto(92.19368503,47.76848048)(92.42545583,47.75025131)(92.81087245,47.75025131)
\lineto(93.02962242,47.75025131)
\lineto(93.09993491,47.68775132)
\lineto(93.09993491,47.37525136)
\lineto(93.03743492,47.30493887)
\curveto(92.30306001,47.34660553)(91.81087257,47.36743886)(91.5608726,47.36743886)
\curveto(91.24316431,47.36743886)(90.87597686,47.34920969)(90.45931024,47.31275136)
\lineto(90.38899775,47.37525136)
\curveto(90.42024774,47.90650129)(90.44368524,48.35702207)(90.45931024,48.72681369)
\curveto(90.13118528,48.47160539)(89.75618533,48.1304596)(89.33431038,47.70337632)
\curveto(89.17285207,47.541918)(88.94108126,47.40650135)(88.63899796,47.29712637)
\curveto(88.33691467,47.18775138)(87.99316471,47.13306389)(87.60774809,47.13306389)
\curveto(87.02441483,47.13306389)(86.56868572,47.22420971)(86.24056076,47.40650135)
\curveto(85.91764413,47.59400133)(85.68847749,47.84920963)(85.55306085,48.17212626)
\curveto(85.4176442,48.50025122)(85.34993587,49.06275115)(85.34993587,49.85962605)
\lineto(85.35774837,50.47681347)
\lineto(85.35774837,52.00025079)
\curveto(85.35774837,52.20858409)(85.3473317,52.49243823)(85.32649837,52.85181318)
\curveto(85.30566504,53.21118814)(85.28743588,53.43254228)(85.27181088,53.5158756)
\curveto(85.26139421,53.59920892)(85.22233172,53.65910475)(85.15462339,53.69556308)
\curveto(85.08691507,53.73722974)(84.95149842,53.75806307)(84.74837344,53.75806307)
\lineto(84.09212353,53.76587557)
\lineto(84.02181103,53.82837556)
\lineto(84.02181103,54.16431302)
\lineto(84.08431103,54.22681301)
\curveto(85.07910257,54.34660466)(85.91764413,54.54191714)(86.59993572,54.81275044)
\closepath
\moveto(88.46712299,55.30493788)
\closepath
\moveto(88.54524798,46.89868892)
\closepath
\moveto(90.08431029,56.45337524)
\curveto(89.84993532,56.45337524)(89.64941451,56.53670856)(89.48274786,56.70337521)
\curveto(89.31608121,56.87004185)(89.23274789,57.07056266)(89.23274789,57.30493763)
\curveto(89.23274789,57.54452094)(89.31347705,57.74504174)(89.47493536,57.90650006)
\curveto(89.64160201,58.0731667)(89.84472698,58.15650003)(90.08431029,58.15650003)
\curveto(90.32389359,58.15650003)(90.5244144,58.0731667)(90.68587271,57.90650006)
\curveto(90.85253936,57.74504174)(90.93587268,57.54452094)(90.93587268,57.30493763)
\curveto(90.93587268,57.07056266)(90.85253936,56.87004185)(90.68587271,56.70337521)
\curveto(90.51920607,56.53670856)(90.31868526,56.45337524)(90.08431029,56.45337524)
\closepath
\moveto(86.77962319,56.45337524)
\curveto(86.54524822,56.45337524)(86.34472741,56.53670856)(86.17806077,56.70337521)
\curveto(86.01139412,56.87004185)(85.9280608,57.07056266)(85.9280608,57.30493763)
\curveto(85.9280608,57.5393126)(86.01139412,57.73983341)(86.17806077,57.90650006)
\curveto(86.34472741,58.0731667)(86.54524822,58.15650003)(86.77962319,58.15650003)
\curveto(87.0192065,58.15650003)(87.21972731,58.0731667)(87.38118562,57.90650006)
\curveto(87.54785227,57.74504174)(87.63118559,57.54452094)(87.63118559,57.30493763)
\curveto(87.63118559,57.07056266)(87.54785227,56.87004185)(87.38118562,56.70337521)
\curveto(87.21451897,56.53670856)(87.01399816,56.45337524)(86.77962319,56.45337524)
\closepath
}
}
{
\newrgbcolor{curcolor}{0 0 0}
\pscustom[linestyle=none,fillstyle=solid,fillcolor=curcolor]
{
\newpath
\moveto(96.20149703,58.93774993)
\lineto(96.34993451,58.84399994)
\curveto(96.29264285,58.24504168)(96.26399702,57.044521)(96.26399702,55.24243789)
\lineto(96.26399702,53.31275062)
\curveto(96.70149697,53.67212558)(97.10774692,54.02889637)(97.48274687,54.38306299)
\curveto(97.59212186,54.48202131)(97.69628851,54.55493797)(97.79524684,54.60181297)
\curveto(97.89420516,54.64868796)(98.0530593,54.69816712)(98.27180928,54.75025045)
\curveto(98.49576758,54.80233377)(98.72493422,54.82837544)(98.95930919,54.82837544)
\curveto(99.35514248,54.82837544)(99.73795493,54.74764628)(100.10774655,54.58618797)
\curveto(100.4827465,54.42472965)(100.76399647,54.23202134)(100.95149645,54.00806304)
\curveto(101.14420476,53.78931307)(101.27701724,53.52889643)(101.3499339,53.22681314)
\curveto(101.42285055,52.92472984)(101.45930888,52.55233405)(101.45930888,52.10962577)
\lineto(101.45930888,50.73462594)
\curveto(101.45930888,50.64608429)(101.47232972,49.99764687)(101.49837138,48.78931368)
\curveto(101.51399638,48.29452208)(101.56868387,48.00025128)(101.66243386,47.90650129)
\curveto(101.76139218,47.8127513)(102.08430881,47.76587631)(102.63118374,47.76587631)
\lineto(102.69368373,47.70337632)
\lineto(102.69368373,47.36743886)
\lineto(102.63118374,47.30493887)
\curveto(101.96972549,47.34660553)(101.52180888,47.36743886)(101.2874339,47.36743886)
\curveto(101.16764225,47.36743886)(100.78743397,47.34660553)(100.14680905,47.30493887)
\lineto(100.05305906,47.39087636)
\curveto(100.12076738,48.0679596)(100.15462154,48.950772)(100.15462154,50.03931353)
\lineto(100.15462154,51.0627509)
\curveto(100.15462154,51.67733416)(100.13899655,52.12004244)(100.10774655,52.39087574)
\curveto(100.08170489,52.66170904)(99.9957674,52.90650067)(99.84993408,53.12525065)
\curveto(99.70410077,53.34920895)(99.50878829,53.52108393)(99.26399665,53.64087558)
\curveto(99.02441335,53.76587557)(98.73274672,53.82837556)(98.38899676,53.82837556)
\curveto(98.03483014,53.82837556)(97.73795518,53.7788964)(97.49837187,53.67993808)
\curveto(97.2639969,53.58618809)(97.0244136,53.42472978)(96.77962196,53.19556314)
\curveto(96.53483032,52.9663965)(96.38639284,52.77368819)(96.33430952,52.61743821)
\curveto(96.28743452,52.46639656)(96.26399702,52.17212577)(96.26399702,51.73462582)
\lineto(96.26399702,50.34400099)
\curveto(96.26399702,50.22941767)(96.27441369,49.85962605)(96.29524702,49.23462613)
\curveto(96.31608035,48.61483454)(96.33691368,48.25285542)(96.35774701,48.14868876)
\curveto(96.38378868,48.04452211)(96.42024701,47.96900128)(96.467122,47.92212629)
\curveto(96.51399699,47.8752513)(96.57128865,47.8440013)(96.63899698,47.8283763)
\curveto(96.7067053,47.81795964)(96.98795527,47.7971263)(97.48274687,47.76587631)
\lineto(97.55305937,47.70337632)
\lineto(97.55305937,47.37525136)
\lineto(97.49055937,47.30493887)
\curveto(96.83951779,47.34660553)(96.2119137,47.36743886)(95.60774711,47.36743886)
\curveto(95.00878885,47.36743886)(94.38378892,47.34660553)(93.73274734,47.30493887)
\lineto(93.66243485,47.37525136)
\lineto(93.66243485,47.70337632)
\lineto(93.73274734,47.76587631)
\curveto(94.23795561,47.7971263)(94.52180974,47.8205638)(94.58430973,47.8361888)
\curveto(94.65201806,47.8518138)(94.70930972,47.88306379)(94.75618471,47.92993879)
\curveto(94.80826804,47.98202212)(94.8421222,48.05754294)(94.8577472,48.15650126)
\curveto(94.87858053,48.26066791)(94.89941386,48.59660537)(94.92024719,49.16431364)
\curveto(94.94628885,49.73723023)(94.95930969,50.16952185)(94.95930969,50.46118848)
\lineto(94.95930969,54.50806298)
\lineto(94.92805969,56.16431277)
\curveto(94.91243469,56.7424377)(94.89420553,57.14347932)(94.8733722,57.36743762)
\curveto(94.8577472,57.59139593)(94.8343097,57.72420841)(94.8030597,57.76587508)
\curveto(94.77180971,57.80754174)(94.71712222,57.83879173)(94.63899722,57.85962506)
\curveto(94.56087223,57.88045839)(94.24576811,57.89087506)(93.69368484,57.89087506)
\lineto(93.62337235,57.96118755)
\lineto(93.62337235,58.28931251)
\lineto(93.68587234,58.359625)
\curveto(94.68066389,58.47420832)(95.51920545,58.66691663)(96.20149703,58.93774993)
\closepath
}
}
{
\newrgbcolor{curcolor}{0 0 0}
\pscustom[linestyle=none,fillstyle=solid,fillcolor=curcolor]
{
\newpath
\moveto(105.85774584,54.81275044)
\lineto(106.00618332,54.71118795)
\curveto(105.97493333,54.40910466)(105.95149583,53.87264639)(105.93587083,53.10181315)
\lineto(106.52180826,53.84400056)
\curveto(106.71451657,54.0887922)(106.88378738,54.27629217)(107.0296207,54.40650049)
\curveto(107.18066234,54.54191714)(107.35514149,54.64608379)(107.55305813,54.71900045)
\curveto(107.75097477,54.79191711)(107.95409975,54.82837544)(108.16243306,54.82837544)
\curveto(108.3915997,54.82837544)(108.6077455,54.78150044)(108.81087048,54.68775046)
\lineto(108.86555797,54.57837547)
\curveto(108.78222465,53.88566722)(108.73534965,53.2762923)(108.72493299,52.75025069)
\lineto(108.37337053,52.75025069)
\curveto(108.16503722,53.2294173)(107.82909976,53.46900061)(107.36555816,53.46900061)
\curveto(107.04264153,53.46900061)(106.76139156,53.36483395)(106.52180826,53.15650064)
\curveto(106.28222496,52.95337567)(106.12076664,52.6955632)(106.03743332,52.38306324)
\curveto(105.95930833,52.07577161)(105.92024583,51.68514666)(105.92024583,51.21118838)
\lineto(105.92024583,50.34400099)
\curveto(105.92024583,50.18775101)(105.9306625,49.80493856)(105.95149583,49.19556363)
\curveto(105.97232916,48.58618871)(105.99316249,48.23462625)(106.01399582,48.14087626)
\curveto(106.04003749,48.04712627)(106.07649581,47.97681378)(106.12337081,47.92993879)
\curveto(106.17545414,47.88827213)(106.24055829,47.8596263)(106.31868328,47.8440013)
\curveto(106.40201661,47.8283763)(106.7744124,47.80233464)(107.43587065,47.76587631)
\lineto(107.50618314,47.70337632)
\lineto(107.50618314,47.37525136)
\lineto(107.43587065,47.30493887)
\curveto(106.7431624,47.34660553)(106.01920415,47.36743886)(105.26399591,47.36743886)
\curveto(104.66503766,47.36743886)(104.04003773,47.34660553)(103.38899615,47.30493887)
\lineto(103.31868365,47.37525136)
\lineto(103.31868365,47.70337632)
\lineto(103.38899615,47.76587631)
\curveto(103.89420442,47.7971263)(104.17805855,47.8205638)(104.24055854,47.8361888)
\curveto(104.30826687,47.8518138)(104.36555853,47.88306379)(104.41243352,47.92993879)
\curveto(104.46451685,47.98202212)(104.49837101,48.05754294)(104.51399601,48.15650126)
\curveto(104.53482934,48.26066791)(104.55566267,48.59660537)(104.576496,49.16431364)
\curveto(104.60253766,49.73723023)(104.61555849,50.16952185)(104.61555849,50.46118848)
\lineto(104.61555849,52.00025079)
\curveto(104.61555849,52.20858409)(104.60514183,52.49243823)(104.5843085,52.85181318)
\curveto(104.56347517,53.21118814)(104.545246,53.43254228)(104.52962101,53.5158756)
\curveto(104.51920434,53.59920892)(104.48014184,53.65910475)(104.41243352,53.69556308)
\curveto(104.34472519,53.73722974)(104.20930854,53.75806307)(104.00618357,53.75806307)
\lineto(103.34993365,53.76587557)
\lineto(103.27962116,53.82837556)
\lineto(103.27962116,54.16431302)
\lineto(103.34212115,54.22681301)
\curveto(104.3369127,54.34660466)(105.17545426,54.54191714)(105.85774584,54.81275044)
\closepath
}
}
{
\newrgbcolor{curcolor}{0 0 0}
\pscustom[linestyle=none,fillstyle=solid,fillcolor=curcolor]
{
\newpath
\moveto(111.8811826,54.81275044)
\lineto(112.02962008,54.71118795)
\curveto(111.97232842,54.02889637)(111.94368259,53.16170898)(111.94368259,52.10962577)
\lineto(111.94368259,50.2893135)
\curveto(111.94368259,49.62264691)(111.99316175,49.1591053)(112.09212007,48.89868867)
\curveto(112.19628673,48.63827203)(112.37337004,48.43775123)(112.62337001,48.29712624)
\curveto(112.87336998,48.16170959)(113.17024494,48.09400127)(113.5139949,48.09400127)
\curveto(113.89420318,48.09400127)(114.24055731,48.16691793)(114.55305727,48.31275124)
\curveto(114.86555723,48.46379289)(115.12597387,48.67473036)(115.33430717,48.94556366)
\curveto(115.54784881,49.21639696)(115.67024463,49.39868861)(115.70149463,49.4924386)
\curveto(115.73274462,49.58618858)(115.75357795,49.84920939)(115.76399462,50.281501)
\lineto(115.78743212,51.15650089)
\lineto(115.78743212,52.00025079)
\curveto(115.78743212,52.20858409)(115.77701545,52.49243823)(115.75618212,52.85181318)
\curveto(115.73534879,53.21118814)(115.71711963,53.43254228)(115.70149463,53.5158756)
\curveto(115.69107796,53.59920892)(115.65201547,53.65910475)(115.58430714,53.69556308)
\curveto(115.51659882,53.73722974)(115.38118217,53.75806307)(115.17805719,53.75806307)
\lineto(114.52180727,53.76587557)
\lineto(114.45149478,53.82837556)
\lineto(114.45149478,54.16431302)
\lineto(114.51399477,54.22681301)
\curveto(115.50878632,54.34660466)(116.34732788,54.54191714)(117.02961946,54.81275044)
\lineto(117.17805695,54.71118795)
\curveto(117.12076529,54.02889637)(117.09211946,53.16170898)(117.09211946,52.10962577)
\lineto(117.09211946,50.73462594)
\curveto(117.09211946,50.65650095)(117.10514029,50.0184802)(117.13118195,48.82056368)
\curveto(117.14159862,48.3778554)(117.16764028,48.1070221)(117.20930694,48.00806378)
\curveto(117.25618194,47.91431379)(117.31868193,47.84660547)(117.39680692,47.8049388)
\curveto(117.47493191,47.76848048)(117.70670271,47.75025131)(118.09211933,47.75025131)
\lineto(118.31086931,47.75025131)
\lineto(118.3811818,47.68775132)
\lineto(118.3811818,47.37525136)
\lineto(118.3186818,47.30493887)
\curveto(117.5843069,47.34660553)(117.09211946,47.36743886)(116.84211949,47.36743886)
\curveto(116.52441119,47.36743886)(116.15722374,47.34920969)(115.74055712,47.31275136)
\lineto(115.67024463,47.37525136)
\curveto(115.70149463,47.90650129)(115.72493212,48.35702207)(115.74055712,48.72681369)
\curveto(115.41243216,48.47160539)(115.03743221,48.1304596)(114.61555726,47.70337632)
\curveto(114.45409895,47.541918)(114.22232814,47.40650135)(113.92024485,47.29712637)
\curveto(113.61816155,47.18775138)(113.27441159,47.13306389)(112.88899497,47.13306389)
\curveto(112.30566171,47.13306389)(111.8499326,47.22420971)(111.52180764,47.40650135)
\curveto(111.19889102,47.59400133)(110.96972438,47.84920963)(110.83430773,48.17212626)
\curveto(110.69889108,48.50025122)(110.63118275,49.06275115)(110.63118275,49.85962605)
\lineto(110.63899525,50.47681347)
\lineto(110.63899525,52.00025079)
\curveto(110.63899525,52.20858409)(110.62857859,52.49243823)(110.60774526,52.85181318)
\curveto(110.58691192,53.21118814)(110.56868276,53.43254228)(110.55305776,53.5158756)
\curveto(110.5426411,53.59920892)(110.5035786,53.65910475)(110.43587028,53.69556308)
\curveto(110.36816195,53.73722974)(110.2327453,53.75806307)(110.02962033,53.75806307)
\lineto(109.37337041,53.76587557)
\lineto(109.30305792,53.82837556)
\lineto(109.30305792,54.16431302)
\lineto(109.36555791,54.22681301)
\curveto(110.36034945,54.34660466)(111.19889102,54.54191714)(111.8811826,54.81275044)
\closepath
\moveto(113.74836987,55.30493788)
\closepath
\moveto(113.82649486,46.89868892)
\closepath
}
}
{
\newrgbcolor{curcolor}{0 0 0}
\pscustom[linestyle=none,fillstyle=solid,fillcolor=curcolor]
{
\newpath
\moveto(121.49836891,54.81275044)
\lineto(121.64680639,54.71118795)
\curveto(121.6155564,54.36222966)(121.59472307,53.90389639)(121.5843064,53.33618812)
\curveto(122.00097302,53.67993808)(122.3968063,54.02889637)(122.77180626,54.38306299)
\curveto(122.88118124,54.48202131)(122.98274373,54.55493797)(123.07649372,54.60181297)
\curveto(123.17545204,54.64868796)(123.33691035,54.69816712)(123.56086866,54.75025045)
\curveto(123.78482696,54.80233377)(124.0139936,54.82837544)(124.24836857,54.82837544)
\curveto(124.64420186,54.82837544)(125.02701431,54.74764628)(125.39680593,54.58618797)
\curveto(125.77180589,54.42472965)(126.05305585,54.23202134)(126.24055583,54.00806304)
\curveto(126.43326414,53.78931307)(126.56607662,53.52889643)(126.63899328,53.22681314)
\curveto(126.71190994,52.92472984)(126.74836827,52.55233405)(126.74836827,52.10962577)
\lineto(126.74836827,50.73462594)
\curveto(126.74836827,50.64608429)(126.7613891,49.99764687)(126.78743076,48.78931368)
\curveto(126.79784743,48.29452208)(126.85253492,48.00025128)(126.95149324,47.90650129)
\curveto(127.05045156,47.8127513)(127.37336819,47.76587631)(127.92024312,47.76587631)
\lineto(127.98274311,47.70337632)
\lineto(127.98274311,47.36743886)
\lineto(127.92024312,47.30493887)
\curveto(127.25878487,47.34660553)(126.81086826,47.36743886)(126.57649329,47.36743886)
\curveto(126.44107664,47.36743886)(126.06086835,47.34660553)(125.43586843,47.30493887)
\lineto(125.34211844,47.39087636)
\curveto(125.40982676,48.0679596)(125.44368093,48.950772)(125.44368093,50.03931353)
\lineto(125.44368093,51.0627509)
\curveto(125.44368093,51.67733416)(125.42805593,52.12004244)(125.39680593,52.39087574)
\curveto(125.37076427,52.66170904)(125.28482678,52.90650067)(125.13899346,53.12525065)
\curveto(124.99316015,53.34920895)(124.79784767,53.52108393)(124.55305604,53.64087558)
\curveto(124.3082644,53.76587557)(124.01659777,53.82837556)(123.67805614,53.82837556)
\curveto(123.40722284,53.82837556)(123.17545204,53.79972973)(122.98274373,53.74243807)
\curveto(122.79003542,53.69035474)(122.57388961,53.57837559)(122.33430631,53.40650061)
\curveto(122.09472301,53.23983397)(121.91763969,53.08097982)(121.80305638,52.92993817)
\curveto(121.68847306,52.78410486)(121.61816056,52.64608404)(121.5921189,52.51587572)
\curveto(121.57128557,52.39087574)(121.56086891,52.12004244)(121.56086891,51.70337582)
\lineto(121.56086891,50.34400099)
\curveto(121.56086891,50.22941767)(121.57128557,49.85962605)(121.5921189,49.23462613)
\curveto(121.61295223,48.61483454)(121.63378556,48.25285542)(121.65461889,48.14868876)
\curveto(121.68066056,48.04452211)(121.71711889,47.96900128)(121.76399388,47.92212629)
\curveto(121.81086887,47.8752513)(121.86816053,47.8440013)(121.93586886,47.8283763)
\curveto(122.00357718,47.81795964)(122.28482715,47.7971263)(122.77961875,47.76587631)
\lineto(122.84993125,47.70337632)
\lineto(122.84993125,47.37525136)
\lineto(122.78743125,47.30493887)
\curveto(122.13638967,47.34660553)(121.50878558,47.36743886)(120.90461899,47.36743886)
\curveto(120.30566073,47.36743886)(119.6806608,47.34660553)(119.02961922,47.30493887)
\lineto(118.95930673,47.37525136)
\lineto(118.95930673,47.70337632)
\lineto(119.02961922,47.76587631)
\curveto(119.53482749,47.7971263)(119.81868162,47.8205638)(119.88118161,47.8361888)
\curveto(119.94888994,47.8518138)(120.0061816,47.88306379)(120.05305659,47.92993879)
\curveto(120.10513992,47.98202212)(120.13899408,48.05754294)(120.15461908,48.15650126)
\curveto(120.17545241,48.26066791)(120.19628574,48.59660537)(120.21711907,49.16431364)
\curveto(120.24316073,49.73723023)(120.25618157,50.16952185)(120.25618157,50.46118848)
\lineto(120.25618157,52.00025079)
\curveto(120.25618157,52.20858409)(120.2457649,52.49243823)(120.22493157,52.85181318)
\curveto(120.20409824,53.21118814)(120.18586907,53.43254228)(120.17024408,53.5158756)
\curveto(120.15982741,53.59920892)(120.12076492,53.65910475)(120.05305659,53.69556308)
\curveto(119.98534827,53.73722974)(119.84993162,53.75806307)(119.64680664,53.75806307)
\lineto(118.99055672,53.76587557)
\lineto(118.92024423,53.82837556)
\lineto(118.92024423,54.16431302)
\lineto(118.98274422,54.22681301)
\curveto(119.97753577,54.34660466)(120.81607733,54.54191714)(121.49836891,54.81275044)
\closepath
}
}
{
\newrgbcolor{curcolor}{0 0 0}
\pscustom[linestyle=none,fillstyle=solid,fillcolor=curcolor]
{
\newpath
\moveto(136.75617953,54.27368801)
\lineto(136.81867952,54.13306302)
\curveto(136.60513788,53.80493806)(136.4723254,53.59139642)(136.42024207,53.4924381)
\lineto(134.96711725,53.4924381)
\curveto(135.11295057,53.2059798)(135.18586723,52.90650067)(135.18586723,52.59400071)
\curveto(135.18586723,52.22420909)(135.09992974,51.86483414)(134.92805476,51.51587585)
\curveto(134.76138811,51.16691756)(134.52701314,50.87004259)(134.22492984,50.62525096)
\curveto(133.92284655,50.38045932)(133.57909659,50.18514684)(133.19367997,50.03931353)
\curveto(132.81347168,49.89348021)(132.3785759,49.82056356)(131.88899263,49.82056356)
\lineto(131.52961768,49.82056356)
\curveto(131.22232605,49.57577192)(131.0244094,49.39348027)(130.93586775,49.27368862)
\curveto(130.84732609,49.15389697)(130.80305527,49.03150115)(130.80305527,48.90650117)
\curveto(130.80305527,48.67733453)(130.90722192,48.51587622)(131.11555523,48.42212623)
\curveto(131.32909687,48.32837624)(131.76659681,48.28150125)(132.42805507,48.28150125)
\lineto(134.17805485,48.30493874)
\curveto(134.73013811,48.30493874)(135.15201306,48.23983458)(135.44367969,48.10962627)
\curveto(135.74055466,47.97941795)(135.98013796,47.74504298)(136.1624296,47.40650135)
\curveto(136.34472125,47.07316806)(136.43586707,46.71900144)(136.43586707,46.34400148)
\curveto(136.43586707,45.77108489)(136.24836709,45.20077246)(135.87336714,44.6330642)
\curveto(135.50357552,44.0601476)(134.96711725,43.62525182)(134.26399234,43.32837686)
\curveto(133.56086743,43.03150189)(132.81086752,42.88306441)(132.01399262,42.88306441)
\curveto(131.56086767,42.88306441)(131.13378439,42.93254357)(130.73274277,43.03150189)
\curveto(130.33690949,43.13046021)(129.9879512,43.28150186)(129.6858679,43.48462684)
\curveto(129.38899294,43.68254348)(129.14940964,43.94296011)(128.96711799,44.26587674)
\curveto(128.78482635,44.58879337)(128.69368053,44.92473083)(128.69368053,45.27368912)
\curveto(128.69368053,45.46639743)(128.72232636,45.6695224)(128.77961802,45.88306404)
\curveto(128.83690967,46.09139735)(128.94628466,46.31795982)(129.10774297,46.56275146)
\lineto(130.5218053,47.35181386)
\curveto(130.06868036,47.49243884)(129.78222206,47.63045966)(129.66243041,47.76587631)
\curveto(129.54784709,47.90650129)(129.49055543,48.07316794)(129.49055543,48.26587625)
\curveto(129.49055543,48.46379289)(129.55565959,48.69816786)(129.6858679,48.96900116)
\lineto(130.94368025,49.87525105)
\curveto(130.20409701,50.06795936)(129.70409707,50.35702182)(129.44368043,50.74243844)
\curveto(129.18847213,51.12785506)(129.06086798,51.54973001)(129.06086798,52.00806329)
\curveto(129.06086798,52.42993823)(129.1520138,52.83097985)(129.33430545,53.21118814)
\curveto(129.51659709,53.59660476)(129.79003456,53.90389639)(130.15461785,54.13306302)
\curveto(130.52440947,54.36222966)(130.93586775,54.54191714)(131.38899269,54.67212546)
\curveto(131.84732597,54.80754211)(132.26399259,54.87525043)(132.63899254,54.87525043)
\curveto(133.29003413,54.87525043)(133.90982572,54.66431296)(134.49836731,54.24243801)
\curveto(135.44107553,54.24243801)(136.1936796,54.25285468)(136.75617953,54.27368801)
\closepath
\moveto(130.40461781,52.46118823)
\curveto(130.40461781,52.14347994)(130.46711781,51.80233414)(130.59211779,51.43775086)
\curveto(130.71711778,51.07316757)(130.91763858,50.79712594)(131.19368022,50.60962596)
\curveto(131.47493018,50.42212598)(131.79003431,50.32837599)(132.1389926,50.32837599)
\curveto(132.60253421,50.32837599)(133.00097166,50.47941764)(133.33430495,50.78150094)
\curveto(133.67284658,51.08358423)(133.84211739,51.55493834)(133.84211739,52.19556326)
\curveto(133.84211739,52.73722986)(133.68847158,53.22420897)(133.38117995,53.65650058)
\curveto(133.07388832,54.0887922)(132.63899254,54.304938)(132.07649261,54.304938)
\curveto(131.60253433,54.304938)(131.20409688,54.14868802)(130.88118026,53.83618806)
\curveto(130.56347196,53.52889643)(130.40461781,53.07056315)(130.40461781,52.46118823)
\closepath
\moveto(132.69368003,47.21900138)
\curveto(131.85513847,47.21900138)(131.37076353,47.20858471)(131.24055521,47.18775138)
\curveto(131.11555523,47.16691805)(130.93586775,47.07837639)(130.70149278,46.92212641)
\curveto(130.46711781,46.76587643)(130.2744095,46.55493896)(130.12336785,46.28931399)
\curveto(129.97753453,46.01848069)(129.90461788,45.71639739)(129.90461788,45.3830641)
\curveto(129.90461788,44.79452251)(130.12336785,44.3153559)(130.5608678,43.94556428)
\curveto(130.99836774,43.57577266)(131.581701,43.39087685)(132.31086758,43.39087685)
\curveto(132.85774251,43.39087685)(133.36295078,43.50546017)(133.82649239,43.73462681)
\curveto(134.290034,43.96379344)(134.63117979,44.27108507)(134.84992977,44.65650169)
\curveto(135.07388807,45.04191831)(135.18586723,45.41170993)(135.18586723,45.76587656)
\curveto(135.18586723,46.12004318)(135.09211724,46.42473064)(134.90461726,46.67993894)
\curveto(134.72232562,46.92993891)(134.48274231,47.08358473)(134.18586735,47.14087639)
\curveto(133.89420072,47.19295971)(133.39680495,47.21900138)(132.69368003,47.21900138)
\closepath
}
}
{
\newrgbcolor{curcolor}{0 0 0}
\pscustom[linestyle=none,fillstyle=solid,fillcolor=curcolor]
{
}
}
{
\newrgbcolor{curcolor}{0 0 0}
\pscustom[linestyle=none,fillstyle=solid,fillcolor=curcolor]
{
\newpath
\moveto(146.29524086,57.96118755)
\lineto(146.29524086,58.28931251)
\lineto(146.35774085,58.359625)
\curveto(147.35253239,58.47420832)(148.19107395,58.66691663)(148.87336554,58.93774993)
\lineto(149.02180302,58.84399994)
\curveto(148.96451136,58.24504168)(148.93586553,57.044521)(148.93586553,55.24243789)
\lineto(148.93586553,50.12525102)
\curveto(148.93586553,49.59920942)(148.94367803,49.13827197)(148.95930303,48.74243869)
\curveto(148.98013636,48.35181374)(149.01399052,48.12004293)(149.06086551,48.04712627)
\curveto(149.10774051,47.97420962)(149.18326133,47.91431379)(149.28742799,47.8674388)
\curveto(149.39680297,47.8205638)(149.70149044,47.77629297)(150.20149037,47.73462631)
\lineto(150.27180286,47.67212632)
\lineto(150.27180286,47.37525136)
\lineto(150.20149037,47.30493887)
\curveto(149.63899044,47.34139719)(149.2014905,47.35962636)(148.88899054,47.35962636)
\curveto(148.60774057,47.35962636)(148.20669895,47.34139719)(147.68586568,47.30493887)
\lineto(147.59992819,47.39087636)
\curveto(147.61555319,47.90129296)(147.62336569,48.25806375)(147.62336569,48.46118872)
\curveto(147.62336569,48.48723039)(147.62596986,48.58618871)(147.63117819,48.75806369)
\curveto(147.34471989,48.54973038)(147.05826159,48.31795957)(146.7718033,48.06275127)
\curveto(146.34992835,47.68775132)(146.07388672,47.45337635)(145.9436784,47.35962636)
\curveto(145.70409509,47.23983471)(145.3629493,47.17993888)(144.92024102,47.17993888)
\curveto(144.20149111,47.17993888)(143.58951202,47.35702219)(143.08430375,47.71118882)
\curveto(142.58430381,48.06535544)(142.22753302,48.51587622)(142.01399138,49.06275115)
\curveto(141.80044974,49.61483441)(141.69367892,50.17993851)(141.69367892,50.75806344)
\curveto(141.69367892,51.34660503)(141.80305391,51.91170913)(142.02180388,52.45337573)
\curveto(142.24576219,53.00025066)(142.56086632,53.39868811)(142.96711627,53.64868808)
\curveto(143.37857455,53.89868805)(143.82388699,54.15650052)(144.3030536,54.42212549)
\curveto(144.78742854,54.69295879)(145.26659515,54.82837544)(145.74055342,54.82837544)
\curveto(146.40201168,54.82837544)(147.03221993,54.66431296)(147.63117819,54.336188)
\lineto(147.59992819,56.07837528)
\curveto(147.58951153,56.68254188)(147.57388653,57.10181266)(147.5530532,57.33618763)
\curveto(147.53221987,57.57577093)(147.50617821,57.71639592)(147.47492821,57.75806258)
\curveto(147.44367821,57.80493757)(147.38899072,57.83879173)(147.31086573,57.85962506)
\curveto(147.23794907,57.88045839)(146.92284494,57.89087506)(146.36555335,57.89087506)
\closepath
\moveto(147.63117819,53.00025066)
\curveto(147.30826156,53.36483395)(146.94888661,53.64347975)(146.55305332,53.83618806)
\curveto(146.16242837,54.02889637)(145.76919925,54.12525052)(145.37336597,54.12525052)
\curveto(144.93586602,54.12525052)(144.52701191,54.00545887)(144.14680362,53.76587557)
\curveto(143.77180367,53.5315006)(143.50097037,53.17993814)(143.33430372,52.7111882)
\curveto(143.16763707,52.24764659)(143.08430375,51.74243832)(143.08430375,51.19556339)
\curveto(143.08430375,50.28410516)(143.31347039,49.55233442)(143.77180367,49.00025116)
\curveto(144.23534528,48.44816789)(144.81607437,48.17212626)(145.51399095,48.17212626)
\curveto(145.89419924,48.17212626)(146.2301367,48.25806375)(146.52180333,48.42993873)
\curveto(146.81867829,48.60702204)(147.06086576,48.84400118)(147.24836574,49.14087614)
\curveto(147.44107405,49.4377511)(147.5530532,49.73462607)(147.5843032,50.03150103)
\curveto(147.61555319,50.32837599)(147.63117819,50.87264676)(147.63117819,51.66431333)
\closepath
}
}
{
\newrgbcolor{curcolor}{0 0 0}
\pscustom[linestyle=none,fillstyle=solid,fillcolor=curcolor]
{
\newpath
\moveto(157.71711445,48.51587622)
\lineto(157.46711448,47.95337629)
\curveto(156.92544788,47.604418)(156.44107294,47.37785552)(156.01398966,47.27368887)
\curveto(155.59211471,47.16952222)(155.21451059,47.11743889)(154.8811773,47.11743889)
\curveto(154.25617737,47.11743889)(153.66242745,47.23983471)(153.09992752,47.48462634)
\curveto(152.54263592,47.72941798)(152.08690681,48.14348043)(151.73274018,48.72681369)
\curveto(151.38378189,49.31014695)(151.20930275,50.01327187)(151.20930275,50.83618843)
\curveto(151.20930275,51.38306336)(151.27701107,51.8752508)(151.41242772,52.31275075)
\curveto(151.54784437,52.75545903)(151.68846936,53.08358399)(151.83430267,53.29712563)
\curveto(151.98534432,53.51066727)(152.23794846,53.7476464)(152.59211508,54.00806304)
\curveto(152.9462817,54.26847967)(153.32128166,54.47681298)(153.71711494,54.63306296)
\curveto(154.11294822,54.78931294)(154.54003151,54.86743793)(154.99836478,54.86743793)
\curveto(155.62336471,54.86743793)(156.1728438,54.71900045)(156.64680208,54.42212549)
\curveto(157.12596869,54.13045886)(157.46190615,53.7554589)(157.65461445,53.29712563)
\curveto(157.84732276,52.83879235)(157.94367692,52.35181324)(157.94367692,51.83618831)
\curveto(157.94367692,51.67472999)(157.93586442,51.51848001)(157.92023942,51.36743836)
\lineto(157.83430193,51.28150088)
\curveto(157.48013531,51.20337588)(157.00357287,51.15129256)(156.40461461,51.12525089)
\curveto(155.80565635,51.09920923)(155.40982306,51.0861884)(155.21711476,51.0861884)
\lineto(152.70930256,51.0861884)
\curveto(152.71971923,50.00806353)(152.99055253,49.2137928)(153.52180246,48.70337619)
\curveto(154.0530524,48.19295959)(154.70409399,47.93775129)(155.47492722,47.93775129)
\curveto(155.83951051,47.93775129)(156.1884688,48.00025128)(156.52180209,48.12525126)
\curveto(156.86034372,48.25025125)(157.21711451,48.42473039)(157.59211446,48.6486887)
\closepath
\moveto(152.70930256,51.71118832)
\curveto(152.80305255,51.69556332)(153.16242751,51.67733416)(153.78742743,51.65650083)
\curveto(154.41763569,51.6356675)(154.88378146,51.62525083)(155.18586476,51.62525083)
\curveto(155.909823,51.62525083)(156.34992712,51.63827166)(156.5061771,51.66431333)
\curveto(156.51138543,51.78931331)(156.5139896,51.88566747)(156.5139896,51.95337579)
\curveto(156.5139896,52.76066736)(156.34992712,53.35962562)(156.02180216,53.75025057)
\curveto(155.6936772,54.14608386)(155.24576058,54.3440005)(154.67805232,54.3440005)
\curveto(154.05826073,54.3440005)(153.57388579,54.12264636)(153.2249275,53.67993808)
\curveto(152.88117754,53.2372298)(152.70930256,52.58097988)(152.70930256,51.71118832)
\closepath
\moveto(154.8889898,55.30493788)
\closepath
\moveto(154.79523981,46.89868892)
\closepath
}
}
{
\newrgbcolor{curcolor}{0 0 0}
\pscustom[linestyle=none,fillstyle=solid,fillcolor=curcolor]
{
\newpath
\moveto(161.47492648,54.81275044)
\lineto(161.62336397,54.71118795)
\curveto(161.59211397,54.40910466)(161.56867647,53.87264639)(161.55305147,53.10181315)
\lineto(162.1389889,53.84400056)
\curveto(162.33169721,54.0887922)(162.50096802,54.27629217)(162.64680134,54.40650049)
\curveto(162.79784299,54.54191714)(162.97232213,54.64608379)(163.17023877,54.71900045)
\curveto(163.36815542,54.79191711)(163.57128039,54.82837544)(163.7796137,54.82837544)
\curveto(164.00878034,54.82837544)(164.22492614,54.78150044)(164.42805112,54.68775046)
\lineto(164.48273861,54.57837547)
\curveto(164.39940529,53.88566722)(164.3525303,53.2762923)(164.34211363,52.75025069)
\lineto(163.99055117,52.75025069)
\curveto(163.78221787,53.2294173)(163.44628041,53.46900061)(162.9827388,53.46900061)
\curveto(162.65982217,53.46900061)(162.37857221,53.36483395)(162.1389889,53.15650064)
\curveto(161.8994056,52.95337567)(161.73794728,52.6955632)(161.65461396,52.38306324)
\curveto(161.57648897,52.07577161)(161.53742648,51.68514666)(161.53742648,51.21118838)
\lineto(161.53742648,50.34400099)
\curveto(161.53742648,50.18775101)(161.54784314,49.80493856)(161.56867647,49.19556363)
\curveto(161.5895098,48.58618871)(161.61034313,48.23462625)(161.63117646,48.14087626)
\curveto(161.65721813,48.04712627)(161.69367646,47.97681378)(161.74055145,47.92993879)
\curveto(161.79263478,47.88827213)(161.85773894,47.8596263)(161.93586393,47.8440013)
\curveto(162.01919725,47.8283763)(162.39159304,47.80233464)(163.05305129,47.76587631)
\lineto(163.12336378,47.70337632)
\lineto(163.12336378,47.37525136)
\lineto(163.05305129,47.30493887)
\curveto(162.36034304,47.34660553)(161.6363848,47.36743886)(160.88117656,47.36743886)
\curveto(160.2822183,47.36743886)(159.65721837,47.34660553)(159.00617679,47.30493887)
\lineto(158.9358643,47.37525136)
\lineto(158.9358643,47.70337632)
\lineto(159.00617679,47.76587631)
\curveto(159.51138506,47.7971263)(159.79523919,47.8205638)(159.85773918,47.8361888)
\curveto(159.92544751,47.8518138)(159.98273917,47.88306379)(160.02961416,47.92993879)
\curveto(160.08169749,47.98202212)(160.11555165,48.05754294)(160.13117665,48.15650126)
\curveto(160.15200998,48.26066791)(160.17284331,48.59660537)(160.19367664,49.16431364)
\curveto(160.21971831,49.73723023)(160.23273914,50.16952185)(160.23273914,50.46118848)
\lineto(160.23273914,52.00025079)
\curveto(160.23273914,52.20858409)(160.22232247,52.49243823)(160.20148914,52.85181318)
\curveto(160.18065581,53.21118814)(160.16242665,53.43254228)(160.14680165,53.5158756)
\curveto(160.13638498,53.59920892)(160.09732249,53.65910475)(160.02961416,53.69556308)
\curveto(159.96190584,53.73722974)(159.82648919,53.75806307)(159.62336421,53.75806307)
\lineto(158.96711429,53.76587557)
\lineto(158.8968018,53.82837556)
\lineto(158.8968018,54.16431302)
\lineto(158.95930179,54.22681301)
\curveto(159.95409334,54.34660466)(160.7926349,54.54191714)(161.47492648,54.81275044)
\closepath
}
}
{
\newrgbcolor{curcolor}{0 0 0}
\pscustom[linestyle=none,fillstyle=solid,fillcolor=curcolor]
{
}
}
{
\newrgbcolor{curcolor}{0 0 0}
\pscustom[linestyle=none,fillstyle=solid,fillcolor=curcolor]
{
\newpath
\moveto(174.17023742,47.22681388)
\curveto(174.09211243,47.45598051)(173.76659163,48.33098041)(173.19367504,49.85181355)
\lineto(170.53742537,56.42212524)
\curveto(170.23534207,57.17212515)(170.04523793,57.59660426)(169.96711294,57.69556258)
\curveto(169.88898795,57.79972924)(169.57648798,57.87004173)(169.02961305,57.90650006)
\lineto(168.95930056,57.97681255)
\lineto(168.95930056,58.33618751)
\lineto(169.02961305,58.4065)
\curveto(169.82648795,58.36483334)(170.5035712,58.344)(171.0608628,58.344)
\curveto(171.58169607,58.344)(172.27180015,58.36483334)(173.13117505,58.4065)
\lineto(173.19367504,58.344)
\lineto(173.19367504,57.91431256)
\lineto(173.13117505,57.85181257)
\curveto(172.50096679,57.85181257)(172.12336267,57.83097923)(171.99836269,57.78931257)
\curveto(171.87857103,57.75285424)(171.81867521,57.67993759)(171.81867521,57.5705626)
\curveto(171.81867521,57.50806261)(171.90200853,57.22941681)(172.06867518,56.7346252)
\curveto(172.23534182,56.24504193)(172.38377931,55.83097948)(172.51398762,55.49243786)
\lineto(174.95929982,49.31275112)
\lineto(176.99836207,54.43775049)
\curveto(177.13898705,54.78670878)(177.34471619,55.34920871)(177.61554949,56.12525028)
\curveto(177.88638279,56.90129185)(178.02179944,57.36743762)(178.02179944,57.52368761)
\curveto(178.02179944,57.65389592)(177.95669528,57.73722925)(177.82648697,57.77368757)
\curveto(177.70148698,57.8101459)(177.3212787,57.85441673)(176.68586211,57.90650006)
\lineto(176.61554962,57.96900005)
\lineto(176.61554962,58.344)
\lineto(176.68586211,58.4065)
\curveto(177.59732033,58.36483334)(178.17804942,58.344)(178.42804939,58.344)
\curveto(178.63117437,58.344)(179.15721597,58.36483334)(180.0061742,58.4065)
\lineto(180.06867419,58.344)
\lineto(180.06867419,57.96900005)
\lineto(180.0061742,57.90650006)
\curveto(179.63117424,57.89087506)(179.40461177,57.86222923)(179.32648678,57.82056257)
\curveto(179.24836179,57.77889591)(179.1780493,57.69816675)(179.11554931,57.5783751)
\curveto(179.05825765,57.45858345)(178.87336184,57.04712516)(178.56086188,56.34400025)
\lineto(176.12336218,50.34400099)
\curveto(175.57648724,49.00545949)(175.18846646,47.96639712)(174.95929982,47.22681388)
\closepath
}
}
{
\newrgbcolor{curcolor}{0 0 0}
\pscustom[linestyle=none,fillstyle=solid,fillcolor=curcolor]
{
\newpath
\moveto(186.12336094,48.51587622)
\lineto(185.87336098,47.95337629)
\curveto(185.33169438,47.604418)(184.84731944,47.37785552)(184.42023615,47.27368887)
\curveto(183.99836121,47.16952222)(183.62075709,47.11743889)(183.28742379,47.11743889)
\curveto(182.66242387,47.11743889)(182.06867394,47.23983471)(181.50617401,47.48462634)
\curveto(180.94888242,47.72941798)(180.49315331,48.14348043)(180.13898668,48.72681369)
\curveto(179.79002839,49.31014695)(179.61554925,50.01327187)(179.61554925,50.83618843)
\curveto(179.61554925,51.38306336)(179.68325757,51.8752508)(179.81867422,52.31275075)
\curveto(179.95409087,52.75545903)(180.09471585,53.08358399)(180.24054917,53.29712563)
\curveto(180.39159082,53.51066727)(180.64419495,53.7476464)(180.99836158,54.00806304)
\curveto(181.3525282,54.26847967)(181.72752815,54.47681298)(182.12336144,54.63306296)
\curveto(182.51919472,54.78931294)(182.946278,54.86743793)(183.40461128,54.86743793)
\curveto(184.0296112,54.86743793)(184.5790903,54.71900045)(185.05304858,54.42212549)
\curveto(185.53221518,54.13045886)(185.86815264,53.7554589)(186.06086095,53.29712563)
\curveto(186.25356926,52.83879235)(186.34992342,52.35181324)(186.34992342,51.83618831)
\curveto(186.34992342,51.67472999)(186.34211092,51.51848001)(186.32648592,51.36743836)
\lineto(186.24054843,51.28150088)
\curveto(185.88638181,51.20337588)(185.40981937,51.15129256)(184.81086111,51.12525089)
\curveto(184.21190285,51.09920923)(183.81606956,51.0861884)(183.62336125,51.0861884)
\lineto(181.11554906,51.0861884)
\curveto(181.12596573,50.00806353)(181.39679903,49.2137928)(181.92804896,48.70337619)
\curveto(182.4592989,48.19295959)(183.11034048,47.93775129)(183.88117372,47.93775129)
\curveto(184.24575701,47.93775129)(184.5947153,48.00025128)(184.92804859,48.12525126)
\curveto(185.26659022,48.25025125)(185.62336101,48.42473039)(185.99836096,48.6486887)
\closepath
\moveto(181.11554906,51.71118832)
\curveto(181.20929905,51.69556332)(181.56867401,51.67733416)(182.19367393,51.65650083)
\curveto(182.82388218,51.6356675)(183.29002796,51.62525083)(183.59211126,51.62525083)
\curveto(184.3160695,51.62525083)(184.75617361,51.63827166)(184.91242359,51.66431333)
\curveto(184.91763193,51.78931331)(184.92023609,51.88566747)(184.92023609,51.95337579)
\curveto(184.92023609,52.76066736)(184.75617361,53.35962562)(184.42804865,53.75025057)
\curveto(184.09992369,54.14608386)(183.65200708,54.3440005)(183.08429882,54.3440005)
\curveto(182.46450723,54.3440005)(181.98013229,54.12264636)(181.631174,53.67993808)
\curveto(181.28742404,53.2372298)(181.11554906,52.58097988)(181.11554906,51.71118832)
\closepath
\moveto(183.29523629,55.30493788)
\closepath
\moveto(183.2014863,46.89868892)
\closepath
}
}
{
\newrgbcolor{curcolor}{0 0 0}
\pscustom[linestyle=none,fillstyle=solid,fillcolor=curcolor]
{
\newpath
\moveto(189.88117298,54.81275044)
\lineto(190.02961046,54.71118795)
\curveto(189.99836047,54.40910466)(189.97492297,53.87264639)(189.95929797,53.10181315)
\lineto(190.5452354,53.84400056)
\curveto(190.73794371,54.0887922)(190.90721452,54.27629217)(191.05304784,54.40650049)
\curveto(191.20408948,54.54191714)(191.37856863,54.64608379)(191.57648527,54.71900045)
\curveto(191.77440191,54.79191711)(191.97752689,54.82837544)(192.1858602,54.82837544)
\curveto(192.41502684,54.82837544)(192.63117264,54.78150044)(192.83429762,54.68775046)
\lineto(192.88898511,54.57837547)
\curveto(192.80565179,53.88566722)(192.75877679,53.2762923)(192.74836013,52.75025069)
\lineto(192.39679767,52.75025069)
\curveto(192.18846436,53.2294173)(191.8525269,53.46900061)(191.3889853,53.46900061)
\curveto(191.06606867,53.46900061)(190.7848187,53.36483395)(190.5452354,53.15650064)
\curveto(190.3056521,52.95337567)(190.14419378,52.6955632)(190.06086046,52.38306324)
\curveto(189.98273547,52.07577161)(189.94367297,51.68514666)(189.94367297,51.21118838)
\lineto(189.94367297,50.34400099)
\curveto(189.94367297,50.18775101)(189.95408964,49.80493856)(189.97492297,49.19556363)
\curveto(189.9957563,48.58618871)(190.01658963,48.23462625)(190.03742296,48.14087626)
\curveto(190.06346463,48.04712627)(190.09992295,47.97681378)(190.14679795,47.92993879)
\curveto(190.19888128,47.88827213)(190.26398543,47.8596263)(190.34211042,47.8440013)
\curveto(190.42544375,47.8283763)(190.79783953,47.80233464)(191.45929779,47.76587631)
\lineto(191.52961028,47.70337632)
\lineto(191.52961028,47.37525136)
\lineto(191.45929779,47.30493887)
\curveto(190.76658954,47.34660553)(190.04263129,47.36743886)(189.28742305,47.36743886)
\curveto(188.68846479,47.36743886)(188.06346487,47.34660553)(187.41242329,47.30493887)
\lineto(187.34211079,47.37525136)
\lineto(187.34211079,47.70337632)
\lineto(187.41242329,47.76587631)
\curveto(187.91763156,47.7971263)(188.20148569,47.8205638)(188.26398568,47.8361888)
\curveto(188.33169401,47.8518138)(188.38898566,47.88306379)(188.43586066,47.92993879)
\curveto(188.48794399,47.98202212)(188.52179815,48.05754294)(188.53742315,48.15650126)
\curveto(188.55825648,48.26066791)(188.57908981,48.59660537)(188.59992314,49.16431364)
\curveto(188.6259648,49.73723023)(188.63898563,50.16952185)(188.63898563,50.46118848)
\lineto(188.63898563,52.00025079)
\curveto(188.63898563,52.20858409)(188.62856897,52.49243823)(188.60773564,52.85181318)
\curveto(188.58690231,53.21118814)(188.56867314,53.43254228)(188.55304814,53.5158756)
\curveto(188.54263148,53.59920892)(188.50356898,53.65910475)(188.43586066,53.69556308)
\curveto(188.36815233,53.73722974)(188.23273568,53.75806307)(188.02961071,53.75806307)
\lineto(187.37336079,53.76587557)
\lineto(187.3030483,53.82837556)
\lineto(187.3030483,54.16431302)
\lineto(187.36554829,54.22681301)
\curveto(188.36033984,54.34660466)(189.1988814,54.54191714)(189.88117298,54.81275044)
\closepath
}
}
{
\newrgbcolor{curcolor}{0 0 0}
\pscustom[linestyle=none,fillstyle=solid,fillcolor=curcolor]
{
\newpath
\moveto(193.99054747,49.66431357)
\lineto(194.32648493,49.66431357)
\lineto(194.39679742,49.59400108)
\curveto(194.40721409,49.15650114)(194.43325575,48.78150118)(194.47492241,48.46900122)
\curveto(194.63638073,48.22420959)(194.91502653,48.02368878)(195.31085981,47.8674388)
\curveto(195.7066931,47.71639715)(196.09731805,47.64087632)(196.48273467,47.64087632)
\curveto(197.03481793,47.64087632)(197.47492204,47.78931381)(197.803047,48.08618877)
\curveto(198.1363803,48.38306373)(198.30304694,48.73723036)(198.30304694,49.14868864)
\curveto(198.30304694,49.37264694)(198.24315112,49.56535525)(198.12335946,49.72681357)
\curveto(198.00356781,49.89348021)(197.82127617,50.02889686)(197.57648453,50.13306352)
\curveto(197.33690123,50.2424385)(196.90460961,50.37004265)(196.27960969,50.51587597)
\curveto(195.74315142,50.64087595)(195.36815147,50.73723011)(195.15460983,50.80493843)
\curveto(194.94106819,50.87785509)(194.73533905,50.99764674)(194.53742241,51.16431339)
\curveto(194.33950576,51.33098004)(194.18846412,51.53410501)(194.08429746,51.77368831)
\curveto(193.98013081,52.01847995)(193.92804748,52.28670908)(193.92804748,52.57837572)
\curveto(193.92804748,53.2762923)(194.20408911,53.83358389)(194.75617238,54.25025051)
\curveto(195.31346398,54.67212546)(196.00617223,54.88306293)(196.83429712,54.88306293)
\curveto(197.18325541,54.88306293)(197.5634637,54.83358377)(197.97492198,54.73462545)
\curveto(198.38638027,54.64087546)(198.69888023,54.54972964)(198.91242187,54.46118798)
\lineto(198.98273436,54.351813)
\curveto(198.9410677,54.14347969)(198.91502603,53.59660476)(198.90460937,52.7111882)
\lineto(198.83429688,52.64087571)
\lineto(198.52179692,52.64087571)
\lineto(198.45148442,52.7111882)
\curveto(198.43065109,53.02889649)(198.40200526,53.25025063)(198.36554693,53.37525062)
\curveto(198.32908861,53.50545893)(198.23533862,53.64347975)(198.08429697,53.78931307)
\curveto(197.93325532,53.94035471)(197.71971368,54.0653547)(197.44367205,54.16431302)
\curveto(197.16763042,54.26847967)(196.87596379,54.320563)(196.56867216,54.320563)
\curveto(196.25096386,54.320563)(195.98273473,54.27368801)(195.76398476,54.17993802)
\curveto(195.55044312,54.08618803)(195.37596397,53.94295888)(195.24054732,53.75025057)
\curveto(195.110339,53.56275059)(195.04523484,53.33358396)(195.04523484,53.06275066)
\curveto(195.04523484,52.86483401)(195.08429734,52.6877507)(195.16242233,52.53150072)
\curveto(195.24575565,52.37525074)(195.37075564,52.24764659)(195.53742228,52.14868827)
\curveto(195.70408893,52.05493828)(195.87856807,51.98722995)(196.06085972,51.94556329)
\lineto(196.91242211,51.72681332)
\curveto(197.64158869,51.54973001)(198.16242196,51.39608419)(198.47492192,51.26587588)
\curveto(198.79263022,51.13566756)(199.03742185,50.93775092)(199.20929683,50.67212595)
\curveto(199.38117181,50.41170932)(199.4671093,50.09400102)(199.4671093,49.71900107)
\curveto(199.4671093,49.00025116)(199.17023434,48.38306373)(198.57648441,47.8674388)
\curveto(197.98273448,47.35181386)(197.21971374,47.09400139)(196.28742219,47.09400139)
\curveto(195.95929723,47.09400139)(195.55044312,47.13045972)(195.06085984,47.20337638)
\curveto(194.5764849,47.27629304)(194.16242245,47.35702219)(193.8186725,47.44556385)
\lineto(193.77961,47.54712634)
\lineto(193.85773499,48.07056377)
\curveto(193.88377665,48.23202208)(193.89940165,48.3934804)(193.90460998,48.55493871)
\curveto(193.90981832,48.72160536)(193.91502665,49.06795948)(193.92023498,49.59400108)
\closepath
}
}
{
\newrgbcolor{curcolor}{0 0 0}
\pscustom[linestyle=none,fillstyle=solid,fillcolor=curcolor]
{
\newpath
\moveto(202.6858589,54.81275044)
\lineto(202.83429638,54.71118795)
\curveto(202.77700472,54.02889637)(202.74835889,53.16170898)(202.74835889,52.10962577)
\lineto(202.74835889,50.2893135)
\curveto(202.74835889,49.62264691)(202.79783805,49.1591053)(202.89679638,48.89868867)
\curveto(203.00096303,48.63827203)(203.17804634,48.43775123)(203.42804631,48.29712624)
\curveto(203.67804628,48.16170959)(203.97492124,48.09400127)(204.3186712,48.09400127)
\curveto(204.69887949,48.09400127)(205.04523361,48.16691793)(205.35773357,48.31275124)
\curveto(205.67023353,48.46379289)(205.93065017,48.67473036)(206.13898348,48.94556366)
\curveto(206.35252512,49.21639696)(206.47492093,49.39868861)(206.50617093,49.4924386)
\curveto(206.53742093,49.58618858)(206.55825426,49.84920939)(206.56867092,50.281501)
\lineto(206.59210842,51.15650089)
\lineto(206.59210842,52.00025079)
\curveto(206.59210842,52.20858409)(206.58169176,52.49243823)(206.56085842,52.85181318)
\curveto(206.54002509,53.21118814)(206.52179593,53.43254228)(206.50617093,53.5158756)
\curveto(206.49575427,53.59920892)(206.45669177,53.65910475)(206.38898345,53.69556308)
\curveto(206.32127512,53.73722974)(206.18585847,53.75806307)(205.9827335,53.75806307)
\lineto(205.32648358,53.76587557)
\lineto(205.25617109,53.82837556)
\lineto(205.25617109,54.16431302)
\lineto(205.31867108,54.22681301)
\curveto(206.31346262,54.34660466)(207.15200418,54.54191714)(207.83429577,54.81275044)
\lineto(207.98273325,54.71118795)
\curveto(207.92544159,54.02889637)(207.89679576,53.16170898)(207.89679576,52.10962577)
\lineto(207.89679576,50.73462594)
\curveto(207.89679576,50.65650095)(207.90981659,50.0184802)(207.93585825,48.82056368)
\curveto(207.94627492,48.3778554)(207.97231658,48.1070221)(208.01398325,48.00806378)
\curveto(208.06085824,47.91431379)(208.12335823,47.84660547)(208.20148322,47.8049388)
\curveto(208.27960821,47.76848048)(208.51137902,47.75025131)(208.89679564,47.75025131)
\lineto(209.11554561,47.75025131)
\lineto(209.1858581,47.68775132)
\lineto(209.1858581,47.37525136)
\lineto(209.12335811,47.30493887)
\curveto(208.3889832,47.34660553)(207.89679576,47.36743886)(207.64679579,47.36743886)
\curveto(207.3290875,47.36743886)(206.96190004,47.34920969)(206.54523343,47.31275136)
\lineto(206.47492093,47.37525136)
\curveto(206.50617093,47.90650129)(206.52960843,48.35702207)(206.54523343,48.72681369)
\curveto(206.21710847,48.47160539)(205.84210851,48.1304596)(205.42023356,47.70337632)
\curveto(205.25877525,47.541918)(205.02700445,47.40650135)(204.72492115,47.29712637)
\curveto(204.42283785,47.18775138)(204.0790879,47.13306389)(203.69367128,47.13306389)
\curveto(203.11033802,47.13306389)(202.65460891,47.22420971)(202.32648395,47.40650135)
\curveto(202.00356732,47.59400133)(201.77440068,47.84920963)(201.63898403,48.17212626)
\curveto(201.50356738,48.50025122)(201.43585906,49.06275115)(201.43585906,49.85962605)
\lineto(201.44367156,50.47681347)
\lineto(201.44367156,52.00025079)
\curveto(201.44367156,52.20858409)(201.43325489,52.49243823)(201.41242156,52.85181318)
\curveto(201.39158823,53.21118814)(201.37335906,53.43254228)(201.35773407,53.5158756)
\curveto(201.3473174,53.59920892)(201.30825491,53.65910475)(201.24054658,53.69556308)
\curveto(201.17283826,53.73722974)(201.03742161,53.75806307)(200.83429663,53.75806307)
\lineto(200.17804671,53.76587557)
\lineto(200.10773422,53.82837556)
\lineto(200.10773422,54.16431302)
\lineto(200.17023421,54.22681301)
\curveto(201.16502576,54.34660466)(202.00356732,54.54191714)(202.6858589,54.81275044)
\closepath
\moveto(204.55304617,55.30493788)
\closepath
\moveto(204.63117116,46.89868892)
\closepath
}
}
{
\newrgbcolor{curcolor}{0 0 0}
\pscustom[linestyle=none,fillstyle=solid,fillcolor=curcolor]
{
\newpath
\moveto(216.41241971,48.20337625)
\lineto(216.09991975,47.67993882)
\curveto(215.42804483,47.3101472)(214.67804492,47.12525139)(213.84992003,47.12525139)
\curveto(212.70408683,47.12525139)(211.81867028,47.46639718)(211.19367035,48.14868876)
\curveto(210.56867043,48.83098034)(210.25617047,49.71118857)(210.25617047,50.78931344)
\curveto(210.25617047,51.27889671)(210.3082538,51.71379249)(210.41242045,52.09400078)
\curveto(210.52179544,52.47941739)(210.65721209,52.79712569)(210.8186704,53.04712566)
\curveto(210.98533705,53.29712563)(211.16762869,53.49504227)(211.36554533,53.64087558)
\curveto(211.56346197,53.7867089)(211.89679527,53.98983387)(212.36554521,54.25025051)
\curveto(212.83950348,54.51066714)(213.19627427,54.67993796)(213.43585758,54.75806295)
\curveto(213.67544088,54.84139627)(214.00617001,54.88306293)(214.42804495,54.88306293)
\curveto(215.23533652,54.88306293)(215.89679477,54.72941712)(216.41241971,54.42212549)
\curveto(216.31346139,53.83358389)(216.23794056,53.17733397)(216.18585724,52.45337573)
\lineto(216.11554475,52.39087574)
\lineto(215.79523229,52.39087574)
\lineto(215.72491979,52.46118823)
\curveto(215.70408646,52.89868818)(215.67544063,53.19295897)(215.6389823,53.34400062)
\curveto(215.60252398,53.49504227)(215.39419067,53.65650058)(215.01398238,53.82837556)
\curveto(214.63898243,54.00025054)(214.23794081,54.08618803)(213.81085753,54.08618803)
\curveto(213.38377425,54.08618803)(213.00356596,53.99504221)(212.67023267,53.81275056)
\curveto(212.33689938,53.63566725)(212.08429524,53.34139645)(211.91242026,52.92993817)
\curveto(211.74575362,52.52368822)(211.6624203,52.03670912)(211.6624203,51.46900085)
\curveto(211.6624203,50.98983424)(211.73012862,50.5210843)(211.86554527,50.06275103)
\curveto(212.00096192,49.60962608)(212.17804523,49.23202196)(212.3967952,48.92993867)
\curveto(212.62075351,48.6330637)(212.91502431,48.3934804)(213.2796076,48.21118875)
\curveto(213.64419088,48.02889711)(214.05044083,47.93775129)(214.49835745,47.93775129)
\curveto(214.77960741,47.93775129)(215.05304488,47.97420962)(215.31866984,48.04712627)
\curveto(215.58950314,48.12004293)(215.89939894,48.23983458)(216.24835723,48.40650123)
\closepath
}
}
{
\newrgbcolor{curcolor}{0 0 0}
\pscustom[linestyle=none,fillstyle=solid,fillcolor=curcolor]
{
\newpath
\moveto(219.38898184,58.93774993)
\lineto(219.53741932,58.84399994)
\curveto(219.48012766,58.24504168)(219.45148183,57.044521)(219.45148183,55.24243789)
\lineto(219.45148183,53.31275062)
\curveto(219.88898178,53.67212558)(220.29523173,54.02889637)(220.67023168,54.38306299)
\curveto(220.77960667,54.48202131)(220.88377332,54.55493797)(220.98273165,54.60181297)
\curveto(221.08168997,54.64868796)(221.24054411,54.69816712)(221.45929409,54.75025045)
\curveto(221.68325239,54.80233377)(221.91241903,54.82837544)(222.146794,54.82837544)
\curveto(222.54262729,54.82837544)(222.92543974,54.74764628)(223.29523136,54.58618797)
\curveto(223.67023131,54.42472965)(223.95148128,54.23202134)(224.13898126,54.00806304)
\curveto(224.33168957,53.78931307)(224.46450205,53.52889643)(224.53741871,53.22681314)
\curveto(224.61033537,52.92472984)(224.64679369,52.55233405)(224.64679369,52.10962577)
\lineto(224.64679369,50.73462594)
\curveto(224.64679369,50.64608429)(224.65981453,49.99764687)(224.68585619,48.78931368)
\curveto(224.70148119,48.29452208)(224.75616868,48.00025128)(224.84991867,47.90650129)
\curveto(224.94887699,47.8127513)(225.27179362,47.76587631)(225.81866855,47.76587631)
\lineto(225.88116854,47.70337632)
\lineto(225.88116854,47.36743886)
\lineto(225.81866855,47.30493887)
\curveto(225.1572103,47.34660553)(224.70929369,47.36743886)(224.47491872,47.36743886)
\curveto(224.35512706,47.36743886)(223.97491878,47.34660553)(223.33429386,47.30493887)
\lineto(223.24054387,47.39087636)
\curveto(223.30825219,48.0679596)(223.34210636,48.950772)(223.34210636,50.03931353)
\lineto(223.34210636,51.0627509)
\curveto(223.34210636,51.67733416)(223.32648136,52.12004244)(223.29523136,52.39087574)
\curveto(223.2691897,52.66170904)(223.18325221,52.90650067)(223.03741889,53.12525065)
\curveto(222.89158558,53.34920895)(222.6962731,53.52108393)(222.45148146,53.64087558)
\curveto(222.21189816,53.76587557)(221.92023153,53.82837556)(221.57648157,53.82837556)
\curveto(221.22231495,53.82837556)(220.92543999,53.7788964)(220.68585668,53.67993808)
\curveto(220.45148171,53.58618809)(220.21189841,53.42472978)(219.96710677,53.19556314)
\curveto(219.72231513,52.9663965)(219.57387765,52.77368819)(219.52179433,52.61743821)
\curveto(219.47491933,52.46639656)(219.45148183,52.17212577)(219.45148183,51.73462582)
\lineto(219.45148183,50.34400099)
\curveto(219.45148183,50.22941767)(219.4618985,49.85962605)(219.48273183,49.23462613)
\curveto(219.50356516,48.61483454)(219.52439849,48.25285542)(219.54523182,48.14868876)
\curveto(219.57127349,48.04452211)(219.60773182,47.96900128)(219.65460681,47.92212629)
\curveto(219.7014818,47.8752513)(219.75877346,47.8440013)(219.82648179,47.8283763)
\curveto(219.89419011,47.81795964)(220.17544008,47.7971263)(220.67023168,47.76587631)
\lineto(220.74054418,47.70337632)
\lineto(220.74054418,47.37525136)
\lineto(220.67804418,47.30493887)
\curveto(220.0270026,47.34660553)(219.39939851,47.36743886)(218.79523192,47.36743886)
\curveto(218.19627366,47.36743886)(217.57127373,47.34660553)(216.92023215,47.30493887)
\lineto(216.84991966,47.37525136)
\lineto(216.84991966,47.70337632)
\lineto(216.92023215,47.76587631)
\curveto(217.42544042,47.7971263)(217.70929455,47.8205638)(217.77179454,47.8361888)
\curveto(217.83950287,47.8518138)(217.89679453,47.88306379)(217.94366952,47.92993879)
\curveto(217.99575285,47.98202212)(218.02960701,48.05754294)(218.04523201,48.15650126)
\curveto(218.06606534,48.26066791)(218.08689867,48.59660537)(218.107732,49.16431364)
\curveto(218.13377366,49.73723023)(218.1467945,50.16952185)(218.1467945,50.46118848)
\lineto(218.1467945,54.50806298)
\lineto(218.1155445,56.16431277)
\curveto(218.0999195,56.7424377)(218.08169034,57.14347932)(218.06085701,57.36743762)
\curveto(218.04523201,57.59139593)(218.02179451,57.72420841)(217.99054451,57.76587508)
\curveto(217.95929452,57.80754174)(217.90460703,57.83879173)(217.82648204,57.85962506)
\curveto(217.74835704,57.88045839)(217.43325292,57.89087506)(216.88116965,57.89087506)
\lineto(216.81085716,57.96118755)
\lineto(216.81085716,58.28931251)
\lineto(216.87335715,58.359625)
\curveto(217.8681487,58.47420832)(218.70669026,58.66691663)(219.38898184,58.93774993)
\closepath
}
}
{
\newrgbcolor{curcolor}{0 0 0}
\pscustom[linestyle=none,fillstyle=solid,fillcolor=curcolor]
{
\newpath
\moveto(226.83429342,49.66431357)
\lineto(227.17023088,49.66431357)
\lineto(227.24054337,49.59400108)
\curveto(227.25096004,49.15650114)(227.2770017,48.78150118)(227.31866836,48.46900122)
\curveto(227.48012668,48.22420959)(227.75877248,48.02368878)(228.15460576,47.8674388)
\curveto(228.55043905,47.71639715)(228.941064,47.64087632)(229.32648062,47.64087632)
\curveto(229.87856388,47.64087632)(230.31866799,47.78931381)(230.64679295,48.08618877)
\curveto(230.98012625,48.38306373)(231.14679289,48.73723036)(231.14679289,49.14868864)
\curveto(231.14679289,49.37264694)(231.08689707,49.56535525)(230.96710541,49.72681357)
\curveto(230.84731376,49.89348021)(230.66502212,50.02889686)(230.42023048,50.13306352)
\curveto(230.18064718,50.2424385)(229.74835557,50.37004265)(229.12335564,50.51587597)
\curveto(228.58689738,50.64087595)(228.21189742,50.73723011)(227.99835578,50.80493843)
\curveto(227.78481414,50.87785509)(227.579085,50.99764674)(227.38116836,51.16431339)
\curveto(227.18325171,51.33098004)(227.03221007,51.53410501)(226.92804341,51.77368831)
\curveto(226.82387676,52.01847995)(226.77179343,52.28670908)(226.77179343,52.57837572)
\curveto(226.77179343,53.2762923)(227.04783506,53.83358389)(227.59991833,54.25025051)
\curveto(228.15720993,54.67212546)(228.84991818,54.88306293)(229.67804307,54.88306293)
\curveto(230.02700136,54.88306293)(230.40720965,54.83358377)(230.81866793,54.73462545)
\curveto(231.23012622,54.64087546)(231.54262618,54.54972964)(231.75616782,54.46118798)
\lineto(231.82648031,54.351813)
\curveto(231.78481365,54.14347969)(231.75877198,53.59660476)(231.74835532,52.7111882)
\lineto(231.67804283,52.64087571)
\lineto(231.36554287,52.64087571)
\lineto(231.29523037,52.7111882)
\curveto(231.27439704,53.02889649)(231.24575121,53.25025063)(231.20929289,53.37525062)
\curveto(231.17283456,53.50545893)(231.07908457,53.64347975)(230.92804292,53.78931307)
\curveto(230.77700127,53.94035471)(230.56345963,54.0653547)(230.287418,54.16431302)
\curveto(230.01137637,54.26847967)(229.71970974,54.320563)(229.41241811,54.320563)
\curveto(229.09470981,54.320563)(228.82648068,54.27368801)(228.60773071,54.17993802)
\curveto(228.39418907,54.08618803)(228.21970992,53.94295888)(228.08429327,53.75025057)
\curveto(227.95408495,53.56275059)(227.88898079,53.33358396)(227.88898079,53.06275066)
\curveto(227.88898079,52.86483401)(227.92804329,52.6877507)(228.00616828,52.53150072)
\curveto(228.0895016,52.37525074)(228.21450159,52.24764659)(228.38116823,52.14868827)
\curveto(228.54783488,52.05493828)(228.72231403,51.98722995)(228.90460567,51.94556329)
\lineto(229.75616806,51.72681332)
\curveto(230.48533464,51.54973001)(231.00616791,51.39608419)(231.31866787,51.26587588)
\curveto(231.63637617,51.13566756)(231.8811678,50.93775092)(232.05304278,50.67212595)
\curveto(232.22491776,50.41170932)(232.31085525,50.09400102)(232.31085525,49.71900107)
\curveto(232.31085525,49.00025116)(232.01398029,48.38306373)(231.42023036,47.8674388)
\curveto(230.82648043,47.35181386)(230.06345969,47.09400139)(229.13116814,47.09400139)
\curveto(228.80304318,47.09400139)(228.39418907,47.13045972)(227.90460579,47.20337638)
\curveto(227.42023085,47.27629304)(227.0061684,47.35702219)(226.66241845,47.44556385)
\lineto(226.62335595,47.54712634)
\lineto(226.70148094,48.07056377)
\curveto(226.7275226,48.23202208)(226.7431476,48.3934804)(226.74835594,48.55493871)
\curveto(226.75356427,48.72160536)(226.7587726,49.06795948)(226.76398093,49.59400108)
\closepath
}
}
{
\newrgbcolor{curcolor}{0 0 0}
\pscustom[linestyle=none,fillstyle=solid,fillcolor=curcolor]
{
\newpath
\moveto(235.82647982,54.81275044)
\lineto(235.9749173,54.71118795)
\curveto(235.9436673,54.40910466)(235.9202298,53.87264639)(235.90460481,53.10181315)
\lineto(236.49054223,53.84400056)
\curveto(236.68325054,54.0887922)(236.85252136,54.27629217)(236.99835467,54.40650049)
\curveto(237.14939632,54.54191714)(237.32387546,54.64608379)(237.52179211,54.71900045)
\curveto(237.71970875,54.79191711)(237.92283372,54.82837544)(238.13116703,54.82837544)
\curveto(238.36033367,54.82837544)(238.57647948,54.78150044)(238.77960445,54.68775046)
\lineto(238.83429194,54.57837547)
\curveto(238.75095862,53.88566722)(238.70408363,53.2762923)(238.69366696,52.75025069)
\lineto(238.34210451,52.75025069)
\curveto(238.1337712,53.2294173)(237.79783374,53.46900061)(237.33429213,53.46900061)
\curveto(237.0113755,53.46900061)(236.73012554,53.36483395)(236.49054223,53.15650064)
\curveto(236.25095893,52.95337567)(236.08950062,52.6955632)(236.00616729,52.38306324)
\curveto(235.9280423,52.07577161)(235.88897981,51.68514666)(235.88897981,51.21118838)
\lineto(235.88897981,50.34400099)
\curveto(235.88897981,50.18775101)(235.89939647,49.80493856)(235.9202298,49.19556363)
\curveto(235.94106313,48.58618871)(235.96189647,48.23462625)(235.9827298,48.14087626)
\curveto(236.00877146,48.04712627)(236.04522979,47.97681378)(236.09210478,47.92993879)
\curveto(236.14418811,47.88827213)(236.20929227,47.8596263)(236.28741726,47.8440013)
\curveto(236.37075058,47.8283763)(236.74314637,47.80233464)(237.40460462,47.76587631)
\lineto(237.47491711,47.70337632)
\lineto(237.47491711,47.37525136)
\lineto(237.40460462,47.30493887)
\curveto(236.71189637,47.34660553)(235.98793813,47.36743886)(235.23272989,47.36743886)
\curveto(234.63377163,47.36743886)(234.00877171,47.34660553)(233.35773012,47.30493887)
\lineto(233.28741763,47.37525136)
\lineto(233.28741763,47.70337632)
\lineto(233.35773012,47.76587631)
\curveto(233.86293839,47.7971263)(234.14679252,47.8205638)(234.20929252,47.8361888)
\curveto(234.27700084,47.8518138)(234.3342925,47.88306379)(234.38116749,47.92993879)
\curveto(234.43325082,47.98202212)(234.46710498,48.05754294)(234.48272998,48.15650126)
\curveto(234.50356331,48.26066791)(234.52439664,48.59660537)(234.54522997,49.16431364)
\curveto(234.57127164,49.73723023)(234.58429247,50.16952185)(234.58429247,50.46118848)
\lineto(234.58429247,52.00025079)
\curveto(234.58429247,52.20858409)(234.5738758,52.49243823)(234.55304247,52.85181318)
\curveto(234.53220914,53.21118814)(234.51397998,53.43254228)(234.49835498,53.5158756)
\curveto(234.48793831,53.59920892)(234.44887582,53.65910475)(234.38116749,53.69556308)
\curveto(234.31345917,53.73722974)(234.17804252,53.75806307)(233.97491754,53.75806307)
\lineto(233.31866762,53.76587557)
\lineto(233.24835513,53.82837556)
\lineto(233.24835513,54.16431302)
\lineto(233.31085513,54.22681301)
\curveto(234.30564667,54.34660466)(235.14418823,54.54191714)(235.82647982,54.81275044)
\closepath
}
}
{
\newrgbcolor{curcolor}{0 0 0}
\pscustom[linestyle=none,fillstyle=solid,fillcolor=curcolor]
{
\newpath
\moveto(246.05304105,48.51587622)
\lineto(245.80304109,47.95337629)
\curveto(245.26137449,47.604418)(244.77699955,47.37785552)(244.34991626,47.27368887)
\curveto(243.92804132,47.16952222)(243.5504372,47.11743889)(243.2171039,47.11743889)
\curveto(242.59210398,47.11743889)(241.99835405,47.23983471)(241.43585412,47.48462634)
\curveto(240.87856253,47.72941798)(240.42283342,48.14348043)(240.06866679,48.72681369)
\curveto(239.7197085,49.31014695)(239.54522936,50.01327187)(239.54522936,50.83618843)
\curveto(239.54522936,51.38306336)(239.61293768,51.8752508)(239.74835433,52.31275075)
\curveto(239.88377098,52.75545903)(240.02439596,53.08358399)(240.17022928,53.29712563)
\curveto(240.32127093,53.51066727)(240.57387506,53.7476464)(240.92804169,54.00806304)
\curveto(241.28220831,54.26847967)(241.65720826,54.47681298)(242.05304155,54.63306296)
\curveto(242.44887483,54.78931294)(242.87595811,54.86743793)(243.33429139,54.86743793)
\curveto(243.95929131,54.86743793)(244.50877041,54.71900045)(244.98272869,54.42212549)
\curveto(245.46189529,54.13045886)(245.79783275,53.7554589)(245.99054106,53.29712563)
\curveto(246.18324937,52.83879235)(246.27960353,52.35181324)(246.27960353,51.83618831)
\curveto(246.27960353,51.67472999)(246.27179103,51.51848001)(246.25616603,51.36743836)
\lineto(246.17022854,51.28150088)
\curveto(245.81606192,51.20337588)(245.33949948,51.15129256)(244.74054122,51.12525089)
\curveto(244.14158296,51.09920923)(243.74574967,51.0861884)(243.55304136,51.0861884)
\lineto(241.04522917,51.0861884)
\curveto(241.05564584,50.00806353)(241.32647914,49.2137928)(241.85772907,48.70337619)
\curveto(242.38897901,48.19295959)(243.04002059,47.93775129)(243.81085383,47.93775129)
\curveto(244.17543712,47.93775129)(244.52439541,48.00025128)(244.8577287,48.12525126)
\curveto(245.19627033,48.25025125)(245.55304112,48.42473039)(245.92804107,48.6486887)
\closepath
\moveto(241.04522917,51.71118832)
\curveto(241.13897916,51.69556332)(241.49835412,51.67733416)(242.12335404,51.65650083)
\curveto(242.75356229,51.6356675)(243.21970807,51.62525083)(243.52179137,51.62525083)
\curveto(244.24574961,51.62525083)(244.68585372,51.63827166)(244.8421037,51.66431333)
\curveto(244.84731204,51.78931331)(244.8499162,51.88566747)(244.8499162,51.95337579)
\curveto(244.8499162,52.76066736)(244.68585372,53.35962562)(244.35772876,53.75025057)
\curveto(244.0296038,54.14608386)(243.58168719,54.3440005)(243.01397893,54.3440005)
\curveto(242.39418734,54.3440005)(241.9098124,54.12264636)(241.56085411,53.67993808)
\curveto(241.21710415,53.2372298)(241.04522917,52.58097988)(241.04522917,51.71118832)
\closepath
\moveto(243.2249164,55.30493788)
\closepath
\moveto(243.13116642,46.89868892)
\closepath
}
}
{
\newrgbcolor{curcolor}{0 0 0}
\pscustom[linestyle=none,fillstyle=solid,fillcolor=curcolor]
{
\newpath
\moveto(249.06085318,58.30493751)
\curveto(249.30564482,58.30493751)(249.51397813,58.21900002)(249.68585311,58.04712504)
\curveto(249.85772809,57.87525006)(249.94366558,57.66691675)(249.94366558,57.42212512)
\curveto(249.94366558,57.18254181)(249.85772809,56.97681267)(249.68585311,56.80493769)
\curveto(249.51397813,56.63306272)(249.30564482,56.54712523)(249.06085318,56.54712523)
\curveto(248.82126988,56.54712523)(248.61293657,56.63045855)(248.43585326,56.7971252)
\curveto(248.26397828,56.96900017)(248.17804079,57.17733348)(248.17804079,57.42212512)
\curveto(248.17804079,57.66691675)(248.26397828,57.87525006)(248.43585326,58.04712504)
\curveto(248.61293657,58.21900002)(248.82126988,58.30493751)(249.06085318,58.30493751)
\closepath
\moveto(249.7249156,54.81275044)
\lineto(249.87335308,54.71118795)
\curveto(249.81606142,54.02889637)(249.78741559,53.16170898)(249.78741559,52.10962577)
\lineto(249.78741559,50.34400099)
\curveto(249.78741559,50.22941767)(249.79783226,49.85962605)(249.81866559,49.23462613)
\curveto(249.83949892,48.61483454)(249.86033225,48.25285542)(249.88116558,48.14868876)
\curveto(249.90720725,48.04452211)(249.94366558,47.96900128)(249.99054057,47.92212629)
\curveto(250.03741556,47.8752513)(250.09470722,47.8440013)(250.16241555,47.8283763)
\curveto(250.23012387,47.81795964)(250.51137384,47.7971263)(251.00616544,47.76587631)
\lineto(251.07647794,47.70337632)
\lineto(251.07647794,47.37525136)
\lineto(251.01397794,47.30493887)
\curveto(250.36293636,47.34660553)(249.73533227,47.36743886)(249.13116568,47.36743886)
\curveto(248.53220742,47.36743886)(247.90720749,47.34660553)(247.25616591,47.30493887)
\lineto(247.18585342,47.37525136)
\lineto(247.18585342,47.70337632)
\lineto(247.25616591,47.76587631)
\curveto(247.76137418,47.7971263)(248.04522831,47.8205638)(248.1077283,47.8361888)
\curveto(248.17543663,47.8518138)(248.23272829,47.88306379)(248.27960328,47.92993879)
\curveto(248.33168661,47.98202212)(248.36554077,48.05754294)(248.38116577,48.15650126)
\curveto(248.4019991,48.26066791)(248.42283243,48.59660537)(248.44366576,49.16431364)
\curveto(248.46970742,49.73723023)(248.48272826,50.16952185)(248.48272826,50.46118848)
\lineto(248.48272826,52.00025079)
\curveto(248.48272826,52.20858409)(248.47231159,52.49243823)(248.45147826,52.85181318)
\curveto(248.43064493,53.21118814)(248.41241576,53.43254228)(248.39679077,53.5158756)
\curveto(248.3863741,53.59920892)(248.34731161,53.65910475)(248.27960328,53.69556308)
\curveto(248.21189496,53.73722974)(248.07647831,53.75806307)(247.87335333,53.75806307)
\lineto(247.21710341,53.76587557)
\lineto(247.14679092,53.82837556)
\lineto(247.14679092,54.16431302)
\lineto(247.20929091,54.22681301)
\curveto(248.20408246,54.34660466)(249.04262402,54.54191714)(249.7249156,54.81275044)
\closepath
\moveto(249.11554068,46.89868892)
\closepath
}
}
{
\newrgbcolor{curcolor}{0 0 0}
\pscustom[linestyle=none,fillstyle=solid,fillcolor=curcolor]
{
\newpath
\moveto(254.12335256,58.93774993)
\lineto(254.27179004,58.84399994)
\curveto(254.21449838,58.24504168)(254.18585255,57.044521)(254.18585255,55.24243789)
\lineto(254.18585255,53.31275062)
\curveto(254.6233525,53.67212558)(255.02960245,54.02889637)(255.4046024,54.38306299)
\curveto(255.51397739,54.48202131)(255.61814404,54.55493797)(255.71710236,54.60181297)
\curveto(255.81606068,54.64868796)(255.97491483,54.69816712)(256.1936648,54.75025045)
\curveto(256.41762311,54.80233377)(256.64678975,54.82837544)(256.88116472,54.82837544)
\curveto(257.276998,54.82837544)(257.65981046,54.74764628)(258.02960208,54.58618797)
\curveto(258.40460203,54.42472965)(258.685852,54.23202134)(258.87335197,54.00806304)
\curveto(259.06606028,53.78931307)(259.19887277,53.52889643)(259.27178942,53.22681314)
\curveto(259.34470608,52.92472984)(259.38116441,52.55233405)(259.38116441,52.10962577)
\lineto(259.38116441,50.73462594)
\curveto(259.38116441,50.64608429)(259.39418524,49.99764687)(259.42022691,48.78931368)
\curveto(259.4358519,48.29452208)(259.4905394,48.00025128)(259.58428939,47.90650129)
\curveto(259.68324771,47.8127513)(260.00616433,47.76587631)(260.55303927,47.76587631)
\lineto(260.61553926,47.70337632)
\lineto(260.61553926,47.36743886)
\lineto(260.55303927,47.30493887)
\curveto(259.89158102,47.34660553)(259.4436644,47.36743886)(259.20928943,47.36743886)
\curveto(259.08949778,47.36743886)(258.70928949,47.34660553)(258.06866457,47.30493887)
\lineto(257.97491458,47.39087636)
\curveto(258.04262291,48.0679596)(258.07647707,48.950772)(258.07647707,50.03931353)
\lineto(258.07647707,51.0627509)
\curveto(258.07647707,51.67733416)(258.06085207,52.12004244)(258.02960208,52.39087574)
\curveto(258.00356041,52.66170904)(257.91762293,52.90650067)(257.77178961,53.12525065)
\curveto(257.62595629,53.34920895)(257.43064382,53.52108393)(257.18585218,53.64087558)
\curveto(256.94626888,53.76587557)(256.65460225,53.82837556)(256.31085229,53.82837556)
\curveto(255.95668567,53.82837556)(255.6598107,53.7788964)(255.4202274,53.67993808)
\curveto(255.18585243,53.58618809)(254.94626912,53.42472978)(254.70147749,53.19556314)
\curveto(254.45668585,52.9663965)(254.30824837,52.77368819)(254.25616504,52.61743821)
\curveto(254.20929005,52.46639656)(254.18585255,52.17212577)(254.18585255,51.73462582)
\lineto(254.18585255,50.34400099)
\curveto(254.18585255,50.22941767)(254.19626922,49.85962605)(254.21710255,49.23462613)
\curveto(254.23793588,48.61483454)(254.25876921,48.25285542)(254.27960254,48.14868876)
\curveto(254.3056442,48.04452211)(254.34210253,47.96900128)(254.38897753,47.92212629)
\curveto(254.43585252,47.8752513)(254.49314418,47.8440013)(254.56085251,47.8283763)
\curveto(254.62856083,47.81795964)(254.9098108,47.7971263)(255.4046024,47.76587631)
\lineto(255.47491489,47.70337632)
\lineto(255.47491489,47.37525136)
\lineto(255.4124149,47.30493887)
\curveto(254.76137331,47.34660553)(254.13376923,47.36743886)(253.52960263,47.36743886)
\curveto(252.93064437,47.36743886)(252.30564445,47.34660553)(251.65460286,47.30493887)
\lineto(251.58429037,47.37525136)
\lineto(251.58429037,47.70337632)
\lineto(251.65460286,47.76587631)
\curveto(252.15981114,47.7971263)(252.44366527,47.8205638)(252.50616526,47.8361888)
\curveto(252.57387358,47.8518138)(252.63116524,47.88306379)(252.67804024,47.92993879)
\curveto(252.73012356,47.98202212)(252.76397773,48.05754294)(252.77960273,48.15650126)
\curveto(252.80043606,48.26066791)(252.82126939,48.59660537)(252.84210272,49.16431364)
\curveto(252.86814438,49.73723023)(252.88116521,50.16952185)(252.88116521,50.46118848)
\lineto(252.88116521,54.50806298)
\lineto(252.84991522,56.16431277)
\curveto(252.83429022,56.7424377)(252.81606105,57.14347932)(252.79522772,57.36743762)
\curveto(252.77960273,57.59139593)(252.75616523,57.72420841)(252.72491523,57.76587508)
\curveto(252.69366524,57.80754174)(252.63897774,57.83879173)(252.56085275,57.85962506)
\curveto(252.48272776,57.88045839)(252.16762363,57.89087506)(251.61554037,57.89087506)
\lineto(251.54522788,57.96118755)
\lineto(251.54522788,58.28931251)
\lineto(251.60772787,58.359625)
\curveto(252.60251941,58.47420832)(253.44106098,58.66691663)(254.12335256,58.93774993)
\closepath
}
}
{
\newrgbcolor{curcolor}{0 0 0}
\pscustom[linestyle=none,fillstyle=solid,fillcolor=curcolor]
{
\newpath
\moveto(267.68585089,48.51587622)
\lineto(267.43585092,47.95337629)
\curveto(266.89418432,47.604418)(266.40980938,47.37785552)(265.9827261,47.27368887)
\curveto(265.56085115,47.16952222)(265.18324703,47.11743889)(264.84991374,47.11743889)
\curveto(264.22491381,47.11743889)(263.63116389,47.23983471)(263.06866396,47.48462634)
\curveto(262.51137236,47.72941798)(262.05564325,48.14348043)(261.70147663,48.72681369)
\curveto(261.35251833,49.31014695)(261.17803919,50.01327187)(261.17803919,50.83618843)
\curveto(261.17803919,51.38306336)(261.24574751,51.8752508)(261.38116416,52.31275075)
\curveto(261.51658081,52.75545903)(261.6572058,53.08358399)(261.80303911,53.29712563)
\curveto(261.95408076,53.51066727)(262.2066849,53.7476464)(262.56085152,54.00806304)
\curveto(262.91501814,54.26847967)(263.2900181,54.47681298)(263.68585138,54.63306296)
\curveto(264.08168467,54.78931294)(264.50876795,54.86743793)(264.96710122,54.86743793)
\curveto(265.59210115,54.86743793)(266.14158024,54.71900045)(266.61553852,54.42212549)
\curveto(267.09470513,54.13045886)(267.43064259,53.7554589)(267.6233509,53.29712563)
\curveto(267.8160592,52.83879235)(267.91241336,52.35181324)(267.91241336,51.83618831)
\curveto(267.91241336,51.67472999)(267.90460086,51.51848001)(267.88897586,51.36743836)
\lineto(267.80303837,51.28150088)
\curveto(267.44887175,51.20337588)(266.97230931,51.15129256)(266.37335105,51.12525089)
\curveto(265.77439279,51.09920923)(265.37855951,51.0861884)(265.1858512,51.0861884)
\lineto(262.678039,51.0861884)
\curveto(262.68845567,50.00806353)(262.95928897,49.2137928)(263.4905389,48.70337619)
\curveto(264.02178884,48.19295959)(264.67283043,47.93775129)(265.44366366,47.93775129)
\curveto(265.80824695,47.93775129)(266.15720524,48.00025128)(266.49053853,48.12525126)
\curveto(266.82908016,48.25025125)(267.18585095,48.42473039)(267.5608509,48.6486887)
\closepath
\moveto(262.678039,51.71118832)
\curveto(262.77178899,51.69556332)(263.13116395,51.67733416)(263.75616387,51.65650083)
\curveto(264.38637213,51.6356675)(264.8525179,51.62525083)(265.1546012,51.62525083)
\curveto(265.87855944,51.62525083)(266.31866356,51.63827166)(266.47491354,51.66431333)
\curveto(266.48012187,51.78931331)(266.48272604,51.88566747)(266.48272604,51.95337579)
\curveto(266.48272604,52.76066736)(266.31866356,53.35962562)(265.9905386,53.75025057)
\curveto(265.66241364,54.14608386)(265.21449703,54.3440005)(264.64678876,54.3440005)
\curveto(264.02699717,54.3440005)(263.54262223,54.12264636)(263.19366394,53.67993808)
\curveto(262.84991398,53.2372298)(262.678039,52.58097988)(262.678039,51.71118832)
\closepath
\moveto(264.85772624,55.30493788)
\closepath
\moveto(264.76397625,46.89868892)
\closepath
}
}
{
\newrgbcolor{curcolor}{0 0 0}
\pscustom[linestyle=none,fillstyle=solid,fillcolor=curcolor]
{
}
}
{
\newrgbcolor{curcolor}{0 0 0}
\pscustom[linestyle=none,fillstyle=solid,fillcolor=curcolor]
{
\newpath
\moveto(281.06084924,54.27368801)
\lineto(281.12334923,54.13306302)
\curveto(280.90980759,53.80493806)(280.77699511,53.59139642)(280.72491178,53.4924381)
\lineto(279.27178696,53.4924381)
\curveto(279.41762027,53.2059798)(279.49053693,52.90650067)(279.49053693,52.59400071)
\curveto(279.49053693,52.22420909)(279.40459944,51.86483414)(279.23272446,51.51587585)
\curveto(279.06605782,51.16691756)(278.83168285,50.87004259)(278.52959955,50.62525096)
\curveto(278.22751625,50.38045932)(277.8837663,50.18514684)(277.49834968,50.03931353)
\curveto(277.11814139,49.89348021)(276.68324561,49.82056356)(276.19366234,49.82056356)
\lineto(275.83428738,49.82056356)
\curveto(275.52699575,49.57577192)(275.32907911,49.39348027)(275.24053746,49.27368862)
\curveto(275.1519958,49.15389697)(275.10772497,49.03150115)(275.10772497,48.90650117)
\curveto(275.10772497,48.67733453)(275.21189163,48.51587622)(275.42022493,48.42212623)
\curveto(275.63376657,48.32837624)(276.07126652,48.28150125)(276.73272477,48.28150125)
\lineto(278.48272456,48.30493874)
\curveto(279.03480782,48.30493874)(279.45668277,48.23983458)(279.7483494,48.10962627)
\curveto(280.04522436,47.97941795)(280.28480767,47.74504298)(280.46709931,47.40650135)
\curveto(280.64939096,47.07316806)(280.74053678,46.71900144)(280.74053678,46.34400148)
\curveto(280.74053678,45.77108489)(280.5530368,45.20077246)(280.17803685,44.6330642)
\curveto(279.80824523,44.0601476)(279.27178696,43.62525182)(278.56866205,43.32837686)
\curveto(277.86553713,43.03150189)(277.11553722,42.88306441)(276.31866232,42.88306441)
\curveto(275.86553738,42.88306441)(275.4384541,42.93254357)(275.03741248,43.03150189)
\curveto(274.6415792,43.13046021)(274.29262091,43.28150186)(273.99053761,43.48462684)
\curveto(273.69366265,43.68254348)(273.45407934,43.94296011)(273.2717877,44.26587674)
\curveto(273.08949605,44.58879337)(272.99835023,44.92473083)(272.99835023,45.27368912)
\curveto(272.99835023,45.46639743)(273.02699606,45.6695224)(273.08428772,45.88306404)
\curveto(273.14157938,46.09139735)(273.25095437,46.31795982)(273.41241268,46.56275146)
\lineto(274.82647501,47.35181386)
\curveto(274.37335006,47.49243884)(274.08689176,47.63045966)(273.96710011,47.76587631)
\curveto(273.85251679,47.90650129)(273.79522513,48.07316794)(273.79522513,48.26587625)
\curveto(273.79522513,48.46379289)(273.86032929,48.69816786)(273.99053761,48.96900116)
\lineto(275.24834995,49.87525105)
\curveto(274.50876671,50.06795936)(274.00876677,50.35702182)(273.74835014,50.74243844)
\curveto(273.49314184,51.12785506)(273.36553769,51.54973001)(273.36553769,52.00806329)
\curveto(273.36553769,52.42993823)(273.45668351,52.83097985)(273.63897515,53.21118814)
\curveto(273.8212668,53.59660476)(274.09470426,53.90389639)(274.45928755,54.13306302)
\curveto(274.82907917,54.36222966)(275.24053746,54.54191714)(275.6936624,54.67212546)
\curveto(276.15199568,54.80754211)(276.56866229,54.87525043)(276.94366225,54.87525043)
\curveto(277.59470383,54.87525043)(278.21449542,54.66431296)(278.80303702,54.24243801)
\curveto(279.74574523,54.24243801)(280.49834931,54.25285468)(281.06084924,54.27368801)
\closepath
\moveto(274.70928752,52.46118823)
\curveto(274.70928752,52.14347994)(274.77178751,51.80233414)(274.8967875,51.43775086)
\curveto(275.02178748,51.07316757)(275.22230829,50.79712594)(275.49834992,50.60962596)
\curveto(275.77959989,50.42212598)(276.09470402,50.32837599)(276.44366231,50.32837599)
\curveto(276.90720392,50.32837599)(277.30564137,50.47941764)(277.63897466,50.78150094)
\curveto(277.97751629,51.08358423)(278.1467871,51.55493834)(278.1467871,52.19556326)
\curveto(278.1467871,52.73722986)(277.99314128,53.22420897)(277.68584965,53.65650058)
\curveto(277.37855803,54.0887922)(276.94366225,54.304938)(276.38116232,54.304938)
\curveto(275.90720404,54.304938)(275.50876659,54.14868802)(275.18584996,53.83618806)
\curveto(274.86814167,53.52889643)(274.70928752,53.07056315)(274.70928752,52.46118823)
\closepath
\moveto(276.99834974,47.21900138)
\curveto(276.15980818,47.21900138)(275.67543324,47.20858471)(275.54522492,47.18775138)
\curveto(275.42022493,47.16691805)(275.24053746,47.07837639)(275.00616248,46.92212641)
\curveto(274.77178751,46.76587643)(274.5790792,46.55493896)(274.42803756,46.28931399)
\curveto(274.28220424,46.01848069)(274.20928758,45.71639739)(274.20928758,45.3830641)
\curveto(274.20928758,44.79452251)(274.42803756,44.3153559)(274.8655375,43.94556428)
\curveto(275.30303745,43.57577266)(275.88637071,43.39087685)(276.61553729,43.39087685)
\curveto(277.16241222,43.39087685)(277.66762049,43.50546017)(278.1311621,43.73462681)
\curveto(278.59470371,43.96379344)(278.9358495,44.27108507)(279.15459947,44.65650169)
\curveto(279.37855778,45.04191831)(279.49053693,45.41170993)(279.49053693,45.76587656)
\curveto(279.49053693,46.12004318)(279.39678694,46.42473064)(279.20928697,46.67993894)
\curveto(279.02699532,46.92993891)(278.78741202,47.08358473)(278.49053706,47.14087639)
\curveto(278.19887042,47.19295971)(277.70147465,47.21900138)(276.99834974,47.21900138)
\closepath
}
}
{
\newrgbcolor{curcolor}{0 0 0}
\pscustom[linestyle=none,fillstyle=solid,fillcolor=curcolor]
{
\newpath
\moveto(288.24834835,48.51587622)
\lineto(287.99834838,47.95337629)
\curveto(287.45668178,47.604418)(286.97230684,47.37785552)(286.54522356,47.27368887)
\curveto(286.12334861,47.16952222)(285.74574449,47.11743889)(285.4124112,47.11743889)
\curveto(284.78741128,47.11743889)(284.19366135,47.23983471)(283.63116142,47.48462634)
\curveto(283.07386982,47.72941798)(282.61814071,48.14348043)(282.26397409,48.72681369)
\curveto(281.9150158,49.31014695)(281.74053665,50.01327187)(281.74053665,50.83618843)
\curveto(281.74053665,51.38306336)(281.80824498,51.8752508)(281.94366163,52.31275075)
\curveto(282.07907828,52.75545903)(282.21970326,53.08358399)(282.36553658,53.29712563)
\curveto(282.51657823,53.51066727)(282.76918236,53.7476464)(283.12334898,54.00806304)
\curveto(283.47751561,54.26847967)(283.85251556,54.47681298)(284.24834885,54.63306296)
\curveto(284.64418213,54.78931294)(285.07126541,54.86743793)(285.52959869,54.86743793)
\curveto(286.15459861,54.86743793)(286.70407771,54.71900045)(287.17803598,54.42212549)
\curveto(287.65720259,54.13045886)(287.99314005,53.7554589)(288.18584836,53.29712563)
\curveto(288.37855667,52.83879235)(288.47491082,52.35181324)(288.47491082,51.83618831)
\curveto(288.47491082,51.67472999)(288.46709832,51.51848001)(288.45147333,51.36743836)
\lineto(288.36553584,51.28150088)
\curveto(288.01136921,51.20337588)(287.53480677,51.15129256)(286.93584851,51.12525089)
\curveto(286.33689025,51.09920923)(285.94105697,51.0861884)(285.74834866,51.0861884)
\lineto(283.24053647,51.0861884)
\curveto(283.25095313,50.00806353)(283.52178643,49.2137928)(284.05303637,48.70337619)
\curveto(284.5842863,48.19295959)(285.23532789,47.93775129)(286.00616113,47.93775129)
\curveto(286.37074442,47.93775129)(286.71970271,48.00025128)(287.053036,48.12525126)
\curveto(287.39157762,48.25025125)(287.74834841,48.42473039)(288.12334837,48.6486887)
\closepath
\moveto(283.24053647,51.71118832)
\curveto(283.33428646,51.69556332)(283.69366141,51.67733416)(284.31866134,51.65650083)
\curveto(284.94886959,51.6356675)(285.41501537,51.62525083)(285.71709866,51.62525083)
\curveto(286.44105691,51.62525083)(286.88116102,51.63827166)(287.037411,51.66431333)
\curveto(287.04261933,51.78931331)(287.0452235,51.88566747)(287.0452235,51.95337579)
\curveto(287.0452235,52.76066736)(286.88116102,53.35962562)(286.55303606,53.75025057)
\curveto(286.2249111,54.14608386)(285.77699449,54.3440005)(285.20928623,54.3440005)
\curveto(284.58949464,54.3440005)(284.1051197,54.12264636)(283.75616141,53.67993808)
\curveto(283.41241145,53.2372298)(283.24053647,52.58097988)(283.24053647,51.71118832)
\closepath
\moveto(285.4202237,55.30493788)
\closepath
\moveto(285.32647371,46.89868892)
\closepath
}
}
{
\newrgbcolor{curcolor}{0 0 0}
\pscustom[linestyle=none,fillstyle=solid,fillcolor=curcolor]
{
\newpath
\moveto(291.8811604,54.81275044)
\lineto(292.02959789,54.71118795)
\curveto(291.99834789,54.36743799)(291.97751456,53.95597971)(291.96709789,53.4768131)
\lineto(292.7092853,54.15650052)
\curveto(292.91761861,54.34920883)(293.05563943,54.47160465)(293.12334775,54.52368798)
\curveto(293.19105608,54.58097964)(293.34991022,54.64608379)(293.59991019,54.71900045)
\curveto(293.84991016,54.79191711)(294.11293096,54.82837544)(294.38897259,54.82837544)
\curveto(294.90980586,54.82837544)(295.36553497,54.69816712)(295.75615993,54.43775049)
\curveto(296.14678488,54.18254218)(296.43584734,53.82837556)(296.62334732,53.37525062)
\curveto(297.32647223,54.0419172)(297.73532635,54.41170882)(297.84990967,54.48462548)
\curveto(297.96970132,54.55754214)(298.1572013,54.6304588)(298.4124096,54.70337545)
\curveto(298.6676179,54.78150044)(298.9228262,54.82056294)(299.1780345,54.82056294)
\curveto(299.58949279,54.82056294)(299.96709691,54.73202128)(300.31084686,54.55493797)
\curveto(300.65459682,54.37785466)(300.92803429,54.14608386)(301.13115926,53.85962556)
\curveto(301.33428424,53.57316726)(301.45147172,53.26327146)(301.48272172,52.92993817)
\curveto(301.51918005,52.59660488)(301.53740921,52.1382716)(301.53740921,51.55493834)
\lineto(301.53740921,50.73462594)
\curveto(301.53740921,50.64608429)(301.55303421,49.99764687)(301.58428421,48.78931368)
\curveto(301.59470087,48.29452208)(301.64938837,48.00025128)(301.74834669,47.90650129)
\curveto(301.84730501,47.8127513)(302.16761747,47.76587631)(302.70928407,47.76587631)
\lineto(302.77178406,47.70337632)
\lineto(302.77178406,47.36743886)
\lineto(302.70928407,47.30493887)
\curveto(302.04782582,47.34660553)(301.59990921,47.36743886)(301.36553423,47.36743886)
\curveto(301.23532592,47.36743886)(300.8577218,47.34660553)(300.23272187,47.30493887)
\lineto(300.13115939,47.39087636)
\curveto(300.19886771,48.0679596)(300.23272187,48.950772)(300.23272187,50.03931353)
\lineto(300.23272187,50.97681341)
\curveto(300.23272187,51.80493831)(300.19365938,52.38306324)(300.11553439,52.7111882)
\curveto(300.0374094,53.04452149)(299.85511775,53.31535479)(299.56865946,53.5236881)
\curveto(299.28220116,53.73722974)(298.94105537,53.84400056)(298.54522208,53.84400056)
\curveto(298.26397212,53.84400056)(298.00095132,53.7867089)(297.75615968,53.67212558)
\curveto(297.51136804,53.56275059)(297.2900139,53.39608395)(297.09209726,53.17212564)
\curveto(296.89938895,52.95337567)(296.79261813,52.74764653)(296.7717848,52.55493822)
\curveto(296.75095147,52.36743824)(296.7405348,52.01847995)(296.7405348,51.50806335)
\lineto(296.7405348,50.53150097)
\curveto(296.7405348,50.30233433)(296.75095147,49.87264688)(296.7717848,49.24243863)
\curveto(296.79261813,48.61223037)(296.81345146,48.24504292)(296.83428479,48.14087626)
\curveto(296.86032646,48.03670961)(296.89678479,47.96118878)(296.94365978,47.91431379)
\curveto(296.99574311,47.87264713)(297.05303477,47.8440013)(297.11553476,47.8283763)
\curveto(297.18324308,47.81795964)(297.46449305,47.7971263)(297.95928465,47.76587631)
\lineto(298.02959715,47.70337632)
\lineto(298.02959715,47.37525136)
\lineto(297.96709715,47.30493887)
\curveto(297.31605557,47.34660553)(296.68845148,47.36743886)(296.08428489,47.36743886)
\curveto(295.52699329,47.36743886)(294.90199336,47.34660553)(294.20928512,47.30493887)
\lineto(294.13897263,47.37525136)
\lineto(294.13897263,47.70337632)
\lineto(294.20928512,47.76587631)
\curveto(294.71449339,47.7971263)(294.99834752,47.8205638)(295.06084751,47.8361888)
\curveto(295.12855584,47.8518138)(295.1858475,47.88306379)(295.23272249,47.92993879)
\curveto(295.28480582,47.98202212)(295.31865998,48.05754294)(295.33428498,48.15650126)
\curveto(295.35511831,48.26066791)(295.37595164,48.59660537)(295.39678497,49.16431364)
\curveto(295.42282663,49.73723023)(295.43584747,50.16952185)(295.43584747,50.46118848)
\lineto(295.43584747,51.35181337)
\curveto(295.43584747,51.94035496)(295.3941808,52.3934799)(295.31084748,52.7111882)
\curveto(295.23272249,53.02889649)(295.06084751,53.29712563)(294.79522254,53.5158756)
\curveto(294.52959758,53.73462557)(294.19626429,53.84400056)(293.79522267,53.84400056)
\curveto(293.48272271,53.84400056)(293.19626441,53.78150057)(292.93584777,53.65650058)
\curveto(292.68063947,53.53670893)(292.469702,53.38566728)(292.30303535,53.20337564)
\curveto(292.14157704,53.02629233)(292.04001455,52.85962568)(291.99834789,52.7033757)
\curveto(291.96188956,52.54712572)(291.9436604,52.20858409)(291.9436604,51.68775083)
\lineto(291.9436604,50.53150097)
\curveto(291.9436604,50.30233433)(291.95407706,49.87264688)(291.97491039,49.24243863)
\curveto(291.99574372,48.61223037)(292.01657705,48.24504292)(292.03741038,48.14087626)
\curveto(292.06345205,48.03670961)(292.09991038,47.96118878)(292.14678537,47.91431379)
\curveto(292.1988687,47.87264713)(292.25616036,47.8440013)(292.31866035,47.8283763)
\curveto(292.38636868,47.81795964)(292.66761864,47.7971263)(293.16241025,47.76587631)
\lineto(293.23272274,47.70337632)
\lineto(293.23272274,47.37525136)
\lineto(293.17022275,47.30493887)
\curveto(292.51918116,47.34660553)(291.89157707,47.36743886)(291.28741048,47.36743886)
\curveto(290.68845222,47.36743886)(290.06345229,47.34660553)(289.41241071,47.30493887)
\lineto(289.34209822,47.37525136)
\lineto(289.34209822,47.70337632)
\lineto(289.41241071,47.76587631)
\curveto(289.91761898,47.7971263)(290.20147311,47.8205638)(290.2639731,47.8361888)
\curveto(290.33168143,47.8518138)(290.38897309,47.88306379)(290.43584808,47.92993879)
\curveto(290.48793141,47.98202212)(290.52178557,48.05754294)(290.53741057,48.15650126)
\curveto(290.5582439,48.26066791)(290.57907723,48.59660537)(290.59991056,49.16431364)
\curveto(290.62595223,49.73723023)(290.63897306,50.16952185)(290.63897306,50.46118848)
\lineto(290.63897306,52.00025079)
\curveto(290.63897306,52.20858409)(290.62855639,52.49243823)(290.60772306,52.85181318)
\curveto(290.58688973,53.21118814)(290.56866057,53.43254228)(290.55303557,53.5158756)
\curveto(290.5426189,53.59920892)(290.50355641,53.65910475)(290.43584808,53.69556308)
\curveto(290.36813976,53.73722974)(290.23272311,53.75806307)(290.02959813,53.75806307)
\lineto(289.37334821,53.76587557)
\lineto(289.30303572,53.82837556)
\lineto(289.30303572,54.16431302)
\lineto(289.36553571,54.22681301)
\curveto(290.36032726,54.34660466)(291.19886882,54.54191714)(291.8811604,54.81275044)
\closepath
}
}
{
\newrgbcolor{curcolor}{0 0 0}
\pscustom[linestyle=none,fillstyle=solid,fillcolor=curcolor]
{
\newpath
\moveto(304.78740881,52.6643132)
\lineto(304.48272135,52.7424382)
\lineto(304.42022136,52.82056319)
\lineto(304.42022136,53.78931307)
\curveto(305.34730458,54.47681298)(306.24834613,54.82056294)(307.12334602,54.82056294)
\curveto(307.72230428,54.82056294)(308.21970006,54.70858379)(308.61553334,54.48462548)
\curveto(309.01136663,54.26066717)(309.29782492,53.97941721)(309.47490823,53.64087558)
\curveto(309.65199155,53.30754229)(309.7405332,52.91691734)(309.7405332,52.46900073)
\lineto(309.70147071,50.89868842)
\lineto(309.70147071,48.57056371)
\curveto(309.70147071,48.25285542)(309.72230404,48.06014711)(309.7639707,47.99243878)
\curveto(309.81084569,47.92473046)(309.86292902,47.87785546)(309.92022068,47.8518138)
\curveto(309.97751234,47.83098047)(310.08428316,47.8127513)(310.24053314,47.7971263)
\lineto(310.68584559,47.75806381)
\lineto(310.74834558,47.68775132)
\lineto(310.74834558,47.37525136)
\lineto(310.68584559,47.31275136)
\curveto(310.3056373,47.34400136)(309.95147068,47.35962636)(309.62334572,47.35962636)
\curveto(309.31084575,47.35962636)(308.9358458,47.34400136)(308.49834586,47.31275136)
\lineto(308.38115837,47.42212635)
\lineto(308.41240837,48.6799387)
\lineto(306.70928358,47.35181386)
\curveto(306.42282528,47.23202221)(306.11032532,47.17212638)(305.77178369,47.17212638)
\curveto(305.35511708,47.17212638)(304.99574212,47.24764721)(304.69365882,47.39868885)
\curveto(304.39678386,47.5497305)(304.16761722,47.76066798)(304.00615891,48.03150128)
\curveto(303.84990893,48.30233458)(303.77178394,48.63045954)(303.77178394,49.01587615)
\curveto(303.77178394,49.78150106)(304.01136724,50.37785515)(304.49053385,50.80493843)
\curveto(304.96970046,51.23723005)(306.27699196,51.6044175)(308.41240837,51.9065008)
\curveto(308.41240837,52.67733404)(308.24053339,53.21900064)(307.89678343,53.5315006)
\curveto(307.55303347,53.84920889)(307.09209603,54.00806304)(306.5139711,54.00806304)
\curveto(306.2118878,54.00806304)(305.93584617,53.96379221)(305.6858462,53.87525056)
\curveto(305.44105457,53.7867089)(305.30042958,53.71379224)(305.26397125,53.65650058)
\curveto(305.22751293,53.60441726)(305.08688794,53.28670896)(304.84209631,52.7033757)
\closepath
\moveto(308.41240837,51.42993836)
\curveto(306.95407521,51.18514672)(306.04522116,50.92212592)(305.6858462,50.64087595)
\curveto(305.32647125,50.35962599)(305.14678377,49.92473021)(305.14678377,49.33618862)
\curveto(305.14678377,48.52368872)(305.55042955,48.11743877)(306.35772112,48.11743877)
\curveto(307.05042937,48.11743877)(307.73532512,48.51848038)(308.41240837,49.32056362)
\closepath
\moveto(307.08428353,55.30493788)
\closepath
\moveto(306.92022105,46.89868892)
\closepath
\moveto(308.73272083,56.45337524)
\curveto(308.49834586,56.45337524)(308.29782505,56.53670856)(308.1311584,56.70337521)
\curveto(307.96449175,56.87004185)(307.88115843,57.07056266)(307.88115843,57.30493763)
\curveto(307.88115843,57.54452094)(307.96188759,57.74504174)(308.1233459,57.90650006)
\curveto(308.29001255,58.0731667)(308.49313752,58.15650003)(308.73272083,58.15650003)
\curveto(308.97230413,58.15650003)(309.17282494,58.0731667)(309.33428325,57.90650006)
\curveto(309.5009499,57.74504174)(309.58428322,57.54452094)(309.58428322,57.30493763)
\curveto(309.58428322,57.07056266)(309.5009499,56.87004185)(309.33428325,56.70337521)
\curveto(309.16761661,56.53670856)(308.9670958,56.45337524)(308.73272083,56.45337524)
\closepath
\moveto(305.42803373,56.45337524)
\curveto(305.19365876,56.45337524)(304.99313795,56.53670856)(304.82647131,56.70337521)
\curveto(304.65980466,56.87004185)(304.57647134,57.07056266)(304.57647134,57.30493763)
\curveto(304.57647134,57.5393126)(304.65980466,57.73983341)(304.82647131,57.90650006)
\curveto(304.99313795,58.0731667)(305.19365876,58.15650003)(305.42803373,58.15650003)
\curveto(305.66761704,58.15650003)(305.86813785,58.0731667)(306.02959616,57.90650006)
\curveto(306.19626281,57.74504174)(306.27959613,57.54452094)(306.27959613,57.30493763)
\curveto(306.27959613,57.07056266)(306.19626281,56.87004185)(306.02959616,56.70337521)
\curveto(305.86292951,56.53670856)(305.6624087,56.45337524)(305.42803373,56.45337524)
\closepath
}
}
{
\newrgbcolor{curcolor}{0 0 0}
\pscustom[linestyle=none,fillstyle=solid,fillcolor=curcolor]
{
\newpath
\moveto(311.62334547,53.5080631)
\lineto(311.62334547,53.68775058)
\lineto(311.68584546,53.76587557)
\lineto(312.62334535,54.25806301)
\lineto(312.62334535,54.82837544)
\curveto(312.62334535,55.37525037)(312.67542867,55.80493782)(312.77959533,56.11743778)
\curveto(312.88897031,56.43514607)(313.07386612,56.75545853)(313.33428276,57.07837516)
\curveto(313.59990773,57.40129179)(313.91761602,57.72160425)(314.28740764,58.03931254)
\curveto(314.6624076,58.35702084)(315.01657422,58.58879164)(315.34990751,58.73462496)
\curveto(315.68844914,58.88045827)(316.07126159,58.95337493)(316.49834487,58.95337493)
\curveto(317.00355314,58.95337493)(317.45407392,58.85441661)(317.8499072,58.65649997)
\curveto(318.24574049,58.45858332)(318.53480295,58.17212503)(318.7170946,57.79712507)
\curveto(318.90459457,57.42733345)(318.99834456,57.013271)(318.99834456,56.55493773)
\curveto(318.99834456,56.13827111)(318.93063624,55.62525034)(318.79521959,55.01587541)
\curveto(318.66501127,54.40650049)(318.54001128,53.93254222)(318.42021963,53.59400059)
\lineto(318.30303215,53.5393131)
\curveto(317.85511553,53.78410473)(317.47230308,53.90650055)(317.15459479,53.90650055)
\curveto(316.90459482,53.90650055)(316.70407401,53.83097973)(316.55303236,53.67993808)
\curveto(316.40719905,53.53410476)(316.33428239,53.35962562)(316.33428239,53.15650064)
\curveto(316.33428239,52.942959)(316.40719905,52.76327153)(316.55303236,52.61743821)
\curveto(316.70407401,52.47681323)(317.0842823,52.23462576)(317.69365722,51.8908758)
\curveto(318.18324049,51.61483417)(318.52699045,51.40389669)(318.72490709,51.25806338)
\curveto(318.92282374,51.11223006)(319.07907372,50.92733425)(319.19365704,50.70337595)
\curveto(319.30824036,50.48462597)(319.36553202,50.21900101)(319.36553202,49.90650104)
\curveto(319.36553202,49.54712609)(319.30042786,49.2137928)(319.17021954,48.90650117)
\curveto(319.04521955,48.60441787)(318.86813624,48.3544179)(318.6389696,48.15650126)
\curveto(318.09209467,47.67733465)(317.74053222,47.40389719)(317.58428223,47.33618886)
\curveto(317.27178227,47.20598054)(316.93324065,47.14087639)(316.56865736,47.14087639)
\curveto(316.07386575,47.14087639)(315.62334498,47.26066804)(315.21709503,47.50025134)
\lineto(315.21709503,47.37525136)
\lineto(315.15459503,47.30493887)
\curveto(314.50876178,47.34660553)(313.88376186,47.36743886)(313.27959527,47.36743886)
\curveto(312.68063701,47.36743886)(312.05303292,47.34660553)(311.396783,47.30493887)
\lineto(311.33428301,47.37525136)
\lineto(311.33428301,47.70337632)
\lineto(311.396783,47.76587631)
\lineto(311.97490793,47.7971263)
\curveto(312.14678291,47.80754297)(312.26136622,47.82577213)(312.31865788,47.8518138)
\curveto(312.37594954,47.88306379)(312.42022037,47.92212629)(312.45147037,47.96900128)
\curveto(312.4879287,48.01587628)(312.51657453,48.09660543)(312.53740786,48.21118875)
\curveto(312.55824119,48.33098041)(312.57647035,48.70858453)(312.59209535,49.34400111)
\curveto(312.61292868,49.98462604)(312.62334535,50.35702182)(312.62334535,50.46118848)
\lineto(312.62334535,53.44556311)
\lineto(311.69365796,53.44556311)
\closepath
\moveto(315.83428245,49.09400115)
\lineto(315.90459494,49.03150115)
\lineto(315.90459494,48.35962624)
\curveto(316.27438656,48.02108461)(316.66240735,47.8518138)(317.0686573,47.8518138)
\curveto(317.40719892,47.8518138)(317.70407389,47.96639712)(317.95928219,48.19556376)
\curveto(318.21449049,48.42993873)(318.34209464,48.73202202)(318.34209464,49.10181364)
\curveto(318.34209464,49.33098028)(318.29001131,49.53150109)(318.18584466,49.70337607)
\curveto(318.08688634,49.87525105)(317.93584469,50.03931353)(317.73271972,50.19556351)
\curveto(317.52959474,50.35181349)(317.08167813,50.65389679)(316.38896988,51.1018134)
\curveto(316.06084492,51.31535504)(315.83949078,51.48462585)(315.72490746,51.60962583)
\curveto(315.61032414,51.73983415)(315.51657416,51.87785497)(315.4436575,52.02368828)
\curveto(315.37594917,52.1695216)(315.34209501,52.33879241)(315.34209501,52.53150072)
\curveto(315.34209501,52.70858403)(315.36032418,52.84920902)(315.3967825,52.95337567)
\curveto(315.43844917,53.05754232)(315.50615749,53.16691731)(315.59990748,53.28150063)
\lineto(316.44365738,54.3283755)
\curveto(316.50615737,54.40650049)(316.58949069,54.46639632)(316.69365734,54.50806298)
\curveto(316.80303233,54.55493797)(316.96188648,54.5913963)(317.17021979,54.61743796)
\curveto(317.37855309,54.64868796)(317.54782391,54.66431296)(317.67803222,54.66431296)
\curveto(317.71969888,54.66431296)(317.80563637,54.66170879)(317.93584469,54.65650046)
\curveto(318.05563634,55.13566707)(318.11553217,55.57577118)(318.11553217,55.9768128)
\curveto(318.11553217,56.66431271)(317.92282386,57.21118764)(317.53740724,57.61743759)
\curveto(317.15199062,58.02889588)(316.65199068,58.23462502)(316.03740743,58.23462502)
\curveto(315.58428248,58.23462502)(315.1884492,58.10702087)(314.84990757,57.85181257)
\curveto(314.51136595,57.59660426)(314.27178264,57.25806264)(314.13115766,56.83618769)
\curveto(313.99574101,56.41431274)(313.92803269,55.77889615)(313.92803269,54.92993793)
\lineto(313.92803269,50.34400099)
\lineto(313.98272018,48.6096262)
\curveto(313.99313684,48.28670958)(314.02699101,48.08098044)(314.08428267,47.99243878)
\curveto(314.14157433,47.90389713)(314.21188682,47.84920963)(314.29522014,47.8283763)
\curveto(314.37855346,47.8127513)(314.64157426,47.79452214)(315.08428254,47.77368881)
\curveto(315.23011586,48.04973044)(315.36032418,48.46900122)(315.47490749,49.03150115)
\lineto(315.54521999,49.09400115)
\closepath
}
}
{
\newrgbcolor{curcolor}{0 0 0}
\pscustom[linestyle=none,fillstyle=solid,fillcolor=curcolor]
{
}
}
{
\newrgbcolor{curcolor}{0 0 0}
\pscustom[linestyle=none,fillstyle=solid,fillcolor=curcolor]
{
\newpath
\moveto(329.40459328,47.22681388)
\curveto(329.32646829,47.45598051)(329.00094749,48.33098041)(328.4280309,49.85181355)
\lineto(325.77178123,56.42212524)
\curveto(325.46969793,57.17212515)(325.27959379,57.59660426)(325.2014688,57.69556258)
\curveto(325.12334381,57.79972924)(324.81084384,57.87004173)(324.26396891,57.90650006)
\lineto(324.19365642,57.97681255)
\lineto(324.19365642,58.33618751)
\lineto(324.26396891,58.4065)
\curveto(325.06084381,58.36483334)(325.73792706,58.344)(326.29521866,58.344)
\curveto(326.81605193,58.344)(327.50615601,58.36483334)(328.36553091,58.4065)
\lineto(328.4280309,58.344)
\lineto(328.4280309,57.91431256)
\lineto(328.36553091,57.85181257)
\curveto(327.73532265,57.85181257)(327.35771853,57.83097923)(327.23271855,57.78931257)
\curveto(327.11292689,57.75285424)(327.05303107,57.67993759)(327.05303107,57.5705626)
\curveto(327.05303107,57.50806261)(327.13636439,57.22941681)(327.30303104,56.7346252)
\curveto(327.46969768,56.24504193)(327.61813516,55.83097948)(327.74834348,55.49243786)
\lineto(330.19365568,49.31275112)
\lineto(332.23271793,54.43775049)
\curveto(332.37334291,54.78670878)(332.57907205,55.34920871)(332.84990535,56.12525028)
\curveto(333.12073865,56.90129185)(333.2561553,57.36743762)(333.2561553,57.52368761)
\curveto(333.2561553,57.65389592)(333.19105114,57.73722925)(333.06084283,57.77368757)
\curveto(332.93584284,57.8101459)(332.55563456,57.85441673)(331.92021797,57.90650006)
\lineto(331.84990548,57.96900005)
\lineto(331.84990548,58.344)
\lineto(331.92021797,58.4065)
\curveto(332.83167619,58.36483334)(333.41240528,58.344)(333.66240525,58.344)
\curveto(333.86553023,58.344)(334.39157183,58.36483334)(335.24053006,58.4065)
\lineto(335.30303005,58.344)
\lineto(335.30303005,57.96900005)
\lineto(335.24053006,57.90650006)
\curveto(334.8655301,57.89087506)(334.63896763,57.86222923)(334.56084264,57.82056257)
\curveto(334.48271765,57.77889591)(334.41240516,57.69816675)(334.34990517,57.5783751)
\curveto(334.29261351,57.45858345)(334.1077177,57.04712516)(333.79521774,56.34400025)
\lineto(331.35771804,50.34400099)
\curveto(330.8108431,49.00545949)(330.42282232,47.96639712)(330.19365568,47.22681388)
\closepath
}
}
{
\newrgbcolor{curcolor}{0 0 0}
\pscustom[linestyle=none,fillstyle=solid,fillcolor=curcolor]
{
\newpath
\moveto(341.3577168,48.51587622)
\lineto(341.10771683,47.95337629)
\curveto(340.56605023,47.604418)(340.08167529,47.37785552)(339.65459201,47.27368887)
\curveto(339.23271707,47.16952222)(338.85511295,47.11743889)(338.52177965,47.11743889)
\curveto(337.89677973,47.11743889)(337.3030298,47.23983471)(336.74052987,47.48462634)
\curveto(336.18323827,47.72941798)(335.72750916,48.14348043)(335.37334254,48.72681369)
\curveto(335.02438425,49.31014695)(334.84990511,50.01327187)(334.84990511,50.83618843)
\curveto(334.84990511,51.38306336)(334.91761343,51.8752508)(335.05303008,52.31275075)
\curveto(335.18844673,52.75545903)(335.32907171,53.08358399)(335.47490503,53.29712563)
\curveto(335.62594668,53.51066727)(335.87855081,53.7476464)(336.23271744,54.00806304)
\curveto(336.58688406,54.26847967)(336.96188401,54.47681298)(337.3577173,54.63306296)
\curveto(337.75355058,54.78931294)(338.18063386,54.86743793)(338.63896714,54.86743793)
\curveto(339.26396706,54.86743793)(339.81344616,54.71900045)(340.28740444,54.42212549)
\curveto(340.76657104,54.13045886)(341.1025085,53.7554589)(341.29521681,53.29712563)
\curveto(341.48792512,52.83879235)(341.58427928,52.35181324)(341.58427928,51.83618831)
\curveto(341.58427928,51.67472999)(341.57646678,51.51848001)(341.56084178,51.36743836)
\lineto(341.47490429,51.28150088)
\curveto(341.12073767,51.20337588)(340.64417522,51.15129256)(340.04521697,51.12525089)
\curveto(339.44625871,51.09920923)(339.05042542,51.0861884)(338.85771711,51.0861884)
\lineto(336.34990492,51.0861884)
\curveto(336.36032159,50.00806353)(336.63115489,49.2137928)(337.16240482,48.70337619)
\curveto(337.69365476,48.19295959)(338.34469634,47.93775129)(339.11552958,47.93775129)
\curveto(339.48011287,47.93775129)(339.82907116,48.00025128)(340.16240445,48.12525126)
\curveto(340.50094608,48.25025125)(340.85771687,48.42473039)(341.23271682,48.6486887)
\closepath
\moveto(336.34990492,51.71118832)
\curveto(336.44365491,51.69556332)(336.80302986,51.67733416)(337.42802979,51.65650083)
\curveto(338.05823804,51.6356675)(338.52438382,51.62525083)(338.82646712,51.62525083)
\curveto(339.55042536,51.62525083)(339.99052947,51.63827166)(340.14677945,51.66431333)
\curveto(340.15198779,51.78931331)(340.15459195,51.88566747)(340.15459195,51.95337579)
\curveto(340.15459195,52.76066736)(339.99052947,53.35962562)(339.66240451,53.75025057)
\curveto(339.33427955,54.14608386)(338.88636294,54.3440005)(338.31865468,54.3440005)
\curveto(337.69886309,54.3440005)(337.21448815,54.12264636)(336.86552986,53.67993808)
\curveto(336.5217799,53.2372298)(336.34990492,52.58097988)(336.34990492,51.71118832)
\closepath
\moveto(338.52959215,55.30493788)
\closepath
\moveto(338.43584216,46.89868892)
\closepath
}
}
{
\newrgbcolor{curcolor}{0 0 0}
\pscustom[linestyle=none,fillstyle=solid,fillcolor=curcolor]
{
\newpath
\moveto(345.11552884,54.81275044)
\lineto(345.26396632,54.71118795)
\curveto(345.23271633,54.40910466)(345.20927883,53.87264639)(345.19365383,53.10181315)
\lineto(345.77959126,53.84400056)
\curveto(345.97229957,54.0887922)(346.14157038,54.27629217)(346.2874037,54.40650049)
\curveto(346.43844534,54.54191714)(346.61292449,54.64608379)(346.81084113,54.71900045)
\curveto(347.00875777,54.79191711)(347.21188275,54.82837544)(347.42021606,54.82837544)
\curveto(347.64938269,54.82837544)(347.8655285,54.78150044)(348.06865348,54.68775046)
\lineto(348.12334097,54.57837547)
\curveto(348.04000765,53.88566722)(347.99313265,53.2762923)(347.98271599,52.75025069)
\lineto(347.63115353,52.75025069)
\curveto(347.42282022,53.2294173)(347.08688276,53.46900061)(346.62334115,53.46900061)
\curveto(346.30042453,53.46900061)(346.01917456,53.36483395)(345.77959126,53.15650064)
\curveto(345.54000795,52.95337567)(345.37854964,52.6955632)(345.29521632,52.38306324)
\curveto(345.21709133,52.07577161)(345.17802883,51.68514666)(345.17802883,51.21118838)
\lineto(345.17802883,50.34400099)
\curveto(345.17802883,50.18775101)(345.1884455,49.80493856)(345.20927883,49.19556363)
\curveto(345.23011216,48.58618871)(345.25094549,48.23462625)(345.27177882,48.14087626)
\curveto(345.29782048,48.04712627)(345.33427881,47.97681378)(345.38115381,47.92993879)
\curveto(345.43323713,47.88827213)(345.49834129,47.8596263)(345.57646628,47.8440013)
\curveto(345.65979961,47.8283763)(346.03219539,47.80233464)(346.69365365,47.76587631)
\lineto(346.76396614,47.70337632)
\lineto(346.76396614,47.37525136)
\lineto(346.69365365,47.30493887)
\curveto(346.0009454,47.34660553)(345.27698715,47.36743886)(344.52177891,47.36743886)
\curveto(343.92282065,47.36743886)(343.29782073,47.34660553)(342.64677914,47.30493887)
\lineto(342.57646665,47.37525136)
\lineto(342.57646665,47.70337632)
\lineto(342.64677914,47.76587631)
\curveto(343.15198742,47.7971263)(343.43584155,47.8205638)(343.49834154,47.8361888)
\curveto(343.56604986,47.8518138)(343.62334152,47.88306379)(343.67021652,47.92993879)
\curveto(343.72229985,47.98202212)(343.75615401,48.05754294)(343.77177901,48.15650126)
\curveto(343.79261234,48.26066791)(343.81344567,48.59660537)(343.834279,49.16431364)
\curveto(343.86032066,49.73723023)(343.87334149,50.16952185)(343.87334149,50.46118848)
\lineto(343.87334149,52.00025079)
\curveto(343.87334149,52.20858409)(343.86292483,52.49243823)(343.8420915,52.85181318)
\curveto(343.82125817,53.21118814)(343.803029,53.43254228)(343.787404,53.5158756)
\curveto(343.77698734,53.59920892)(343.73792484,53.65910475)(343.67021652,53.69556308)
\curveto(343.60250819,53.73722974)(343.46709154,53.75806307)(343.26396657,53.75806307)
\lineto(342.60771665,53.76587557)
\lineto(342.53740416,53.82837556)
\lineto(342.53740416,54.16431302)
\lineto(342.59990415,54.22681301)
\curveto(343.59469569,54.34660466)(344.43323726,54.54191714)(345.11552884,54.81275044)
\closepath
}
}
{
\newrgbcolor{curcolor}{0 0 0}
\pscustom[linestyle=none,fillstyle=solid,fillcolor=curcolor]
{
\newpath
\moveto(349.22490333,49.66431357)
\lineto(349.56084079,49.66431357)
\lineto(349.63115328,49.59400108)
\curveto(349.64156995,49.15650114)(349.66761161,48.78150118)(349.70927827,48.46900122)
\curveto(349.87073659,48.22420959)(350.14938239,48.02368878)(350.54521567,47.8674388)
\curveto(350.94104896,47.71639715)(351.33167391,47.64087632)(351.71709053,47.64087632)
\curveto(352.26917379,47.64087632)(352.7092779,47.78931381)(353.03740286,48.08618877)
\curveto(353.37073616,48.38306373)(353.5374028,48.73723036)(353.5374028,49.14868864)
\curveto(353.5374028,49.37264694)(353.47750698,49.56535525)(353.35771532,49.72681357)
\curveto(353.23792367,49.89348021)(353.05563203,50.02889686)(352.81084039,50.13306352)
\curveto(352.57125709,50.2424385)(352.13896547,50.37004265)(351.51396555,50.51587597)
\curveto(350.97750728,50.64087595)(350.60250733,50.73723011)(350.38896569,50.80493843)
\curveto(350.17542405,50.87785509)(349.96969491,50.99764674)(349.77177827,51.16431339)
\curveto(349.57386162,51.33098004)(349.42281998,51.53410501)(349.31865332,51.77368831)
\curveto(349.21448667,52.01847995)(349.16240334,52.28670908)(349.16240334,52.57837572)
\curveto(349.16240334,53.2762923)(349.43844497,53.83358389)(349.99052824,54.25025051)
\curveto(350.54781984,54.67212546)(351.24052808,54.88306293)(352.06865298,54.88306293)
\curveto(352.41761127,54.88306293)(352.79781956,54.83358377)(353.20927784,54.73462545)
\curveto(353.62073612,54.64087546)(353.93323609,54.54972964)(354.14677773,54.46118798)
\lineto(354.21709022,54.351813)
\curveto(354.17542356,54.14347969)(354.14938189,53.59660476)(354.13896523,52.7111882)
\lineto(354.06865274,52.64087571)
\lineto(353.75615277,52.64087571)
\lineto(353.68584028,52.7111882)
\curveto(353.66500695,53.02889649)(353.63636112,53.25025063)(353.59990279,53.37525062)
\curveto(353.56344447,53.50545893)(353.46969448,53.64347975)(353.31865283,53.78931307)
\curveto(353.16761118,53.94035471)(352.95406954,54.0653547)(352.67802791,54.16431302)
\curveto(352.40198627,54.26847967)(352.11031964,54.320563)(351.80302802,54.320563)
\curveto(351.48531972,54.320563)(351.21709059,54.27368801)(350.99834061,54.17993802)
\curveto(350.78479897,54.08618803)(350.61031983,53.94295888)(350.47490318,53.75025057)
\curveto(350.34469486,53.56275059)(350.2795907,53.33358396)(350.2795907,53.06275066)
\curveto(350.2795907,52.86483401)(350.3186532,52.6877507)(350.39677819,52.53150072)
\curveto(350.48011151,52.37525074)(350.6051115,52.24764659)(350.77177814,52.14868827)
\curveto(350.93844479,52.05493828)(351.11292393,51.98722995)(351.29521558,51.94556329)
\lineto(352.14677797,51.72681332)
\curveto(352.87594455,51.54973001)(353.39677782,51.39608419)(353.70927778,51.26587588)
\curveto(354.02698607,51.13566756)(354.27177771,50.93775092)(354.44365269,50.67212595)
\curveto(354.61552767,50.41170932)(354.70146516,50.09400102)(354.70146516,49.71900107)
\curveto(354.70146516,49.00025116)(354.40459019,48.38306373)(353.81084027,47.8674388)
\curveto(353.21709034,47.35181386)(352.4540696,47.09400139)(351.52177805,47.09400139)
\curveto(351.19365309,47.09400139)(350.78479897,47.13045972)(350.2952157,47.20337638)
\curveto(349.81084076,47.27629304)(349.39677831,47.35702219)(349.05302835,47.44556385)
\lineto(349.01396586,47.54712634)
\lineto(349.09209085,48.07056377)
\curveto(349.11813251,48.23202208)(349.13375751,48.3934804)(349.13896584,48.55493871)
\curveto(349.14417418,48.72160536)(349.14938251,49.06795948)(349.15459084,49.59400108)
\closepath
}
}
{
\newrgbcolor{curcolor}{0 0 0}
\pscustom[linestyle=none,fillstyle=solid,fillcolor=curcolor]
{
\newpath
\moveto(357.92021476,54.81275044)
\lineto(358.06865224,54.71118795)
\curveto(358.01136058,54.02889637)(357.98271475,53.16170898)(357.98271475,52.10962577)
\lineto(357.98271475,50.2893135)
\curveto(357.98271475,49.62264691)(358.03219391,49.1591053)(358.13115224,48.89868867)
\curveto(358.23531889,48.63827203)(358.4124022,48.43775123)(358.66240217,48.29712624)
\curveto(358.91240214,48.16170959)(359.2092771,48.09400127)(359.55302706,48.09400127)
\curveto(359.93323535,48.09400127)(360.27958947,48.16691793)(360.59208943,48.31275124)
\curveto(360.90458939,48.46379289)(361.16500603,48.67473036)(361.37333934,48.94556366)
\curveto(361.58688098,49.21639696)(361.70927679,49.39868861)(361.74052679,49.4924386)
\curveto(361.77177679,49.58618858)(361.79261012,49.84920939)(361.80302678,50.281501)
\lineto(361.82646428,51.15650089)
\lineto(361.82646428,52.00025079)
\curveto(361.82646428,52.20858409)(361.81604761,52.49243823)(361.79521428,52.85181318)
\curveto(361.77438095,53.21118814)(361.75615179,53.43254228)(361.74052679,53.5158756)
\curveto(361.73011012,53.59920892)(361.69104763,53.65910475)(361.6233393,53.69556308)
\curveto(361.55563098,53.73722974)(361.42021433,53.75806307)(361.21708935,53.75806307)
\lineto(360.56083944,53.76587557)
\lineto(360.49052694,53.82837556)
\lineto(360.49052694,54.16431302)
\lineto(360.55302694,54.22681301)
\curveto(361.54781848,54.34660466)(362.38636004,54.54191714)(363.06865163,54.81275044)
\lineto(363.21708911,54.71118795)
\curveto(363.15979745,54.02889637)(363.13115162,53.16170898)(363.13115162,52.10962577)
\lineto(363.13115162,50.73462594)
\curveto(363.13115162,50.65650095)(363.14417245,50.0184802)(363.17021411,48.82056368)
\curveto(363.18063078,48.3778554)(363.20667244,48.1070221)(363.2483391,48.00806378)
\curveto(363.2952141,47.91431379)(363.35771409,47.84660547)(363.43583908,47.8049388)
\curveto(363.51396407,47.76848048)(363.74573488,47.75025131)(364.1311515,47.75025131)
\lineto(364.34990147,47.75025131)
\lineto(364.42021396,47.68775132)
\lineto(364.42021396,47.37525136)
\lineto(364.35771397,47.30493887)
\curveto(363.62333906,47.34660553)(363.13115162,47.36743886)(362.88115165,47.36743886)
\curveto(362.56344336,47.36743886)(362.1962559,47.34920969)(361.77958929,47.31275136)
\lineto(361.70927679,47.37525136)
\curveto(361.74052679,47.90650129)(361.76396429,48.35702207)(361.77958929,48.72681369)
\curveto(361.45146433,48.47160539)(361.07646437,48.1304596)(360.65458942,47.70337632)
\curveto(360.49313111,47.541918)(360.26136031,47.40650135)(359.95927701,47.29712637)
\curveto(359.65719371,47.18775138)(359.31344376,47.13306389)(358.92802714,47.13306389)
\curveto(358.34469388,47.13306389)(357.88896477,47.22420971)(357.56083981,47.40650135)
\curveto(357.23792318,47.59400133)(357.00875654,47.84920963)(356.87333989,48.17212626)
\curveto(356.73792324,48.50025122)(356.67021492,49.06275115)(356.67021492,49.85962605)
\lineto(356.67802741,50.47681347)
\lineto(356.67802741,52.00025079)
\curveto(356.67802741,52.20858409)(356.66761075,52.49243823)(356.64677742,52.85181318)
\curveto(356.62594409,53.21118814)(356.60771492,53.43254228)(356.59208992,53.5158756)
\curveto(356.58167326,53.59920892)(356.54261076,53.65910475)(356.47490244,53.69556308)
\curveto(356.40719411,53.73722974)(356.27177746,53.75806307)(356.06865249,53.75806307)
\lineto(355.41240257,53.76587557)
\lineto(355.34209008,53.82837556)
\lineto(355.34209008,54.16431302)
\lineto(355.40459007,54.22681301)
\curveto(356.39938162,54.34660466)(357.23792318,54.54191714)(357.92021476,54.81275044)
\closepath
\moveto(359.78740203,55.30493788)
\closepath
\moveto(359.86552702,46.89868892)
\closepath
}
}
{
\newrgbcolor{curcolor}{0 0 0}
\pscustom[linestyle=none,fillstyle=solid,fillcolor=curcolor]
{
\newpath
\moveto(371.64677557,48.20337625)
\lineto(371.33427561,47.67993882)
\curveto(370.66240069,47.3101472)(369.91240078,47.12525139)(369.08427588,47.12525139)
\curveto(367.93844269,47.12525139)(367.05302614,47.46639718)(366.42802621,48.14868876)
\curveto(365.80302629,48.83098034)(365.49052633,49.71118857)(365.49052633,50.78931344)
\curveto(365.49052633,51.27889671)(365.54260965,51.71379249)(365.64677631,52.09400078)
\curveto(365.75615129,52.47941739)(365.89156794,52.79712569)(366.05302626,53.04712566)
\curveto(366.2196929,53.29712563)(366.40198455,53.49504227)(366.59990119,53.64087558)
\curveto(366.79781783,53.7867089)(367.13115113,53.98983387)(367.59990107,54.25025051)
\curveto(368.07385934,54.51066714)(368.43063013,54.67993796)(368.67021344,54.75806295)
\curveto(368.90979674,54.84139627)(369.24052587,54.88306293)(369.66240081,54.88306293)
\curveto(370.46969238,54.88306293)(371.13115063,54.72941712)(371.64677557,54.42212549)
\curveto(371.54781725,53.83358389)(371.47229642,53.17733397)(371.4202131,52.45337573)
\lineto(371.34990061,52.39087574)
\lineto(371.02958814,52.39087574)
\lineto(370.95927565,52.46118823)
\curveto(370.93844232,52.89868818)(370.90979649,53.19295897)(370.87333816,53.34400062)
\curveto(370.83687984,53.49504227)(370.62854653,53.65650058)(370.24833824,53.82837556)
\curveto(369.87333829,54.00025054)(369.47229667,54.08618803)(369.04521339,54.08618803)
\curveto(368.61813011,54.08618803)(368.23792182,53.99504221)(367.90458853,53.81275056)
\curveto(367.57125524,53.63566725)(367.3186511,53.34139645)(367.14677612,52.92993817)
\curveto(366.98010948,52.52368822)(366.89677615,52.03670912)(366.89677615,51.46900085)
\curveto(366.89677615,50.98983424)(366.96448448,50.5210843)(367.09990113,50.06275103)
\curveto(367.23531778,49.60962608)(367.41240109,49.23202196)(367.63115106,48.92993867)
\curveto(367.85510937,48.6330637)(368.14938017,48.3934804)(368.51396345,48.21118875)
\curveto(368.87854674,48.02889711)(369.28479669,47.93775129)(369.7327133,47.93775129)
\curveto(370.01396327,47.93775129)(370.28740074,47.97420962)(370.5530257,48.04712627)
\curveto(370.823859,48.12004293)(371.1337548,48.23983458)(371.48271309,48.40650123)
\closepath
}
}
{
\newrgbcolor{curcolor}{0 0 0}
\pscustom[linestyle=none,fillstyle=solid,fillcolor=curcolor]
{
\newpath
\moveto(374.6233377,58.93774993)
\lineto(374.77177518,58.84399994)
\curveto(374.71448352,58.24504168)(374.68583769,57.044521)(374.68583769,55.24243789)
\lineto(374.68583769,53.31275062)
\curveto(375.12333764,53.67212558)(375.52958759,54.02889637)(375.90458754,54.38306299)
\curveto(376.01396253,54.48202131)(376.11812918,54.55493797)(376.21708751,54.60181297)
\curveto(376.31604583,54.64868796)(376.47489997,54.69816712)(376.69364995,54.75025045)
\curveto(376.91760825,54.80233377)(377.14677489,54.82837544)(377.38114986,54.82837544)
\curveto(377.77698315,54.82837544)(378.1597956,54.74764628)(378.52958722,54.58618797)
\curveto(378.90458717,54.42472965)(379.18583714,54.23202134)(379.37333712,54.00806304)
\curveto(379.56604543,53.78931307)(379.69885791,53.52889643)(379.77177457,53.22681314)
\curveto(379.84469122,52.92472984)(379.88114955,52.55233405)(379.88114955,52.10962577)
\lineto(379.88114955,50.73462594)
\curveto(379.88114955,50.64608429)(379.89417039,49.99764687)(379.92021205,48.78931368)
\curveto(379.93583705,48.29452208)(379.99052454,48.00025128)(380.08427453,47.90650129)
\curveto(380.18323285,47.8127513)(380.50614948,47.76587631)(381.05302441,47.76587631)
\lineto(381.1155244,47.70337632)
\lineto(381.1155244,47.36743886)
\lineto(381.05302441,47.30493887)
\curveto(380.39156616,47.34660553)(379.94364955,47.36743886)(379.70927457,47.36743886)
\curveto(379.58948292,47.36743886)(379.20927464,47.34660553)(378.56864972,47.30493887)
\lineto(378.47489973,47.39087636)
\curveto(378.54260805,48.0679596)(378.57646221,48.950772)(378.57646221,50.03931353)
\lineto(378.57646221,51.0627509)
\curveto(378.57646221,51.67733416)(378.56083722,52.12004244)(378.52958722,52.39087574)
\curveto(378.50354556,52.66170904)(378.41760807,52.90650067)(378.27177475,53.12525065)
\curveto(378.12594144,53.34920895)(377.93062896,53.52108393)(377.68583732,53.64087558)
\curveto(377.44625402,53.76587557)(377.15458739,53.82837556)(376.81083743,53.82837556)
\curveto(376.45667081,53.82837556)(376.15979585,53.7788964)(375.92021254,53.67993808)
\curveto(375.68583757,53.58618809)(375.44625427,53.42472978)(375.20146263,53.19556314)
\curveto(374.95667099,52.9663965)(374.80823351,52.77368819)(374.75615019,52.61743821)
\curveto(374.70927519,52.46639656)(374.68583769,52.17212577)(374.68583769,51.73462582)
\lineto(374.68583769,50.34400099)
\curveto(374.68583769,50.22941767)(374.69625436,49.85962605)(374.71708769,49.23462613)
\curveto(374.73792102,48.61483454)(374.75875435,48.25285542)(374.77958768,48.14868876)
\curveto(374.80562935,48.04452211)(374.84208767,47.96900128)(374.88896267,47.92212629)
\curveto(374.93583766,47.8752513)(374.99312932,47.8440013)(375.06083765,47.8283763)
\curveto(375.12854597,47.81795964)(375.40979594,47.7971263)(375.90458754,47.76587631)
\lineto(375.97490003,47.70337632)
\lineto(375.97490003,47.37525136)
\lineto(375.91240004,47.30493887)
\curveto(375.26135846,47.34660553)(374.63375437,47.36743886)(374.02958777,47.36743886)
\curveto(373.43062952,47.36743886)(372.80562959,47.34660553)(372.15458801,47.30493887)
\lineto(372.08427551,47.37525136)
\lineto(372.08427551,47.70337632)
\lineto(372.15458801,47.76587631)
\curveto(372.65979628,47.7971263)(372.94365041,47.8205638)(373.0061504,47.8361888)
\curveto(373.07385873,47.8518138)(373.13115039,47.88306379)(373.17802538,47.92993879)
\curveto(373.23010871,47.98202212)(373.26396287,48.05754294)(373.27958787,48.15650126)
\curveto(373.3004212,48.26066791)(373.32125453,48.59660537)(373.34208786,49.16431364)
\curveto(373.36812952,49.73723023)(373.38115035,50.16952185)(373.38115035,50.46118848)
\lineto(373.38115035,54.50806298)
\lineto(373.34990036,56.16431277)
\curveto(373.33427536,56.7424377)(373.3160462,57.14347932)(373.29521287,57.36743762)
\curveto(373.27958787,57.59139593)(373.25615037,57.72420841)(373.22490037,57.76587508)
\curveto(373.19365038,57.80754174)(373.13896288,57.83879173)(373.06083789,57.85962506)
\curveto(372.9827129,57.88045839)(372.66760878,57.89087506)(372.11552551,57.89087506)
\lineto(372.04521302,57.96118755)
\lineto(372.04521302,58.28931251)
\lineto(372.10771301,58.359625)
\curveto(373.10250456,58.47420832)(373.94104612,58.66691663)(374.6233377,58.93774993)
\closepath
}
}
{
\newrgbcolor{curcolor}{0 0 0}
\pscustom[linestyle=none,fillstyle=solid,fillcolor=curcolor]
{
\newpath
\moveto(382.06864928,49.66431357)
\lineto(382.40458674,49.66431357)
\lineto(382.47489923,49.59400108)
\curveto(382.4853159,49.15650114)(382.51135756,48.78150118)(382.55302422,48.46900122)
\curveto(382.71448254,48.22420959)(382.99312834,48.02368878)(383.38896162,47.8674388)
\curveto(383.78479491,47.71639715)(384.17541986,47.64087632)(384.56083648,47.64087632)
\curveto(385.11291974,47.64087632)(385.55302385,47.78931381)(385.88114881,48.08618877)
\curveto(386.21448211,48.38306373)(386.38114875,48.73723036)(386.38114875,49.14868864)
\curveto(386.38114875,49.37264694)(386.32125293,49.56535525)(386.20146127,49.72681357)
\curveto(386.08166962,49.89348021)(385.89937798,50.02889686)(385.65458634,50.13306352)
\curveto(385.41500304,50.2424385)(384.98271142,50.37004265)(384.3577115,50.51587597)
\curveto(383.82125323,50.64087595)(383.44625328,50.73723011)(383.23271164,50.80493843)
\curveto(383.01917,50.87785509)(382.81344086,50.99764674)(382.61552422,51.16431339)
\curveto(382.41760757,51.33098004)(382.26656593,51.53410501)(382.16239927,51.77368831)
\curveto(382.05823262,52.01847995)(382.00614929,52.28670908)(382.00614929,52.57837572)
\curveto(382.00614929,53.2762923)(382.28219092,53.83358389)(382.83427419,54.25025051)
\curveto(383.39156579,54.67212546)(384.08427404,54.88306293)(384.91239893,54.88306293)
\curveto(385.26135722,54.88306293)(385.64156551,54.83358377)(386.05302379,54.73462545)
\curveto(386.46448207,54.64087546)(386.77698204,54.54972964)(386.99052368,54.46118798)
\lineto(387.06083617,54.351813)
\curveto(387.01916951,54.14347969)(386.99312784,53.59660476)(386.98271118,52.7111882)
\lineto(386.91239869,52.64087571)
\lineto(386.59989872,52.64087571)
\lineto(386.52958623,52.7111882)
\curveto(386.5087529,53.02889649)(386.48010707,53.25025063)(386.44364874,53.37525062)
\curveto(386.40719042,53.50545893)(386.31344043,53.64347975)(386.16239878,53.78931307)
\curveto(386.01135713,53.94035471)(385.79781549,54.0653547)(385.52177386,54.16431302)
\curveto(385.24573223,54.26847967)(384.95406559,54.320563)(384.64677397,54.320563)
\curveto(384.32906567,54.320563)(384.06083654,54.27368801)(383.84208656,54.17993802)
\curveto(383.62854492,54.08618803)(383.45406578,53.94295888)(383.31864913,53.75025057)
\curveto(383.18844081,53.56275059)(383.12333665,53.33358396)(383.12333665,53.06275066)
\curveto(383.12333665,52.86483401)(383.16239915,52.6877507)(383.24052414,52.53150072)
\curveto(383.32385746,52.37525074)(383.44885745,52.24764659)(383.61552409,52.14868827)
\curveto(383.78219074,52.05493828)(383.95666988,51.98722995)(384.13896153,51.94556329)
\lineto(384.99052392,51.72681332)
\curveto(385.7196905,51.54973001)(386.24052377,51.39608419)(386.55302373,51.26587588)
\curveto(386.87073202,51.13566756)(387.11552366,50.93775092)(387.28739864,50.67212595)
\curveto(387.45927362,50.41170932)(387.54521111,50.09400102)(387.54521111,49.71900107)
\curveto(387.54521111,49.00025116)(387.24833614,48.38306373)(386.65458622,47.8674388)
\curveto(386.06083629,47.35181386)(385.29781555,47.09400139)(384.365524,47.09400139)
\curveto(384.03739904,47.09400139)(383.62854492,47.13045972)(383.13896165,47.20337638)
\curveto(382.65458671,47.27629304)(382.24052426,47.35702219)(381.8967743,47.44556385)
\lineto(381.85771181,47.54712634)
\lineto(381.9358368,48.07056377)
\curveto(381.96187846,48.23202208)(381.97750346,48.3934804)(381.98271179,48.55493871)
\curveto(381.98792013,48.72160536)(381.99312846,49.06795948)(381.99833679,49.59400108)
\closepath
}
}
{
\newrgbcolor{curcolor}{0 0 0}
\pscustom[linestyle=none,fillstyle=solid,fillcolor=curcolor]
{
\newpath
\moveto(390.95146069,54.90650043)
\lineto(391.09989817,54.81275044)
\curveto(391.04781484,54.21900051)(391.01916901,53.69295891)(391.01396068,53.23462563)
\curveto(391.72229393,53.88045889)(392.21448137,54.30754217)(392.490523,54.51587548)
\curveto(392.77177296,54.72420878)(393.25354374,54.82837544)(393.93583532,54.82837544)
\curveto(394.3629186,54.82837544)(394.75093939,54.76847961)(395.09989768,54.64868796)
\curveto(395.44885597,54.53410464)(395.77177259,54.34139633)(396.06864756,54.07056303)
\curveto(396.36552252,53.80493806)(396.59989749,53.46900061)(396.77177247,53.06275066)
\curveto(396.94364745,52.66170904)(397.02958494,52.19556326)(397.02958494,51.66431333)
\curveto(397.02958494,51.28931337)(396.96968911,50.90910509)(396.84989746,50.52368847)
\curveto(396.73010581,50.13827185)(396.57125166,49.78150106)(396.37333502,49.4533761)
\curveto(396.18062671,49.12525114)(396.02437673,48.90650117)(395.90458508,48.79712618)
\curveto(395.79000176,48.69295953)(395.55041845,48.53410538)(395.18583517,48.32056374)
\curveto(394.85771021,48.1226471)(394.53218941,47.90910546)(394.20927279,47.67993882)
\curveto(393.98531448,47.51848051)(393.8290645,47.41431385)(393.74052284,47.36743886)
\curveto(393.65198119,47.32056386)(393.50354371,47.27368887)(393.2952104,47.22681388)
\curveto(393.08687709,47.17993888)(392.86552295,47.15650138)(392.63114798,47.15650138)
\curveto(392.08948138,47.15650138)(391.55041895,47.2945222)(391.01396068,47.57056383)
\lineto(391.01396068,45.90650154)
\curveto(391.01396068,45.79191822)(391.02437735,45.42473076)(391.04521068,44.80493917)
\curveto(391.06604401,44.17993925)(391.08687734,43.81535596)(391.10771067,43.71118931)
\curveto(391.13375233,43.60702265)(391.17021066,43.53150183)(391.21708566,43.48462684)
\curveto(391.26396065,43.43775184)(391.32125231,43.40910601)(391.38896063,43.39868935)
\curveto(391.45666896,43.38306435)(391.73791892,43.35962685)(392.23271053,43.32837686)
\lineto(392.30302302,43.26587686)
\lineto(392.30302302,42.9377519)
\lineto(392.24052303,42.86743941)
\curveto(391.58948144,42.90910607)(390.96448152,42.92993941)(390.36552326,42.92993941)
\curveto(389.766565,42.92993941)(389.14156508,42.90910607)(388.49052349,42.86743941)
\lineto(388.420211,42.9377519)
\lineto(388.420211,43.26587686)
\lineto(388.49052349,43.32837686)
\curveto(388.99052343,43.35962685)(389.27437756,43.38306435)(389.34208589,43.39868935)
\curveto(389.40979421,43.41431435)(389.46708587,43.44816851)(389.51396087,43.50025183)
\curveto(389.56604419,43.54712683)(389.59989835,43.62525182)(389.61552335,43.73462681)
\curveto(389.63635668,43.84400179)(389.65719001,44.19296008)(389.67802335,44.78150168)
\curveto(389.70406501,45.37004327)(389.71708584,45.81275155)(389.71708584,46.10962651)
\lineto(389.71708584,52.09400078)
\curveto(389.71708584,52.36483408)(389.70406501,52.67733404)(389.67802335,53.03150066)
\curveto(389.65719001,53.39087562)(389.63896085,53.59920892)(389.62333585,53.65650058)
\curveto(389.60771085,53.71379224)(389.56864836,53.76066724)(389.50614837,53.79712557)
\curveto(389.44885671,53.83358389)(389.31604422,53.85181306)(389.10771092,53.85181306)
\lineto(388.451461,53.85962556)
\lineto(388.38114851,53.92212555)
\lineto(388.38114851,54.25806301)
\lineto(388.4436485,54.320563)
\curveto(389.43844004,54.44035465)(390.27437744,54.63566713)(390.95146069,54.90650043)
\closepath
\moveto(391.01396068,48.91431367)
\curveto(391.25354398,48.66952203)(391.55041895,48.45598039)(391.90458557,48.27368875)
\curveto(392.25875219,48.09660543)(392.64937715,48.00806378)(393.07646043,48.00806378)
\curveto(393.5816687,48.00806378)(394.02698114,48.13566793)(394.41239776,48.39087623)
\curveto(394.80302271,48.64608453)(395.10510601,49.01848032)(395.31864765,49.50806359)
\curveto(395.53218929,50.0028552)(395.63896011,50.54973013)(395.63896011,51.14868839)
\curveto(395.63896011,51.63306333)(395.54260596,52.08358411)(395.34989765,52.50025073)
\curveto(395.16239767,52.91691734)(394.86812687,53.2450423)(394.46708525,53.4846256)
\curveto(394.07125197,53.72420891)(393.64937702,53.84400056)(393.20146041,53.84400056)
\curveto(392.92541878,53.84400056)(392.65718964,53.7945214)(392.39677301,53.69556308)
\curveto(392.13635638,53.60181309)(391.89156474,53.46118811)(391.6623981,53.27368813)
\curveto(391.43843979,53.08618815)(391.27698148,52.90650067)(391.17802316,52.7346257)
\curveto(391.07906484,52.56795905)(391.02698151,52.42733407)(391.02177318,52.31275075)
\curveto(391.01656485,52.20337576)(391.01396068,52.08097994)(391.01396068,51.94556329)
\closepath
}
}
{
\newrgbcolor{curcolor}{0 0 0}
\pscustom[linestyle=none,fillstyle=solid,fillcolor=curcolor]
{
\newpath
\moveto(400.599897,58.93774993)
\lineto(400.74833448,58.84399994)
\curveto(400.69104282,58.24504168)(400.66239699,57.044521)(400.66239699,55.24243789)
\lineto(400.66239699,50.34400099)
\curveto(400.66239699,50.22941767)(400.67281366,49.85962605)(400.69364699,49.23462613)
\curveto(400.71448032,48.61483454)(400.73531365,48.25285542)(400.75614698,48.14868876)
\curveto(400.78218864,48.04452211)(400.81864697,47.96900128)(400.86552197,47.92212629)
\curveto(400.91239696,47.8752513)(400.96968862,47.8440013)(401.03739694,47.8283763)
\curveto(401.10510527,47.81795964)(401.38635523,47.7971263)(401.88114684,47.76587631)
\lineto(401.95145933,47.70337632)
\lineto(401.95145933,47.37525136)
\lineto(401.88895934,47.30493887)
\curveto(401.23791775,47.34660553)(400.61031366,47.36743886)(400.00614707,47.36743886)
\curveto(399.40718881,47.36743886)(398.78218889,47.34660553)(398.1311473,47.30493887)
\lineto(398.06083481,47.37525136)
\lineto(398.06083481,47.70337632)
\lineto(398.1311473,47.76587631)
\curveto(398.63635557,47.7971263)(398.92020971,47.8205638)(398.9827097,47.8361888)
\curveto(399.05041802,47.8518138)(399.10770968,47.88306379)(399.15458468,47.92993879)
\curveto(399.206668,47.98202212)(399.24052217,48.05754294)(399.25614716,48.15650126)
\curveto(399.2769805,48.26066791)(399.29781383,48.59660537)(399.31864716,49.16431364)
\curveto(399.34468882,49.73723023)(399.35770965,50.16952185)(399.35770965,50.46118848)
\lineto(399.35770965,54.50806298)
\lineto(399.32645966,56.16431277)
\curveto(399.31083466,56.7424377)(399.29260549,57.14347932)(399.27177216,57.36743762)
\curveto(399.25614716,57.59139593)(399.23270967,57.72420841)(399.20145967,57.76587508)
\curveto(399.17020967,57.80754174)(399.11552218,57.83879173)(399.03739719,57.85962506)
\curveto(398.9592722,57.88045839)(398.64416807,57.89087506)(398.09208481,57.89087506)
\lineto(398.02177232,57.96118755)
\lineto(398.02177232,58.28931251)
\lineto(398.08427231,58.359625)
\curveto(399.07906385,58.47420832)(399.91760542,58.66691663)(400.599897,58.93774993)
\closepath
}
}
{
\newrgbcolor{curcolor}{0 0 0}
\pscustom[linestyle=none,fillstyle=solid,fillcolor=curcolor]
{
\newpath
\moveto(403.99052158,52.6643132)
\lineto(403.68583412,52.7424382)
\lineto(403.62333413,52.82056319)
\lineto(403.62333413,53.78931307)
\curveto(404.55041734,54.47681298)(405.4514589,54.82056294)(406.32645879,54.82056294)
\curveto(406.92541705,54.82056294)(407.42281282,54.70858379)(407.81864611,54.48462548)
\curveto(408.21447939,54.26066717)(408.50093769,53.97941721)(408.678021,53.64087558)
\curveto(408.85510431,53.30754229)(408.94364597,52.91691734)(408.94364597,52.46900073)
\lineto(408.90458347,50.89868842)
\lineto(408.90458347,48.57056371)
\curveto(408.90458347,48.25285542)(408.92541681,48.06014711)(408.96708347,47.99243878)
\curveto(409.01395846,47.92473046)(409.06604179,47.87785546)(409.12333345,47.8518138)
\curveto(409.18062511,47.83098047)(409.28739593,47.8127513)(409.44364591,47.7971263)
\lineto(409.88895835,47.75806381)
\lineto(409.95145835,47.68775132)
\lineto(409.95145835,47.37525136)
\lineto(409.88895835,47.31275136)
\curveto(409.50875007,47.34400136)(409.15458344,47.35962636)(408.82645848,47.35962636)
\curveto(408.51395852,47.35962636)(408.13895857,47.34400136)(407.70145862,47.31275136)
\lineto(407.58427114,47.42212635)
\lineto(407.61552113,48.6799387)
\lineto(405.91239634,47.35181386)
\curveto(405.62593805,47.23202221)(405.31343808,47.17212638)(404.97489646,47.17212638)
\curveto(404.55822984,47.17212638)(404.19885489,47.24764721)(403.89677159,47.39868885)
\curveto(403.59989663,47.5497305)(403.37072999,47.76066798)(403.20927168,48.03150128)
\curveto(403.0530217,48.30233458)(402.97489671,48.63045954)(402.97489671,49.01587615)
\curveto(402.97489671,49.78150106)(403.21448001,50.37785515)(403.69364662,50.80493843)
\curveto(404.17281322,51.23723005)(405.48010473,51.6044175)(407.61552113,51.9065008)
\curveto(407.61552113,52.67733404)(407.44364615,53.21900064)(407.0998962,53.5315006)
\curveto(406.75614624,53.84920889)(406.2952088,54.00806304)(405.71708387,54.00806304)
\curveto(405.41500057,54.00806304)(405.13895894,53.96379221)(404.88895897,53.87525056)
\curveto(404.64416733,53.7867089)(404.50354235,53.71379224)(404.46708402,53.65650058)
\curveto(404.43062569,53.60441726)(404.29000071,53.28670896)(404.04520907,52.7033757)
\closepath
\moveto(407.61552113,51.42993836)
\curveto(406.15718798,51.18514672)(405.24833393,50.92212592)(404.88895897,50.64087595)
\curveto(404.52958401,50.35962599)(404.34989654,49.92473021)(404.34989654,49.33618862)
\curveto(404.34989654,48.52368872)(404.75354232,48.11743877)(405.56083389,48.11743877)
\curveto(406.25354213,48.11743877)(406.93843788,48.51848038)(407.61552113,49.32056362)
\closepath
\moveto(406.2873963,55.30493788)
\closepath
\moveto(406.12333382,46.89868892)
\closepath
}
}
{
\newrgbcolor{curcolor}{0 0 0}
\pscustom[linestyle=none,fillstyle=solid,fillcolor=curcolor]
{
\newpath
\moveto(413.00614547,54.81275044)
\lineto(413.15458295,54.71118795)
\curveto(413.12333295,54.36222966)(413.10249962,53.90389639)(413.09208296,53.33618812)
\curveto(413.50874957,53.67993808)(413.90458286,54.02889637)(414.27958281,54.38306299)
\curveto(414.3889578,54.48202131)(414.49052029,54.55493797)(414.58427027,54.60181297)
\curveto(414.6832286,54.64868796)(414.84468691,54.69816712)(415.06864521,54.75025045)
\curveto(415.29260352,54.80233377)(415.52177016,54.82837544)(415.75614513,54.82837544)
\curveto(416.15197841,54.82837544)(416.53479087,54.74764628)(416.90458249,54.58618797)
\curveto(417.27958244,54.42472965)(417.56083241,54.23202134)(417.74833238,54.00806304)
\curveto(417.94104069,53.78931307)(418.07385318,53.52889643)(418.14676984,53.22681314)
\curveto(418.21968649,52.92472984)(418.25614482,52.55233405)(418.25614482,52.10962577)
\lineto(418.25614482,50.73462594)
\curveto(418.25614482,50.64608429)(418.26916565,49.99764687)(418.29520732,48.78931368)
\curveto(418.30562398,48.29452208)(418.36031148,48.00025128)(418.4592698,47.90650129)
\curveto(418.55822812,47.8127513)(418.88114474,47.76587631)(419.42801968,47.76587631)
\lineto(419.49051967,47.70337632)
\lineto(419.49051967,47.36743886)
\lineto(419.42801968,47.30493887)
\curveto(418.76656143,47.34660553)(418.31864481,47.36743886)(418.08426984,47.36743886)
\curveto(417.94885319,47.36743886)(417.56864491,47.34660553)(416.94364498,47.30493887)
\lineto(416.84989499,47.39087636)
\curveto(416.91760332,48.0679596)(416.95145748,48.950772)(416.95145748,50.03931353)
\lineto(416.95145748,51.0627509)
\curveto(416.95145748,51.67733416)(416.93583248,52.12004244)(416.90458249,52.39087574)
\curveto(416.87854082,52.66170904)(416.79260334,52.90650067)(416.64677002,53.12525065)
\curveto(416.5009367,53.34920895)(416.30562423,53.52108393)(416.06083259,53.64087558)
\curveto(415.81604096,53.76587557)(415.52437433,53.82837556)(415.1858327,53.82837556)
\curveto(414.9149994,53.82837556)(414.6832286,53.79972973)(414.49052029,53.74243807)
\curveto(414.29781198,53.69035474)(414.08166617,53.57837559)(413.84208287,53.40650061)
\curveto(413.60249956,53.23983397)(413.42541625,53.08097982)(413.31083293,52.92993817)
\curveto(413.19624961,52.78410486)(413.12593712,52.64608404)(413.09989546,52.51587572)
\curveto(413.07906213,52.39087574)(413.06864546,52.12004244)(413.06864546,51.70337582)
\lineto(413.06864546,50.34400099)
\curveto(413.06864546,50.22941767)(413.07906213,49.85962605)(413.09989546,49.23462613)
\curveto(413.12072879,48.61483454)(413.14156212,48.25285542)(413.16239545,48.14868876)
\curveto(413.18843711,48.04452211)(413.22489544,47.96900128)(413.27177044,47.92212629)
\curveto(413.31864543,47.8752513)(413.37593709,47.8440013)(413.44364541,47.8283763)
\curveto(413.51135374,47.81795964)(413.79260371,47.7971263)(414.28739531,47.76587631)
\lineto(414.3577078,47.70337632)
\lineto(414.3577078,47.37525136)
\lineto(414.29520781,47.30493887)
\curveto(413.64416622,47.34660553)(413.01656213,47.36743886)(412.41239554,47.36743886)
\curveto(411.81343728,47.36743886)(411.18843736,47.34660553)(410.53739577,47.30493887)
\lineto(410.46708328,47.37525136)
\lineto(410.46708328,47.70337632)
\lineto(410.53739577,47.76587631)
\curveto(411.04260404,47.7971263)(411.32645818,47.8205638)(411.38895817,47.8361888)
\curveto(411.45666649,47.8518138)(411.51395815,47.88306379)(411.56083315,47.92993879)
\curveto(411.61291647,47.98202212)(411.64677064,48.05754294)(411.66239563,48.15650126)
\curveto(411.68322897,48.26066791)(411.7040623,48.59660537)(411.72489563,49.16431364)
\curveto(411.75093729,49.73723023)(411.76395812,50.16952185)(411.76395812,50.46118848)
\lineto(411.76395812,52.00025079)
\curveto(411.76395812,52.20858409)(411.75354146,52.49243823)(411.73270813,52.85181318)
\curveto(411.7118748,53.21118814)(411.69364563,53.43254228)(411.67802063,53.5158756)
\curveto(411.66760397,53.59920892)(411.62854147,53.65910475)(411.56083315,53.69556308)
\curveto(411.49312482,53.73722974)(411.35770817,53.75806307)(411.1545832,53.75806307)
\lineto(410.49833328,53.76587557)
\lineto(410.42802079,53.82837556)
\lineto(410.42802079,54.16431302)
\lineto(410.49052078,54.22681301)
\curveto(411.48531232,54.34660466)(412.32385389,54.54191714)(413.00614547,54.81275044)
\closepath
}
}
{
\newrgbcolor{curcolor}{1 1 1}
\pscustom[linestyle=none,fillstyle=solid,fillcolor=curcolor]
{
\newpath
\moveto(27.30123519,117.29909122)
\curveto(27.30123519,113.12434076)(23.91693126,109.74003683)(19.7421808,109.74003683)
\curveto(15.56743033,109.74003683)(12.1831264,113.12434076)(12.1831264,117.29909122)
\curveto(12.1831264,121.47384169)(15.56743033,124.85814562)(19.7421808,124.85814562)
\curveto(23.91693126,124.85814562)(27.30123519,121.47384169)(27.30123519,117.29909122)
\closepath
}
}
{
\newrgbcolor{curcolor}{0.15686275 0.16078432 0.16470589}
\pscustom[linewidth=2.88359956,linecolor=curcolor]
{
\newpath
\moveto(27.30123519,117.29909122)
\curveto(27.30123519,113.12434076)(23.91693126,109.74003683)(19.7421808,109.74003683)
\curveto(15.56743033,109.74003683)(12.1831264,113.12434076)(12.1831264,117.29909122)
\curveto(12.1831264,121.47384169)(15.56743033,124.85814562)(19.7421808,124.85814562)
\curveto(23.91693126,124.85814562)(27.30123519,121.47384169)(27.30123519,117.29909122)
\closepath
}
}
{
\newrgbcolor{curcolor}{0 0 0}
\pscustom[linestyle=none,fillstyle=solid,fillcolor=curcolor]
{
\newpath
\moveto(23.30468036,113.76784166)
\curveto(23.24218036,113.15846673)(23.23436786,112.58034181)(23.28124286,112.03346687)
\lineto(23.18749287,111.92409189)
\curveto(21.75520138,111.98138355)(20.58332652,112.01002938)(19.6718683,112.01002938)
\curveto(18.71874342,112.01002938)(17.58853523,111.98398771)(16.28124372,111.93190439)
\lineto(16.17968123,112.02565437)
\lineto(16.17968123,112.73659179)
\lineto(17.77343104,114.36159159)
\curveto(18.96613922,115.5751331)(19.74478496,116.49700799)(20.10936825,117.12721625)
\curveto(20.32290989,117.49700787)(20.47395154,117.83294532)(20.56249319,118.13502862)
\curveto(20.65103485,118.44232025)(20.69530568,118.76784104)(20.69530568,119.111591)
\curveto(20.69530568,120.35638251)(20.12499325,120.97877827)(18.98436839,120.97877827)
\curveto(18.3489518,120.97877827)(17.84374353,120.7365908)(17.46874358,120.25221586)
\curveto(17.41145192,120.02825755)(17.30988943,119.6376326)(17.16405611,119.080341)
\lineto(17.07811862,119.00221601)
\lineto(16.63280618,119.00221601)
\lineto(16.55468119,119.080341)
\lineto(16.84374365,121.68971568)
\curveto(17.49999357,122.3459656)(18.41405596,122.67409056)(19.58593081,122.67409056)
\curveto(20.63280569,122.67409056)(21.44009725,122.4006531)(22.00780552,121.85377816)
\curveto(22.53384712,121.35377822)(22.79686792,120.68971581)(22.79686792,119.86159091)
\curveto(22.79686792,119.52304928)(22.75780542,119.21054932)(22.67968043,118.92409102)
\curveto(22.51822212,118.34075776)(22.18749299,117.75221617)(21.68749306,117.15846624)
\curveto(21.27082644,116.66367464)(20.64061818,116.01263305)(19.79686829,115.20534148)
\curveto(19.48957666,114.90325819)(19.02343088,114.45794574)(18.39843096,113.86940415)
\lineto(18.41405596,113.76784166)
\lineto(19.79686829,113.76784166)
\curveto(20.19790991,113.76784166)(20.57811819,113.77304999)(20.93749315,113.78346666)
\lineto(22.28905548,113.82252915)
\curveto(22.63280544,113.83294582)(22.93228457,113.84857082)(23.18749287,113.86940415)
\closepath
}
}
{
\newrgbcolor{curcolor}{0 0 0}
\pscustom[linestyle=none,fillstyle=solid,fillcolor=curcolor]
{
\newpath
\moveto(34.26400416,123.38543487)
\lineto(35.24056654,123.35418488)
\curveto(35.88119146,123.33335155)(36.30046224,123.32293488)(36.49837888,123.32293488)
\curveto(36.6754622,123.32293488)(37.42285794,123.34376821)(38.74056611,123.38543487)
\lineto(38.8030661,123.33074738)
\lineto(38.8030661,122.94012243)
\lineto(38.74056611,122.88543494)
\curveto(38.38119115,122.87501827)(38.10254535,122.85939327)(37.90462871,122.83855994)
\curveto(37.7119204,122.81772661)(37.57129542,122.77866412)(37.48275376,122.72137246)
\curveto(37.39421211,122.6640808)(37.33431628,122.52866415)(37.30306628,122.31512251)
\curveto(37.27702462,122.1067892)(37.26400379,121.81512257)(37.26400379,121.44012261)
\lineto(37.24837879,119.64324784)
\lineto(37.24837879,116.02606078)
\lineto(37.25619129,114.50262347)
\curveto(37.26139962,113.91408188)(37.27702462,113.53387359)(37.30306628,113.36199861)
\curveto(37.32910795,113.19012363)(37.36556628,113.07554031)(37.41244127,113.01824865)
\curveto(37.4645246,112.96095699)(37.56087875,112.91147783)(37.70150374,112.86981117)
\curveto(37.84212872,112.83335284)(38.18848284,112.80470701)(38.74056611,112.78387368)
\lineto(38.8030661,112.73699869)
\lineto(38.8030661,112.34637374)
\lineto(38.74056611,112.28387374)
\curveto(38.71452444,112.28387374)(38.25619117,112.29949874)(37.36556628,112.33074874)
\curveto(36.99577466,112.3411654)(36.70671219,112.34637374)(36.49837888,112.34637374)
\curveto(36.33692057,112.34637374)(35.592129,112.32554041)(34.26400416,112.28387374)
\lineto(34.20150417,112.34637374)
\lineto(34.20150417,112.73699869)
\lineto(34.26400416,112.78387368)
\curveto(34.7483791,112.80470701)(35.07129573,112.83074868)(35.23275404,112.86199867)
\curveto(35.39421235,112.89324867)(35.50098317,112.93231116)(35.5530665,112.97918616)
\curveto(35.61035816,113.03126948)(35.65462899,113.1380403)(35.68587898,113.29949862)
\curveto(35.72233731,113.46095693)(35.74317064,113.82033189)(35.74837898,114.37762349)
\lineto(35.75619148,116.02606078)
\lineto(35.75619148,119.64324784)
\lineto(35.74056648,121.16668515)
\curveto(35.73535814,121.76043508)(35.71973315,122.14324753)(35.69369148,122.31512251)
\curveto(35.67285815,122.48699749)(35.63639982,122.6015808)(35.5843165,122.65887246)
\curveto(35.5374415,122.71616412)(35.44369151,122.76303912)(35.30306653,122.79949745)
\curveto(35.16244155,122.83595578)(34.81608742,122.86460161)(34.26400416,122.88543494)
\lineto(34.20150417,122.94012243)
\lineto(34.20150417,123.33074738)
\closepath
\moveto(36.60775387,123.9635598)
\closepath
\moveto(36.49837888,111.87762379)
\closepath
}
}
{
\newrgbcolor{curcolor}{0 0 0}
\pscustom[linestyle=none,fillstyle=solid,fillcolor=curcolor]
{
\newpath
\moveto(44.4515029,122.94012243)
\lineto(44.4515029,123.26824739)
\lineto(44.5140029,123.33855988)
\curveto(45.50879444,123.4531432)(46.347336,123.64585151)(47.02962759,123.91668481)
\lineto(47.17806507,123.82293482)
\curveto(47.12077341,123.22397656)(47.09212758,122.02345588)(47.09212758,120.22137276)
\lineto(47.09212758,115.1041859)
\curveto(47.09212758,114.57814429)(47.09994008,114.11720685)(47.11556507,113.72137357)
\curveto(47.13639841,113.33074861)(47.17025257,113.09897781)(47.21712756,113.02606115)
\curveto(47.26400256,112.95314449)(47.33952338,112.89324867)(47.44369003,112.84637367)
\curveto(47.55306502,112.79949868)(47.85775248,112.75522785)(48.35775242,112.71356119)
\lineto(48.42806491,112.6510612)
\lineto(48.42806491,112.35418623)
\lineto(48.35775242,112.28387374)
\curveto(47.79525249,112.32033207)(47.35775254,112.33856124)(47.04525258,112.33856124)
\curveto(46.76400262,112.33856124)(46.362961,112.32033207)(45.84212773,112.28387374)
\lineto(45.75619024,112.36981123)
\curveto(45.77181524,112.88022784)(45.77962774,113.23699863)(45.77962774,113.4401236)
\curveto(45.77962774,113.46616526)(45.78223191,113.56512359)(45.78744024,113.73699856)
\curveto(45.50098194,113.52866526)(45.21452364,113.29689445)(44.92806534,113.04168615)
\curveto(44.5061904,112.6666862)(44.23014876,112.43231123)(44.09994045,112.33856124)
\curveto(43.86035714,112.21876958)(43.51921135,112.15887376)(43.07650307,112.15887376)
\curveto(42.35775316,112.15887376)(41.74577407,112.33595707)(41.2405658,112.69012369)
\curveto(40.74056586,113.04429032)(40.38379507,113.49481109)(40.17025343,114.04168603)
\curveto(39.95671179,114.59376929)(39.84994097,115.15887339)(39.84994097,115.73699832)
\curveto(39.84994097,116.32553991)(39.95931596,116.89064401)(40.17806593,117.43231061)
\curveto(40.40202424,117.97918554)(40.71712836,118.37762299)(41.12337831,118.62762296)
\curveto(41.5348366,118.87762293)(41.98014904,119.1354354)(42.45931565,119.40106037)
\curveto(42.94369059,119.67189367)(43.4228572,119.80731032)(43.89681547,119.80731032)
\curveto(44.55827372,119.80731032)(45.18848198,119.64324784)(45.78744024,119.31512288)
\lineto(45.75619024,121.05731016)
\curveto(45.74577358,121.66147675)(45.73014858,122.08074754)(45.70931525,122.31512251)
\curveto(45.68848192,122.55470581)(45.66244025,122.69533079)(45.63119026,122.73699745)
\curveto(45.59994026,122.78387245)(45.54525277,122.81772661)(45.46712778,122.83855994)
\curveto(45.39421112,122.85939327)(45.07910699,122.86980994)(44.52181539,122.86980994)
\closepath
\moveto(45.78744024,117.97918554)
\curveto(45.46452361,118.34376883)(45.10514866,118.62241463)(44.70931537,118.81512294)
\curveto(44.31869042,119.00783125)(43.9254613,119.1041854)(43.52962802,119.1041854)
\curveto(43.09212807,119.1041854)(42.68327395,118.98439375)(42.30306567,118.74481045)
\curveto(41.92806571,118.51043548)(41.65723241,118.15887302)(41.49056577,117.69012308)
\curveto(41.32389912,117.22658147)(41.2405658,116.7213732)(41.2405658,116.17449826)
\curveto(41.2405658,115.26304004)(41.46973244,114.5312693)(41.92806571,113.97918603)
\curveto(42.39160732,113.42710277)(42.97233642,113.15106114)(43.670253,113.15106114)
\curveto(44.05046129,113.15106114)(44.38639874,113.23699863)(44.67806538,113.4088736)
\curveto(44.97494034,113.58595692)(45.21712781,113.82293605)(45.40462779,114.11981102)
\curveto(45.59733609,114.41668598)(45.70931525,114.71356094)(45.74056524,115.01043591)
\curveto(45.77181524,115.30731087)(45.78744024,115.85158164)(45.78744024,116.64324821)
\closepath
}
}
{
\newrgbcolor{curcolor}{0 0 0}
\pscustom[linestyle=none,fillstyle=solid,fillcolor=curcolor]
{
\newpath
\moveto(55.87337649,113.49481109)
\lineto(55.62337653,112.93231116)
\curveto(55.08170993,112.58335287)(54.59733499,112.3567904)(54.1702517,112.25262375)
\curveto(53.74837676,112.14845709)(53.37077264,112.09637377)(53.03743934,112.09637377)
\curveto(52.41243942,112.09637377)(51.81868949,112.21876958)(51.25618956,112.46356122)
\curveto(50.69889797,112.70835286)(50.24316886,113.12241531)(49.88900223,113.70574857)
\curveto(49.54004394,114.28908183)(49.3655648,114.99220674)(49.3655648,115.81512331)
\curveto(49.3655648,116.36199824)(49.43327312,116.85418568)(49.56868977,117.29168563)
\curveto(49.70410642,117.7343939)(49.8447314,118.06251886)(49.99056472,118.2760605)
\curveto(50.14160637,118.48960215)(50.3942105,118.72658128)(50.74837713,118.98699792)
\curveto(51.10254375,119.24741455)(51.4775437,119.45574786)(51.87337699,119.61199784)
\curveto(52.26921027,119.76824782)(52.69629355,119.84637281)(53.15462683,119.84637281)
\curveto(53.77962675,119.84637281)(54.32910585,119.69793533)(54.80306413,119.40106037)
\curveto(55.28223073,119.10939374)(55.61816819,118.73439378)(55.8108765,118.2760605)
\curveto(56.00358481,117.81772723)(56.09993897,117.33074812)(56.09993897,116.81512318)
\curveto(56.09993897,116.65366487)(56.09212647,116.49741489)(56.07650147,116.34637324)
\lineto(55.99056398,116.26043575)
\curveto(55.63639736,116.18231076)(55.15983492,116.13022744)(54.56087666,116.10418577)
\curveto(53.9619184,116.07814411)(53.56608511,116.06512328)(53.3733768,116.06512328)
\lineto(50.86556461,116.06512328)
\curveto(50.87598128,114.98699841)(51.14681458,114.19272767)(51.67806451,113.68231107)
\curveto(52.20931445,113.17189447)(52.86035603,112.91668617)(53.63118927,112.91668617)
\curveto(53.99577256,112.91668617)(54.34473085,112.97918616)(54.67806414,113.10418614)
\curveto(55.01660577,113.22918613)(55.37337656,113.40366527)(55.74837651,113.62762358)
\closepath
\moveto(50.86556461,116.6901232)
\curveto(50.9593146,116.6744982)(51.31868956,116.65626904)(51.94368948,116.63543571)
\curveto(52.57389773,116.61460238)(53.04004351,116.60418571)(53.34212681,116.60418571)
\curveto(54.06608505,116.60418571)(54.50618916,116.61720654)(54.66243914,116.64324821)
\curveto(54.66764748,116.76824819)(54.67025164,116.86460235)(54.67025164,116.93231067)
\curveto(54.67025164,117.73960224)(54.50618916,118.3385605)(54.1780642,118.72918545)
\curveto(53.84993924,119.12501873)(53.40202263,119.32293538)(52.83431437,119.32293538)
\curveto(52.21452278,119.32293538)(51.73014784,119.10158124)(51.38118955,118.65887296)
\curveto(51.03743959,118.21616468)(50.86556461,117.55991476)(50.86556461,116.6901232)
\closepath
\moveto(53.04525184,120.28387276)
\closepath
\moveto(52.95150185,111.87762379)
\closepath
}
}
{
\newrgbcolor{curcolor}{0 0 0}
\pscustom[linestyle=none,fillstyle=solid,fillcolor=curcolor]
{
\newpath
\moveto(59.30306357,119.79168532)
\lineto(59.45150105,119.69012283)
\curveto(59.42025106,119.34116454)(59.39941773,118.88283126)(59.38900106,118.315123)
\curveto(59.80566768,118.65887296)(60.20150096,119.00783125)(60.57650091,119.36199787)
\curveto(60.6858759,119.46095619)(60.78743839,119.53387285)(60.88118838,119.58074784)
\curveto(60.9801467,119.62762284)(61.14160501,119.677102)(61.36556332,119.72918533)
\curveto(61.58952162,119.78126865)(61.81868826,119.80731032)(62.05306323,119.80731032)
\curveto(62.44889652,119.80731032)(62.83170897,119.72658116)(63.20150059,119.56512285)
\curveto(63.57650054,119.40366453)(63.85775051,119.21095622)(64.04525049,118.98699792)
\curveto(64.2379588,118.76824794)(64.37077128,118.50783131)(64.44368794,118.20574801)
\curveto(64.5166046,117.90366472)(64.55306292,117.53126893)(64.55306292,117.08856065)
\lineto(64.55306292,115.71356082)
\curveto(64.55306292,115.62501916)(64.56608376,114.97658174)(64.59212542,113.76824856)
\curveto(64.60254208,113.27345695)(64.65722958,112.97918616)(64.7561879,112.88543617)
\curveto(64.85514622,112.79168618)(65.17806285,112.74481119)(65.72493778,112.74481119)
\lineto(65.78743777,112.68231119)
\lineto(65.78743777,112.34637374)
\lineto(65.72493778,112.28387374)
\curveto(65.06347953,112.32554041)(64.61556292,112.34637374)(64.38118795,112.34637374)
\curveto(64.2457713,112.34637374)(63.86556301,112.32554041)(63.24056309,112.28387374)
\lineto(63.1468131,112.36981123)
\curveto(63.21452142,113.04689448)(63.24837559,113.92970687)(63.24837559,115.01824841)
\lineto(63.24837559,116.04168578)
\curveto(63.24837559,116.65626904)(63.23275059,117.09897732)(63.20150059,117.36981062)
\curveto(63.17545893,117.64064392)(63.08952144,117.88543555)(62.94368812,118.10418553)
\curveto(62.79785481,118.32814383)(62.60254233,118.50001881)(62.3577507,118.61981046)
\curveto(62.11295906,118.74481045)(61.82129243,118.80731044)(61.4827508,118.80731044)
\curveto(61.2119175,118.80731044)(60.9801467,118.77866461)(60.78743839,118.72137295)
\curveto(60.59473008,118.66928962)(60.37858427,118.55731047)(60.13900097,118.38543549)
\curveto(59.89941766,118.21876885)(59.72233435,118.0599147)(59.60775103,117.90887305)
\curveto(59.49316771,117.76303973)(59.42285522,117.62501892)(59.39681356,117.4948106)
\curveto(59.37598023,117.36981062)(59.36556356,117.09897732)(59.36556356,116.6823107)
\lineto(59.36556356,115.32293587)
\curveto(59.36556356,115.20835255)(59.37598023,114.83856093)(59.39681356,114.21356101)
\curveto(59.41764689,113.59376942)(59.43848022,113.23179029)(59.45931355,113.12762364)
\curveto(59.48535522,113.02345699)(59.52181354,112.94793616)(59.56868854,112.90106117)
\curveto(59.61556353,112.85418617)(59.67285519,112.82293618)(59.74056352,112.80731118)
\curveto(59.80827184,112.79689451)(60.08952181,112.77606118)(60.58431341,112.74481119)
\lineto(60.65462591,112.68231119)
\lineto(60.65462591,112.35418623)
\lineto(60.59212591,112.28387374)
\curveto(59.94108433,112.32554041)(59.31348024,112.34637374)(58.70931364,112.34637374)
\curveto(58.11035539,112.34637374)(57.48535546,112.32554041)(56.83431388,112.28387374)
\lineto(56.76400138,112.35418623)
\lineto(56.76400138,112.68231119)
\lineto(56.83431388,112.74481119)
\curveto(57.33952215,112.77606118)(57.62337628,112.79949868)(57.68587627,112.81512368)
\curveto(57.7535846,112.83074868)(57.81087626,112.86199867)(57.85775125,112.90887367)
\curveto(57.90983458,112.96095699)(57.94368874,113.03647782)(57.95931374,113.13543614)
\curveto(57.98014707,113.23960279)(58.0009804,113.57554025)(58.02181373,114.14324851)
\curveto(58.04785539,114.71616511)(58.06087622,115.14845672)(58.06087622,115.44012335)
\lineto(58.06087622,116.97918566)
\curveto(58.06087622,117.18751897)(58.05045956,117.4713731)(58.02962623,117.83074806)
\curveto(58.0087929,118.19012302)(57.99056373,118.41147715)(57.97493874,118.49481048)
\curveto(57.96452207,118.5781438)(57.92545957,118.63803963)(57.85775125,118.67449796)
\curveto(57.79004292,118.71616462)(57.65462627,118.73699795)(57.4515013,118.73699795)
\lineto(56.79525138,118.74481045)
\lineto(56.72493889,118.80731044)
\lineto(56.72493889,119.1432479)
\lineto(56.78743888,119.20574789)
\curveto(57.78223043,119.32553954)(58.62077199,119.52085202)(59.30306357,119.79168532)
\closepath
}
}
{
\newrgbcolor{curcolor}{0 0 0}
\pscustom[linestyle=none,fillstyle=solid,fillcolor=curcolor]
{
\newpath
\moveto(66.30306271,118.47918548)
\lineto(66.30306271,118.68231045)
\lineto(66.3577502,118.76043545)
\curveto(66.84733347,118.94272709)(67.24316676,119.1119979)(67.54525006,119.26824788)
\curveto(67.54525006,120.62762271)(67.52962506,121.41928928)(67.49837506,121.64324759)
\curveto(68.03483333,121.83074757)(68.47493744,122.03126838)(68.8186874,122.24481002)
\lineto(69.00618738,122.08856003)
\curveto(68.95410405,121.76043508)(68.89160406,120.80731019)(68.8186874,119.22918539)
\curveto(69.07910403,119.22397705)(69.360354,119.22137289)(69.66243729,119.22137289)
\curveto(70.27702055,119.22137289)(70.71712466,119.23699789)(70.98274963,119.26824788)
\lineto(71.03743712,119.21356039)
\lineto(70.88899964,118.55731047)
\lineto(70.82649965,118.48699798)
\curveto(70.56087468,118.49220631)(70.26660389,118.49481048)(69.94368726,118.49481048)
\curveto(69.65202063,118.49481048)(69.27702068,118.49220631)(68.8186874,118.48699798)
\lineto(68.7718124,115.30731087)
\curveto(68.7718124,114.57293596)(68.7874374,114.09116519)(68.8186874,113.86199855)
\curveto(68.85514573,113.63804024)(68.94889572,113.46095693)(69.09993736,113.33074861)
\curveto(69.25618734,113.20574863)(69.48535398,113.14324864)(69.78743728,113.14324864)
\curveto(70.13639557,113.14324864)(70.4593122,113.23439446)(70.75618716,113.4166861)
\lineto(70.94368714,113.13543614)
\curveto(70.81868715,113.04689448)(70.51920802,112.78647785)(70.04524975,112.35418623)
\curveto(69.77441645,112.22918625)(69.49837481,112.16668626)(69.21712485,112.16668626)
\curveto(68.04524999,112.16668626)(67.45931257,112.73439452)(67.45931257,113.86981105)
\curveto(67.45931257,114.28647766)(67.46972923,114.64064429)(67.49056256,114.93231092)
\curveto(67.49577089,115.02085257)(67.49837506,115.11199839)(67.49837506,115.20574838)
\lineto(67.49837506,118.44793548)
\lineto(67.1780626,118.44793548)
\curveto(66.94368763,118.44793548)(66.6754585,118.43751882)(66.3733752,118.41668549)
\closepath
}
}
{
\newrgbcolor{curcolor}{0 0 0}
\pscustom[linestyle=none,fillstyle=solid,fillcolor=curcolor]
{
\newpath
\moveto(73.41243683,123.28387239)
\curveto(73.65722847,123.28387239)(73.86556178,123.1979349)(74.03743675,123.02605992)
\curveto(74.20931173,122.85418494)(74.29524922,122.64585163)(74.29524922,122.40106)
\curveto(74.29524922,122.16147669)(74.20931173,121.95574755)(74.03743675,121.78387257)
\curveto(73.86556178,121.61199759)(73.65722847,121.5260601)(73.41243683,121.5260601)
\curveto(73.17285353,121.5260601)(72.96452022,121.60939343)(72.78743691,121.77606007)
\curveto(72.61556193,121.94793505)(72.52962444,122.15626836)(72.52962444,122.40106)
\curveto(72.52962444,122.64585163)(72.61556193,122.85418494)(72.78743691,123.02605992)
\curveto(72.96452022,123.1979349)(73.17285353,123.28387239)(73.41243683,123.28387239)
\closepath
\moveto(74.07649925,119.79168532)
\lineto(74.22493673,119.69012283)
\curveto(74.16764507,119.00783125)(74.13899924,118.14064385)(74.13899924,117.08856065)
\lineto(74.13899924,115.32293587)
\curveto(74.13899924,115.20835255)(74.14941591,114.83856093)(74.17024924,114.21356101)
\curveto(74.19108257,113.59376942)(74.2119159,113.23179029)(74.23274923,113.12762364)
\curveto(74.25879089,113.02345699)(74.29524922,112.94793616)(74.34212422,112.90106117)
\curveto(74.38899921,112.85418617)(74.44629087,112.82293618)(74.5139992,112.80731118)
\curveto(74.58170752,112.79689451)(74.86295749,112.77606118)(75.35774909,112.74481119)
\lineto(75.42806158,112.68231119)
\lineto(75.42806158,112.35418623)
\lineto(75.36556159,112.28387374)
\curveto(74.71452,112.32554041)(74.08691592,112.34637374)(73.48274932,112.34637374)
\curveto(72.88379106,112.34637374)(72.25879114,112.32554041)(71.60774955,112.28387374)
\lineto(71.53743706,112.35418623)
\lineto(71.53743706,112.68231119)
\lineto(71.60774955,112.74481119)
\curveto(72.11295783,112.77606118)(72.39681196,112.79949868)(72.45931195,112.81512368)
\curveto(72.52702027,112.83074868)(72.58431193,112.86199867)(72.63118693,112.90887367)
\curveto(72.68327026,112.96095699)(72.71712442,113.03647782)(72.73274942,113.13543614)
\curveto(72.75358275,113.23960279)(72.77441608,113.57554025)(72.79524941,114.14324851)
\curveto(72.82129107,114.71616511)(72.8343119,115.14845672)(72.8343119,115.44012335)
\lineto(72.8343119,116.97918566)
\curveto(72.8343119,117.18751897)(72.82389524,117.4713731)(72.80306191,117.83074806)
\curveto(72.78222858,118.19012302)(72.76399941,118.41147715)(72.74837441,118.49481048)
\curveto(72.73795775,118.5781438)(72.69889525,118.63803963)(72.63118693,118.67449796)
\curveto(72.5634786,118.71616462)(72.42806195,118.73699795)(72.22493698,118.73699795)
\lineto(71.56868706,118.74481045)
\lineto(71.49837457,118.80731044)
\lineto(71.49837457,119.1432479)
\lineto(71.56087456,119.20574789)
\curveto(72.5556661,119.32553954)(73.39420767,119.52085202)(74.07649925,119.79168532)
\closepath
\moveto(73.46712433,111.87762379)
\closepath
}
}
{
\newrgbcolor{curcolor}{0 0 0}
\pscustom[linestyle=none,fillstyle=solid,fillcolor=curcolor]
{
\newpath
\moveto(78.79524867,119.26043538)
\lineto(80.73274843,119.26043538)
\curveto(81.39420668,119.26043538)(81.92545662,119.29689371)(82.32649823,119.36981037)
\curveto(82.73274818,119.44793536)(83.20931062,119.58856034)(83.75618556,119.79168532)
\lineto(83.89681054,119.69012283)
\curveto(83.84472721,119.00783125)(83.81868555,118.14064385)(83.81868555,117.08856065)
\lineto(83.81868555,115.32293587)
\curveto(83.81868555,115.20835255)(83.82910221,114.83856093)(83.84993554,114.21356101)
\curveto(83.87076888,113.59376942)(83.89160221,113.23179029)(83.91243554,113.12762364)
\curveto(83.9384772,113.02345699)(83.97493553,112.94793616)(84.02181052,112.90106117)
\curveto(84.06868552,112.85418617)(84.12597718,112.82293618)(84.1936855,112.80731118)
\curveto(84.26139383,112.79689451)(84.54264379,112.77606118)(85.0374354,112.74481119)
\lineto(85.10774789,112.68231119)
\lineto(85.10774789,112.35418623)
\lineto(85.0374354,112.28387374)
\curveto(84.39160214,112.32554041)(83.76660222,112.34637374)(83.16243563,112.34637374)
\curveto(82.5999357,112.34637374)(81.97493578,112.32554041)(81.28743586,112.28387374)
\lineto(81.21712337,112.35418623)
\lineto(81.21712337,112.68231119)
\lineto(81.28743586,112.74481119)
\curveto(81.7874358,112.77606118)(82.07128993,112.79949868)(82.13899826,112.81512368)
\curveto(82.20670658,112.83074868)(82.26399824,112.86199867)(82.31087323,112.90887367)
\curveto(82.36295656,112.96095699)(82.39681072,113.03647782)(82.41243572,113.13543614)
\curveto(82.43326905,113.23960279)(82.45410238,113.57554025)(82.47493571,114.14324851)
\curveto(82.49576905,114.71616511)(82.50618571,115.14845672)(82.50618571,115.44012335)
\lineto(82.50618571,116.97918566)
\curveto(82.50618571,117.1927273)(82.49576905,117.50783143)(82.47493571,117.92449805)
\curveto(82.45410238,118.34116466)(82.42806072,118.56772714)(82.39681072,118.60418546)
\curveto(82.36556073,118.64064379)(82.3212899,118.67189379)(82.26399824,118.69793545)
\curveto(82.21191491,118.72918545)(81.99056077,118.74481045)(81.59993582,118.74481045)
\lineto(80.84212342,118.73699795)
\lineto(80.79524842,118.51043548)
\lineto(80.71712343,118.47137298)
\lineto(78.79524867,118.47137298)
\lineto(78.79524867,115.32293587)
\curveto(78.79524867,115.15626922)(78.80566533,114.77345677)(78.82649866,114.17449851)
\curveto(78.85254033,113.58074858)(78.87597782,113.23179029)(78.89681116,113.12762364)
\curveto(78.91764449,113.02345699)(78.95149865,112.94793616)(78.99837364,112.90106117)
\curveto(79.05045697,112.85418617)(79.13118613,112.82033201)(79.24056111,112.79949868)
\curveto(79.35514443,112.77866535)(79.64941523,112.76043618)(80.1233735,112.74481119)
\lineto(80.1858735,112.68231119)
\lineto(80.1858735,112.35418623)
\lineto(80.1233735,112.28387374)
\curveto(79.95149853,112.28908208)(79.71191522,112.29949874)(79.40462359,112.31512374)
\curveto(78.99837364,112.33595707)(78.57910286,112.34637374)(78.14681125,112.34637374)
\curveto(77.58431132,112.34637374)(76.95670723,112.32554041)(76.26399898,112.28387374)
\lineto(76.20149899,112.35418623)
\lineto(76.20149899,112.68231119)
\lineto(76.26399898,112.74481119)
\curveto(76.76920725,112.77606118)(77.05566555,112.79949868)(77.12337387,112.81512368)
\curveto(77.1910822,112.83074868)(77.24837386,112.86199867)(77.29524885,112.90887367)
\curveto(77.34212385,112.96095699)(77.37337384,113.03647782)(77.38899884,113.13543614)
\curveto(77.40983217,113.23960279)(77.4306655,113.57554025)(77.45149883,114.14324851)
\curveto(77.4775405,114.71616511)(77.49056133,115.14845672)(77.49056133,115.44012335)
\lineto(77.49056133,118.45574798)
\lineto(76.48274895,118.40106049)
\lineto(76.41243646,118.46356048)
\lineto(76.41243646,118.65887296)
\lineto(76.46712396,118.73699795)
\lineto(77.49056133,119.26043538)
\lineto(77.49056133,119.72918533)
\curveto(77.49056133,120.25001859)(77.51399883,120.64324771)(77.56087382,120.90887268)
\curveto(77.60774881,121.17970598)(77.6936863,121.42710178)(77.81868629,121.65106009)
\curveto(77.94889461,121.88022673)(78.17285291,122.16668503)(78.49056121,122.51043498)
\curveto(78.8082695,122.85939327)(79.0869153,123.1432474)(79.3264986,123.36199738)
\curveto(79.56608191,123.58595568)(79.77441521,123.73439317)(79.95149853,123.80730982)
\curveto(80.12858184,123.88543481)(80.33691514,123.92449731)(80.57649845,123.92449731)
\curveto(80.77441509,123.92449731)(80.98795673,123.88543481)(81.21712337,123.80730982)
\lineto(81.20931087,122.52605998)
\lineto(81.02962339,122.45574749)
\curveto(80.73795676,122.71616412)(80.39941514,122.84637244)(80.01399852,122.84637244)
\curveto(79.74316522,122.84637244)(79.50879025,122.78908078)(79.3108736,122.67449746)
\curveto(79.11295696,122.56512248)(78.97754031,122.38283083)(78.90462365,122.12762253)
\curveto(78.831707,121.87762256)(78.79524867,121.46095595)(78.79524867,120.87762268)
\closepath
\moveto(83.09212314,123.28387239)
\curveto(83.33170644,123.28387239)(83.53743558,123.1979349)(83.70931056,123.02605992)
\curveto(83.88118554,122.85939327)(83.96712303,122.65105997)(83.96712303,122.40106)
\curveto(83.96712303,122.16147669)(83.88118554,121.95574755)(83.70931056,121.78387257)
\curveto(83.54264392,121.61199759)(83.33691477,121.5260601)(83.09212314,121.5260601)
\curveto(82.8473315,121.5260601)(82.63899819,121.60939343)(82.46712322,121.77606007)
\curveto(82.29524824,121.94793505)(82.20931075,122.15626836)(82.20931075,122.40106)
\curveto(82.20931075,122.64585163)(82.29524824,122.85418494)(82.46712322,123.02605992)
\curveto(82.64420653,123.1979349)(82.85253983,123.28387239)(83.09212314,123.28387239)
\closepath
}
}
{
\newrgbcolor{curcolor}{0 0 0}
\pscustom[linestyle=none,fillstyle=solid,fillcolor=curcolor]
{
\newpath
\moveto(85.92024779,112.28387374)
\lineto(85.8499353,112.34637374)
\lineto(85.77962281,112.6432487)
\curveto(86.45149772,113.35679028)(87.09472681,114.09897769)(87.70931007,114.86981092)
\lineto(89.43587236,117.01043566)
\curveto(89.97753896,117.66668558)(90.48274723,118.34376883)(90.95149717,119.04168541)
\lineto(89.1077474,119.04168541)
\curveto(88.96191408,119.04168541)(88.62597662,119.02866458)(88.09993502,119.00262292)
\curveto(87.57389342,118.98178958)(87.28222679,118.94012292)(87.22493513,118.87762293)
\curveto(87.16764347,118.82033127)(87.08951848,118.48178965)(86.99056016,117.86199806)
\lineto(86.94368516,117.61199809)
\lineto(86.88118517,117.54949809)
\lineto(86.56868521,117.54949809)
\lineto(86.49837272,117.61981059)
\curveto(86.55045605,118.44272715)(86.57649771,118.91928959)(86.57649771,119.04949791)
\curveto(86.57649771,119.17449789)(86.57128938,119.35158121)(86.56087271,119.58074784)
\lineto(86.6233727,119.63543534)
\curveto(88.56087246,119.59376868)(90.10514311,119.57293534)(91.25618463,119.57293534)
\curveto(91.87076789,119.57293534)(92.35514283,119.59376868)(92.70930945,119.63543534)
\lineto(92.77180944,119.57293534)
\lineto(92.77180944,119.19793539)
\curveto(92.41243449,118.80210211)(91.85253872,118.11981052)(91.09212215,117.15106064)
\lineto(89.0452474,114.54949846)
\curveto(88.44108081,113.78387356)(88.0166017,113.22918613)(87.77181006,112.88543617)
\lineto(88.31868499,112.88543617)
\lineto(89.8811848,112.90887367)
\curveto(90.4541014,112.914082)(90.93587217,112.9375195)(91.32649712,112.97918616)
\curveto(91.71712207,113.02085282)(91.94368455,113.05470698)(92.00618454,113.08074865)
\curveto(92.07389286,113.11199864)(92.11555953,113.1380403)(92.13118452,113.15887364)
\curveto(92.14680952,113.1849153)(92.19108035,113.34897778)(92.26399701,113.65106108)
\curveto(92.342122,113.95314437)(92.40201782,114.23439434)(92.44368449,114.49481097)
\lineto(92.51399698,114.55731096)
\lineto(92.86555943,114.55731096)
\lineto(92.93587192,114.48699847)
\curveto(92.81087194,113.55991525)(92.74055945,112.84637367)(92.72493445,112.34637374)
\lineto(92.66243446,112.28387374)
\curveto(91.09993465,112.33074874)(89.91503896,112.35418623)(89.1077474,112.35418623)
\curveto(88.39420582,112.35418623)(87.33170595,112.33074874)(85.92024779,112.28387374)
\closepath
}
}
{
\newrgbcolor{curcolor}{0 0 0}
\pscustom[linestyle=none,fillstyle=solid,fillcolor=curcolor]
{
\newpath
\moveto(95.74837158,123.28387239)
\curveto(95.99316321,123.28387239)(96.20149652,123.1979349)(96.3733715,123.02605992)
\curveto(96.54524648,122.85418494)(96.63118397,122.64585163)(96.63118397,122.40106)
\curveto(96.63118397,122.16147669)(96.54524648,121.95574755)(96.3733715,121.78387257)
\curveto(96.20149652,121.61199759)(95.99316321,121.5260601)(95.74837158,121.5260601)
\curveto(95.50878827,121.5260601)(95.30045497,121.60939343)(95.12337165,121.77606007)
\curveto(94.95149668,121.94793505)(94.86555919,122.15626836)(94.86555919,122.40106)
\curveto(94.86555919,122.64585163)(94.95149668,122.85418494)(95.12337165,123.02605992)
\curveto(95.30045497,123.1979349)(95.50878827,123.28387239)(95.74837158,123.28387239)
\closepath
\moveto(96.412434,119.79168532)
\lineto(96.56087148,119.69012283)
\curveto(96.50357982,119.00783125)(96.47493399,118.14064385)(96.47493399,117.08856065)
\lineto(96.47493399,115.32293587)
\curveto(96.47493399,115.20835255)(96.48535065,114.83856093)(96.50618398,114.21356101)
\curveto(96.52701732,113.59376942)(96.54785065,113.23179029)(96.56868398,113.12762364)
\curveto(96.59472564,113.02345699)(96.63118397,112.94793616)(96.67805896,112.90106117)
\curveto(96.72493396,112.85418617)(96.78222562,112.82293618)(96.84993394,112.80731118)
\curveto(96.91764227,112.79689451)(97.19889223,112.77606118)(97.69368384,112.74481119)
\lineto(97.76399633,112.68231119)
\lineto(97.76399633,112.35418623)
\lineto(97.70149634,112.28387374)
\curveto(97.05045475,112.32554041)(96.42285066,112.34637374)(95.81868407,112.34637374)
\curveto(95.21972581,112.34637374)(94.59472589,112.32554041)(93.9436843,112.28387374)
\lineto(93.87337181,112.35418623)
\lineto(93.87337181,112.68231119)
\lineto(93.9436843,112.74481119)
\curveto(94.44889257,112.77606118)(94.7327467,112.79949868)(94.7952467,112.81512368)
\curveto(94.86295502,112.83074868)(94.92024668,112.86199867)(94.96712167,112.90887367)
\curveto(95.019205,112.96095699)(95.05305916,113.03647782)(95.06868416,113.13543614)
\curveto(95.08951749,113.23960279)(95.11035082,113.57554025)(95.13118415,114.14324851)
\curveto(95.15722582,114.71616511)(95.17024665,115.14845672)(95.17024665,115.44012335)
\lineto(95.17024665,116.97918566)
\curveto(95.17024665,117.18751897)(95.15982998,117.4713731)(95.13899665,117.83074806)
\curveto(95.11816332,118.19012302)(95.09993416,118.41147715)(95.08430916,118.49481048)
\curveto(95.07389249,118.5781438)(95.03483,118.63803963)(94.96712167,118.67449796)
\curveto(94.89941335,118.71616462)(94.7639967,118.73699795)(94.56087172,118.73699795)
\lineto(93.90462181,118.74481045)
\lineto(93.83430931,118.80731044)
\lineto(93.83430931,119.1432479)
\lineto(93.89680931,119.20574789)
\curveto(94.89160085,119.32553954)(95.73014241,119.52085202)(96.412434,119.79168532)
\closepath
\moveto(95.80305907,111.87762379)
\closepath
}
}
{
\newrgbcolor{curcolor}{0 0 0}
\pscustom[linestyle=none,fillstyle=solid,fillcolor=curcolor]
{
\newpath
\moveto(105.06087043,113.49481109)
\lineto(104.81087046,112.93231116)
\curveto(104.26920386,112.58335287)(103.78482892,112.3567904)(103.35774564,112.25262375)
\curveto(102.93587069,112.14845709)(102.55826657,112.09637377)(102.22493328,112.09637377)
\curveto(101.59993336,112.09637377)(101.00618343,112.21876958)(100.4436835,112.46356122)
\curveto(99.8863919,112.70835286)(99.43066279,113.12241531)(99.07649617,113.70574857)
\curveto(98.72753788,114.28908183)(98.55305873,114.99220674)(98.55305873,115.81512331)
\curveto(98.55305873,116.36199824)(98.62076706,116.85418568)(98.75618371,117.29168563)
\curveto(98.89160036,117.7343939)(99.03222534,118.06251886)(99.17805865,118.2760605)
\curveto(99.3291003,118.48960215)(99.58170444,118.72658128)(99.93587106,118.98699792)
\curveto(100.29003768,119.24741455)(100.66503764,119.45574786)(101.06087092,119.61199784)
\curveto(101.45670421,119.76824782)(101.88378749,119.84637281)(102.34212076,119.84637281)
\curveto(102.96712069,119.84637281)(103.51659979,119.69793533)(103.99055806,119.40106037)
\curveto(104.46972467,119.10939374)(104.80566213,118.73439378)(104.99837044,118.2760605)
\curveto(105.19107875,117.81772723)(105.2874329,117.33074812)(105.2874329,116.81512318)
\curveto(105.2874329,116.65366487)(105.2796204,116.49741489)(105.2639954,116.34637324)
\lineto(105.17805792,116.26043575)
\curveto(104.82389129,116.18231076)(104.34732885,116.13022744)(103.74837059,116.10418577)
\curveto(103.14941233,116.07814411)(102.75357905,116.06512328)(102.56087074,116.06512328)
\lineto(100.05305855,116.06512328)
\curveto(100.06347521,114.98699841)(100.33430851,114.19272767)(100.86555845,113.68231107)
\curveto(101.39680838,113.17189447)(102.04784997,112.91668617)(102.81868321,112.91668617)
\curveto(103.18326649,112.91668617)(103.53222478,112.97918616)(103.86555808,113.10418614)
\curveto(104.2040997,113.22918613)(104.56087049,113.40366527)(104.93587045,113.62762358)
\closepath
\moveto(100.05305855,116.6901232)
\curveto(100.14680854,116.6744982)(100.50618349,116.65626904)(101.13118341,116.63543571)
\curveto(101.76139167,116.61460238)(102.22753745,116.60418571)(102.52962074,116.60418571)
\curveto(103.25357899,116.60418571)(103.6936831,116.61720654)(103.84993308,116.64324821)
\curveto(103.85514141,116.76824819)(103.85774558,116.86460235)(103.85774558,116.93231067)
\curveto(103.85774558,117.73960224)(103.6936831,118.3385605)(103.36555814,118.72918545)
\curveto(103.03743318,119.12501873)(102.58951657,119.32293538)(102.0218083,119.32293538)
\curveto(101.40201671,119.32293538)(100.91764177,119.10158124)(100.56868348,118.65887296)
\curveto(100.22493353,118.21616468)(100.05305855,117.55991476)(100.05305855,116.6901232)
\closepath
\moveto(102.23274578,120.28387276)
\closepath
\moveto(102.13899579,111.87762379)
\closepath
}
}
{
\newrgbcolor{curcolor}{0 0 0}
\pscustom[linestyle=none,fillstyle=solid,fillcolor=curcolor]
{
\newpath
\moveto(108.81868247,119.79168532)
\lineto(108.96711995,119.69012283)
\curveto(108.93586995,119.38803953)(108.91243245,118.85158127)(108.89680746,118.08074803)
\lineto(109.48274488,118.82293544)
\curveto(109.67545319,119.06772707)(109.84472401,119.25522705)(109.99055732,119.38543537)
\curveto(110.14159897,119.52085202)(110.31607811,119.62501867)(110.51399476,119.69793533)
\curveto(110.7119114,119.77085199)(110.91503637,119.80731032)(111.12336968,119.80731032)
\curveto(111.35253632,119.80731032)(111.56868213,119.76043532)(111.7718071,119.66668533)
\lineto(111.8264946,119.55731035)
\curveto(111.74316127,118.8646021)(111.69628628,118.25522717)(111.68586961,117.72918557)
\lineto(111.33430716,117.72918557)
\curveto(111.12597385,118.20835218)(110.79003639,118.44793548)(110.32649478,118.44793548)
\curveto(110.00357815,118.44793548)(109.72232819,118.34376883)(109.48274488,118.13543552)
\curveto(109.24316158,117.93231055)(109.08170327,117.67449808)(108.99836994,117.36199812)
\curveto(108.92024495,117.05470649)(108.88118246,116.66408154)(108.88118246,116.19012326)
\lineto(108.88118246,115.32293587)
\curveto(108.88118246,115.16668589)(108.89159912,114.78387344)(108.91243245,114.17449851)
\curveto(108.93326579,113.56512359)(108.95409912,113.21356113)(108.97493245,113.11981114)
\curveto(109.00097411,113.02606115)(109.03743244,112.95574866)(109.08430743,112.90887367)
\curveto(109.13639076,112.867207)(109.20149492,112.83856118)(109.27961991,112.82293618)
\curveto(109.36295323,112.80731118)(109.73534902,112.78126952)(110.39680727,112.74481119)
\lineto(110.46711976,112.68231119)
\lineto(110.46711976,112.35418623)
\lineto(110.39680727,112.28387374)
\curveto(109.70409902,112.32554041)(108.98014078,112.34637374)(108.22493254,112.34637374)
\curveto(107.62597428,112.34637374)(107.00097436,112.32554041)(106.34993277,112.28387374)
\lineto(106.27962028,112.35418623)
\lineto(106.27962028,112.68231119)
\lineto(106.34993277,112.74481119)
\curveto(106.85514104,112.77606118)(107.13899517,112.79949868)(107.20149517,112.81512368)
\curveto(107.26920349,112.83074868)(107.32649515,112.86199867)(107.37337014,112.90887367)
\curveto(107.42545347,112.96095699)(107.45930763,113.03647782)(107.47493263,113.13543614)
\curveto(107.49576596,113.23960279)(107.51659929,113.57554025)(107.53743262,114.14324851)
\curveto(107.56347429,114.71616511)(107.57649512,115.14845672)(107.57649512,115.44012335)
\lineto(107.57649512,116.97918566)
\curveto(107.57649512,117.18751897)(107.56607845,117.4713731)(107.54524512,117.83074806)
\curveto(107.52441179,118.19012302)(107.50618263,118.41147715)(107.49055763,118.49481048)
\curveto(107.48014096,118.5781438)(107.44107847,118.63803963)(107.37337014,118.67449796)
\curveto(107.30566182,118.71616462)(107.17024517,118.73699795)(106.96712019,118.73699795)
\lineto(106.31087028,118.74481045)
\lineto(106.24055778,118.80731044)
\lineto(106.24055778,119.1432479)
\lineto(106.30305778,119.20574789)
\curveto(107.29784932,119.32553954)(108.13639088,119.52085202)(108.81868247,119.79168532)
\closepath
}
}
{
\newrgbcolor{curcolor}{0 0 0}
\pscustom[linestyle=none,fillstyle=solid,fillcolor=curcolor]
{
\newpath
\moveto(114.84211922,119.79168532)
\lineto(114.99055671,119.69012283)
\curveto(114.93326505,119.00783125)(114.90461922,118.14064385)(114.90461922,117.08856065)
\lineto(114.90461922,115.26824838)
\curveto(114.90461922,114.60158179)(114.95409838,114.13804018)(115.0530567,113.87762355)
\curveto(115.15722335,113.61720691)(115.33430666,113.4166861)(115.58430663,113.27606112)
\curveto(115.8343066,113.14064447)(116.13118156,113.07293615)(116.47493152,113.07293615)
\curveto(116.85513981,113.07293615)(117.20149393,113.1458528)(117.51399389,113.29168612)
\curveto(117.82649386,113.44272777)(118.08691049,113.65366524)(118.2952438,113.92449854)
\curveto(118.50878544,114.19533184)(118.63118126,114.37762349)(118.66243125,114.47137347)
\curveto(118.69368125,114.56512346)(118.71451458,114.82814426)(118.72493124,115.26043588)
\lineto(118.74836874,116.13543577)
\lineto(118.74836874,116.97918566)
\curveto(118.74836874,117.18751897)(118.73795208,117.4713731)(118.71711875,117.83074806)
\curveto(118.69628541,118.19012302)(118.67805625,118.41147715)(118.66243125,118.49481048)
\curveto(118.65201459,118.5781438)(118.61295209,118.63803963)(118.54524377,118.67449796)
\curveto(118.47753544,118.71616462)(118.34211879,118.73699795)(118.13899382,118.73699795)
\lineto(117.4827439,118.74481045)
\lineto(117.41243141,118.80731044)
\lineto(117.41243141,119.1432479)
\lineto(117.4749314,119.20574789)
\curveto(118.46972294,119.32553954)(119.30826451,119.52085202)(119.99055609,119.79168532)
\lineto(120.13899357,119.69012283)
\curveto(120.08170191,119.00783125)(120.05305608,118.14064385)(120.05305608,117.08856065)
\lineto(120.05305608,115.71356082)
\curveto(120.05305608,115.63543583)(120.06607691,114.99741508)(120.09211858,113.79949856)
\curveto(120.10253524,113.35679028)(120.12857691,113.08595698)(120.17024357,112.98699866)
\curveto(120.21711856,112.89324867)(120.27961855,112.82554034)(120.35774354,112.78387368)
\curveto(120.43586853,112.74741535)(120.66763934,112.72918619)(121.05305596,112.72918619)
\lineto(121.27180593,112.72918619)
\lineto(121.34211842,112.6666862)
\lineto(121.34211842,112.35418623)
\lineto(121.27961843,112.28387374)
\curveto(120.54524352,112.32554041)(120.05305608,112.34637374)(119.80305611,112.34637374)
\curveto(119.48534782,112.34637374)(119.11816036,112.32814457)(118.70149375,112.29168624)
\lineto(118.63118126,112.35418623)
\curveto(118.66243125,112.88543617)(118.68586875,113.33595695)(118.70149375,113.70574857)
\curveto(118.37336879,113.45054027)(117.99836883,113.10939448)(117.57649389,112.68231119)
\curveto(117.41503557,112.52085288)(117.18326477,112.38543623)(116.88118147,112.27606124)
\curveto(116.57909818,112.16668626)(116.23534822,112.11199876)(115.8499316,112.11199876)
\curveto(115.26659834,112.11199876)(114.81086923,112.20314459)(114.48274427,112.38543623)
\curveto(114.15982764,112.57293621)(113.930661,112.82814451)(113.79524435,113.15106114)
\curveto(113.6598277,113.4791861)(113.59211938,114.04168603)(113.59211938,114.83856093)
\lineto(113.59993188,115.45574835)
\lineto(113.59993188,116.97918566)
\curveto(113.59993188,117.18751897)(113.58951521,117.4713731)(113.56868188,117.83074806)
\curveto(113.54784855,118.19012302)(113.52961939,118.41147715)(113.51399439,118.49481048)
\curveto(113.50357772,118.5781438)(113.46451523,118.63803963)(113.3968069,118.67449796)
\curveto(113.32909858,118.71616462)(113.19368193,118.73699795)(112.99055695,118.73699795)
\lineto(112.33430703,118.74481045)
\lineto(112.26399454,118.80731044)
\lineto(112.26399454,119.1432479)
\lineto(112.32649453,119.20574789)
\curveto(113.32128608,119.32553954)(114.15982764,119.52085202)(114.84211922,119.79168532)
\closepath
\moveto(116.70930649,120.28387276)
\closepath
\moveto(116.78743148,111.87762379)
\closepath
}
}
{
\newrgbcolor{curcolor}{0 0 0}
\pscustom[linestyle=none,fillstyle=solid,fillcolor=curcolor]
{
\newpath
\moveto(124.45930554,119.79168532)
\lineto(124.60774302,119.69012283)
\curveto(124.57649302,119.34116454)(124.55565969,118.88283126)(124.54524303,118.315123)
\curveto(124.96190964,118.65887296)(125.35774293,119.00783125)(125.73274288,119.36199787)
\curveto(125.84211787,119.46095619)(125.94368035,119.53387285)(126.03743034,119.58074784)
\curveto(126.13638866,119.62762284)(126.29784698,119.677102)(126.52180528,119.72918533)
\curveto(126.74576359,119.78126865)(126.97493023,119.80731032)(127.2093052,119.80731032)
\curveto(127.60513848,119.80731032)(127.98795094,119.72658116)(128.35774256,119.56512285)
\curveto(128.73274251,119.40366453)(129.01399248,119.21095622)(129.20149245,118.98699792)
\curveto(129.39420076,118.76824794)(129.52701325,118.50783131)(129.5999299,118.20574801)
\curveto(129.67284656,117.90366472)(129.70930489,117.53126893)(129.70930489,117.08856065)
\lineto(129.70930489,115.71356082)
\curveto(129.70930489,115.62501916)(129.72232572,114.97658174)(129.74836739,113.76824856)
\curveto(129.75878405,113.27345695)(129.81347154,112.97918616)(129.91242987,112.88543617)
\curveto(130.01138819,112.79168618)(130.33430481,112.74481119)(130.88117975,112.74481119)
\lineto(130.94367974,112.68231119)
\lineto(130.94367974,112.34637374)
\lineto(130.88117975,112.28387374)
\curveto(130.21972149,112.32554041)(129.77180488,112.34637374)(129.53742991,112.34637374)
\curveto(129.40201326,112.34637374)(129.02180498,112.32554041)(128.39680505,112.28387374)
\lineto(128.30305506,112.36981123)
\curveto(128.37076339,113.04689448)(128.40461755,113.92970687)(128.40461755,115.01824841)
\lineto(128.40461755,116.04168578)
\curveto(128.40461755,116.65626904)(128.38899255,117.09897732)(128.35774256,117.36981062)
\curveto(128.33170089,117.64064392)(128.2457634,117.88543555)(128.09993009,118.10418553)
\curveto(127.95409677,118.32814383)(127.7587843,118.50001881)(127.51399266,118.61981046)
\curveto(127.26920102,118.74481045)(126.97753439,118.80731044)(126.63899277,118.80731044)
\curveto(126.36815947,118.80731044)(126.13638866,118.77866461)(125.94368035,118.72137295)
\curveto(125.75097205,118.66928962)(125.53482624,118.55731047)(125.29524293,118.38543549)
\curveto(125.05565963,118.21876885)(124.87857632,118.0599147)(124.763993,117.90887305)
\curveto(124.64940968,117.76303973)(124.57909719,117.62501892)(124.55305553,117.4948106)
\curveto(124.5322222,117.36981062)(124.52180553,117.09897732)(124.52180553,116.6823107)
\lineto(124.52180553,115.32293587)
\curveto(124.52180553,115.20835255)(124.5322222,114.83856093)(124.55305553,114.21356101)
\curveto(124.57388886,113.59376942)(124.59472219,113.23179029)(124.61555552,113.12762364)
\curveto(124.64159718,113.02345699)(124.67805551,112.94793616)(124.7249305,112.90106117)
\curveto(124.7718055,112.85418617)(124.82909716,112.82293618)(124.89680548,112.80731118)
\curveto(124.96451381,112.79689451)(125.24576377,112.77606118)(125.74055538,112.74481119)
\lineto(125.81086787,112.68231119)
\lineto(125.81086787,112.35418623)
\lineto(125.74836788,112.28387374)
\curveto(125.09732629,112.32554041)(124.4697222,112.34637374)(123.86555561,112.34637374)
\curveto(123.26659735,112.34637374)(122.64159743,112.32554041)(121.99055584,112.28387374)
\lineto(121.92024335,112.35418623)
\lineto(121.92024335,112.68231119)
\lineto(121.99055584,112.74481119)
\curveto(122.49576411,112.77606118)(122.77961824,112.79949868)(122.84211824,112.81512368)
\curveto(122.90982656,112.83074868)(122.96711822,112.86199867)(123.01399322,112.90887367)
\curveto(123.06607654,112.96095699)(123.09993071,113.03647782)(123.1155557,113.13543614)
\curveto(123.13638903,113.23960279)(123.15722236,113.57554025)(123.1780557,114.14324851)
\curveto(123.20409736,114.71616511)(123.21711819,115.14845672)(123.21711819,115.44012335)
\lineto(123.21711819,116.97918566)
\curveto(123.21711819,117.18751897)(123.20670153,117.4713731)(123.18586819,117.83074806)
\curveto(123.16503486,118.19012302)(123.1468057,118.41147715)(123.1311807,118.49481048)
\curveto(123.12076404,118.5781438)(123.08170154,118.63803963)(123.01399322,118.67449796)
\curveto(122.94628489,118.71616462)(122.81086824,118.73699795)(122.60774327,118.73699795)
\lineto(121.95149335,118.74481045)
\lineto(121.88118086,118.80731044)
\lineto(121.88118086,119.1432479)
\lineto(121.94368085,119.20574789)
\curveto(122.93847239,119.32553954)(123.77701396,119.52085202)(124.45930554,119.79168532)
\closepath
}
}
{
\newrgbcolor{curcolor}{0 0 0}
\pscustom[linestyle=none,fillstyle=solid,fillcolor=curcolor]
{
\newpath
\moveto(139.71711616,119.25262288)
\lineto(139.77961615,119.1119979)
\curveto(139.56607451,118.78387294)(139.43326202,118.5703313)(139.3811787,118.47137298)
\lineto(137.92805388,118.47137298)
\curveto(138.07388719,118.18491468)(138.14680385,117.88543555)(138.14680385,117.57293559)
\curveto(138.14680385,117.20314397)(138.06086636,116.84376901)(137.88899138,116.49481072)
\curveto(137.72232474,116.14585243)(137.48794976,115.84897747)(137.18586647,115.60418583)
\curveto(136.88378317,115.3593942)(136.54003321,115.16408172)(136.1546166,115.01824841)
\curveto(135.77440831,114.87241509)(135.33951253,114.79949843)(134.84992926,114.79949843)
\lineto(134.4905543,114.79949843)
\curveto(134.18326267,114.5547068)(133.98534603,114.37241515)(133.89680437,114.2526235)
\curveto(133.80826272,114.13283185)(133.76399189,114.01043603)(133.76399189,113.88543605)
\curveto(133.76399189,113.65626941)(133.86815854,113.49481109)(134.07649185,113.40106111)
\curveto(134.29003349,113.30731112)(134.72753344,113.26043612)(135.38899169,113.26043612)
\lineto(137.13899147,113.28387362)
\curveto(137.69107474,113.28387362)(138.11294969,113.21876946)(138.40461632,113.08856114)
\curveto(138.70149128,112.95835283)(138.94107459,112.72397786)(139.12336623,112.38543623)
\curveto(139.30565787,112.05210294)(139.3968037,111.69793632)(139.3968037,111.32293636)
\curveto(139.3968037,110.75001977)(139.20930372,110.17970734)(138.83430377,109.61199907)
\curveto(138.46451214,109.03908248)(137.92805388,108.6041867)(137.22492896,108.30731173)
\curveto(136.52180405,108.01043677)(135.77180414,107.86199929)(134.97492924,107.86199929)
\curveto(134.5218043,107.86199929)(134.09472102,107.91147845)(133.6936794,108.01043677)
\curveto(133.29784611,108.10939509)(132.94888782,108.26043674)(132.64680453,108.46356171)
\curveto(132.34992956,108.66147836)(132.11034626,108.92189499)(131.92805462,109.24481162)
\curveto(131.74576297,109.56772825)(131.65461715,109.9036657)(131.65461715,110.25262399)
\curveto(131.65461715,110.4453323)(131.68326298,110.64845728)(131.74055464,110.86199892)
\curveto(131.7978463,111.07033223)(131.90722129,111.2968947)(132.0686796,111.54168634)
\lineto(133.48274193,112.33074874)
\curveto(133.02961698,112.47137372)(132.74315868,112.60939454)(132.62336703,112.74481119)
\curveto(132.50878371,112.88543617)(132.45149205,113.05210282)(132.45149205,113.24481113)
\curveto(132.45149205,113.44272777)(132.51659621,113.67710274)(132.64680453,113.94793604)
\lineto(133.90461687,114.85418593)
\curveto(133.16503363,115.04689424)(132.66503369,115.3359567)(132.40461706,115.72137332)
\curveto(132.14940876,116.10678994)(132.02180461,116.52866489)(132.02180461,116.98699816)
\curveto(132.02180461,117.40887311)(132.11295043,117.80991473)(132.29524207,118.19012302)
\curveto(132.47753372,118.57553963)(132.75097118,118.88283126)(133.11555447,119.1119979)
\curveto(133.48534609,119.34116454)(133.89680437,119.52085202)(134.34992932,119.65106034)
\curveto(134.80826259,119.78647699)(135.22492921,119.85418531)(135.59992916,119.85418531)
\curveto(136.25097075,119.85418531)(136.87076234,119.64324784)(137.45930393,119.22137289)
\curveto(138.40201215,119.22137289)(139.15461623,119.23178955)(139.71711616,119.25262288)
\closepath
\moveto(133.36555444,117.44012311)
\curveto(133.36555444,117.12241481)(133.42805443,116.78126902)(133.55305442,116.41668573)
\curveto(133.6780544,116.05210245)(133.87857521,115.77606081)(134.15461684,115.58856084)
\curveto(134.43586681,115.40106086)(134.75097094,115.30731087)(135.09992923,115.30731087)
\curveto(135.56347084,115.30731087)(135.96190829,115.45835252)(136.29524158,115.76043581)
\curveto(136.6337832,116.06251911)(136.80305402,116.53387322)(136.80305402,117.17449814)
\curveto(136.80305402,117.71616474)(136.6494082,118.20314385)(136.34211657,118.63543546)
\curveto(136.03482494,119.06772707)(135.59992916,119.28387288)(135.03742923,119.28387288)
\curveto(134.56347096,119.28387288)(134.16503351,119.1276229)(133.84211688,118.81512294)
\curveto(133.52440859,118.50783131)(133.36555444,118.04949803)(133.36555444,117.44012311)
\closepath
\moveto(135.65461666,112.19793625)
\curveto(134.81607509,112.19793625)(134.33170015,112.18751959)(134.20149184,112.16668626)
\curveto(134.07649185,112.14585293)(133.89680437,112.05731127)(133.6624294,111.90106129)
\curveto(133.42805443,111.74481131)(133.23534612,111.53387384)(133.08430447,111.26824887)
\curveto(132.93847116,110.99741557)(132.8655545,110.69533227)(132.8655545,110.36199898)
\curveto(132.8655545,109.77345739)(133.08430447,109.29429078)(133.52180442,108.92449916)
\curveto(133.95930437,108.55470754)(134.54263763,108.36981173)(135.2718042,108.36981173)
\curveto(135.81867914,108.36981173)(136.32388741,108.48439505)(136.78742902,108.71356168)
\curveto(137.25097063,108.94272832)(137.59211642,109.25001995)(137.81086639,109.63543657)
\curveto(138.0348247,110.02085319)(138.14680385,110.39064481)(138.14680385,110.74481143)
\curveto(138.14680385,111.09897806)(138.05305386,111.40366552)(137.86555388,111.65887382)
\curveto(137.68326224,111.90887379)(137.44367894,112.0625196)(137.14680397,112.11981126)
\curveto(136.85513734,112.17189459)(136.35774157,112.19793625)(135.65461666,112.19793625)
\closepath
}
}
{
\newrgbcolor{curcolor}{0 0 0}
\pscustom[linestyle=none,fillstyle=solid,fillcolor=curcolor]
{
}
}
{
\newrgbcolor{curcolor}{0 0 0}
\pscustom[linestyle=none,fillstyle=solid,fillcolor=curcolor]
{
\newpath
\moveto(147.98274014,112.20574875)
\lineto(147.4983652,113.4323111)
\lineto(146.92024027,114.76824844)
\lineto(145.31867797,118.49481048)
\curveto(145.21451131,118.73960211)(145.12076132,118.90106043)(145.037428,118.97918542)
\curveto(144.95930301,119.05731041)(144.87857385,119.1041854)(144.79524053,119.1198104)
\curveto(144.71190721,119.1354354)(144.54524056,119.14845623)(144.29524059,119.1588729)
\lineto(144.2327406,119.22137289)
\lineto(144.2327406,119.55731035)
\lineto(144.30305309,119.62762284)
\curveto(145.03221967,119.59116451)(145.6103446,119.57293534)(146.03742788,119.57293534)
\curveto(146.64159447,119.57293534)(147.24836523,119.59116451)(147.85774015,119.62762284)
\lineto(147.92024014,119.56512285)
\lineto(147.92024014,119.22137289)
\lineto(147.85774015,119.1588729)
\curveto(147.31607355,119.1432479)(147.00096942,119.1041854)(146.91242777,119.04168541)
\curveto(146.82388611,118.98439375)(146.77961529,118.90887293)(146.77961529,118.81512294)
\curveto(146.77961529,118.65887296)(146.88378194,118.3229355)(147.09211525,117.80731056)
\lineto(148.04524013,115.43231086)
\lineto(148.79524004,113.65887357)
\lineto(149.57648994,115.43231086)
\lineto(150.22492736,117.04168566)
\curveto(150.47492733,117.65106058)(150.63638564,118.0833522)(150.7093023,118.3385605)
\curveto(150.78221896,118.5937688)(150.81867729,118.76303961)(150.81867729,118.84637293)
\curveto(150.81867729,118.94533126)(150.76138563,119.01564375)(150.64680231,119.05731041)
\curveto(150.53221899,119.1041854)(150.24315653,119.13803957)(149.77961492,119.1588729)
\lineto(149.71711492,119.22137289)
\lineto(149.71711492,119.55731035)
\lineto(149.78742741,119.62762284)
\curveto(150.43326067,119.59116451)(150.91763561,119.57293534)(151.24055224,119.57293534)
\curveto(151.51138554,119.57293534)(151.97492714,119.59116451)(152.63117706,119.62762284)
\lineto(152.70148956,119.55731035)
\lineto(152.70148956,119.22137289)
\lineto(152.63898956,119.1588729)
\curveto(152.33169793,119.13283123)(152.12596879,119.0885604)(152.02180214,119.02606041)
\curveto(151.92284382,118.96356042)(151.8238855,118.8411646)(151.72492718,118.65887296)
\curveto(151.63117719,118.48178965)(151.35513555,117.92449805)(150.89680228,116.98699816)
\lineto(150.14680237,115.38543586)
\curveto(150.02180239,115.11460256)(149.57128161,114.05470686)(148.79524004,112.20574875)
\closepath
}
}
{
\newrgbcolor{curcolor}{0 0 0}
\pscustom[linestyle=none,fillstyle=solid,fillcolor=curcolor]
{
\newpath
\moveto(153.56867695,115.92449829)
\curveto(153.56867695,116.54428988)(153.6962811,117.14064398)(153.9514894,117.71356057)
\curveto(154.2066977,118.2916855)(154.64940598,118.78908127)(155.27961424,119.20574789)
\curveto(155.91503083,119.62762284)(156.6676349,119.83856031)(157.53742646,119.83856031)
\curveto(158.63117632,119.83856031)(159.52180121,119.49220619)(160.20930113,118.79949794)
\curveto(160.89680104,118.11199802)(161.240551,117.22658147)(161.240551,116.14324827)
\curveto(161.240551,114.95054008)(160.84471772,113.96876937)(160.05305115,113.19793613)
\curveto(159.26659291,112.42710289)(158.3160722,112.04168627)(157.201489,112.04168627)
\curveto(156.47232242,112.04168627)(155.82128084,112.23699875)(155.24836424,112.6276237)
\curveto(154.67544765,113.02345699)(154.25096853,113.50522776)(153.9749269,114.07293602)
\curveto(153.7040936,114.64064429)(153.56867695,115.25783171)(153.56867695,115.92449829)
\closepath
\moveto(155.04523927,116.46356073)
\curveto(155.04523927,115.73439415)(155.13898925,115.0807484)(155.32648923,114.50262347)
\curveto(155.51398921,113.92970687)(155.81086417,113.46876943)(156.21711412,113.11981114)
\curveto(156.62336407,112.77085285)(157.08690568,112.59637371)(157.60773895,112.59637371)
\curveto(158.22232221,112.59637371)(158.73013465,112.83856118)(159.13117626,113.32293612)
\curveto(159.53221788,113.81251939)(159.73273869,114.5468943)(159.73273869,115.52606084)
\curveto(159.73273869,116.63543571)(159.51398872,117.54428976)(159.07648877,118.25262301)
\curveto(158.64419716,118.96095625)(158.01919723,119.31512288)(157.201489,119.31512288)
\curveto(156.52440575,119.31512288)(155.99575998,119.07033124)(155.6155517,118.58074797)
\curveto(155.23534341,118.09116469)(155.04523927,117.38543561)(155.04523927,116.46356073)
\closepath
\moveto(157.41242647,120.28387276)
\closepath
\moveto(157.29523899,111.87762379)
\closepath
}
}
{
\newrgbcolor{curcolor}{0 0 0}
\pscustom[linestyle=none,fillstyle=solid,fillcolor=curcolor]
{
\newpath
\moveto(164.44367561,119.79168532)
\lineto(164.59211309,119.69012283)
\curveto(164.56086309,119.34116454)(164.54002976,118.88283126)(164.5296131,118.315123)
\curveto(164.94627971,118.65887296)(165.342113,119.00783125)(165.71711295,119.36199787)
\curveto(165.82648794,119.46095619)(165.92805042,119.53387285)(166.02180041,119.58074784)
\curveto(166.12075873,119.62762284)(166.28221705,119.677102)(166.50617535,119.72918533)
\curveto(166.73013366,119.78126865)(166.9593003,119.80731032)(167.19367527,119.80731032)
\curveto(167.58950855,119.80731032)(167.97232101,119.72658116)(168.34211263,119.56512285)
\curveto(168.71711258,119.40366453)(168.99836255,119.21095622)(169.18586252,118.98699792)
\curveto(169.37857083,118.76824794)(169.51138332,118.50783131)(169.58429997,118.20574801)
\curveto(169.65721663,117.90366472)(169.69367496,117.53126893)(169.69367496,117.08856065)
\lineto(169.69367496,115.71356082)
\curveto(169.69367496,115.62501916)(169.70669579,114.97658174)(169.73273746,113.76824856)
\curveto(169.74315412,113.27345695)(169.79784161,112.97918616)(169.89679994,112.88543617)
\curveto(169.99575826,112.79168618)(170.31867488,112.74481119)(170.86554982,112.74481119)
\lineto(170.92804981,112.68231119)
\lineto(170.92804981,112.34637374)
\lineto(170.86554982,112.28387374)
\curveto(170.20409156,112.32554041)(169.75617495,112.34637374)(169.52179998,112.34637374)
\curveto(169.38638333,112.34637374)(169.00617504,112.32554041)(168.38117512,112.28387374)
\lineto(168.28742513,112.36981123)
\curveto(168.35513346,113.04689448)(168.38898762,113.92970687)(168.38898762,115.01824841)
\lineto(168.38898762,116.04168578)
\curveto(168.38898762,116.65626904)(168.37336262,117.09897732)(168.34211263,117.36981062)
\curveto(168.31607096,117.64064392)(168.23013347,117.88543555)(168.08430016,118.10418553)
\curveto(167.93846684,118.32814383)(167.74315437,118.50001881)(167.49836273,118.61981046)
\curveto(167.25357109,118.74481045)(166.96190446,118.80731044)(166.62336284,118.80731044)
\curveto(166.35252954,118.80731044)(166.12075873,118.77866461)(165.92805042,118.72137295)
\curveto(165.73534211,118.66928962)(165.51919631,118.55731047)(165.279613,118.38543549)
\curveto(165.0400297,118.21876885)(164.86294639,118.0599147)(164.74836307,117.90887305)
\curveto(164.63377975,117.76303973)(164.56346726,117.62501892)(164.5374256,117.4948106)
\curveto(164.51659227,117.36981062)(164.5061756,117.09897732)(164.5061756,116.6823107)
\lineto(164.5061756,115.32293587)
\curveto(164.5061756,115.20835255)(164.51659227,114.83856093)(164.5374256,114.21356101)
\curveto(164.55825893,113.59376942)(164.57909226,113.23179029)(164.59992559,113.12762364)
\curveto(164.62596725,113.02345699)(164.66242558,112.94793616)(164.70930057,112.90106117)
\curveto(164.75617557,112.85418617)(164.81346723,112.82293618)(164.88117555,112.80731118)
\curveto(164.94888388,112.79689451)(165.23013384,112.77606118)(165.72492545,112.74481119)
\lineto(165.79523794,112.68231119)
\lineto(165.79523794,112.35418623)
\lineto(165.73273795,112.28387374)
\curveto(165.08169636,112.32554041)(164.45409227,112.34637374)(163.84992568,112.34637374)
\curveto(163.25096742,112.34637374)(162.6259675,112.32554041)(161.97492591,112.28387374)
\lineto(161.90461342,112.35418623)
\lineto(161.90461342,112.68231119)
\lineto(161.97492591,112.74481119)
\curveto(162.48013418,112.77606118)(162.76398831,112.79949868)(162.82648831,112.81512368)
\curveto(162.89419663,112.83074868)(162.95148829,112.86199867)(162.99836329,112.90887367)
\curveto(163.05044661,112.96095699)(163.08430078,113.03647782)(163.09992577,113.13543614)
\curveto(163.1207591,113.23960279)(163.14159243,113.57554025)(163.16242577,114.14324851)
\curveto(163.18846743,114.71616511)(163.20148826,115.14845672)(163.20148826,115.44012335)
\lineto(163.20148826,116.97918566)
\curveto(163.20148826,117.18751897)(163.1910716,117.4713731)(163.17023826,117.83074806)
\curveto(163.14940493,118.19012302)(163.13117577,118.41147715)(163.11555077,118.49481048)
\curveto(163.10513411,118.5781438)(163.06607161,118.63803963)(162.99836329,118.67449796)
\curveto(162.93065496,118.71616462)(162.79523831,118.73699795)(162.59211334,118.73699795)
\lineto(161.93586342,118.74481045)
\lineto(161.86555093,118.80731044)
\lineto(161.86555093,119.1432479)
\lineto(161.92805092,119.20574789)
\curveto(162.92284246,119.32553954)(163.76138402,119.52085202)(164.44367561,119.79168532)
\closepath
}
}
{
\newrgbcolor{curcolor}{0 0 0}
\pscustom[linestyle=none,fillstyle=solid,fillcolor=curcolor]
{
}
}
{
\newrgbcolor{curcolor}{0 0 0}
\pscustom[linestyle=none,fillstyle=solid,fillcolor=curcolor]
{
\newpath
\moveto(175.65461173,123.33074738)
\lineto(175.71711172,123.38543487)
\curveto(177.50356983,123.35418488)(178.64679886,123.33855988)(179.14679879,123.33855988)
\curveto(180.85513192,123.33855988)(182.37596506,123.35418488)(183.70929823,123.38543487)
\lineto(183.74836073,123.33074738)
\curveto(183.61815241,122.83074744)(183.52179826,122.04689337)(183.45929826,120.97918517)
\lineto(183.41242327,120.91668518)
\lineto(183.02961082,120.91668518)
\lineto(182.95929832,120.97918517)
\curveto(182.96450666,121.02606017)(182.96711082,121.06772683)(182.96711082,121.10418516)
\curveto(182.96711082,121.54689343)(182.93325666,121.96876838)(182.86554834,122.36981)
\curveto(182.53742338,122.51564332)(182.08169427,122.6093933)(181.498361,122.65105997)
\curveto(180.92023608,122.69793496)(180.4853403,122.72137246)(180.19367367,122.72137246)
\curveto(179.78221538,122.72137246)(179.25617378,122.69272663)(178.61554886,122.63543497)
\lineto(178.58429886,121.38543512)
\lineto(178.56086137,119.9557478)
\lineto(178.53742387,118.36981049)
\curveto(179.0999238,118.3385605)(179.6181529,118.3229355)(180.09211118,118.3229355)
\curveto(180.57127779,118.3229355)(180.9723194,118.3385605)(181.29523603,118.36981049)
\curveto(181.61815266,118.40106049)(181.80565263,118.43231049)(181.85773596,118.46356048)
\curveto(181.91502762,118.50001881)(181.95669428,118.56512297)(181.98273594,118.65887296)
\curveto(182.01398594,118.75783128)(182.0400276,118.92970626)(182.06086094,119.17449789)
\lineto(182.09211093,119.65887283)
\lineto(182.15461092,119.72137283)
\lineto(182.51398588,119.72137283)
\lineto(182.57648587,119.65887283)
\curveto(182.56606921,119.09116457)(182.55304837,118.52085214)(182.53742338,117.94793555)
\lineto(182.57648587,116.19012326)
\lineto(182.51398588,116.12762327)
\lineto(182.15461092,116.12762327)
\lineto(182.09211093,116.19012326)
\lineto(182.03742344,116.6588732)
\curveto(181.99054844,117.02345649)(181.95669428,117.2396023)(181.93586095,117.30731062)
\curveto(181.91502762,117.38022728)(181.86554846,117.44272727)(181.78742347,117.4948106)
\curveto(181.71450681,117.54689393)(181.5217985,117.58856059)(181.20929854,117.61981059)
\curveto(180.89679858,117.65106058)(180.46711113,117.66668558)(179.9202362,117.66668558)
\curveto(179.6181529,117.66668558)(179.15721546,117.65366475)(178.53742387,117.62762308)
\curveto(178.51138221,117.19012314)(178.49836137,116.46095656)(178.49836137,115.44012335)
\curveto(178.49836137,114.34637349)(178.51138221,113.54168609)(178.53742387,113.02606115)
\lineto(179.97492369,113.02606115)
\curveto(181.24054854,113.02606115)(182.03221511,113.04168615)(182.3499234,113.07293615)
\curveto(182.66763169,113.10418614)(182.96971499,113.17970697)(183.25617329,113.29949862)
\curveto(183.30304828,113.41929027)(183.37856911,113.6953319)(183.48273576,114.12762352)
\curveto(183.59211075,114.56512346)(183.65461074,114.83856093)(183.67023574,114.94793592)
\lineto(183.74836073,115.00262341)
\lineto(184.12336068,115.00262341)
\lineto(184.18586067,114.94793592)
\curveto(184.13898568,114.72397761)(184.09211068,114.33335266)(184.04523569,113.77606106)
\curveto(184.00356903,113.21876946)(183.9827357,112.74220702)(183.9827357,112.34637374)
\lineto(183.9280482,112.28387374)
\curveto(182.93325666,112.31512374)(181.63117349,112.33074874)(180.02179869,112.33074874)
\lineto(177.89679895,112.34637374)
\curveto(177.40200734,112.34637374)(176.88638241,112.32554041)(176.34992414,112.28387374)
\lineto(176.28742415,112.34637374)
\lineto(176.28742415,112.6041862)
\lineto(176.34992414,112.68231119)
\curveto(176.65721577,112.83856118)(176.83169491,112.93491533)(176.87336157,112.97137366)
\curveto(176.92023657,113.01304032)(176.9566949,113.46876943)(176.98273656,114.33856099)
\curveto(177.01398656,115.21356088)(177.02961156,115.88022747)(177.02961156,116.33856074)
\lineto(177.02961156,119.20574789)
\lineto(177.02179906,120.97918517)
\curveto(177.02179906,121.36460179)(177.01659072,121.67710175)(177.00617406,121.91668506)
\curveto(176.99575739,122.16147669)(176.97752823,122.33595584)(176.95148657,122.44012249)
\curveto(176.9254449,122.54428915)(176.89419491,122.6172058)(176.85773658,122.65887246)
\curveto(176.82648658,122.70053913)(176.77179909,122.73439329)(176.6936741,122.76043495)
\curveto(176.62075744,122.79168495)(176.49315329,122.81772661)(176.31086164,122.83855994)
\lineto(175.71711172,122.88543494)
\lineto(175.65461173,122.94012243)
\closepath
\moveto(180.11554868,123.9635598)
\closepath
\moveto(180.08429868,111.87762379)
\closepath
}
}
{
\newrgbcolor{curcolor}{0 0 0}
\pscustom[linestyle=none,fillstyle=solid,fillcolor=curcolor]
{
\newpath
\moveto(187.10773531,123.28387239)
\curveto(187.35252695,123.28387239)(187.56086026,123.1979349)(187.73273524,123.02605992)
\curveto(187.90461021,122.85418494)(187.9905477,122.64585163)(187.9905477,122.40106)
\curveto(187.9905477,122.16147669)(187.90461021,121.95574755)(187.73273524,121.78387257)
\curveto(187.56086026,121.61199759)(187.35252695,121.5260601)(187.10773531,121.5260601)
\curveto(186.86815201,121.5260601)(186.6598187,121.60939343)(186.48273539,121.77606007)
\curveto(186.31086041,121.94793505)(186.22492292,122.15626836)(186.22492292,122.40106)
\curveto(186.22492292,122.64585163)(186.31086041,122.85418494)(186.48273539,123.02605992)
\curveto(186.6598187,123.1979349)(186.86815201,123.28387239)(187.10773531,123.28387239)
\closepath
\moveto(187.77179773,119.79168532)
\lineto(187.92023521,119.69012283)
\curveto(187.86294355,119.00783125)(187.83429772,118.14064385)(187.83429772,117.08856065)
\lineto(187.83429772,115.32293587)
\curveto(187.83429772,115.20835255)(187.84471439,114.83856093)(187.86554772,114.21356101)
\curveto(187.88638105,113.59376942)(187.90721438,113.23179029)(187.92804771,113.12762364)
\curveto(187.95408938,113.02345699)(187.9905477,112.94793616)(188.0374227,112.90106117)
\curveto(188.08429769,112.85418617)(188.14158935,112.82293618)(188.20929768,112.80731118)
\curveto(188.277006,112.79689451)(188.55825597,112.77606118)(189.05304757,112.74481119)
\lineto(189.12336006,112.68231119)
\lineto(189.12336006,112.35418623)
\lineto(189.06086007,112.28387374)
\curveto(188.40981849,112.32554041)(187.7822144,112.34637374)(187.1780478,112.34637374)
\curveto(186.57908954,112.34637374)(185.95408962,112.32554041)(185.30304804,112.28387374)
\lineto(185.23273554,112.35418623)
\lineto(185.23273554,112.68231119)
\lineto(185.30304804,112.74481119)
\curveto(185.80825631,112.77606118)(186.09211044,112.79949868)(186.15461043,112.81512368)
\curveto(186.22231876,112.83074868)(186.27961042,112.86199867)(186.32648541,112.90887367)
\curveto(186.37856874,112.96095699)(186.4124229,113.03647782)(186.4280479,113.13543614)
\curveto(186.44888123,113.23960279)(186.46971456,113.57554025)(186.49054789,114.14324851)
\curveto(186.51658955,114.71616511)(186.52961038,115.14845672)(186.52961038,115.44012335)
\lineto(186.52961038,116.97918566)
\curveto(186.52961038,117.18751897)(186.51919372,117.4713731)(186.49836039,117.83074806)
\curveto(186.47752706,118.19012302)(186.45929789,118.41147715)(186.44367289,118.49481048)
\curveto(186.43325623,118.5781438)(186.39419373,118.63803963)(186.32648541,118.67449796)
\curveto(186.25877708,118.71616462)(186.12336043,118.73699795)(185.92023546,118.73699795)
\lineto(185.26398554,118.74481045)
\lineto(185.19367305,118.80731044)
\lineto(185.19367305,119.1432479)
\lineto(185.25617304,119.20574789)
\curveto(186.25096459,119.32553954)(187.08950615,119.52085202)(187.77179773,119.79168532)
\closepath
\moveto(187.16242281,111.87762379)
\closepath
}
}
{
\newrgbcolor{curcolor}{0 0 0}
\pscustom[linestyle=none,fillstyle=solid,fillcolor=curcolor]
{
\newpath
\moveto(192.18585969,119.79168532)
\lineto(192.33429717,119.69012283)
\curveto(192.30304717,119.34116454)(192.28221384,118.88283126)(192.27179718,118.315123)
\curveto(192.68846379,118.65887296)(193.08429708,119.00783125)(193.45929703,119.36199787)
\curveto(193.56867202,119.46095619)(193.6702345,119.53387285)(193.76398449,119.58074784)
\curveto(193.86294281,119.62762284)(194.02440113,119.677102)(194.24835943,119.72918533)
\curveto(194.47231774,119.78126865)(194.70148438,119.80731032)(194.93585935,119.80731032)
\curveto(195.33169263,119.80731032)(195.71450509,119.72658116)(196.08429671,119.56512285)
\curveto(196.45929666,119.40366453)(196.74054663,119.21095622)(196.9280466,118.98699792)
\curveto(197.12075491,118.76824794)(197.2535674,118.50783131)(197.32648405,118.20574801)
\curveto(197.39940071,117.90366472)(197.43585904,117.53126893)(197.43585904,117.08856065)
\lineto(197.43585904,115.71356082)
\curveto(197.43585904,115.62501916)(197.44887987,114.97658174)(197.47492153,113.76824856)
\curveto(197.4853382,113.27345695)(197.54002569,112.97918616)(197.63898401,112.88543617)
\curveto(197.73794234,112.79168618)(198.06085896,112.74481119)(198.60773389,112.74481119)
\lineto(198.67023389,112.68231119)
\lineto(198.67023389,112.34637374)
\lineto(198.60773389,112.28387374)
\curveto(197.94627564,112.32554041)(197.49835903,112.34637374)(197.26398406,112.34637374)
\curveto(197.12856741,112.34637374)(196.74835912,112.32554041)(196.1233592,112.28387374)
\lineto(196.02960921,112.36981123)
\curveto(196.09731754,113.04689448)(196.1311717,113.92970687)(196.1311717,115.01824841)
\lineto(196.1311717,116.04168578)
\curveto(196.1311717,116.65626904)(196.1155467,117.09897732)(196.08429671,117.36981062)
\curveto(196.05825504,117.64064392)(195.97231755,117.88543555)(195.82648424,118.10418553)
\curveto(195.68065092,118.32814383)(195.48533845,118.50001881)(195.24054681,118.61981046)
\curveto(194.99575517,118.74481045)(194.70408854,118.80731044)(194.36554692,118.80731044)
\curveto(194.09471362,118.80731044)(193.86294281,118.77866461)(193.6702345,118.72137295)
\curveto(193.47752619,118.66928962)(193.26138039,118.55731047)(193.02179708,118.38543549)
\curveto(192.78221378,118.21876885)(192.60513047,118.0599147)(192.49054715,117.90887305)
\curveto(192.37596383,117.76303973)(192.30565134,117.62501892)(192.27960968,117.4948106)
\curveto(192.25877634,117.36981062)(192.24835968,117.09897732)(192.24835968,116.6823107)
\lineto(192.24835968,115.32293587)
\curveto(192.24835968,115.20835255)(192.25877634,114.83856093)(192.27960968,114.21356101)
\curveto(192.30044301,113.59376942)(192.32127634,113.23179029)(192.34210967,113.12762364)
\curveto(192.36815133,113.02345699)(192.40460966,112.94793616)(192.45148465,112.90106117)
\curveto(192.49835965,112.85418617)(192.55565131,112.82293618)(192.62335963,112.80731118)
\curveto(192.69106796,112.79689451)(192.97231792,112.77606118)(193.46710953,112.74481119)
\lineto(193.53742202,112.68231119)
\lineto(193.53742202,112.35418623)
\lineto(193.47492203,112.28387374)
\curveto(192.82388044,112.32554041)(192.19627635,112.34637374)(191.59210976,112.34637374)
\curveto(190.9931515,112.34637374)(190.36815158,112.32554041)(189.71710999,112.28387374)
\lineto(189.6467975,112.35418623)
\lineto(189.6467975,112.68231119)
\lineto(189.71710999,112.74481119)
\curveto(190.22231826,112.77606118)(190.50617239,112.79949868)(190.56867239,112.81512368)
\curveto(190.63638071,112.83074868)(190.69367237,112.86199867)(190.74054736,112.90887367)
\curveto(190.79263069,112.96095699)(190.82648485,113.03647782)(190.84210985,113.13543614)
\curveto(190.86294318,113.23960279)(190.88377651,113.57554025)(190.90460984,114.14324851)
\curveto(190.93065151,114.71616511)(190.94367234,115.14845672)(190.94367234,115.44012335)
\lineto(190.94367234,116.97918566)
\curveto(190.94367234,117.18751897)(190.93325567,117.4713731)(190.91242234,117.83074806)
\curveto(190.89158901,118.19012302)(190.87335985,118.41147715)(190.85773485,118.49481048)
\curveto(190.84731819,118.5781438)(190.80825569,118.63803963)(190.74054736,118.67449796)
\curveto(190.67283904,118.71616462)(190.53742239,118.73699795)(190.33429742,118.73699795)
\lineto(189.6780475,118.74481045)
\lineto(189.607735,118.80731044)
\lineto(189.607735,119.1432479)
\lineto(189.670235,119.20574789)
\curveto(190.66502654,119.32553954)(191.5035681,119.52085202)(192.18585969,119.79168532)
\closepath
}
}
{
\newrgbcolor{curcolor}{0 0 0}
\pscustom[linestyle=none,fillstyle=solid,fillcolor=curcolor]
{
\newpath
\moveto(201.803046,119.26043538)
\lineto(202.6389834,119.26043538)
\curveto(203.07127501,119.26043538)(203.48012913,119.27606038)(203.86554575,119.30731038)
\lineto(203.93585824,119.22137289)
\lineto(203.79523326,118.53387297)
\lineto(203.72492076,118.45574798)
\curveto(203.56867078,118.46095632)(203.4671083,118.46356048)(203.4202333,118.46356048)
\lineto(202.40460843,118.47137298)
\lineto(201.803046,118.47137298)
\lineto(201.803046,115.32293587)
\curveto(201.803046,115.15626922)(201.81346267,114.77345677)(201.834296,114.17449851)
\curveto(201.86033766,113.58074858)(201.88377516,113.23179029)(201.90460849,113.12762364)
\curveto(201.92544182,113.02345699)(201.95929598,112.94793616)(202.00617098,112.90106117)
\curveto(202.0582543,112.85418617)(202.13898346,112.82033201)(202.24835845,112.79949868)
\curveto(202.36294177,112.77866535)(202.65721256,112.76043618)(203.13117084,112.74481119)
\lineto(203.19367083,112.68231119)
\lineto(203.19367083,112.35418623)
\lineto(203.13117084,112.28387374)
\curveto(202.89679587,112.29429041)(202.60512924,112.30731124)(202.25617095,112.32293624)
\curveto(201.95408765,112.33856124)(201.58690019,112.34637374)(201.15460858,112.34637374)
\curveto(200.59210865,112.34637374)(199.96450456,112.32554041)(199.27179631,112.28387374)
\lineto(199.20929632,112.35418623)
\lineto(199.20929632,112.68231119)
\lineto(199.27179631,112.74481119)
\curveto(199.77700458,112.77606118)(200.06346288,112.79949868)(200.13117121,112.81512368)
\curveto(200.19887953,112.83074868)(200.25617119,112.86199867)(200.30304619,112.90887367)
\curveto(200.34992118,112.96095699)(200.38117118,113.03647782)(200.39679617,113.13543614)
\curveto(200.41762951,113.23960279)(200.43846284,113.57554025)(200.45929617,114.14324851)
\curveto(200.48533783,114.71616511)(200.49835866,115.14845672)(200.49835866,115.44012335)
\lineto(200.49835866,118.45574798)
\lineto(199.49054629,118.40106049)
\lineto(199.42023379,118.46356048)
\lineto(199.42023379,118.65887296)
\lineto(199.47492129,118.73699795)
\lineto(200.49835866,119.26043538)
\lineto(200.49835866,119.72918533)
\curveto(200.49835866,120.25001859)(200.52179616,120.64324771)(200.56867115,120.90887268)
\curveto(200.61554615,121.17970598)(200.70148364,121.42710178)(200.82648362,121.65106009)
\curveto(200.95148361,121.88022673)(201.17544191,122.16928919)(201.49835854,122.51824748)
\curveto(201.82127517,122.86720577)(202.10252513,123.1510599)(202.34210843,123.36980988)
\curveto(202.58169174,123.58855985)(202.79002505,123.73439317)(202.96710836,123.80730982)
\curveto(203.14419167,123.88543481)(203.33950414,123.92449731)(203.55304579,123.92449731)
\curveto(203.76137909,123.92449731)(204.03742073,123.88543481)(204.38117068,123.80730982)
\lineto(204.38117068,123.34637238)
\curveto(205.3238789,123.45574737)(206.1363788,123.64585151)(206.81867038,123.91668481)
\lineto(206.96710786,123.82293482)
\curveto(206.9098162,123.22397656)(206.88117037,122.02345588)(206.88117037,120.22137276)
\lineto(206.88117037,115.32293587)
\curveto(206.88117037,115.20835255)(206.89158704,114.83856093)(206.91242037,114.21356101)
\curveto(206.9332537,113.59376942)(206.95408703,113.23179029)(206.97492036,113.12762364)
\curveto(207.00096203,113.02345699)(207.03742036,112.94793616)(207.08429535,112.90106117)
\curveto(207.13637868,112.85418617)(207.19367034,112.82293618)(207.25617033,112.80731118)
\curveto(207.32387865,112.79689451)(207.60512862,112.77606118)(208.09992022,112.74481119)
\lineto(208.17023272,112.68231119)
\lineto(208.17023272,112.35418623)
\lineto(208.10773272,112.28387374)
\curveto(207.45669114,112.32554041)(206.82908705,112.34637374)(206.22492046,112.34637374)
\curveto(205.66762886,112.34637374)(205.04262893,112.32554041)(204.34992069,112.28387374)
\lineto(204.28742069,112.35418623)
\lineto(204.28742069,112.68231119)
\lineto(204.34992069,112.74481119)
\curveto(204.85512896,112.77606118)(205.13898309,112.79949868)(205.20148308,112.81512368)
\curveto(205.26919141,112.83074868)(205.32648307,112.86199867)(205.37335806,112.90887367)
\curveto(205.42544139,112.96095699)(205.45929555,113.03647782)(205.47492055,113.13543614)
\curveto(205.49575388,113.23960279)(205.51658721,113.57554025)(205.53742054,114.14324851)
\curveto(205.5634622,114.71616511)(205.57648304,115.14845672)(205.57648304,115.44012335)
\lineto(205.57648304,119.48699786)
\lineto(205.53742054,121.36199762)
\curveto(205.52179554,122.08595587)(205.49835805,122.50001832)(205.46710805,122.60418497)
\curveto(205.44106639,122.70835162)(205.39419139,122.77866412)(205.32648307,122.81512245)
\curveto(205.25877474,122.85158077)(204.94367061,122.86980994)(204.38117068,122.86980994)
\lineto(204.38117068,122.52605998)
\lineto(204.2092957,122.45574749)
\curveto(203.82908742,122.71616412)(203.43325413,122.84637244)(203.02179585,122.84637244)
\curveto(202.71450422,122.84637244)(202.46710842,122.78387245)(202.27960844,122.65887246)
\curveto(202.0973168,122.53387248)(201.97231681,122.34897667)(201.90460849,122.10418503)
\curveto(201.83690016,121.86460173)(201.803046,121.45574761)(201.803046,120.87762268)
\closepath
}
}
{
\newrgbcolor{curcolor}{0 0 0}
\pscustom[linestyle=none,fillstyle=solid,fillcolor=curcolor]
{
\newpath
\moveto(211.25616984,119.79168532)
\lineto(211.40460732,119.69012283)
\curveto(211.34731566,119.00783125)(211.31866983,118.14064385)(211.31866983,117.08856065)
\lineto(211.31866983,115.26824838)
\curveto(211.31866983,114.60158179)(211.36814899,114.13804018)(211.46710731,113.87762355)
\curveto(211.57127396,113.61720691)(211.74835727,113.4166861)(211.99835724,113.27606112)
\curveto(212.24835721,113.14064447)(212.54523218,113.07293615)(212.88898213,113.07293615)
\curveto(213.26919042,113.07293615)(213.61554454,113.1458528)(213.92804451,113.29168612)
\curveto(214.24054447,113.44272777)(214.5009611,113.65366524)(214.70929441,113.92449854)
\curveto(214.92283605,114.19533184)(215.04523187,114.37762349)(215.07648186,114.47137347)
\curveto(215.10773186,114.56512346)(215.12856519,114.82814426)(215.13898186,115.26043588)
\lineto(215.16241935,116.13543577)
\lineto(215.16241935,116.97918566)
\curveto(215.16241935,117.18751897)(215.15200269,117.4713731)(215.13116936,117.83074806)
\curveto(215.11033603,118.19012302)(215.09210686,118.41147715)(215.07648186,118.49481048)
\curveto(215.0660652,118.5781438)(215.0270027,118.63803963)(214.95929438,118.67449796)
\curveto(214.89158605,118.71616462)(214.7561694,118.73699795)(214.55304443,118.73699795)
\lineto(213.89679451,118.74481045)
\lineto(213.82648202,118.80731044)
\lineto(213.82648202,119.1432479)
\lineto(213.88898201,119.20574789)
\curveto(214.88377355,119.32553954)(215.72231512,119.52085202)(216.4046067,119.79168532)
\lineto(216.55304418,119.69012283)
\curveto(216.49575252,119.00783125)(216.46710669,118.14064385)(216.46710669,117.08856065)
\lineto(216.46710669,115.71356082)
\curveto(216.46710669,115.63543583)(216.48012752,114.99741508)(216.50616919,113.79949856)
\curveto(216.51658585,113.35679028)(216.54262752,113.08595698)(216.58429418,112.98699866)
\curveto(216.63116917,112.89324867)(216.69366916,112.82554034)(216.77179416,112.78387368)
\curveto(216.84991915,112.74741535)(217.08168995,112.72918619)(217.46710657,112.72918619)
\lineto(217.68585654,112.72918619)
\lineto(217.75616903,112.6666862)
\lineto(217.75616903,112.35418623)
\lineto(217.69366904,112.28387374)
\curveto(216.95929413,112.32554041)(216.46710669,112.34637374)(216.21710672,112.34637374)
\curveto(215.89939843,112.34637374)(215.53221097,112.32814457)(215.11554436,112.29168624)
\lineto(215.04523187,112.35418623)
\curveto(215.07648186,112.88543617)(215.09991936,113.33595695)(215.11554436,113.70574857)
\curveto(214.7874194,113.45054027)(214.41241945,113.10939448)(213.9905445,112.68231119)
\curveto(213.82908618,112.52085288)(213.59731538,112.38543623)(213.29523208,112.27606124)
\curveto(212.99314879,112.16668626)(212.64939883,112.11199876)(212.26398221,112.11199876)
\curveto(211.68064895,112.11199876)(211.22491984,112.20314459)(210.89679488,112.38543623)
\curveto(210.57387825,112.57293621)(210.34471161,112.82814451)(210.20929496,113.15106114)
\curveto(210.07387831,113.4791861)(210.00616999,114.04168603)(210.00616999,114.83856093)
\lineto(210.01398249,115.45574835)
\lineto(210.01398249,116.97918566)
\curveto(210.01398249,117.18751897)(210.00356582,117.4713731)(209.98273249,117.83074806)
\curveto(209.96189916,118.19012302)(209.94367,118.41147715)(209.928045,118.49481048)
\curveto(209.91762833,118.5781438)(209.87856584,118.63803963)(209.81085751,118.67449796)
\curveto(209.74314919,118.71616462)(209.60773254,118.73699795)(209.40460756,118.73699795)
\lineto(208.74835764,118.74481045)
\lineto(208.67804515,118.80731044)
\lineto(208.67804515,119.1432479)
\lineto(208.74054515,119.20574789)
\curveto(209.73533669,119.32553954)(210.57387825,119.52085202)(211.25616984,119.79168532)
\closepath
\moveto(213.12335711,120.28387276)
\closepath
\moveto(213.2014821,111.87762379)
\closepath
}
}
{
\newrgbcolor{curcolor}{0 0 0}
\pscustom[linestyle=none,fillstyle=solid,fillcolor=curcolor]
{
\newpath
\moveto(218.99054388,114.64324845)
\lineto(219.32648134,114.64324845)
\lineto(219.39679383,114.57293596)
\curveto(219.4072105,114.13543602)(219.43325216,113.76043606)(219.47491882,113.4479361)
\curveto(219.63637714,113.20314446)(219.91502293,113.00262365)(220.31085622,112.84637367)
\curveto(220.7066895,112.69533203)(221.09731446,112.6198112)(221.48273107,112.6198112)
\curveto(222.03481434,112.6198112)(222.47491845,112.76824868)(222.80304341,113.06512365)
\curveto(223.1363767,113.36199861)(223.30304335,113.71616523)(223.30304335,114.12762352)
\curveto(223.30304335,114.35158182)(223.24314752,114.54429013)(223.12335587,114.70574844)
\curveto(223.00356422,114.87241509)(222.82127258,115.00783174)(222.57648094,115.11199839)
\curveto(222.33689764,115.22137338)(221.90460602,115.34897753)(221.2796061,115.49481085)
\curveto(220.74314783,115.61981083)(220.36814788,115.71616499)(220.15460624,115.78387331)
\curveto(219.9410646,115.85678997)(219.73533546,115.97658162)(219.53741881,116.14324827)
\curveto(219.33950217,116.30991491)(219.18846052,116.51303989)(219.08429387,116.75262319)
\curveto(218.98012722,116.99741483)(218.92804389,117.26564396)(218.92804389,117.55731059)
\curveto(218.92804389,118.25522717)(219.20408552,118.81251877)(219.75616879,119.22918539)
\curveto(220.31346039,119.65106034)(221.00616863,119.86199781)(221.83429353,119.86199781)
\curveto(222.18325182,119.86199781)(222.56346011,119.81251865)(222.97491839,119.71356033)
\curveto(223.38637667,119.61981034)(223.69887663,119.52866452)(223.91241827,119.44012286)
\lineto(223.98273077,119.33074787)
\curveto(223.9410641,119.12241457)(223.91502244,118.57553963)(223.90460578,117.69012308)
\lineto(223.83429328,117.61981059)
\lineto(223.52179332,117.61981059)
\lineto(223.45148083,117.69012308)
\curveto(223.4306475,118.00783137)(223.40200167,118.22918551)(223.36554334,118.3541855)
\curveto(223.32908501,118.48439381)(223.23533502,118.62241463)(223.08429338,118.76824794)
\curveto(222.93325173,118.91928959)(222.71971009,119.04428958)(222.44366846,119.1432479)
\curveto(222.16762682,119.24741455)(221.87596019,119.29949788)(221.56866856,119.29949788)
\curveto(221.25096027,119.29949788)(220.98273114,119.25262288)(220.76398116,119.1588729)
\curveto(220.55043952,119.06512291)(220.37596038,118.92189376)(220.24054373,118.72918545)
\curveto(220.11033541,118.54168547)(220.04523125,118.31251883)(220.04523125,118.04168553)
\curveto(220.04523125,117.84376889)(220.08429375,117.66668558)(220.16241874,117.5104356)
\curveto(220.24575206,117.35418562)(220.37075204,117.22658147)(220.53741869,117.12762315)
\curveto(220.70408534,117.03387316)(220.87856448,116.96616483)(221.06085613,116.92449817)
\lineto(221.91241852,116.7057482)
\curveto(222.6415851,116.52866489)(223.16241837,116.37501907)(223.47491833,116.24481076)
\curveto(223.79262662,116.11460244)(224.03741826,115.9166858)(224.20929324,115.65106083)
\curveto(224.38116822,115.39064419)(224.46710571,115.0729359)(224.46710571,114.69793595)
\curveto(224.46710571,113.97918603)(224.17023074,113.36199861)(223.57648082,112.84637367)
\curveto(222.98273089,112.33074874)(222.21971015,112.07293627)(221.2874186,112.07293627)
\curveto(220.95929364,112.07293627)(220.55043952,112.1093946)(220.06085625,112.18231126)
\curveto(219.57648131,112.25522791)(219.16241886,112.33595707)(218.8186689,112.42449873)
\lineto(218.77960641,112.52606121)
\lineto(218.8577314,113.04949865)
\curveto(218.88377306,113.21095696)(218.89939806,113.37241528)(218.90460639,113.53387359)
\curveto(218.90981472,113.70054024)(218.91502306,114.04689436)(218.92023139,114.57293596)
\closepath
}
}
{
\newrgbcolor{curcolor}{0 0 0}
\pscustom[linestyle=none,fillstyle=solid,fillcolor=curcolor]
{
\newpath
\moveto(225.77179305,114.64324845)
\lineto(226.1077305,114.64324845)
\lineto(226.178043,114.57293596)
\curveto(226.18845966,114.13543602)(226.21450132,113.76043606)(226.25616799,113.4479361)
\curveto(226.4176263,113.20314446)(226.6962721,113.00262365)(227.09210538,112.84637367)
\curveto(227.48793867,112.69533203)(227.87856362,112.6198112)(228.26398024,112.6198112)
\curveto(228.8160635,112.6198112)(229.25616762,112.76824868)(229.58429258,113.06512365)
\curveto(229.91762587,113.36199861)(230.08429251,113.71616523)(230.08429251,114.12762352)
\curveto(230.08429251,114.35158182)(230.02439669,114.54429013)(229.90460504,114.70574844)
\curveto(229.78481338,114.87241509)(229.60252174,115.00783174)(229.3577301,115.11199839)
\curveto(229.1181468,115.22137338)(228.68585519,115.34897753)(228.06085526,115.49481085)
\curveto(227.524397,115.61981083)(227.14939704,115.71616499)(226.9358554,115.78387331)
\curveto(226.72231376,115.85678997)(226.51658462,115.97658162)(226.31866798,116.14324827)
\curveto(226.12075134,116.30991491)(225.96970969,116.51303989)(225.86554303,116.75262319)
\curveto(225.76137638,116.99741483)(225.70929305,117.26564396)(225.70929305,117.55731059)
\curveto(225.70929305,118.25522717)(225.98533469,118.81251877)(226.53741795,119.22918539)
\curveto(227.09470955,119.65106034)(227.7874178,119.86199781)(228.61554269,119.86199781)
\curveto(228.96450099,119.86199781)(229.34470927,119.81251865)(229.75616755,119.71356033)
\curveto(230.16762584,119.61981034)(230.4801258,119.52866452)(230.69366744,119.44012286)
\lineto(230.76397993,119.33074787)
\curveto(230.72231327,119.12241457)(230.6962716,118.57553963)(230.68585494,117.69012308)
\lineto(230.61554245,117.61981059)
\lineto(230.30304249,117.61981059)
\lineto(230.23273,117.69012308)
\curveto(230.21189666,118.00783137)(230.18325083,118.22918551)(230.14679251,118.3541855)
\curveto(230.11033418,118.48439381)(230.01658419,118.62241463)(229.86554254,118.76824794)
\curveto(229.71450089,118.91928959)(229.50095925,119.04428958)(229.22491762,119.1432479)
\curveto(228.94887599,119.24741455)(228.65720936,119.29949788)(228.34991773,119.29949788)
\curveto(228.03220943,119.29949788)(227.7639803,119.25262288)(227.54523033,119.1588729)
\curveto(227.33168869,119.06512291)(227.15720954,118.92189376)(227.02179289,118.72918545)
\curveto(226.89158457,118.54168547)(226.82648042,118.31251883)(226.82648042,118.04168553)
\curveto(226.82648042,117.84376889)(226.86554291,117.66668558)(226.9436679,117.5104356)
\curveto(227.02700122,117.35418562)(227.15200121,117.22658147)(227.31866785,117.12762315)
\curveto(227.4853345,117.03387316)(227.65981365,116.96616483)(227.84210529,116.92449817)
\lineto(228.69366769,116.7057482)
\curveto(229.42283426,116.52866489)(229.94366753,116.37501907)(230.25616749,116.24481076)
\curveto(230.57387579,116.11460244)(230.81866742,115.9166858)(230.9905424,115.65106083)
\curveto(231.16241738,115.39064419)(231.24835487,115.0729359)(231.24835487,114.69793595)
\curveto(231.24835487,113.97918603)(230.95147991,113.36199861)(230.35772998,112.84637367)
\curveto(229.76398005,112.33074874)(229.00095931,112.07293627)(228.06866776,112.07293627)
\curveto(227.7405428,112.07293627)(227.33168869,112.1093946)(226.84210541,112.18231126)
\curveto(226.35773047,112.25522791)(225.94366802,112.33595707)(225.59991807,112.42449873)
\lineto(225.56085557,112.52606121)
\lineto(225.63898056,113.04949865)
\curveto(225.66502223,113.21095696)(225.68064722,113.37241528)(225.68585556,113.53387359)
\curveto(225.69106389,113.70054024)(225.69627222,114.04689436)(225.70148055,114.57293596)
\closepath
}
}
{
\newrgbcolor{curcolor}{0 0 0}
\pscustom[linestyle=none,fillstyle=solid,fillcolor=curcolor]
{
\newpath
\moveto(234.65460445,119.88543531)
\lineto(234.80304193,119.79168532)
\curveto(234.75095861,119.19793539)(234.72231278,118.67189379)(234.71710444,118.21356051)
\curveto(235.42543769,118.85939377)(235.91762513,119.28647705)(236.19366676,119.49481035)
\curveto(236.47491673,119.70314366)(236.9566875,119.80731032)(237.63897908,119.80731032)
\curveto(238.06606236,119.80731032)(238.45408315,119.74741449)(238.80304144,119.62762284)
\curveto(239.15199973,119.51303952)(239.47491636,119.32033121)(239.77179132,119.04949791)
\curveto(240.06866628,118.78387294)(240.30304125,118.44793548)(240.47491623,118.04168553)
\curveto(240.64679121,117.64064392)(240.7327287,117.17449814)(240.7327287,116.64324821)
\curveto(240.7327287,116.26824825)(240.67283287,115.88803997)(240.55304122,115.50262335)
\curveto(240.43324957,115.11720673)(240.27439542,114.76043594)(240.07647878,114.43231098)
\curveto(239.88377047,114.10418602)(239.72752049,113.88543605)(239.60772884,113.77606106)
\curveto(239.49314552,113.67189441)(239.25356222,113.51304026)(238.88897893,113.29949862)
\curveto(238.56085397,113.10158198)(238.23533318,112.88804034)(237.91241655,112.6588737)
\curveto(237.68845824,112.49741538)(237.53220826,112.39324873)(237.44366661,112.34637374)
\curveto(237.35512495,112.29949874)(237.20668747,112.25262375)(236.99835416,112.20574875)
\curveto(236.79002085,112.15887376)(236.56866671,112.13543626)(236.33429174,112.13543626)
\curveto(235.79262514,112.13543626)(235.25356271,112.27345708)(234.71710444,112.54949871)
\lineto(234.71710444,110.88543642)
\curveto(234.71710444,110.7708531)(234.72752111,110.40366564)(234.74835444,109.78387405)
\curveto(234.76918777,109.15887413)(234.7900211,108.79429084)(234.81085443,108.69012419)
\curveto(234.83689609,108.58595753)(234.87335442,108.51043671)(234.92022942,108.46356171)
\curveto(234.96710441,108.41668672)(235.02439607,108.38804089)(235.0921044,108.37762423)
\curveto(235.15981272,108.36199923)(235.44106269,108.33856173)(235.93585429,108.30731173)
\lineto(236.00616678,108.24481174)
\lineto(236.00616678,107.91668678)
\lineto(235.94366679,107.84637429)
\curveto(235.2926252,107.88804095)(234.66762528,107.90887428)(234.06866702,107.90887428)
\curveto(233.46970876,107.90887428)(232.84470884,107.88804095)(232.19366725,107.84637429)
\lineto(232.12335476,107.91668678)
\lineto(232.12335476,108.24481174)
\lineto(232.19366725,108.30731173)
\curveto(232.69366719,108.33856173)(232.97752132,108.36199923)(233.04522965,108.37762423)
\curveto(233.11293797,108.39324922)(233.17022963,108.42710339)(233.21710463,108.47918671)
\curveto(233.26918795,108.52606171)(233.30304212,108.6041867)(233.31866711,108.71356168)
\curveto(233.33950045,108.82293667)(233.36033378,109.17189496)(233.38116711,109.76043655)
\curveto(233.40720877,110.34897815)(233.4202296,110.79168643)(233.4202296,111.08856139)
\lineto(233.4202296,117.07293565)
\curveto(233.4202296,117.34376895)(233.40720877,117.65626891)(233.38116711,118.01043554)
\curveto(233.36033378,118.36981049)(233.34210461,118.5781438)(233.32647961,118.63543546)
\curveto(233.31085462,118.69272712)(233.27179212,118.73960211)(233.20929213,118.77606044)
\curveto(233.15200047,118.81251877)(233.01918799,118.83074794)(232.81085468,118.83074794)
\lineto(232.15460476,118.83856044)
\lineto(232.08429227,118.90106043)
\lineto(232.08429227,119.23699789)
\lineto(232.14679226,119.29949788)
\curveto(233.1415838,119.41928953)(233.9775212,119.61460201)(234.65460445,119.88543531)
\closepath
\moveto(234.71710444,113.89324855)
\curveto(234.95668775,113.64845691)(235.25356271,113.43491527)(235.60772933,113.25262362)
\curveto(235.96189596,113.07554031)(236.35252091,112.98699866)(236.77960419,112.98699866)
\curveto(237.28481246,112.98699866)(237.7301249,113.11460281)(238.11554152,113.36981111)
\curveto(238.50616648,113.62501941)(238.80824977,113.9974152)(239.02179141,114.48699847)
\curveto(239.23533305,114.98179008)(239.34210387,115.52866501)(239.34210387,116.12762327)
\curveto(239.34210387,116.61199821)(239.24574972,117.06251899)(239.05304141,117.4791856)
\curveto(238.86554143,117.89585222)(238.57127063,118.22397718)(238.17022902,118.46356048)
\curveto(237.77439573,118.70314379)(237.35252078,118.82293544)(236.90460417,118.82293544)
\curveto(236.62856254,118.82293544)(236.36033341,118.77345628)(236.09991677,118.67449796)
\curveto(235.83950014,118.58074797)(235.5947085,118.44012298)(235.36554186,118.25262301)
\curveto(235.14158356,118.06512303)(234.98012524,117.88543555)(234.88116692,117.71356057)
\curveto(234.7822086,117.54689393)(234.73012527,117.40626895)(234.72491694,117.29168563)
\curveto(234.71970861,117.18231064)(234.71710444,117.05991482)(234.71710444,116.92449817)
\closepath
}
}
{
\newrgbcolor{curcolor}{0 0 0}
\pscustom[linestyle=none,fillstyle=solid,fillcolor=curcolor]
{
\newpath
\moveto(243.03741592,117.64324808)
\lineto(242.73272845,117.72137307)
\lineto(242.67022846,117.79949806)
\lineto(242.67022846,118.76824794)
\curveto(243.59731168,119.45574786)(244.49835324,119.79949782)(245.37335313,119.79949782)
\curveto(245.97231139,119.79949782)(246.46970716,119.68751866)(246.86554044,119.46356036)
\curveto(247.26137373,119.23960205)(247.54783203,118.95835209)(247.72491534,118.61981046)
\curveto(247.90199865,118.28647717)(247.99054031,117.89585222)(247.99054031,117.44793561)
\lineto(247.95147781,115.8776233)
\lineto(247.95147781,113.54949859)
\curveto(247.95147781,113.23179029)(247.97231114,113.03908198)(248.0139778,112.97137366)
\curveto(248.0608528,112.90366533)(248.11293612,112.85679034)(248.17022778,112.83074868)
\curveto(248.22751944,112.80991535)(248.33429026,112.79168618)(248.49054024,112.77606118)
\lineto(248.93585269,112.73699869)
\lineto(248.99835268,112.6666862)
\lineto(248.99835268,112.35418623)
\lineto(248.93585269,112.29168624)
\curveto(248.5556444,112.32293624)(248.20147778,112.33856124)(247.87335282,112.33856124)
\curveto(247.56085286,112.33856124)(247.18585291,112.32293624)(246.74835296,112.29168624)
\lineto(246.63116547,112.40106123)
\lineto(246.66241547,113.65887357)
\lineto(244.95929068,112.33074874)
\curveto(244.67283238,112.21095709)(244.36033242,112.15106126)(244.0217908,112.15106126)
\curveto(243.60512418,112.15106126)(243.24574922,112.22658208)(242.94366593,112.37762373)
\curveto(242.64679096,112.52866538)(242.41762433,112.73960285)(242.25616601,113.01043615)
\curveto(242.09991603,113.28126945)(242.02179104,113.60939441)(242.02179104,113.99481103)
\curveto(242.02179104,114.76043594)(242.26137435,115.35679003)(242.74054095,115.78387331)
\curveto(243.21970756,116.21616493)(244.52699907,116.58335238)(246.66241547,116.88543568)
\curveto(246.66241547,117.65626891)(246.49054049,118.19793551)(246.14679053,118.51043548)
\curveto(245.80304058,118.82814377)(245.34210313,118.98699792)(244.7639782,118.98699792)
\curveto(244.46189491,118.98699792)(244.18585327,118.94272709)(243.93585331,118.85418543)
\curveto(243.69106167,118.76564378)(243.55043669,118.69272712)(243.51397836,118.63543546)
\curveto(243.47752003,118.58335213)(243.33689505,118.26564384)(243.09210341,117.68231058)
\closepath
\moveto(246.66241547,116.40887323)
\curveto(245.20408232,116.1640816)(244.29522826,115.9010608)(243.93585331,115.61981083)
\curveto(243.57647835,115.33856087)(243.39679087,114.90366509)(243.39679087,114.31512349)
\curveto(243.39679087,113.50262359)(243.80043666,113.09637364)(244.60772822,113.09637364)
\curveto(245.30043647,113.09637364)(245.98533222,113.49741526)(246.66241547,114.2994985)
\closepath
\moveto(245.33429063,120.28387276)
\closepath
\moveto(245.17022815,111.87762379)
\closepath
}
}
{
\newrgbcolor{curcolor}{0 0 0}
\pscustom[linestyle=none,fillstyle=solid,fillcolor=curcolor]
{
\newpath
\moveto(252.38116476,119.79168532)
\lineto(252.52960225,119.69012283)
\curveto(252.49835225,119.38803953)(252.47491475,118.85158127)(252.45928975,118.08074803)
\lineto(253.04522718,118.82293544)
\curveto(253.23793549,119.06772707)(253.4072063,119.25522705)(253.55303962,119.38543537)
\curveto(253.70408127,119.52085202)(253.87856041,119.62501867)(254.07647706,119.69793533)
\curveto(254.2743937,119.77085199)(254.47751867,119.80731032)(254.68585198,119.80731032)
\curveto(254.91501862,119.80731032)(255.13116443,119.76043532)(255.3342894,119.66668533)
\lineto(255.38897689,119.55731035)
\curveto(255.30564357,118.8646021)(255.25876858,118.25522717)(255.24835191,117.72918557)
\lineto(254.89678945,117.72918557)
\curveto(254.68845615,118.20835218)(254.35251869,118.44793548)(253.88897708,118.44793548)
\curveto(253.56606045,118.44793548)(253.28481049,118.34376883)(253.04522718,118.13543552)
\curveto(252.80564388,117.93231055)(252.64418557,117.67449808)(252.56085224,117.36199812)
\curveto(252.48272725,117.05470649)(252.44366476,116.66408154)(252.44366476,116.19012326)
\lineto(252.44366476,115.32293587)
\curveto(252.44366476,115.16668589)(252.45408142,114.78387344)(252.47491475,114.17449851)
\curveto(252.49574808,113.56512359)(252.51658141,113.21356113)(252.53741475,113.11981114)
\curveto(252.56345641,113.02606115)(252.59991474,112.95574866)(252.64678973,112.90887367)
\curveto(252.69887306,112.867207)(252.76397722,112.83856118)(252.84210221,112.82293618)
\curveto(252.92543553,112.80731118)(253.29783132,112.78126952)(253.95928957,112.74481119)
\lineto(254.02960206,112.68231119)
\lineto(254.02960206,112.35418623)
\lineto(253.95928957,112.28387374)
\curveto(253.26658132,112.32554041)(252.54262308,112.34637374)(251.78741484,112.34637374)
\curveto(251.18845658,112.34637374)(250.56345666,112.32554041)(249.91241507,112.28387374)
\lineto(249.84210258,112.35418623)
\lineto(249.84210258,112.68231119)
\lineto(249.91241507,112.74481119)
\curveto(250.41762334,112.77606118)(250.70147747,112.79949868)(250.76397746,112.81512368)
\curveto(250.83168579,112.83074868)(250.88897745,112.86199867)(250.93585244,112.90887367)
\curveto(250.98793577,112.96095699)(251.02178993,113.03647782)(251.03741493,113.13543614)
\curveto(251.05824826,113.23960279)(251.07908159,113.57554025)(251.09991492,114.14324851)
\curveto(251.12595659,114.71616511)(251.13897742,115.14845672)(251.13897742,115.44012335)
\lineto(251.13897742,116.97918566)
\curveto(251.13897742,117.18751897)(251.12856075,117.4713731)(251.10772742,117.83074806)
\curveto(251.08689409,118.19012302)(251.06866493,118.41147715)(251.05303993,118.49481048)
\curveto(251.04262326,118.5781438)(251.00356077,118.63803963)(250.93585244,118.67449796)
\curveto(250.86814412,118.71616462)(250.73272747,118.73699795)(250.52960249,118.73699795)
\lineto(249.87335257,118.74481045)
\lineto(249.80304008,118.80731044)
\lineto(249.80304008,119.1432479)
\lineto(249.86554007,119.20574789)
\curveto(250.86033162,119.32553954)(251.69887318,119.52085202)(252.38116476,119.79168532)
\closepath
}
}
{
\newrgbcolor{curcolor}{0 0 0}
\pscustom[linestyle=none,fillstyle=solid,fillcolor=curcolor]
{
\newpath
\moveto(257.35772665,117.64324808)
\lineto(257.05303919,117.72137307)
\lineto(256.9905392,117.79949806)
\lineto(256.9905392,118.76824794)
\curveto(257.91762242,119.45574786)(258.81866397,119.79949782)(259.69366386,119.79949782)
\curveto(260.29262212,119.79949782)(260.79001789,119.68751866)(261.18585118,119.46356036)
\curveto(261.58168446,119.23960205)(261.86814276,118.95835209)(262.04522607,118.61981046)
\curveto(262.22230938,118.28647717)(262.31085104,117.89585222)(262.31085104,117.44793561)
\lineto(262.27178854,115.8776233)
\lineto(262.27178854,113.54949859)
\curveto(262.27178854,113.23179029)(262.29262188,113.03908198)(262.33428854,112.97137366)
\curveto(262.38116353,112.90366533)(262.43324686,112.85679034)(262.49053852,112.83074868)
\curveto(262.54783018,112.80991535)(262.654601,112.79168618)(262.81085098,112.77606118)
\lineto(263.25616342,112.73699869)
\lineto(263.31866342,112.6666862)
\lineto(263.31866342,112.35418623)
\lineto(263.25616342,112.29168624)
\curveto(262.87595514,112.32293624)(262.52178851,112.33856124)(262.19366355,112.33856124)
\curveto(261.88116359,112.33856124)(261.50616364,112.32293624)(261.06866369,112.29168624)
\lineto(260.95147621,112.40106123)
\lineto(260.9827262,113.65887357)
\lineto(259.27960141,112.33074874)
\curveto(258.99314312,112.21095709)(258.68064315,112.15106126)(258.34210153,112.15106126)
\curveto(257.92543491,112.15106126)(257.56605996,112.22658208)(257.26397666,112.37762373)
\curveto(256.9671017,112.52866538)(256.73793506,112.73960285)(256.57647675,113.01043615)
\curveto(256.42022677,113.28126945)(256.34210178,113.60939441)(256.34210178,113.99481103)
\curveto(256.34210178,114.76043594)(256.58168508,115.35679003)(257.06085169,115.78387331)
\curveto(257.5400183,116.21616493)(258.8473098,116.58335238)(260.9827262,116.88543568)
\curveto(260.9827262,117.65626891)(260.81085123,118.19793551)(260.46710127,118.51043548)
\curveto(260.12335131,118.82814377)(259.66241387,118.98699792)(259.08428894,118.98699792)
\curveto(258.78220564,118.98699792)(258.50616401,118.94272709)(258.25616404,118.85418543)
\curveto(258.0113724,118.76564378)(257.87074742,118.69272712)(257.83428909,118.63543546)
\curveto(257.79783076,118.58335213)(257.65720578,118.26564384)(257.41241414,117.68231058)
\closepath
\moveto(260.9827262,116.40887323)
\curveto(259.52439305,116.1640816)(258.615539,115.9010608)(258.25616404,115.61981083)
\curveto(257.89678908,115.33856087)(257.71710161,114.90366509)(257.71710161,114.31512349)
\curveto(257.71710161,113.50262359)(258.12074739,113.09637364)(258.92803896,113.09637364)
\curveto(259.62074721,113.09637364)(260.30564295,113.49741526)(260.9827262,114.2994985)
\closepath
\moveto(259.65460137,120.28387276)
\closepath
\moveto(259.49053889,111.87762379)
\closepath
}
}
{
\newrgbcolor{curcolor}{0 0 0}
\pscustom[linestyle=none,fillstyle=solid,fillcolor=curcolor]
{
\newpath
\moveto(266.57647551,119.79168532)
\lineto(266.724913,119.69012283)
\curveto(266.693663,119.34637287)(266.67282967,118.93491459)(266.662413,118.45574798)
\lineto(267.40460041,119.1354354)
\curveto(267.61293372,119.32814371)(267.75095454,119.45053953)(267.81866286,119.50262285)
\curveto(267.88637119,119.55991451)(268.04522533,119.62501867)(268.2952253,119.69793533)
\curveto(268.54522527,119.77085199)(268.80824607,119.80731032)(269.0842877,119.80731032)
\curveto(269.60512097,119.80731032)(270.06085008,119.677102)(270.45147504,119.41668536)
\curveto(270.84209999,119.16147706)(271.13116245,118.80731044)(271.31866243,118.3541855)
\curveto(272.02178734,119.02085208)(272.43064146,119.3906437)(272.54522478,119.46356036)
\curveto(272.66501643,119.53647702)(272.85251641,119.60939367)(273.10772471,119.68231033)
\curveto(273.36293301,119.76043532)(273.61814131,119.79949782)(273.87334961,119.79949782)
\curveto(274.2848079,119.79949782)(274.66241202,119.71095616)(275.00616197,119.53387285)
\curveto(275.34991193,119.35678954)(275.6233494,119.12501873)(275.82647437,118.83856044)
\curveto(276.02959935,118.55210214)(276.14678683,118.24220634)(276.17803683,117.90887305)
\curveto(276.21449516,117.57553976)(276.23272432,117.11720648)(276.23272432,116.53387322)
\lineto(276.23272432,115.71356082)
\curveto(276.23272432,115.62501916)(276.24834932,114.97658174)(276.27959932,113.76824856)
\curveto(276.29001598,113.27345695)(276.34470348,112.97918616)(276.4436618,112.88543617)
\curveto(276.54262012,112.79168618)(276.86293258,112.74481119)(277.40459918,112.74481119)
\lineto(277.46709917,112.68231119)
\lineto(277.46709917,112.34637374)
\lineto(277.40459918,112.28387374)
\curveto(276.74314093,112.32554041)(276.29522432,112.34637374)(276.06084934,112.34637374)
\curveto(275.93064103,112.34637374)(275.55303691,112.32554041)(274.92803698,112.28387374)
\lineto(274.8264745,112.36981123)
\curveto(274.89418282,113.04689448)(274.92803698,113.92970687)(274.92803698,115.01824841)
\lineto(274.92803698,115.95574829)
\curveto(274.92803698,116.78387319)(274.88897449,117.36199812)(274.8108495,117.69012308)
\curveto(274.73272451,118.02345637)(274.55043286,118.29428967)(274.26397457,118.50262298)
\curveto(273.97751627,118.71616462)(273.63637048,118.82293544)(273.24053719,118.82293544)
\curveto(272.95928723,118.82293544)(272.69626643,118.76564378)(272.45147479,118.65106046)
\curveto(272.20668315,118.54168547)(271.98532901,118.37501883)(271.78741237,118.15106052)
\curveto(271.59470406,117.93231055)(271.48793324,117.72658141)(271.46709991,117.5338731)
\curveto(271.44626658,117.34637312)(271.43584991,116.99741483)(271.43584991,116.48699823)
\lineto(271.43584991,115.51043585)
\curveto(271.43584991,115.28126921)(271.44626658,114.85158176)(271.46709991,114.2213735)
\curveto(271.48793324,113.59116525)(271.50876657,113.22397779)(271.5295999,113.11981114)
\curveto(271.55564157,113.01564449)(271.5920999,112.94012366)(271.63897489,112.89324867)
\curveto(271.69105822,112.85158201)(271.74834988,112.82293618)(271.81084987,112.80731118)
\curveto(271.87855819,112.79689451)(272.15980816,112.77606118)(272.65459976,112.74481119)
\lineto(272.72491226,112.68231119)
\lineto(272.72491226,112.35418623)
\lineto(272.66241226,112.28387374)
\curveto(272.01137068,112.32554041)(271.38376659,112.34637374)(270.7796,112.34637374)
\curveto(270.2223084,112.34637374)(269.59730847,112.32554041)(268.90460023,112.28387374)
\lineto(268.83428774,112.35418623)
\lineto(268.83428774,112.68231119)
\lineto(268.90460023,112.74481119)
\curveto(269.4098085,112.77606118)(269.69366263,112.79949868)(269.75616262,112.81512368)
\curveto(269.82387095,112.83074868)(269.88116261,112.86199867)(269.9280376,112.90887367)
\curveto(269.98012093,112.96095699)(270.01397509,113.03647782)(270.02960009,113.13543614)
\curveto(270.05043342,113.23960279)(270.07126675,113.57554025)(270.09210008,114.14324851)
\curveto(270.11814174,114.71616511)(270.13116258,115.14845672)(270.13116258,115.44012335)
\lineto(270.13116258,116.33074824)
\curveto(270.13116258,116.91928984)(270.08949591,117.37241478)(270.00616259,117.69012308)
\curveto(269.9280376,118.00783137)(269.75616262,118.2760605)(269.49053765,118.49481048)
\curveto(269.22491269,118.71356045)(268.8915794,118.82293544)(268.49053778,118.82293544)
\curveto(268.17803782,118.82293544)(267.89157952,118.76043545)(267.63116288,118.63543546)
\curveto(267.37595458,118.51564381)(267.16501711,118.36460216)(266.99835046,118.18231052)
\curveto(266.83689215,118.0052272)(266.73532966,117.83856056)(266.693663,117.68231058)
\curveto(266.65720467,117.5260606)(266.63897551,117.18751897)(266.63897551,116.6666857)
\lineto(266.63897551,115.51043585)
\curveto(266.63897551,115.28126921)(266.64939217,114.85158176)(266.6702255,114.2213735)
\curveto(266.69105883,113.59116525)(266.71189216,113.22397779)(266.73272549,113.11981114)
\curveto(266.75876716,113.01564449)(266.79522549,112.94012366)(266.84210048,112.89324867)
\curveto(266.89418381,112.85158201)(266.95147547,112.82293618)(267.01397546,112.80731118)
\curveto(267.08168379,112.79689451)(267.36293375,112.77606118)(267.85772536,112.74481119)
\lineto(267.92803785,112.68231119)
\lineto(267.92803785,112.35418623)
\lineto(267.86553786,112.28387374)
\curveto(267.21449627,112.32554041)(266.58689218,112.34637374)(265.98272559,112.34637374)
\curveto(265.38376733,112.34637374)(264.7587674,112.32554041)(264.10772582,112.28387374)
\lineto(264.03741333,112.35418623)
\lineto(264.03741333,112.68231119)
\lineto(264.10772582,112.74481119)
\curveto(264.61293409,112.77606118)(264.89678822,112.79949868)(264.95928821,112.81512368)
\curveto(265.02699654,112.83074868)(265.0842882,112.86199867)(265.13116319,112.90887367)
\curveto(265.18324652,112.96095699)(265.21710068,113.03647782)(265.23272568,113.13543614)
\curveto(265.25355901,113.23960279)(265.27439234,113.57554025)(265.29522567,114.14324851)
\curveto(265.32126734,114.71616511)(265.33428817,115.14845672)(265.33428817,115.44012335)
\lineto(265.33428817,116.97918566)
\curveto(265.33428817,117.18751897)(265.3238715,117.4713731)(265.30303817,117.83074806)
\curveto(265.28220484,118.19012302)(265.26397568,118.41147715)(265.24835068,118.49481048)
\curveto(265.23793401,118.5781438)(265.19887152,118.63803963)(265.13116319,118.67449796)
\curveto(265.06345487,118.71616462)(264.92803822,118.73699795)(264.72491324,118.73699795)
\lineto(264.06866332,118.74481045)
\lineto(263.99835083,118.80731044)
\lineto(263.99835083,119.1432479)
\lineto(264.06085082,119.20574789)
\curveto(265.05564237,119.32553954)(265.89418393,119.52085202)(266.57647551,119.79168532)
\closepath
}
}
{
\newrgbcolor{curcolor}{0 0 0}
\pscustom[linestyle=none,fillstyle=solid,fillcolor=curcolor]
{
\newpath
\moveto(284.73272328,113.49481109)
\lineto(284.48272331,112.93231116)
\curveto(283.94105671,112.58335287)(283.45668177,112.3567904)(283.02959849,112.25262375)
\curveto(282.60772354,112.14845709)(282.23011942,112.09637377)(281.89678613,112.09637377)
\curveto(281.2717862,112.09637377)(280.67803628,112.21876958)(280.11553634,112.46356122)
\curveto(279.55824475,112.70835286)(279.10251564,113.12241531)(278.74834901,113.70574857)
\curveto(278.39939072,114.28908183)(278.22491158,114.99220674)(278.22491158,115.81512331)
\curveto(278.22491158,116.36199824)(278.2926199,116.85418568)(278.42803655,117.29168563)
\curveto(278.5634532,117.7343939)(278.70407819,118.06251886)(278.8499115,118.2760605)
\curveto(279.00095315,118.48960215)(279.25355728,118.72658128)(279.60772391,118.98699792)
\curveto(279.96189053,119.24741455)(280.33689048,119.45574786)(280.73272377,119.61199784)
\curveto(281.12855705,119.76824782)(281.55564033,119.84637281)(282.01397361,119.84637281)
\curveto(282.63897353,119.84637281)(283.18845263,119.69793533)(283.66241091,119.40106037)
\curveto(284.14157751,119.10939374)(284.47751497,118.73439378)(284.67022328,118.2760605)
\curveto(284.86293159,117.81772723)(284.95928575,117.33074812)(284.95928575,116.81512318)
\curveto(284.95928575,116.65366487)(284.95147325,116.49741489)(284.93584825,116.34637324)
\lineto(284.84991076,116.26043575)
\curveto(284.49574414,116.18231076)(284.0191817,116.13022744)(283.42022344,116.10418577)
\curveto(282.82126518,116.07814411)(282.42543189,116.06512328)(282.23272358,116.06512328)
\lineto(279.72491139,116.06512328)
\curveto(279.73532806,114.98699841)(280.00616136,114.19272767)(280.53741129,113.68231107)
\curveto(281.06866123,113.17189447)(281.71970281,112.91668617)(282.49053605,112.91668617)
\curveto(282.85511934,112.91668617)(283.20407763,112.97918616)(283.53741092,113.10418614)
\curveto(283.87595255,113.22918613)(284.23272334,113.40366527)(284.60772329,113.62762358)
\closepath
\moveto(279.72491139,116.6901232)
\curveto(279.81866138,116.6744982)(280.17803634,116.65626904)(280.80303626,116.63543571)
\curveto(281.43324452,116.61460238)(281.89939029,116.60418571)(282.20147359,116.60418571)
\curveto(282.92543183,116.60418571)(283.36553594,116.61720654)(283.52178592,116.64324821)
\curveto(283.52699426,116.76824819)(283.52959842,116.86460235)(283.52959842,116.93231067)
\curveto(283.52959842,117.73960224)(283.36553594,118.3385605)(283.03741098,118.72918545)
\curveto(282.70928602,119.12501873)(282.26136941,119.32293538)(281.69366115,119.32293538)
\curveto(281.07386956,119.32293538)(280.58949462,119.10158124)(280.24053633,118.65887296)
\curveto(279.89678637,118.21616468)(279.72491139,117.55991476)(279.72491139,116.6901232)
\closepath
\moveto(281.90459862,120.28387276)
\closepath
\moveto(281.81084864,111.87762379)
\closepath
}
}
{
\newrgbcolor{curcolor}{0 0 0}
\pscustom[linestyle=none,fillstyle=solid,fillcolor=curcolor]
{
\newpath
\moveto(285.84991064,118.47918548)
\lineto(285.84991064,118.68231045)
\lineto(285.90459813,118.76043545)
\curveto(286.3941814,118.94272709)(286.79001469,119.1119979)(287.09209798,119.26824788)
\curveto(287.09209798,120.62762271)(287.07647299,121.41928928)(287.04522299,121.64324759)
\curveto(287.58168126,121.83074757)(288.02178537,122.03126838)(288.36553533,122.24481002)
\lineto(288.5530353,122.08856003)
\curveto(288.50095198,121.76043508)(288.43845199,120.80731019)(288.36553533,119.22918539)
\curveto(288.62595196,119.22397705)(288.90720193,119.22137289)(289.20928522,119.22137289)
\curveto(289.82386848,119.22137289)(290.26397259,119.23699789)(290.52959756,119.26824788)
\lineto(290.58428505,119.21356039)
\lineto(290.43584757,118.55731047)
\lineto(290.37334758,118.48699798)
\curveto(290.10772261,118.49220631)(289.81345182,118.49481048)(289.49053519,118.49481048)
\curveto(289.19886856,118.49481048)(288.8238686,118.49220631)(288.36553533,118.48699798)
\lineto(288.31866033,115.30731087)
\curveto(288.31866033,114.57293596)(288.33428533,114.09116519)(288.36553533,113.86199855)
\curveto(288.40199366,113.63804024)(288.49574364,113.46095693)(288.64678529,113.33074861)
\curveto(288.80303527,113.20574863)(289.03220191,113.14324864)(289.33428521,113.14324864)
\curveto(289.6832435,113.14324864)(290.00616013,113.23439446)(290.30303509,113.4166861)
\lineto(290.49053507,113.13543614)
\curveto(290.36553508,113.04689448)(290.06605595,112.78647785)(289.59209768,112.35418623)
\curveto(289.32126438,112.22918625)(289.04522274,112.16668626)(288.76397278,112.16668626)
\curveto(287.59209792,112.16668626)(287.0061605,112.73439452)(287.0061605,113.86981105)
\curveto(287.0061605,114.28647766)(287.01657716,114.64064429)(287.03741049,114.93231092)
\curveto(287.04261882,115.02085257)(287.04522299,115.11199839)(287.04522299,115.20574838)
\lineto(287.04522299,118.44793548)
\lineto(286.72491053,118.44793548)
\curveto(286.49053556,118.44793548)(286.22230643,118.43751882)(285.92022313,118.41668549)
\closepath
}
}
{
\newrgbcolor{curcolor}{0 0 0}
\pscustom[linestyle=none,fillstyle=solid,fillcolor=curcolor]
{
\newpath
\moveto(297.61553419,113.49481109)
\lineto(297.36553422,112.93231116)
\curveto(296.82386762,112.58335287)(296.33949268,112.3567904)(295.9124094,112.25262375)
\curveto(295.49053445,112.14845709)(295.11293033,112.09637377)(294.77959704,112.09637377)
\curveto(294.15459711,112.09637377)(293.56084719,112.21876958)(292.99834726,112.46356122)
\curveto(292.44105566,112.70835286)(291.98532655,113.12241531)(291.63115992,113.70574857)
\curveto(291.28220163,114.28908183)(291.10772249,114.99220674)(291.10772249,115.81512331)
\curveto(291.10772249,116.36199824)(291.17543081,116.85418568)(291.31084746,117.29168563)
\curveto(291.44626411,117.7343939)(291.5868891,118.06251886)(291.73272241,118.2760605)
\curveto(291.88376406,118.48960215)(292.1363682,118.72658128)(292.49053482,118.98699792)
\curveto(292.84470144,119.24741455)(293.2197014,119.45574786)(293.61553468,119.61199784)
\curveto(294.01136796,119.76824782)(294.43845125,119.84637281)(294.89678452,119.84637281)
\curveto(295.52178445,119.84637281)(296.07126354,119.69793533)(296.54522182,119.40106037)
\curveto(297.02438843,119.10939374)(297.36032589,118.73439378)(297.55303419,118.2760605)
\curveto(297.7457425,117.81772723)(297.84209666,117.33074812)(297.84209666,116.81512318)
\curveto(297.84209666,116.65366487)(297.83428416,116.49741489)(297.81865916,116.34637324)
\lineto(297.73272167,116.26043575)
\curveto(297.37855505,116.18231076)(296.90199261,116.13022744)(296.30303435,116.10418577)
\curveto(295.70407609,116.07814411)(295.3082428,116.06512328)(295.1155345,116.06512328)
\lineto(292.6077223,116.06512328)
\curveto(292.61813897,114.98699841)(292.88897227,114.19272767)(293.4202222,113.68231107)
\curveto(293.95147214,113.17189447)(294.60251373,112.91668617)(295.37334696,112.91668617)
\curveto(295.73793025,112.91668617)(296.08688854,112.97918616)(296.42022183,113.10418614)
\curveto(296.75876346,113.22918613)(297.11553425,113.40366527)(297.4905342,113.62762358)
\closepath
\moveto(292.6077223,116.6901232)
\curveto(292.70147229,116.6744982)(293.06084725,116.65626904)(293.68584717,116.63543571)
\curveto(294.31605543,116.61460238)(294.7822012,116.60418571)(295.0842845,116.60418571)
\curveto(295.80824274,116.60418571)(296.24834686,116.61720654)(296.40459684,116.64324821)
\curveto(296.40980517,116.76824819)(296.41240934,116.86460235)(296.41240934,116.93231067)
\curveto(296.41240934,117.73960224)(296.24834686,118.3385605)(295.9202219,118.72918545)
\curveto(295.59209694,119.12501873)(295.14418032,119.32293538)(294.57647206,119.32293538)
\curveto(293.95668047,119.32293538)(293.47230553,119.10158124)(293.12334724,118.65887296)
\curveto(292.77959728,118.21616468)(292.6077223,117.55991476)(292.6077223,116.6901232)
\closepath
\moveto(294.78740954,120.28387276)
\closepath
\moveto(294.69365955,111.87762379)
\closepath
}
}
{
\newrgbcolor{curcolor}{0 0 0}
\pscustom[linestyle=none,fillstyle=solid,fillcolor=curcolor]
{
\newpath
\moveto(301.37334622,119.79168532)
\lineto(301.52178371,119.69012283)
\curveto(301.49053371,119.38803953)(301.46709621,118.85158127)(301.45147121,118.08074803)
\lineto(302.03740864,118.82293544)
\curveto(302.23011695,119.06772707)(302.39938776,119.25522705)(302.54522108,119.38543537)
\curveto(302.69626273,119.52085202)(302.87074187,119.62501867)(303.06865851,119.69793533)
\curveto(303.26657516,119.77085199)(303.46970013,119.80731032)(303.67803344,119.80731032)
\curveto(303.90720008,119.80731032)(304.12334588,119.76043532)(304.32647086,119.66668533)
\lineto(304.38115835,119.55731035)
\curveto(304.29782503,118.8646021)(304.25095004,118.25522717)(304.24053337,117.72918557)
\lineto(303.88897091,117.72918557)
\curveto(303.68063761,118.20835218)(303.34470015,118.44793548)(302.88115854,118.44793548)
\curveto(302.55824191,118.44793548)(302.27699195,118.34376883)(302.03740864,118.13543552)
\curveto(301.79782534,117.93231055)(301.63636702,117.67449808)(301.5530337,117.36199812)
\curveto(301.47490871,117.05470649)(301.43584622,116.66408154)(301.43584622,116.19012326)
\lineto(301.43584622,115.32293587)
\curveto(301.43584622,115.16668589)(301.44626288,114.78387344)(301.46709621,114.17449851)
\curveto(301.48792954,113.56512359)(301.50876287,113.21356113)(301.5295962,113.11981114)
\curveto(301.55563787,113.02606115)(301.5920962,112.95574866)(301.63897119,112.90887367)
\curveto(301.69105452,112.867207)(301.75615868,112.83856118)(301.83428367,112.82293618)
\curveto(301.91761699,112.80731118)(302.29001278,112.78126952)(302.95147103,112.74481119)
\lineto(303.02178352,112.68231119)
\lineto(303.02178352,112.35418623)
\lineto(302.95147103,112.28387374)
\curveto(302.25876278,112.32554041)(301.53480454,112.34637374)(300.7795963,112.34637374)
\curveto(300.18063804,112.34637374)(299.55563811,112.32554041)(298.90459653,112.28387374)
\lineto(298.83428404,112.35418623)
\lineto(298.83428404,112.68231119)
\lineto(298.90459653,112.74481119)
\curveto(299.4098048,112.77606118)(299.69365893,112.79949868)(299.75615892,112.81512368)
\curveto(299.82386725,112.83074868)(299.88115891,112.86199867)(299.9280339,112.90887367)
\curveto(299.98011723,112.96095699)(300.01397139,113.03647782)(300.02959639,113.13543614)
\curveto(300.05042972,113.23960279)(300.07126305,113.57554025)(300.09209638,114.14324851)
\curveto(300.11813805,114.71616511)(300.13115888,115.14845672)(300.13115888,115.44012335)
\lineto(300.13115888,116.97918566)
\curveto(300.13115888,117.18751897)(300.12074221,117.4713731)(300.09990888,117.83074806)
\curveto(300.07907555,118.19012302)(300.06084639,118.41147715)(300.04522139,118.49481048)
\curveto(300.03480472,118.5781438)(299.99574223,118.63803963)(299.9280339,118.67449796)
\curveto(299.86032558,118.71616462)(299.72490893,118.73699795)(299.52178395,118.73699795)
\lineto(298.86553403,118.74481045)
\lineto(298.79522154,118.80731044)
\lineto(298.79522154,119.1432479)
\lineto(298.85772153,119.20574789)
\curveto(299.85251308,119.32553954)(300.69105464,119.52085202)(301.37334622,119.79168532)
\closepath
}
}
{
\newrgbcolor{curcolor}{0 0 0}
\pscustom[linestyle=none,fillstyle=solid,fillcolor=curcolor]
{
\newpath
\moveto(307.36553298,119.79168532)
\lineto(307.51397047,119.69012283)
\curveto(307.48272047,119.34116454)(307.46188714,118.88283126)(307.45147047,118.315123)
\curveto(307.86813709,118.65887296)(308.26397037,119.00783125)(308.63897033,119.36199787)
\curveto(308.74834531,119.46095619)(308.8499078,119.53387285)(308.94365779,119.58074784)
\curveto(309.04261611,119.62762284)(309.20407442,119.677102)(309.42803273,119.72918533)
\curveto(309.65199104,119.78126865)(309.88115767,119.80731032)(310.11553265,119.80731032)
\curveto(310.51136593,119.80731032)(310.89417838,119.72658116)(311.26397,119.56512285)
\curveto(311.63896996,119.40366453)(311.92021992,119.21095622)(312.1077199,118.98699792)
\curveto(312.30042821,118.76824794)(312.43324069,118.50783131)(312.50615735,118.20574801)
\curveto(312.57907401,117.90366472)(312.61553234,117.53126893)(312.61553234,117.08856065)
\lineto(312.61553234,115.71356082)
\curveto(312.61553234,115.62501916)(312.62855317,114.97658174)(312.65459483,113.76824856)
\curveto(312.6650115,113.27345695)(312.71969899,112.97918616)(312.81865731,112.88543617)
\curveto(312.91761563,112.79168618)(313.24053226,112.74481119)(313.78740719,112.74481119)
\lineto(313.84990719,112.68231119)
\lineto(313.84990719,112.34637374)
\lineto(313.78740719,112.28387374)
\curveto(313.12594894,112.32554041)(312.67803233,112.34637374)(312.44365736,112.34637374)
\curveto(312.30824071,112.34637374)(311.92803242,112.32554041)(311.3030325,112.28387374)
\lineto(311.20928251,112.36981123)
\curveto(311.27699084,113.04689448)(311.310845,113.92970687)(311.310845,115.01824841)
\lineto(311.310845,116.04168578)
\curveto(311.310845,116.65626904)(311.29522,117.09897732)(311.26397,117.36981062)
\curveto(311.23792834,117.64064392)(311.15199085,117.88543555)(311.00615754,118.10418553)
\curveto(310.86032422,118.32814383)(310.66501174,118.50001881)(310.42022011,118.61981046)
\curveto(310.17542847,118.74481045)(309.88376184,118.80731044)(309.54522022,118.80731044)
\curveto(309.27438692,118.80731044)(309.04261611,118.77866461)(308.8499078,118.72137295)
\curveto(308.65719949,118.66928962)(308.44105369,118.55731047)(308.20147038,118.38543549)
\curveto(307.96188708,118.21876885)(307.78480377,118.0599147)(307.67022045,117.90887305)
\curveto(307.55563713,117.76303973)(307.48532464,117.62501892)(307.45928297,117.4948106)
\curveto(307.43844964,117.36981062)(307.42803298,117.09897732)(307.42803298,116.6823107)
\lineto(307.42803298,115.32293587)
\curveto(307.42803298,115.20835255)(307.43844964,114.83856093)(307.45928297,114.21356101)
\curveto(307.4801163,113.59376942)(307.50094963,113.23179029)(307.52178297,113.12762364)
\curveto(307.54782463,113.02345699)(307.58428296,112.94793616)(307.63115795,112.90106117)
\curveto(307.67803295,112.85418617)(307.73532461,112.82293618)(307.80303293,112.80731118)
\curveto(307.87074126,112.79689451)(308.15199122,112.77606118)(308.64678283,112.74481119)
\lineto(308.71709532,112.68231119)
\lineto(308.71709532,112.35418623)
\lineto(308.65459533,112.28387374)
\curveto(308.00355374,112.32554041)(307.37594965,112.34637374)(306.77178306,112.34637374)
\curveto(306.1728248,112.34637374)(305.54782488,112.32554041)(304.89678329,112.28387374)
\lineto(304.8264708,112.35418623)
\lineto(304.8264708,112.68231119)
\lineto(304.89678329,112.74481119)
\curveto(305.40199156,112.77606118)(305.68584569,112.79949868)(305.74834568,112.81512368)
\curveto(305.81605401,112.83074868)(305.87334567,112.86199867)(305.92022066,112.90887367)
\curveto(305.97230399,112.96095699)(306.00615815,113.03647782)(306.02178315,113.13543614)
\curveto(306.04261648,113.23960279)(306.06344981,113.57554025)(306.08428314,114.14324851)
\curveto(306.11032481,114.71616511)(306.12334564,115.14845672)(306.12334564,115.44012335)
\lineto(306.12334564,116.97918566)
\curveto(306.12334564,117.18751897)(306.11292897,117.4713731)(306.09209564,117.83074806)
\curveto(306.07126231,118.19012302)(306.05303315,118.41147715)(306.03740815,118.49481048)
\curveto(306.02699148,118.5781438)(305.98792899,118.63803963)(305.92022066,118.67449796)
\curveto(305.85251234,118.71616462)(305.71709569,118.73699795)(305.51397071,118.73699795)
\lineto(304.85772079,118.74481045)
\lineto(304.7874083,118.80731044)
\lineto(304.7874083,119.1432479)
\lineto(304.84990829,119.20574789)
\curveto(305.84469984,119.32553954)(306.6832414,119.52085202)(307.36553298,119.79168532)
\closepath
}
}
{
\newrgbcolor{curcolor}{0 0 0}
\pscustom[linestyle=none,fillstyle=solid,fillcolor=curcolor]
{
}
}
{
\newrgbcolor{curcolor}{0 0 0}
\pscustom[linestyle=none,fillstyle=solid,fillcolor=curcolor]
{
\newpath
\moveto(320.70928134,119.79168532)
\lineto(320.85771882,119.69012283)
\curveto(320.80042716,119.00783125)(320.77178133,118.14064385)(320.77178133,117.08856065)
\lineto(320.77178133,115.26824838)
\curveto(320.77178133,114.60158179)(320.82126049,114.13804018)(320.92021881,113.87762355)
\curveto(321.02438547,113.61720691)(321.20146878,113.4166861)(321.45146875,113.27606112)
\curveto(321.70146872,113.14064447)(321.99834368,113.07293615)(322.34209364,113.07293615)
\curveto(322.72230192,113.07293615)(323.06865605,113.1458528)(323.38115601,113.29168612)
\curveto(323.69365597,113.44272777)(323.95407261,113.65366524)(324.16240591,113.92449854)
\curveto(324.37594755,114.19533184)(324.49834337,114.37762349)(324.52959337,114.47137347)
\curveto(324.56084336,114.56512346)(324.5816767,114.82814426)(324.59209336,115.26043588)
\lineto(324.61553086,116.13543577)
\lineto(324.61553086,116.97918566)
\curveto(324.61553086,117.18751897)(324.60511419,117.4713731)(324.58428086,117.83074806)
\curveto(324.56344753,118.19012302)(324.54521837,118.41147715)(324.52959337,118.49481048)
\curveto(324.5191767,118.5781438)(324.48011421,118.63803963)(324.41240588,118.67449796)
\curveto(324.34469756,118.71616462)(324.20928091,118.73699795)(324.00615593,118.73699795)
\lineto(323.34990601,118.74481045)
\lineto(323.27959352,118.80731044)
\lineto(323.27959352,119.1432479)
\lineto(323.34209351,119.20574789)
\curveto(324.33688506,119.32553954)(325.17542662,119.52085202)(325.8577182,119.79168532)
\lineto(326.00615569,119.69012283)
\curveto(325.94886403,119.00783125)(325.9202182,118.14064385)(325.9202182,117.08856065)
\lineto(325.9202182,115.71356082)
\curveto(325.9202182,115.63543583)(325.93323903,114.99741508)(325.95928069,113.79949856)
\curveto(325.96969736,113.35679028)(325.99573902,113.08595698)(326.03740568,112.98699866)
\curveto(326.08428068,112.89324867)(326.14678067,112.82554034)(326.22490566,112.78387368)
\curveto(326.30303065,112.74741535)(326.53480145,112.72918619)(326.92021807,112.72918619)
\lineto(327.13896805,112.72918619)
\lineto(327.20928054,112.6666862)
\lineto(327.20928054,112.35418623)
\lineto(327.14678055,112.28387374)
\curveto(326.41240564,112.32554041)(325.9202182,112.34637374)(325.67021823,112.34637374)
\curveto(325.35250993,112.34637374)(324.98532248,112.32814457)(324.56865586,112.29168624)
\lineto(324.49834337,112.35418623)
\curveto(324.52959337,112.88543617)(324.55303087,113.33595695)(324.56865586,113.70574857)
\curveto(324.2405309,113.45054027)(323.86553095,113.10939448)(323.443656,112.68231119)
\curveto(323.28219769,112.52085288)(323.05042688,112.38543623)(322.74834359,112.27606124)
\curveto(322.44626029,112.16668626)(322.10251033,112.11199876)(321.71709372,112.11199876)
\curveto(321.13376045,112.11199876)(320.67803134,112.20314459)(320.34990638,112.38543623)
\curveto(320.02698976,112.57293621)(319.79782312,112.82814451)(319.66240647,113.15106114)
\curveto(319.52698982,113.4791861)(319.45928149,114.04168603)(319.45928149,114.83856093)
\lineto(319.46709399,115.45574835)
\lineto(319.46709399,116.97918566)
\curveto(319.46709399,117.18751897)(319.45667733,117.4713731)(319.435844,117.83074806)
\curveto(319.41501067,118.19012302)(319.3967815,118.41147715)(319.3811565,118.49481048)
\curveto(319.37073984,118.5781438)(319.33167734,118.63803963)(319.26396902,118.67449796)
\curveto(319.19626069,118.71616462)(319.06084404,118.73699795)(318.85771907,118.73699795)
\lineto(318.20146915,118.74481045)
\lineto(318.13115666,118.80731044)
\lineto(318.13115666,119.1432479)
\lineto(318.19365665,119.20574789)
\curveto(319.18844819,119.32553954)(320.02698976,119.52085202)(320.70928134,119.79168532)
\closepath
\moveto(322.57646861,120.28387276)
\closepath
\moveto(322.6545936,111.87762379)
\closepath
}
}
{
\newrgbcolor{curcolor}{0 0 0}
\pscustom[linestyle=none,fillstyle=solid,fillcolor=curcolor]
{
\newpath
\moveto(330.32646765,119.79168532)
\lineto(330.47490514,119.69012283)
\curveto(330.44365514,119.34116454)(330.42282181,118.88283126)(330.41240514,118.315123)
\curveto(330.82907176,118.65887296)(331.22490504,119.00783125)(331.599905,119.36199787)
\curveto(331.70927998,119.46095619)(331.81084247,119.53387285)(331.90459246,119.58074784)
\curveto(332.00355078,119.62762284)(332.16500909,119.677102)(332.3889674,119.72918533)
\curveto(332.6129257,119.78126865)(332.84209234,119.80731032)(333.07646731,119.80731032)
\curveto(333.4723006,119.80731032)(333.85511305,119.72658116)(334.22490467,119.56512285)
\curveto(334.59990463,119.40366453)(334.88115459,119.21095622)(335.06865457,118.98699792)
\curveto(335.26136288,118.76824794)(335.39417536,118.50783131)(335.46709202,118.20574801)
\curveto(335.54000868,117.90366472)(335.57646701,117.53126893)(335.57646701,117.08856065)
\lineto(335.57646701,115.71356082)
\curveto(335.57646701,115.62501916)(335.58948784,114.97658174)(335.6155295,113.76824856)
\curveto(335.62594617,113.27345695)(335.68063366,112.97918616)(335.77959198,112.88543617)
\curveto(335.8785503,112.79168618)(336.20146693,112.74481119)(336.74834186,112.74481119)
\lineto(336.81084185,112.68231119)
\lineto(336.81084185,112.34637374)
\lineto(336.74834186,112.28387374)
\curveto(336.08688361,112.32554041)(335.638967,112.34637374)(335.40459203,112.34637374)
\curveto(335.26917538,112.34637374)(334.88896709,112.32554041)(334.26396717,112.28387374)
\lineto(334.17021718,112.36981123)
\curveto(334.2379255,113.04689448)(334.27177967,113.92970687)(334.27177967,115.01824841)
\lineto(334.27177967,116.04168578)
\curveto(334.27177967,116.65626904)(334.25615467,117.09897732)(334.22490467,117.36981062)
\curveto(334.19886301,117.64064392)(334.11292552,117.88543555)(333.9670922,118.10418553)
\curveto(333.82125889,118.32814383)(333.62594641,118.50001881)(333.38115478,118.61981046)
\curveto(333.13636314,118.74481045)(332.84469651,118.80731044)(332.50615488,118.80731044)
\curveto(332.23532158,118.80731044)(332.00355078,118.77866461)(331.81084247,118.72137295)
\curveto(331.61813416,118.66928962)(331.40198835,118.55731047)(331.16240505,118.38543549)
\curveto(330.92282175,118.21876885)(330.74573844,118.0599147)(330.63115512,117.90887305)
\curveto(330.5165718,117.76303973)(330.44625931,117.62501892)(330.42021764,117.4948106)
\curveto(330.39938431,117.36981062)(330.38896765,117.09897732)(330.38896765,116.6823107)
\lineto(330.38896765,115.32293587)
\curveto(330.38896765,115.20835255)(330.39938431,114.83856093)(330.42021764,114.21356101)
\curveto(330.44105097,113.59376942)(330.4618843,113.23179029)(330.48271763,113.12762364)
\curveto(330.5087593,113.02345699)(330.54521763,112.94793616)(330.59209262,112.90106117)
\curveto(330.63896762,112.85418617)(330.69625927,112.82293618)(330.7639676,112.80731118)
\curveto(330.83167592,112.79689451)(331.11292589,112.77606118)(331.6077175,112.74481119)
\lineto(331.67802999,112.68231119)
\lineto(331.67802999,112.35418623)
\lineto(331.61552999,112.28387374)
\curveto(330.96448841,112.32554041)(330.33688432,112.34637374)(329.73271773,112.34637374)
\curveto(329.13375947,112.34637374)(328.50875954,112.32554041)(327.85771796,112.28387374)
\lineto(327.78740547,112.35418623)
\lineto(327.78740547,112.68231119)
\lineto(327.85771796,112.74481119)
\curveto(328.36292623,112.77606118)(328.64678036,112.79949868)(328.70928035,112.81512368)
\curveto(328.77698868,112.83074868)(328.83428034,112.86199867)(328.88115533,112.90887367)
\curveto(328.93323866,112.96095699)(328.96709282,113.03647782)(328.98271782,113.13543614)
\curveto(329.00355115,113.23960279)(329.02438448,113.57554025)(329.04521781,114.14324851)
\curveto(329.07125947,114.71616511)(329.08428031,115.14845672)(329.08428031,115.44012335)
\lineto(329.08428031,116.97918566)
\curveto(329.08428031,117.18751897)(329.07386364,117.4713731)(329.05303031,117.83074806)
\curveto(329.03219698,118.19012302)(329.01396782,118.41147715)(328.99834282,118.49481048)
\curveto(328.98792615,118.5781438)(328.94886366,118.63803963)(328.88115533,118.67449796)
\curveto(328.81344701,118.71616462)(328.67803036,118.73699795)(328.47490538,118.73699795)
\lineto(327.81865546,118.74481045)
\lineto(327.74834297,118.80731044)
\lineto(327.74834297,119.1432479)
\lineto(327.81084296,119.20574789)
\curveto(328.80563451,119.32553954)(329.64417607,119.52085202)(330.32646765,119.79168532)
\closepath
}
}
{
\newrgbcolor{curcolor}{0 0 0}
\pscustom[linestyle=none,fillstyle=solid,fillcolor=curcolor]
{
\newpath
\moveto(342.22490369,122.94012243)
\lineto(342.22490369,123.26824739)
\lineto(342.28740368,123.33855988)
\curveto(343.28219522,123.4531432)(344.12073679,123.64585151)(344.80302837,123.91668481)
\lineto(344.95146585,123.82293482)
\curveto(344.89417419,123.22397656)(344.86552836,122.02345588)(344.86552836,120.22137276)
\lineto(344.86552836,115.1041859)
\curveto(344.86552836,114.57814429)(344.87334086,114.11720685)(344.88896586,113.72137357)
\curveto(344.90979919,113.33074861)(344.94365335,113.09897781)(344.99052835,113.02606115)
\curveto(345.03740334,112.95314449)(345.11292416,112.89324867)(345.21709082,112.84637367)
\curveto(345.3264658,112.79949868)(345.63115327,112.75522785)(346.1311532,112.71356119)
\lineto(346.2014657,112.6510612)
\lineto(346.2014657,112.35418623)
\lineto(346.1311532,112.28387374)
\curveto(345.56865327,112.32033207)(345.13115333,112.33856124)(344.81865337,112.33856124)
\curveto(344.5374034,112.33856124)(344.13636178,112.32033207)(343.61552851,112.28387374)
\lineto(343.52959103,112.36981123)
\curveto(343.54521602,112.88022784)(343.55302852,113.23699863)(343.55302852,113.4401236)
\curveto(343.55302852,113.46616526)(343.55563269,113.56512359)(343.56084102,113.73699856)
\curveto(343.27438272,113.52866526)(342.98792443,113.29689445)(342.70146613,113.04168615)
\curveto(342.27959118,112.6666862)(342.00354955,112.43231123)(341.87334123,112.33856124)
\curveto(341.63375793,112.21876958)(341.29261213,112.15887376)(340.84990386,112.15887376)
\curveto(340.13115394,112.15887376)(339.51917485,112.33595707)(339.01396658,112.69012369)
\curveto(338.51396664,113.04429032)(338.15719585,113.49481109)(337.94365421,114.04168603)
\curveto(337.73011257,114.59376929)(337.62334175,115.15887339)(337.62334175,115.73699832)
\curveto(337.62334175,116.32553991)(337.73271674,116.89064401)(337.95146671,117.43231061)
\curveto(338.17542502,117.97918554)(338.49052915,118.37762299)(338.8967791,118.62762296)
\curveto(339.30823738,118.87762293)(339.75354982,119.1354354)(340.23271643,119.40106037)
\curveto(340.71709137,119.67189367)(341.19625798,119.80731032)(341.67021625,119.80731032)
\curveto(342.33167451,119.80731032)(342.96188276,119.64324784)(343.56084102,119.31512288)
\lineto(343.52959103,121.05731016)
\curveto(343.51917436,121.66147675)(343.50354936,122.08074754)(343.48271603,122.31512251)
\curveto(343.4618827,122.55470581)(343.43584104,122.69533079)(343.40459104,122.73699745)
\curveto(343.37334104,122.78387245)(343.31865355,122.81772661)(343.24052856,122.83855994)
\curveto(343.1676119,122.85939327)(342.85250778,122.86980994)(342.29521618,122.86980994)
\closepath
\moveto(343.56084102,117.97918554)
\curveto(343.23792439,118.34376883)(342.87854944,118.62241463)(342.48271615,118.81512294)
\curveto(342.0920912,119.00783125)(341.69886208,119.1041854)(341.3030288,119.1041854)
\curveto(340.86552885,119.1041854)(340.45667474,118.98439375)(340.07646645,118.74481045)
\curveto(339.7014665,118.51043548)(339.4306332,118.15887302)(339.26396655,117.69012308)
\curveto(339.09729991,117.22658147)(339.01396658,116.7213732)(339.01396658,116.17449826)
\curveto(339.01396658,115.26304004)(339.24313322,114.5312693)(339.7014665,113.97918603)
\curveto(340.16500811,113.42710277)(340.7457372,113.15106114)(341.44365378,113.15106114)
\curveto(341.82386207,113.15106114)(342.15979953,113.23699863)(342.45146616,113.4088736)
\curveto(342.74834112,113.58595692)(342.99052859,113.82293605)(343.17802857,114.11981102)
\curveto(343.37073688,114.41668598)(343.48271603,114.71356094)(343.51396603,115.01043591)
\curveto(343.54521602,115.30731087)(343.56084102,115.85158164)(343.56084102,116.64324821)
\closepath
}
}
{
\newrgbcolor{curcolor}{0 0 0}
\pscustom[linestyle=none,fillstyle=solid,fillcolor=curcolor]
{
}
}
{
\newrgbcolor{curcolor}{0 0 0}
\pscustom[linestyle=none,fillstyle=solid,fillcolor=curcolor]
{
\newpath
\moveto(351.35771506,115.0338734)
\lineto(351.42021505,115.0963734)
\lineto(351.80302751,115.0963734)
\lineto(351.8655275,115.0338734)
\curveto(351.89156916,114.36720682)(351.92542332,113.9583527)(351.96708999,113.80731106)
\curveto(352.01396498,113.66147774)(352.15458996,113.50522776)(352.38896493,113.33856111)
\curveto(352.62854824,113.1771028)(352.94365236,113.04429032)(353.33427732,112.94012366)
\curveto(353.7301106,112.83595701)(354.12854805,112.78387368)(354.52958967,112.78387368)
\curveto(355.08167293,112.78387368)(355.57646454,112.88543617)(356.01396449,113.08856114)
\curveto(356.45667276,113.29168612)(356.80042272,113.58856108)(357.04521436,113.97918603)
\curveto(357.290006,114.37501932)(357.41240181,114.8177276)(357.41240181,115.30731087)
\curveto(357.41240181,115.65106083)(357.35250599,115.95053996)(357.23271434,116.20574826)
\curveto(357.11813102,116.46095656)(356.95406854,116.66408154)(356.7405269,116.81512318)
\curveto(356.53219359,116.97137317)(356.29261029,117.08595648)(356.02177699,117.15887314)
\curveto(355.75094369,117.23699813)(355.3472979,117.32033146)(354.81083963,117.40887311)
\curveto(354.29000637,117.49220643)(353.88115225,117.56772726)(353.58427729,117.63543558)
\curveto(353.29261066,117.70314391)(353.00094402,117.80731056)(352.70927739,117.94793555)
\curveto(352.41761076,118.08856053)(352.17281913,118.26564384)(351.97490248,118.47918548)
\curveto(351.78219417,118.69793545)(351.62594419,118.96356042)(351.50615254,119.27606038)
\curveto(351.39156922,119.59376868)(351.33427756,119.9323103)(351.33427756,120.29168526)
\curveto(351.33427756,121.26564347)(351.67281919,122.06251837)(352.34990244,122.68230996)
\curveto(353.02698569,123.30730988)(353.93583974,123.61980985)(355.0764646,123.61980985)
\curveto(355.52958955,123.61980985)(356.02177699,123.56251819)(356.55302692,123.44793487)
\curveto(357.08948519,123.33855988)(357.57646429,123.1744974)(358.01396424,122.95574743)
\lineto(358.06865173,122.86199744)
\curveto(357.95406841,122.38283083)(357.88375592,121.70053925)(357.85771426,120.81512269)
\lineto(357.78740177,120.7526227)
\lineto(357.38115182,120.7526227)
\lineto(357.31865183,120.80731019)
\curveto(357.30823516,121.43231012)(357.29261016,121.82553923)(357.27177683,121.98699755)
\curveto(357.2509435,122.14845586)(357.01396436,122.34897667)(356.56083942,122.58855997)
\curveto(356.10771447,122.82814328)(355.61552704,122.94793493)(355.0842771,122.94793493)
\curveto(354.64156882,122.94793493)(354.22750637,122.85939327)(353.84208975,122.68230996)
\curveto(353.45667313,122.50522665)(353.1650065,122.22658085)(352.96708986,121.84637256)
\curveto(352.76917322,121.46616428)(352.6702149,121.07814349)(352.6702149,120.68231021)
\curveto(352.6702149,120.37501858)(352.73011072,120.10158111)(352.84990238,119.86199781)
\curveto(352.96969403,119.62762284)(353.12333984,119.44012286)(353.31083982,119.29949788)
\curveto(353.50354813,119.16408123)(353.72490227,119.06251874)(353.97490224,118.99481042)
\curveto(354.23011054,118.92710209)(354.69365215,118.84376877)(355.36552707,118.74481045)
\curveto(356.28740195,118.61460213)(356.9566727,118.45574798)(357.37333932,118.26824801)
\curveto(357.79521427,118.08595636)(358.13115173,117.79428973)(358.38115169,117.39324811)
\curveto(358.63636,116.9922065)(358.76396415,116.50783156)(358.76396415,115.94012329)
\curveto(358.76396415,114.83074843)(358.3108392,113.90626938)(357.40458931,113.16668613)
\curveto(356.50354776,112.42710289)(355.40979789,112.05731127)(354.12333972,112.05731127)
\curveto(353.04521485,112.05731127)(352.09990247,112.25001958)(351.28740257,112.6354362)
\lineto(351.24052757,112.73699869)
\curveto(351.2926109,113.06512365)(351.3316734,113.83074855)(351.35771506,115.0338734)
\closepath
}
}
{
\newrgbcolor{curcolor}{0 0 0}
\pscustom[linestyle=none,fillstyle=solid,fillcolor=curcolor]
{
\newpath
\moveto(359.49052656,108.10418676)
\curveto(359.66760987,108.54168671)(359.80042235,108.94272832)(359.88896401,109.30731161)
\lineto(360.07646399,109.30731161)
\curveto(360.33167229,109.03126998)(360.60771392,108.89324916)(360.90458888,108.89324916)
\curveto(361.2743805,108.89324916)(361.59469296,109.07814497)(361.86552626,109.44793659)
\curveto(362.1415679,109.81251988)(362.52958868,110.62241562)(363.02958862,111.87762379)
\curveto(362.83167198,112.43491539)(362.63115117,112.96095699)(362.4280262,113.4557486)
\lineto(361.18583885,116.53387322)
\curveto(361.05563053,116.85158151)(360.86292222,117.30470646)(360.60771392,117.89324805)
\curveto(360.35250562,118.48699798)(360.1988598,118.81251877)(360.14677648,118.86981043)
\curveto(360.09990148,118.93231042)(360.03740149,118.98960208)(359.9592765,119.04168541)
\curveto(359.88115151,119.09376874)(359.67021404,119.1432479)(359.32646408,119.19012289)
\lineto(359.26396409,119.25262288)
\lineto(359.26396409,119.56512285)
\lineto(359.33427658,119.62762284)
\curveto(359.93323484,119.59116451)(360.54781809,119.57293534)(361.17802635,119.57293534)
\curveto(361.91760959,119.57293534)(362.48792202,119.59116451)(362.88896364,119.62762284)
\lineto(362.95146363,119.56512285)
\lineto(362.95146363,119.25262288)
\lineto(362.89677614,119.19012289)
\curveto(362.83948448,119.18491456)(362.67802616,119.17189373)(362.4124012,119.1510604)
\curveto(362.15198456,119.1354354)(361.98792208,119.09376874)(361.92021376,119.02606041)
\curveto(361.85771377,118.96356042)(361.82646377,118.89064376)(361.82646377,118.80731044)
\curveto(361.82646377,118.71876878)(361.90719293,118.42970632)(362.06865124,117.94012305)
\curveto(362.23010955,117.45574811)(362.38635953,117.02606066)(362.53740118,116.65106071)
\lineto(363.02958862,115.43231086)
\curveto(363.35250525,114.63022762)(363.60250522,114.03908186)(363.77958853,113.65887357)
\lineto(364.19365098,114.57293596)
\curveto(364.33948429,114.90626925)(364.5478176,115.41147752)(364.8186509,116.08856077)
\lineto(365.46708832,117.74481057)
\curveto(365.63375497,118.17189385)(365.72750496,118.43751882)(365.74833829,118.54168547)
\curveto(365.77437995,118.64585213)(365.78740078,118.72137295)(365.78740078,118.76824794)
\curveto(365.78740078,118.88283126)(365.74573412,118.96095625)(365.6624008,119.00262292)
\curveto(365.57906747,119.04428958)(365.39417166,119.08074791)(365.10771336,119.1119979)
\lineto(364.77958841,119.1588729)
\lineto(364.72490091,119.22137289)
\lineto(364.72490091,119.56512285)
\lineto(364.7874009,119.62762284)
\curveto(365.22490085,119.59116451)(365.74312995,119.57293534)(366.34208821,119.57293534)
\curveto(366.88896315,119.57293534)(367.32646309,119.59116451)(367.65458805,119.62762284)
\lineto(367.71708804,119.56512285)
\lineto(367.71708804,119.22137289)
\lineto(367.65458805,119.1588729)
\curveto(367.46187974,119.15366456)(367.30823393,119.1276229)(367.19365061,119.08074791)
\curveto(367.07906729,119.03387291)(366.9696923,118.95053959)(366.86552565,118.83074794)
\curveto(366.76135899,118.71095628)(366.57385902,118.37241466)(366.30302572,117.81512306)
\curveto(366.03219242,117.2630398)(365.76135912,116.6901232)(365.49052582,116.09637327)
\lineto(364.72490091,114.41668598)
\lineto(362.92021363,110.1979365)
\curveto(362.76396365,109.83335321)(362.56865118,109.44272826)(362.33427621,109.02606165)
\curveto(362.10510957,108.60939503)(361.82125544,108.3099159)(361.48271381,108.12762426)
\curveto(361.14938052,107.94012428)(360.7952139,107.84637429)(360.42021394,107.84637429)
\curveto(360.09208898,107.84637429)(359.78219319,107.93231178)(359.49052656,108.10418676)
\closepath
\moveto(363.67021354,120.28387276)
\closepath
\moveto(365.50615082,111.87762379)
\closepath
}
}
{
\newrgbcolor{curcolor}{0 0 0}
\pscustom[linestyle=none,fillstyle=solid,fillcolor=curcolor]
{
\newpath
\moveto(368.82646291,114.64324845)
\lineto(369.16240036,114.64324845)
\lineto(369.23271286,114.57293596)
\curveto(369.24312952,114.13543602)(369.26917119,113.76043606)(369.31083785,113.4479361)
\curveto(369.47229616,113.20314446)(369.75094196,113.00262365)(370.14677524,112.84637367)
\curveto(370.54260853,112.69533203)(370.93323348,112.6198112)(371.3186501,112.6198112)
\curveto(371.87073336,112.6198112)(372.31083748,112.76824868)(372.63896244,113.06512365)
\curveto(372.97229573,113.36199861)(373.13896237,113.71616523)(373.13896237,114.12762352)
\curveto(373.13896237,114.35158182)(373.07906655,114.54429013)(372.9592749,114.70574844)
\curveto(372.83948324,114.87241509)(372.6571916,115.00783174)(372.41239996,115.11199839)
\curveto(372.17281666,115.22137338)(371.74052505,115.34897753)(371.11552512,115.49481085)
\curveto(370.57906686,115.61981083)(370.2040669,115.71616499)(369.99052526,115.78387331)
\curveto(369.77698362,115.85678997)(369.57125448,115.97658162)(369.37333784,116.14324827)
\curveto(369.1754212,116.30991491)(369.02437955,116.51303989)(368.92021289,116.75262319)
\curveto(368.81604624,116.99741483)(368.76396291,117.26564396)(368.76396291,117.55731059)
\curveto(368.76396291,118.25522717)(369.04000455,118.81251877)(369.59208781,119.22918539)
\curveto(370.14937941,119.65106034)(370.84208766,119.86199781)(371.67021256,119.86199781)
\curveto(372.01917085,119.86199781)(372.39937913,119.81251865)(372.81083742,119.71356033)
\curveto(373.2222957,119.61981034)(373.53479566,119.52866452)(373.7483373,119.44012286)
\lineto(373.81864979,119.33074787)
\curveto(373.77698313,119.12241457)(373.75094147,118.57553963)(373.7405248,117.69012308)
\lineto(373.67021231,117.61981059)
\lineto(373.35771235,117.61981059)
\lineto(373.28739986,117.69012308)
\curveto(373.26656653,118.00783137)(373.2379207,118.22918551)(373.20146237,118.3541855)
\curveto(373.16500404,118.48439381)(373.07125405,118.62241463)(372.9202124,118.76824794)
\curveto(372.76917075,118.91928959)(372.55562911,119.04428958)(372.27958748,119.1432479)
\curveto(372.00354585,119.24741455)(371.71187922,119.29949788)(371.40458759,119.29949788)
\curveto(371.08687929,119.29949788)(370.81865016,119.25262288)(370.59990019,119.1588729)
\curveto(370.38635855,119.06512291)(370.2118794,118.92189376)(370.07646275,118.72918545)
\curveto(369.94625443,118.54168547)(369.88115028,118.31251883)(369.88115028,118.04168553)
\curveto(369.88115028,117.84376889)(369.92021277,117.66668558)(369.99833776,117.5104356)
\curveto(370.08167108,117.35418562)(370.20667107,117.22658147)(370.37333772,117.12762315)
\curveto(370.54000436,117.03387316)(370.71448351,116.96616483)(370.89677515,116.92449817)
\lineto(371.74833755,116.7057482)
\curveto(372.47750412,116.52866489)(372.99833739,116.37501907)(373.31083735,116.24481076)
\curveto(373.62854565,116.11460244)(373.87333728,115.9166858)(374.04521226,115.65106083)
\curveto(374.21708724,115.39064419)(374.30302473,115.0729359)(374.30302473,114.69793595)
\curveto(374.30302473,113.97918603)(374.00614977,113.36199861)(373.41239984,112.84637367)
\curveto(372.81864991,112.33074874)(372.05562917,112.07293627)(371.12333762,112.07293627)
\curveto(370.79521266,112.07293627)(370.38635855,112.1093946)(369.89677527,112.18231126)
\curveto(369.41240033,112.25522791)(368.99833789,112.33595707)(368.65458793,112.42449873)
\lineto(368.61552543,112.52606121)
\lineto(368.69365042,113.04949865)
\curveto(368.71969209,113.21095696)(368.73531708,113.37241528)(368.74052542,113.53387359)
\curveto(368.74573375,113.70054024)(368.75094208,114.04689436)(368.75615042,114.57293596)
\closepath
}
}
{
\newrgbcolor{curcolor}{0 0 0}
\pscustom[linestyle=none,fillstyle=solid,fillcolor=curcolor]
{
\newpath
\moveto(375.17802462,118.47918548)
\lineto(375.17802462,118.68231045)
\lineto(375.23271212,118.76043545)
\curveto(375.72229539,118.94272709)(376.11812867,119.1119979)(376.42021197,119.26824788)
\curveto(376.42021197,120.62762271)(376.40458697,121.41928928)(376.37333698,121.64324759)
\curveto(376.90979524,121.83074757)(377.34989936,122.03126838)(377.69364931,122.24481002)
\lineto(377.88114929,122.08856003)
\curveto(377.82906596,121.76043508)(377.76656597,120.80731019)(377.69364931,119.22918539)
\curveto(377.95406595,119.22397705)(378.23531591,119.22137289)(378.53739921,119.22137289)
\curveto(379.15198247,119.22137289)(379.59208658,119.23699789)(379.85771155,119.26824788)
\lineto(379.91239904,119.21356039)
\lineto(379.76396156,118.55731047)
\lineto(379.70146157,118.48699798)
\curveto(379.4358366,118.49220631)(379.1415658,118.49481048)(378.81864917,118.49481048)
\curveto(378.52698254,118.49481048)(378.15198259,118.49220631)(377.69364931,118.48699798)
\lineto(377.64677432,115.30731087)
\curveto(377.64677432,114.57293596)(377.66239932,114.09116519)(377.69364931,113.86199855)
\curveto(377.73010764,113.63804024)(377.82385763,113.46095693)(377.97489928,113.33074861)
\curveto(378.13114926,113.20574863)(378.3603159,113.14324864)(378.66239919,113.14324864)
\curveto(379.01135748,113.14324864)(379.33427411,113.23439446)(379.63114907,113.4166861)
\lineto(379.81864905,113.13543614)
\curveto(379.69364907,113.04689448)(379.39416994,112.78647785)(378.92021166,112.35418623)
\curveto(378.64937836,112.22918625)(378.37333673,112.16668626)(378.09208676,112.16668626)
\curveto(376.92021191,112.16668626)(376.33427448,112.73439452)(376.33427448,113.86981105)
\curveto(376.33427448,114.28647766)(376.34469115,114.64064429)(376.36552448,114.93231092)
\curveto(376.37073281,115.02085257)(376.37333698,115.11199839)(376.37333698,115.20574838)
\lineto(376.37333698,118.44793548)
\lineto(376.05302452,118.44793548)
\curveto(375.81864954,118.44793548)(375.55042041,118.43751882)(375.24833711,118.41668549)
\closepath
}
}
{
\newrgbcolor{curcolor}{0 0 0}
\pscustom[linestyle=none,fillstyle=solid,fillcolor=curcolor]
{
\newpath
\moveto(386.94364817,113.49481109)
\lineto(386.6936482,112.93231116)
\curveto(386.1519816,112.58335287)(385.66760666,112.3567904)(385.24052338,112.25262375)
\curveto(384.81864843,112.14845709)(384.44104431,112.09637377)(384.10771102,112.09637377)
\curveto(383.4827111,112.09637377)(382.88896117,112.21876958)(382.32646124,112.46356122)
\curveto(381.76916964,112.70835286)(381.31344053,113.12241531)(380.95927391,113.70574857)
\curveto(380.61031562,114.28908183)(380.43583647,114.99220674)(380.43583647,115.81512331)
\curveto(380.43583647,116.36199824)(380.5035448,116.85418568)(380.63896145,117.29168563)
\curveto(380.7743781,117.7343939)(380.91500308,118.06251886)(381.0608364,118.2760605)
\curveto(381.21187805,118.48960215)(381.46448218,118.72658128)(381.8186488,118.98699792)
\curveto(382.17281543,119.24741455)(382.54781538,119.45574786)(382.94364867,119.61199784)
\curveto(383.33948195,119.76824782)(383.76656523,119.84637281)(384.22489851,119.84637281)
\curveto(384.84989843,119.84637281)(385.39937753,119.69793533)(385.8733358,119.40106037)
\curveto(386.35250241,119.10939374)(386.68843987,118.73439378)(386.88114818,118.2760605)
\curveto(387.07385649,117.81772723)(387.17021064,117.33074812)(387.17021064,116.81512318)
\curveto(387.17021064,116.65366487)(387.16239815,116.49741489)(387.14677315,116.34637324)
\lineto(387.06083566,116.26043575)
\curveto(386.70666903,116.18231076)(386.23010659,116.13022744)(385.63114833,116.10418577)
\curveto(385.03219007,116.07814411)(384.63635679,116.06512328)(384.44364848,116.06512328)
\lineto(381.93583629,116.06512328)
\curveto(381.94625296,114.98699841)(382.21708626,114.19272767)(382.74833619,113.68231107)
\curveto(383.27958612,113.17189447)(383.93062771,112.91668617)(384.70146095,112.91668617)
\curveto(385.06604424,112.91668617)(385.41500253,112.97918616)(385.74833582,113.10418614)
\curveto(386.08687744,113.22918613)(386.44364823,113.40366527)(386.81864819,113.62762358)
\closepath
\moveto(381.93583629,116.6901232)
\curveto(382.02958628,116.6744982)(382.38896123,116.65626904)(383.01396116,116.63543571)
\curveto(383.64416941,116.61460238)(384.11031519,116.60418571)(384.41239848,116.60418571)
\curveto(385.13635673,116.60418571)(385.57646084,116.61720654)(385.73271082,116.64324821)
\curveto(385.73791915,116.76824819)(385.74052332,116.86460235)(385.74052332,116.93231067)
\curveto(385.74052332,117.73960224)(385.57646084,118.3385605)(385.24833588,118.72918545)
\curveto(384.92021092,119.12501873)(384.47229431,119.32293538)(383.90458605,119.32293538)
\curveto(383.28479446,119.32293538)(382.80041952,119.10158124)(382.45146123,118.65887296)
\curveto(382.10771127,118.21616468)(381.93583629,117.55991476)(381.93583629,116.6901232)
\closepath
\moveto(384.11552352,120.28387276)
\closepath
\moveto(384.02177353,111.87762379)
\closepath
}
}
{
\newrgbcolor{curcolor}{0 0 0}
\pscustom[linestyle=none,fillstyle=solid,fillcolor=curcolor]
{
\newpath
\moveto(390.57646022,119.79168532)
\lineto(390.72489771,119.69012283)
\curveto(390.69364771,119.34637287)(390.67281438,118.93491459)(390.66239771,118.45574798)
\lineto(391.40458512,119.1354354)
\curveto(391.61291843,119.32814371)(391.75093925,119.45053953)(391.81864757,119.50262285)
\curveto(391.8863559,119.55991451)(392.04521004,119.62501867)(392.29521001,119.69793533)
\curveto(392.54520998,119.77085199)(392.80823078,119.80731032)(393.08427242,119.80731032)
\curveto(393.60510568,119.80731032)(394.06083479,119.677102)(394.45145975,119.41668536)
\curveto(394.8420847,119.16147706)(395.13114716,118.80731044)(395.31864714,118.3541855)
\curveto(396.02177205,119.02085208)(396.43062617,119.3906437)(396.54520949,119.46356036)
\curveto(396.66500114,119.53647702)(396.85250112,119.60939367)(397.10770942,119.68231033)
\curveto(397.36291772,119.76043532)(397.61812602,119.79949782)(397.87333432,119.79949782)
\curveto(398.28479261,119.79949782)(398.66239673,119.71095616)(399.00614669,119.53387285)
\curveto(399.34989664,119.35678954)(399.62333411,119.12501873)(399.82645908,118.83856044)
\curveto(400.02958406,118.55210214)(400.14677154,118.24220634)(400.17802154,117.90887305)
\curveto(400.21447987,117.57553976)(400.23270903,117.11720648)(400.23270903,116.53387322)
\lineto(400.23270903,115.71356082)
\curveto(400.23270903,115.62501916)(400.24833403,114.97658174)(400.27958403,113.76824856)
\curveto(400.29000069,113.27345695)(400.34468819,112.97918616)(400.44364651,112.88543617)
\curveto(400.54260483,112.79168618)(400.86291729,112.74481119)(401.40458389,112.74481119)
\lineto(401.46708388,112.68231119)
\lineto(401.46708388,112.34637374)
\lineto(401.40458389,112.28387374)
\curveto(400.74312564,112.32554041)(400.29520903,112.34637374)(400.06083406,112.34637374)
\curveto(399.93062574,112.34637374)(399.55302162,112.32554041)(398.92802169,112.28387374)
\lineto(398.82645921,112.36981123)
\curveto(398.89416753,113.04689448)(398.92802169,113.92970687)(398.92802169,115.01824841)
\lineto(398.92802169,115.95574829)
\curveto(398.92802169,116.78387319)(398.8889592,117.36199812)(398.81083421,117.69012308)
\curveto(398.73270922,118.02345637)(398.55041757,118.29428967)(398.26395928,118.50262298)
\curveto(397.97750098,118.71616462)(397.63635519,118.82293544)(397.2405219,118.82293544)
\curveto(396.95927194,118.82293544)(396.69625114,118.76564378)(396.4514595,118.65106046)
\curveto(396.20666786,118.54168547)(395.98531372,118.37501883)(395.78739708,118.15106052)
\curveto(395.59468877,117.93231055)(395.48791795,117.72658141)(395.46708462,117.5338731)
\curveto(395.44625129,117.34637312)(395.43583463,116.99741483)(395.43583463,116.48699823)
\lineto(395.43583463,115.51043585)
\curveto(395.43583463,115.28126921)(395.44625129,114.85158176)(395.46708462,114.2213735)
\curveto(395.48791795,113.59116525)(395.50875128,113.22397779)(395.52958461,113.11981114)
\curveto(395.55562628,113.01564449)(395.59208461,112.94012366)(395.6389596,112.89324867)
\curveto(395.69104293,112.85158201)(395.74833459,112.82293618)(395.81083458,112.80731118)
\curveto(395.8785429,112.79689451)(396.15979287,112.77606118)(396.65458448,112.74481119)
\lineto(396.72489697,112.68231119)
\lineto(396.72489697,112.35418623)
\lineto(396.66239697,112.28387374)
\curveto(396.01135539,112.32554041)(395.3837513,112.34637374)(394.77958471,112.34637374)
\curveto(394.22229311,112.34637374)(393.59729319,112.32554041)(392.90458494,112.28387374)
\lineto(392.83427245,112.35418623)
\lineto(392.83427245,112.68231119)
\lineto(392.90458494,112.74481119)
\curveto(393.40979321,112.77606118)(393.69364734,112.79949868)(393.75614733,112.81512368)
\curveto(393.82385566,112.83074868)(393.88114732,112.86199867)(393.92802231,112.90887367)
\curveto(393.98010564,112.96095699)(394.0139598,113.03647782)(394.0295848,113.13543614)
\curveto(394.05041813,113.23960279)(394.07125146,113.57554025)(394.09208479,114.14324851)
\curveto(394.11812645,114.71616511)(394.13114729,115.14845672)(394.13114729,115.44012335)
\lineto(394.13114729,116.33074824)
\curveto(394.13114729,116.91928984)(394.08948062,117.37241478)(394.0061473,117.69012308)
\curveto(393.92802231,118.00783137)(393.75614733,118.2760605)(393.49052237,118.49481048)
\curveto(393.2248974,118.71356045)(392.89156411,118.82293544)(392.49052249,118.82293544)
\curveto(392.17802253,118.82293544)(391.89156423,118.76043545)(391.63114759,118.63543546)
\curveto(391.37593929,118.51564381)(391.16500182,118.36460216)(390.99833517,118.18231052)
\curveto(390.83687686,118.0052272)(390.73531437,117.83856056)(390.69364771,117.68231058)
\curveto(390.65718938,117.5260606)(390.63896022,117.18751897)(390.63896022,116.6666857)
\lineto(390.63896022,115.51043585)
\curveto(390.63896022,115.28126921)(390.64937688,114.85158176)(390.67021021,114.2213735)
\curveto(390.69104354,113.59116525)(390.71187687,113.22397779)(390.73271021,113.11981114)
\curveto(390.75875187,113.01564449)(390.7952102,112.94012366)(390.84208519,112.89324867)
\curveto(390.89416852,112.85158201)(390.95146018,112.82293618)(391.01396017,112.80731118)
\curveto(391.0816685,112.79689451)(391.36291846,112.77606118)(391.85771007,112.74481119)
\lineto(391.92802256,112.68231119)
\lineto(391.92802256,112.35418623)
\lineto(391.86552257,112.28387374)
\curveto(391.21448098,112.32554041)(390.58687689,112.34637374)(389.9827103,112.34637374)
\curveto(389.38375204,112.34637374)(388.75875212,112.32554041)(388.10771053,112.28387374)
\lineto(388.03739804,112.35418623)
\lineto(388.03739804,112.68231119)
\lineto(388.10771053,112.74481119)
\curveto(388.6129188,112.77606118)(388.89677293,112.79949868)(388.95927292,112.81512368)
\curveto(389.02698125,112.83074868)(389.08427291,112.86199867)(389.1311479,112.90887367)
\curveto(389.18323123,112.96095699)(389.21708539,113.03647782)(389.23271039,113.13543614)
\curveto(389.25354372,113.23960279)(389.27437705,113.57554025)(389.29521038,114.14324851)
\curveto(389.32125205,114.71616511)(389.33427288,115.14845672)(389.33427288,115.44012335)
\lineto(389.33427288,116.97918566)
\curveto(389.33427288,117.18751897)(389.32385621,117.4713731)(389.30302288,117.83074806)
\curveto(389.28218955,118.19012302)(389.26396039,118.41147715)(389.24833539,118.49481048)
\curveto(389.23791872,118.5781438)(389.19885623,118.63803963)(389.1311479,118.67449796)
\curveto(389.06343958,118.71616462)(388.92802293,118.73699795)(388.72489795,118.73699795)
\lineto(388.06864803,118.74481045)
\lineto(387.99833554,118.80731044)
\lineto(387.99833554,119.1432479)
\lineto(388.06083553,119.20574789)
\curveto(389.05562708,119.32553954)(389.89416864,119.52085202)(390.57646022,119.79168532)
\closepath
}
}
{
\newrgbcolor{curcolor}{0 0 0}
\pscustom[linestyle=none,fillstyle=solid,fillcolor=curcolor]
{
\newpath
\moveto(403.48270863,117.64324808)
\lineto(403.17802117,117.72137307)
\lineto(403.11552118,117.79949806)
\lineto(403.11552118,118.76824794)
\curveto(404.0426044,119.45574786)(404.94364595,119.79949782)(405.81864585,119.79949782)
\curveto(406.4176041,119.79949782)(406.91499988,119.68751866)(407.31083316,119.46356036)
\curveto(407.70666645,119.23960205)(407.99312474,118.95835209)(408.17020806,118.61981046)
\curveto(408.34729137,118.28647717)(408.43583302,117.89585222)(408.43583302,117.44793561)
\lineto(408.39677053,115.8776233)
\lineto(408.39677053,113.54949859)
\curveto(408.39677053,113.23179029)(408.41760386,113.03908198)(408.45927052,112.97137366)
\curveto(408.50614551,112.90366533)(408.55822884,112.85679034)(408.6155205,112.83074868)
\curveto(408.67281216,112.80991535)(408.77958298,112.79168618)(408.93583296,112.77606118)
\lineto(409.38114541,112.73699869)
\lineto(409.4436454,112.6666862)
\lineto(409.4436454,112.35418623)
\lineto(409.38114541,112.29168624)
\curveto(409.00093712,112.32293624)(408.6467705,112.33856124)(408.31864554,112.33856124)
\curveto(408.00614558,112.33856124)(407.63114562,112.32293624)(407.19364568,112.29168624)
\lineto(407.07645819,112.40106123)
\lineto(407.10770819,113.65887357)
\lineto(405.4045834,112.33074874)
\curveto(405.1181251,112.21095709)(404.80562514,112.15106126)(404.46708351,112.15106126)
\curveto(404.0504169,112.15106126)(403.69104194,112.22658208)(403.38895864,112.37762373)
\curveto(403.09208368,112.52866538)(402.86291704,112.73960285)(402.70145873,113.01043615)
\curveto(402.54520875,113.28126945)(402.46708376,113.60939441)(402.46708376,113.99481103)
\curveto(402.46708376,114.76043594)(402.70666706,115.35679003)(403.18583367,115.78387331)
\curveto(403.66500028,116.21616493)(404.97229178,116.58335238)(407.10770819,116.88543568)
\curveto(407.10770819,117.65626891)(406.93583321,118.19793551)(406.59208325,118.51043548)
\curveto(406.24833329,118.82814377)(405.78739585,118.98699792)(405.20927092,118.98699792)
\curveto(404.90718762,118.98699792)(404.63114599,118.94272709)(404.38114602,118.85418543)
\curveto(404.13635439,118.76564378)(403.9957294,118.69272712)(403.95927107,118.63543546)
\curveto(403.92281275,118.58335213)(403.78218776,118.26564384)(403.53739613,117.68231058)
\closepath
\moveto(407.10770819,116.40887323)
\curveto(405.64937503,116.1640816)(404.74052098,115.9010608)(404.38114602,115.61981083)
\curveto(404.02177107,115.33856087)(403.84208359,114.90366509)(403.84208359,114.31512349)
\curveto(403.84208359,113.50262359)(404.24572937,113.09637364)(405.05302094,113.09637364)
\curveto(405.74572919,113.09637364)(406.43062494,113.49741526)(407.10770819,114.2994985)
\closepath
\moveto(405.77958335,120.28387276)
\closepath
\moveto(405.61552087,111.87762379)
\closepath
}
}
{
\newrgbcolor{curcolor}{0 0 0}
\pscustom[linestyle=none,fillstyle=solid,fillcolor=curcolor]
{
\newpath
\moveto(412.49833252,119.79168532)
\lineto(412.64677,119.69012283)
\curveto(412.61552001,119.34116454)(412.59468668,118.88283126)(412.58427001,118.315123)
\curveto(413.00093663,118.65887296)(413.39676991,119.00783125)(413.77176986,119.36199787)
\curveto(413.88114485,119.46095619)(413.98270734,119.53387285)(414.07645733,119.58074784)
\curveto(414.17541565,119.62762284)(414.33687396,119.677102)(414.56083227,119.72918533)
\curveto(414.78479057,119.78126865)(415.01395721,119.80731032)(415.24833218,119.80731032)
\curveto(415.64416547,119.80731032)(416.02697792,119.72658116)(416.39676954,119.56512285)
\curveto(416.77176949,119.40366453)(417.05301946,119.21095622)(417.24051944,118.98699792)
\curveto(417.43322775,118.76824794)(417.56604023,118.50783131)(417.63895689,118.20574801)
\curveto(417.71187355,117.90366472)(417.74833187,117.53126893)(417.74833187,117.08856065)
\lineto(417.74833187,115.71356082)
\curveto(417.74833187,115.62501916)(417.76135271,114.97658174)(417.78739437,113.76824856)
\curveto(417.79781103,113.27345695)(417.85249853,112.97918616)(417.95145685,112.88543617)
\curveto(418.05041517,112.79168618)(418.3733318,112.74481119)(418.92020673,112.74481119)
\lineto(418.98270672,112.68231119)
\lineto(418.98270672,112.34637374)
\lineto(418.92020673,112.28387374)
\curveto(418.25874848,112.32554041)(417.81083187,112.34637374)(417.5764569,112.34637374)
\curveto(417.44104025,112.34637374)(417.06083196,112.32554041)(416.43583204,112.28387374)
\lineto(416.34208205,112.36981123)
\curveto(416.40979037,113.04689448)(416.44364453,113.92970687)(416.44364453,115.01824841)
\lineto(416.44364453,116.04168578)
\curveto(416.44364453,116.65626904)(416.42801954,117.09897732)(416.39676954,117.36981062)
\curveto(416.37072788,117.64064392)(416.28479039,117.88543555)(416.13895707,118.10418553)
\curveto(415.99312376,118.32814383)(415.79781128,118.50001881)(415.55301964,118.61981046)
\curveto(415.30822801,118.74481045)(415.01656138,118.80731044)(414.67801975,118.80731044)
\curveto(414.40718645,118.80731044)(414.17541565,118.77866461)(413.98270734,118.72137295)
\curveto(413.78999903,118.66928962)(413.57385322,118.55731047)(413.33426992,118.38543549)
\curveto(413.09468661,118.21876885)(412.9176033,118.0599147)(412.80301998,117.90887305)
\curveto(412.68843666,117.76303973)(412.61812417,117.62501892)(412.59208251,117.4948106)
\curveto(412.57124918,117.36981062)(412.56083251,117.09897732)(412.56083251,116.6823107)
\lineto(412.56083251,115.32293587)
\curveto(412.56083251,115.20835255)(412.57124918,114.83856093)(412.59208251,114.21356101)
\curveto(412.61291584,113.59376942)(412.63374917,113.23179029)(412.6545825,113.12762364)
\curveto(412.68062417,113.02345699)(412.71708249,112.94793616)(412.76395749,112.90106117)
\curveto(412.81083248,112.85418617)(412.86812414,112.82293618)(412.93583247,112.80731118)
\curveto(413.00354079,112.79689451)(413.28479076,112.77606118)(413.77958236,112.74481119)
\lineto(413.84989485,112.68231119)
\lineto(413.84989485,112.35418623)
\lineto(413.78739486,112.28387374)
\curveto(413.13635328,112.32554041)(412.50874919,112.34637374)(411.90458259,112.34637374)
\curveto(411.30562434,112.34637374)(410.68062441,112.32554041)(410.02958283,112.28387374)
\lineto(409.95927033,112.35418623)
\lineto(409.95927033,112.68231119)
\lineto(410.02958283,112.74481119)
\curveto(410.5347911,112.77606118)(410.81864523,112.79949868)(410.88114522,112.81512368)
\curveto(410.94885355,112.83074868)(411.00614521,112.86199867)(411.0530202,112.90887367)
\curveto(411.10510353,112.96095699)(411.13895769,113.03647782)(411.15458269,113.13543614)
\curveto(411.17541602,113.23960279)(411.19624935,113.57554025)(411.21708268,114.14324851)
\curveto(411.24312434,114.71616511)(411.25614517,115.14845672)(411.25614517,115.44012335)
\lineto(411.25614517,116.97918566)
\curveto(411.25614517,117.18751897)(411.24572851,117.4713731)(411.22489518,117.83074806)
\curveto(411.20406185,118.19012302)(411.18583268,118.41147715)(411.17020769,118.49481048)
\curveto(411.15979102,118.5781438)(411.12072852,118.63803963)(411.0530202,118.67449796)
\curveto(410.98531187,118.71616462)(410.84989522,118.73699795)(410.64677025,118.73699795)
\lineto(409.99052033,118.74481045)
\lineto(409.92020784,118.80731044)
\lineto(409.92020784,119.1432479)
\lineto(409.98270783,119.20574789)
\curveto(410.97749938,119.32553954)(411.81604094,119.52085202)(412.49833252,119.79168532)
\closepath
}
}
{
\newrgbcolor{curcolor}{0 0 0}
\pscustom[linestyle=none,fillstyle=solid,fillcolor=curcolor]
{
\newpath
\moveto(419.49833166,118.47918548)
\lineto(419.49833166,118.68231045)
\lineto(419.55301915,118.76043545)
\curveto(420.04260242,118.94272709)(420.43843571,119.1119979)(420.74051901,119.26824788)
\curveto(420.74051901,120.62762271)(420.72489401,121.41928928)(420.69364401,121.64324759)
\curveto(421.23010228,121.83074757)(421.67020639,122.03126838)(422.01395635,122.24481002)
\lineto(422.20145633,122.08856003)
\curveto(422.149373,121.76043508)(422.08687301,120.80731019)(422.01395635,119.22918539)
\curveto(422.27437298,119.22397705)(422.55562295,119.22137289)(422.85770624,119.22137289)
\curveto(423.4722895,119.22137289)(423.91239361,119.23699789)(424.17801858,119.26824788)
\lineto(424.23270607,119.21356039)
\lineto(424.08426859,118.55731047)
\lineto(424.0217686,118.48699798)
\curveto(423.75614363,118.49220631)(423.46187284,118.49481048)(423.13895621,118.49481048)
\curveto(422.84728958,118.49481048)(422.47228962,118.49220631)(422.01395635,118.48699798)
\lineto(421.96708135,115.30731087)
\curveto(421.96708135,114.57293596)(421.98270635,114.09116519)(422.01395635,113.86199855)
\curveto(422.05041468,113.63804024)(422.14416467,113.46095693)(422.29520631,113.33074861)
\curveto(422.45145629,113.20574863)(422.68062293,113.14324864)(422.98270623,113.14324864)
\curveto(423.33166452,113.14324864)(423.65458115,113.23439446)(423.95145611,113.4166861)
\lineto(424.13895609,113.13543614)
\curveto(424.0139561,113.04689448)(423.71447697,112.78647785)(423.2405187,112.35418623)
\curveto(422.9696854,112.22918625)(422.69364376,112.16668626)(422.4123938,112.16668626)
\curveto(421.24051894,112.16668626)(420.65458152,112.73439452)(420.65458152,113.86981105)
\curveto(420.65458152,114.28647766)(420.66499818,114.64064429)(420.68583151,114.93231092)
\curveto(420.69103984,115.02085257)(420.69364401,115.11199839)(420.69364401,115.20574838)
\lineto(420.69364401,118.44793548)
\lineto(420.37333155,118.44793548)
\curveto(420.13895658,118.44793548)(419.87072745,118.43751882)(419.56864415,118.41668549)
\closepath
}
}
{
\newrgbcolor{curcolor}{0 0 0}
\pscustom[linestyle=none,fillstyle=solid,fillcolor=curcolor]
{
\newpath
\moveto(427.70926815,112.19793625)
\lineto(426.65458078,115.69793582)
\curveto(426.45666413,116.35418574)(426.26395582,116.96356067)(426.07645585,117.5260606)
\curveto(425.8941642,118.08856053)(425.77176838,118.44533132)(425.70926839,118.59637297)
\curveto(425.65197673,118.75262295)(425.58947674,118.8646021)(425.52176842,118.93231042)
\curveto(425.45406009,119.00001875)(425.38114343,119.04689374)(425.30301844,119.07293541)
\curveto(425.22489345,119.09897707)(425.00353931,119.13283123)(424.63895602,119.17449789)
\lineto(424.58426853,119.23699789)
\lineto(424.58426853,119.55731035)
\lineto(424.65458102,119.62762284)
\curveto(425.02958098,119.59116451)(425.61031007,119.57293534)(426.39676831,119.57293534)
\curveto(427.24572654,119.57293534)(427.88374729,119.59116451)(428.31083057,119.62762284)
\lineto(428.38114306,119.55731035)
\lineto(428.38114306,119.23699789)
\lineto(428.32645557,119.17449789)
\curveto(427.75874731,119.16928956)(427.42020568,119.13803957)(427.3108307,119.08074791)
\curveto(427.20666404,119.02345625)(427.15458071,118.93491459)(427.15458071,118.81512294)
\curveto(427.15458071,118.69533129)(427.23010154,118.37501883)(427.38114319,117.85418556)
\lineto(428.1077056,115.33856087)
\curveto(428.22228892,114.94793592)(428.36812223,114.4922068)(428.54520554,113.97137354)
\curveto(428.68583053,114.30991516)(428.88374717,114.7630401)(429.13895547,115.33074837)
\lineto(430.28739283,117.90106055)
\curveto(430.56343446,118.51043548)(430.81864276,119.12501873)(431.05301773,119.74481032)
\lineto(431.43583019,119.74481032)
\lineto(432.1389551,117.85418556)
\lineto(433.03739249,115.58856084)
\lineto(433.7327049,113.86981105)
\lineto(434.31864233,115.40887336)
\curveto(434.65718396,116.28908158)(434.92020476,117.02345649)(435.10770473,117.61199809)
\curveto(435.29520471,118.20053968)(435.3889547,118.5781438)(435.3889547,118.74481045)
\curveto(435.3889547,118.91147709)(435.33166304,119.02085208)(435.21707972,119.07293541)
\curveto(435.1024964,119.12501873)(434.79520477,119.1588729)(434.29520483,119.17449789)
\lineto(434.22489234,119.23699789)
\lineto(434.22489234,119.56512285)
\lineto(434.29520483,119.62762284)
\curveto(434.89937143,119.59116451)(435.40718386,119.57293534)(435.81864215,119.57293534)
\curveto(436.33947542,119.57293534)(436.82905869,119.59116451)(437.28739196,119.62762284)
\lineto(437.34989196,119.56512285)
\lineto(437.34989196,119.24481039)
\lineto(437.28739196,119.17449789)
\curveto(437.006142,119.16408123)(436.81343369,119.13283123)(436.70926704,119.08074791)
\curveto(436.61030872,119.02866458)(436.4957254,118.90106043)(436.36551708,118.69793545)
\curveto(436.24051709,118.50001881)(436.04780878,118.09376886)(435.78739215,117.4791856)
\lineto(435.01395475,115.64324833)
\lineto(434.66239229,114.76824844)
\lineto(434.10770486,113.31512362)
\curveto(433.93582988,112.86199867)(433.81082989,112.48960288)(433.7327049,112.19793625)
\lineto(432.99832999,112.19793625)
\lineto(432.62333004,113.17449863)
\lineto(430.84208026,117.65106058)
\lineto(430.09208035,116.03387328)
\lineto(429.22489296,114.09637352)
\lineto(428.45926805,112.19793625)
\closepath
}
}
{
\newrgbcolor{curcolor}{0 0 0}
\pscustom[linestyle=none,fillstyle=solid,fillcolor=curcolor]
{
\newpath
\moveto(438.23270435,115.92449829)
\curveto(438.23270435,116.54428988)(438.3603085,117.14064398)(438.6155168,117.71356057)
\curveto(438.8707251,118.2916855)(439.31343338,118.78908127)(439.94364164,119.20574789)
\curveto(440.57905823,119.62762284)(441.3316623,119.83856031)(442.20145386,119.83856031)
\curveto(443.29520372,119.83856031)(444.18582861,119.49220619)(444.87332853,118.79949794)
\curveto(445.56082844,118.11199802)(445.9045784,117.22658147)(445.9045784,116.14324827)
\curveto(445.9045784,114.95054008)(445.50874512,113.96876937)(444.71707855,113.19793613)
\curveto(443.93062031,112.42710289)(442.9800996,112.04168627)(441.8655164,112.04168627)
\curveto(441.13634982,112.04168627)(440.48530824,112.23699875)(439.91239164,112.6276237)
\curveto(439.33947505,113.02345699)(438.91499593,113.50522776)(438.6389543,114.07293602)
\curveto(438.368121,114.64064429)(438.23270435,115.25783171)(438.23270435,115.92449829)
\closepath
\moveto(439.70926667,116.46356073)
\curveto(439.70926667,115.73439415)(439.80301665,115.0807484)(439.99051663,114.50262347)
\curveto(440.17801661,113.92970687)(440.47489157,113.46876943)(440.88114152,113.11981114)
\curveto(441.28739147,112.77085285)(441.75093308,112.59637371)(442.27176635,112.59637371)
\curveto(442.88634961,112.59637371)(443.39416205,112.83856118)(443.79520366,113.32293612)
\curveto(444.19624528,113.81251939)(444.39676609,114.5468943)(444.39676609,115.52606084)
\curveto(444.39676609,116.63543571)(444.17801612,117.54428976)(443.74051617,118.25262301)
\curveto(443.30822456,118.96095625)(442.68322463,119.31512288)(441.8655164,119.31512288)
\curveto(441.18843315,119.31512288)(440.65978738,119.07033124)(440.2795791,118.58074797)
\curveto(439.89937081,118.09116469)(439.70926667,117.38543561)(439.70926667,116.46356073)
\closepath
\moveto(442.07645387,120.28387276)
\closepath
\moveto(441.95926639,111.87762379)
\closepath
}
}
{
\newrgbcolor{curcolor}{0 0 0}
\pscustom[linestyle=none,fillstyle=solid,fillcolor=curcolor]
{
\newpath
\moveto(449.43582797,119.79168532)
\lineto(449.58426545,119.69012283)
\curveto(449.55301545,119.38803953)(449.52957796,118.85158127)(449.51395296,118.08074803)
\lineto(450.09989039,118.82293544)
\curveto(450.29259869,119.06772707)(450.46186951,119.25522705)(450.60770282,119.38543537)
\curveto(450.75874447,119.52085202)(450.93322362,119.62501867)(451.13114026,119.69793533)
\curveto(451.3290569,119.77085199)(451.53218188,119.80731032)(451.74051518,119.80731032)
\curveto(451.96968182,119.80731032)(452.18582763,119.76043532)(452.3889526,119.66668533)
\lineto(452.4436401,119.55731035)
\curveto(452.36030677,118.8646021)(452.31343178,118.25522717)(452.30301511,117.72918557)
\lineto(451.95145266,117.72918557)
\curveto(451.74311935,118.20835218)(451.40718189,118.44793548)(450.94364028,118.44793548)
\curveto(450.62072365,118.44793548)(450.33947369,118.34376883)(450.09989039,118.13543552)
\curveto(449.86030708,117.93231055)(449.69884877,117.67449808)(449.61551544,117.36199812)
\curveto(449.53739045,117.05470649)(449.49832796,116.66408154)(449.49832796,116.19012326)
\lineto(449.49832796,115.32293587)
\curveto(449.49832796,115.16668589)(449.50874462,114.78387344)(449.52957796,114.17449851)
\curveto(449.55041129,113.56512359)(449.57124462,113.21356113)(449.59207795,113.11981114)
\curveto(449.61811961,113.02606115)(449.65457794,112.95574866)(449.70145293,112.90887367)
\curveto(449.75353626,112.867207)(449.81864042,112.83856118)(449.89676541,112.82293618)
\curveto(449.98009873,112.80731118)(450.35249452,112.78126952)(451.01395277,112.74481119)
\lineto(451.08426526,112.68231119)
\lineto(451.08426526,112.35418623)
\lineto(451.01395277,112.28387374)
\curveto(450.32124452,112.32554041)(449.59728628,112.34637374)(448.84207804,112.34637374)
\curveto(448.24311978,112.34637374)(447.61811986,112.32554041)(446.96707827,112.28387374)
\lineto(446.89676578,112.35418623)
\lineto(446.89676578,112.68231119)
\lineto(446.96707827,112.74481119)
\curveto(447.47228654,112.77606118)(447.75614067,112.79949868)(447.81864067,112.81512368)
\curveto(447.88634899,112.83074868)(447.94364065,112.86199867)(447.99051565,112.90887367)
\curveto(448.04259897,112.96095699)(448.07645313,113.03647782)(448.09207813,113.13543614)
\curveto(448.11291146,113.23960279)(448.13374479,113.57554025)(448.15457812,114.14324851)
\curveto(448.18061979,114.71616511)(448.19364062,115.14845672)(448.19364062,115.44012335)
\lineto(448.19364062,116.97918566)
\curveto(448.19364062,117.18751897)(448.18322395,117.4713731)(448.16239062,117.83074806)
\curveto(448.14155729,118.19012302)(448.12332813,118.41147715)(448.10770313,118.49481048)
\curveto(448.09728647,118.5781438)(448.05822397,118.63803963)(447.99051565,118.67449796)
\curveto(447.92280732,118.71616462)(447.78739067,118.73699795)(447.5842657,118.73699795)
\lineto(446.92801578,118.74481045)
\lineto(446.85770328,118.80731044)
\lineto(446.85770328,119.1432479)
\lineto(446.92020328,119.20574789)
\curveto(447.91499482,119.32553954)(448.75353638,119.52085202)(449.43582797,119.79168532)
\closepath
}
}
{
\newrgbcolor{curcolor}{0 0 0}
\pscustom[linestyle=none,fillstyle=solid,fillcolor=curcolor]
{
\newpath
\moveto(453.11551501,118.47918548)
\lineto(453.11551501,118.68231045)
\lineto(453.17020251,118.76043545)
\curveto(453.65978578,118.94272709)(454.05561906,119.1119979)(454.35770236,119.26824788)
\curveto(454.35770236,120.62762271)(454.34207736,121.41928928)(454.31082737,121.64324759)
\curveto(454.84728563,121.83074757)(455.28738975,122.03126838)(455.6311397,122.24481002)
\lineto(455.81863968,122.08856003)
\curveto(455.76655635,121.76043508)(455.70405636,120.80731019)(455.6311397,119.22918539)
\curveto(455.89155634,119.22397705)(456.1728063,119.22137289)(456.4748896,119.22137289)
\curveto(457.08947286,119.22137289)(457.52957697,119.23699789)(457.79520194,119.26824788)
\lineto(457.84988943,119.21356039)
\lineto(457.70145195,118.55731047)
\lineto(457.63895196,118.48699798)
\curveto(457.37332699,118.49220631)(457.07905619,118.49481048)(456.75613956,118.49481048)
\curveto(456.46447293,118.49481048)(456.08947298,118.49220631)(455.6311397,118.48699798)
\lineto(455.58426471,115.30731087)
\curveto(455.58426471,114.57293596)(455.59988971,114.09116519)(455.6311397,113.86199855)
\curveto(455.66759803,113.63804024)(455.76134802,113.46095693)(455.91238967,113.33074861)
\curveto(456.06863965,113.20574863)(456.29780629,113.14324864)(456.59988958,113.14324864)
\curveto(456.94884787,113.14324864)(457.2717645,113.23439446)(457.56863946,113.4166861)
\lineto(457.75613944,113.13543614)
\curveto(457.63113946,113.04689448)(457.33166033,112.78647785)(456.85770205,112.35418623)
\curveto(456.58686875,112.22918625)(456.31082712,112.16668626)(456.02957715,112.16668626)
\curveto(454.8577023,112.16668626)(454.27176487,112.73439452)(454.27176487,113.86981105)
\curveto(454.27176487,114.28647766)(454.28218154,114.64064429)(454.30301487,114.93231092)
\curveto(454.3082232,115.02085257)(454.31082737,115.11199839)(454.31082737,115.20574838)
\lineto(454.31082737,118.44793548)
\lineto(453.99051491,118.44793548)
\curveto(453.75613993,118.44793548)(453.4879108,118.43751882)(453.1858275,118.41668549)
\closepath
}
}
\end{pspicture}

    \caption{Ablaufschritte der statistischen Versuchsplanung (\ac{DoE} Steps)}
    \label{fig:ma_abb2.02_doe_steps}
\end{figure}
Das erklärte Ziel ist es, den kausalen Zusammenhang zwischen Einflussfaktoren und Systemantwort funktional abzubilden.
Darin abgebildete Einflussfaktoren sollen also per se statistisch signifikant und somit relevant für das Systemverhalten sein.
So kann beispielsweise die zufallsverteilte Lebensdauer $\sym{tau}$ in Abhängigkeit von $\sym{k} \geq 2$ technischen Beanspruchungen zunächst empirisch untersucht und anschließend modelliert sowie optimiert werden (\textit{Schritt~1} in Abbildung~\ref{fig:ma_abb2.02_doe_steps}).

Im Zentrum der Betrachtung steht damit generell ein technisches \textbf{System}, welches abstrakt als Produkt oder Prozess verstanden wird und den Zustand der Ausgangsgröße in Abhängigkeit der Eingangsgrößen definiert.
Die zu untersuchende oder zu optimierende Ausgangsgröße wird als \textbf{Systemantwort} $\sym{y}$ (engl. Response) bezeichnet.
Die gezielt kontrollierbaren und variierten Eingangsgrößen sind \textbf{Faktoren} (\textbf{Steuergrößen}), während nicht kontrollierbare oder unbekannte Einflüsse als \textbf{Störgrößen} (engl. Noise) klassifiziert werden (\textit{Schritt~2}, vgl. \cite{Kleppmann.2016}).
Eine visuelle Aufstellung des genannten Zusammenspiels der Parameter kann dem Parameterdiagramm, kurz \textbf{P-Diagramm}, in Abbildung~\ref{fig:ma_abb2.03_p_diagramm} entnommen werden \cite{Montgomery.2020}.
\begin{figure}[htbp]
    \centering
    \def\svgwidth{0.8\textwidth}
    \input{plots/ma_abb2.03_p_diagramm.pdf_tex}
    \caption{Parameter-Diagramm (P-Diagramm)}
    \label{fig:ma_abb2.03_p_diagramm}
\end{figure}
Um das Systemverhalten zu charakterisieren, werden die Faktoren als kategoriale oder kontinuierliche Parameter im Versuch auf diskreten Werten, den sogenannten \textbf{Faktorstufen} (engl. Level), variiert (\textit{Schritt~3}).
Dies erfolgt in aller Regel in kodierter Darstellung, so entsprechen gemäß der gängigsten Konvention die Stufe \textit{-1} der niedrigen und \textit{+1} der hohen Einstellstufe.
Die planerische Kombination verschiedener Faktorstufen äußert sich in spezifischen \textbf{Versuchspunkten} innerhalb des Parameterraums und entspricht der Versuchsplan-Matrix.
Der \textbf{Versuchsraum} (engl. Design Space) wird hierbei durch die Gesamtheit der technisch realisierbaren und im Versuch einstellbaren Parameterkombinationen aufgespannt.
Die Auswahl geeigneter statistischer Versuchspläne (\textit{Schritt~4}) für die Durchführung (\textit{Schritt~5}) wird in Abschnitt~\ref{subsec:pläne} detailliert behandelt.

Die aus der Variation resultierende Änderung der Systemantwort quantifiziert den Einfluss des Faktors, der statistisch als \textbf{Effekt} $\sym{Eff}$ bezeichnet wird und den Mittelwertunterschiede zweier Faktorstufen beschreibt (\textit{Schritte~6-7}).
So können mittels \ac{DoE} strukturiert, effizient und verbindlich Informationen gewonnen werden, die über die direkten Effekte hinausgehen und differenziert Aufschluss über \textbf{Haupteffekte} sowie etwaige \textbf{Wechselwirkungen} der Faktoren auf die Antwort des Systems geben - vergleiche Abb.~\ref{fig:ma_abb2.04_effekt} sowie \textcite{Montgomery.2020,Kleppmann.2016,Siebertz.2017,Kremer.2021}.
Abbildung~\ref{fig:ma_abb2.04_effekt} visualisiert derartige Effekte.
So gibt die Darstellung eines Haupteffekts in Abhängigkeit des Vorzeichens und der Steigung (\textbf{positiver} oder \textbf{negativer} Haupteffekt) sozusagen die Einflussstärke und -richtung wieder, während bei Wechselwirkung der Effekt in Abhängigkeit der Einstellung eines \textbf{Co-Faktors} dargestellt wird.
\begin{figure}[h]
    \centering
    \begin{subfigure}[b]{0.48\textwidth}
        \centering
        \setlength\figurewidth{0.9\linewidth}
        \setlength\figureheight{4cm}
        % This file was created by matlab2tikz.
%
\definecolor{mycolor1}{rgb}{0.12941,0.12941,0.12941}%
%
\begin{tikzpicture}

\begin{axis}[%
width=\figurewidth,
height=0.967\figureheight,
at={(0\figurewidth,0\figureheight)},
scale only axis,
xmin=-1.5,
xmax=1.5,
xtick={-1,1},
xlabel style={font=\color{mycolor1}},
xlabel={$x_1$},
ymin=0,
ymax=4,
ytick={\empty},
ylabel style={font=\color{mycolor1}},
ylabel={$y$},
axis background/.style={fill=white}
]
\addplot [color=black, line width=1.5pt, mark=*, mark options={solid, fill=black, black}, forget plot]
  table[row sep=crcr]{%
-1	1\\
1	3\\
};
\addplot [color=black, dotted, line width=1.0pt, forget plot]
  table[row sep=crcr]{%
0	1\\
1.2	1\\
};
\addplot [color=black, dotted, line width=1.0pt, forget plot]
  table[row sep=crcr]{%
1	3\\
1.2	3\\
};
\node[right, align=left, inner sep=0, font=\color{mycolor1}]
at (axis cs:1.1,2) {$+E_1$};
\end{axis}
\end{tikzpicture}%
        \caption{Positiver Haupteffekt $\sym{Eff}_1$}
        \label{fig:ma_abb2.04.1_effekt}
    \end{subfigure}
    \hfill
    \begin{subfigure}[b]{0.48\textwidth}
        \centering
        \setlength\figurewidth{0.9\linewidth}
        \setlength\figureheight{4cm}
        % This file was created by matlab2tikz.
%
\definecolor{mycolor1}{rgb}{0.12941,0.12941,0.12941}%
%
\begin{tikzpicture}

\begin{axis}[%
width=\figurewidth,
height=0.967\figureheight,
at={(0\figurewidth,0\figureheight)},
scale only axis,
xmin=-1.5,
xmax=1.5,
xtick={-1,1},
xlabel style={font=\color{mycolor1}},
xlabel={$x_1$},
ymin=0,
ymax=5,
ytick={\empty},
ylabel style={font=\color{mycolor1}},
ylabel={$y$},
axis background/.style={fill=white}
]
\addplot [color=black, line width=1.5pt, mark=*, mark options={solid, fill=black, black}, forget plot]
  table[row sep=crcr]{%
-1	2\\
1	4\\
};
\node[right, align=left, inner sep=0, rotate=28, font=\color{mycolor1}]
at (axis cs:-0.8,2.8) {$x_2 = +1$};
\addplot [color=black, dashed, line width=1.5pt, mark=*, mark options={solid, fill=black, black}, forget plot]
  table[row sep=crcr]{%
-1	1\\
1	1.5\\
};
\node[right, align=left, inner sep=0, rotate=9, font=\color{mycolor1}]
at (axis cs:-0.8,1.4) {$x_2 = -1$};
\addplot [color=black, dotted, line width=1.0pt, forget plot]
  table[row sep=crcr]{%
1	2\\
1.2	2\\
};
\addplot [color=black, dotted, line width=1.0pt, forget plot]
  table[row sep=crcr]{%
1	4\\
1.2	4\\
};
\node[right, align=left, inner sep=0, font=\color{mycolor1}]
at (axis cs:1.1,3) {$+E_2$};
\addplot [color=black, dotted, line width=1.0pt, forget plot]
  table[row sep=crcr]{%
1	1\\
1.2	1\\
};
\addplot [color=black, dotted, line width=1.0pt, forget plot]
  table[row sep=crcr]{%
1	1.5\\
1.2	1.5\\
};
\node[right, align=left, inner sep=0, font=\color{mycolor1}]
at (axis cs:1.1,1.25) {$+E_3$};
\end{axis}
\end{tikzpicture}%
        \caption{Wechselwirkungseffekte $\sym{Eff}_2$ und $\sym{Eff}_3$}
        \label{fig:ma_abb2.04.2_effekt}
    \end{subfigure}
    \caption{Schematische Darstellung der Effekte: (a) positiver Haupteffekt von Faktor $\sym{x}_1$, (b) Wechselwirkungseffekte zwischen $\sym{x}_1$ und $\sym{x}_2$.}
    \label{fig:ma_abb2.04_effekt}
\end{figure}
Diese Zusammenhänge können mathematisch positiv oder negativ beschrieben sowie durch Polynomfunktionen höherer Ordnung approximiert werden, um zusätzlich beispielsweise \textbf{quadratische Effekte} oder \textbf{Mehrfachwechselwirkungen} abzubilden.

Sollen zunächst perspektivisch relevante Faktoren für eine versuchstechnische Untersuchung identifiziert werden, kann ein \textbf{Parameter-Screening} durchgeführt werden.
Dessen Durchführung kann sowohl heuristisch als auch versuchstechnisch erfolgen.

\subsubsection{Parameter-Screening} \label{subsubsec:screening}
Angesichts der potenziell hohen Komplexität durch Wechselwirkungen und Nichtlinearitäten sind die in Abbildung~\ref{fig:ma_abb2.02_doe_steps} beschriebenen \textit{Schritte~2-3} als propädeutische Arbeiten für ein effizientes Testdesign zu interpretieren.
Methodisch lassen sich diese unter dem Terminus \textbf{Screening} subsumieren.
Screening-Schritte sind zwischen der Definition des Untersuchungsziels und der Durchführung der physischen Screening-Experimente angeordnet (vgl. \textit{Schritt~3} in Abbildung~\ref{fig:ma_abb2.02_doe_steps} sowie Abschnitt~\ref{subsec:pläne}).
Daraus folgend dienen Screening-Methoden und -Versuchspläne dem Ziel, Informationsverluste bei einer minimalen Anzahl an Versuchsläufen zu begrenzen und die vitalen (\textit{Steuergrößen}) von den trivialen (\textit{Störgrößen}) Faktoren zu separieren.

Im Hinblick auf die Realisierung eines unter Zeit- und Kostenrestriktionen hochgradig effizienten \ac{DoE} ist die effiziente Ausgestaltung der Screening-Strategie selbst schon von primärem Interesse.
In traditionellen \ac{DoE}-Ansätzen impliziert dies den Einsatz von \textbf{Kreativmethoden}, wie sie Standardliteratur von \textcite{Montgomery.2020} aufführen oder exemplarisch durch \textcite{Kremer.2021} und \textcite{Gundlach.2004} zusammengefasst werden.
Hierbei ist ein Rückgriff auf Ergebnisse aus Experimenten, die explizit für das Forschungsziel ausgelegt wären, in dieser Phase unter Umständen noch nicht möglich.
Es gilt damit zunächst, qualitativ eine rein rational erlesene Sammlung an potenziellen Einflussparameter zu erstellen, um diese dann anhand ihrer geschätzten Einflüsse auf die Systemantwort zu priorisieren.
Ansätze aus der Kreativmethodik können dazu genutzt werden und fundieren auf der technischen \textbf{Systemanalyse}, die sowohl mit als auch ohne spezifisches Vorwissen über das System erfolgen kann \cite{Bertsche.2022}.
Hilfsmittel zur Priorisierung einer hier erstellten Parametersammlung können beispielsweise Entscheidungsfindungsprotokolle, wie die \ac{DSM}, und Methoden aus dem Komplexitätsmanagement, z.B. \textbf{Ishikawa-Diagramm}, sein - siehe hierzu auch weiterführende Werke von \textcite{Mayers.1997}, \textcite{Pahl.2007},\textcite{Wu.2021, Daenzer.2002} sowie \textcite{Lindemann.2008}.
Das Screening liefert somit eine rational festgestellte Auswahl an möglichst wenigen Einflussparametern, die mutmaßlich den entscheidenden Anteil an statistisch begründeter Manipulation der Systemantwort tragen und sich daher für eine Untersuchung in Versuchsplänen qualifizieren.
Entsprechend ist daraufhin ein geeigneter Versuchsplan für die physischen Datenerhebungen zu wählen.

%%%%%%%%%%%%%%%%%%%%%%%%%%%%%%%%%%%%%%%%%%%%%%%%%%%%%%%%%%%%%%%%%%%%%%%%%%%%%%%%%%%%%%%%%%%%%%%%%%%%%%%%%%%%%%%%%%%%%%%%%%
%%%%%%%%%%%%%%%%%%%%%%%%%%%%%%%%%%%%%%%%%%%%%%%%%%%%%%%%%%%%%%%%%%%%%%%%%%%%%%%%%%%%%%%%%%%%%%%%%%%%%%%%%%%%%%%%%%%%%%%%%%
%%%%%%%%%%%%%%%%%%%%%%%%%%%%%%%%%%%%%%%%%%%%%%%%%%%%%%%%%%%%%%%%%%%%%%%%%%%%%%%%%%%%%%%%%%%%%%%%%%%%%%%%%%%%%%%%%%%%%%%%%%
%%%%%%%%%%%%%%%%%%%%%%%%%%%%%%%%%%%%%%%%%%%%%%%%%%%%%%%%%%%%%%%%%%%%%%%%%%%%%%%%%%%%%%%%%%%%%%%%%%%%%%%%%%%%%%%%%%%%%%%%%%
\subsection{Statistische Versuchsplanung zur Lebensdauererprobung} \label{subsec:pläne}
Standardprotokolle aus dem \ac{DoE} wie der \textbf{$2^{\sym{k}}$ voll-faktorielle Versuchsplan} eignen sich grundsätzlich auch für Lebensdaueruntersuchungen, da sich hier analog zu vergleichbar statistisch verteilter Datenlage Effekte stets als (Mittelwert-) Unterschiede in der Beobachtung der Systemantwort aus dem Vergleich zweier Einstellstufen eines oder mehrerer Faktoren ergeben \cite{Kleppmann.2016}.

\subsubsection{$2^{\sym{k}}$ Faktorielle Versuchspläne}
Demzufolge kann auch ein Lebensdauer-beeinflussendes Parameterset -  beispielhaft $(\sym{x}_1,\sym{x}_2)$ - voll-faktoriell auf zwei Stufen variiert und vollständig kombiniert werden, vgl. Abbildung~\ref{fig:abb2.05_vfdesign}.
\begin{figure}[h]
    \centering
    \begin{subfigure}[b]{0.4\textwidth}
        \centering
        \setlength\figurewidth{0.9\linewidth}
        \setlength\figureheight{\figurewidth}
        % This file was created by matlab2tikz.
%
\definecolor{mycolor1}{rgb}{0.12941,0.12941,0.12941}%
%
\begin{tikzpicture}

\begin{axis}[%
width=\figurewidth,
height=\figureheight,
at={(0\figurewidth,0\figureheight)},
scale only axis,
xmin=-1.8,
xmax=1.8,
xtick={-1,  0,  1},
xlabel style={font=\color{mycolor1}},
xlabel={$x_1$},
ymin=-1.8,
ymax=1.8,
ytick={-1,  0,  1},
ylabel style={font=\color{mycolor1}},
ylabel={$x_2$},
axis background/.style={fill=white},
xmajorgrids,
ymajorgrids,
grid style={dotted, opacity=0.25}
]
\addplot [color=black, dashed, line width=1.0pt, forget plot]
  table[row sep=crcr]{%
-1	-1\\
1	-1\\
1	1\\
-1	1\\
-1	-1\\
};
\addplot[only marks, mark=*, mark options={}, mark size=2.2361pt, color=black, fill=black, forget plot] table[row sep=crcr]{%
x	y\\
-1	-1\\
1	-1\\
-1	1\\
1	1\\
};
\end{axis}
\end{tikzpicture}%
        \caption{Versuchsplan}
        \label{fig:abb2.05.1_vfdesign_plot}
    \end{subfigure}
    \hfill
    \begin{subfigure}[b]{0.55\textwidth}
        \centering
        \vspace*{1cm}
        \begin{tabular}{cccc}
            \toprule
            \multicolumn{1}{c}{Faktorstufen-} & \multicolumn{3}{c}{Faktoren und}                                         \\
            \multicolumn{1}{c}{Kombination}   & \multicolumn{3}{c}{Wechselwirkung}                                       \\
            \cmidrule(r){1-1} \cmidrule(l){2-4}
            \#                                & $\sym{x}_1$                        & $\sym{x}_2$ & $\sym{x}_1 \sym{x}_2$ \\
            \midrule
            1                                 & $-1$                               & $-1$        & $+1$                  \\
            2                                 & $+1$                               & $-1$        & $-1$                  \\
            3                                 & $-1$                               & $+1$        & $-1$                  \\
            4                                 & $+1$                               & $+1$        & $+1$                  \\
            \bottomrule
        \end{tabular}
        \vspace*{1.5cm}
        \caption{Versuchsplanmatrix zu Abb.~\ref{fig:abb2.05.1_vfdesign_plot}}
        \label{fig:abb2.05.2_vfdesign_matrix}
    \end{subfigure}
    \caption{Standard voll-faktorieller Versuchsplan}
    \label{fig:abb2.05_vfdesign}
\end{figure}
Ein derartiges Setup erlaubt es, die perspektivische Differenz erreichbarer \ac{EoL}-Werte durch tiefe und hohe Beanspruchungswerte der Faktoren zu beobachten \cite{Yang.2007,Meeker.2022}.
Der voll-faktorielle Versuchsplan bildet somit den Standard-Versuchsplan im \ac{DoE} und fordert bei einmaliger Durchführung (\textbf{Replikation} $\sym{r}=1$)
\begin{equation}
    \sym{n}=2^{\sym{k}}
    \label{eq:ffvp_n}
\end{equation}
Versuche.
Dieser Stichprobenumfang stellt sicher, dass das resultierende Gleichungssystem \textbf{gesättigt} ist: Mit $\sym{n}$ Versuchen lassen sich $\sym{n}-1$ Effekte für Hauptfaktoren und Wechselwirkungen eindeutig bestimmen.
Von entscheidender Bedeutung für die Aussagekraft des Versuchsplans ist die Wahl der Faktorstufen (vgl. Abbildung~\ref{fig:ma_abb2.06_effektabstand}).
Die Differenz der gewählten Level muss bereits im Vorfeld definiert werden, sodass signifikante Effekte sicher detektiert werden („\textbf{Signal-to-Noise}“), wobei gleichzeitig zu geringe Abstände (Rauschen) sowie zu große Intervalle (Gefahr unerkannter Nichtlinearitäten) zu vermeiden sind - vgl. Abbildung~\ref{fig:ma_abb2.06_effektabstand} sowie \textcite{Wu.2021,Siebertz.2017,Kleppmann.2016}.
\begin{figure}[htbp]
    \centering
    \setlength\figurewidth{0.9\textwidth}
    \setlength\figureheight{4cm}
    % This file was created by matlab2tikz.
%
\definecolor{mycolor1}{rgb}{0.12941,0.12941,0.12941}%
%
\begin{tikzpicture}

\begin{axis}[%
width=0.275\figurewidth,
height=0.93\figureheight,
at={(0\figurewidth,0\figureheight)},
scale only axis,
xmin=-1,
xmax=2,
xtick={-0.5,-0.3},
xticklabels={{$-$},{$+$}},
xlabel style={font=\color{mycolor1}},
xlabel={$x_1$},
ymin=0,
ymax=1.4,
ytick={\empty},
ylabel style={font=\color{mycolor1}},
ylabel={$y$},
axis background/.style={fill=white}
]
\addplot [color=black, line width=1.0pt, forget plot]
  table[row sep=crcr]{%
-1	-1.05\\
-0.984924623115578	-1.00500113633494\\
-0.969849246231156	-0.960456806646297\\
-0.954773869346734	-0.916367010934067\\
-0.939698492462312	-0.872731749198252\\
-0.924623115577889	-0.829551021438853\\
-0.909547738693467	-0.786824827655867\\
-0.894472361809045	-0.744553167849297\\
-0.879396984924623	-0.702736042019141\\
-0.864321608040201	-0.6613734501654\\
-0.849246231155779	-0.620465392288074\\
-0.834170854271357	-0.580011868387162\\
-0.819095477386935	-0.540012878462665\\
-0.804020100502513	-0.500468422514583\\
-0.78894472361809	-0.461378500542916\\
-0.773869346733668	-0.422743112547663\\
-0.758793969849246	-0.384562258528825\\
-0.743718592964824	-0.346835938486402\\
-0.728643216080402	-0.309564152420393\\
-0.71356783919598	-0.2727469003308\\
-0.698492462311558	-0.236384182217621\\
-0.683417085427136	-0.200475998080857\\
-0.668341708542714	-0.165022347920507\\
-0.653266331658291	-0.130023231736572\\
-0.638190954773869	-0.0954786495290523\\
-0.623115577889447	-0.0613886012979472\\
-0.608040201005025	-0.0277530870432567\\
-0.592964824120603	0.00542789323501891\\
-0.577889447236181	0.0381543395368804\\
-0.562814070351759	0.0704262518623266\\
-0.547738693467337	0.102243630211358\\
-0.532663316582915	0.133606474583975\\
-0.517587939698492	0.164514784980177\\
-0.50251256281407	0.194968561399965\\
-0.487437185929648	0.224967803843337\\
-0.472361809045226	0.254512512310295\\
-0.457286432160804	0.283602686800838\\
-0.442211055276382	0.312238327314967\\
-0.42713567839196	0.34041943385268\\
-0.412060301507538	0.368146006413979\\
-0.396984924623116	0.395418044998864\\
-0.381909547738693	0.422235549607333\\
-0.366834170854271	0.448598520239388\\
-0.351758793969849	0.474506956895028\\
-0.336683417085427	0.499960859574253\\
-0.321608040201005	0.524960228277064\\
-0.306532663316583	0.549505063003459\\
-0.291457286432161	0.57359536375344\\
-0.276381909547739	0.597231130527007\\
-0.261306532663317	0.620412363324158\\
-0.246231155778894	0.643139062144895\\
-0.231155778894472	0.665411226989217\\
-0.21608040201005	0.687228857857125\\
-0.201005025125628	0.708591954748617\\
-0.185929648241206	0.729500517663695\\
-0.170854271356784	0.749954546602358\\
-0.155778894472362	0.769954041564607\\
-0.14070351758794	0.78949900255044\\
-0.125628140703518	0.80858942955986\\
-0.110552763819096	0.827225322592864\\
-0.0954773869346733	0.845406681649453\\
-0.0804020100502513	0.863133506729628\\
-0.0653266331658291	0.880405797833388\\
-0.050251256281407	0.897223554960733\\
-0.035175879396985	0.913586778111664\\
-0.0201005025125628	0.92949546728618\\
-0.00502512562814073	0.944949622484281\\
0.0100502512562815	0.959949243705967\\
0.0251256281407035	0.974494330951239\\
0.0402010050251256	0.988584884220095\\
0.0552763819095476	1.00222090351254\\
0.0703517587939699	1.01540238882857\\
0.085427135678392	1.02812934016818\\
0.100502512562814	1.04040175753138\\
0.115577889447236	1.05221964091816\\
0.130653266331658	1.06358299032853\\
0.14572864321608	1.07449180576248\\
0.160804020100503	1.08494608722002\\
0.175879396984925	1.09494583470114\\
0.190954773869347	1.10449104820585\\
0.206030150753769	1.11358172773415\\
0.221105527638191	1.12221787328603\\
0.236180904522613	1.13039948486149\\
0.251256281407035	1.13812656246054\\
0.266331658291457	1.14539910608318\\
0.28140703517588	1.1522171157294\\
0.296482412060302	1.15858059139921\\
0.311557788944724	1.1644895330926\\
0.326633165829146	1.16994394080958\\
0.341708542713568	1.17494381455014\\
0.35678391959799	1.17948915431428\\
0.371859296482412	1.18357996010202\\
0.386934673366834	1.18721623191334\\
0.402010050251256	1.19039796974824\\
0.417085427135678	1.19312517360673\\
0.432160804020101	1.1953978434888\\
0.447236180904523	1.19721597939446\\
0.462311557788945	1.1985795813237\\
0.477386934673367	1.19948864927653\\
0.492462311557789	1.19994318325295\\
0.507537688442211	1.19994318325295\\
0.522613065326633	1.19948864927653\\
0.537688442211055	1.1985795813237\\
0.552763819095477	1.19721597939446\\
0.567839195979899	1.1953978434888\\
0.582914572864322	1.19312517360673\\
0.597989949748744	1.19039796974824\\
0.613065326633166	1.18721623191334\\
0.628140703517588	1.18357996010202\\
0.64321608040201	1.17948915431428\\
0.658291457286432	1.17494381455014\\
0.673366834170854	1.16994394080958\\
0.688442211055276	1.1644895330926\\
0.703517587939698	1.15858059139921\\
0.71859296482412	1.1522171157294\\
0.733668341708543	1.14539910608318\\
0.748743718592965	1.13812656246054\\
0.763819095477387	1.13039948486149\\
0.778894472361809	1.12221787328603\\
0.793969849246231	1.11358172773415\\
0.809045226130653	1.10449104820585\\
0.824120603015075	1.09494583470114\\
0.839195979899497	1.08494608722002\\
0.85427135678392	1.07449180576248\\
0.869346733668342	1.06358299032853\\
0.884422110552764	1.05221964091816\\
0.899497487437186	1.04040175753138\\
0.914572864321608	1.02812934016818\\
0.92964824120603	1.01540238882857\\
0.944723618090452	1.00222090351254\\
0.959798994974874	0.988584884220095\\
0.974874371859296	0.974494330951239\\
0.989949748743719	0.959949243705967\\
1.00502512562814	0.944949622484281\\
1.02010050251256	0.929495467286179\\
1.03517587939698	0.913586778111664\\
1.05025125628141	0.897223554960733\\
1.06532663316583	0.880405797833388\\
1.08040201005025	0.863133506729628\\
1.09547738693467	0.845406681649453\\
1.1105527638191	0.827225322592864\\
1.12562814070352	0.80858942955986\\
1.14070351758794	0.78949900255044\\
1.15577889447236	0.769954041564607\\
1.17085427135678	0.749954546602358\\
1.18592964824121	0.729500517663696\\
1.20100502512563	0.708591954748617\\
1.21608040201005	0.687228857857125\\
1.23115577889447	0.665411226989218\\
1.24623115577889	0.643139062144895\\
1.26130653266332	0.620412363324159\\
1.27638190954774	0.597231130527007\\
1.29145728643216	0.57359536375344\\
1.30653266331658	0.54950506300346\\
1.32160804020101	0.524960228277064\\
1.33668341708543	0.499960859574254\\
1.35175879396985	0.474506956895028\\
1.36683417085427	0.448598520239388\\
1.38190954773869	0.422235549607333\\
1.39698492462312	0.395418044998864\\
1.41206030150754	0.36814600641398\\
1.42713567839196	0.34041943385268\\
1.44221105527638	0.312238327314966\\
1.4572864321608	0.283602686800839\\
1.47236180904523	0.254512512310295\\
1.48743718592965	0.224967803843338\\
1.50251256281407	0.194968561399965\\
1.51758793969849	0.164514784980177\\
1.53266331658291	0.133606474583975\\
1.54773869346734	0.102243630211358\\
1.56281407035176	0.0704262518623262\\
1.57788944723618	0.0381543395368804\\
1.5929648241206	0.00542789323501891\\
1.60804020100502	-0.0277530870432563\\
1.62311557788945	-0.0613886012979472\\
1.63819095477387	-0.0954786495290527\\
1.65326633165829	-0.130023231736572\\
1.66834170854271	-0.165022347920507\\
1.68341708542714	-0.200475998080856\\
1.69849246231156	-0.236384182217621\\
1.71356783919598	-0.2727469003308\\
1.7286432160804	-0.309564152420393\\
1.74371859296482	-0.346835938486402\\
1.75879396984925	-0.384562258528825\\
1.77386934673367	-0.422743112547663\\
1.78894472361809	-0.461378500542916\\
1.80402010050251	-0.500468422514583\\
1.81909547738693	-0.540012878462665\\
1.83417085427136	-0.580011868387162\\
1.84924623115578	-0.620465392288074\\
1.8643216080402	-0.661373450165401\\
1.87939698492462	-0.702736042019141\\
1.89447236180905	-0.744553167849297\\
1.90954773869347	-0.786824827655867\\
1.92462311557789	-0.829551021438853\\
1.93969849246231	-0.872731749198253\\
1.95477386934673	-0.916367010934067\\
1.96984924623116	-0.960456806646297\\
1.98492462311558	-1.00500113633494\\
2	-1.05\\
};
\addplot [color=black, line width=1.5pt, forget plot]
  table[row sep=crcr]{%
-0.5	0.2\\
-0.3	0.56\\
};
\addplot[only marks, mark=*, mark options={}, mark size=1.5811pt, draw=black, fill=gray, forget plot] table[row sep=crcr]{%
x	y\\
-0.5	0.2\\
-0.3	0.56\\
};
\addplot [color=black, line width=0.8pt, forget plot]
  table[row sep=crcr]{%
-1	4.94057116545912e-10\\
-0.984924623115578	1.57602276748064e-09\\
-0.969849246231156	4.85205611542638e-09\\
-0.954773869346734	1.44167425497681e-08\\
-0.939698492462312	4.13415229494018e-08\\
-0.924623115577889	1.14415219901712e-07\\
-0.909547738693467	3.05604066037433e-07\\
-0.894472361809045	7.87793615663737e-07\\
-0.879396984924623	1.95994455544759e-06\\
-0.864321608040201	4.70601291743604e-06\\
-0.849246231155779	1.09053709836089e-05\\
-0.834170854271357	2.43896615423924e-05\\
-0.819095477386935	5.26440270021226e-05\\
-0.804020100502513	0.000109665603513466\\
-0.78894472361809	0.000220480285668284\\
-0.773869346733668	0.000427806218451707\\
-0.758793969849246	0.000801128945257342\\
-0.743718592964824	0.00144789061734106\\
-0.728643216080402	0.00252549832279041\\
-0.71356783919598	0.00425144353314529\\
-0.698492462311558	0.00690722727455912\\
-0.683417085427136	0.0108305142315072\\
-0.668341708542714	0.016389753002101\\
-0.653266331658291	0.0239372215927844\\
-0.638190954773869	0.0337406210220738\\
-0.623115577889447	0.0458997588353218\\
-0.608040201005025	0.0602623045175218\\
-0.592964824120603	0.076358795283038\\
-0.577889447236181	0.0933792540392318\\
-0.562814070351759	0.11020967781249\\
-0.547738693467337	0.125535647068942\\
-0.532663316582915	0.138004230025995\\
-0.517587939698492	0.146418425165253\\
-0.50251256281407	0.149926038101147\\
-0.487437185929648	0.148161852006634\\
-0.472361809045226	0.141310271747869\\
-0.457286432160804	0.130073571571152\\
-0.442211055276382	0.115553312379778\\
-0.42713567839196	0.0990726394180448\\
-0.412060301507538	0.0819790851628189\\
-0.396984924623116	0.0654682002244067\\
-0.381909547738693	0.0504586627332136\\
-0.366834170854271	0.0375335006100462\\
-0.351758793969849	0.0269451376333582\\
-0.336683417085427	0.0186689435016723\\
-0.321608040201005	0.0124835204393518\\
-0.306532663316583	0.00805624030634114\\
-0.291457286432161	0.00501771219792656\\
-0.276381909547739	0.00301617880345267\\
-0.261306532663317	0.00174979197149739\\
-0.246231155778894	0.000979701460978718\\
-0.231155778894472	0.00052939409408853\\
-0.21608040201005	0.000276084756470587\\
-0.201005025125628	0.000138958067324504\\
-0.185929648241206	6.74998957371878e-05\\
-0.170854271356784	3.16446603768558e-05\\
-0.155778894472362	1.43177830072836e-05\\
-0.14070351758794	6.25214603522081e-06\\
-0.125628140703518	2.63487763442609e-06\\
-0.110552763819096	1.07169150642828e-06\\
-0.0954773869346733	4.20685096491226e-07\\
-0.0804020100502513	1.59375831900113e-07\\
-0.0653266331658291	5.82727887989975e-08\\
-0.050251256281407	2.0563031523068e-08\\
-0.035175879396985	7.00303795033874e-09\\
-0.0201005025125628	2.30178003387978e-09\\
-0.00502512562814073	7.30161882950325e-10\\
0.0100502512562815	2.23538584508942e-10\\
0.0251256281407035	6.60486333507968e-11\\
0.0402010050251256	1.88344600914967e-11\\
0.0552763819095476	5.18346911064005e-12\\
0.0703517587939699	1.37678414045859e-12\\
0.085427135678392	3.52930502547627e-13\\
0.100502512562814	8.73153370218821e-14\\
0.115577889447236	2.08482628999997e-14\\
0.130653266331658	4.80426788284014e-15\\
0.14572864321608	1.06847054926449e-15\\
0.160804020100503	2.293379345176e-16\\
0.175879396984925	4.75080535998551e-17\\
0.190954773869347	9.49809652749516e-18\\
0.206030150753769	1.83266852111727e-18\\
0.221105527638191	3.41278761247874e-19\\
0.236180904522613	6.13355979363771e-20\\
0.251256281407035	1.06388318608236e-20\\
0.266331658291457	1.78095639878592e-21\\
0.28140703517588	2.87733633100887e-22\\
0.296482412060302	4.48648243476834e-23\\
0.311557788944724	6.75148511159821e-24\\
0.326633165829146	9.80552061285523e-25\\
0.341708542713568	1.37442168882428e-25\\
0.35678391959799	1.85929089646744e-26\\
0.371859296482412	2.42746337503807e-27\\
0.386934673366834	3.05869398817556e-28\\
0.402010050251256	3.71960971241608e-29\\
0.417085427135678	4.36552738374366e-30\\
0.432160804020101	4.94486068784572e-31\\
0.447236180904523	5.40566844359041e-32\\
0.462311557788945	5.70325446290064e-33\\
0.477386934673367	5.80729774725765e-34\\
0.492462311557789	5.70694172659544e-35\\
0.507537688442211	5.41266044031043e-36\\
0.522613065326633	4.95445773244763e-37\\
0.537688442211055	4.37682792651257e-38\\
0.552763819095477	3.73164926603188e-39\\
0.567839195979899	3.0705782090626e-40\\
0.582914572864322	2.43847051792633e-41\\
0.597989949748744	1.86892922367551e-42\\
0.613065326633166	1.38243971220511e-43\\
0.628140703517588	9.86909994390187e-45\\
0.64321608040201	6.79965523954378e-46\\
0.658291457286432	4.52141363728256e-47\\
0.673366834170854	2.90161365610459e-48\\
0.688442211055276	1.79714423209735e-49\\
0.703517587939698	1.07424732364763e-50\\
0.71859296482412	6.19731580354211e-52\\
0.733668341708543	3.45049160324826e-53\\
0.748743718592965	1.85411352709972e-54\\
0.763819095477387	9.6154512444845e-56\\
0.778894472361809	4.81261386557408e-57\\
0.793969849246231	2.32471846939545e-58\\
0.809045226130653	1.08377147018656e-59\\
0.824120603015075	4.87621743082316e-61\\
0.839195979899497	2.11741707602194e-62\\
0.85427135678392	8.87376166071854e-64\\
0.869346733668342	3.58911280870636e-65\\
0.884422110552764	1.40102039622597e-66\\
0.899497487437186	5.27812665085848e-68\\
0.914572864321608	1.91908034074213e-69\\
0.92964824120603	6.73417667380368e-71\\
0.944723618090452	2.28062486945852e-72\\
0.959798994974874	7.45420282973179e-74\\
0.974874371859296	2.35140022753671e-75\\
0.989949748743719	7.15862983603782e-77\\
1.00502512562814	2.10334859928141e-78\\
1.02010050251256	5.96445281528211e-80\\
1.03517587939698	1.63232998342091e-81\\
1.05025125628141	4.31144955379258e-83\\
1.06532663316583	1.099047999615e-84\\
1.08040201005025	2.70388372325828e-86\\
1.09547738693467	6.42003451535516e-88\\
1.1105527638191	1.47117628166674e-89\\
1.12562814070352	3.25364425381868e-91\\
1.14070351758794	6.94469886085328e-93\\
1.15577889447236	1.43058863505929e-94\\
1.17085427135678	2.84416064620836e-96\\
1.18592964824121	5.45722016773012e-98\\
1.20100502512563	1.01057099266459e-99\\
1.21608040201005	1.80609329772476e-101\\
1.23115577889447	3.1152401680726e-103\\
1.24623115577889	5.18586118295871e-105\\
1.26130653266332	8.33159672542081e-107\\
1.27638190954774	1.29185447563516e-108\\
1.29145728643216	1.93320071853718e-110\\
1.30653266331658	2.79201856539966e-112\\
1.32160804020101	3.89168477964236e-114\\
1.33668341708543	5.23522088366185e-116\\
1.35175879396985	6.79689192639872e-118\\
1.36683417085427	8.51655036505505e-120\\
1.38190954773869	1.0299000372269e-121\\
1.39698492462312	1.20199989235265e-123\\
1.41206030150754	1.35391619733106e-125\\
1.42713567839196	1.47182826596156e-127\\
1.44221105527638	1.54418914276882e-129\\
1.4572864321608	1.56358627726146e-131\\
1.47236180904523	1.5279924417714e-133\\
1.48743718592965	1.44111474570082e-135\\
1.50251256281407	1.31175859381067e-137\\
1.51758793969849	1.15235761060435e-139\\
1.53266331658291	9.7700914965437e-142\\
1.54773869346734	7.99443889830478e-144\\
1.56281407035176	6.31328426274197e-146\\
1.57788944723618	4.81172392320207e-148\\
1.5929648241206	3.53935445921421e-150\\
1.60804020100502	2.51261192063529e-152\\
1.62311557788945	1.72149106676805e-154\\
1.63819095477387	1.13831414114531e-156\\
1.65326633165829	7.26436218364783e-159\\
1.66834170854271	4.47415316105865e-161\\
1.68341708542714	2.65951353658765e-163\\
1.69849246231156	1.52570849216465e-165\\
1.71356783919598	8.44731952777952e-168\\
1.7286432160804	4.51382046057725e-170\\
1.74371859296482	2.32781059049058e-172\\
1.75879396984925	1.15858792732176e-174\\
1.77386934673367	5.56529715339483e-177\\
1.78894472361809	2.58003571227894e-179\\
1.80402010050251	1.15435948830055e-181\\
1.81909547738693	4.98464732871307e-184\\
1.83417085427136	2.0773314522254e-186\\
1.84924623115578	8.3551674220519e-189\\
1.8643216080402	3.24326560997671e-191\\
1.87939698492462	1.21503245181336e-193\\
1.89447236180905	4.39310150611083e-196\\
1.90954773869347	1.53296626912309e-198\\
1.92462311557789	5.16264140113166e-201\\
1.93969849246231	1.67798981710585e-203\\
1.95477386934673	5.26362186473754e-206\\
1.96984924623116	1.59352192693446e-208\\
1.98492462311558	4.65596176719113e-211\\
2	1.31292152108604e-213\\
};
\addplot [color=black, line width=0.8pt, forget plot]
  table[row sep=crcr]{%
-1	3.55431262101029e-18\\
-0.984924623115578	1.81609884596062e-17\\
-0.969849246231156	8.95573720731097e-17\\
-0.954773869346734	4.26227238656083e-16\\
-0.939698492462312	1.95775823637214e-15\\
-0.924623115577889	8.67870485556227e-15\\
-0.909547738693467	3.71303274102413e-14\\
-0.894472361809045	1.53313584829357e-13\\
-0.879396984924623	6.10956793859674e-13\\
-0.864321608040201	2.34973214755473e-12\\
-0.849246231155779	8.72176174951484e-12\\
-0.834170854271357	3.1244104338487e-11\\
-0.819095477386935	1.08021434444808e-10\\
-0.804020100502513	3.60437360701495e-10\\
-0.78894472361809	1.16072051905236e-09\\
-0.773869346733668	3.6074766029745e-09\\
-0.758793969849246	1.08207505253954e-08\\
-0.743718592964824	3.13248677345787e-08\\
-0.728643216080402	8.75183472523818e-08\\
-0.71356783919598	2.35986396924585e-07\\
-0.698492462311558	6.14119420555289e-07\\
-0.683417085427136	1.5423988532712e-06\\
-0.668341708542714	3.7386822494762e-06\\
-0.653266331658291	8.74618015156444e-06\\
-0.638190954773869	1.97467797092196e-05\\
-0.623115577889447	4.30280975139258e-05\\
-0.608040201005025	9.04869619478003e-05\\
-0.592964824120603	0.000183652933809039\\
-0.577889447236181	0.000359739174620673\\
-0.562814070351759	0.000680073100952153\\
-0.547738693467337	0.0012407990754747\\
-0.532663316582915	0.00218486860960069\\
-0.517587939698492	0.00371301915092076\\
-0.50251256281407	0.00608985642084095\\
-0.487437185929648	0.00963973106645835\\
-0.472361809045226	0.0147265420746967\\
-0.457286432160804	0.0217127407700539\\
-0.442211055276382	0.0308963027649698\\
-0.42713567839196	0.0424303268657564\\
-0.412060301507538	0.0562372716460321\\
-0.396984924623116	0.0719366332331492\\
-0.381909547738693	0.0888083852921676\\
-0.366834170854271	0.105812223267767\\
-0.351758793969849	0.121673417265781\\
-0.336683417085427	0.13503102731758\\
-0.321608040201005	0.144627020114942\\
-0.306532663316583	0.149500727017824\\
-0.291457286432161	0.149147222911492\\
-0.276381909547739	0.143603504366961\\
-0.261306532663317	0.133442111451205\\
-0.246231155778894	0.119673714846927\\
-0.231155778894472	0.103581605858988\\
-0.21608040201005	0.0865255776358202\\
-0.201005025125628	0.069756450696946\\
-0.185929648241206	0.054275301279128\\
-0.170854271356784	0.0407566153750663\\
-0.155778894472362	0.0295373806056307\\
-0.14070351758794	0.0206596922333797\\
-0.125628140703518	0.0139461302227939\\
-0.110552763819096	0.00908576678094173\\
-0.0954773869346733	0.00571277939705933\\
-0.0804020100502513	0.00346666058666087\\
-0.0653266331658291	0.0020302671706642\\
-0.050251256281407	0.00114755381628874\\
-0.035175879396985	0.00062599509629525\\
-0.0201005025125628	0.000329569339017116\\
-0.00502512562814073	0.000167455983755197\\
0.0100502512562815	8.21168990776936e-05\\
0.0251256281407035	3.88635448584388e-05\\
0.0402010050251256	1.77513048275818e-05\\
0.0552763819095476	7.82521272454388e-06\\
0.0703517587939699	3.32920105260662e-06\\
0.085427135678392	1.36697916185204e-06\\
0.100502512562814	5.41703654419998e-07\\
0.115577889447236	2.07176092053804e-07\\
0.130653266331658	7.647077793046e-08\\
0.14572864321608	2.72413972833871e-08\\
0.160804020100503	9.36572176650138e-09\\
0.175879396984925	3.10764301220918e-09\\
0.190954773869347	9.95173942493791e-10\\
0.206030150753769	3.0757063434393e-10\\
0.221105527638191	9.17421134910817e-11\\
0.236180904522613	2.64101345384594e-11\\
0.251256281407035	7.33754007928847e-12\\
0.266331658291457	1.96747108986481e-12\\
0.28140703517588	5.09148223467565e-13\\
0.296482412060302	1.27162221441985e-13\\
0.311557788944724	3.06513762898085e-14\\
0.326633165829146	7.13049785325629e-15\\
0.341708542713568	1.60091298568325e-15\\
0.35678391959799	3.46891454600484e-16\\
0.371859296482412	7.25433280534619e-17\\
0.386934673366834	1.46412902485596e-17\\
0.402010050251256	2.85193238452744e-18\\
0.417085427135678	5.36138629123624e-19\\
0.432160804020101	9.7273148793476e-20\\
0.447236180904523	1.70328314803765e-20\\
0.462311557788945	2.87845023160171e-21\\
0.477386934673367	4.69470816261789e-22\\
0.492462311557789	7.38986451246489e-23\\
0.507537688442211	1.12264475537458e-23\\
0.522613065326633	1.64598634420917e-24\\
0.537688442211055	2.32909983145351e-25\\
0.552763819095477	3.18073860713411e-26\\
0.567839195979899	4.1922372907246e-27\\
0.582914572864322	5.33263346118372e-28\\
0.597989949748744	6.54659704763229e-29\\
0.613065326633166	7.75653019221057e-30\\
0.628140703517588	8.86946377929167e-31\\
0.64321608040201	9.78825425536666e-32\\
0.658291457286432	1.04253614113891e-32\\
0.673366834170854	1.07165500623314e-33\\
0.688442211055276	1.06315570742002e-34\\
0.703517587939698	1.01792727909847e-35\\
0.71859296482412	9.4062091514638e-37\\
0.733668341708543	8.38861977051669e-38\\
0.748743718592965	7.22011954315451e-39\\
0.763819095477387	5.99758300162638e-40\\
0.778894472361809	4.80824009771488e-41\\
0.793969849246231	3.72026627236228e-42\\
0.809045226130653	2.77804901623269e-43\\
0.824120603015075	2.00209086303519e-44\\
0.839195979899497	1.39253353689166e-45\\
0.85427135678392	9.34771672587968e-47\\
0.869346733668342	6.05596592641516e-48\\
0.884422110552764	3.78651169987244e-49\\
0.899497487437186	2.2849314920672e-50\\
0.914572864321608	1.33071503523943e-51\\
0.92964824120603	7.47954095369583e-53\\
0.944723618090452	4.05735329155278e-54\\
0.959798994974874	2.12416723811486e-55\\
0.974874371859296	1.07327888993012e-56\\
0.989949748743719	5.23376801450119e-58\\
1.00502512562814	2.46316979445995e-59\\
1.02010050251256	1.1187994183065e-60\\
1.03517587939698	4.90442518278308e-62\\
1.05025125628141	2.07492305251633e-63\\
1.06532663316583	8.4721547122708e-65\\
1.08040201005025	3.33859516655462e-66\\
1.09547738693467	1.26973076603021e-67\\
1.1105527638191	4.66055379816209e-69\\
1.12562814070352	1.6509784647997e-70\\
1.14070351758794	5.64447158975996e-72\\
1.15577889447236	1.86244379171118e-73\\
1.17085427135678	5.93090623472841e-75\\
1.18592964824121	1.82279127715636e-76\\
1.20100502512563	5.40668198195167e-78\\
1.21608040201005	1.5477567591289e-79\\
1.23115577889447	4.27614667441517e-81\\
1.24623115577889	1.14019853224272e-82\\
1.26130653266332	2.93417736178343e-84\\
1.27638190954774	7.28736037196695e-86\\
1.29145728643216	1.746755548321e-87\\
1.30653266331658	4.04084336603742e-89\\
1.32160804020101	9.0217315778946e-91\\
1.33668341708543	1.9439531172889e-92\\
1.35175879396985	4.04258976583385e-94\\
1.36683417085427	8.11356197490833e-96\\
1.38190954773869	1.57159791381339e-97\\
1.39698492462312	2.93798334575841e-99\\
1.41206030150754	5.3007242923906e-101\\
1.42713567839196	9.22994476536941e-103\\
1.44221105527638	1.55110425278128e-104\\
1.4572864321608	2.51571173288238e-106\\
1.47236180904523	3.93784631011554e-108\\
1.48743718592965	5.94887233766287e-110\\
1.50251256281407	8.67338323675302e-112\\
1.51758793969849	1.2204512137518e-113\\
1.53266331658291	1.65741135986735e-115\\
1.54773869346734	2.17229198027859e-117\\
1.56281407035176	2.74779346758635e-119\\
1.57788944723618	3.35450137926678e-121\\
1.5929648241206	3.95229964174975e-123\\
1.60804020100502	4.49417263908639e-125\\
1.62311557788945	4.93205192176921e-127\\
1.63819095477387	5.2237637665589e-129\\
1.65326633165829	5.33970688292729e-131\\
1.66834170854271	5.26780034822843e-133\\
1.68341708542714	5.01555729494176e-135\\
1.69849246231156	4.60879175562921e-137\\
1.71356783919598	4.08726668080394e-139\\
1.7286432160804	3.49829850681712e-141\\
1.74371859296482	2.88973990096915e-143\\
1.75879396984925	2.30376748674689e-145\\
1.77386934673367	1.772542002225e-147\\
1.78894472361809	1.31623194062673e-149\\
1.80402010050251	9.43292358145945e-152\\
1.81909547738693	6.52436436367394e-154\\
1.83417085427136	4.35519994357291e-156\\
1.84924623115578	2.8057953364635e-158\\
1.8643216080402	1.74454381073637e-160\\
1.87939698492462	1.04685316848186e-162\\
1.89447236180905	6.06272212504302e-165\\
1.90954773869347	3.38865655986551e-167\\
1.92462311557789	1.82795477753842e-169\\
1.93969849246231	9.51658957011775e-172\\
1.95477386934673	4.78162189970433e-174\\
1.96984924623116	2.31871368067417e-176\\
1.98492462311558	1.08516809628053e-178\\
2	4.90145376102018e-181\\
};
\node[centered, align=center, inner sep=0, font=\color{mycolor1}]
at (axis cs:0.5,1.3) {Abstand zu klein};
\end{axis}

\begin{axis}[%
width=0.275\figurewidth,
height=0.93\figureheight,
at={(0.362\figurewidth,0\figureheight)},
scale only axis,
xmin=-1,
xmax=2,
xtick={-0.5,0.3},
xticklabels={{$-$},{$+$}},
xlabel style={font=\color{mycolor1}},
xlabel={$x_1$},
ymin=0,
ymax=1.4,
ytick={\empty},
ylabel style={font=\color{mycolor1}},
ylabel={$y$},
axis background/.style={fill=white}
]
\addplot [color=black, line width=1.0pt, forget plot]
  table[row sep=crcr]{%
-1	-1.05\\
-0.984924623115578	-1.00500113633494\\
-0.969849246231156	-0.960456806646297\\
-0.954773869346734	-0.916367010934067\\
-0.939698492462312	-0.872731749198252\\
-0.924623115577889	-0.829551021438853\\
-0.909547738693467	-0.786824827655867\\
-0.894472361809045	-0.744553167849297\\
-0.879396984924623	-0.702736042019141\\
-0.864321608040201	-0.6613734501654\\
-0.849246231155779	-0.620465392288074\\
-0.834170854271357	-0.580011868387162\\
-0.819095477386935	-0.540012878462665\\
-0.804020100502513	-0.500468422514583\\
-0.78894472361809	-0.461378500542916\\
-0.773869346733668	-0.422743112547663\\
-0.758793969849246	-0.384562258528825\\
-0.743718592964824	-0.346835938486402\\
-0.728643216080402	-0.309564152420393\\
-0.71356783919598	-0.2727469003308\\
-0.698492462311558	-0.236384182217621\\
-0.683417085427136	-0.200475998080857\\
-0.668341708542714	-0.165022347920507\\
-0.653266331658291	-0.130023231736572\\
-0.638190954773869	-0.0954786495290523\\
-0.623115577889447	-0.0613886012979472\\
-0.608040201005025	-0.0277530870432567\\
-0.592964824120603	0.00542789323501891\\
-0.577889447236181	0.0381543395368804\\
-0.562814070351759	0.0704262518623266\\
-0.547738693467337	0.102243630211358\\
-0.532663316582915	0.133606474583975\\
-0.517587939698492	0.164514784980177\\
-0.50251256281407	0.194968561399965\\
-0.487437185929648	0.224967803843337\\
-0.472361809045226	0.254512512310295\\
-0.457286432160804	0.283602686800838\\
-0.442211055276382	0.312238327314967\\
-0.42713567839196	0.34041943385268\\
-0.412060301507538	0.368146006413979\\
-0.396984924623116	0.395418044998864\\
-0.381909547738693	0.422235549607333\\
-0.366834170854271	0.448598520239388\\
-0.351758793969849	0.474506956895028\\
-0.336683417085427	0.499960859574253\\
-0.321608040201005	0.524960228277064\\
-0.306532663316583	0.549505063003459\\
-0.291457286432161	0.57359536375344\\
-0.276381909547739	0.597231130527007\\
-0.261306532663317	0.620412363324158\\
-0.246231155778894	0.643139062144895\\
-0.231155778894472	0.665411226989217\\
-0.21608040201005	0.687228857857125\\
-0.201005025125628	0.708591954748617\\
-0.185929648241206	0.729500517663695\\
-0.170854271356784	0.749954546602358\\
-0.155778894472362	0.769954041564607\\
-0.14070351758794	0.78949900255044\\
-0.125628140703518	0.80858942955986\\
-0.110552763819096	0.827225322592864\\
-0.0954773869346733	0.845406681649453\\
-0.0804020100502513	0.863133506729628\\
-0.0653266331658291	0.880405797833388\\
-0.050251256281407	0.897223554960733\\
-0.035175879396985	0.913586778111664\\
-0.0201005025125628	0.92949546728618\\
-0.00502512562814073	0.944949622484281\\
0.0100502512562815	0.959949243705967\\
0.0251256281407035	0.974494330951239\\
0.0402010050251256	0.988584884220095\\
0.0552763819095476	1.00222090351254\\
0.0703517587939699	1.01540238882857\\
0.085427135678392	1.02812934016818\\
0.100502512562814	1.04040175753138\\
0.115577889447236	1.05221964091816\\
0.130653266331658	1.06358299032853\\
0.14572864321608	1.07449180576248\\
0.160804020100503	1.08494608722002\\
0.175879396984925	1.09494583470114\\
0.190954773869347	1.10449104820585\\
0.206030150753769	1.11358172773415\\
0.221105527638191	1.12221787328603\\
0.236180904522613	1.13039948486149\\
0.251256281407035	1.13812656246054\\
0.266331658291457	1.14539910608318\\
0.28140703517588	1.1522171157294\\
0.296482412060302	1.15858059139921\\
0.311557788944724	1.1644895330926\\
0.326633165829146	1.16994394080958\\
0.341708542713568	1.17494381455014\\
0.35678391959799	1.17948915431428\\
0.371859296482412	1.18357996010202\\
0.386934673366834	1.18721623191334\\
0.402010050251256	1.19039796974824\\
0.417085427135678	1.19312517360673\\
0.432160804020101	1.1953978434888\\
0.447236180904523	1.19721597939446\\
0.462311557788945	1.1985795813237\\
0.477386934673367	1.19948864927653\\
0.492462311557789	1.19994318325295\\
0.507537688442211	1.19994318325295\\
0.522613065326633	1.19948864927653\\
0.537688442211055	1.1985795813237\\
0.552763819095477	1.19721597939446\\
0.567839195979899	1.1953978434888\\
0.582914572864322	1.19312517360673\\
0.597989949748744	1.19039796974824\\
0.613065326633166	1.18721623191334\\
0.628140703517588	1.18357996010202\\
0.64321608040201	1.17948915431428\\
0.658291457286432	1.17494381455014\\
0.673366834170854	1.16994394080958\\
0.688442211055276	1.1644895330926\\
0.703517587939698	1.15858059139921\\
0.71859296482412	1.1522171157294\\
0.733668341708543	1.14539910608318\\
0.748743718592965	1.13812656246054\\
0.763819095477387	1.13039948486149\\
0.778894472361809	1.12221787328603\\
0.793969849246231	1.11358172773415\\
0.809045226130653	1.10449104820585\\
0.824120603015075	1.09494583470114\\
0.839195979899497	1.08494608722002\\
0.85427135678392	1.07449180576248\\
0.869346733668342	1.06358299032853\\
0.884422110552764	1.05221964091816\\
0.899497487437186	1.04040175753138\\
0.914572864321608	1.02812934016818\\
0.92964824120603	1.01540238882857\\
0.944723618090452	1.00222090351254\\
0.959798994974874	0.988584884220095\\
0.974874371859296	0.974494330951239\\
0.989949748743719	0.959949243705967\\
1.00502512562814	0.944949622484281\\
1.02010050251256	0.929495467286179\\
1.03517587939698	0.913586778111664\\
1.05025125628141	0.897223554960733\\
1.06532663316583	0.880405797833388\\
1.08040201005025	0.863133506729628\\
1.09547738693467	0.845406681649453\\
1.1105527638191	0.827225322592864\\
1.12562814070352	0.80858942955986\\
1.14070351758794	0.78949900255044\\
1.15577889447236	0.769954041564607\\
1.17085427135678	0.749954546602358\\
1.18592964824121	0.729500517663696\\
1.20100502512563	0.708591954748617\\
1.21608040201005	0.687228857857125\\
1.23115577889447	0.665411226989218\\
1.24623115577889	0.643139062144895\\
1.26130653266332	0.620412363324159\\
1.27638190954774	0.597231130527007\\
1.29145728643216	0.57359536375344\\
1.30653266331658	0.54950506300346\\
1.32160804020101	0.524960228277064\\
1.33668341708543	0.499960859574254\\
1.35175879396985	0.474506956895028\\
1.36683417085427	0.448598520239388\\
1.38190954773869	0.422235549607333\\
1.39698492462312	0.395418044998864\\
1.41206030150754	0.36814600641398\\
1.42713567839196	0.34041943385268\\
1.44221105527638	0.312238327314966\\
1.4572864321608	0.283602686800839\\
1.47236180904523	0.254512512310295\\
1.48743718592965	0.224967803843338\\
1.50251256281407	0.194968561399965\\
1.51758793969849	0.164514784980177\\
1.53266331658291	0.133606474583975\\
1.54773869346734	0.102243630211358\\
1.56281407035176	0.0704262518623262\\
1.57788944723618	0.0381543395368804\\
1.5929648241206	0.00542789323501891\\
1.60804020100502	-0.0277530870432563\\
1.62311557788945	-0.0613886012979472\\
1.63819095477387	-0.0954786495290527\\
1.65326633165829	-0.130023231736572\\
1.66834170854271	-0.165022347920507\\
1.68341708542714	-0.200475998080856\\
1.69849246231156	-0.236384182217621\\
1.71356783919598	-0.2727469003308\\
1.7286432160804	-0.309564152420393\\
1.74371859296482	-0.346835938486402\\
1.75879396984925	-0.384562258528825\\
1.77386934673367	-0.422743112547663\\
1.78894472361809	-0.461378500542916\\
1.80402010050251	-0.500468422514583\\
1.81909547738693	-0.540012878462665\\
1.83417085427136	-0.580011868387162\\
1.84924623115578	-0.620465392288074\\
1.8643216080402	-0.661373450165401\\
1.87939698492462	-0.702736042019141\\
1.89447236180905	-0.744553167849297\\
1.90954773869347	-0.786824827655867\\
1.92462311557789	-0.829551021438853\\
1.93969849246231	-0.872731749198253\\
1.95477386934673	-0.916367010934067\\
1.96984924623116	-0.960456806646297\\
1.98492462311558	-1.00500113633494\\
2	-1.05\\
};
\addplot [color=black, line width=1.5pt, forget plot]
  table[row sep=crcr]{%
-0.5	0.2\\
0.3	1.16\\
};
\addplot[only marks, mark=*, mark options={}, mark size=1.5811pt, draw=black, fill=gray, forget plot] table[row sep=crcr]{%
x	y\\
-0.5	0.2\\
0.3	1.16\\
};
\addplot [color=black, line width=0.8pt, forget plot]
  table[row sep=crcr]{%
-1	4.94057116545912e-10\\
-0.984924623115578	1.57602276748064e-09\\
-0.969849246231156	4.85205611542638e-09\\
-0.954773869346734	1.44167425497681e-08\\
-0.939698492462312	4.13415229494018e-08\\
-0.924623115577889	1.14415219901712e-07\\
-0.909547738693467	3.05604066037433e-07\\
-0.894472361809045	7.87793615663737e-07\\
-0.879396984924623	1.95994455544759e-06\\
-0.864321608040201	4.70601291743604e-06\\
-0.849246231155779	1.09053709836089e-05\\
-0.834170854271357	2.43896615423924e-05\\
-0.819095477386935	5.26440270021226e-05\\
-0.804020100502513	0.000109665603513466\\
-0.78894472361809	0.000220480285668284\\
-0.773869346733668	0.000427806218451707\\
-0.758793969849246	0.000801128945257342\\
-0.743718592964824	0.00144789061734106\\
-0.728643216080402	0.00252549832279041\\
-0.71356783919598	0.00425144353314529\\
-0.698492462311558	0.00690722727455912\\
-0.683417085427136	0.0108305142315072\\
-0.668341708542714	0.016389753002101\\
-0.653266331658291	0.0239372215927844\\
-0.638190954773869	0.0337406210220738\\
-0.623115577889447	0.0458997588353218\\
-0.608040201005025	0.0602623045175218\\
-0.592964824120603	0.076358795283038\\
-0.577889447236181	0.0933792540392318\\
-0.562814070351759	0.11020967781249\\
-0.547738693467337	0.125535647068942\\
-0.532663316582915	0.138004230025995\\
-0.517587939698492	0.146418425165253\\
-0.50251256281407	0.149926038101147\\
-0.487437185929648	0.148161852006634\\
-0.472361809045226	0.141310271747869\\
-0.457286432160804	0.130073571571152\\
-0.442211055276382	0.115553312379778\\
-0.42713567839196	0.0990726394180448\\
-0.412060301507538	0.0819790851628189\\
-0.396984924623116	0.0654682002244067\\
-0.381909547738693	0.0504586627332136\\
-0.366834170854271	0.0375335006100462\\
-0.351758793969849	0.0269451376333582\\
-0.336683417085427	0.0186689435016723\\
-0.321608040201005	0.0124835204393518\\
-0.306532663316583	0.00805624030634114\\
-0.291457286432161	0.00501771219792656\\
-0.276381909547739	0.00301617880345267\\
-0.261306532663317	0.00174979197149739\\
-0.246231155778894	0.000979701460978718\\
-0.231155778894472	0.00052939409408853\\
-0.21608040201005	0.000276084756470587\\
-0.201005025125628	0.000138958067324504\\
-0.185929648241206	6.74998957371878e-05\\
-0.170854271356784	3.16446603768558e-05\\
-0.155778894472362	1.43177830072836e-05\\
-0.14070351758794	6.25214603522081e-06\\
-0.125628140703518	2.63487763442609e-06\\
-0.110552763819096	1.07169150642828e-06\\
-0.0954773869346733	4.20685096491226e-07\\
-0.0804020100502513	1.59375831900113e-07\\
-0.0653266331658291	5.82727887989975e-08\\
-0.050251256281407	2.0563031523068e-08\\
-0.035175879396985	7.00303795033874e-09\\
-0.0201005025125628	2.30178003387978e-09\\
-0.00502512562814073	7.30161882950325e-10\\
0.0100502512562815	2.23538584508942e-10\\
0.0251256281407035	6.60486333507968e-11\\
0.0402010050251256	1.88344600914967e-11\\
0.0552763819095476	5.18346911064005e-12\\
0.0703517587939699	1.37678414045859e-12\\
0.085427135678392	3.52930502547627e-13\\
0.100502512562814	8.73153370218821e-14\\
0.115577889447236	2.08482628999997e-14\\
0.130653266331658	4.80426788284014e-15\\
0.14572864321608	1.06847054926449e-15\\
0.160804020100503	2.293379345176e-16\\
0.175879396984925	4.75080535998551e-17\\
0.190954773869347	9.49809652749516e-18\\
0.206030150753769	1.83266852111727e-18\\
0.221105527638191	3.41278761247874e-19\\
0.236180904522613	6.13355979363771e-20\\
0.251256281407035	1.06388318608236e-20\\
0.266331658291457	1.78095639878592e-21\\
0.28140703517588	2.87733633100887e-22\\
0.296482412060302	4.48648243476834e-23\\
0.311557788944724	6.75148511159821e-24\\
0.326633165829146	9.80552061285523e-25\\
0.341708542713568	1.37442168882428e-25\\
0.35678391959799	1.85929089646744e-26\\
0.371859296482412	2.42746337503807e-27\\
0.386934673366834	3.05869398817556e-28\\
0.402010050251256	3.71960971241608e-29\\
0.417085427135678	4.36552738374366e-30\\
0.432160804020101	4.94486068784572e-31\\
0.447236180904523	5.40566844359041e-32\\
0.462311557788945	5.70325446290064e-33\\
0.477386934673367	5.80729774725765e-34\\
0.492462311557789	5.70694172659544e-35\\
0.507537688442211	5.41266044031043e-36\\
0.522613065326633	4.95445773244763e-37\\
0.537688442211055	4.37682792651257e-38\\
0.552763819095477	3.73164926603188e-39\\
0.567839195979899	3.0705782090626e-40\\
0.582914572864322	2.43847051792633e-41\\
0.597989949748744	1.86892922367551e-42\\
0.613065326633166	1.38243971220511e-43\\
0.628140703517588	9.86909994390187e-45\\
0.64321608040201	6.79965523954378e-46\\
0.658291457286432	4.52141363728256e-47\\
0.673366834170854	2.90161365610459e-48\\
0.688442211055276	1.79714423209735e-49\\
0.703517587939698	1.07424732364763e-50\\
0.71859296482412	6.19731580354211e-52\\
0.733668341708543	3.45049160324826e-53\\
0.748743718592965	1.85411352709972e-54\\
0.763819095477387	9.6154512444845e-56\\
0.778894472361809	4.81261386557408e-57\\
0.793969849246231	2.32471846939545e-58\\
0.809045226130653	1.08377147018656e-59\\
0.824120603015075	4.87621743082316e-61\\
0.839195979899497	2.11741707602194e-62\\
0.85427135678392	8.87376166071854e-64\\
0.869346733668342	3.58911280870636e-65\\
0.884422110552764	1.40102039622597e-66\\
0.899497487437186	5.27812665085848e-68\\
0.914572864321608	1.91908034074213e-69\\
0.92964824120603	6.73417667380368e-71\\
0.944723618090452	2.28062486945852e-72\\
0.959798994974874	7.45420282973179e-74\\
0.974874371859296	2.35140022753671e-75\\
0.989949748743719	7.15862983603782e-77\\
1.00502512562814	2.10334859928141e-78\\
1.02010050251256	5.96445281528211e-80\\
1.03517587939698	1.63232998342091e-81\\
1.05025125628141	4.31144955379258e-83\\
1.06532663316583	1.099047999615e-84\\
1.08040201005025	2.70388372325828e-86\\
1.09547738693467	6.42003451535516e-88\\
1.1105527638191	1.47117628166674e-89\\
1.12562814070352	3.25364425381868e-91\\
1.14070351758794	6.94469886085328e-93\\
1.15577889447236	1.43058863505929e-94\\
1.17085427135678	2.84416064620836e-96\\
1.18592964824121	5.45722016773012e-98\\
1.20100502512563	1.01057099266459e-99\\
1.21608040201005	1.80609329772476e-101\\
1.23115577889447	3.1152401680726e-103\\
1.24623115577889	5.18586118295871e-105\\
1.26130653266332	8.33159672542081e-107\\
1.27638190954774	1.29185447563516e-108\\
1.29145728643216	1.93320071853718e-110\\
1.30653266331658	2.79201856539966e-112\\
1.32160804020101	3.89168477964236e-114\\
1.33668341708543	5.23522088366185e-116\\
1.35175879396985	6.79689192639872e-118\\
1.36683417085427	8.51655036505505e-120\\
1.38190954773869	1.0299000372269e-121\\
1.39698492462312	1.20199989235265e-123\\
1.41206030150754	1.35391619733106e-125\\
1.42713567839196	1.47182826596156e-127\\
1.44221105527638	1.54418914276882e-129\\
1.4572864321608	1.56358627726146e-131\\
1.47236180904523	1.5279924417714e-133\\
1.48743718592965	1.44111474570082e-135\\
1.50251256281407	1.31175859381067e-137\\
1.51758793969849	1.15235761060435e-139\\
1.53266331658291	9.7700914965437e-142\\
1.54773869346734	7.99443889830478e-144\\
1.56281407035176	6.31328426274197e-146\\
1.57788944723618	4.81172392320207e-148\\
1.5929648241206	3.53935445921421e-150\\
1.60804020100502	2.51261192063529e-152\\
1.62311557788945	1.72149106676805e-154\\
1.63819095477387	1.13831414114531e-156\\
1.65326633165829	7.26436218364783e-159\\
1.66834170854271	4.47415316105865e-161\\
1.68341708542714	2.65951353658765e-163\\
1.69849246231156	1.52570849216465e-165\\
1.71356783919598	8.44731952777952e-168\\
1.7286432160804	4.51382046057725e-170\\
1.74371859296482	2.32781059049058e-172\\
1.75879396984925	1.15858792732176e-174\\
1.77386934673367	5.56529715339483e-177\\
1.78894472361809	2.58003571227894e-179\\
1.80402010050251	1.15435948830055e-181\\
1.81909547738693	4.98464732871307e-184\\
1.83417085427136	2.0773314522254e-186\\
1.84924623115578	8.3551674220519e-189\\
1.8643216080402	3.24326560997671e-191\\
1.87939698492462	1.21503245181336e-193\\
1.89447236180905	4.39310150611083e-196\\
1.90954773869347	1.53296626912309e-198\\
1.92462311557789	5.16264140113166e-201\\
1.93969849246231	1.67798981710585e-203\\
1.95477386934673	5.26362186473754e-206\\
1.96984924623116	1.59352192693446e-208\\
1.98492462311558	4.65596176719113e-211\\
2	1.31292152108604e-213\\
};
\addplot [color=black, line width=0.8pt, forget plot]
  table[row sep=crcr]{%
-1	6.8493277019912e-59\\
-0.984924623115578	1.43822686623393e-57\\
-0.969849246231156	2.91463951975169e-56\\
-0.954773869346734	5.70059649521197e-55\\
-0.939698492462312	1.07605322607897e-53\\
-0.924623115577889	1.96031203617885e-52\\
-0.909547738693467	3.44662996315189e-51\\
-0.894472361809045	5.84846792114822e-50\\
-0.879396984924623	9.57783972357778e-49\\
-0.864321608040201	1.51380877556303e-47\\
-0.849246231155779	2.30915173764227e-46\\
-0.834170854271357	3.39947557840153e-45\\
-0.819095477386935	4.83002500803532e-44\\
-0.804020100502513	6.62315369582974e-43\\
-0.78894472361809	8.76512850145944e-42\\
-0.773869346733668	1.11951454851226e-40\\
-0.758793969849246	1.38000034905347e-39\\
-0.743718592964824	1.64174858596057e-38\\
-0.728643216080402	1.88500326041787e-37\\
-0.71356783919598	2.08879375664382e-36\\
-0.698492462311558	2.23386547423101e-35\\
-0.683417085427136	2.30566640265031e-34\\
-0.668341708542714	2.29675106665317e-33\\
-0.653266331658291	2.20805243490529e-32\\
-0.638190954773869	2.0487210904563e-31\\
-0.623115577889447	1.8345700292747e-30\\
-0.608040201005025	1.58549088831272e-29\\
-0.592964824120603	1.32242547519364e-28\\
-0.577889447236181	1.06452695672951e-27\\
-0.562814070351759	8.27027822529765e-27\\
-0.547738693467337	6.20099779683591e-26\\
-0.532663316582915	4.48725793390612e-25\\
-0.517587939698492	3.13385206362226e-24\\
-0.50251256281407	2.11229197344011e-23\\
-0.487437185929648	1.37406540772869e-22\\
-0.472361809045226	8.62658416958921e-22\\
-0.457286432160804	5.22694971440549e-21\\
-0.442211055276382	3.05658009167148e-20\\
-0.42713567839196	1.72504821616104e-19\\
-0.412060301507538	9.39603653554335e-19\\
-0.396984924623116	4.93930853640692e-18\\
-0.381909547738693	2.50591083671817e-17\\
-0.366834170854271	1.22699579162819e-16\\
-0.351758793969849	5.79827130347142e-16\\
-0.336683417085427	2.64442948973482e-15\\
-0.321608040201005	1.1639744599651e-14\\
-0.306532663316583	4.94461988130744e-14\\
-0.291457286432161	2.02721761177089e-13\\
-0.276381909547739	8.02131978312006e-13\\
-0.261306532663317	3.06315723891553e-12\\
-0.246231155778894	1.12893971418547e-11\\
-0.231155778894472	4.01559783638835e-11\\
-0.21608040201005	1.37850286242349e-10\\
-0.201005025125628	4.56712747023503e-10\\
-0.185929648241206	1.46034878653212e-09\\
-0.170854271356784	4.50659006895807e-09\\
-0.155778894472362	1.34220087218964e-08\\
-0.14070351758794	3.85802447314335e-08\\
-0.125628140703518	1.07026282314714e-07\\
-0.110552763819096	2.86545704068012e-07\\
-0.0954773869346733	7.40415264934752e-07\\
-0.0804020100502513	1.84643841118667e-06\\
-0.0653266331658291	4.4439819728022e-06\\
-0.050251256281407	1.03225679677087e-05\\
-0.035175879396985	2.31409515928824e-05\\
-0.0201005025125628	5.00671276659756e-05\\
-0.00502512562814073	0.000104544728687137\\
0.0100502512562815	0.000210683053035108\\
0.0251256281407035	0.000409765194476735\\
0.0402010050251256	0.000769163257750072\\
0.0552763819095476	0.00139341343623919\\
0.0703517587939699	0.00243623656942126\\
0.085427135678392	0.00411090009104515\\
0.100502512562814	0.0066947195155291\\
0.115577889447236	0.0105221830032947\\
0.130653266331658	0.0159608977533265\\
0.14572864321608	0.0233661295529732\\
0.160804020100503	0.0330137034066676\\
0.175879396984925	0.0450173276410492\\
0.190954773869347	0.0592438366465508\\
0.206030150753769	0.0752462119081839\\
0.221105527638191	0.0922367744155688\\
0.236180904522613	0.109119300376484\\
0.251256281407035	0.124588236571305\\
0.266331658291457	0.13728734460741\\
0.28140703517588	0.146003064499325\\
0.296482412060302	0.149855068988588\\
0.311557788944724	0.148442720372069\\
0.326633165829146	0.141913715796969\\
0.341708542713568	0.130938643863252\\
0.35678391959799	0.116597518025902\\
0.371859296482412	0.100204857146412\\
0.386934673366834	0.0831124798770908\\
0.402010050251256	0.0665306411242142\\
0.417085427135678	0.0513990599884718\\
0.432160804020101	0.0383236305293732\\
0.447236180904523	0.0275775774421602\\
0.462311557788945	0.0191524166618557\\
0.477386934673367	0.0128371628278504\\
0.492462311557789	0.00830409901024845\\
0.507537688442211	0.00518434611023925\\
0.522613065326633	0.00312372977128873\\
0.537688442211055	0.00181648127785973\\
0.552763819095477	0.00101945109221272\\
0.567839195979899	0.000552178973603402\\
0.582914572864322	0.000288649849205513\\
0.597989949748744	0.000145626631837943\\
0.613065326633166	7.09068634864491e-05\\
0.628140703517588	3.33206715322245e-05\\
0.64321608040201	1.51118354481397e-05\\
0.658291457286432	6.61452537119759e-06\\
0.673366834170854	2.79420430517544e-06\\
0.688442211055276	1.13918859528062e-06\\
0.703517587939698	4.48240505951625e-07\\
0.71859296482412	1.70217642404199e-07\\
0.733668341708543	6.23844057578812e-08\\
0.748743718592965	2.20660965593118e-08\\
0.763819095477387	7.53274014877157e-09\\
0.778894472361809	2.48175244387289e-09\\
0.793969849246231	7.8911796469827e-10\\
0.809045226130653	2.42160554747685e-10\\
0.824120603015075	7.17204284645568e-11\\
0.839195979899497	2.05003062128995e-11\\
0.85427135678392	5.65530225182869e-12\\
0.869346733668342	1.50566825562585e-12\\
0.884422110552764	3.8688399825355e-13\\
0.899497487437186	9.59423254869063e-14\\
0.914572864321608	2.29624230774115e-14\\
0.92964824120603	5.30399600441583e-15\\
0.944723618090452	1.18240610167668e-15\\
0.959798994974874	2.54394745236841e-16\\
0.974874371859296	5.28235531677505e-17\\
0.989949748743719	1.05858341383548e-17\\
1.00502512562814	2.04738991495923e-18\\
1.02010050251256	3.82167749183725e-19\\
1.03517587939698	6.88470774919187e-20\\
1.05025125628141	1.19700229935844e-20\\
1.06532663316583	2.00854913347827e-21\\
1.08040201005025	3.25272940288304e-22\\
1.09547738693467	5.0838345835915e-23\\
1.1105527638191	7.66854352716832e-24\\
1.12562814070352	1.11638081111364e-24\\
1.14070351758794	1.56851920004244e-25\\
1.15577889447236	2.1268914017999e-26\\
1.17085427135678	2.78342031154454e-27\\
1.18592964824121	3.51552563240852e-28\\
1.20100502512563	4.2852853453004e-29\\
1.21608040201005	5.04135432976769e-30\\
1.23115577889447	5.72390872763086e-31\\
1.24623115577889	6.27214626383576e-32\\
1.26130653266332	6.63311695637867e-33\\
1.27638190954774	6.77013194210741e-34\\
1.29145728643216	6.66890622443611e-35\\
1.30653266331658	6.34001212813184e-36\\
1.32160804020101	5.81706031579322e-37\\
1.33668341708543	5.15104145862591e-38\\
1.35175879396985	4.40214682959121e-39\\
1.36683417085427	3.63088089407869e-40\\
1.38190954773869	2.89026378807301e-41\\
1.39698492462312	2.22044980878024e-42\\
1.41206030150754	1.64635103142841e-43\\
1.42713567839196	1.17809935861205e-44\\
1.44221105527638	8.13615847608739e-46\\
1.4572864321608	5.4229409940991e-47\\
1.47236180904523	3.48841671657882e-48\\
1.48743718592965	2.16570769247293e-49\\
1.50251256281407	1.29762534480196e-50\\
1.51758793969849	7.50372310825173e-52\\
1.53266331658291	4.18776478527071e-53\\
1.54773869346734	2.25561908978704e-54\\
1.56281407035176	1.17253885342462e-55\\
1.57788944723618	5.88256485105805e-57\\
1.5929648241206	2.84828999936561e-58\\
1.60804020100502	1.33100499883068e-59\\
1.62311557788945	6.00279024704189e-61\\
1.63819095477387	2.61279087186008e-62\\
1.65326633165829	1.09757481257323e-63\\
1.66834170854271	4.44981150332529e-65\\
1.68341708542714	1.7411135556756e-66\\
1.69849246231156	6.57492161106601e-68\\
1.71356783919598	2.39624964800869e-69\\
1.7286432160804	8.42852433393525e-71\\
1.74371859296482	2.86120534768046e-72\\
1.75879396984925	9.37399054932213e-74\\
1.77386934673367	2.96399860401685e-75\\
1.78894472361809	9.04501889346669e-77\\
1.80402010050251	2.66390640551074e-78\\
1.81909547738693	7.5719274837066e-80\\
1.83417085427136	2.07716946728902e-81\\
1.84924623115578	5.49940133150177e-83\\
1.8643216080402	1.4051959808193e-84\\
1.87939698492462	3.46526445307128e-86\\
1.89447236180905	8.24733942035108e-88\\
1.90954773869347	1.89438982916651e-89\\
1.92462311557789	4.19955079439079e-91\\
1.93969849246231	8.98492225999689e-93\\
1.95477386934673	1.85525584845646e-94\\
1.96984924623116	3.69718595868385e-96\\
1.98492462311558	7.11077244524251e-98\\
2	1.3198976840684e-99\\
};
\node[centered, align=center, inner sep=0, font=\color{mycolor1}]
at (axis cs:0.5,1.3) {Abstand gut};
\end{axis}

\begin{axis}[%
width=0.275\figurewidth,
height=0.93\figureheight,
at={(0.725\figurewidth,0\figureheight)},
scale only axis,
xmin=-1,
xmax=2,
xtick={-0.5,1.5},
xticklabels={{$-$},{$+$}},
xlabel style={font=\color{mycolor1}},
xlabel={$x_1$},
ymin=0,
ymax=1.4,
ytick={\empty},
ylabel style={font=\color{mycolor1}},
ylabel={$y$},
axis background/.style={fill=white}
]
\addplot [color=black, line width=1.0pt, forget plot]
  table[row sep=crcr]{%
-1	-1.05\\
-0.984924623115578	-1.00500113633494\\
-0.969849246231156	-0.960456806646297\\
-0.954773869346734	-0.916367010934067\\
-0.939698492462312	-0.872731749198252\\
-0.924623115577889	-0.829551021438853\\
-0.909547738693467	-0.786824827655867\\
-0.894472361809045	-0.744553167849297\\
-0.879396984924623	-0.702736042019141\\
-0.864321608040201	-0.6613734501654\\
-0.849246231155779	-0.620465392288074\\
-0.834170854271357	-0.580011868387162\\
-0.819095477386935	-0.540012878462665\\
-0.804020100502513	-0.500468422514583\\
-0.78894472361809	-0.461378500542916\\
-0.773869346733668	-0.422743112547663\\
-0.758793969849246	-0.384562258528825\\
-0.743718592964824	-0.346835938486402\\
-0.728643216080402	-0.309564152420393\\
-0.71356783919598	-0.2727469003308\\
-0.698492462311558	-0.236384182217621\\
-0.683417085427136	-0.200475998080857\\
-0.668341708542714	-0.165022347920507\\
-0.653266331658291	-0.130023231736572\\
-0.638190954773869	-0.0954786495290523\\
-0.623115577889447	-0.0613886012979472\\
-0.608040201005025	-0.0277530870432567\\
-0.592964824120603	0.00542789323501891\\
-0.577889447236181	0.0381543395368804\\
-0.562814070351759	0.0704262518623266\\
-0.547738693467337	0.102243630211358\\
-0.532663316582915	0.133606474583975\\
-0.517587939698492	0.164514784980177\\
-0.50251256281407	0.194968561399965\\
-0.487437185929648	0.224967803843337\\
-0.472361809045226	0.254512512310295\\
-0.457286432160804	0.283602686800838\\
-0.442211055276382	0.312238327314967\\
-0.42713567839196	0.34041943385268\\
-0.412060301507538	0.368146006413979\\
-0.396984924623116	0.395418044998864\\
-0.381909547738693	0.422235549607333\\
-0.366834170854271	0.448598520239388\\
-0.351758793969849	0.474506956895028\\
-0.336683417085427	0.499960859574253\\
-0.321608040201005	0.524960228277064\\
-0.306532663316583	0.549505063003459\\
-0.291457286432161	0.57359536375344\\
-0.276381909547739	0.597231130527007\\
-0.261306532663317	0.620412363324158\\
-0.246231155778894	0.643139062144895\\
-0.231155778894472	0.665411226989217\\
-0.21608040201005	0.687228857857125\\
-0.201005025125628	0.708591954748617\\
-0.185929648241206	0.729500517663695\\
-0.170854271356784	0.749954546602358\\
-0.155778894472362	0.769954041564607\\
-0.14070351758794	0.78949900255044\\
-0.125628140703518	0.80858942955986\\
-0.110552763819096	0.827225322592864\\
-0.0954773869346733	0.845406681649453\\
-0.0804020100502513	0.863133506729628\\
-0.0653266331658291	0.880405797833388\\
-0.050251256281407	0.897223554960733\\
-0.035175879396985	0.913586778111664\\
-0.0201005025125628	0.92949546728618\\
-0.00502512562814073	0.944949622484281\\
0.0100502512562815	0.959949243705967\\
0.0251256281407035	0.974494330951239\\
0.0402010050251256	0.988584884220095\\
0.0552763819095476	1.00222090351254\\
0.0703517587939699	1.01540238882857\\
0.085427135678392	1.02812934016818\\
0.100502512562814	1.04040175753138\\
0.115577889447236	1.05221964091816\\
0.130653266331658	1.06358299032853\\
0.14572864321608	1.07449180576248\\
0.160804020100503	1.08494608722002\\
0.175879396984925	1.09494583470114\\
0.190954773869347	1.10449104820585\\
0.206030150753769	1.11358172773415\\
0.221105527638191	1.12221787328603\\
0.236180904522613	1.13039948486149\\
0.251256281407035	1.13812656246054\\
0.266331658291457	1.14539910608318\\
0.28140703517588	1.1522171157294\\
0.296482412060302	1.15858059139921\\
0.311557788944724	1.1644895330926\\
0.326633165829146	1.16994394080958\\
0.341708542713568	1.17494381455014\\
0.35678391959799	1.17948915431428\\
0.371859296482412	1.18357996010202\\
0.386934673366834	1.18721623191334\\
0.402010050251256	1.19039796974824\\
0.417085427135678	1.19312517360673\\
0.432160804020101	1.1953978434888\\
0.447236180904523	1.19721597939446\\
0.462311557788945	1.1985795813237\\
0.477386934673367	1.19948864927653\\
0.492462311557789	1.19994318325295\\
0.507537688442211	1.19994318325295\\
0.522613065326633	1.19948864927653\\
0.537688442211055	1.1985795813237\\
0.552763819095477	1.19721597939446\\
0.567839195979899	1.1953978434888\\
0.582914572864322	1.19312517360673\\
0.597989949748744	1.19039796974824\\
0.613065326633166	1.18721623191334\\
0.628140703517588	1.18357996010202\\
0.64321608040201	1.17948915431428\\
0.658291457286432	1.17494381455014\\
0.673366834170854	1.16994394080958\\
0.688442211055276	1.1644895330926\\
0.703517587939698	1.15858059139921\\
0.71859296482412	1.1522171157294\\
0.733668341708543	1.14539910608318\\
0.748743718592965	1.13812656246054\\
0.763819095477387	1.13039948486149\\
0.778894472361809	1.12221787328603\\
0.793969849246231	1.11358172773415\\
0.809045226130653	1.10449104820585\\
0.824120603015075	1.09494583470114\\
0.839195979899497	1.08494608722002\\
0.85427135678392	1.07449180576248\\
0.869346733668342	1.06358299032853\\
0.884422110552764	1.05221964091816\\
0.899497487437186	1.04040175753138\\
0.914572864321608	1.02812934016818\\
0.92964824120603	1.01540238882857\\
0.944723618090452	1.00222090351254\\
0.959798994974874	0.988584884220095\\
0.974874371859296	0.974494330951239\\
0.989949748743719	0.959949243705967\\
1.00502512562814	0.944949622484281\\
1.02010050251256	0.929495467286179\\
1.03517587939698	0.913586778111664\\
1.05025125628141	0.897223554960733\\
1.06532663316583	0.880405797833388\\
1.08040201005025	0.863133506729628\\
1.09547738693467	0.845406681649453\\
1.1105527638191	0.827225322592864\\
1.12562814070352	0.80858942955986\\
1.14070351758794	0.78949900255044\\
1.15577889447236	0.769954041564607\\
1.17085427135678	0.749954546602358\\
1.18592964824121	0.729500517663696\\
1.20100502512563	0.708591954748617\\
1.21608040201005	0.687228857857125\\
1.23115577889447	0.665411226989218\\
1.24623115577889	0.643139062144895\\
1.26130653266332	0.620412363324159\\
1.27638190954774	0.597231130527007\\
1.29145728643216	0.57359536375344\\
1.30653266331658	0.54950506300346\\
1.32160804020101	0.524960228277064\\
1.33668341708543	0.499960859574254\\
1.35175879396985	0.474506956895028\\
1.36683417085427	0.448598520239388\\
1.38190954773869	0.422235549607333\\
1.39698492462312	0.395418044998864\\
1.41206030150754	0.36814600641398\\
1.42713567839196	0.34041943385268\\
1.44221105527638	0.312238327314966\\
1.4572864321608	0.283602686800839\\
1.47236180904523	0.254512512310295\\
1.48743718592965	0.224967803843338\\
1.50251256281407	0.194968561399965\\
1.51758793969849	0.164514784980177\\
1.53266331658291	0.133606474583975\\
1.54773869346734	0.102243630211358\\
1.56281407035176	0.0704262518623262\\
1.57788944723618	0.0381543395368804\\
1.5929648241206	0.00542789323501891\\
1.60804020100502	-0.0277530870432563\\
1.62311557788945	-0.0613886012979472\\
1.63819095477387	-0.0954786495290527\\
1.65326633165829	-0.130023231736572\\
1.66834170854271	-0.165022347920507\\
1.68341708542714	-0.200475998080856\\
1.69849246231156	-0.236384182217621\\
1.71356783919598	-0.2727469003308\\
1.7286432160804	-0.309564152420393\\
1.74371859296482	-0.346835938486402\\
1.75879396984925	-0.384562258528825\\
1.77386934673367	-0.422743112547663\\
1.78894472361809	-0.461378500542916\\
1.80402010050251	-0.500468422514583\\
1.81909547738693	-0.540012878462665\\
1.83417085427136	-0.580011868387162\\
1.84924623115578	-0.620465392288074\\
1.8643216080402	-0.661373450165401\\
1.87939698492462	-0.702736042019141\\
1.89447236180905	-0.744553167849297\\
1.90954773869347	-0.786824827655867\\
1.92462311557789	-0.829551021438853\\
1.93969849246231	-0.872731749198253\\
1.95477386934673	-0.916367010934067\\
1.96984924623116	-0.960456806646297\\
1.98492462311558	-1.00500113633494\\
2	-1.05\\
};
\addplot [color=black, line width=1.5pt, forget plot]
  table[row sep=crcr]{%
-0.5	0.2\\
1.5	0.2\\
};
\addplot[only marks, mark=*, mark options={}, mark size=1.5811pt, draw=black, fill=gray, forget plot] table[row sep=crcr]{%
x	y\\
-0.5	0.2\\
1.5	0.2\\
};
\addplot [color=black, line width=0.8pt, forget plot]
  table[row sep=crcr]{%
-1	4.94057116545912e-10\\
-0.984924623115578	1.57602276748064e-09\\
-0.969849246231156	4.85205611542638e-09\\
-0.954773869346734	1.44167425497681e-08\\
-0.939698492462312	4.13415229494018e-08\\
-0.924623115577889	1.14415219901712e-07\\
-0.909547738693467	3.05604066037433e-07\\
-0.894472361809045	7.87793615663737e-07\\
-0.879396984924623	1.95994455544759e-06\\
-0.864321608040201	4.70601291743604e-06\\
-0.849246231155779	1.09053709836089e-05\\
-0.834170854271357	2.43896615423924e-05\\
-0.819095477386935	5.26440270021226e-05\\
-0.804020100502513	0.000109665603513466\\
-0.78894472361809	0.000220480285668284\\
-0.773869346733668	0.000427806218451707\\
-0.758793969849246	0.000801128945257342\\
-0.743718592964824	0.00144789061734106\\
-0.728643216080402	0.00252549832279041\\
-0.71356783919598	0.00425144353314529\\
-0.698492462311558	0.00690722727455912\\
-0.683417085427136	0.0108305142315072\\
-0.668341708542714	0.016389753002101\\
-0.653266331658291	0.0239372215927844\\
-0.638190954773869	0.0337406210220738\\
-0.623115577889447	0.0458997588353218\\
-0.608040201005025	0.0602623045175218\\
-0.592964824120603	0.076358795283038\\
-0.577889447236181	0.0933792540392318\\
-0.562814070351759	0.11020967781249\\
-0.547738693467337	0.125535647068942\\
-0.532663316582915	0.138004230025995\\
-0.517587939698492	0.146418425165253\\
-0.50251256281407	0.149926038101147\\
-0.487437185929648	0.148161852006634\\
-0.472361809045226	0.141310271747869\\
-0.457286432160804	0.130073571571152\\
-0.442211055276382	0.115553312379778\\
-0.42713567839196	0.0990726394180448\\
-0.412060301507538	0.0819790851628189\\
-0.396984924623116	0.0654682002244067\\
-0.381909547738693	0.0504586627332136\\
-0.366834170854271	0.0375335006100462\\
-0.351758793969849	0.0269451376333582\\
-0.336683417085427	0.0186689435016723\\
-0.321608040201005	0.0124835204393518\\
-0.306532663316583	0.00805624030634114\\
-0.291457286432161	0.00501771219792656\\
-0.276381909547739	0.00301617880345267\\
-0.261306532663317	0.00174979197149739\\
-0.246231155778894	0.000979701460978718\\
-0.231155778894472	0.00052939409408853\\
-0.21608040201005	0.000276084756470587\\
-0.201005025125628	0.000138958067324504\\
-0.185929648241206	6.74998957371878e-05\\
-0.170854271356784	3.16446603768558e-05\\
-0.155778894472362	1.43177830072836e-05\\
-0.14070351758794	6.25214603522081e-06\\
-0.125628140703518	2.63487763442609e-06\\
-0.110552763819096	1.07169150642828e-06\\
-0.0954773869346733	4.20685096491226e-07\\
-0.0804020100502513	1.59375831900113e-07\\
-0.0653266331658291	5.82727887989975e-08\\
-0.050251256281407	2.0563031523068e-08\\
-0.035175879396985	7.00303795033874e-09\\
-0.0201005025125628	2.30178003387978e-09\\
-0.00502512562814073	7.30161882950325e-10\\
0.0100502512562815	2.23538584508942e-10\\
0.0251256281407035	6.60486333507968e-11\\
0.0402010050251256	1.88344600914967e-11\\
0.0552763819095476	5.18346911064005e-12\\
0.0703517587939699	1.37678414045859e-12\\
0.085427135678392	3.52930502547627e-13\\
0.100502512562814	8.73153370218821e-14\\
0.115577889447236	2.08482628999997e-14\\
0.130653266331658	4.80426788284014e-15\\
0.14572864321608	1.06847054926449e-15\\
0.160804020100503	2.293379345176e-16\\
0.175879396984925	4.75080535998551e-17\\
0.190954773869347	9.49809652749516e-18\\
0.206030150753769	1.83266852111727e-18\\
0.221105527638191	3.41278761247874e-19\\
0.236180904522613	6.13355979363771e-20\\
0.251256281407035	1.06388318608236e-20\\
0.266331658291457	1.78095639878592e-21\\
0.28140703517588	2.87733633100887e-22\\
0.296482412060302	4.48648243476834e-23\\
0.311557788944724	6.75148511159821e-24\\
0.326633165829146	9.80552061285523e-25\\
0.341708542713568	1.37442168882428e-25\\
0.35678391959799	1.85929089646744e-26\\
0.371859296482412	2.42746337503807e-27\\
0.386934673366834	3.05869398817556e-28\\
0.402010050251256	3.71960971241608e-29\\
0.417085427135678	4.36552738374366e-30\\
0.432160804020101	4.94486068784572e-31\\
0.447236180904523	5.40566844359041e-32\\
0.462311557788945	5.70325446290064e-33\\
0.477386934673367	5.80729774725765e-34\\
0.492462311557789	5.70694172659544e-35\\
0.507537688442211	5.41266044031043e-36\\
0.522613065326633	4.95445773244763e-37\\
0.537688442211055	4.37682792651257e-38\\
0.552763819095477	3.73164926603188e-39\\
0.567839195979899	3.0705782090626e-40\\
0.582914572864322	2.43847051792633e-41\\
0.597989949748744	1.86892922367551e-42\\
0.613065326633166	1.38243971220511e-43\\
0.628140703517588	9.86909994390187e-45\\
0.64321608040201	6.79965523954378e-46\\
0.658291457286432	4.52141363728256e-47\\
0.673366834170854	2.90161365610459e-48\\
0.688442211055276	1.79714423209735e-49\\
0.703517587939698	1.07424732364763e-50\\
0.71859296482412	6.19731580354211e-52\\
0.733668341708543	3.45049160324826e-53\\
0.748743718592965	1.85411352709972e-54\\
0.763819095477387	9.6154512444845e-56\\
0.778894472361809	4.81261386557408e-57\\
0.793969849246231	2.32471846939545e-58\\
0.809045226130653	1.08377147018656e-59\\
0.824120603015075	4.87621743082316e-61\\
0.839195979899497	2.11741707602194e-62\\
0.85427135678392	8.87376166071854e-64\\
0.869346733668342	3.58911280870636e-65\\
0.884422110552764	1.40102039622597e-66\\
0.899497487437186	5.27812665085848e-68\\
0.914572864321608	1.91908034074213e-69\\
0.92964824120603	6.73417667380368e-71\\
0.944723618090452	2.28062486945852e-72\\
0.959798994974874	7.45420282973179e-74\\
0.974874371859296	2.35140022753671e-75\\
0.989949748743719	7.15862983603782e-77\\
1.00502512562814	2.10334859928141e-78\\
1.02010050251256	5.96445281528211e-80\\
1.03517587939698	1.63232998342091e-81\\
1.05025125628141	4.31144955379258e-83\\
1.06532663316583	1.099047999615e-84\\
1.08040201005025	2.70388372325828e-86\\
1.09547738693467	6.42003451535516e-88\\
1.1105527638191	1.47117628166674e-89\\
1.12562814070352	3.25364425381868e-91\\
1.14070351758794	6.94469886085328e-93\\
1.15577889447236	1.43058863505929e-94\\
1.17085427135678	2.84416064620836e-96\\
1.18592964824121	5.45722016773012e-98\\
1.20100502512563	1.01057099266459e-99\\
1.21608040201005	1.80609329772476e-101\\
1.23115577889447	3.1152401680726e-103\\
1.24623115577889	5.18586118295871e-105\\
1.26130653266332	8.33159672542081e-107\\
1.27638190954774	1.29185447563516e-108\\
1.29145728643216	1.93320071853718e-110\\
1.30653266331658	2.79201856539966e-112\\
1.32160804020101	3.89168477964236e-114\\
1.33668341708543	5.23522088366185e-116\\
1.35175879396985	6.79689192639872e-118\\
1.36683417085427	8.51655036505505e-120\\
1.38190954773869	1.0299000372269e-121\\
1.39698492462312	1.20199989235265e-123\\
1.41206030150754	1.35391619733106e-125\\
1.42713567839196	1.47182826596156e-127\\
1.44221105527638	1.54418914276882e-129\\
1.4572864321608	1.56358627726146e-131\\
1.47236180904523	1.5279924417714e-133\\
1.48743718592965	1.44111474570082e-135\\
1.50251256281407	1.31175859381067e-137\\
1.51758793969849	1.15235761060435e-139\\
1.53266331658291	9.7700914965437e-142\\
1.54773869346734	7.99443889830478e-144\\
1.56281407035176	6.31328426274197e-146\\
1.57788944723618	4.81172392320207e-148\\
1.5929648241206	3.53935445921421e-150\\
1.60804020100502	2.51261192063529e-152\\
1.62311557788945	1.72149106676805e-154\\
1.63819095477387	1.13831414114531e-156\\
1.65326633165829	7.26436218364783e-159\\
1.66834170854271	4.47415316105865e-161\\
1.68341708542714	2.65951353658765e-163\\
1.69849246231156	1.52570849216465e-165\\
1.71356783919598	8.44731952777952e-168\\
1.7286432160804	4.51382046057725e-170\\
1.74371859296482	2.32781059049058e-172\\
1.75879396984925	1.15858792732176e-174\\
1.77386934673367	5.56529715339483e-177\\
1.78894472361809	2.58003571227894e-179\\
1.80402010050251	1.15435948830055e-181\\
1.81909547738693	4.98464732871307e-184\\
1.83417085427136	2.0773314522254e-186\\
1.84924623115578	8.3551674220519e-189\\
1.8643216080402	3.24326560997671e-191\\
1.87939698492462	1.21503245181336e-193\\
1.89447236180905	4.39310150611083e-196\\
1.90954773869347	1.53296626912309e-198\\
1.92462311557789	5.16264140113166e-201\\
1.93969849246231	1.67798981710585e-203\\
1.95477386934673	5.26362186473754e-206\\
1.96984924623116	1.59352192693446e-208\\
1.98492462311558	4.65596176719113e-211\\
2	1.31292152108604e-213\\
};
\addplot [color=black, line width=0.8pt, forget plot]
  table[row sep=crcr]{%
-1	1.31292152108604e-213\\
-0.984924623115578	4.6559617671906e-211\\
-0.969849246231156	1.59352192693446e-208\\
-0.954773869346734	5.26362186473754e-206\\
-0.939698492462312	1.67798981710585e-203\\
-0.924623115577889	5.16264140113166e-201\\
-0.909547738693467	1.53296626912309e-198\\
-0.894472361809045	4.39310150611083e-196\\
-0.879396984924623	1.21503245181336e-193\\
-0.864321608040201	3.24326560997707e-191\\
-0.849246231155779	8.3551674220519e-189\\
-0.834170854271357	2.07733145222505e-186\\
-0.819095477386935	4.98464732871307e-184\\
-0.804020100502513	1.15435948830055e-181\\
-0.78894472361809	2.58003571227894e-179\\
-0.773869346733668	5.56529715339483e-177\\
-0.758793969849246	1.15858792732176e-174\\
-0.743718592964824	2.32781059049058e-172\\
-0.728643216080402	4.51382046057725e-170\\
-0.71356783919598	8.44731952778096e-168\\
-0.698492462311558	1.52570849216465e-165\\
-0.683417085427136	2.65951353658735e-163\\
-0.668341708542714	4.47415316105865e-161\\
-0.653266331658291	7.26436218364783e-159\\
-0.638190954773869	1.13831414114531e-156\\
-0.623115577889447	1.72149106676805e-154\\
-0.608040201005025	2.51261192063529e-152\\
-0.592964824120603	3.53935445921421e-150\\
-0.577889447236181	4.81172392320207e-148\\
-0.562814070351759	6.31328426274341e-146\\
-0.547738693467337	7.99443889830478e-144\\
-0.532663316582915	9.7700914965437e-142\\
-0.517587939698492	1.15235761060435e-139\\
-0.50251256281407	1.31175859381067e-137\\
-0.487437185929648	1.44111474570074e-135\\
-0.472361809045226	1.5279924417714e-133\\
-0.457286432160804	1.56358627726146e-131\\
-0.442211055276382	1.54418914276882e-129\\
-0.42713567839196	1.47182826596156e-127\\
-0.412060301507538	1.35391619733098e-125\\
-0.396984924623116	1.20199989235265e-123\\
-0.381909547738693	1.0299000372269e-121\\
-0.366834170854271	8.51655036505505e-120\\
-0.351758793969849	6.79689192639872e-118\\
-0.336683417085427	5.23522088366126e-116\\
-0.321608040201005	3.89168477964236e-114\\
-0.306532663316583	2.79201856539966e-112\\
-0.291457286432161	1.93320071853718e-110\\
-0.276381909547739	1.29185447563516e-108\\
-0.261306532663317	8.3315967254201e-107\\
-0.246231155778894	5.18586118295871e-105\\
-0.231155778894472	3.1152401680726e-103\\
-0.21608040201005	1.80609329772476e-101\\
-0.201005025125628	1.01057099266459e-99\\
-0.185929648241206	5.45722016772981e-98\\
-0.170854271356784	2.84416064620836e-96\\
-0.155778894472362	1.43058863505929e-94\\
-0.14070351758794	6.94469886085328e-93\\
-0.125628140703518	3.25364425381868e-91\\
-0.110552763819096	1.47117628166665e-89\\
-0.0954773869346733	6.42003451535516e-88\\
-0.0804020100502513	2.70388372325828e-86\\
-0.0653266331658291	1.09904799961506e-84\\
-0.050251256281407	4.31144955379258e-83\\
-0.035175879396985	1.63232998342081e-81\\
-0.0201005025125628	5.96445281528211e-80\\
-0.00502512562814073	2.10334859928141e-78\\
0.0100502512562815	7.15862983603782e-77\\
0.0251256281407035	2.35140022753671e-75\\
0.0402010050251256	7.45420282973179e-74\\
0.0552763819095476	2.28062486945852e-72\\
0.0703517587939699	6.73417667380368e-71\\
0.085427135678392	1.91908034074213e-69\\
0.100502512562814	5.27812665085848e-68\\
0.115577889447236	1.40102039622597e-66\\
0.130653266331658	3.58911280870636e-65\\
0.14572864321608	8.87376166071854e-64\\
0.160804020100503	2.11741707602194e-62\\
0.175879396984925	4.87621743082316e-61\\
0.190954773869347	1.08377147018656e-59\\
0.206030150753769	2.32471846939545e-58\\
0.221105527638191	4.81261386557408e-57\\
0.236180904522613	9.6154512444845e-56\\
0.251256281407035	1.85411352709972e-54\\
0.266331658291457	3.45049160324826e-53\\
0.28140703517588	6.19731580354211e-52\\
0.296482412060302	1.07424732364763e-50\\
0.311557788944724	1.79714423209735e-49\\
0.326633165829146	2.90161365610459e-48\\
0.341708542713568	4.52141363728256e-47\\
0.35678391959799	6.79965523954378e-46\\
0.371859296482412	9.86909994390187e-45\\
0.386934673366834	1.38243971220511e-43\\
0.402010050251256	1.86892922367551e-42\\
0.417085427135678	2.43847051792633e-41\\
0.432160804020101	3.0705782090626e-40\\
0.447236180904523	3.73164926603188e-39\\
0.462311557788945	4.37682792651257e-38\\
0.477386934673367	4.95445773244763e-37\\
0.492462311557789	5.41266044031043e-36\\
0.507537688442211	5.70694172659544e-35\\
0.522613065326633	5.80729774725765e-34\\
0.537688442211055	5.70325446290064e-33\\
0.552763819095477	5.40566844359041e-32\\
0.567839195979899	4.94486068784572e-31\\
0.582914572864322	4.36552738374366e-30\\
0.597989949748744	3.71960971241608e-29\\
0.613065326633166	3.05869398817556e-28\\
0.628140703517588	2.42746337503807e-27\\
0.64321608040201	1.85929089646744e-26\\
0.658291457286432	1.37442168882428e-25\\
0.673366834170854	9.80552061285523e-25\\
0.688442211055276	6.75148511159821e-24\\
0.703517587939698	4.48648243476834e-23\\
0.71859296482412	2.87733633100887e-22\\
0.733668341708543	1.78095639878592e-21\\
0.748743718592965	1.06388318608236e-20\\
0.763819095477387	6.13355979363771e-20\\
0.778894472361809	3.41278761247874e-19\\
0.793969849246231	1.83266852111727e-18\\
0.809045226130653	9.49809652749516e-18\\
0.824120603015075	4.75080535998551e-17\\
0.839195979899497	2.293379345176e-16\\
0.85427135678392	1.06847054926449e-15\\
0.869346733668342	4.80426788284014e-15\\
0.884422110552764	2.08482628999997e-14\\
0.899497487437186	8.73153370218821e-14\\
0.914572864321608	3.52930502547627e-13\\
0.92964824120603	1.37678414045859e-12\\
0.944723618090452	5.18346911064005e-12\\
0.959798994974874	1.88344600914967e-11\\
0.974874371859296	6.60486333507968e-11\\
0.989949748743719	2.23538584508942e-10\\
1.00502512562814	7.3016188295032e-10\\
1.02010050251256	2.30178003387981e-09\\
1.03517587939698	7.00303795033864e-09\\
1.05025125628141	2.0563031523068e-08\\
1.06532663316583	5.82727887989985e-08\\
1.08040201005025	1.59375831900111e-07\\
1.09547738693467	4.20685096491229e-07\\
1.1105527638191	1.07169150642826e-06\\
1.12562814070352	2.63487763442609e-06\\
1.14070351758794	6.25214603522084e-06\\
1.15577889447236	1.43177830072835e-05\\
1.17085427135678	3.16446603768559e-05\\
1.18592964824121	6.74998957371871e-05\\
1.20100502512563	0.000138958067324504\\
1.21608040201005	0.000276084756470588\\
1.23115577889447	0.000529394094088528\\
1.24623115577889	0.000979701460978723\\
1.26130653266332	0.00174979197149738\\
1.27638190954774	0.00301617880345267\\
1.29145728643216	0.00501771219792657\\
1.30653266331658	0.00805624030634112\\
1.32160804020101	0.0124835204393518\\
1.33668341708543	0.0186689435016722\\
1.35175879396985	0.0269451376333582\\
1.36683417085427	0.0375335006100463\\
1.38190954773869	0.0504586627332135\\
1.39698492462312	0.0654682002244067\\
1.41206030150754	0.0819790851628187\\
1.42713567839196	0.0990726394180448\\
1.44221105527638	0.115553312379779\\
1.4572864321608	0.130073571571152\\
1.47236180904523	0.141310271747869\\
1.48743718592965	0.148161852006634\\
1.50251256281407	0.149926038101147\\
1.51758793969849	0.146418425165253\\
1.53266331658291	0.138004230025996\\
1.54773869346734	0.125535647068942\\
1.56281407035176	0.11020967781249\\
1.57788944723618	0.0933792540392318\\
1.5929648241206	0.0763587952830379\\
1.60804020100502	0.060262304517522\\
1.62311557788945	0.0458997588353218\\
1.63819095477387	0.0337406210220736\\
1.65326633165829	0.0239372215927844\\
1.66834170854271	0.0163897530021009\\
1.68341708542714	0.0108305142315073\\
1.69849246231156	0.00690722727455912\\
1.71356783919598	0.00425144353314526\\
1.7286432160804	0.00252549832279041\\
1.74371859296482	0.00144789061734106\\
1.75879396984925	0.000801128945257349\\
1.77386934673367	0.000427806218451707\\
1.78894472361809	0.000220480285668281\\
1.80402010050251	0.000109665603513467\\
1.81909547738693	5.26440270021223e-05\\
1.83417085427136	2.43896615423926e-05\\
1.84924623115578	1.09053709836089e-05\\
1.8643216080402	4.70601291743597e-06\\
1.87939698492462	1.95994455544761e-06\\
1.89447236180905	7.8779361566373e-07\\
1.90954773869347	3.05604066037438e-07\\
1.92462311557789	1.14415219901712e-07\\
1.93969849246231	4.13415229494012e-08\\
1.95477386934673	1.44167425497682e-08\\
1.96984924623116	4.85205611542633e-09\\
1.98492462311558	1.57602276748067e-09\\
2	4.94057116545912e-10\\
};
\node[centered, align=center, inner sep=0, font=\color{mycolor1}]
at (axis cs:0.5,1.3) {Abstand zu groß};
\end{axis}
\end{tikzpicture}%
    \caption{Einfluss der Schrittweite auf die Approximation des Effekts $\sym{Eff}$}
    \label{fig:ma_abb2.06_effektabstand}
\end{figure}
Auf Basis eines solchen zweistufigen Setups lässt sich der Zusammenhang zwischen Einflussgrößen und Lebensdauer interpolieren und in einem linearen Modell abbilden, welches zudem die Schätzung von Wechselwirkungen erlaubt.
Werden davon abweichende Modellterme zur Abbildung der Systemantwort erwartet, berücksichtigen alternative Versuchspläne typischerweise drei bis fünf Faktorstufen.
Der voll-faktorielle Versuchsplan nimmt dabei eine entscheidende Schlüsselrolle in der strategischen Modellbildung ein - insbesondere im Hinblick auf Lebensdauerdaten und Zuverlässigkeitstechnik.\
Darin aufgeführte Versuchspunkte können auf Basis der Struktur ihrer Zuordnung üblicherweise ideal aus Voruntersuchungen übernommen oder durch weiterführende Untersuchungen nachfolgend erweitert werden.
Ausgehend vom qualitativen Parameter-Screening (vgl. Abschnitt~\ref{subsec:begriffedoe}) ist ohne initiale Experimente oft unklar, welche Faktoren die Antwortvariable, also beispielsweise die Lebensdauer eines Systems, nun tatsächlich signifikant beeinflussen.
Folglich ist es essenziell, diese Fragestellung vor der eigentlichen Versuchsplanumsetzung effizient zu klären.

\subsubsection{$2^{\sym{k}-\sym{p}}$ Fraktionell Faktorielle Versuchspläne}
Sind nach Anwendung der Kreativmethoden (qualitatives Screening) weiterhin so viele Einflussfaktoren als relevant eingestuft, dass ein voll-faktorieller Ansatz gemäß Gleichung~\ref{eq:ffvp_n} zu einem wirtschaftlich nicht vertretbaren Versuchsumfang führen würde, muss die Strategie hin zu physikalischen Screening-Tests verschärft werden.
Dies empfiehlt sich insbesondere für Systeme mit $\sym{k} > 5$ Faktoren, um die experimentelle Effizienz zu gewährleisten.
Zur Veranschaulichung der Notwendigkeit: Bereits eine einzelne Replikation eines voll-faktoriellen Experiments mit $\sym{k}=8$ Faktoren würde $2^8 = 256$ Versuchsdurchläufe erfordern, was in der Lebensdauererprobung meist illusorisch ist.

Für derartige Selektionsaufgaben eignen sich daher \textbf{Screening-Versuchspläne}, wie der \textbf{teil-faktorielle Versuchsplan} (Fractional Factorial Design) oder alternativ der \textbf{Plackett-Burman-Plan} \cite{Kleppmann.2016,Siebertz.2017,Montgomery.2020}.
Bei diesem Ansatz wird lediglich eine selektive Teilmenge (Fraktion) der voll-faktoriellen Versuchsagenda umgesetzt, um mit minimalem Informationsverlust die für den Anwendungsfall signifikanten Effekte zu beschreiben.
Mathematisch wird die Anzahl der Versuche dabei auf
\begin{equation} \label{eq:teilfakt_n}
    \sym{n} = 2^{\sym{k}-\sym{p}}
\end{equation}
reduziert, wobei $\sym{p}$ den Grad der Fraktionierung (die Anzahl der Generatoren) angibt.
Die Validität dieses Vorgehens stützt sich auf zwei fundamentale empirische Postulate \cite{Montgomery.2021,Rigdon.2022}:
\begin{itemize}
    \item Die \textbf{Effekthierarchie} besagt, dass Effekte niedrigerer Ordnung -- primär Haupteffekte -- in der Regel eine größere Amplitude aufweisen und mit höherer Wahrscheinlichkeit signifikant sind als Effekte höherer Ordnung.
    \item Die \textbf{Effektvererbung} impliziert, dass das Auftreten signifikanter Wechselwirkungen oder quadratischer Terme strukturell an die Signifikanz ihrer korrespondierenden Haupteffekte gekoppelt ist.
\end{itemize}
Eine direkte Konsequenz dieser Reduktion („Der Preis der Einsparung“) ist jedoch, dass sich in teil-faktoriellen Versuchsplänen bestimmte Effekte nicht mehr isoliert betrachten lassen (vgl. Abbildung~\ref{fig:abb2.05.1_vfdesign_plot} für den Fall einer Fraktionierung).
So sind beispielsweise Haupteffekte unter Umständen nicht mehr zweifelsfrei von Wechselwirkungen höherer Ordnung zu unterscheiden. Da sich diese Effekte statistisch überlagern, spricht man von einer \textbf{Vermengung} (engl. \textbf{Aliasing}).
Die Schwere dieser Vermengung wird dabei über die \textbf{Auflösung} (engl. Resolution) des Versuchsplans klassifiziert (z.\,B. Auflösung III, IV oder V).
Wird dieser Informationsverlust jedoch bewusst in Kauf genommen und ingenieurwissenschaftlich bewertet, ermöglicht dies eine signifikante Reduktion des Versuchsumfangs, um effizient die dominanten Faktoren aus der initialen Parametermenge zu isolieren.
Diese Vorgehensweise ist von hoher Relevanz, da Lebensdauertests -- ob als beschleunigte Prüfung mittels \ac{ALT} oder unter Feldbedingungen -- durch die inhärente Zeitabhängigkeit der Systemantwort erhebliche Kapazitäten binden und Ergebnisse nicht ad hoc verfügbar sind.
Im Sinne einer ressourceneffizienten Gesamtstrategie sollten die Screening-Versuche daher idealerweise so konzipiert sein, dass sie nahtlos in einen nachfolgenden, höher aufgelösten Versuchsplan integriert (\textbf{augmentiert}) werden können.

% \subsubsection{Wirkungsflächenversuchspläne}
% Perspektivisch ist zudem entscheidend, wie diese Datenbasis weitergenutzt werden kann, falls resultierende lineare Modelle im weiteren Verlauf z. B. zur Abbildung von \textbf{Krümmungen} - engl. \textbf{Curvature} detailliert werden müssen.
% Dies kann eben durch die Integration weiterer Versuchspunkte erfolgen, was erlaubt, nicht-lineare Zusammenhänge in der Systemantwort abzubilden.
% Die \ac{RSM} behandelt als Teildisziplin des \ac{DoE} derartige Herangehensweisen und bietet Wirkungsflächenversuchspläne, engl. \acp{RSD}, an.
% Um zu entscheiden, inwiefern Krümmungen vorliegen und damit die Systemantwort von einer linearen Modellierung abweicht, können sogenannte Zentralpunkte, engl. \acp{CP}, zusätzlich getestet werden.
% Hier werden alle Faktoren auf die Stufe \textit{0} gesetzt wobei im Fall diskreter Faktorstufen alle kategorialen Einstellungen abgetestet werden
% Diese geben schließlich Aufschluss über Diskrepanzen zu linearen Zusammenhängen zwischen Faktor und Antwort.


% \textbf{Box-Behnken-Design}
% \textbf{Sternpunkten} \textbf{Axial Points} \ac{CCD} \textbf{Rotierbarkeit}


\subsubsection{Wirkungsflächenversuchspläne} \label{subsubsec:rsm}
Die bisher diskutierten Versuchspläne beschränken sich auf die Untersuchung von Faktoren auf jeweils zwei Stufen ($\pm 1$). Dies ermöglicht zwar eine effiziente Darstellung linearer Beziehungen und Interaktionen, jedoch ist die Modellierung komplexerer, nicht-linearer Effekte aufgrund fehlender Stützstellen im Versuchsraum damit physikalisch nicht möglich.
Perspektivisch ist daher entscheidend, wie die bestehende Datenbasis weitergenutzt und augmentiert werden kann, falls die Systemantwort signifikante \textbf{Krümmungen} (engl. \textbf{Curvature}) aufweist und die quantitativen Beziehungen zwischen Faktoren und Zielgröße für eine Optimierung detaillierter beschrieben werden müssen.

Die \ac{RSM} behandelt als Teildisziplin des \ac{DoE} derartige Herangehensweisen und bietet hierfür spezielle \textbf{Wirkungsflächenversuchspläne} - engl. \acp{RSD} - an.
Der erste Schritt zur Detektion von Nichtlinearitäten besteht in der Integration von $\sym{n_center}$ sogenannten \textbf{Zentralpunkten}, engl. \acp{CP}, in den faktoriellen Basisplan. Hierbei werden alle Faktoren auf die kodierte Stufe \textit{0} (die Mitte des Versuchsraums) gesetzt.
Weicht der Mittelwert der Systemantwort in den Zentralpunkten signifikant vom Mittelwert über die faktoriellen Eckpunkte ab, deutet dies auf eine Krümmung der Antwortfläche hin \cite{Montgomery.2020}.

Um diese quadratischen Zusammenhänge explizit zu bestimmen, gilt der \textbf{Zentral-Zusammengesetzte-Versuchsplan}, engl. \ac{CCD}, als etablierter Standard.
Ein \ac{CCD} entsteht durch die Augmentierung der $\sym{n_fact}$ Eckpunkte des ursprünglichen voll- (oder teil-)faktoriellen Plans um:
\begin{itemize}
    \item eine definierte Anzahl $\sym{n_center}$ an wiederholten Zentralpunkten (üblicherweise $3 \leq \sym{n_center} \leq 5$ zur Abschätzung des reinen Fehlers) sowie
    \item $\sym{n_star} = 2\sym{k}$ zusätzliche \textbf{Sternpunkte} (engl. Axial Points), die auf den Achsen des Koordinatensystems im Abstand $\pm \sym{alpha_rot}$ vom Zentrum liegen.
\end{itemize}

Während die Zentralpunkte das Vorhandensein quadratischer Effekte validieren, ermöglichen die Sternpunkte deren genaue geometrische Bestimmung.
Ein wesentliches Qualitätsmerkmal solcher Pläne ist die \textbf{Drehbarkeit} (engl. Rotatability), welche sicherstellt, dass die Varianz der Vorhersage nur vom Abstand zum Zentrum abhängt und nicht von der Richtung.
Um diese Eigenschaft zu gewährleisten, orientiert sich der Abstand $\sym{alpha_rot}$ der Sternpunkte vom Zentrum an der Anzahl der faktoriellen Versuchspunkte $\sym{n_fact}$ \cite{Kleppmann.2016,Montgomery.2020}:
\begin{equation} \label{eq:ccd_alpha}
    \sym{alpha_rot} = (\sym{n_fact})^{1/4}.
\end{equation}
Abwandlungen dieses Konzepts, wie das \textbf{Box-Behnken-Design} (welches extreme Ecken vermeidet) oder flächenzentrierte Pläne (engl. Face-Centered CCD mit $\sym{alpha_rot}=1$), erfüllen spezifische Randbedingungen hinsichtlich des Versuchsaufwands und der physikalischen Machbarkeit extremer Einstellungen \cite{Myers.2016}.

\subsubsection{Strategische Vorgehensweisen}

\begin{itemize}

    \item CCD mit \cite{Box.1988}

\end{itemize}



%%%%%%%%%%%%%%%%%%%%%%%%%%%%%%%%%%%%%%%%%%%%%%%%%%%%%%%%%%%%%%%%%%%%%%%%%%%%%%%%%%%%%%%%%%%%%%%%%%%%%%%%%%%%%%%%%%%%%%%%%%
%%%%%%%%%%%%%%%%%%%%%%%%%%%%%%%%%%%%%%%%%%%%%%%%%%%%%%%%%%%%%%%%%%%%%%%%%%%%%%%%%%%%%%%%%%%%%%%%%%%%%%%%%%%%%%%%%%%%%%%%%%
%%%%%%%%%%%%%%%%%%%%%%%%%%%%%%%%%%%%%%%%%%%%%%%%%%%%%%%%%%%%%%%%%%%%%%%%%%%%%%%%%%%%%%%%%%%%%%%%%%%%%%%%%%%%%%%%%%%%%%%%%%
%%%%%%%%%%%%%%%%%%%%%%%%%%%%%%%%%%%%%%%%%%%%%%%%%%%%%%%%%%%%%%%%%%%%%%%%%%%%%%%%%%%%%%%%%%%%%%%%%%%%%%%%%%%%%%%%%%%%%%%%%%
\subsection{Statistische Modellbildung} \label{subsec:model}

\begin{itemize}
    \item GLL-Weibull
    \item LR-Ratio
    \item p-Werte
    \item Trennschärfe etwa $80\%$, Effekt $\sym{Eff}>2\sigma$ \cite[S.116]{Rigdon.2022}
    \item Varianzen
    \item Modellaufbau
    \item Extrapolation: bedarf in der Praxis / oder sind doch dann mal VPs verfügbar und das modell wird einfach erweitert
    \item Optimalitäten
    \item Risiduen (Cox-Snell \cite{Rigdon.2022})
\end{itemize}
\begin{itemize}
    \item \colorbox{yellow}{wu}
    \item \colorbox{yellow}{wütherich}
    \item \colorbox{yellow}{yang p282 erster abschnitt}
    \item \colorbox{yellow}{russell doe for glm kap1.4 p12}
\end{itemize}
\subsubsection{Metriken zur optimalen Versuchsplanung}

Ein Versuchsplan wird hier als \textbf{orthogonal} bezeichnet, wenn keine Korrelation zwischen jeweils zwei Spalten der Matrix vorliegt - deren Skalarprodukt also null ergibt.
So können Effekte eindeutig identifiziert werden.
\textbf{Ausgewogenheit} liegt vor, sofern für einen jeweiligen Faktor alle anderen Faktoreinstellungen gleichmäßig aufgeteilt sind \cite{Siebertz.2017}.



\cite{RisbergEllekjr.1998, Bisgaard.1992,Bisgaard.2011,Myers.2016,Montgomery.2020,Zahran.2003,GiovannittiJensen.1989,Goos.2011,Hinkelmann.2012,Khuri.2006,Rittmaier.2025,Johnson.2011,Donev.2004,G.E.P.Box.1951,Ardakani.2011,Jones.2012,Jones.2021,Rigdon.2022,Wu.2021,Khuri.2006,Escobar.1995,Modarres.2017,Ahn.2015,Rasch.2018,Xu.2002,Wald.1943,Rencher.2008,Box.2007,Kleppmann.2016,Siebertz.2017,Fisher.1935,Bisgaard.1997,Myers.2010,Box.1988}