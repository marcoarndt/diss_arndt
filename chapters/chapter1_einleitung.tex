%%%%%%%%%%%%%%%%%%%%%%%%%%%%%%%%%%%%%%%%%%%%%%%%%%%%%%%%%%%%%%%%%%%%%%%%%%%%%%%%%%%%%%%%%%%%%%%%%%%%%%%%%%%%%%%%%
%% Kapitel 1 - Einleitung %%%%%%%%%%%%%%%%%%%%%%%%%%%%%%%%%%%%%%%%%%%%%%%%%%%%%%%%%%%%%%%%%%%%%%%%%%%%%%%%%%%%%%%
%%%%%%%%%%%%%%%%%%%%%%%%%%%%%%%%%%%%%%%%%%%%%%%%%%%%%%%%%%%%%%%%%%%%%%%%%%%%%%%%%%%%%%%%%%%%%%%%%%%%%%%%%%%%%%%%%

\chapter{Einleitung}
Die Fähigkeit, die Zuverlässigkeit technischer Systeme und ingenieurwissenschaftlicher Erzeugnisse präzise vorherzusagen, ist vermehrt von zentraler, unternehmerischer Bedeutung.
Prognosen über die Lebensdauer und Ausfallwahrscheinlichkeit rücken damit in den Fokus strategischer Entscheidungen – von der Auslegung sicherheitskritischer Komponenten bis hin zur wirtschaftlichen Optimierung von Wartungsintervallen. Dabei ist die Modellbildung umso komplexer, je mehr Einflussgrößen simultan zu berücksichtigen sind. Sie übernimmt sogar eine zunehmend dominierende Rolle: Elektrifizierung, Gewichts- und Performance-Optimierung durch neuartige Material- und Strukturkonglomerate sowie Digitalisierung lassen vielfältige Expositionen gegenüber Beanspruchungen und Ausfallmechanismen realisieren, deren strukturelles Verhalten nicht ohne weiteres modellierbar ist. Genau hier setzt die multivariate Zuverlässigkeitsmodellierung an: sie erlaubt nicht nur eine realitätsnahe Abbildung des Systemverhaltens, sondern schafft die Voraussetzung für belastbare Aussagen über die Wirkung kombinierter Belastungsfaktoren gemäß klassischer Zuverlässigkeitsmethodik. Besonders in frühen Entwicklungsphasen, in denen physikalische Ausfallmechanismen (Physics of Failure, PoF) noch nicht vollständig verstanden sind, wird ein systematischer experimenteller Ansatz unverzichtbar.  Um dennoch belastbare Modelle zu erzeugen oder Nachweise zu erbringen, kommen beschleunigte Lebensdauertests (Accelerated Life Testing, ALT) zum Einsatz. Dabei werden gezielt erhöhte Belastungsniveaus appliziert, um Versagensprozesse zu initiieren, ohne den eigentlichen Ausfallmechanismus zu verfälschen. Die resultierenden Daten dienen der Generierung von Lebensdauermodellen, häufig mittels verallgemeinerter log-linearer Regressionsmodelle (GLMs)..

Da in der Regel mehrere kontinuierliche Einflussgrößen gleichzeitig auf das Versagensverhalten wirken, ist die Entwicklung eines statistisch fundierten Versuchsplans essenziell. Die Response Surface Methodology (RSM) bietet hierfür ein leistungsfähiges Instrumentarium, insbesondere durch die Anwendung zentral zusammengesetzter Versuchspläne wie dem Central Composite Design (CCD). Diese Designs ermöglichen durch ihre orthogonale und rotierbare Struktur eine unabhängige Schätzung linearer, quadratischer und interaktiver Effekte – ein entscheidender Vorteil für die Modellierung komplexer Systeme, deren Parameterraum zunächst nur unzureichend bekannt ist.

CCD zeichnet sich insbesondere durch seine Fähigkeit aus, gleichmäßige Vorhersagegenauigkeiten radial zum Zentrum des Versuchsraumes zu gewährleisten. Dies prädestiniert den Ansatz für explorative Teststrategien in Fällen mit hohem Unsicherheitsgrad, etwa bei neuartigen Technologien oder unvollständig erforschten Werkstoffverhalten. Im Vergleich zu anderen optimalen Designs, die spezifisches Vorwissen über relevante Einflussgrößen erfordern, liefert CCD eine robuste Ausgangsbasis für die strukturierte Erfassung von Wechselwirkungen und nichtlinearen Zusammenhängen. Dies ist von unschätzbarem Wert für Produkte, deren physikalische Ausfallmodelle noch nicht vollständig formuliert sind.

Die konsequente Anwendung dieser Methodik erfordert jedoch auch eine Erweiterung klassischer Modellierungsansätze. Insbesondere bei klinischen oder biomedizinischen Anwendungen – zunehmend aber auch bei technischen Systemen mit diskreten oder korrelierten Antwortgrößen – stoßen klassische lineare Modelle an ihre Grenzen. Hier bietet die Klasse der verallgemeinerten linearen Modelle (GLMs) ein erweitertes Rahmenwerk, das sowohl binäre als auch Poisson-verteilte oder andere nicht-normalverteilte Zielgrößen adäquat abbilden kann. Allerdings stellt die Abhängigkeit optimaler Versuchspläne von a priori unbekannten Modellparametern eine Herausforderung dar, die bislang in der Literatur nur begrenzt adressiert wurde.

Diese Arbeit entwickelt vor diesem Hintergrund ein umfassendes Modellkonzept, das die Prinzipien der Versuchsplanung mit der Analyse ökonomischer und qualitativer Zielgrößen in der Modellbildung verknüpft. Ziel ist es, eine strukturierte und effiziente Methodik zur Durchführung multivariater Lebensdauertests zu etablieren – als Grundlage für moderne, belastbare und praxisgerechte Zuverlässigkeitsprognosen.



Die Untersuchung von Zuverlässigkeit in der Ingenieurwissenschaft und der Produktentwicklung stellt eine essenzielle Herausforderung dar, insbesondere da zuverlässige technische Systeme und Produkte unter normalen Einsatzbedingungen Lebensdauern aufweisen können, die für Testzwecke unpraktisch sind. Um die Grundlage für Lebensdauermodelle zu schaffen oder die Zuverlässigkeit nachzuweisen, werden häufig beschleunigte Lebensdauertests (Accelerated Life Testing, ALT) durchgeführt.
Dabei werden Belastungsniveaus erhöht, um Versagensmechanismen in angemessener Zeit auszulösen, ohne dabei den Versagensmodus am Ende der Lebensdauer (End-of-Life, EoL) zu verändern. Anhand der Testergebnisse wird ein Regressionsmodell, häufig unter Verwendung verallgemeinerter log-linearer Modelle (GLM) und der Maximum-Likelihood-Methode (MLE), entwickelt, um ein Lebensdauer-Belastungs-Modell auf Basis empirischer Daten abzuleiten.
Zahlreiche Studien zeigen, dass die Modellierung der Lebensdauer häufig von mehreren kontinuierlichen Einflussfaktoren abhängt, was ein statistisches Versuchsdesign unerlässlich macht.
Die Response Surface Methodology (RSM) bietet dabei entscheidende Vorteile für experimentelle Designs im Zuverlässigkeitsengineering. Insbesondere factorial designs wie das Central Composite Design (CCD) sind aufgrund ihrer orthogonalen und rotierbaren Struktur besonders geeignet.
CCD ermöglicht die unabhängige Schätzung von Effekten und Wechselwirkungen zwischen mehreren Faktoren, was eine zentrale Voraussetzung für die Testung industrieller Anwendungen darstellt.
Die resultierenden Antwortflächen liefern zudem konstante Vorhersagegüten oder Vorhersagevarianzen, die radial vom Zentrum des experimentellen Designs ausgehen.
Dies ist ideal für die Entwicklung eines funktionsfähigen Zuverlässigkeitsmodells, insbesondere wenn der ideale Parameterraum für die Testung einer Technologie zu Beginn nicht bekannt ist – eine häufige Situation in der Praxis. CCD ordnet die Testpunkte auf effiziente Weise an und ermöglicht die Erfassung von Wechselwirkungen und quadratischen Effekten.
Dies ist besonders vorteilhaft bei technischen Produkten, bei denen die zugrunde liegenden physikalischen Versagensmechanismen (Physics of Failure, PoF) noch nicht vollständig verstanden sind.
Während alternative optimale Designs auf spezifische Leistungsanforderungen zugeschnitten sind, bietet CCD eine robuste Grundlage für Szenarien ohne Vorwissen.
Dies macht CCD zu einem wesentlichen Werkzeug für die Untersuchung von technischen Produkten, deren optimale Testbedingungen oder Zuverlässigkeitsmodelle erst entwickelt werden müssen.
Abschließend wird in dieser Arbeit ein umfassendes Modell vorgestellt, das die Prinzipien der Versuchsplanung mit der Analyse von Kosten und Qualität in der Modellbildung verbindet. Ziel ist es, eine methodische Grundlage für Zuverlässigkeitstests zu schaffen, die sowohl technische als auch ökonomische Anforderungen erfüllt.

\section{Forschungperspektive und Problembeschreibung}
\begin{itemize}
    \item {Ausgangssituation und Problemstellung}
\end{itemize}

\section{Beitrag dieser Arbeit}
\begin{itemize}
    \item {Ziele}
\end{itemize}

\section{Aufbau der Arbeit}
\begin{itemize}
    \item {Kapitelzusammensetzung und Struktur etc}
\end{itemize}