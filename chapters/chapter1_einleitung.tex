%%%%%%%%%%%%%%%%%%%%%%%%%%%%%%%%%%%%%%%%%%%%%%%%%%%%%%%%%%%%%%%%%%%%%%%%%%%%%%%%%%%%%%%%%%%%%%%%%%%%%%%%%%%%%%%%%
%% Kapitel 1 - Einleitung %%%%%%%%%%%%%%%%%%%%%%%%%%%%%%%%%%%%%%%%%%%%%%%%%%%%%%%%%%%%%%%%%%%%%%%%%%%%%%%%%%%%%%%
%%%%%%%%%%%%%%%%%%%%%%%%%%%%%%%%%%%%%%%%%%%%%%%%%%%%%%%%%%%%%%%%%%%%%%%%%%%%%%%%%%%%%%%%%%%%%%%%%%%%%%%%%%%%%%%%%

\chapter{Einleitung}
Die Absicherung technischer Produkte und Systeme hinsichtlich ihrer Funktionalität bildet einen zentralen Bestandteil ingenieurwissenschaftlicher Bestreben im Produktentwicklungsprozess.
Sie erfolgt als Erfüllung von Produktversprechen gegenüber der potenziellen Käuferschaft sowie bestehender Kunden, der Wahrung eines internern Selbstverständnisses der Marke sowie rein regulatorischer Vorgaben.
Über den gesamten Produktlebenszyklus hinweg unterstützen hier die Methoden der Zuverlässigkeitstechnik dabei, diese Anforderungen systematisch zu erfüllen.
Hierzu zählen Maßnahmen im Bereich Safety, explorative Datenanalysen zur Untersuchung der Produktperformance im Betrieb oder Test, effiziente Versuchsplanung zum Lebensdauernachweis, beschleunigte Prüfungsmethoden und der Aufbau probabilistischer Lebensdauermodelle sowie umfassendes Risikomanagement.
Das zentrale Ergebnis besteht in der Ermittlung der Ausfallwahrscheinlichkeit als Komplement zur Zuverlässigkeit: der Wahrscheinlichkeit, dass ein Produkt unter definierten Randbedingungen eine vorgegebene Zeitdauer ohne funktionskritischen Ausfall übersteht.
Gelingt keine hinreichend genaue Quantifizierung dieser probabilistischen Metrik, liegen die Ursachen jedoch nicht zwangsläufig allein in den ökonomischen Einschränkungen wie dem Zeit- und Kostenrahmen oder fehlendem Know-How zur Methodik oder der Auswertung - vielmehr können die Einflussfaktoren auf die Zuverlässigkeit multidimensional und/oder komplex wechselwirkend sein.
Moderne Produkte können schlichtweg durch vielfältigere Mechanismen ausfallen.

\section{Forschungperspektive und Problembeschreibung}
Da trotz genannter Randbedingungen die Kundenanforderungen und Garantiebedingungen üblicherweise als unveränderlich gelten, können verbleibende Unsicherheiten in der Zuverlässigkeit daher häufig nur durch präventive Wartungsstrategien oder durch Tolerierung von Restrisiken unter Inkaufnahme von ex-post Schadensbegrenzung gehandhabt werden.
Führt dies jedoch zu erheblichen Regress- oder Kulanzkosten, müssen ausfallschutzorientierte Maßnahmen bis hin zu Rückrufaktionen frühzeitig geplant und umgesetzt werden, um (selten nachhaltig) Image- und Kostenrisiken zu minimieren.
Somit ist also im Vorfeld den Randbedingungen aufgrund fortschreitend und hoch integrativer Elektrifizierung sowie Digitalisierung, verkürzten Entwicklungszyklen, verschärftem Kostendruck, generisch sich ändernder Priorisierungen, sich wandelnder Materialauswahl und -Komposition, auf Performance getrimmte Beanspruchung sowie Beanspruchbarkeit, intensivierte Einsatzbedingungen und nicht zuletzt effiziente Ressourcenausnutzung in einem Produkt zu begegnen:
beispielsweise wäre ein kompfortabler gleichwohl wie ausfallfreier Betrieb eines Automobils gegenüber verschiedenster Schadenshergänge einerseits stets zu erhalten, andererseits inzwischen bedingt durch den Funktionserhalt von einem bis zu dreistelligen Vielfachen an ECUs, wohingegen es in jüngster Vergangenheit noch eine kleine bis mittlerere zweistelliger Anzahl mit reduzierten Funktionen üblich war.

Das damit betrachtete System bildet somit eine vielfältig komplexeres Netz aus teils wechselwirkenden oder direkten Eigenschaften, die zu Versagen führen könnten.
Damit sei gemeint: Produkt-Designs werden gegenüber steigenden Kundenanforderungen zunhemend raffinierter und hinsichtlich des Effizienzgedankens optmiert (mehr Sensorik mit fortschrittlichen Assistenzsystemen), die Art und Weise des möglichen Funktionsverlustes wird jedoch zunehmend komplexer.
So rückt also die Fähigkeit, Prognosen über die Lebensdauer und Ausfallwahrscheinlichkeit in Anhängigkeit von mehreren Einflüssen treffen zu können, zunehmend in den unternehmerischen Fokus für strategische Entscheidungen.
Jenseits der klassischen Testplanung im Rahmen der Zuverlässigkeitstechnik, erfordert dies jedoch zunehmend Methoden aus der statistischen Versuchsplanung (engl.: Design of Experiments, DoE) unter Berücksichtigung mehrerer Einflussfaktoren.


\section{Beitrag dieser Arbeit}
Mit der Problembeschreibung lässt sich der übergeordnete Beitrag dieser Arbeit damit formulieren: liegen komplexe technische Systeme vor und sollen diese hinsichtlich ihrer Lebensdauer empirisch untersucht werden, um etwaige Prognosen hinsichtlich der Funktionalität über die Betriebszeit treffen zu können, müssen mehrdimensionale Lebensdaueruntersuchungen nach dem Vorbild von DoE geplant werden.
Neben der schieren Implementierung von Tests für Lebensdauerversuche berücksichtigt dies jedoch auch:
\begin{itemize}
    \item eine effiziente Methodik zur Vorauswahl der Faktoren aus der Gesamtheit der zu identifizierenden Systemparameter mit dem Ziel, deren Einfluss auf die Lebensdauer zu untersuchen;
    \item die Auswahl geeigneter Strategien sowie passender Testpläne zur statistisch abgesicherten Quantifizierung von Einflüssen auf die Lebensdauer in Kombination mit konventionellen Zuverlässigkeitsmethoden wie der beschleunigten Testplanung (engl. Accelerated Lifetime Testing, ALT);
    \item die präzise Parameterschätzung für die mathematische Beschreibung der Effekte auf Basis ermittelter Einflüssen;
    \item Bilanzierung geeigneter Testpläne, relativ zur Abweichung etablierter Standardpläne.
\end{itemize}

\section{Aufbau der Arbeit}
Zur Orientierung soll eine makroskopische Beschreibung zum Aufbau der Arbeit dienen, welcher \colorbox{yellow}{Abbildung1} zu entnehmen ist.