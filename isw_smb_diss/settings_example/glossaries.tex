
\RequirePackage{shellesc}

\usepackage[toc=true, % add glossaries to toc
acronyms, % we use it for acronyms
shortcuts=ac, % use acronyms with \ac{} shortcuts
symbols, % use it for table of symbols
nogroupskip,
nomain, % there is no "main" glossary
nonumberlist, %suppress page numbers where the acronym is used in list of acronyms
automake % so we won't need to call glossaries by hand
]{glossaries-extra}
%\renewcommand*{\glstextformat}[1]{\textcolor{uniSgreen!50!black}{#1}}
%\renewcommand*{\glstextformat}[1]{\textcolor{black}{#1}} % just color glossary links them black

%\usepackage{glossary-longbooktabs} % long-booktabs styles: https://www.dickimaw-books.com/gallery/glossaries-styles/#longbooktabs


% acronym style is: Long Forms (short)
% define acronyms before \begin{document} with 
%	\newacronym{ABSC}{ABSC}{Adaptive Backstepping Controller}
%
% and use them with 
% - \ac{ABSC} ... LaTeX uses long form first with introduciton of short form, then short form only
% - \acs{ABSC} ... explicitly use short form
% - \acl{ABSC} ... explicitly use long form
% - \ac{ABSC}["=Gerät] ... compose hyphenated form with suffix
% - \glsresetall to reset all acronyms and introduce them again
%
% but also allows German composed words with \ac{EoL}["=Prüfung] => End-of-Line-Prüfung
\setabbreviationstyle[acronym]{long-hyphen-short-hyphen} % requires glossaries-extra package, otherwise use style long-short

% additionally needed to also replace spaces by hyphens in long form
\glssetcategoryattribute{\acronymtype}{markwords}{true}

% for more flexibility we define our own style here (based on longtable)
% Define glossary entries (before \begin{document}) with: \glsxtrnewsymbol[description={Tischposition},symbol={\si{m}}]{xT}{\ensuremath{x_T}}
% Refer to them in document with \gls{xT}
\newglossarystyle{symbunitlong}{%
	\renewenvironment{theglossary}%
	{\begin{longtable}{@{}lcp{0.6\textwidth}@{}}}%
		{\end{longtable}}%
	\renewcommand*{\glossaryheader}{%
		\toprule
		\bfseries Symbol & \bfseries Einheit & \bfseries Beschreibung \tabularnewline \midrule\endhead
		\bottomrule\endfoot}%
	\renewcommand{\glossentry}[2]{%
		\glsentryitem{##1}\glstarget{##1}{\glossentryname{##1}} &
		\glossentrysymbol{##1} &
		\glossentrydesc{##1}\\%
	}%
	\ifglsnogroupskip
	\renewcommand*{\glsgroupskip}{}%
	\else
	\renewcommand*{\glsgroupskip}{ & & & \tabularnewline}%
	\fi
}

\newglossarystyle{alttreetwocols}{%
	\glossarystyle{alttree}%
	\BeforeBeginEnvironment{theglossary}{\begin{multicols}{2}}%
	\AfterEndEnvironment{theglossary}{\end{multicols}}%
	}

\makeglossaries

%automatically label glossaries:
\renewcommand*{\glossarypreamble}{%
	\label{glo:\currentglossary}%
}

% fix automake glossaries with lualatex: https://tex.stackexchange.com/a/468820
\usepackage{etoolbox}
\makeatletter
\patchcmd\@gls@automake{\write18}{\DelayedShellEscape}{}{\fail}
\makeatother